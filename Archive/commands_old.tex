


% ----------------------------------------------------------------------------
% Syntax-Abkürzungen für logische und mengentheoretische Operatoren
% ----------------------------------------------------------------------------
% Dieser Abschnitt definiert Kurzformen (Abkürzungen) für häufig verwendete logische
% und mengentheoretische Operatoren, die in diesem Dokumentin Form von Commands 
% verwendet werden. Diese Kurzformen sind entworfen, um die Lesbarkeit des Codes zu 
% verbessern und eine effiziente Bearbeitung zu ermöglichen.
%
% Komma:
% w - with
%
% Klammern:
% Lp - Linke runde Klammer (Left Parenthesis)
% Rp - Rechte runde Klammer (Right Parenthesis)
% Lb - Linke eckige Klammer (Left Bracket)
% Rb - Rechte eckige Klammer (Right Bracket)
% Lc - Linke geschweifte Klammer (Left Curly)
% Rc - Rechte geschweifte Klammer (Right Curly)
% La - Linker Winkel (Left Angle Bracket oder Left Chevron)
% Ra - Rechter Winkel (Right Angle Bracket oder Right Chevron)
%
% Logische Operatoren:
% Or  - Logisches ODER (\lor)
% And - Logisches UND (\land)
% n   - Negation (\neg)
% Bot - Widerspruch (\bot)
% To  - Implikation (\rightarrow)
% Lr  - Äquivalenz (\leftrightarrow)
% Fa  - Für alle (\forall)
% Ex  - Existiert (\exists)
% Exu - Existiert genau eines (\exists!)
%
% Folgerungszeichen:
% Imp - ⊢ (\vdash)
% Eqv - ⊣⊢ (\dashv\vdash)
% Mengentheoretische Operatoren:
% Cup - Vereinigung (\cup)
% Cap - Schnittmenge (\cap)
% Times - Kartesisches Produkt (\times)
% Powerset - Potenzmenge (\powerset)
% Subseteq - Teilmenge (\subseteq)
% Emptyset - Leere Menge (\emptyset)
% Notin - Kein Element (\notin)
% In - Element (\in)
%
% Beispielverwendung im Dokument: \newcommand{\PLrQwPImpQ}[1]{\hyperref[PLrQwPImpQ]{\ensuremath{P \leftrightarrow Q, P \vdash Q}(#1)}}
% ----------------------------------------------------------------------------

    \newcommand{\rA}{\hyperref[rule:A]{\ensuremath{A}}}
    % Regeln zum Umgang mit dem ∧-Symbol
    \newcommand{\rAI}[1]{\hyperref[rule:AI]{\ensuremath{\land}I(#1)}}
    \newcommand{\rAEa}[1]{\hyperref[rule:AE1]{\ensuremath{\land}E1(#1)}}
    \newcommand{\rAEb}[1]{\hyperref[rule:AE2]{\ensuremath{\land}E2(#1)}}
	
    % Regeln zum Umgang mit dem ∨-Symbol
    \newcommand{\rOIa}[1]{\hyperref[rule:OI1]{\ensuremath{\lor}I1(#1)}}
    \newcommand{\rOIb}[1]{\hyperref[rule:OI2]{\ensuremath{\lor}I2(#1)}}
    \newcommand{\rOE}[1]{\hyperref[rule:OE]{\ensuremath{\lor}E(#1)}}	
	
    % Regeln zum Umgang mit dem →-Symbol
    \newcommand{\rRI}[1]{\hyperref[rule:RI]{\ensuremath{\rightarrow}I(#1)}}	
    \newcommand{\rRE}[1]{\hyperref[rule:RE]{\ensuremath{\rightarrow}E(#1)}}	
	
    % Regeln zu Umgang mit dem ↔.Symbol
    \newcommand{\rLRI}[1]{\hyperref[rule:LRI]{\ensuremath{\leftrightarrow}I(#1)}}	
    \newcommand{\rLREa}[1]{\hyperref[rule:LRE1]
    {\ensuremath{\leftrightarrow}E1(#1)}}	
    \newcommand{\rLREb}[1]{\hyperref[rule:LRE2]{\ensuremath{\leftrightarrow}E2(#1)}}

    \newcommand{\rLRS}[1]{\hyperref[rule:LRSubst]{\ensuremath{\leftrightarrow}S(#1)}}

	
    % Regeln zum Umgang mit dem ∀-Symbol
    \newcommand{\rUI}[1]{\hyperref[rule:UI]{\ensuremath{\forall}I(#1)}}		
    \newcommand{\rUE}[1]{\hyperref[rule:UE]{\ensuremath{\forall}E(#1)}}	
	
	
    % Regeln zum Umgang mit dem ∃-Symbol	
    \newcommand{\rEI}[1]{\hyperref[rule:EI]{\ensuremath{\exists}I(#1)}}				
    \newcommand{\rEE}[1]{\hyperref[rule:EE]{\ensuremath{\exists}E(#1)}}	
	
    % Regeln zum Umgang mit dem ∃!-Symbol
    \newcommand{\UEI}[1]{\hyperref[rule:UEI]{\ensuremath{\exists!}I(#1)}}			
    \newcommand{\UEE}[1]{\hyperref[rule:UEE]{\ensuremath{\exists!}E(#1)}}	

    % ∃!x(P(x))↔∃x(P(x)∧∀y(P(y)→x=y))
    \newcommand{\ExonlyonexLpPLpxRpRpLrExxLpPLpxRpAndFayLpPLpyRpToxEqualsyRpRp}[1]{\hyperref[ExonlyonexLpPLpxRpRpLrExxLpPLpxRpAndFayLpPLpyRpToxEqualsyRpRp]{\ensuremath{\exists! x(P(x)) \leftrightarrow \exists x(P(x) \land \forall y(P(y) \rightarrow x = y))}(#1)}}
	
    % Regeln zum Umgang mit dem =-Symbol
    \newcommand{\rII}{\hyperref[rule:II]{\ensuremath{=}I}}
    \newcommand{\rIE}[1]{\hyperref[rule:IE]{\ensuremath{=}E(#1)}}	

    % Regeln zum Umgang mit dem =-Symbol dreier Gleichheiten
    \newcommand{\rIIb}[1]{\hyperref[rule:rIIb]{\ensuremath{=}I(#1)}}
    \newcommand{\rIEb}[1]{\hyperref[rule:rIEb]{\ensuremath{=}E(#1)}}	
	
    \newcommand{\rNeq}{\hyperref[rule:Neq]{\ensuremath{\neq}}}	
	
    % Regeln zum Umgang mit dem ¬-Symbol als auch dem ⊥-Symbol	
    \newcommand{\rBI}[1]{\hyperref[rule:BI]{\ensuremath{\bot}I(#1)}}
    \newcommand{\rCI}[1]{\hyperref[rule:CI]{\ensuremath{\neg}I(#1)}}
    \newcommand{\rCE}[1]{\hyperref[rule:CE]{\ensuremath{\neg}E(#1)}}
    \newcommand{\rDN}[1]{\hyperref[rule:DN]{\ensuremath{}DN(#1)}}

    % Regeln zum Umgang mit Definitionen
    % Q(x_1,...,x_n):=φ(x_1,...,x_n)
    
    %
    \newcommand{\QLpxSubOnewDotswxSubnRpDefEqVarphiLpxSubOnewDotswxSubnRp}[1]{\hyperref[QLpxSubOnewDotswxSubnRpDefEqVarphiLpxSubOnewDotswxSubnRp]{\textnormal{Def. \nameref*{QLpxSubOnewDotswxSubnRpDefEqVarphiLpxSubOnewDotswxSubnRp}}(#1) \textnormal{, p.~\pageref*{QLpxSubOnewDotswxSubnRpDefEqVarphiLpxSubOnewDotswxSubnRp}}}}


    % C(x_1,...,x_n)→Q(x_1,...,x_n):=φ(x_1,...,x_n)
    \newcommand{\CLpxSubOnewDotswxSubnRpToQLpxSubOnewDotswxSubnRpDefEqVarphiLpxSubOnewDotswxSubnRp}[1]{\hyperref[CLpxSubOnewDotswxSubnRpToQLpxSubOnewDotswxSubnRpDefEqVarphiLpxSubOnewDotswxSubnRp]{\textnormal{Def. \nameref*{CLpxSubOnewDotswxSubnRpToQLpxSubOnewDotswxSubnRpDefEqVarphiLpxSubOnewDotswxSubnRp}}(#1) \textnormal{, p.~\pageref*{CLpxSubOnewDotswxSubnRpToQLpxSubOnewDotswxSubnRpDefEqVarphiLpxSubOnewDotswxSubnRp}}}}

    % σ(a_1,...,a_n):=b, σ(x_1,...,x_n)=f(σ(y_1,...,y_m),z_1,...,z_p)
    \newcommand{\dfSigmaGenRecursive}[1]{\hyperref[dfSigmaGenRecursive]{\textnormal{Def. \nameref*{dfSigmaGenRecursive}}(#1) \textnormal{, p.~\pageref*{dfSigmaGenRecursive}}}}

    % σ wird implizit definiert durch Axiome ϕ
    \newcommand{\dfSigmaGenImplicitAxiom}{\hyperref[dfSigmaGenImplicitAxiom]{\textnormal{Def. \nameref*{dfSigmaGenImplicitAxiom}} \phi_i \textnormal{, p.~\pageref*{dfSigmaGenImplicitAxiom}}}}

    % Eine Regel wird über ein Theorem festgelegt
    \newcommand{\dfSigmaGenImplicitRule}[1]{\hyperref[dfSigmaGenImplicitRule]{\textnormal{\nameref*{dfSigmaGenImplicitRule}}(#1) \textnormal{, p.~\pageref*{dfSigmaGenImplicitRule}}}}

    % σ wird über eine ι-Definition festgelegt
    \newcommand{\dfSigmaGenIota}{\hyperref[dfSigmaGenIota]{\textnormal{ [2;D(\nameref*{dfSigmaGenIota})}; \textnormal{\pageref*{dfSigmaGenIota}}]}}

    % Eine Regel wird über ein Theorem festgelegt
    \newcommand{\theoremUsageRule}[1]{\hyperref[theoremUsageRule]{\textnormal{[2;T(2.8.25);\pageref*{theoremUsageRule}]}(#1)}}

    % Regel zur Kettennotation
    \newcommand{\rChain}[1]{\hyperref[rule:Chain]{\ensuremath{\mathsf{Tr.}}(#1)}}

    % Regeln zum Umgang mit der Kontravalenz
    % P⊻Q↔(P∧¬Q)∨(¬P∧Q)
    \newcommand*\lxor{\mathbin{\veebar}}
    
    \newcommand{\PXOrQLrLpPAndnQRpOrLpnPAndQRp}[1]{\hyperref[XOr]{\ensuremath{P \lxor Q \leftrightarrow (P \land \neg Q) \lor (\neg P \land Q)}(#1)}}
    \newcommand{\rXOrE}[1]{\hyperref[rule:XOrE]{\ensuremath{\lxor}E(#1)}}
    \newcommand{\rXOrIa}[1]{\hyperref[rule:XOrI1]{\ensuremath{\lxor}I1(#1)}}	
    \newcommand{\rXOrIb}[1]{\hyperref[rule:XOrI2]{\ensuremath{\lxor}I2(#1)}}	

		
    % Grundlegende Beweise der Logik

    % Band 1

    %Und-Elimination verschachtelter Konjugationen
    % P∧(Q∧R))⊢Q
    \newcommand{\PAndLpQAndRRpRpImpQ}[1]{%
      \hyperref[PAndLpQAndRRpRpImpQ]{2.\ref*{PAndLpQAndRRpRpImpQ}(#1)}%
    }
    
    % (P∧(Q∧R))⊢R
    \newcommand{\LpPAndLpQAndRRpRpImpR}[1]{%
      \hyperref[LpPAndLpQAndRRpRpImpR]{2.\ref*{LpPAndLpQAndRRpRpImpR}(#1)}%
    }
    
    % ((P∧Q)∧R)⊢P
    \newcommand{\LpLpPAndQRpAndRRpImpP}[1]{%
      \hyperref[LpLpPAndQRpAndRRpImpP]{2.\ref*{LpLpPAndQRpAndRRpImpP}(#1)}%
    }
    
    % ((P∧Q)∧R)⊢Q
    \newcommand{\LpLpPAndQRpAndRRpImpQ}[1]{%
      \hyperref[LpLpPAndQRpAndRRpImpQ]{2.\ref*{LpLpPAndQRpAndRRpImpQ}(#1)}%
    }
    
    % Selbstbezüglichkeiten
    % P→P
    \newcommand{\PToP}[1]{\hyperref[PToP]{2.\ref*{PToP}(#1)}}
    
    % P↔P
    \newcommand{\PLrP}[1]{\hyperref[PLrP]{2.\ref*{PLrP}(#1)}}
    
    % Äquivalenz als Folgerung
    % P ↔ Q, P ⊢ Q
    \newcommand{\PLrQwPImpQ}[1]{%
      \hyperref[PLrQwPImpQ]{2.\ref*{PLrQwPImpQ}(#1)}%
    }
    
    % P ↔ Q, Q ⊢ P
    \newcommand{\PLrQwQImpP}[1]{%
      \hyperref[PLrQwQImpP]{2.\ref*{PLrQwQImpP}(#1)}%
    }
    
    % P→Q,P→R⊢P→(Q∧R)
    \newcommand{\PToQwPToRImpPToLpQAndRRp}[1]{%
      \hyperref[PToQwPToRImpPToLpQAndRRp]{2.\ref*{PToQwPToRImpPToLpQAndRRp}(#1)}%
    }

    % P→Q,P→R⊢P→(R∧Q)
    \newcommand{\PToQwPToRImpPToLpRAndQRp}[1]{\hyperref[PToQwPToRImpPToLpRAndQRp]{2.\ref*{PToQwPToRImpPToLpRAndQRp}(#1)}}
    
    % P→Q⊢P↔(P∧Q)
    \newcommand{\PToQImpPLrLpPAndQRp}[1]{%
      \hyperref[PToQImpPLrLpPAndQRp]{2.\ref*{PToQImpPLrLpPAndQRp}(#1)}%
    }

    % P→Q⊢P↔(Q∧P)
    \newcommand{\PToQImpPLrLpQAndPRp}[1]{\hyperref[PToQImpPLrLpQAndPRp]{2.\ref*{PToQImpPLrLpQAndPRp}(#1)}}
    
    % P ∨ P ⊣⊢ P
    \newcommand{\POrPEqvP}[1]{%
      \hyperref[POrPEqvP]{2.\ref*{POrPEqvP}(#1)}%
    }
    
    % P ∧ P ⊣⊢ P
    \newcommand{\PAndPEqvP}[1]{%
      \hyperref[PAndPEqvP]{2.\ref*{PAndPEqvP}(#1)}%
    }
    
    % P ∨ Q ⊢ Q ∨ P
    \newcommand{\POrQImpQOrP}[1]{%
      \hyperref[POrQImpQOrP]{2.\ref*{POrQImpQOrP}(#1)}%
    }

    % P∨Q⊣⊢Q∨P
    \newcommand{\POrQEqvQOrP}[1]{\hyperref[POrQEqvQOrP]{2.\ref*{POrQEqvQOrP}(#1)}}

    
    % P ∧ Q ⊢ Q ∧ P
    \newcommand{\PAndQImpQAndP}[1]{%
      \hyperref[PAndQImpQAndP]{2.\ref*{PAndQImpQAndP}(#1)}%
    }
    
    % P ↔ Q ⊣⊢ Q ↔ P
    \newcommand{\PLrQImpQLrP}[1]{%
      \hyperref[PLrQImpQLrP]{2.\ref*{PLrQImpQLrP}(#1)}%
    }
    
    % ∀x(P(x)↔Q(x))⊢∀x(Q(x)↔P(x))
    \newcommand{\FaxLpPLpxRpLrQLpxRpRpImpFaxLpQLpxRpLrPLpxRpRp}[1]{%
      \hyperref[FaxLpPLpxRpLrQLpxRpRpImpFaxLpQLpxRpLrPLpxRpRp]{2.\ref*{FaxLpPLpxRpLrQLpxRpRpImpFaxLpQLpxRpLrPLpxRpRp}(#1)}%
    }
    
    % P→Q,Q→R,P⊢R
    \newcommand{\PToQwQToRwPImpR}[1]{%
      \hyperref[PToQwQToRwPImpR]{2.\ref*{PToQwQToRwPImpR}(#1)}%
    }
    
    % P→Q,Q→R⊢P→R (Implikationstransitivität)
    \newcommand{\PToQwQToRImpPToR}[1]{%
      \hyperref[PToQwQToRImpPToR]{2.\ref*{PToQwQToRImpPToR}(#1)}%
    }
    
    % P↔Q,Q↔R⊢P→R
    \newcommand{\PLrQwQLrRImpPToR}[1]{%
      \hyperref[PLrQwQLrRImpPToR]{2.\ref*{PLrQwQLrRImpPToR}(#1)}%
    }
    
    % P↔Q,Q↔R⊢R→P
    \newcommand{\PLrQwQLrRImpRToP}[1]{%
      \hyperref[PLrQwQLrRImpRToP]{2.\ref*{PLrQwQLrRImpRToP}(#1)}%
    }
    
    % P↔Q,Q↔R⊢P↔R (Äquivalenztransitivität)
    \newcommand{\PLrQwQLrRImpPLrR}[1]{%
      \hyperref[PLrQwQLrRImpPLrR]{2.\ref*{PLrQwQLrRImpPLrR}(#1)}%
    }
    
    % P↔Q,R↔Q⊢P↔R (Äquivalenztransitivität)
    \newcommand{\PLrQwRLrQImpPLrR}[1]{%
      \hyperref[PLrQwRLrQImpPLrR]{2.\ref*{PLrQwRLrQImpPLrR}(#1)}%
    }
    
    % P→Q,Q↔R⊢P→R (Kettenregel)
    \newcommand{\PToQwQLrRImpPToR}[1]{%
      \hyperref[PToQwQLrRImpPToR]{2.\ref*{PToQwQLrRImpPToR}(#1)}%
    }
    
    % ∀x(P(x)→Q(x)),∀x(Q(x)→R(x))⊢∀x(P(x)→R(x))
    \newcommand{\FaxLpPLpxRpToQLpxRpRpwFaxLpQLpxRpToRLpxRpRpImpFaxLpPLpxRpToRLpxRpRp}[1]{%
      \hyperref[FaxLpPLpxRpToQLpxRpRpwFaxLpQLpxRpToRLpxRpRpImpFaxLpPLpxRpToRLpxRpRp]{2.\ref*{FaxLpPLpxRpToQLpxRpRpwFaxLpQLpxRpToRLpxRpRpImpFaxLpPLpxRpToRLpxRpRp}(#1)}%
    }

    % ∀x(P(x)↔Q(x)),∀x(Q(x)↔R(x))⊢∀x(P(x)↔R(x))
    \newcommand{\FaxLpPLpxRpLrQLpxRpRpwFaxLpQLpxRpLrRLpxRpRpImpFaxLpPLpxRpLrRLpxRpRp}[1]{\hyperref[FaxLpPLpxRpLrQLpxRpRpwFaxLpQLpxRpLrRLpxRpRpImpFaxLpPLpxRpLrRLpxRpRp]{2.\ref*{FaxLpPLpxRpLrQLpxRpRpwFaxLpQLpxRpLrRLpxRpRpImpFaxLpPLpxRpLrRLpxRpRp}(#1)}}
    
    % ∀x(P(x)↔Q(x)),∀x(Q(x)↔R(x))⊢∀x(P(x)↔R(x))
    \newcommand{\FaxLpPLpxRpLrQLpxRpRpImpFaxLpPLpxRpLrRLpxRpRp}[1]{%
      \hyperref[FaxLpPLpxRpLrQLpxRpRpImpFaxLpPLpxRpLrRLpxRpRp]{2.\ref*{FaxLpPLpxRpLrQLpxRpRpImpFaxLpPLpxRpLrRLpxRpRp}(#1)}%
    }
    
    % ∀x(P(x)↔Q(x)),∀x(R(x)↔Q(x))⊢∀x(P(x)↔R(x))
    \newcommand{\FaxLpPLpxRpLrQLpxRpRpwFaxLpRLpxRpLrQLpxRpRpImpFaxLpPLpxRpLrRLpxRpRp}[1]{%
      \hyperref[FaxLpPLpxRpLrQLpxRpRpwFaxLpRLpxRpLrQLpxRpRpImpFaxLpPLpxRpLrRLpxRpRp]{2.\ref*{FaxLpPLpxRpLrQLpxRpRpwFaxLpRLpxRpLrQLpxRpRpImpFaxLpPLpxRpLrRLpxRpRp}(#1)}%
    }
    
    % P ∨ (Q ∨ R) ⊣⊢ (P ∨ Q) ∨ R
    \newcommand{\POrLpQOrRRpImpLpPOrQRpOrQ}[1]{%
      \hyperref[POrLpQOrRRpImpLpPOrQRpOrQ]{2.\ref*{POrLpQOrRRpImpLpPOrQRpOrQ}(#1)}%
    }
    
    % P ∧ (Q ∧ R) ⊣⊢ (P ∧ Q) ∧ R
    \newcommand{\PAndLpQAndRRpImpLpPAndQRpAndQ}[1]{%
      \hyperref[PAndLpQAndRRpImpLpPAndQRpAndQ]{2.\ref*{PAndLpQAndRRpImpLpPAndQRpAndQ}(#1)}%
    }
    
 
    
    % P → Q, ¬Q ⊢ ¬P
    \newcommand{\PToQwnQImpnP}[1]{%
      \hyperref[PToQwnQImpnP]{2.\ref*{PToQwnQImpnP}(#1)}%
    }
    
    % ¬P → ¬Q, Q ⊢ P
    \newcommand{\nPTonQwQImpP}[1]{%
      \hyperref[nPTonQwQImpP]{2.\ref*{nPTonQwQImpP}(#1)}%
    }
    
    % P ↔ Q, ¬Q ⊢ ¬P
    \newcommand{\PLrQwnQImpnP}[1]{%
      \hyperref[PLrQwnQImpnP]{2.\ref*{PLrQwnQImpnP}(#1)}%
    }
    
    % P ↔ Q, ¬P ⊢ ¬Q
    \newcommand{\PLrQwnPImpnQ}[1]{%
      \hyperref[PLrQwnPImpnQ]{2.\ref*{PLrQwnPImpnQ}(#1)}%
    }
    
    % ¬P ↔ ¬Q, Q ⊢ P
    \newcommand{\nPLrnQwQImpP}[1]{%
      \hyperref[nPLrnQwQImpP]{2.\ref*{nPLrnQwQImpP}(#1)}%
    }
    
    % ¬P ↔ ¬Q, P ⊢ Q
    \newcommand{\nPLrnQwPImpQ}[1]{%
      \hyperref[nPLrnQwPImpQ]{2.\ref*{nPLrnQwPImpQ}(#1)}%
    }

    % P→Q⊣⊢¬Q→¬P
    \newcommand{\PToQEqvnQTonP}[1]{\hyperref[PToQEqvnQTonP]{2.\ref*{PToQEqvnQTonP}(#1)}}

    % (P∧Q)→R⊣⊢P→(Q→R)
    \newcommand{\LpPAndQRpToREqvPToLpQToRRp}[1]{\hyperref[LpPAndQRpToREqvPToLpQToRRp]{2.\ref*{LpPAndQRpToREqvPToLpQToRRp}(#1)}}

    % (P∧Q)→R⊣⊢Q→(P→R)
    \newcommand{\LpPAndQRpToREqvQToLpPToRRp}[1]{\hyperref[LpPAndQRpToREqvQToLpPToRRp]{2.\ref*{LpPAndQRpToREqvQToLpPToRRp}(#1)}}

    % P∨Q⊣⊢¬¬P∨¬¬Q
    \newcommand{\POrQEqvnnPOrnnQ}[1]{\hyperref[POrQEqvnnPOrnnQ]{2.\ref*{POrQEqvnnPOrnnQ}(#1)}}

    % P∧Q⊣⊢¬¬P∧¬¬Q
    \newcommand{\PAndQEqvnnPAndnnQ}[1]{\hyperref[PAndQEqvnnPAndnnQ]{2.\ref*{PAndQEqvnnPAndnnQ}(#1)}}
    
    % Theoreme für Law Execution Middle, ⊢P∨¬P
    \newcommand{\ImpPOrnP}{%
      \hyperref[ImpPOrnP]{2.\ref*{ImpPOrnP}}%
    }
    
    % ¬(P ∨ Q) ⊣⊢ ¬ P ∧ ¬ Q (De Morgan)
    \newcommand{\nLpPOrQRpEqvnPAndnQ}[1]{%
      \hyperref[nLpPOrQRpEqvnPAndnQ]{2.\ref*{nLpPOrQRpEqvnPAndnQ}(#1)}%
    }
    
    % ¬(P ∧ Q) ⊣⊢ ¬P ∨ ¬Q (De Morgan 2)
    \newcommand{\nLpPAndQRpEqvnPOrnQ}[1]{%
      \hyperref[nLpPAndQRpEqvnPOrnQ]{2.\ref*{nLpPAndQRpEqvnPOrnQ}(#1)}%
    }
    
    % ¬(¬ P ∧ ¬ Q) ⊣⊢  P ∨  Q
    \newcommand{\nLpnPAndnQRpEqvPOrQ}[1]{%
      \hyperref[nLpnPAndnQRpEqvPOrQ]{2.\ref*{nLpnPAndnQRpEqvPOrQ}(#1)}%
    }
    
    % ¬(¬ P ∨ ¬ Q) ⊣⊢  P ∧  Q
    \newcommand{\nLpnPOrnQRpEqvPAndQ}[1]{%
      \hyperref[nLpnPOrnQRpEqvPAndQ]{2.\ref*{nLpnPOrnQRpEqvPAndQ}(#1)}%
    }
    
    % P∧(Q∨R)⊣⊢(P∧Q)∨(P∧R)
    \newcommand{\PAndLpQOrRRpEqvLpPAndQRpOrLpPAndRRp}[1]{%
      \hyperref[PAndLpQOrRRpEqvLpPAndQRpOrLpPAndRRp]{2.\ref*{PAndLpQOrRRpEqvLpPAndQRpOrLpPAndRRp}(#1)}%
    }
    
    % (P∨Q)∧R⊣⊢(P∧R)∨(Q∧R)
    \newcommand{\LpPOrQRpAndREqvLpPAndRRpOrLpQAndRRp}[1]{%
      \hyperref[LpPOrQRpAndREqvLpPAndRRpOrLpQAndRRp]{2.\ref*{LpPOrQRpAndREqvLpPAndRRpOrLpQAndRRp}(#1)}%
    }
    
    % P∨(Q∧R)⊣⊢(P∨Q)∧(P∨R)
    \newcommand{\POrLpQAndRRpEqvLpPOrQRpAndLpPOrRRp}[1]{%
      \hyperref[POrLpQAndRRpEqvLpPOrQRpAndLpPOrRRp]{2.\ref*{POrLpQAndRRpEqvLpPOrQRpAndLpPOrRRp}(#1)}%
    }
    
    % (P∧Q)∨R⊣⊢(P∨R)∧(Q∨R)
    \newcommand{\LpPAndQRpOrREqvLpPOrRRpAndLpQOrRRp}[1]{%
      \hyperref[LpPAndQRpOrREqvLpPOrRRpAndLpQOrRRp]{2.\ref*{LpPAndQRpOrREqvLpPOrRRpAndLpQOrRRp}(#1)}%
    }

    % P→P∨Q
    \newcommand{\PToPOrQ}[1]{\hyperref[PToPOrQ]{2.\ref*{PToPOrQ}(#1)}}

    % P→Q∨P
    \newcommand{\PToQOrP}[1]{\hyperref[PToQOrP]{2.\ref*{PToQOrP}(#1)}}

    % P→Q⊣⊢¬P∨Q
    \newcommand{\PToQEqvnPOrQ}[1]{\hyperref[PToQEqvnPOrQ]{2.\ref*{PToQEqvnPOrQ}(#1)}}

    % ¬P→Q⊣⊢P∨Q
    \newcommand{\nPToQEqvPOrQ}[1]{\hyperref[nPToQEqvPOrQ]{2.\ref*{nPToQEqvPOrQ}(#1)}}

    % ¬Q→P⊣⊢P∨Q
    \newcommand{\nQToPEqvPOrQ}[1]{\hyperref[nQToPEqvPOrQ]{2.\ref*{nQToPEqvPOrQ}(#1)}}
    
    
    % P∨Q,¬P⊢Q
    \newcommand{\POrQwnPImpQ}[1]{%
      \hyperref[POrQwnPImpQ]{2.\ref*{POrQwnPImpQ}(#1)}%
    }
    
    % P∨Q,¬Q⊢P
    \newcommand{\POrQwnQImpP}[1]{%
      \hyperref[POrQwnQImpP]{2.\ref*{POrQwnQImpP}(#1)}%
    }
    
    % ¬(P→Q)⊣⊢P∧¬Q
    \newcommand{\nLpPToQRpEqvPAndnQ}[1]{%
      \hyperref[nLpPToQRpEqvPAndnQ]{2.\ref*{nLpPToQRpEqvPAndnQ}(#1)}%
    }
    
    % Q⊢P→Q
    \newcommand{\QImpPToQ}[1]{%
      \hyperref[QImpPToQ]{2.\ref*{QImpPToQ}(#1)}%
    }
    
    % ¬P⊢P→Q
    \newcommand{\nPImpPToQ}[1]{%
      \hyperref[nPImpPToQ]{2.\ref*{nPImpPToQ}(#1)}%
    }

    % P→Q,R→Q,P∨R⊢Q
    \newcommand{\PToQwRToQwPOrRImpQ}[1]{\hyperref[PToQwRToQwPOrRImpQ]{2.\ref*{PToQwRToQwPOrRImpQ}(#1)}}

    % P→Q,P∨R⊢Q∨R
    \newcommand{\PToQwPOrRImpQOrR}[1]{\hyperref[PToQwPOrRImpQOrR]{2.\ref*{PToQwPOrRImpQOrR}(#1)}}

    % P→Q,P∧R⊢Q∧R
    \newcommand{\PToQwPAndRImpQAndR}[1]{\hyperref[PToQwPAndRImpQAndR]{2.\ref*{PToQwPAndRImpQAndR}(#1)}}

    % P→Q,R∧P⊢R∧Q
    \newcommand{\PToQwRAndPImpRAndQ}[1]{\hyperref[PToQwRAndPImpRAndQ]{2.\ref*{PToQwRAndPImpRAndQ}(#1)}}

    % P→Q⊢(P∧Q)↔P
    \newcommand{\PToQImpLpPAndQRpLrP}[1]{\hyperref[PToQImpLpPAndQRpLrP]{2.\ref*{PToQImpLpPAndQRpLrP}(#1)}}
    
    % P→Q⊢(Q∧P)↔P
    \newcommand{\PToQImpLpQAndPRpLrP}[1]{\hyperref[PToQImpLpQAndPRpLrP]{2.\ref*{PToQImpLpQAndPRpLrP}(#1)}}
    
    % P↔Q⊣⊢¬P↔¬Q
    \newcommand{\PLrQEqvnPLrnQ}[1]{%
      \hyperref[PLrQEqvnPLrnQ]{2.\ref*{PLrQEqvnPLrnQ}(#1)}%
    }

    % ∀x(P(x)→Q(x))⊢∀x((P(x)∧Q(x))↔P(x))
    \newcommand{\FaxLpPLpxRpToQLpxRpRpImpFaxLpLpPLpxRpAndQLpxRpRpLrPLpxRpRp}[1]{\hyperref[FaxLpPLpxRpToQLpxRpRpImpFaxLpLpPLpxRpAndQLpxRpRpLrPLpxRpRp]{2.\ref*{FaxLpPLpxRpToQLpxRpRpImpFaxLpLpPLpxRpAndQLpxRpRpLrPLpxRpRp}(#1)}}

    % ∀x(P(x)→Q(x))⊢∀x((Q(x)∧P(x))↔P(x))
    \newcommand{\FaxLpPLpxRpToQLpxRpRpImpFaxLpLpQLpxRpAndPLpxRpRpLrPLpxRpRp}[1]{\hyperref[FaxLpPLpxRpToQLpxRpRpImpFaxLpLpQLpxRpAndPLpxRpRpLrPLpxRpRp]{2.\ref*{FaxLpPLpxRpToQLpxRpRpImpFaxLpLpQLpxRpAndPLpxRpRpLrPLpxRpRp}(#1)}}
    
    % ∀x(P(x)↔Q(x))⊣⊢∀x(¬P(x)↔¬Q(x))
    \newcommand{\FaxLpPLpxRpLrQLpxRpRpEqvFaxLpnPLpxRpLrnQLpxRpRp}[1]{%
      \hyperref[FaxLpPLpxRpLrQLpxRpRpEqvFaxLpnPLpxRpLrnQLpxRpRp]{2.\ref*{FaxLpPLpxRpLrQLpxRpRpEqvFaxLpnPLpxRpLrnQLpxRpRp}(#1)}%
    }
    
    % P↔Q⊣⊢(P∧Q)∨(¬P∧¬Q)
    \newcommand{\PLrQEqvLpPAndQRpOrLpnPAndnQRp}[1]{%
      \hyperref[PLrQEqvLpPAndQRpOrLpnPAndnQRp]{2.\ref*{PLrQEqvLpPAndQRpOrLpnPAndnQRp}(#1)}%
    }
    
    % P↔Q⊣⊢(P→Q)∧(Q→P)
    \newcommand{\PLrQEqvLpPToQRpAndLpQToPRp}[1]{%
      \hyperref[PLrQEqvLpPToQRpAndLpQToPRp]{2.\ref*{PLrQEqvLpPToQRpAndLpQToPRp}(#1)}%
    }
    
    % P↔Q⊣⊢(¬P∨Q)∧(¬Q∨P)
    \newcommand{\PLrQEqvLpnPOrQRpAndLpnQOrPRp}[1]{%
      \hyperref[PLrQEqvLpnPOrQRpAndLpnQOrPRp]{2.\ref*{PLrQEqvLpnPOrQRpAndLpnQOrPRp}(#1)}%
    }
    
    % ∀x(P(x)↔Q(x))⊣⊢∀x(P(x)→Q(x))∧∀x(Q(x)→P(x))
    \newcommand{\FaxLpPLpxRpLrQLpxRpRpEqvFaxLpPLpxRpToQLpxRpRpAndFaxLpQLpxRpToPLpxRpRp}[1]{%
      \hyperref[FaxLpPLpxRpLrQLpxRpRpEqvFaxLpPLpxRpToQLpxRpRpAndFaxLpQLpxRpToPLpxRpRp]{2.\ref*{FaxLpPLpxRpLrQLpxRpRpEqvFaxLpPLpxRpToQLpxRpRpAndFaxLpQLpxRpToPLpxRpRp}(#1)}%
    }
    
    % ∀x(P(x)→Q(x))⊣⊢∀x(¬P(x)∨Q(x))
    \newcommand{\FaxLpPLpxRpToQLpxRpRpEqvFaxLpnPLpxRpOrQLpxRpRp}[1]{%
      \hyperref[FaxLpPLpxRpToQLpxRpRpEqvFaxLpnPLpxRpOrQLpxRpRp]{2.\ref*{FaxLpPLpxRpToQLpxRpRpEqvFaxLpnPLpxRpOrQLpxRpRp}(#1)}%
    }
    
    % ∀x(P(x)↔Q(x))⊣⊢∀x(¬P(x)∨Q(x))∧∀x(¬Q(x)∨P(x))
    \newcommand{\FaxLpPLpxRpLrQLpxRpRpEqvFaxLpnPLpxRpOrQLpxRpRpAndFaxLpnQLpxRpOrPLpxRpRp}[1]{%
      \hyperref[FaxLpPLpxRpLrQLpxRpRpEqvFaxLpnPLpxRpOrQLpxRpRpAndFaxLpnQLpxRpOrPLpxRpRp]{2.\ref*{FaxLpPLpxRpLrQLpxRpRpEqvFaxLpnPLpxRpOrQLpxRpRpAndFaxLpnQLpxRpOrPLpxRpRp}(#1)}%
    }
    
    % ¬(P↔Q)⊣⊢(¬P∧Q)∨(P∧¬Q)
    \newcommand{\nLpPLrQRpEqvLpnPAndQRpOrLpPAndnQRp}[1]{%
      \hyperref[nLpPLrQRpEqvLpnPAndQRpOrLpPAndnQRp]{2.\ref*{nLpPLrQRpEqvLpnPAndQRpOrLpPAndnQRp}(#1)}%
    }
    
    % ∃x(P(x))⊣⊢¬∀x(¬(P(x))
    \newcommand{\ExxLpPLpxRpRpEqvnFaxLpnLpPLpxRpRp}[1]{%
      \hyperref[ExxLpPLpxRpRpEqvnFaxLpnLpPLpxRpRp]{2.\ref*{ExxLpPLpxRpRpEqvnFaxLpnLpPLpxRpRp}(#1)}%
    }

    % ∀x(¬Q(x))⊢P↔P∨Q(a)
    \newcommand{\FaxLpnQLpxRpRpImpPLrPOrQLpaRp}[1]{\hyperref[FaxLpnQLpxRpRpImpPLrPOrQLpaRp]{3.\ref*{FaxLpnQLpxRpRpImpPLrPOrQLpaRp}(#1)}}
    
    % ∃x(¬P(x))⊣⊢¬∀x(P(x))
    \newcommand{\ExxLpnPLpxRpRpEqvnFaxLpPLpxRpRp}[1]{%
      \hyperref[ExxLpnPLpxRpRpEqvnFaxLpPLpxRpRp]{2.\ref*{ExxLpnPLpxRpRpEqvnFaxLpPLpxRpRp}(#1)}%
    }
    
    % (¬∀x(P(x)→¬Q(x))⊣⊢∃x(P(x)∧Q(x))
    \newcommand{\LpnFaxLpPLpxRpTonQLpxRpRpEqvExxLpPLpxRpAndQLpxRpRp}[1]{%
      \hyperref[LpnFaxLpPLpxRpTonQLpxRpRpEqvExxLpPLpxRpAndQLpxRpRp]{2.\ref*{LpnFaxLpPLpxRpTonQLpxRpRpEqvExxLpPLpxRpAndQLpxRpRp}(#1)}%
    }
    
    % ¬∀x(P(x)→Q(x))⊣⊢∃x(P(x)∧¬Q(x))
    \newcommand{\nFaxLpPLpxRpToQLpxRpRpEqvExxLpPLpxRpAndnQLpxRpRp}[1]{%
      \hyperref[nFaxLpPLpxRpToQLpxRpRpEqvExxLpPLpxRpAndnQLpxRpRp]{2.\ref*{nFaxLpPLpxRpToQLpxRpRpEqvExxLpPLpxRpAndnQLpxRpRp}(#1)}%
    }
    
    % ∀x(P(x))⊣⊢¬∃x¬(P(x))
    \newcommand{\FaxLpPLpxRpRpEqvnExxnLpPLpxRpRp}[1]{%
      \hyperref[FaxLpPLpxRpRpEqvnExxnLpPLpxRpRp]{2.\ref*{FaxLpPLpxRpRpEqvnExxnLpPLpxRpRp}(#1)}%
    }
    
    % ∀x(¬P(x))⊣⊢¬∃x(P(x))
    \newcommand{\FaxLpnPLpxRpRpEqvnExxLpPLpxRpRp}[1]{%
      \hyperref[FaxLpnPLpxRpRpEqvnExxLpPLpxRpRp]{2.\ref*{FaxLpnPLpxRpRpEqvnExxLpPLpxRpRp}(#1)}%
    }
    
    % ∀x(P(x)→Q(x))⊣⊢¬∃x(P(x)∧¬Q(x))
    \newcommand{\FaxLpPLpxRpToQLpxRpRpEqvnExxLpPLpxRpAndnQLpxRpRp}[1]{%
      \hyperref[FaxLpPLpxRpToQLpxRpRpEqvnExxLpPLpxRpAndnQLpxRpRp]{2.\ref*{FaxLpPLpxRpToQLpxRpRpEqvnExxLpPLpxRpAndnQLpxRpRp}(#1)}%
    }
    
    % ∀x(P(x)→¬Q(x))⊣⊢¬∃x(P(x)∧Q(x))
    \newcommand{\FaxLpPLpxRpTonQLpxRpRpEqvnExxLpPLpxRpAndQLpxRpRp}[1]{%
      \hyperref[FaxLpPLpxRpTonQLpxRpRpEqvnExxLpPLpxRpAndQLpxRpRp]{2.\ref*{FaxLpPLpxRpTonQLpxRpRpEqvnExxLpPLpxRpAndQLpxRpRp}(#1)}%
    }
    
    % ∀x(P(x)↔Q(x))⊣⊢¬∃x(P(x)∧¬Q(x))∧¬∃x(Q(x)∧¬P(x))
    \newcommand{\FaxLpPLpxRpLrQLpxRpRpEqvnExxLpPLpxRpAndnQLpxRpRpAndnExxLpQLpxRpAndnPLpxRpRp}[1]{%
      \hyperref[FaxLpPLpxRpLrQLpxRpRpEqvnExxLpPLpxRpAndnQLpxRpRpAndnExxLpQLpxRpAndnPLpxRpRp]{2.\ref*{FaxLpPLpxRpLrQLpxRpRpEqvnExxLpPLpxRpAndnQLpxRpRpAndnExxLpQLpxRpAndnPLpxRpRp}(#1)}%
    }
    
    % ¬∀x(P(x)↔Q(x))⊣⊢∃x(P(x)∧¬Q(x))∨∃x(Q(x)∧¬P(x))
    \newcommand{\nFaxLpPLpxRpLrQLpxRpRpEqvExxLpPLpxRpAndnQLpxRpRpOrExxLpQLpxRpAndnPLpxRpRp}[1]{%
      \hyperref[nFaxLpPLpxRpLrQLpxRpRpEqvExxLpPLpxRpAndnQLpxRpRpOrExxLpQLpxRpAndnPLpxRpRp]{2.\ref*{nFaxLpPLpxRpLrQLpxRpRpEqvExxLpPLpxRpAndnQLpxRpRpOrExxLpQLpxRpAndnPLpxRpRp}(#1)}%
    }
    
    % ¬∀P(x)∃Q(y)(R(x,y))⊣⊢∃P(x)∀Q(y)(¬P(x,y))
    \newcommand{\nFaPLpxRpExQLpyRpLpRLpxwyRpRpEqvExPLpxRpFaQLpyRpLpnPLpxwyRpRp}[1]{%
      \hyperref[nFaPLpxRpExQLpyRpLpRLpxwyRpRpEqvExPLpxRpFaQLpyRpLpnPLpxwyRpRp]{2.\ref*{nFaPLpxRpExQLpyRpLpRLpxwyRpRpEqvExPLpxRpFaQLpyRpLpnPLpxwyRpRp}(#1)}%
    }

    % P(a),∀x(P(x)→Q(x))⊢Q(a)
    \newcommand{\PLpaRpwFaxLpPLpxRpToQLpxRpRpImpQLpaRp}[1]{\hyperref[PLpaRpwFaxLpPLpxRpToQLpxRpRpImpQLpaRp]{2.\ref*{PLpaRpwFaxLpPLpxRpToQLpxRpRpImpQLpaRp}(#1)}}
    
    % P(a),∀x(P(x)↔Q(x))⊢Q(a)
    \newcommand{\PLpaRpwFaxLpPLpxRpLrQLpxRpRpImpQLpaRp}[1]{%
      \hyperref[PLpaRpwFaxLpPLpxRpLrQLpxRpRpImpQLpaRp]{2.\ref*{PLpaRpwFaxLpPLpxRpLrQLpxRpRpImpQLpaRp}(#1)}%
    }
    
    % Q(a),∀x(P(x)↔Q(x))⊢P(a)
    \newcommand{\QLpaRpwFaxLpPLpxRpLrQLpxRpRpImpPLpaRp}[1]{%
      \hyperref[QLpaRpwFaxLpPLpxRpLrQLpxRpRpImpPLpaRp]{2.\ref*{QLpaRpwFaxLpPLpxRpLrQLpxRpRpImpPLpaRp}(#1)}%
    }
    
    % ∀x(P→F(x))⊣⊢P→∀x(F(x))
    \newcommand{\FaxLpPToFLpxRpRpEqvPToFaxLpFLpxRpRp}[1]{%
      \hyperref[FaxLpPToFLpxRpRpEqvPToFaxLpFLpxRpRp]{2.\ref*{FaxLpPToFLpxRpRpEqvPToFaxLpFLpxRpRp}(#1)}%
    }
    
    % ∀x(P∧F(x))⊣⊢P∧∀x(F(x))
    \newcommand{\FaxLpPAndFLpxRpRpEqvPAndFaxLpFLpxRpRp}[1]{%
      \hyperref[FaxLpPAndFLpxRpRpEqvPAndFaxLpFLpxRpRp]{2.\ref*{FaxLpPAndFLpxRpRpEqvPAndFaxLpFLpxRpRp}(#1)}%
    }
    
    % ∃x(P∧F(x))⊣⊢P∧∃x(F(x))
    \newcommand{\ExxLpPAndFLpxRpRpEqvPAndExxLpFLpxRpRp}[1]{%
      \hyperref[ExxLpPAndFLpxRpRpEqvPAndExxLpFLpxRpRp]{2.\ref*{ExxLpPAndFLpxRpRpEqvPAndExxLpFLpxRpRp}(#1)}%
    }
    
    % ∀x(P∨F(x))⊣⊢P∨∀x(F(x))
    \newcommand{\FaxLpPOrFLpxRpRpEqvPOrFaxLpFLpxRpRp}[1]{%
      \hyperref[FaxLpPOrFLpxRpRpEqvPOrFaxLpFLpxRpRp]{2.\ref*{FaxLpPOrFLpxRpRpEqvPOrFaxLpFLpxRpRp}(#1)}%
    }
    
    % ∃x(P∨F(x))⊣⊢P∨∃x(F(x))
    \newcommand{\ExxLpPOrFLpxRpRpEqvPOrExxLpFLpxRpRp}[1]{%
      \hyperref[ExxLpPOrFLpxRpRpEqvPOrExxLpFLpxRpRp]{2.\ref*{ExxLpPOrFLpxRpRpEqvPOrExxLpFLpxRpRp}(#1)}%
    }
    
    % ∃x(F(x)→P)⊣⊢∀x(F(x))→P
    \newcommand{\ExxLpFLpxRpToPRpEqvFaxLpFLpxRpRpToP}[1]{%
      \hyperref[ExxLpFLpxRpToPRpEqvFaxLpFLpxRpRpToP]{2.\ref*{ExxLpFLpxRpToPRpEqvFaxLpFLpxRpRpToP}(#1)}%
    }
    
    % ∀x(F(x)∧G(x))⊣⊢∀x(F(x))∧∀x(G(x))
    \newcommand{\FaxLpFLpxRpAndGLpxRpRpEqvFaxLpFLpxRpRpAndFaxLpGLpxRpRp}[1]{%
      \hyperref[FaxLpFLpxRpAndGLpxRpRpEqvFaxLpFLpxRpRpAndFaxLpGLpxRpRp]{2.\ref*{FaxLpFLpxRpAndGLpxRpRpEqvFaxLpFLpxRpRpAndFaxLpGLpxRpRp}(#1)}%
    }
    
    % ∀x(F(x))∨∀x(G(x))⊢∀x(F(x)∨G(x))
    \newcommand{\FaxLpFLpxRpRpOrFaxLpGLpxRpRpImpFaxLpFLpxRpOrGLpxRpRp}[1]{%
      \hyperref[FaxLpFLpxRpRpOrFaxLpGLpxRpRpImpFaxLpFLpxRpOrGLpxRpRp]{2.\ref*{FaxLpFLpxRpRpOrFaxLpGLpxRpRpImpFaxLpFLpxRpOrGLpxRpRp}(#1)}%
    }
    
    % ∃x(F(x)∨G(x))⊣⊢∃x(F(x))∨∃x(G(x))
    \newcommand{\ExxLpFLpxRpOrGLpxRpRpEqvExxLpFLpxRpRpOrExxLpGLpxRpRp}[1]{%
      \hyperref[ExxLpFLpxRpOrGLpxRpRpEqvExxLpFLpxRpRpOrExxLpGLpxRpRp]{2.\ref*{ExxLpFLpxRpOrGLpxRpRpEqvExxLpFLpxRpRpOrExxLpGLpxRpRp}(#1)}%
    }
    
    % ∃x∃yF(x,y)⊢∃y∃xF(x,y)
    \newcommand{\ExxExyFLpxwyRpImpExyExxFLpxwyRp}[1]{%
      \hyperref[ExxExyFLpxwyRpImpExyExxFLpxwyRp]{2.\ref*{ExxExyFLpxwyRpImpExyExxFLpxwyRp}(#1)}%
    }
    
    % ∀x∀yF(x,y)⊢∀y∀xF(x,y)
    \newcommand{\FaxFayFLpxwyRpImpFayFaxFLpxwyRp}[1]{%
      \hyperref[FaxFayFLpxwyRpImpFayFaxFLpxwyRp]{2.\ref*{FaxFayFLpxwyRpImpFayFaxFLpxwyRp}(#1)}%
    }
    
    % Theoreme zum Identitätssymbol
    
    % Fa⊣⊢∃x(x=a∧Fx)
    \newcommand{\FaEqvExxLpxIdaAndFxRp}[1]{%
      \hyperref[FaEqvExxLpxIdaAndFxRp]{2.\ref*{FaEqvExxLpxIdaAndFxRp}(#1)}%
    }

    % ∃c((c=a∨c=b)∧P(c))⊣⊢P(a)∨P(b)
    \newcommand{\ExcLpLpcEqualsaOrcEqualsbRpAndPLpcRpRpEqvPLpaRpOrPLpbRp}[1]{\hyperref[ExcLpLpcEqualsaOrcEqualsbRpAndPLpcRpRpEqvPLpaRpOrPLpbRp]{2.\ref*{ExcLpLpcEqualsaOrcEqualsbRpAndPLpcRpRpEqvPLpaRpOrPLpbRp}(#1)}}
    
    % a=b⊢b=a
    \newcommand{\aIdbImpbIda}[1]{%
      \hyperref[aIdbImpbIda]{2.\ref*{aIdbImpbIda}(#1)}%
    }
    
    % a=b,b=c⊢a=c
    \newcommand{\aIdbwbIdcImpaIdc}[1]{%
      \hyperref[aIdbwbIdcImpaIdc]{2.\ref*{aIdbwbIdcImpaIdc}(#1)}%
    }
    
    % a=b,c=b⊢a=c
    \newcommand{\aIdbwcIdbImpaIdc}[1]{%
      \hyperref[aIdbwcIdbImpaIdc]{2.\ref*{aIdbwcIdbImpaIdc}(#1)}%
    }
    
    % Theoreme zum Nicht-Gleichheitszeichen
    
    % a≠b⊢b≠a
    \newcommand{\aNotEqualsbImpbNotEqualsa}[1]{%
      \hyperref[aNotEqualsbImpbNotEqualsa]{2.\ref*{aNotEqualsbImpbNotEqualsa}(#1)}%
    }
    
    %%%
    %
    % Mengenlehre
    %
    %%%
    
    % Hinweis: Hier hatten manche schon [3;A(...)] oder [3;D(...)] statt T(...). 
    % Wir reduzieren trotzdem auf "3.\ref*{Label}(#1)"
    
    % A=B⊣⊢∀x(x∈A↔x∈B)
    \newcommand{\AIdBEqvFaxLpxInALrxInBRp}[1]{%
      \hyperref[AIdBEqvFaxLpxInALrxInBRp]{3.\ref*{AIdBEqvFaxLpxInALrxInBRp}(#1)}%
    }
    
    % ∃O(∀x(x∉O))
    \newcommand{\ExOLpFaxLpxNotinORpRp}[1]{%
      \hyperref[ExOLpFaxLpxNotinORpRp]{3.\ref*{ExOLpFaxLpxNotinORpRp}(#1)}%
    }
    
    % x∉∅
    \newcommand{\xNotinEmptyset}[1]{%
      \hyperref[xNotinEmptyset]{3.\ref*{xNotinEmptyset}(#1)}%
    }
    
    % x∈{x∈A|P(x)}⊣⊢x∈A∧P(x)
    \newcommand{\xInLbxInAMidPLpxRpRbEqvxInAAndPLpxRp}[1]{%
      \hyperref[xInLbxInAMidPLpxRpRbEqvxInAAndPLpxRp]{3.\ref*{xInLbxInAMidPLpxRpRbEqvxInAAndPLpxRp}(#1)}%
    }

    % ∀A,B∃C(∀x(x∈C↔x=A∨x=B))
    \newcommand{\FaAwBExCLpFaxLpxInCLrxEqualsAOrxEqualsBRpRp}[1]{\hyperref[FaAwBExCLpFaxLpxInCLrxEqualsAOrxEqualsBRpRp]{3.\ref*{FaAwBExCLpFaxLpxInCLrxEqualsAOrxEqualsBRpRp}(#1)}}

    % x∈{A,B}
    \newcommand{\DefxInLbAwBRb}[1]{\hyperref[DefxInLbAwBRb]{3.\ref*{DefxInLbAwBRb}(#1)}}

    % (a,b)
    \newcommand{\DefLpawbRp}[1]{\hyperref[DefLpawbRp]{3.\ref*{DefLpawbRp}(#1)}}

    % x∈⋃A
    \newcommand{\DefxInBigcupA}[1]{\hyperref[DefxInBigcupA]{3.\ref*{DefxInBigcupA}(#1)}}
    
    % ∃B(∀x(x∈B↔x∈A∧P(x)))
    \newcommand{\ExBLpFaxLpxInBLrxInAAndPLpxRpRpRp}[1]{%
      \hyperref[ExBLpFaxLpxInBLrxInAAndPLpxRpRpRp]{3.\ref*{ExBLpFaxLpxInBLrxInAAndPLpxRpRpRp}(#1)}%
    }
    
    % Def(⊆)
    \newcommand{\DefLpSubseteqRp}[1]{%
      \hyperref[DefLpSubseteqRp]{3.\ref*{DefLpSubseteqRp}(#1)}%
    }

    % Def(℘)
    \newcommand{\DefPowerset}[1]{\hyperref[DefPowerset]{3.\ref*{DefPowerset}(#1)}}
    
    % A≠B⊣⊢∃x(x∉A∧x∈B)∨∃x(x∈A∧x∉B)
    \newcommand{\ANotIdBEqvExxLpxNotInAAndxInBRpOrExxLpxInAAndxNotInBRp}[1]{%
      \hyperref[ANotIdBEqvExxLpxNotInAAndxInBRpOrExxLpxInAAndxNotInBRp]{3.\ref*{ANotIdBEqvExxLpxNotInAAndxInBRpOrExxLpxInAAndxNotInBRp}(#1)}%
    }
    
    % ∃!O∀x(x∉O)
    \newcommand{\ExonlyoneOFaxLpxNotinORp}[1]{%
      \hyperref[ExonlyoneOFaxLpxNotinORp]{3.\ref*{ExonlyoneOFaxLpxNotinORp}(#1)}%
    }
    
    % x∈A∧x∉B⊢A≠B
    \newcommand{\xInAAndxNotinBImpANotEqualsB}[1]{%
      \hyperref[xInAAndxNotinBImpANotEqualsB]{3.\ref*{xInAAndxNotinBImpANotEqualsB}(#1)}%
    }
    
    % ∀x(x∈B↔P(x))∧∀x(x∈C↔P(x))⊢B=C
    \newcommand{\FaxLpxInBLrPLpxRpRpAndFaxLpxInCLrPLpxRpRpImpBEqualsC}[1]{%
      \hyperref[FaxLpxInBLrPLpxRpRpAndFaxLpxInCLrPLpxRpRpImpBEqualsC]{3.\ref*{FaxLpxInBLrPLpxRpRpAndFaxLpxInCLrPLpxRpRpImpBEqualsC}(#1)}%
    }
    
    % ∃B(∀x(x∈B↔P(x)))⊢∃!B(∀x(x∈B↔P(x)))
    \newcommand{\ExBLpFaxLpxInBLrPLpxRpRpRpImpExonlyoneBLpFaxLpxInBLrPLpxRpRpRp}[1]{%
      \hyperref[ExBLpFaxLpxInBLrPLpxRpRpRpImpExonlyoneBLpFaxLpxInBLrPLpxRpRpRp]{3.\ref*{ExBLpFaxLpxInBLrPLpxRpRpRpImpExonlyoneBLpFaxLpxInBLrPLpxRpRpRp}(#1)}%
    }

    % A≠∅⊢∃x∈A(x∩A=∅)
    \newcommand{\ANotEqualsEmptysetImpExxInALpxcaAEqualsEmptysetRp}[1]{\hyperref[ANotEqualsEmptysetImpExxInALpxcaAEqualsEmptysetRp]{3.\ref*{ANotEqualsEmptysetImpExxInALpxcaAEqualsEmptysetRp}(#1)}}

    % ∀x(P(x)→x∈A)⊢∃!B(∀x(x∈B↔P(x)))
    \newcommand{\FaxLpPLpxRpToxInARpImpExonlyoneBLpFaxLpxInBLrPLpxRpRpRp}[1]{\hyperref[FaxLpPLpxRpToxInARpImpExonlyoneBLpFaxLpxInBLrPLpxRpRpRp]{3.\ref*{FaxLpPLpxRpToxInARpImpExonlyoneBLpFaxLpxInBLrPLpxRpRpRp}(#1)}}
    
    % A⊆B,x∈A⊢x∈B
    \newcommand{\ASubseteqBwxInAImpxInB}[1]{%
      \hyperref[ASubseteqBwxInAImpxInB]{3.\ref*{ASubseteqBwxInAImpxInB}(#1)}%
    }
    
    % A=B,x∈A⊢x∈B
    \newcommand{\AEqualsBwxInAImpxInB}[1]{%
      \hyperref[AEqualsBwxInAImpxInB]{3.\ref*{AEqualsBwxInAImpxInB}(#1)}%
    }

    % a∈A,b∉A⊢a≠b
    \newcommand{\aInAwbNotinAImpaNotEqualsb}[1]{\hyperref[aInAwbNotinAImpaNotEqualsb]{3.\ref*{aInAwbNotinAImpaNotEqualsb}(#1)}}
    
    % A⊆A
    \newcommand{\ASubseteqA}[1]{%
      \hyperref[ASubseteqA]{3.\ref*{ASubseteqA}(#1)}%
    }
    
    % A⊆B∧B⊆A⊣⊢A=B
    \newcommand{\ASubseteqBAndBSubseteqAEqvAEqualsB}[1]{%
      \hyperref[ASubseteqBAndBSubseteqAEqvAEqualsB]{3.\ref*{ASubseteqBAndBSubseteqAEqvAEqualsB}(#1)}%
    }
    
    % A⊆B,B⊆C⊢A⊆C
    \newcommand{\ASubseteqBwBSubseteqCImpASubseteqC}[1]{%
      \hyperref[ASubseteqBwBSubseteqCImpASubseteqC]{3.\ref*{ASubseteqBwBSubseteqCImpASubseteqC}(#1)}%
    }
    
    % A⊆B,B=C⊢A⊆C
    \newcommand{\ASubseteqBwBEqualsCImpASubseteqC}[1]{%
      \hyperref[ASubseteqBwBEqualsCImpASubseteqC]{3.\ref*{ASubseteqBwBEqualsCImpASubseteqC}(#1)}%
    }
    
    % ∀A(∅⊆A)
    \newcommand{\FaALpEmptysetSubseteqARp}[1]{%
      \hyperref[FaALpEmptysetSubseteqARp]{3.\ref*{FaALpEmptysetSubseteqARp}(#1)}%
    }
    
    % A⊆B,∀x∈B(x∉A)⊢A=∅
    \newcommand{\ASubseteqBwFaxInBLpxNotinARpImpAEqualsEmptyset}[1]{%
      \hyperref[ASubseteqBwFaxInBLpxNotinARpImpAEqualsEmptyset]{3.\ref*{ASubseteqBwFaxInBLpxNotinARpImpAEqualsEmptyset}(#1)}%
    }

    % a∈S⊢S≠∅
    \newcommand{\aInSImpSNotEqualsEmptyset}[1]{\hyperref[aInSImpSNotEqualsEmptyset]{3.\ref*{aInSImpSNotEqualsEmptyset}(#1)}}
    
    % ∃x(x∈S)⊢S≠∅
    \newcommand{\ExxLpxInSRpImpSNotEqualsEmptyset}[1]{%
      \hyperref[ExxLpxInSRpImpSNotEqualsEmptyset]{3.\ref*{ExxLpxInSRpImpSNotEqualsEmptyset}(#1)}%
    }
    
    % ∀x(x∉{x∈A|P(x)}↔(x∉A∨¬P(x)))
    \newcommand{\FaxLpxNotinLbxInAMidPLpxRpRbLrLpxNotinAOrnPLpxRpRpRp}[1]{%
      \hyperref[FaxLpxNotinLbxInAMidPLpxRpRbLrLpxNotinAOrnPLpxRpRpRp]{3.\ref*{FaxLpxNotinLbxInAMidPLpxRpRbLrLpxNotinAOrnPLpxRpRpRp}(#1)}%
    }
    
    % {x∈A|P(x)}⊆A
    \newcommand{\ImpLbxInAMidPLpxRpRbSubseteqA}[1]{%
      \hyperref[ImpLbxInAMidPLpxRpRbSubseteqA]{3.\ref*{ImpLbxInAMidPLpxRpRbSubseteqA}(#1)}%
    }
    
    % ∀x∈A(P(x)),y∈A⊢P(y)
    \newcommand{\FaxInALpPLpxRpRpwyInAImpPLpyRp}[1]{%
      \hyperref[FaxInALpPLpxRpRpwyInAImpPLpyRp]{3.\ref*{FaxInALpPLpxRpRpwyInAImpPLpyRp}(#1)}%
    }
    
    % ∀x∈M(P(x))⊣⊢M={x∈M|P(x)}
    \newcommand{\FaxInMLpPLpxRpRpEqvMEqualsLbxInMMidPLpxRpRb}[1]{%
      \hyperref[FaxInMLpPLpxRpRpEqvMEqualsLbxInMMidPLpxRpRb]{\ensuremath{\forall x \in M(P(x)) \dashv\vdash M = \{x \in M \mid P(x)\}}(#1)}%
    }
    
    % Definition des Schnitts
    % A∩B:={x∈A|x∈B}
    \newcommand{\DefLpCapRp}[1]{%
      \hyperref[DefLpCapRp]{3.\ref*{DefLpCapRp}(#1)}%
    }
    
    % x∈A∩B⊢x∈A
    \newcommand{\xInAcaBImpxInA}[1]{%
      \hyperref[xInAcaBImpxInA]{3.\ref*{xInAcaBImpxInA}(#1)}%
    }
    
    % x∈A∩B⊢x∈B
    \newcommand{\xInAcaBImpxInB}[1]{%
      \hyperref[xInAcaBImpxInB]{3.\ref*{xInAcaBImpxInB}(#1)}%
    }
    
    % x∈A∩B⊣⊢x∈A∧x∈B
    \newcommand{\xInAcaBEqvxInAAndxInB}[1]{%
      \hyperref[xInAcaBEqvxInAAndxInB]{3.\ref*{xInAcaBEqvxInAAndxInB}(#1)}%
    }
    
    % x∈(A∩B)∩C⊣⊢(x∈A∧x∈B)∧x∈C
    \newcommand{\xInLpAcaBRpcaCEqvLpxInAAndxInBRpAndxInC}[1]{%
      \hyperref[xInLpAcaBRpcaCEqvLpxInAAndxInBRpAndxInC]{3.\ref*{xInLpAcaBRpcaCEqvLpxInAAndxInBRpAndxInC}(#1)}%
    }
    
    % x∈A∩(B∩C)⊣⊢x∈A∧(x∈B∧x∈C)
    \newcommand{\xInAcaLpBcaCRpEqvxInAAndLpxInBAndxInCRp}[1]{%
      \hyperref[xInAcaLpBcaCRpEqvxInAAndLpxInBAndxInCRp]{3.\ref*{xInAcaLpBcaCRpEqvxInAAndLpxInBAndxInCRp}(#1)}%
    }
    
    % A=A∩A
    \newcommand{\AEqualsAcaA}[1]{%
      \hyperref[AEqualsAcaA]{3.\ref*{AEqualsAcaA}(#1)}%
    }
    
    % A∩B=B∩A
    \newcommand{\AcaBEqualsBcaA}[1]{%
      \hyperref[AcaBEqualsBcaA]{3.\ref*{AcaBEqualsBcaA}(#1)}%
    }
    
    % (A∩B)∩C=A∩(B∩C)
    \newcommand{\LpAcaBRpcaCEqualsAcaLpBcaCRp}[1]{%
      \hyperref[LpAcaBRpcaCEqualsAcaLpBcaCRp]{3.\ref*{LpAcaBRpcaCEqualsAcaLpBcaCRp}(#1)}%
    }
    
    % ¬∃U∀A(A∈U)
    \newcommand{\nExUFaALpAInURp}[1]{%
      \hyperref[nExUFaALpAInURp]{3.\ref*{nExUFaALpAInURp}(#1)}%
    }
    
    % R:={x∈U|x∉x}
    \newcommand{\RDefineEqualsLbxInUMidxNotinxRb}[1]{%
      \hyperref[RDefineEqualsLbxInUMidxNotinxRb]{3.\ref*{RDefineEqualsLbxInUMidxNotinxRb}(#1)}%
    }
    
    % A∩B⊆A
    \newcommand{\AcaBSubseteqA}[1]{%
      \hyperref[AcaBSubseteqA]{3.\ref*{AcaBSubseteqA}(#1)}%
    }
    
    % A∩B⊆B
    \newcommand{\AcaBSubseteqB}[1]{%
      \hyperref[AcaBSubseteqB]{3.\ref*{AcaBSubseteqB}(#1)}%
    }
    
    % A⊆B,x∈A⊢x∈B
    \newcommand{\ASubseteqBImpAcaCSubseteqBcaC}[1]{%
      \hyperref[ASubseteqBImpAcaCSubseteqBcaC]{3.\ref*{ASubseteqBImpAcaCSubseteqBcaC}(#1)}%
    }
    
    % A⊆B⊣⊢A∩B=A
    \newcommand{\ASubseteqBEqvAcaBEqualsA}[1]{%
      \hyperref[ASubseteqBEqvAcaBEqualsA]{3.\ref*{ASubseteqBEqvAcaBEqualsA}(#1)}%
    }
    
    % B⊆A⊣⊢A∩B=B
    \newcommand{\BSubseteqAEqvAcaBEqualsB}[1]{%
      \hyperref[BSubseteqAEqvAcaBEqualsB]{3.\ref*{BSubseteqAEqvAcaBEqualsB}(#1)}%
    }
    
    % ∅∩A=∅
    \newcommand{\EmptysetcaAEqualsEmptyset}[1]{%
      \hyperref[EmptysetcaAEqualsEmptyset]{3.\ref*{EmptysetcaAEqualsEmptyset}(#1)}%
    }
    
    % A∩∅=∅
    \newcommand{\AcaEmptysetEqualsEmptyset}[1]{%
      \hyperref[AcaEmptysetEqualsEmptyset]{3.\ref*{AcaEmptysetEqualsEmptyset}(#1)}%
    }
    
    % A∩B=∅,x∈A⊢x∉B
    \newcommand{\AcaBEqualsEmptysetwxInAImpxNotinB}[1]{%
      \hyperref[AcaBEqualsEmptysetwxInAImpxNotinB]{3.\ref*{AcaBEqualsEmptysetwxInAImpxNotinB}(#1)}%
    }
    
    % A∩B=∅,x∈B⊢x∉A
    \newcommand{\AcaBEqualsEmptysetwxInBImpxNotinA}[1]{%
      \hyperref[AcaBEqualsEmptysetwxInBImpxNotinA]{3.\ref*{AcaBEqualsEmptysetwxInBImpxNotinA}(#1)}%
    }
    
    % Der Unendliche Schnitt
    % I_B
    \newcommand{\DefBigcapLbAMidPLpARpRbMidSubB}[1]{%
      \hyperref[DefBigcapLbAMidPLpARpRbMidSubB]{3.\ref*{DefBigcapLbAMidPLpARpRbMidSubB}(#1)}%
    }
    
    % P(C)⊢I_B⊆I_C
    \newcommand{\PLpCRpImpISubBSubseteqISubC}[1]{\hyperref[PLpCRpImpISubBSubseteqISubC]{3.\ref*{PLpCRpImpISubBSubseteqISubC}(#1)}}
    
    % P(B),P(C)⊢I_B=I_C
    \newcommand{\PLpBRpwPLpCRpImpISubBEqualsISubC}[1]{\hyperref[PLpBRpwPLpCRpImpISubBEqualsISubC]{3.\ref*{PLpBRpwPLpCRpImpISubBEqualsISubC}(#1)}}

    % ∃D(P(D))⊢∃C∀B(P(B)→C=I_B)
    \newcommand{\ExDLpPLpDRpRpImpExCFaBLpPLpBRpToCEqualsISubBRp}[1]{\hyperref[ExDLpPLpDRpRpImpExCFaBLpPLpBRpToCEqualsISubBRp]{3.\ref*{ExDLpPLpDRpRpImpExCFaBLpPLpBRpToCEqualsISubBRp}(#1)}}

    % ∀D(¬P(D))⊢∃C∀B(P(B)→C=I_B)
    \newcommand{\FaDLpnPLpDRpRpImpExCFaBLpPLpBRpToCEqualsISubBRp}[1]{\hyperref[FaDLpnPLpDRpRpImpExCFaBLpPLpBRpToCEqualsISubBRp]{3.\ref*{FaDLpnPLpDRpRpImpExCFaBLpPLpBRpToCEqualsISubBRp}(#1)}}
    
    % P(D)⊢∃C∀B(P(B)→C={x∈B|∀A(P(A)→x∈A)})
    \newcommand{\PLpDRpImpExCFaBLpPLpBRpToCEqualsLbxInBMidFaALpPLpARpToxInARpRb}[1]{%
      \hyperref[PLpDRpImpExCFaBLpPLpBRpToCEqualsLbxInBMidFaALpPLpARpToxInARpRb]{3.\ref*{PLpDRpImpExCFaBLpPLpBRpToCEqualsLbxInBMidFaALpPLpARpToxInARpRb}(#1)}%
    }

    % P(B_0),∀B(P(B)→C=I_B),∀B(P(B)→D=I_B)⊢C=D
    \newcommand{\PLpBSubZeroRpwFaBLpPLpBRpToCEqualsISubBRpwFaBLpPLpBRpToDEqualsISubBRpImpCEqualsD}[1]{\hyperref[PLpBSubZeroRpwFaBLpPLpBRpToCEqualsISubBRpwFaBLpPLpBRpToDEqualsISubBRpImpCEqualsD]{3.\ref*{PLpBSubZeroRpwFaBLpPLpBRpToCEqualsISubBRpwFaBLpPLpBRpToDEqualsISubBRpImpCEqualsD}(#1)}}

    % P(D)⊢∃!C∀B(P(B)→C=I_B)
    \newcommand{\PLpDRpImpExonlyoneCFaBLpPLpBRpToCEqualsISubBRp}[1]{\hyperref[PLpDRpImpExonlyoneCFaBLpPLpBRpToCEqualsISubBRp]{3.\ref*{PLpDRpImpExonlyoneCFaBLpPLpBRpToCEqualsISubBRp}(#1)}}

    % ⋂_{P(A)}A
    \newcommand{\DefBigcapSubLbPLpARpRbA}[1]{\hyperref[DefBigcapSubLbPLpARpRbA]{3.\ref*{DefBigcapSubLbPLpARpRbA}(#1)}}

    % P(A)⊢⋂_{P(B)}B=I_A
    \newcommand{\PLpARpImpBigcapSubLbPLpBRpRbBEqualsISubA}[1]{\hyperref[PLpARpImpBigcapSubLbPLpBRpRbBEqualsISubA]{3.\ref*{PLpARpImpBigcapSubLbPLpBRpRbBEqualsISubA}(#1)}}

    % P(A)⊢x∈⋂_{P(B)}B↔∀C(P(C)→x∈C)
    \newcommand{\PLpARpImpxInBigcapSubLbPLpBRpRbBLrFaCLpPLpCRpToxInCRp}[1]{\hyperref[PLpARpImpxInBigcapSubLbPLpBRpRbBLrFaCLpPLpCRpToxInCRp]{3.\ref*{PLpARpImpxInBigcapSubLbPLpBRpRbBLrFaCLpPLpCRpToxInCRp}(#1)}}


    % P(C)⊢⋂_{P(A)}A⊆C
    \newcommand{\PLpCRpImpBigcapSubLbPLpARpRbASubseteqC}[1]{\hyperref[PLpCRpImpBigcapSubLbPLpARpRbASubseteqC]{\ensuremath{P(C)\vdash \bigcap_{P(A)} A\subseteq C}(#1)}}

    % ∀A∀B∃C∀x(x∈C↔(x=A∨x=B))
    \newcommand{\FaAFaBExCFaxLpxInCLrLpxEqualsAOrxEqualsBRpRp}[1]{\hyperref[FaAFaBExCFaxLpxInCLrLpxEqualsAOrxEqualsBRpRp]{\ensuremath{\forall A \forall B \exists C \forall x (x \in C \leftrightarrow (x = A \lor x = B))}(#1)}}

    % ∀x(x∈C↔(x=A∨x=B))∧∀x(x∈D↔(x=A∨x=B))⊢C=D
    \newcommand{\FaxLpxInCLrLpxEqualsAOrxEqualsBRpRpAndFaxLpxInDLrLpxEqualsAOrxEqualsBRpRpImpCEqualsD}[1]{\hyperref[FaxLpxInCLrLpxEqualsAOrxEqualsBRpRpAndFaxLpxInDLrLpxEqualsAOrxEqualsBRpRpImpCEqualsD]{\ensuremath{\forall x (x \in C \leftrightarrow (x = A \lor x = B)) \land \forall x (x \in D \leftrightarrow (x = A \lor x = B)) \vdash C = D}(#1)}}

	
    % Einführung und Elimination der Paarmenge
    \newcommand{\pairI}[1]{\hyperref[rule:pairI]{\ensuremath{\in}I2(#1)}}
    \newcommand{\pairIb}{\hyperref[rule:pairIb]{\ensuremath{\in}I2}}
    \newcommand{\pairE}[1]{\hyperref[rule:pairE]{\ensuremath{\in}E2(#1)}}
    \newcommand{\nPairI}[1]{\hyperref[rule:nPairI]{\ensuremath{\notin}I2(#1)}}
    \newcommand{\nPairE}[1]{\hyperref[rule:nPairE]{\ensuremath{\notin}E2(#1)}}

    % Die Paarmenge ist eine Menge

    % x∉{a,b}⊣⊢x≠a∧x≠b
    \newcommand{\xNotinLbawbRbEqvxNotEqualsaAndxNotEqualsb}[1]{\hyperref[xNotinLbawbRbEqvxNotEqualsaAndxNotEqualsb]{3.\ref*{xNotinLbawbRbEqvxNotEqualsaAndxNotEqualsb}(#1)}}

    % a∈{a,b}
    \newcommand{\aInLbawbRb}[1]{\hyperref[aInLbawbRb]{3.\ref*{aInLbawbRb}(#1)}}

    % b∈{a,b}
    \newcommand{\bInLbawbRb}[1]{\hyperref[bInLbawbRb]{3.\ref*{bInLbawbRb}(#1)}}

    % {a,b}={b,a}
    \newcommand{\LbawbRbEqualsLbbwaRb}[1]{\hyperref[LbawbRbEqualsLbbwaRb]{3.\ref*{LbawbRbEqualsLbbwaRb}(#1)}}

    % {a,b}≠∅
    \newcommand{\LbawbRbNotEqualsEmptyset}[1]{\hyperref[LbawbRbNotEqualsEmptyset]{3.\ref*{LbawbRbNotEqualsEmptyset}(#1)}}

    % Def {a}:={a,a}
    \newcommand{\DefLbaRb}[1]{\hyperref[DefLbaRb]{3.\ref*{DefLbaRb}(#1)}}

    % a∈{a}
    \newcommand{\aInLbaRb}[1]{\hyperref[aInLbaRb]{3.\ref*{aInLbaRb}(#1)}}

    % x∈{a}⊣⊢x=a
    \newcommand{\xInLbaRbEqvxEqualsa}[1]{\hyperref[xInLbaRbEqvxEqualsa]{3.\ref*{xInLbaRbEqvxEqualsa}(#1)}}

    % x∉{a}⊣⊢x≠a
    \newcommand{\xNotinLbaRbEqvxNotEqualsa}[1]{\hyperref[xNotinLbaRbEqvxNotEqualsa]{3.\ref*{xNotinLbaRbEqvxNotEqualsa}(#1)}}

    % a∈A⊢{a}⊆A
    \newcommand{\aInAImpLbaRbSubseteqA}[1]{\hyperref[aInAImpLbaRbSubseteqA]{3.\ref*{aInAImpLbaRbSubseteqA}(#1)}}
    
    % Einführung und Elimination der Differenzmenge
    % A\B
    \newcommand{\DefASetminusB}[1]{\hyperref[DefASetminusB]{3.\ref*{DefASetminusB}(#1)}}

    % x∈A\B↔x∈A∧x∉B
    \newcommand{\xInASetminusBLrxInAAndxNotinB}[1]{\hyperref[xInASetminusBLrxInAAndxNotinB]{3.\ref*{xInASetminusBLrxInAAndxNotinB}(#1)}}

    % c∈A\{a}⊢c≠a
    \newcommand{\cInASetminusLbaRbImpcNotEqualsa}[1]{\hyperref[cInASetminusLbaRbImpcNotEqualsa]{3.\ref*{cInASetminusLbaRbImpcNotEqualsa}(#1)}}

    % c∈A\{a,b}⊢c≠a 
    \newcommand{\cInASetminusLbawbRbImpcNotEqualsa}[1]{\hyperref[cInASetminusLbawbRbImpcNotEqualsa]{3.\ref*{cInASetminusLbawbRbImpcNotEqualsa}(#1)}}

    % c∈A\{a,b}⊢c≠b
    \newcommand{\cInASetminusLbawbRbImpcNotEqualsb}[1]{\hyperref[cInASetminusLbawbRbImpcNotEqualsb]{3.\ref*{cInASetminusLbawbRbImpcNotEqualsb}(#1)}}

    % c∈A\B,b∈B⊢c≠b
    \newcommand{\cInASetminusBwbInBImpcNotEqualsb}[1]{\hyperref[cInASetminusBwbInBImpcNotEqualsb]{3.\ref*{cInASetminusBwbInBImpcNotEqualsb}(#1)}}

    % a∉A\{a}
    \newcommand{\aNotinASetminusLbaRb}[1]{\hyperref[aNotinASetminusLbaRb]{3.\ref*{aNotinASetminusLbaRb}(#1)}}

    % a∉A\{a,b}
    \newcommand{\aNotinASetminusLbawbRb}[1]{\hyperref[aNotinASetminusLbawbRb]{3.\ref*{aNotinASetminusLbawbRb}(#1)}}

    % b∉A\{a,b}
    \newcommand{\bNotinASetminusLbawbRb}[1]{\hyperref[bNotinASetminusLbawbRb]{3.\ref*{bNotinASetminusLbawbRb}(#1)}}

    % a∈A,a≠b⊢a∈A\{b}
    \newcommand{\aInAwaNotEqualsbImpaInASetminusLbbRb}[1]{\hyperref[aInAwaNotEqualsbImpaInASetminusLbbRb]{3.\ref*{aInAwaNotEqualsbImpaInASetminusLbbRb}(#1)}}

    % a∈A,b≠a⊢a∈A\{b}
    \newcommand{\aInAwbNotEqualsaImpaInASetminusLbbRb}[1]{\hyperref[aInAwbNotEqualsaImpaInASetminusLbbRb]{3.\ref*{aInAwbNotEqualsaImpaInASetminusLbbRb}(#1)}}

    % Definition A∪B
    \newcommand{\DefAcuB}[1]{\hyperref[DefAcuB]{3.\ref*{DefAcuB}(#1)}}

    % A⊆B⊢⋃A⊆⋃B
    \newcommand{\ASubseteqBImpBigcupASubseteqBigcupB}[1]{\hyperref[ASubseteqBImpBigcupASubseteqBigcupB]{3.\ref*{ASubseteqBImpBigcupASubseteqBigcupB}(#1)}}

    % z∈A∪B⊣⊢(z∈A∨z∈B)
    \newcommand{\zInAcuBEqvLpzInAOrzInBRp}[1]{\hyperref[zInAcuBEqvLpzInAOrzInBRp]{3.\ref*{zInAcuBEqvLpzInAOrzInBRp}(#1)}}

    % z∈A⊢z∈A∪B
    \newcommand{\zInAImpzInAcuB}[1]{\hyperref[zInAImpzInAcuB]{3.\ref*{zInAImpzInAcuB}(#1)}}

    % z∈B⊢z∈A∪B
    \newcommand{\zInBImpzInAcuB}[1]{\hyperref[zInBImpzInAcuB]{3.\ref*{zInBImpzInAcuB}(#1)}}

    % x∈{a,b}⊣⊢x∈{a}∪{b}
    \newcommand{\xInLbawbRbEqvxInLbaRbcuLbbRb}[1]{\hyperref[xInLbawbRbEqvxInLbaRbcuLbbRb]{3.\ref*{xInLbawbRbEqvxInLbaRbcuLbbRb}(#1)}}

    % {a,b}={a}∪{b}
    \newcommand{\LbawbRbEqualsLbaRbcuLbbRb}[1]{\hyperref[LbawbRbEqualsLbaRbcuLbbRb]{3.\ref*{LbawbRbEqualsLbaRbcuLbbRb}(#1)}}

    % a∈A∪{a}
    \newcommand{\aInAcuLbaRb}[1]{\hyperref[aInAcuLbaRb]{3.\ref*{aInAcuLbaRb}(#1)}}

    % a∈{a}∪A
    \newcommand{\aInLbaRbcuA}[1]{\hyperref[aInLbaRbcuA]{3.\ref*{aInLbaRbcuA}(#1)}}

    % a∉B,A=B∪{a}⊢A⊈B
    \newcommand{\aNotinBwAEqualsBcuLbaRbImpANsubseteqB}[1]{\hyperref[aNotinBwAEqualsBcuLbaRbImpANsubseteqB]{3.\ref*{aNotinBwAEqualsBcuLbaRbImpANsubseteqB}(#1)}}

    % x∈A⊣⊢x∈A∪A
    \newcommand{\xInAEqvxInAcuA}[1]{\hyperref[xInAEqvxInAcuA]{3.\ref*{xInAEqvxInAcuA}(#1)}}

    % A=A∪A
    \newcommand{\AEqualsAcuA}[1]{\hyperref[AEqualsAcuA]{3.\ref*{AEqualsAcuA}(#1)}}

    % x∈A∪B=x∈B∪A
    \newcommand{\xInAcuBEqualsxInBcuA}[1]{\hyperref[xInAcuBEqualsxInBcuA]{3.\ref*{xInAcuBEqualsxInBcuA}(#1)}}

    % A∪B=B∪A
    \newcommand{\AcuBEqualsBcuA}[1]{\hyperref[AcuBEqualsBcuA]{3.\ref*{AcuBEqualsBcuA}(#1)}}

    % x∈A⊣⊢x∈A∪∅
    \newcommand{\xInAEqvxInAcuEmptyset}[1]{\hyperref[xInAEqvxInAcuEmptyset]{3.\ref*{xInAEqvxInAcuEmptyset}(#1)}}
    
    % A∪B = B∪A
    \newcommand{\kommCup}[1]{\hyperref[rule:kommCup]{\ensuremath{\cup}Komm.(#1)}}

    % A=A∪∅
    \newcommand{\AEqualsAcuEmptyset}[1]{\hyperref[AEqualsAcuEmptyset]{3.\ref*{AEqualsAcuEmptyset}(#1)}}

    % A=∅∪A
    \newcommand{\AEqualsEmptysetcuA}[1]{\hyperref[AEqualsEmptysetcuA]{3.\ref*{AEqualsEmptysetcuA}(#1)}}

    % A∪{A}={∅}=⊣⊢A=∅
    \newcommand{\AcuLbARbEqualsLbEmptysetRbEqualsEqvAEqualsEmptyset}[1]{\hyperref[AcuLbARbEqualsLbEmptysetRbEqualsEqvAEqualsEmptyset]{3.\ref*{AcuLbARbEqualsLbEmptysetRbEqualsEqvAEqualsEmptyset}(#1)}}

    % z∈A∪B⊣⊢z∉A→z∈B
    \newcommand{\zInAcuBEqvzNotinATozInB}[1]{\hyperref[zInAcuBEqvzNotinATozInB]{3.\ref*{zInAcuBEqvzNotinATozInB}(#1)}}

    % z∈A∪B⊣⊢z∉B→z∈A
    \newcommand{\zInAcuBEqvzNotinBTozInA}[1]{\hyperref[zInAcuBEqvzNotinBTozInA]{\ensuremath{z\in A \cup B \dashv\vdash z\notin B \rightarrow z\in A}(#1)}}

    % z∈(A∪B)∪C⊣⊢(z∈A∨z∈B)∨z∈C
    \newcommand{\zInLpAcuBRpcuCEqvLpzInAOrzInBRpOrzInC}[1]{\hyperref[zInLpAcuBRpcuCEqvLpzInAOrzInBRpOrzInC]{3.\ref*{zInLpAcuBRpcuCEqvLpzInAOrzInBRpOrzInC}(#1)}}

    % z∈A∪(B∪C)⊣⊢z∈A∨(z∈B∨z∈C)
    \newcommand{\zInAcuLpBcuCRpEqvzInAOrLpzInBOrzInCRp}[1]{\hyperref[zInAcuLpBcuCRpEqvzInAOrLpzInBOrzInCRp]{3.\ref*{zInAcuLpBcuCRpEqvzInAOrLpzInBOrzInCRp}(#1)}}

    % z∈(A∪B)∪C⊣⊢z∈A∪(B∪C)
    \newcommand{\zInLpAcuBRpcuCEqvzInAcuLpBcuCRp}[1]{\hyperref[zInLpAcuBRpcuCEqvzInAcuLpBcuCRp]{3.\ref*{zInLpAcuBRpcuCEqvzInAcuLpBcuCRp}(#1)}}

    % (A∪B)∪C=A∪(B∪C)
    \newcommand{\LpAcuBRpcuCEqualsAcuLpBcuCRp}[1]{\hyperref[LpAcuBRpcuCEqualsAcuLpBcuCRp]{3.\ref*{LpAcuBRpcuCEqualsAcuLpBcuCRp}(#1)}}

    % z∈A∪(B∩C)⊣⊢z∈A∨(z∈B∧z∈C)
    \newcommand{\zInAcuLpBcaCRpEqvzInAOrLpzInBAndzInCRp}[1]{\hyperref[zInAcuLpBcaCRpEqvzInAOrLpzInBAndzInCRp]{3.\ref*{zInAcuLpBcaCRpEqvzInAOrLpzInBAndzInCRp}(#1)}}

    % z∈(A∩B)∪C⊣⊢(z∈A∧z∈B)∨z∈C
    \newcommand{\zInLpAcaBRpcuCEqvLpzInAAndzInBRpOrzInC}[1]{\hyperref[zInLpAcaBRpcuCEqvLpzInAAndzInBRpOrzInC]{3.\ref*{zInLpAcaBRpcuCEqvLpzInAAndzInBRpOrzInC}(#1)}}


    % z∈A∩(B∪C)⊣⊢z∈A∧(z∈B∨z∈C)
    \newcommand{\zInAcaLpBcuCRpEqvzInAAndLpzInBOrzInCRp}[1]{\hyperref[zInAcaLpBcuCRpEqvzInAAndLpzInBOrzInCRp]{3.\ref*{zInAcaLpBcuCRpEqvzInAAndLpzInBOrzInCRp}(#1)}}

    % z∈(A∪B)∩C⊣⊢z∈(z∈A∨z∈C)∧z∈C
    \newcommand{\zInLpAcuBRpcaCEqvzInLpzInAOrzInCRpAndzInC}[1]{\hyperref[zInLpAcuBRpcaCEqvzInLpzInAOrzInCRpAndzInC]{3.\ref*{zInLpAcuBRpcaCEqvzInLpzInAOrzInCRpAndzInC}(#1)}}

    % z∈(A∪B)∩(C∪D)⊣⊢(z∈A∨z∈B)∧(z∈C∨z∈D)
    \newcommand{\zInLpAcuBRpcaLpCcuDRpEqvLpzInAOrzInBRpAndLpzInCOrzInDRp}[1]{\hyperref[zInLpAcuBRpcaLpCcuDRpEqvLpzInAOrzInBRpAndLpzInCOrzInDRp]{3.\ref*{zInLpAcuBRpcaLpCcuDRpEqvLpzInAOrzInBRpAndLpzInCOrzInDRp}(#1)}}

    % z∈(A∩B)∪(C∩D)⊣⊢(z∈A∧z∈B)∨(z∈C∧z∈D)
    \newcommand{\zInLpAcaBRpcuLpCcaDRpEqvLpzInAAndzInBRpOrLpzInCAndzInDRp}[1]{\hyperref[zInLpAcaBRpcuLpCcaDRpEqvLpzInAAndzInBRpOrLpzInCAndzInDRp]{3.\ref*{zInLpAcaBRpcuLpCcaDRpEqvLpzInAAndzInBRpOrLpzInCAndzInDRp}(#1)}}

    % z∈A∩(B∪C)⊣⊢z∈(A∩B)∪(A∩C)
    \newcommand{\zInAcaLpBcuCRpEqvzInLpAcaBRpcuLpAcaCRp}[1]{\hyperref[zInAcaLpBcuCRpEqvzInLpAcaBRpcuLpAcaCRp]{3.\ref*{zInAcaLpBcuCRpEqvzInLpAcaBRpcuLpAcaCRp}(#1)}}

    % z∈(A∪B)∩C⊣⊢z∈(A∩C)∪(B∩C)
    \newcommand{\zInLpAcuBRpcaCEqvzInLpAcaCRpcuLpBcaCRp}[1]{\hyperref[zInLpAcuBRpcaCEqvzInLpAcaCRpcuLpBcaCRp]{3.\ref*{zInLpAcuBRpcaCEqvzInLpAcaCRpcuLpBcaCRp}(#1)}}

    % z∈(A∩B)∪C⊣⊢z∈(A∪C)∩(B∪C)
    \newcommand{\zInLpAcaBRpcuCEqvzInLpAcuCRpcaLpBcuCRp}[1]{\hyperref[zInLpAcaBRpcuCEqvzInLpAcuCRpcaLpBcuCRp]{3.\ref*{zInLpAcaBRpcuCEqvzInLpAcuCRpcaLpBcuCRp}(#1)}}

    % z∈A∪(B∩C)⊣⊢z∈(A∪B)∩(A∪C)
    \newcommand{\zInAcuLpBcaCRpEqvzInLpAcuBRpcaLpAcuCRp}[1]{\hyperref[zInAcuLpBcaCRpEqvzInLpAcuBRpcaLpAcuCRp]{3.\ref*{zInAcuLpBcaCRpEqvzInLpAcuBRpcaLpAcuCRp}(#1)}}



    % A∪(B∩C)=(A∪B)∩(A∪C)
    \newcommand{\AcuLpBcaCRpEqualsLpAcuBRpcaLpAcuCRp}[1]{\hyperref[AcuLpBcaCRpEqualsLpAcuBRpcaLpAcuCRp]{3.\ref*{AcuLpBcaCRpEqualsLpAcuBRpcaLpAcuCRp}(#1)}}

    % (A∩B)∪C=(A∪C)∩(B∪C)
    \newcommand{\LpAcaBRpcuCEqualsLpAcuCRpcaLpBcuCRp}[1]{\hyperref[LpAcaBRpcuCEqualsLpAcuCRpcaLpBcuCRp]{3.\ref*{LpAcaBRpcuCEqualsLpAcuCRpcaLpBcuCRp}(#1)}}

    % A∩(B∪C)=(A∩B)∪(A∩C)
    \newcommand{\AcaLpBcuCRpEqualsLpAcaBRpcuLpAcaCRp}[1]{\hyperref[AcaLpBcuCRpEqualsLpAcaBRpcuLpAcaCRp]{3.\ref*{AcaLpBcuCRpEqualsLpAcaBRpcuLpAcaCRp}(#1)}}

    % (A∪B)∩C=(A∩C)∪(B∩C)
    \newcommand{\LpAcuBRpcaCEqualsLpAcaCRpcuLpBcaCRp}[1]{\hyperref[LpAcuBRpcaCEqualsLpAcaCRpcuLpBcaCRp]{3.\ref*{LpAcuBRpcaCEqualsLpAcaCRpcuLpBcaCRp}(#1)}}

    % A⊆A∪B
    \newcommand{\ASubseteqAcuB}[1]{\hyperref[ASubseteqAcuB]{3.\ref*{ASubseteqAcuB}(#1)}}

    % A⊆B∪A
    \newcommand{\ASubseteqBcuA}[1]{\hyperref[ASubseteqBcuA]{3.\ref*{ASubseteqBcuA}(#1)}}

    % A⊆C,B⊆C⊢A∪B⊆C
    \newcommand{\ASubseteqCwBSubseteqCImpAcuBSubseteqC}[1]{\hyperref[ASubseteqCwBSubseteqCImpAcuBSubseteqC]{3.\ref*{ASubseteqCwBSubseteqCImpAcuBSubseteqC}(#1)}}

    % A⊆B⊢A∪C⊆B∪C
    \newcommand{\ASubseteqBImpAcuCSubseteqBcuC}[1]{\hyperref[ASubseteqBImpAcuCSubseteqBcuC]{3.\ref*{ASubseteqBImpAcuCSubseteqBcuC}(#1)}}

    % A⊆B⊢C∪A⊆C∪B
    \newcommand{\ASubseteqBImpCcuASubseteqCcuB}[1]{\hyperref[ASubseteqBImpCcuASubseteqCcuB]{3.\ref*{ASubseteqBImpCcuASubseteqCcuB}(#1)}}

    % A⊆B,C⊆D⊢A∪C⊆B∪D
    \newcommand{\ASubseteqBwCSubseteqDImpAcuCSubseteqBcuD}[1]{\hyperref[ASubseteqBwCSubseteqDImpAcuCSubseteqBcuD]{3.\ref*{ASubseteqBwCSubseteqDImpAcuCSubseteqBcuD}(#1)}}

    % A⊆B,C⊆D⊢A∩C⊆B∩D
    \newcommand{\ASubseteqBwCSubseteqDImpAcaCSubseteqBcaD}[1]{\hyperref[ASubseteqBwCSubseteqDImpAcaCSubseteqBcaD]{3.\ref*{ASubseteqBwCSubseteqDImpAcaCSubseteqBcaD}(#1)}}

    % a∈A,b∈B⊢{a,b}⊆A∪B
    \newcommand{\aInAwbInBImpLbawbRbSubseteqAcuB}[1]{\hyperref[aInAwbInBImpLbawbRbSubseteqAcuB]{3.\ref*{aInAwbInBImpLbawbRbSubseteqAcuB}(#1)}}

    % a∈A,b∈A⊢{a,b}⊆A
    \newcommand{\aInAwbInAImpLbawbRbSubseteqA}[1]{\hyperref[aInAwbInAImpLbawbRbSubseteqA]{3.\ref*{aInAwbInAImpLbawbRbSubseteqA}(#1)}}

    % Die Potenzmenge

    % A⊆B⊢℘(A)⊆℘(B)
    \newcommand{\ASubseteqBImpPowersetLpARpSubseteqPowersetLpBRp}[1]{\hyperref[ASubseteqBImpPowersetLpARpSubseteqPowersetLpBRp]{3.\ref*{ASubseteqBImpPowersetLpARpSubseteqPowersetLpBRp}(#1)}}

    % a∈A⊢{a}∈℘(A)
    \newcommand{\aInAImpLbaRbInPowersetLpARp}[1]{\hyperref[aInAImpLbaRbInPowersetLpARp]{3.\ref*{aInAImpLbaRbInPowersetLpARp}(#1)}}

    % A⊆B,a∈℘(A)⊢a∈℘(B)
    \newcommand{\ASubseteqBwaInPowersetLpARpImpaInPowersetLpBRp}[1]{\hyperref[ASubseteqBwaInPowersetLpARpImpaInPowersetLpBRp]{3.\ref*{ASubseteqBwaInPowersetLpARpImpaInPowersetLpBRp}(#1)}}

    % a∈℘(A)⊢∀B(a∈℘(A∪B))
    \newcommand{\aInPowersetLpARpImpFaBLpaInPowersetLpAcuBRpRp}[1]{\hyperref[aInPowersetLpARpImpFaBLpaInPowersetLpAcuBRpRp]{3.\ref*{aInPowersetLpARpImpFaBLpaInPowersetLpAcuBRpRp}(#1)}}

    % a∈A,b∈B⊢(a,b)∈℘(℘(A∪B))
    \newcommand{\aInAwbInBImpLpawbRpInPowersetLpPowersetLpAcuBRpRp}[1]{\hyperref[aInAwbInBImpLpawbRpInPowersetLpPowersetLpAcuBRpRp]{3.\ref*{aInAwbInBImpLpawbRpInPowersetLpPowersetLpAcuBRpRp}(#1)}}

    % ∃!C∀(a,b)((a,b)∈C↔(a=A∧b∈B))
    \newcommand{\ExonlyoneCFaLpawbRpLpLpawbRpInCLrLpaEqualsAAndbInBRpRp}[1]{\hyperref[ExonlyoneCFaLpawbRpLpLpawbRpInCLrLpaEqualsAAndbInBRpRp]{3.\ref*{ExonlyoneCFaLpawbRpLpLpawbRpInCLrLpaEqualsAAndbInBRpRp}(#1)}}

    % Def. AxB
    \newcommand{\DefAtiB}[1]{\hyperref[DefAtiB]{3.\ref*{DefAtiB}(#1)}}
    
    % a∈b⊢b∉a
    \newcommand{\aInbImpbNotina}[1]{\hyperref[aInbImpbNotina]{\ensuremath{a\in b \vdash b\notin a}(#1)}}


    %Regeln für die natürlichen Zahlen
    \newcommand{\NaturalI}[1]{\hyperref[rule:NaturalI]{\ensuremath{\in\mathbb{N}}I(#1)}}
    \newcommand{\NaturalE}[1]{\hyperref[rule:NaturalE]{\ensuremath{\in\mathbb{N}}E(#1)}}
    
    %Regeln für die natürlichen Zahlen
    \newcommand{\NotNaturalI}[1]{\hyperref[rule:NaturalI]{\ensuremath{\notin\mathbb{N}}I(#1)}}
    \newcommand{\NotNaturalE}[1]{\hyperref[rule:NaturalE]{\ensuremath{\notin\mathbb{N}}E(#1)}}

    % ∅∈ℕ
    \newcommand{\EmptysetInNatural}[1]{\hyperref[EmptysetInNatural]{\ensuremath{\emptyset\in\mathbb{N}}(#1)}}

    % n∈ℕ⊢n∪{n}∈ℕ
    \newcommand{\nInNaturalImpncuLbnRbInNatural}[1]{\hyperref[nInNaturalImpncuLbnRbInNatural]{\ensuremath{n\in\mathbb{N}\vdash n\cup\{n\}\in\mathbb{N}}(#1)}}

    % 0:=∅
    \newcommand{\ZeroDefineEqualsEmptyset}[1]{\hyperref[ZeroDefineEqualsEmptyset]{\ensuremath{0:=\emptyset}(#1)}}

    % 1:=0∪{0}
    \newcommand{\OneDefineEqualsZerocuLbZeroRb}[1]{\hyperref[OneDefineEqualsZerocuLbZeroRb]{\ensuremath{1 := 0\cup \{0\}}(#1)}}

    % Regeln für die Definition der natürlichen Zahlen
    
    % 0:=∅
    \newcommand{\zeroSetDefinition}{\hyperref[rule:zeroSetDefinition]{\ensuremath{0 := \emptyset}}}
    
    % 1:=0∪{0}
    \newcommand{\oneSetDefinition}{\hyperref[rule:oneSetDefinition]{\ensuremath{1 := 0 \cup \{0\}}}}
    
    % n∈ℕ→n+1:=n∪{n}
    \newcommand{\successorSetDefinition}[1]{\hyperref[rule:successorSetDefinition]{\ensuremath{n+1 := n \cup \{n\}}(#1)}}

    % 1={∅}
    \newcommand{\OneEqualsLbEmptysetRb}[1]{\hyperref[OneEqualsLbEmptysetRb]{\ensuremath{1=\{\emptyset\}}(#1)}}

    % n∈ℕ⊢n∈n+1
    \newcommand{\nInNaturalImpnInnPlusOne}[1]{\hyperref[nInNaturalImpnInnPlusOne]{\ensuremath{n\in\mathbb{N}\vdash n\in n+1}(#1)}}

    % n∈ℕ⊢n+1≠0
    \newcommand{\nInNaturalImpnPlusOneNotEqualsZero}[1]{\hyperref[nInNaturalImpnPlusOneNotEqualsZero]{\ensuremath{n\in\mathbb{N}\vdash n+1\neq 0}(#1)}}

    % ⊢1≠0
    \newcommand{\ImpOneNotEqualsZero}[1]{\hyperref[ImpOneNotEqualsZero]{\ensuremath{\vdash 1\neq 0}(#1)}}

    % Regeln für die Zugehörigkeit zu natürlichen Zahlen
    \newcommand{\zeroIsNaturalNumber}{\hyperref[rule:zeroIsNaturalNumber]{\ensuremath{0\in\mathbb{N}}}}
    \newcommand{\oneIsNaturalNumber}{\hyperref[rule:oneIsNaturalNumber]{\ensuremath{1\in\mathbb{N}}}}
    \newcommand{\successorIsNaturalNumber}[1]{\hyperref[rule:successorIsNaturalNumber]{\ensuremath{(n+1)\in\mathbb{N}}(#1)}}

    % m∈ℕ,m≠0⊢∃x∈ℕ(x+1=m)
    \newcommand{\mInNaturalwmNotEqualsZeroImpExxInNaturalLpxPlusOneEqualsmRp}[1]{\hyperref[mInNaturalwmNotEqualsZeroImpExxInNaturalLpxPlusOneEqualsmRp]{\ensuremath{m\in\mathbb{N}, m\neq 0\vdash\exists x\in\mathbb{N}(x+1=m)}(#1)}}


    % m∈ℕ,m≠0⊢∃!x∈ℕ(x+1=m)
    \newcommand{\mInNaturalwmNotEqualsZeroImpExonlyonexInNaturalLpxPlusOneEqualsmRp}[1]{\hyperref[mInNaturalwmNotEqualsZeroImpExonlyonexInNaturalLpxPlusOneEqualsmRp]{\ensuremath{m\in\mathbb{N}, m\neq 0\vdash\exists! x\in\mathbb{N}(x+1=m)}(#1)}}

    
    %(n-1)∈ℕ
    \newcommand{\rPredecessorI}[1]{\hyperref[rule:rPredecessorI]{\ensuremath{(n-1)\in\mathbb{N}}(#1)}}

    % n-1→n
    \newcommand{\rPredecessorEa}[1]{\hyperref[rule:rPredecessorEa]{\ensuremath{n-1\rightarrow n}(#1)}}	

    % (n-1)∈ℕ→n≠0
    \newcommand{\rPredecessorEb}[1]{\hyperref[rule:rPredecessorEb]{\ensuremath{(n-1)\in\mathbb{N}\rightarrow n\neq 0}(#1)}}

    % n=(n+1)-1
    \newcommand{\rPredecessorEc}[1]{\hyperref[rule:rPredecessorEc]{\ensuremath{n=(n+1)-1}(#1)}}	
    
    % m=n⊢m-1=n-1
    \newcommand{\rPredecessorUniqueness}[1]{\hyperref[rule:rPredecessorUniqueness]{m=n\rightarrow(m-1=n-1)(#1)}}

    % n∈ℕ,n≠0⊢n-1∈n
    \newcommand{\nInNaturalwnNotEqualsZeroImpnMinusOneInn}[1]{\hyperref[nInNaturalwnNotEqualsZeroImpnMinusOneInn]{\ensuremath{n\in\mathbb{N}, n\neq 0\vdash n-1\in n}(#1)}}

    % P(0),∀n∈ℕ(P(n)→P(n+1))⊢∀n∈ℕP(n)
    \newcommand{\PLpZeroRpwFanInNaturalLpPLpnRpToPLpnPlusOneRpRpImpFanInNaturalPLpnRp}[1]{\hyperref[PLpZeroRpwFanInNaturalLpPLpnRpToPLpnPlusOneRpRpImpFanInNaturalPLpnRp]{\ensuremath{P(0), \forall n \in \mathbb{N} (P(n) \rightarrow P(n+1)) \vdash \forall n \in \mathbb{N} P(n)}(#1)}}

    % s∈S⊢s∈ℕ
    \newcommand{\PLpZeroRpwFanInNaturalLpPLpnRpToPLpnPlusOneRpRpImpFanInNaturalPLpnRpLo}[1]{\hyperref[PLpZeroRpwFanInNaturalLpPLpnRpToPLpnPlusOneRpRpImpFanInNaturalPLpnRpLo]{\mbox{Lemma Nr. }\ensuremath{1}(#1)}}

    % s∈S⊢¬P(s)
    \newcommand{\PLpZeroRpwFanInNaturalLpPLpnRpToPLpnPlusOneRpRpImpFanInNaturalPLpnRpLoo}[1]{\hyperref[PLpZeroRpwFanInNaturalLpPLpnRpToPLpnPlusOneRpRpImpFanInNaturalPLpnRpLoo]{\mbox{Lemma Nr. }\ensuremath{2}(#1)}}

    % s∉S,s∈ℕ⊢P(s)
    \newcommand{\PLpZeroRpwFanInNaturalLpPLpnRpToPLpnPlusOneRpRpImpFanInNaturalPLpnRpLooo}[1]{\hyperref[PLpZeroRpwFanInNaturalLpPLpnRpToPLpnPlusOneRpRpImpFanInNaturalPLpnRpLooo]{\mbox{Lemma Nr. }\ensuremath{3}(#1)}}
    
    % ¬(∀n∈ℕP(n))⊢S≠∅
    \newcommand{\PLpZeroRpwFanInNaturalLpPLpnRpToPLpnPlusOneRpRpImpFanInNaturalPLpnRpLoooo}[1]{\hyperref[PLpZeroRpwFanInNaturalLpPLpnRpToPLpnPlusOneRpRpImpFanInNaturalPLpnRpLoooo]{\mbox{Lemma Nr. }\ensuremath{4}(#1)}}
    
    % S≠∅,P(0)⊢∃x∈SP(s-1)
    \newcommand{\PLpZeroRpwFanInNaturalLpPLpnRpToPLpnPlusOneRpRpImpFanInNaturalPLpnRpLooooo}[1]{\hyperref[PLpZeroRpwFanInNaturalLpPLpnRpToPLpnPlusOneRpRpImpFanInNaturalPLpnRpLooooo]{\mbox{Lemma Nr. }\ensuremath{5}(#1)}}

    % Regel für das Prinzip der vollständigen Induktion über ℕ
    \newcommand{\rInductionN}[1]{\hyperref[rule:rInductionN]{\ensuremath{P(0),P(n)\rightarrow P(n+1)}(#1)}}

    % Regel für das n-fache karthesische Produkt
    \newcommand{\zeroPowerSet}[1]{\hyperref[rule:zeroPowerSet]{\ensuremath{A^0=\emptyset}(#1)}}
    
    \newcommand{\nextPowerSet}[1]{\hyperref[rule:nextPowerSet]{\ensuremath{A^{n+1}=A^n\times A}(#1)}}

    % Das n-fache karthesische Produkt
    \newcommand{\PowerSetExists}[1]{\hyperref[rule:PowerSetExists]{\ensuremath{A^n\text{ ist eine Menge}({#1})}}}


    % Einführungsregel für Funktionen (f: A -> B)
    \newcommand{\toI}[1]{\hyperref[rule:toI]{\ensuremath{f:A\rightarrow B}(#1)}}

    % Eliminierungsregel für Funktionen (f(a) = b)
    \newcommand{\toE}[1]{\hyperref[rule:toE]{\ensuremath{f:A\rightarrow B}(#1)}}

    % Einführungsregel binärer Operationen
    \newcommand{\cdotI}[1]{\hyperref[rule:cdotI]{\ensuremath{\cdot I}(#1)}}

    % Einführungsregel für Injektivität
    \newcommand{\InjI}[1]{\hyperref[rule:InjI]{\ensuremath{\text{Inj} \, I}(#1)}}

    % Eliminierungsregel für Injektivität
    \newcommand{\InjE}[1]{\hyperref[rule:InjE]{\ensuremath{\text{Inj} \, E}(#1)}}

    % Einführungsregel für Surjektivität
    \newcommand{\SurjI}[1]{\hyperref[rule:SurjI]{\ensuremath{\text{Surj} \, I}(#1)}}

    % Eliminierungsregel für Surjektivität
    \newcommand{\SurjE}[1]{\hyperref[rule:SurjE]{\ensuremath{\text{Surj} \, E}(#1)}}

    % Einführungsregel für Bijektivität
    \newcommand{\BijectionI}[1]{\hyperref[rule:BijectionI]{\ensuremath{\text{Bij} \, I}(#1)}}

    % Eliminierungsregel für Bijektivität
    \newcommand{\BijectionE}[1]{\hyperref[rule:BijectionE]{\ensuremath{\text{Bij} \, E}(#1)}}

    % Halbgruppeneinführung
    \newcommand{\rSemigroupI}[1]{\hyperref[rule:rSemigroupI]{\text{HG\textsubscript{I}}(#1)}}
    
    % Halbgruppenelimination
    \newcommand{\rSemigroupE}[1]{\hyperref[rule:rSemigroupI]{\text{HG\textsubscript{E}}(#1)}}

    % Regel der Assoziativität
    \newcommand{\rAssociativityHG}[1]{\hyperref[rule:rAssociativityHG]{\text{Assoziativität}(#1)}}

    % Abelsche Halbgruppeneinführung
    \newcommand{\rAbelianSemigroupI}[1]{\hyperref[rule:rAbelianSemigroupI]{\text{AHG\textsubscript{I}}(#1)}}
    
    % Abelsche Halbgruppenelimination
    \newcommand{\rAbelianSemigroupE}[1]{\hyperref[rule:rAbelianSemigroupE]{\text{AHG\textsubscript{E}}(#1)}}    

    % Kommutativität
    \newcommand{\rCommutativeSemigroup}[1]{\hyperref[rule:rCommutativeSemigroup]{\text{Kommutativität}(#1)}}

    % Monoideinführung
    \newcommand{\rMonoidI}[1]{\hyperref[rule:rMonoidI]{\text{Monoid\textsubscript{I}}(#1)}}
    
    % Monoidelimination
    \newcommand{\rMonoidE}[1]{\hyperref[rule:rMonoidE]{\text{Monoid\textsubscript{E}}(#1)}}  

    % Regel der Assoziativität
    \newcommand{\rAssociativityMonoid}[1]{\hyperref[rule:rAssociativityMonoid]{\text{Regel der Assoziativität}(#1)}}

    % Regel des neutralen Elements
    \newcommand{\rNeutralElementMonoid}[1]{\hyperref[rule:rNeutralElementMonoid]{\text{Regel des neutralen Elements}(#1)}}

    % ∃e,e'∈M(∀a∈M(e·a=a·e=a∧e'·a=a·e'=a))⊢e=e'
    \newcommand{\ExeweApostropheInMLpFaaInMLpeMultaEqualsaMulteEqualsaAndeApostropheMultaEqualsaMulteApostropheEqualsaRpRpImpeEqualseApostrophe}[1]{\hyperref[ExeweApostropheInMLpFaaInMLpeMultaEqualsaMulteEqualsaAndeApostropheMultaEqualsaMulteApostropheEqualsaRpRpImpeEqualseApostrophe]{\ensuremath{\exists e, e' \in M (\forall a \in M (e \cdot a = a \cdot e = a \land e' \cdot a = a \cdot e' = a)) \vdash e = e'}(#1)}}

    % (M,·,e)∈Monoid.⊢(M,·)∈SemiGroup.
    \newcommand{\LpMwMultweRpInMonoidImpLpMwMultRpInSemiGroup}[1]{\hyperref[LpMwMultweRpInMonoidImpLpMwMultRpInSemiGroup]{\ensuremath{(M,\cdot, e) \text{ ist ein Monoid} \vdash (M,\cdot) \text{ ist eine Halbgruppe}}(#1)}}

    % Abelsche Monoideinführung
    \newcommand{\rAbelianMonoidI}[1]{\hyperref[rule:rAbelianMonoidI]{\text{abelschler Monoid\textsubscript{I}}(#1)}}
    
    % Abelsche Monoidelimination
    \newcommand{\rAbelianMonoidE}[1]{\hyperref[rule:rAbelianMonoidE]{\text{abelschler Monoid\textsubscript{E}}(#1)}}  

    % Kommutativität
    \newcommand{\rCommutativeMonoid}[1]{\hyperref[rule:rCommutativeMonoid]{\text{Kommutativität}(#1)}}

    % (M,·,e)∈AbelMonoid.⊢(M,·)∈AbelSemiGroup.
    \newcommand{\LpMwMultweRpInAbelMonoidImpLpMwMultRpInAbelSemiGroup}[1]{\hyperref[LpMwMultweRpInAbelMonoidImpLpMwMultRpInAbelSemiGroup]{\ensuremath{(M,\cdot, e) \text{ ist ein abelscher Monoid} \vdash (M,\cdot) \text{ ist eine abelsche Halbgruppe}}(#1)}}

    % a∈M,b∈M,c∈M⊢(a+b)+c=(a+c)+b
    \newcommand{\aInMwbInMwcInMImpLpaPlusbRpPluscEqualsLpaPluscRpPlusb}[1]{\hyperref[aInMwbInMwcInMImpLpaPlusbRpPluscEqualsLpaPluscRpPlusb]{\ensuremath{a,b, c\in M \vdash (ab)c=(ac)b}(#1)}}

    % M∈AbelMonoid,a,b,c∈M⊢(ab)c=(ca)b
    \newcommand{\MInAbelMonoidwawbwcInMImpLpabRpcEqualsLpcaRpb}[1]{\hyperref[MInAbelMonoidwawbwcInMImpLpabRpcEqualsLpcaRpb]{\ensuremath{M\text{ ist ein abelscher Monoid},a,b,c\in M \vdash (ab)c=(ca)b}(#1)}}

    % a∈M,b∈M,c∈M⊢a+(b+c)=(a+c)+b
    \newcommand{\aInMwbInMwcInMImpaPlusLpbPluscRpEqualsLpaPluscRpPlusb}[1]{\hyperref[aInMwbInMwcInMImpaPlusLpbPluscRpEqualsLpaPluscRpPlusb]{\ensuremath{a, b, c\in M \vdash a(bc)=(ac)b}(#1)}}

    % a∈M,b∈M,c∈M,d∈M⊢(a+b)+(c+d)=(a+c)+(b+d)
    \newcommand{\aInMwbInMwcInMwdInMImpLpaPlusbRpPlusLpcPlusdRpEqualsLpaPluscRpPlusLpbPlusdRp}[1]{\hyperref[aInMwbInMwcInMwdInMImpLpaPlusbRpPlusLpcPlusdRpEqualsLpaPluscRpPlusLpbPlusdRp]{\ensuremath{a, b, c, d\in M \vdash (ab)(cd)=(ac)(bd)}(#1)}}

    % Halbringeinführung
    \newcommand{\rSemiringI}[1]{\hyperref[rule:rSemiringI]{\text{Halbring\textsubscript{I}}(#1)}}
    
    % Halbringelimination
    \newcommand{\rSemiringE}[1]{\hyperref[rule:rSemiringE]{\text{Halbring\textsubscript{E}}(#1)}} 

    % Regel der Linksdistributivität
    \newcommand{\rLeftDistributiveSemigroup}[1]{\hyperref[rule:rLeftDistributiveSemigroup]{\text{Regel der Linksdistributivität}(#1)}}

    % Regel der Rechtsdistributivität
    \newcommand{\rRightDistributiveSemigroup}[1]{\hyperref[rule:rRightDistributiveSemigroup]{\text{Regel der Rechtsdistributivität}(#1)}}

    % abelsche Halbringeinführung
    \newcommand{\rAbelianSemiringI}[1]{\hyperref[rule:rAbelianSemiringI]{\text{abelscher Halbring\textsubscript{I}}(#1)}}
    
    % abelsche Halbringelimination
    \newcommand{\rAbelianSemiringE}[1]{\hyperref[rule:rAbelianSemiringE]{\text{abelscher Halbring\textsubscript{E}}(#1)}} 

    % Regel der Linksdistributivität
    \newcommand{\rLeftDistributiveAbelianSemigroup}[1]{\hyperref[rule:rLeftDistributiveAbelianSemigroup]{\text{Regel der Linksdistributivität}(#1)}}

    % Regel der Rechtsdistributivität
    \newcommand{\rRightDistributiveAbelianSemigroup}[1]{\hyperref[rule:rRightDistributiveAbelianSemigroup]{\text{Regel der Rechtsdistributivität}(#1)}}

    % Einführungsregeln für Summen- und Produktzeichen
    \newcommand{\rSumI}[1]{\hyperref[rule:sumIntro]{\ensuremath{\sum}I(#1)}}
    \newcommand{\rProdI}[1]{\hyperref[rule:prodIntro]{\ensuremath{\prod}I(#1)}}

    % Einführungsregel für Potenzen
    \newcommand{\rPowerI}[1]{\hyperref[rule:rPowerI]{\ensuremath{\text{Pow}_I}(#1)}}

    % Einführung binärer Relationen
    \newcommand{\rBinaryRelationI}[1]{\hyperref[rule:rBinaryRelationI]{\text{Binäre Relation\textsubscript{I}}(#1)}}

    % Eliminierung binärer Relationen
    \newcommand{\rBinaryRelationE}[1]{\hyperref[rule:rBinaryRelationE]{\text{Binäre Relation\textsubscript{E}}(#1)}}

    % Einführung Äquivalenzrelationen
    \newcommand{\rEquivalenceRelationI}[1]{\hyperref[rule:rEquivalenceRelationI]{\text{Äquivalenzrelation\textsubscript{I}}(#1)}}

    % Eliminierung Äquivalenzrelationen
    \newcommand{\rEquivalenceRelationE}[1]{\hyperref[rule:rEquivalenceRelationE]{\text{Äquivalenzrelation\textsubscript{E}}(#1)}}

    % Einführung Reflexivität
    \newcommand{\rReflexivityEqRI}[1]{\hyperref[rule:rReflexivityEqRI]{\text{Reflexiv\textsubscript{I}}(#1)}}

    % Einführung Symmetrie
    \newcommand{\rSymmetryEqRI}[1]{\hyperref[rule:rSymmetryEqRI]{\text{Symetrisch\textsubscript{I}}(#1)}}

    % Einführung Transitivität
    \newcommand{\rTransitivityEqRI}[1]{\hyperref[rule:rTransitivityEqRI]{\text{Transitiv\textsubscript{I}}(#1)}}

    % ∀S(=∈EquivalencerelationS)
    \newcommand{\FaSLpEqualsInEquivalencerelationSRp}[1]{\hyperref[FaSLpEqualsInEquivalencerelationSRp]{\ensuremath{\forall S(= \text{ ist eine Äquivalenzrelation auf } S)}(#1)}}

    % Einführung Halbordnungsrelationen
    \newcommand{\rPartialOrderRelationI}[1]{\hyperref[rule:rPartialOrderRelationI]{\text{Halbordnung\textsubscript{I}}(#1)}}

    % Eliminierung Halbordnungsrelationen
    \newcommand{\rPartialOrderRelationE}[1]{\hyperref[rule:rPartialOrderRelationE]{\text{Halbordnung\textsubscript{E}}(#1)}}
    
    % Einführung Reflexivität
    \newcommand{\rReflexivityOrdRI}[1]{\hyperref[rule:rReflexivityOrdRI]{\text{Reflexiv\textsubscript{I}}(#1)}}

    % Einführung Antisymmetrie
    \newcommand{\rAntisymmetryOrdRI}[1]{\hyperref[rule:rAntisymmetryOrdRI]{\text{Antisymmetrisch\textsubscript{I}}(#1)}}

    % Einführung Transitivität
    \newcommand{\rTransitivityOrdRI}[1]{\hyperref[rule:rTransitivityOrdRI]{\text{Transitiv\textsubscript{I}}(#1)}}

    % Einführung ≥
    \newcommand{\rgeqI}[1]{\hyperref[rule:rgeqI]{\ensuremath{\geq\text{I}}(#1)}}

    % Eliminierung ≥
    \newcommand{\rgeqE}[1]{\hyperref[rule:rgeqE]{\ensuremath{\geq\text{E}}(#1)}}

    % Einführung totale Ordnung
    \newcommand{\rTotalOrderI}[1]{\hyperref[rule:rTotalOrderI]{\text{TotaleOrdnung\textsubscript{I}}(#1)}}

    % Eliminierung totale Ordnung
    \newcommand{\rTotalOrderE}[1]{\hyperref[rule:rTotalOrderE]{\text{TotaleOrdnung\textsubscript{E}}(#1)}}

    % Einführung Totalität (Totale Strikte Ordnung)
    \newcommand{\rTotalityOrdRI}[1]{\hyperref[rule:rTotalityOrdRI]{\text{Totalität\textsubscript{I}}(#1)}}

    % Einführung Transitivität (Strikte Ordnung)
    \newcommand{\rTransitivityStrictRI}[1]{\hyperref[rule:rTransitivityStrictRI]{\text{Transitiv\textsubscript{I}}(#1)}}

    % Einführung >
    \newcommand{\rgtI}[1]{\hyperref[rule:rgtI]{\ensuremath{>\text{I}}(#1)}}

    % Eliminierung >
    \newcommand{\rgtE}[1]{\hyperref[rule:rgtE]{\ensuremath{>\text{E}}(#1)}}

    % a>b⊢a≥b
    \newcommand{\aGneqbImpaGeqb}[1]{\hyperref[aGneqbImpaGeqb]{\ensuremath{a > b\vdash a\geq b}(#1)}}

    % a<b⊢¬(b<a)
    \newcommand{\aLneqbImpnLpbLneqaRp}[1]{\hyperref[aLneqbImpnLpbLneqaRp]{\ensuremath{a<b\vdash \neg(b<a)}(#1)}}
    
    % a=b⊢¬(a<b)
    \newcommand{\aEqualsbImpnLpaLneqbRp}[1]{\hyperref[aEqualsbImpnLpaLneqbRp]{\ensuremath{a=b\vdash \neg(a<b)}(#1)}}

    % a=b⊢¬(b<a)
    \newcommand{\aEqualsbImpnLpbLneqaRp}[1]{\hyperref[aEqualsbImpnLpbLneqaRp]{\ensuremath{a=b\vdash \neg(b<a)}(#1)}}

    % Einführung Strikte Ordnung
    \newcommand{\rStrictOrderRelationI}[1]{\hyperref[rule:rStrictOrderRelationI]{\text{Strikt\textsubscript{I}}(#1)}}

    % Eliminierung Strikte Ordnung
    \newcommand{\rStrictOrderRelationE}[1]{\hyperref[rule:rStrictOrderRelationE]{\text{Strikt\textsubscript{E}}(#1)}}

    % Einführung Irreflexivität
    \newcommand{\rIrreflexivityStrictRI}[1]{\hyperref[rule:rIrreflexivityStrictRI]{\text{Irreflexiv\textsubscript{I}}(#1)}}

    % Einführung Totale Strikte Ordnung
    \newcommand{\rTotalStrictOrderI}[1]{\hyperref[rule:rTotalStrictOrderI]{\text{TotalStrikt\textsubscript{I}}(#1)}}

    % Eliminierung Totale Strikte Ordnung
    \newcommand{\rTotalStrictOrderE}[1]{\hyperref[rule:rTotalStrictOrderE]{\text{TotalStrikt\textsubscript{E}}(#1)}}

    % Einführung Totalität (Totale Strikte Ordnung)
    \newcommand{\rTotalityStrictRI}[1]{\hyperref[rule:rTotalityStrictRI]{\text{Totalität\textsubscript{I}}(#1)}}

    % Eliminierung in die induzierte strikte Ordnung
    \newcommand{\InducedStrictOrderE}[1]{\hyperref[rule:InducedStrictOrderE]{\ensuremath{<\text{E}}(#1)}}
    % Eliminierung in die induzierte strikte Ordnung
    \newcommand{\InducedStrictOrderI}[1]{\hyperref[rule:InducedStrictOrderI]{\ensuremath{<\text{I}}(#1)}}

    % ⊢¬(a<a)
    \newcommand{\InducedStrictOrderImpnLpaLneqaRp}[1]{\hyperref[InducedStrictOrderImpnLpaLneqaRp]{\ensuremath{\vdash \neg (a < a)}(#1)}}

    % a<b,b<c⊢a<c
    \newcommand{\InducedStrictOrderaLneqbwbLneqcImpaLneqc}[1]{\hyperref[InducedStrictOrderaLneqbwbLneqcImpaLneqc]{\ensuremath{a<b,b<c\vdash a < c}(#1)}}

    % a∈S,b∈S,a≤b∨b≤a⊢(a<b∨b<a)∨a=b
    \newcommand{\InducedStrictOrderFromHalfOrder}[1]{\hyperref[InducedStrictOrderFromHalfOrder]{\text{Induzierte Ordnung } \ensuremath{<}\text{ ist strikt}(#1)}}

    % ⊢(a<b∨b<a)∨a=b
    \newcommand{\ImpLpaLneqbOrbLneqaRpOraEqualsb}[1]{\hyperref[ImpLpaLneqbOrbLneqaRpOraEqualsb]{\ensuremath{\vdash (a<b\lor b<a)\lor a=b}(#1)}}

    % a<b∨(b<a∨a=b)
    \newcommand{\aLneqbOrLpbLneqaOraEqualsbRp}[1]{\hyperref[aLneqbOrLpbLneqaOraEqualsbRp]{\ensuremath{a<b\lor (b<a\lor a=b)}(#1)}}

    % (a≤b)⊣⊢a<b∨a=b
    \newcommand{\LpaLeqbRpEqvaLneqbOraEqualsb}[1]{\hyperref[LpaLeqbRpEqvaLneqbOraEqualsb]{\ensuremath{(a\leq b)\dashv\vdash a<b\lor a=b}(#1)}}

    % ¬(a<b)⊣⊢b≤a
    \newcommand{\nLpaLneqbRpEqvbLeqa}[1]{\hyperref[nLpaLneqbRpEqvbLeqa]{\ensuremath{\neg(a<b)\dashv\vdash b\leq a}(#1)}}

    % a<b⊣⊢¬(b≤a)
    \newcommand{\aLneqbEqvnLpbLeqaRp}[1]{\hyperref[aLneqbEqvnLpbLeqaRp]{\ensuremath{a<b\dashv\vdash \neg(b\leq a)}(#1)}}


    % Verschachtelte Ordnungen
    \newcommand{\rLeqLltI}[1]{\hyperref[rule:rLeqLltI]{\ensuremath{(a \leq b < c)_I}(#1)}}
    \newcommand{\rLeqLltE}[1]{\hyperref[rule:rLeqLltE]{\ensuremath{(a \leq b < c)_E}(#1)}}
    \newcommand{\rLltLeqI}[1]{\hyperref[rule:rLltLeqI]{\ensuremath{(a < b \leq c)_I}(#1)}}
    \newcommand{\rLltLeqE}[1]{\hyperref[rule:rLltLeqE]{\ensuremath{(a < b \leq c)_E}(#1)}}
    \newcommand{\rLltLltI}[1]{\hyperref[rule:rLltLltI]{\ensuremath{(a < b < c)_I}(#1)}}
    \newcommand{\rLltLltE}[1]{\hyperref[rule:rLltLltE]{\ensuremath{(a < b < c)_E}(#1)}}
    \newcommand{\rLeqLeqI}[1]{\hyperref[rule:rLeqLeqI]{\ensuremath{(a \leq b \leq c)_I}(#1)}}
    \newcommand{\rLeqLeqE}[1]{\hyperref[rule:rLeqLeqE]{\ensuremath{(a \leq b \leq c)_E}(#1)}}

    % Einführungsregeln für Intervalle
    \newcommand{\rClosedIntervalI}[1]{\hyperref[rule:rClosedIntervalI]{\ensuremath{x \in [a, b]}I(#1)}}
    \newcommand{\rOpenIntervalI}[1]{\hyperref[rule:rOpenIntervalI]{\ensuremath{x \in (a, b)}I(#1)}}
    \newcommand{\rClosedOpenIntervalI}[1]{\hyperref[rule:rClosedOpenIntervalI]{\ensuremath{x \in [a, b)}I(#1)}}
    \newcommand{\rOpenClosedIntervalI}[1]{\hyperref[rule:rOpenClosedIntervalI]{\ensuremath{x \in (a, b]}I(#1)}}

    % Eliminationssregeln für Intervalle

    \newcommand{\rClosedIntervalE}[1]{\hyperref[rule:rClosedIntervalE]{\ensuremath{x \in [a, b]}E(#1)}}
    \newcommand{\rOpenIntervalE}[1]{\hyperref[rule:rOpenIntervalE]{\ensuremath{x \in (a, b)}E(#1)}}
    \newcommand{\rClosedOpenIntervalE}[1]{\hyperref[rule:rClosedOpenIntervalE]{\ensuremath{x \in [a, b)}E(#1)}}
    \newcommand{\rOpenClosedIntervalE}[1]{\hyperref[rule:rOpenClosedIntervalE]{\ensuremath{x \in (a, b]}E(#1)}}

    % M∈S,N∈S,∀t∈S(t≤M),∀t∈S(t≤N)⊢M=N
    \newcommand{\MInSwNInSwFatInSLptLeqMRpwFatInSLptLeqNRpImpMEqualsN}[1]{\hyperref[MInSwNInSwFatInSLptLeqMRpwFatInSLptLeqNRpImpMEqualsN]{\ensuremath{M\in S, N\in S, \forall t \in S(t \leq M), \forall t \in S(t \leq N) \vdash M = N}(#1)}}

    % P∈S,Q∈S,∀t∈S(P≤t),∀t∈S(Q≤t)⊢P=Q
    \newcommand{\PInSwQInSwFatInSLpPLeqtRpwFatInSLpQLeqtRpImpPEqualsQ}[1]{\hyperref[PInSwQInSwFatInSLpPLeqtRpwFatInSLpQLeqtRpImpPEqualsQ]{\ensuremath{P \in S, Q \in S, \forall t \in S(P \leq t), \forall t \in S(Q \leq t) \vdash P = Q}(#1)}}

    % Regeln zur Einführung- und Elimination vom Minimum und Maximum
    \newcommand{\rMaxI}[1]{\hyperref[rule:rMaxI]{\ensuremath{\max(S)}I(#1)}}
    \newcommand{\rMaxE}[1]{\hyperref[rule:rMaxE]{\ensuremath{\max(S)}E(#1)}}
    \newcommand{\rMinI}[1]{\hyperref[rule:rMinI]{\ensuremath{\min(S)}I(#1)}}
    \newcommand{\rMinE}[1]{\hyperref[rule:rMinE]{\ensuremath{\min(S)}E(#1)}}

    % Regeln zur Einführung- und Elimination von oberen und unteren Schranken
    \newcommand{\rUBSI}[1]{\hyperref[rule:rUBSI]{\ensuremath{\text{UB}_S(T)}I(#1)}}
    \newcommand{\rUBSE}[1]{\hyperref[rule:rUBSE]{\ensuremath{\text{UB}_S(T)}E(#1)}}
    \newcommand{\rLBSI}[1]{\hyperref[rule:rLBSI]{\ensuremath{\text{LB}_S(T)}I(#1)}}
    \newcommand{\rLBSE}[1]{\hyperref[rule:rLBSE]{\ensuremath{\text{LB}_S(T)}E(#1)}}


    % u∈UB,v∈UB,∀w∈UB(u≤w),∀w∈UB(v≤w)⊢u=v
    \newcommand{\uInUBwvInUBwFawInUBLpuLeqwRpwFawInUBLpvLeqwRpImpuEqualsv}[1]{\hyperref[uInUBwvInUBwFawInUBLpuLeqwRpwFawInUBLpvLeqwRpImpuEqualsv]{\ensuremath{u \in \text{UB}_S(T), v \in \text{UB}_S(T), \forall w \in \text{UB}_S(T) (u \leq w), \forall w \in \text{UB}_S(T) (v \leq w) \vdash u = v}(#1)}}

    % l∈LB,m∈LB,∀n∈LB(n≤l),∀n∈LB(n≤m)⊢l=m
    \newcommand{\lInLBwmInLBwFanInLBLpnLeqlRpwFanInLBLpnLeqmRpImplEqualsm}[1]{\hyperref[lInLBwmInLBwFanInLBLpnLeqlRpwFanInLBLpnLeqmRpImplEqualsm]{\ensuremath{l \in \text{LB}_S(T), m \in \text{LB}_S(T), \forall n \in \text{LB}_S(T) (n \leq l), \forall n \in \text{LB}_S(T) (n \leq m) \vdash l = m}(#1)}}

    % Regeln zur Einführung- und Elimination von Supremum und Infimum
    \newcommand{\rSupSI}[1]{\hyperref[rule:rSupSI]{\ensuremath{\sup_S(T)}I(#1)}}
    \newcommand{\rSupSE}[1]{\hyperref[rule:rSupSE]{\ensuremath{\sup_S(T)}E(#1)}}
    \newcommand{\rInfSI}[1]{\hyperref[rule:rInfSI]{\ensuremath{\inf_S(T)}I(#1)}}
    \newcommand{\rInfSE}[1]{\hyperref[rule:rInfSE]{\ensuremath{\inf_S(T)}E(#1)}}

    % Regeln zur Einführung und Elimination der Vollständigkeit
    \newcommand{\rCompleteI}[1]{\hyperref[rule:rCompleteI]{\ensuremath{\text{Vollständig}(S)}I(#1)}}
    \newcommand{\rCompleteE}[1]{\hyperref[rule:rCompleteE]{\ensuremath{\text{Vollständig}(S)}E(#1)}}
    
    
    %Regeln für die Addition
    \newcommand{\rAddI}[1]{\hyperref[rule:rAddI]{\ensuremath{+}I(#1)}}

    % a∈ℕ,b∈ℕ⊢a+b∈ℕ
    \newcommand{\aInNaturalwbInNaturalImpaPlusbInNatural}[1]{\hyperref[aInNaturalwbInNaturalImpaPlusbInNatural]{\ensuremath{a\in\mathbb{N}, b\in\mathbb{N}\vdash a+b\in\mathbb{N}}(#1)}}

    % a∈ℕ⊢a=0+a
    \newcommand{\aInNaturalImpaEqualsZeroPlusa}[1]{\hyperref[aInNaturalImpaEqualsZeroPlusa]{\ensuremath{a\in\mathbb{N}\vdash a=0+a}(#1)}}

    % a∈ℕ⊢a+0=0+a
    \newcommand{\aInNaturalImpaPlusZeroEqualsZeroPlusa}[1]{\hyperref[aInNaturalImpaPlusZeroEqualsZeroPlusa]{\ensuremath{a\in\mathbb{N}\vdash a+0=0+a}(#1)}}

    % a∈ℕ⊢a=a+0
    \newcommand{\aInNaturalImpaEqualsaPlusZero}[1]{\hyperref[aInNaturalImpaEqualsaPlusZero]{\ensuremath{a\in\mathbb{N}\vdash a=a+0}(#1)}}

    % a∈ℕ⊢0+a=a+0=a
    \newcommand{\aInNaturalImpZeroPlusaEqualsaPlusZeroEqualsa}[1]{\hyperref[aInNaturalImpZeroPlusaEqualsaPlusZeroEqualsa]{\ensuremath{a\in\mathbb{N}\vdash 0+a=a+0=a}(#1)}}

    % a∈ℕ⊢1+a=a+1
    \newcommand{\aInNaturalImpOnePlusaEqualsaPlusOne}[1]{\hyperref[aInNaturalImpOnePlusaEqualsaPlusOne]{\ensuremath{a\in\mathbb{N}\vdash 1+a=a+1}(#1)}}

    % a∈ℕ,b∈ℕ,c∈ℕ⊢a+(b+c)=(a+b)+c
    \newcommand{\aInNaturalwbInNaturalwcInNaturalImpaPlusLpbPluscRpEqualsLpaPlusbRpPlusc}[1]{\hyperref[aInNaturalwbInNaturalwcInNaturalImpaPlusLpbPluscRpEqualsLpaPlusbRpPlusc]{\ensuremath{a\in\mathbb{N}, b\in\mathbb{N}, c\in\mathbb{N}\vdash a+(b+c)=(a+b)+c}(#1)}}

    % a∈ℕ,b∈ℕ⊢a+b=b+a
    \newcommand{\aInNaturalwbInNaturalImpaPlusbEqualsbPlusa}[1]{\hyperref[aInNaturalwbInNaturalImpaPlusbEqualsbPlusa]{\ensuremath{a\in\mathbb{N}, b\in\mathbb{N}\vdash a+b=b+a}(#1)}}

    % ⊢(ℕ,+,0)∈Monoid.
    \newcommand{\ImpLpNaturalwPluswZeroRpInMonoid}[1]{\hyperref[ImpLpNaturalwPluswZeroRpInMonoid]{\ensuremath{\vdash (\mathbb{N},+,0) \text{ ist ein Monoid}}(#1)}}

    % ⊢(ℕ,+,0)∈AbelMonoid.
    \newcommand{\ImpLpNaturalwPluswZeroRpInAbelMonoid}[1]{\hyperref[ImpLpNaturalwPluswZeroRpInAbelMonoid]{\ensuremath{\vdash (\mathbb{N},+,0) \text{ ist ein abelscher Monoid}}(#1)}}


    % a∈ℕ,b∈ℕ,c∈ℕ⊢(a+b)+c=(a+c)+b
    \newcommand{\aInNaturalwbInNaturalwcInNaturalImpLpaPlusbRpPluscEqualsLpaPluscRpPlusb}[1]{\hyperref[aInNaturalwbInNaturalwcInNaturalImpLpaPlusbRpPluscEqualsLpaPluscRpPlusb]{\ensuremath{a\in\mathbb{N}, b\in\mathbb{N}, c\in\mathbb{N} \vdash (a+b)+c=(a+c)+b}(#1)}}

    % a∈ℕ,b∈ℕ,c∈ℕ,d∈ℕ⊢(a+b)+(c+d)=(a+c)+(b+d)
    \newcommand{\aInNaturalwbInNaturalwcInNaturalwdInNaturalImpLpaPlusbRpPlusLpcPlusdRpEqualsLpaPluscRpPlusLpbPlusdRp}[1]{\hyperref[aInNaturalwbInNaturalwcInNaturalwdInNaturalImpLpaPlusbRpPlusLpcPlusdRpEqualsLpaPluscRpPlusLpbPlusdRp]{\ensuremath{a\in\mathbb{N}, b\in\mathbb{N}, c\in\mathbb{N}, d\in\mathbb{N} \vdash (a+b)+(c+d)=(a+c)+(b+d)}(#1)}}

    % a≠0⊢a+b≠0
    \newcommand{\aNotEqualsZeroImpaPlusbNotEqualsZero}[1]{\hyperref[aNotEqualsZeroImpaPlusbNotEqualsZero]{\ensuremath{a\neq 0\vdash a+b\neq 0}(#1)}}

    % b≠0⊢a+b≠0
    \newcommand{\bNotEqualsZeroImpaPlusbNotEqualsZero}[1]{\hyperref[bNotEqualsZeroImpaPlusbNotEqualsZero]{\ensuremath{b\neq 0\vdash a+b\neq 0}(#1)}}

    % a=b⊣⊢a+c=b+c
    \newcommand{\aEqualsbEqvaPluscEqualsbPlusc}[1]{\hyperref[aEqualsbEqvaPluscEqualsbPlusc]{\ensuremath{a=b\dashv\vdash a+c=b+c}(#1)}}

    % a≠b⊣⊢a+c≠b+c
    \newcommand{\aNotEqualsbEqvaPluscNotEqualsbPlusc}[1]{\hyperref[aNotEqualsbEqvaPluscNotEqualsbPlusc]{\ensuremath{a\neq b\dashv \vdash a+c\neq b+c}(#1)}}

    % a=b⊣⊢c+a=c+b
    \newcommand{\aEqualsbEqvcPlusaEqualscPlusb}[1]{\hyperref[aEqualsbEqvcPlusaEqualscPlusb]{\ensuremath{a=b\dashv\vdash c+a=c+b}(#1)}}

    % a≠b⊣⊢c+a≠c+b
    \newcommand{\aNotEqualsbEqvcPlusaNotEqualscPlusb}[1]{\hyperref[aNotEqualsbEqvcPlusaNotEqualscPlusb]{\ensuremath{a\neq b\dashv \vdash c+a\neq c+b}(#1)}}

    % a∈ℕ,b∈ℕ,a+b=a⊢b=0
    \newcommand{\aInNaturalwbInNaturalwaPlusbEqualsaImpbEqualsZero}[1]{\hyperref[aInNaturalwbInNaturalwaPlusbEqualsaImpbEqualsZero]{\ensuremath{a\in\mathbb{N}, b\in\mathbb{N}, a+b=a\vdash b=0}(#1)}}

    % a∈ℕ,b∈ℕ,a+b=0⊢a=0∧b=0
    \newcommand{\aInNaturalwbInNaturalwaPlusbEqualsZeroImpaEqualsZeroAndbEqualsZero}[1]{\hyperref[aInNaturalwbInNaturalwaPlusbEqualsZeroImpaEqualsZeroAndbEqualsZero]{\ensuremath{a\in\mathbb{N}, b\in\mathbb{N}, a+b=0\vdash a=0\land b=0}(#1)}}


    % Kleiner-Relation

    \newcommand{\rLeqNI}[1]{\hyperref[rule:rLeqNI]{\ensuremath{\leq}I(#1)}}
    \newcommand{\rLeqNE}[1]{\hyperref[rule:rLeqNE]{\ensuremath{\leq}E(#1)}}
    
    \newcommand{\rLneqNI}[1]{\hyperref[rule:rLneqNI]{\ensuremath{<}I(#1)}}

    % ∃c_1,c_2∈ℕ(a+c_1=b∧a+c_2=b)⊢c_1=c_2
    \newcommand{\ExcSubOnewcSubTwoInNaturalLpaPluscSubOneEqualsbAndaPluscSubTwoEqualsbRpImpcSubOneEqualscSubTwo}[1]{\hyperref[ExcSubOnewcSubTwoInNaturalLpaPluscSubOneEqualsbAndaPluscSubTwoEqualsbRpImpcSubOneEqualscSubTwo]{\ensuremath{\exists c_1, c_2 \in \mathbb{N} (a + c_1 = b \land a + c_2 = b) \vdash c_1 = c_2}(#1)}}
    
    % Einführung von b-a, sofern b≤a
    \newcommand{\minusI}[1]{\hyperref[rule:minusI]{\ensuremath{-}I(#1)}}

    % Einführung von b-a, sofern b≤a
    \newcommand{\minusE}[1]{\hyperref[rule:minusE]{\ensuremath{-}E(#1)}}

    % a,b∈ℕ,a=b⊢a≤b
    \newcommand{\awbInNaturalwaEqualsbImpaLeqb}[1]{\hyperref[awbInNaturalwaEqualsbImpaLeqb]{\ensuremath{a,b\in\mathbb{N},a=b\vdash a\leq b}(#1)}}

    % a,b∈ℕ,a=b⊢b≤a
    \newcommand{\awbInNaturalwaEqualsbImpbLeqa}[1]{\hyperref[awbInNaturalwaEqualsbImpbLeqa]{\ensuremath{a,b\in\mathbb{N},a=b\vdash b\leq a}(#1)}}

    % a,b∈ℕa≤b⊢b-a≤b
    \newcommand{\awbInNaturalaLeqbImpbMinusaLeqb}[1]{\hyperref[awbInNaturalaLeqbImpbMinusaLeqb]{\ensuremath{a,b\in\mathbb{N}a\leq b\vdash b-a\leq b}(#1)}}

    % a∈ℕ⊢(a+1)-a=1
    \newcommand{\aInNaturalImpLpaPlusOneRpMinusaEqualsOne}[1]{\hyperref[aInNaturalImpLpaPlusOneRpMinusaEqualsOne]{\ensuremath{a\in\mathbb{N}\vdash (a+1)-a=1}(#1)}}

    % a∈ℕ,a≠0⊢a-(a-1)=1
    \newcommand{\aInNaturalwaNotEqualsZeroImpaMinusLpaMinusOneRpEqualsOne}[1]{\hyperref[aInNaturalwaNotEqualsZeroImpaMinusLpaMinusOneRpEqualsOne]{\ensuremath{a\in\mathbb{N},a\neq 0\vdash a-(a-1)=1}(#1)}}

    % a∈ℕ⊢a≤a+1
    \newcommand{\aInNaturalImpaLeqaPlusOne}[1]{\hyperref[aInNaturalImpaLeqaPlusOne]{\ensuremath{a\in\mathbb{N}\vdash a\leq a+1}(#1)}}

    % a∈ℕ,b∈ℕ,a=b⊢a≤b+1
    \newcommand{\aInNaturalwbInNaturalwaEqualsbImpaLeqbPlusOne}[1]{\hyperref[aInNaturalwbInNaturalwaEqualsbImpaLeqbPlusOne]{\ensuremath{a\in\mathbb{N},b\in\mathbb{N},a=b\vdash a\leq b+1}(#1)}}

    % a∈ℕ,b∈ℕ,a=b⊢b≤a+1
    \newcommand{\aInNaturalwbInNaturalwaEqualsbImpbLeqaPlusOne}[1]{\hyperref[aInNaturalwbInNaturalwaEqualsbImpbLeqaPlusOne]{\ensuremath{a\in\mathbb{N},b\in\mathbb{N},a=b\vdash b\leq a+1}(#1)}}

    % a∈ℕ,a≠0⊢(a-1)≤a
    \newcommand{\aInNaturalwaNotEqualsZeroImpLpaMinusOneRpLeqa}[1]{\hyperref[aInNaturalwaNotEqualsZeroImpLpaMinusOneRpLeqa]{\ensuremath{a\in\mathbb{N},a\neq 0\vdash (a-1)\leq a}(#1)}}

    % a∈ℕ,b∈ℕ,a≠0,a=b⊢a-1≤b
    \newcommand{\aInNaturalwbInNaturalwaNotEqualsZerowaEqualsbImpaMinusOneLeqb}[1]{\hyperref[aInNaturalwbInNaturalwaNotEqualsZerowaEqualsbImpaMinusOneLeqb]{\ensuremath{a\in\mathbb{N},b\in\mathbb{N},a\neq 0, a=b\vdash a-1\leq b}(#1)}}

    % a∈ℕ,b∈ℕ,a≤b,a≠b⊢a+1≤b
    \newcommand{\aInNaturalwbInNaturalwaLeqbwaNotEqualsbImpaPlusOneLeqb}[1]{\hyperref[aInNaturalwbInNaturalwaLeqbwaNotEqualsbImpaPlusOneLeqb]{\ensuremath{a\in\mathbb{N},b\in\mathbb{N}, a\leq b,a\neq b\vdash a+1\leq b}(#1)}}

    % a∈ℕ,b∈ℕ,a≠0,a=b⊢b-1≤a
    \newcommand{\aInNaturalwbInNaturalwaNotEqualsZerowaEqualsbImpbMinusOneLeqa}[1]{\hyperref[aInNaturalwbInNaturalwaNotEqualsZerowaEqualsbImpbMinusOneLeqa]{\ensuremath{a\in\mathbb{N},b\in\mathbb{N},a\neq 0, a=b\vdash b-1\leq a}(#1)}}

    % a∈ℕ⊢a-0=a
    \newcommand{\aInNaturalImpaMinusZeroEqualsa}[1]{\hyperref[aInNaturalImpaMinusZeroEqualsa]{\ensuremath{a\in\mathbb{N}\vdash a-0=a}(#1)}}

    % a∈ℕ⊢a-a=0
    \newcommand{\aInNaturalImpaMinusaEqualsZero}[1]{\hyperref[aInNaturalImpaMinusaEqualsZero]{\ensuremath{a\in\mathbb{N}\vdash a-a=0}(#1)}}

    % a,b∈ℕ(a=b⊣⊢a-b=0)
    \newcommand{\awbInNaturalLpaEqualsbEqvaMinusbEqualsZeroRp}[1]{\hyperref[awbInNaturalLpaEqualsbEqvaMinusbEqualsZeroRp]{\ensuremath{a,b\in\mathbb{N}(a=b\dashv\vdash a-b=0)}(#1)}}

    % a,b∈ℕ(b=a⊣⊢a-b=0)
    \newcommand{\awbInNaturalLpbEqualsaEqvaMinusbEqualsZeroRp}[1]{\hyperref[awbInNaturalLpbEqualsaEqvaMinusbEqualsZeroRp]{\ensuremath{a,b\in\mathbb{N}(b=a\dashv\vdash a-b=0)}(#1)}}

    % a≠b⊣⊢a-b≠0
    \newcommand{\aNotEqualsbEqvaMinusbNotEqualsZero}[1]{\hyperref[aNotEqualsbEqvaMinusbNotEqualsZero]{\ensuremath{a\neq b\dashv\vdash a-b\neq 0}(#1)}}

    % b≠a⊣⊢a-b≠0
    \newcommand{\bNotEqualsaEqvaMinusbNotEqualsZero}[1]{\hyperref[bNotEqualsaEqvaMinusbNotEqualsZero]{\ensuremath{b\neq a\dashv\vdash a-b\neq 0}(#1)}}

    % a,b∈ℕ(a≤b⊣⊢0≤b-a)
    \newcommand{\awbInNaturalLpaLeqbEqvZeroLeqbMinusaRp}[1]{\hyperref[awbInNaturalLpaLeqbEqvZeroLeqbMinusaRp]{\ensuremath{a,b\in\mathbb{N}(a\leq b\dashv\vdash 0\leq b-a)}(#1)}}

    % ⊢0≤a
    \newcommand{\ImpZeroLeqa}[1]{\hyperref[ImpZeroLeqa]{\ensuremath{\vdash 0\leq a}(#1)}}

    % a∈ℕ,a≠0,1≤a
    \newcommand{\aInNaturalwaNotEqualsZerowOneLeqa}[1]{\hyperref[aInNaturalwaNotEqualsZerowOneLeqa]{\ensuremath{a\in\mathbb{N}, a\neq 0, 1\leq a}(#1)}}

    % a∈ℕ,a≤0⊢a=0
    \newcommand{\aInNaturalwaLeqZeroImpaEqualsZero}[1]{\hyperref[aInNaturalwaLeqZeroImpaEqualsZero]{\ensuremath{a\in\mathbb{N},a\leq 0\vdash a=0}(#1)}}

    % a∈ℕ⊢a≤a
    \newcommand{\aInNaturalImpaLeqa}[1]{\hyperref[aInNaturalImpaLeqa]{\ensuremath{a\in\mathbb{N}\vdash a\leq a}(#1)}}

    % a∈ℕ,b∈ℕ,a≤b,b≤a⊢a=b
    \newcommand{\aInNaturalwbInNaturalwaLeqbwbLeqaImpaEqualsb}[1]{\hyperref[aInNaturalwbInNaturalwaLeqbwbLeqaImpaEqualsb]{\ensuremath{a\in\mathbb{N},b\in\mathbb{N},a\leq b,b\leq a\vdash a = b}(#1)}}

    % a∈ℕ,b∈ℕ,a≤b,b≤c⊢a≤c
    \newcommand{\aInNaturalwbInNaturalwaLeqbwbLeqcImpaLeqc}[1]{\hyperref[aInNaturalwbInNaturalwaLeqbwbLeqcImpaLeqc]{\ensuremath{a\in\mathbb{N},b\in\mathbb{N},a\leq b,b\leq c\vdash a\leq c}(#1)}}

    % ⊢≤ ist eine Halbordnung auf ℕ
    \newcommand{\LeqIsHalfOrderOnNaturalNumbers}[1]{\hyperref[LeqIsHalfOrderOnNaturalNumbers]{\ensuremath{\leq \text{ ist eine Halbordnung auf } \mathbb{N} }(#1)}}

    % a,b,c∈ℕ(a≤b⊣⊢a+c≤b+c)
    \newcommand{\awbwcInNaturalLpaLeqbEqvaPluscLeqbPluscRp}[1]{\hyperref[awbwcInNaturalLpaLeqbEqvaPluscLeqbPluscRp]{\ensuremath{a,b,c\in\mathbb{N}(a\leq b\dashv\vdash a+c\leq b+c)}(#1)}}

    % a,b,c∈ℕ(a≤b⊣⊢c+a≤c+b)
    \newcommand{\awbwcInNaturalLpaLeqbEqvcPlusaLeqcPlusbRp}[1]{\hyperref[awbwcInNaturalLpaLeqbEqvcPlusaLeqcPlusbRp]{\ensuremath{a,b,c\in\mathbb{N}(a\leq b\dashv\vdash c+a\leq c+b)}(#1)}}

    % a,b,c,d∈ℕ,a≤b,c≤d⊢a+c≤b+d
    \newcommand{\awbwcwdInNaturalwaLeqbwcLeqdImpaPluscLeqbPlusd}[1]{\hyperref[awbwcwdInNaturalwaLeqbwcLeqdImpaPluscLeqbPlusd]{\ensuremath{a,b,c,d\in\mathbb{N},a\leq b, c\leq d\vdash a+c\leq b+d}(#1)}}

    % a∈ℕ,b∈ℕ,c∈ℕ,a≤b⊢a≤b+c
    \newcommand{\aInNaturalwbInNaturalwcInNaturalwaLeqbImpaLeqbPlusc}[1]{\hyperref[aInNaturalwbInNaturalwcInNaturalwaLeqbImpaLeqbPlusc]{\ensuremath{a\in\mathbb{N},b\in\mathbb{N},c\in\mathbb{N}, a\leq b\vdash a\leq b+c}(#1)}}

    % a∈ℕ,b∈ℕ,c∈ℕ,a≤b⊢a≤c+b
    \newcommand{\aInNaturalwbInNaturalwcInNaturalwaLeqbImpaLeqcPlusb}[1]{\hyperref[aInNaturalwbInNaturalwcInNaturalwaLeqbImpaLeqcPlusb]{\ensuremath{a\in\mathbb{N},b\in\mathbb{N},c\in\mathbb{N}, a\leq b\vdash a\leq c+b}(#1)}}

    % a,b∈ℕ⊢a≤a+b
    \newcommand{\awbInNaturalImpaLeqaPlusb}[1]{\hyperref[awbInNaturalImpaLeqaPlusb]{\ensuremath{a,b\in\mathbb{N}\vdash a\leq a+b}(#1)}}

    % a,b∈ℕ⊢a≤b+a
    \newcommand{\awbInNaturalImpaLeqbPlusa}[1]{\hyperref[awbInNaturalImpaLeqbPlusa]{\ensuremath{a,b\in\mathbb{N}\vdash a\leq b+a}(#1)}}

    % a,b,c∈ℕ(a+b≤c⊢a≤c)
    \newcommand{\awbwcInNaturalLpaPlusbLeqcImpaLeqcRp}[1]{\hyperref[awbwcInNaturalLpaPlusbLeqcImpaLeqcRp]{\ensuremath{a,b,c\in\mathbb{N}(a+b\leq c\vdash a\leq c)}(#1)}}

    % a,b,c∈ℕ(a+b≤c⊢b≤c)
    \newcommand{\awbwcInNaturalLpaPlusbLeqcImpbLeqcRp}[1]{\hyperref[awbwcInNaturalLpaPlusbLeqcImpbLeqcRp]{\ensuremath{a,b,c\in\mathbb{N}(a+b\leq c\vdash b\leq c)}(#1)}}

    % b≤c,a≤b⊢b-c≤c
    \newcommand{\bLeqcwaLeqbImpbMinuscLeqc}[1]{\hyperref[bLeqcwaLeqbImpbMinuscLeqc]{\ensuremath{b\leq c, a\leq b\vdash b-c\leq c}(#1)}}

    % ⊢a≤b∨b≤a
    \newcommand{\ImpaLeqbOrbLeqa}[1]{\hyperref[ImpaLeqbOrbLeqa]{\ensuremath{\vdash a\leq b\lor b\leq a}(#1)}}

    % ⊢≤ ist eine totale Ordnung auf ℕ
    \newcommand{\LeqIsTotalOrderOnNaturalNumbers}[1]{\hyperref[LeqIsTotalOrderOnNaturalNumbers]{\ensuremath{\leq \text{ ist eine totale Ordnung auf } \mathbb{N} }(#1)}}

    % a,b,c∈ℕ,c≤a⊢(a+b)-c=b+(a-c)
    \newcommand{\awbwcInNaturalwcLeqaImpLpaPlusbRpMinuscEqualsbPlusLpaMinuscRp}[1]{\hyperref[awbwcInNaturalwcLeqaImpLpaPlusbRpMinuscEqualsbPlusLpaMinuscRp]{\ensuremath{a,b,c\in\mathbb{N},c\leq a\vdash (a+b)-c=b+(a-c)}(#1)}}

    % a,b,c∈ℕ,c≤a⊢(a+b)-c=(a-c)+b
    \newcommand{\awbwcInNaturalwcLeqaImpLpaPlusbRpMinuscEqualsLpaMinuscRpPlusb}[1]{\hyperref[awbwcInNaturalwcLeqaImpLpaPlusbRpMinuscEqualsLpaMinuscRpPlusb]{\ensuremath{a,b,c\in\mathbb{N},c\leq a\vdash (a+b)-c=(a-c)+b}(#1)}}

    % a,b,c∈ℕ,c≤b⊢(a+b)-c=a+(b-c)
    \newcommand{\awbwcInNaturalwcLeqbImpLpaPlusbRpMinuscEqualsaPlusLpbMinuscRp}[1]{\hyperref[awbwcInNaturalwcLeqbImpLpaPlusbRpMinuscEqualsaPlusLpbMinuscRp]{\ensuremath{a,b,c\in\mathbb{N}, c\leq b\vdash (a+b)-c=a+(b-c)}(#1)}}

    % a,b,c∈ℕ,c≤b⊢(a+b)-c=(b-c)+a
    \newcommand{\awbwcInNaturalwcLeqbImpLpaPlusbRpMinuscEqualsLpbMinuscRpPlusa}[1]{\hyperref[awbwcInNaturalwcLeqbImpLpaPlusbRpMinuscEqualsLpbMinuscRpPlusa]{\ensuremath{a,b,c\in\mathbb{N}, c\leq b\vdash (a+b)-c=(b-c)+a}(#1)}}

    % a,b∈ℕ⊢(a+b)-b=a+(b-b)=a
    \newcommand{\awbInNaturalImpLpaPlusbRpMinusbEqualsaPlusLpbMinusbRpEqualsa}[1]{\hyperref[awbInNaturalImpLpaPlusbRpMinusbEqualsaPlusLpbMinusbRpEqualsa]{\ensuremath{a,b\in\mathbb{N}\vdash (a+b)-b=a+(b-b)=a}(#1)}}

    % a,b∈ℕ⊢(a+b)-a=(a-a)+b=b
    \newcommand{\awbInNaturalImpLpaPlusbRpMinusaEqualsLpaMinusaRpPlusbEqualsb}[1]{\hyperref[awbInNaturalImpLpaPlusbRpMinusaEqualsLpaMinusaRpPlusbEqualsb]{\ensuremath{a,b\in\mathbb{N}\vdash (a+b)-a=(a-a)+b=b}(#1)}}

    % a,b∈ℕ,b≤a⊢(a-b)+b=a
    \newcommand{\awbInNaturalwbLeqaImpLpaMinusbRpPlusbEqualsa}[1]{\hyperref[awbInNaturalwbLeqaImpLpaMinusbRpPlusbEqualsa]{\ensuremath{a,b\in\mathbb{N},b\leq a \vdash (a-b)+b=a}(#1)}}

    % a,b∈ℕ,a-b=a⊢b=0
    \newcommand{\awbInNaturalwaMinusbEqualsaImpbEqualsZero}[1]{\hyperref[awbInNaturalwaMinusbEqualsaImpbEqualsZero]{\ensuremath{a,b\in\mathbb{N}, a-b=a\vdash b=0}(#1)}}

    % a,b∈ℕ,b≤a⊢b+(a-b)=a
    \newcommand{\awbInNaturalwbLeqaImpbPlusLpaMinusbRpEqualsa}[1]{\hyperref[awbInNaturalwbLeqaImpbPlusLpaMinusbRpEqualsa]{\ensuremath{a,b\in\mathbb{N},b\leq a\vdash (a-b)+b=b+(a-b)=a}(#1)}}

    % a,b,c∈ℕ,c≤a(a=b⊣⊢a-c=b-c)
    \newcommand{\awbwcInNaturalwcLeqaLpaEqualsbEqvaMinuscEqualsbMinuscRp}[1]{\hyperref[awbwcInNaturalwcLeqaLpaEqualsbEqvaMinuscEqualsbMinuscRp]{\ensuremath{a,b,c\in\mathbb{N},c\leq a(a=b\dashv\vdash a-c=b-c)}(#1)}}

    % a,b,c∈ℕ,c≤a(a≠b⊣⊢a-c≠b-c)
    \newcommand{\awbwcInNaturalwcLeqaLpaNotEqualsbEqvaMinuscNotEqualsbMinuscRp}[1]{\hyperref[awbwcInNaturalwcLeqaLpaNotEqualsbEqvaMinuscNotEqualsbMinuscRp]{\ensuremath{a,b,c\in\mathbb{N}, c\leq a(a\neq b\dashv\vdash a-c\neq b-c)}(#1)}}

    % a,b,c∈ℕ,a≤b,b≤c⊢(b-a)+(c-b)=(c-a)
    \newcommand{\awbwcInNaturalwaLeqbwbLeqcImpLpbMinusaRpPlusLpcMinusbRpEqualsLpcMinusaRp}[1]{\hyperref[awbwcInNaturalwaLeqbwbLeqcImpLpbMinusaRpPlusLpcMinusbRpEqualsLpcMinusaRp]{\ensuremath{a,b,c \in \mathbb{N},a\leq b, b\leq c\vdash (b-a)+(c-b)=(c-a)}(#1)}}


    % a,b,c∈ℕ,a≤b(b≤c⊣⊢b-a≤c-a)
    \newcommand{\awbwcInNaturalwaLeqbLpbLeqcEqvbMinusaLeqcMinusaRp}[1]{\hyperref[awbwcInNaturalwaLeqbLpbLeqcEqvbMinusaLeqcMinusaRp]{\ensuremath{a,b,c\in\mathbb{N},a\leq b(b\leq c \dashv\vdash b-a\leq c-a)}(#1)}}

    % a,b,c∈ℕ,a+b≤c⊢a≤c-b
    \newcommand{\awbwcInNaturalwaPlusbLeqcImpaLeqcMinusb}[1]{\hyperref[awbwcInNaturalwaPlusbLeqcImpaLeqcMinusb]{\ensuremath{a,b,c\in\mathbb{N},a+b\leq c \vdash a\leq c-b}(#1)}}

    % a,b,c∈ℕ,a+b≤c⊢b≤c-a
    \newcommand{\awbwcInNaturalwaPlusbLeqcImpbLeqcMinusa}[1]{\hyperref[awbwcInNaturalwaPlusbLeqcImpbLeqcMinusa]{\ensuremath{a,b,c\in\mathbb{N},a+b\leq c \vdash b\leq c-a}(#1)}}

    % a,b,c∈ℕ,b+c≤a⊢a-(b+c)=(a-b)-c
    \newcommand{\awbwcInNaturalwbPluscLeqaImpaMinusLpbPluscRpEqualsLpaMinusbRpMinusc}[1]{\hyperref[awbwcInNaturalwbPluscLeqaImpaMinusLpbPluscRpEqualsLpaMinusbRpMinusc]{\ensuremath{a,b,c\in\mathbb{N},b+c\leq a \vdash a-(b+c)=(a-b)-c}(#1)}}

    % a,b,c∈ℕ,b+c≤a⊢a-(b+c)=(a-c)-b
    \newcommand{\awbwcInNaturalwbPluscLeqaImpaMinusLpbPluscRpEqualsLpaMinuscRpMinusb}[1]{\hyperref[awbwcInNaturalwbPluscLeqaImpaMinusLpbPluscRpEqualsLpaMinuscRpMinusb]{\ensuremath{a,b,c\in\mathbb{N},b+c\leq a \vdash a-(b+c)=(a-c)-b}(#1)}}

    % a,b,c∈ℕ,b+c≤a⊢(a-b)-c=(a-c)-b
    \newcommand{\awbwcInNaturalwbPluscLeqaImpLpaMinusbRpMinuscEqualsLpaMinuscRpMinusb}[1]{\hyperref[awbwcInNaturalwbPluscLeqaImpLpaMinusbRpMinuscEqualsLpaMinuscRpMinusb]{\ensuremath{a,b,c\in\mathbb{N},b+c\leq a \vdash (a-b)-c=(a-c)-b}(#1)}}

    % a,b,c,d∈ℕ,a≤c,b≤d⊢(c-a)+(d-b)=(c+d)-(a+b)
    \newcommand{\awbwcwdInNaturalwaLeqcwbLeqdImpLpcMinusaRpPlusLpdMinusbRpEqualsLpcPlusdRpMinusLpaPlusbRp}[1]{\hyperref[awbwcwdInNaturalwaLeqcwbLeqdImpLpcMinusaRpPlusLpdMinusbRpEqualsLpcPlusdRpMinusLpaPlusbRp]{\ensuremath{a,b,c,d\in\mathbb{N},a\leq c, b\leq d \vdash (c-a)+(d-b)=(c+d)-(a+b)}(#1)}}

    % a,b,c,d∈ℕ,c≤a,b≤d(a+b=c+d⊣⊢a-c=d-b)
    \newcommand{\awbwcwdInNaturalwcLeqawbLeqdLpaPlusbEqualscPlusdEqvaMinuscEqualsdMinusbRp}[1]{\hyperref[awbwcwdInNaturalwcLeqawbLeqdLpaPlusbEqualscPlusdEqvaMinuscEqualsdMinusbRp]{\ensuremath{a,b,c,d\in\mathbb{N},c\leq a,b\leq d (a+b=c+d \dashv\vdash a-c=d-b)}(#1)}}

    % a,b,c,d,f∈ℕ,f(a-b)=(c-d),(c-d)<f⊢a=b∧c=d
    \newcommand{\awbwcwdwfInNaturalwfLpaMinusbRpEqualsLpcMinusdRpwLpcMinusdRpLneqfImpaEqualsbAndcEqualsd}[1]{\hyperref[awbwcwdwfInNaturalwfLpaMinusbRpEqualsLpcMinusdRpwLpcMinusdRpLneqfImpaEqualsbAndcEqualsd]{\ensuremath{a,b,c,d,f\in\mathbb{N}, f(a-b)=(c-d), (c-d)<f\vdash a=b\land c=d}(#1)}}
    
    % a<b⊣⊢b-a≠0∧a+(b-a)=b
    \newcommand{\aLneqbEqvExnInNaturalLpnNotEqualsZeroAndaPlusnEqualsbRp}[1]{\hyperref[aLneqbEqvExnInNaturalLpnNotEqualsZeroAndaPlusnEqualsbRp]{\ensuremath{a<b\dashv\vdash b-a\neq 0\land a+(b-a)=b}(#1)}}

    % a∈ℕ⊢a=0∨0<a
    \newcommand{\aInNaturalImpaEqualsZeroOrZeroLneqa}[1]{\hyperref[aInNaturalImpaEqualsZeroOrZeroLneqa]{\ensuremath{a\in\mathbb{N}\vdash a=0\lor 0<a}(#1)}}

    % a∈ℕ,0<a⊢∃b∈ℕ(1+b=a)
    \newcommand{\aInNaturalwZeroLneqaImpExbInNaturalLpOnePlusbEqualsaRp}[1]{\hyperref[aInNaturalwZeroLneqaImpExbInNaturalLpOnePlusbEqualsaRp]{\ensuremath{a\in\mathbb{N},0<a\vdash\exists b\in\mathbb{N}(1+b=a)}(#1)}}

    % a<b⊣⊢a≤b-1
    \newcommand{\aLneqbEqvaLeqbMinusOne}[1]{\hyperref[aLneqbEqvaLeqbMinusOne]{\ensuremath{a<b\dashv\vdash a\leq b-1}(#1)}}

    % a<b+1⊣⊢a≤b
    \newcommand{\aLneqbPlusOneEqvaLeqb}[1]{\hyperref[aLneqbPlusOneEqvaLeqb]{\ensuremath{a<b+1\dashv\vdash a\leq b}(#1)}}

    % a∈ℕ(a<1⊣⊢a=0)
    \newcommand{\aInNaturalLpaLneqOneEqvaEqualsZeroRp}[1]{\hyperref[aInNaturalLpaLneqOneEqvaEqualsZeroRp]{\ensuremath{a\in\mathbb{N}(a<1\dashv\vdash a=0)}(#1)}}

    % a∈ℕ,a≥1⊣⊢a≠0
    \newcommand{\aInNaturalwaGeqOneEqvaNotEqualsZero}[1]{\hyperref[aInNaturalwaGeqOneEqvaNotEqualsZero]{\ensuremath{a\in\mathbb{N},a\geq 1\dashv\vdash a\neq 0}(#1)}}

    % a∈ℕ,a>1⊢a≠0
    \newcommand{\aInNaturalwaGneqOneImpaNotEqualsZero}[1]{\hyperref[aInNaturalwaGneqOneImpaNotEqualsZero]{\ensuremath{a\in\mathbb{N},a> 1\vdash a\neq 0}(#1)}}

    % n∈ℕ,n>1⊢0∈{0,1,...,n-1}
    \newcommand{\nInNaturalwnGneqOneImpZeroInLbZerowOnewDotswnMinusOneRb}[1]{\hyperref[nInNaturalwnGneqOneImpZeroInLbZerowOnewDotswnMinusOneRb]{\ensuremath{n\in\mathbb{N},n>1\vdash 0\in\{0, 1, \dots, n-1\}}(#1)}}

    % n∈ℕ,n>1⊢0≤n-1
    \newcommand{\nInNaturalwnGneqOneImpZeroLeqnMinusOne}[1]{\hyperref[nInNaturalwnGneqOneImpZeroLeqnMinusOne]{\ensuremath{n\in\mathbb{N}, n>1\vdash 0\leq n-1}(#1)}}

    % a,b,c∈ℕ(a<b⊣⊢a+c<b+c)
    \newcommand{\awbwcInNaturalLpaLneqbEqvaPluscLneqbPluscRp}[1]{\hyperref[awbwcInNaturalLpaLneqbEqvaPluscLneqbPluscRp]{\ensuremath{a,b,c\in\mathbb{N}(a<b\dashv\vdash a+c<b+c)}(#1)}}

    % a<b⊣⊢c+a<c+b
    \newcommand{\aLneqbEqvcPlusaLneqcPlusb}[1]{\hyperref[aLneqbEqvcPlusaLneqcPlusb]{\ensuremath{a<b\dashv\vdash c+a<c+b}(#1)}}

    % a∈ℕ,b∈ℕ,c∈ℕ,a≤b,c≠0⊢a<b+c
    \newcommand{\aInNaturalwbInNaturalwcInNaturalwaLeqbwcNotEqualsZeroImpaLneqbPlusc}[1]{\hyperref[aInNaturalwbInNaturalwcInNaturalwaLeqbwcNotEqualsZeroImpaLneqbPlusc]{\ensuremath{a\in\mathbb{N},b\in\mathbb{N},c\in\mathbb{N}, a\leq b,c\neq 0\vdash a<b+c}(#1)}}

    % a∈ℕ,b∈ℕ,c∈ℕ,a≤b,c≠0⊢a<c+b
    \newcommand{\aInNaturalwbInNaturalwcInNaturalwaLeqbwcNotEqualsZeroImpaLneqcPlusb}[1]{\hyperref[aInNaturalwbInNaturalwcInNaturalwaLeqbwcNotEqualsZeroImpaLneqcPlusb]{\ensuremath{a\in\mathbb{N},b\in\mathbb{N},c\in\mathbb{N}, a\leq b,c\neq 0 \vdash a<c+b}(#1)}}

    % a,b∈ℕ,b≠0⊢a<a+b
    \newcommand{\awbInNaturalwbNotEqualsZeroImpaLneqaPlusb}[1]{\hyperref[awbInNaturalwbNotEqualsZeroImpaLneqaPlusb]{\ensuremath{a,b\in\mathbb{N},b\neq 0\vdash a<a+b}(#1)}}

    % a,b∈ℕb≠0⊢a<b+a
    \newcommand{\awbInNaturalbNotEqualsZeroImpaLneqbPlusa}[1]{\hyperref[awbInNaturalbNotEqualsZeroImpaLneqbPlusa]{\ensuremath{a,b\in\mathbb{N}b\neq 0\vdash a<b+a}(#1)}}

    % a,b,c∈ℕ(a+b≤c,b≠0⊢a<c
    \newcommand{\awbwcInNaturalLpaPlusbLeqcwbNotEqualsZeroImpaLneqc}[1]{\hyperref[awbwcInNaturalLpaPlusbLeqcwbNotEqualsZeroImpaLneqc]{\ensuremath{a,b,c\in\mathbb{N}(a+b\leq c,b\neq 0 \vdash a<c}(#1)}}

    % a,b,c∈ℕ(a+b≤c⊢b<c)
    \newcommand{\awbwcInNaturalLpaPlusbLeqcImpbLneqcRp}[1]{\hyperref[awbwcInNaturalLpaPlusbLeqcImpbLneqcRp]{\ensuremath{a,b,c\in\mathbb{N}(a+b\leq c\vdash b<c)}(#1)}}

    % a,b∈ℕ(a<b⊣⊢0<b-a)
    \newcommand{\awbInNaturalLpaLneqbEqvZeroLneqbMinusaRp}[1]{\hyperref[awbInNaturalLpaLneqbEqvZeroLneqbMinusaRp]{\ensuremath{a,b\in\mathbb{N}(a<b\dashv\vdash 0<b-a)}(#1)}}

    % a,b,c∈ℕ,a≤b(b<c⊣⊢b-a<c-a)
    \newcommand{\awbwcInNaturalwaLeqbLpbLneqcEqvbMinusaLneqcMinusaRp}[1]{\hyperref[awbwcInNaturalwaLeqbLpbLneqcEqvbMinusaLneqcMinusaRp]{\ensuremath{a,b,c\in\mathbb{N},a\leq b(b<c \dashv\vdash b-a<c-a)}(#1)}}

    % b<c,a≤b⊢b-a<c
    \newcommand{\bLneqcwaLeqbImpbMinusaLneqc}[1]{\hyperref[bLneqcwaLeqbImpbMinusaLneqc]{\ensuremath{b<c, a\leq b\vdash b-a<c}(#1)}}

    % Einführungs und Eliminationsregeln für endliche Teilmengen
    \newcommand{\rSegmentI}[1]{\hyperref[rule:rSegmentI]{\ensuremath{x \in \{i, i+1, \dots, n\}}I(#1)}}
    \newcommand{\rSegmentZeroI}[1]{\hyperref[rule:rSegmentZeroI]{\ensuremath{x \in \{0, 1, \dots, n-1\}}I(#1)}}
    \newcommand{\rSegmentE}[1]{\hyperref[rule:rSegmentE]{\ensuremath{x \in \{i, i+1, \dots, n\}}E(#1)}}
    \newcommand{\rSegmentZeroE}[1]{\hyperref[rule:rSegmentZeroE]{\ensuremath{x \in \{0, 1, \dots, n-1\}}E(#1)}}

    % n∈ℕ,n≠0⊢0∈{0,1,...,n-1}
    \newcommand{\nInNaturalwnNotEqualsZeroImpZeroInLbZerowOnewDotswnMinusOneRb}[1]{\hyperref[nInNaturalwnNotEqualsZeroImpZeroInLbZerowOnewDotswnMinusOneRb]{\ensuremath{n\in\mathbb{N},n\neq 0\vdash 0\in\{0, 1, \dots, n-1\}}(#1)}}

    % a∈ℕ,b∈ℕ,a≤b+1⊢a≤b∨a=b+1
    \newcommand{\aInNaturalwbInNaturalwaLeqbPlusOneImpaLeqbOraEqualsbPlusOne}[1]{\hyperref[aInNaturalwbInNaturalwaLeqbPlusOneImpaLeqbOraEqualsbPlusOne]{\ensuremath{a,b\in\mathbb{N},a\leq b+1\vdash a\leq b\lor a=b+1}(#1)}}

    % Regel für das starke Induktionsprinzip über ℕ
    \newcommand{\rStrongInductionN}[1]{\hyperref[rule:rStrongInductionN]{\ensuremath{P(0),\forall k \leq n \, P(k) \rightarrow P(n+1)}(#1)}}

    % ∀n∈ℕ(∀k∈ℕ(k≤n→P(k)⊢∀n∈ℕ(P(n))))
    \newcommand{\FanInNaturalLpFakInNaturalLpkLeqnToPLpkRpImpFanInNaturalLpPLpnRpRpRpRp}[1]{\hyperref[FanInNaturalLpFakInNaturalLpkLeqnToPLpkRpImpFanInNaturalLpPLpnRpRpRpRp]{\ensuremath{\forall n\in\mathbb{N}(\forall k\in\mathbb{N}(k\leq n\rightarrow P(k)\vdash \forall n\in\mathbb{N}(P(n))))}(#1)}}

    % P(0),∀n∈ℕ((∀k∈ℕ(k≤n→P(k))→P(n+1))⊢∀n∈ℕ(P(n))
    \newcommand{\PLpZeroRpwFanInNaturalLpLpFakInNaturalLpkLeqnToPLpkRpRpToPLpnPlusOneRpRpImpFanInNaturalLpPLpnRpRp}[1]{\hyperref[PLpZeroRpwFanInNaturalLpLpFakInNaturalLpkLeqnToPLpkRpRpToPLpnPlusOneRpRpImpFanInNaturalLpPLpnRpRp]{\ensuremath{P(0), \forall n \in \mathbb{N} ((\forall k\in\mathbb{N}(k\leq n\rightarrow P(k)) \rightarrow P(n+1)) \vdash \forall n \in \mathbb{N}(P(n))}(#1)}}

    % Wohlordnungsprinzip
    % A⊆ℕ,A≠∅⊢∃n∈A(n=min(A))
    \newcommand{\ASubseteqNaturalwANotEqualsEmptysetImpExnInALpnEqualsMinLpARpRp}[1]{\hyperref[ASubseteqNaturalwANotEqualsEmptysetImpExnInALpnEqualsMinLpARpRp]{\ensuremath{A\subseteq\mathbb{N},A\neq\emptyset\vdash\exists n\in A(n=\min(A))}(#1)}}

        % Multiplikation
    \newcommand{\rMultI}[1]{\hyperref[rule:rMultI]{\ensuremath{\cdot}I(#1)}}

    % a∈ℕ⊢0=0·a
    \newcommand{\aInNaturalImpZeroEqualsZeroMulta}[1]{\hyperref[aInNaturalImpZeroEqualsZeroMulta]{\ensuremath{a\in\mathbb{N}\vdash 0=0\cdot a}(#1)}}

    % a∈ℕ⊢a=a·1
    \newcommand{\aInNaturalImpaEqualsaMultOne}[1]{\hyperref[aInNaturalImpaEqualsaMultOne]{\ensuremath{a\in\mathbb{N}\vdash a=a\cdot 1}(#1)}}

    % a∈ℕ⊢a=1·a
    \newcommand{\aInNaturalImpaEqualsOneMulta}[1]{\hyperref[aInNaturalImpaEqualsOneMulta]{\ensuremath{a\in\mathbb{N}\vdash a=1\cdot a}(#1)}}

    % a∈ℕ⊢a·1=1·a=a
    \newcommand{\aInNaturalImpaMultOneEqualsOneMultaEqualsa}[1]{\hyperref[aInNaturalImpaMultOneEqualsOneMultaEqualsa]{\ensuremath{a\in\mathbb{N}\vdash a\cdot 1=1\cdot a = a}(#1)}}

    % a∈ℕ,b∈ℕ⊢a·b∈ℕ
    \newcommand{\aInNaturalwbInNaturalImpaMultbInNatural}[1]{\hyperref[aInNaturalwbInNaturalImpaMultbInNatural]{\ensuremath{a\in\mathbb{N},b\in\mathbb{N}\vdash a\cdot b\in\mathbb{N}}(#1)}}

    % a∈ℕ,b∈ℕ,c∈ℕ⊢a(b+c)=ab+ac
    \newcommand{\aInNaturalwbInNaturalwcInNaturalImpaLpbPluscRpEqualsabPlusac}[1]{\hyperref[aInNaturalwbInNaturalwcInNaturalImpaLpbPluscRpEqualsabPlusac]{\ensuremath{a\in\mathbb{N},b\in\mathbb{N},c\in\mathbb{N}\vdash a(b+c)=ab+ac}(#1)}}

    % a∈ℕ,b∈ℕ,c∈ℕ⊢(a+b)c=ac+bc
    \newcommand{\aInNaturalwbInNaturalwcInNaturalImpLpaPlusbRpcEqualsacPlusbc}[1]{\hyperref[aInNaturalwbInNaturalwcInNaturalImpLpaPlusbRpcEqualsacPlusbc]{\ensuremath{a\in\mathbb{N},b\in\mathbb{N},c\in\mathbb{N}\vdash (a+b)c=ac+bc}(#1)}}

    % a∈ℕ,b∈ℕ⊢(a+1)b=ab+b
    \newcommand{\aInNaturalwbInNaturalImpLpaPlusOneRpbEqualsabPlusb}[1]{\hyperref[aInNaturalwbInNaturalImpLpaPlusOneRpbEqualsabPlusb]{\ensuremath{a\in\mathbb{N},b\in\mathbb{N}\vdash (a+1)b=ab+b}(#1)}}

    % a∈ℕ,b∈ℕ,c∈ℕ⊢a(bc)=(ab)c
    \newcommand{\aInNaturalwbInNaturalwcInNaturalImpaLpbcRpEqualsLpabRpc}[1]{\hyperref[aInNaturalwbInNaturalwcInNaturalImpaLpbcRpEqualsLpabRpc]{\ensuremath{a\in\mathbb{N},b\in\mathbb{N},c\in\mathbb{N}\vdash a(bc)=(ab)c}(#1)}}

    % a∈ℕ,b∈ℕ⊢ab=ba
    \newcommand{\aInNaturalwbInNaturalImpabEqualsba}[1]{\hyperref[aInNaturalwbInNaturalImpabEqualsba]{\ensuremath{a\in\mathbb{N},b\in\mathbb{N}\vdash ab=ba}(#1)}}

    % ⊢(ℕ,·,1)∈Monoid.
    \newcommand{\ImpLpNaturalwMultwOneRpInMonoid}[1]{\hyperref[ImpLpNaturalwMultwOneRpInMonoid]{\ensuremath{\vdash (\mathbb{N},\cdot,1) \text{ ist ein Monoid.}}(#1)}}

    % ⊢(ℕ,·,1)∈AbelMonoid.
    \newcommand{\ImpLpNaturalwMultwOneRpInAbelMonoid}[1]{\hyperref[ImpLpNaturalwMultwOneRpInAbelMonoid]{\ensuremath{\vdash (\mathbb{N},\cdot,1) \text{ ist ein abelscher Monoid.}}(#1)}}

    % ⊢(ℕ,·,1)∈AbelSemiRing.
    \newcommand{\ImpLpNaturalwMultwOneRpInAbelSemiRing}[1]{\hyperref[ImpLpNaturalwMultwOneRpInAbelSemiRing]{\ensuremath{\vdash (\mathbb{N},\cdot,1) \text{ ist ein abelscher Halbring.}}(#1)}}

    % a∈ℕ⊢0=a·0
    \newcommand{\aInNaturalImpZeroEqualsaMultZero}[1]{\hyperref[aInNaturalImpZeroEqualsaMultZero]{\ensuremath{a\in\mathbb{N}\vdash 0=a\cdot 0}(#1)}}

    % a∈ℕ,a=0⊢ab=0
    \newcommand{\aInNaturalwaEqualsZeroImpabEqualsZero}[1]{\hyperref[aInNaturalwaEqualsZeroImpabEqualsZero]{\ensuremath{a\in\mathbb{N},a=0\vdash ab=0}(#1)}}

    % a∈ℕ,ab≠0⊢a≠0
    \newcommand{\aInNaturalwabNotEqualsZeroImpaNotEqualsZero}[1]{\hyperref[aInNaturalwabNotEqualsZeroImpaNotEqualsZero]{\ensuremath{a\in\mathbb{N},ab\neq 0\vdash a\neq 0}(#1)}}

    % a∈ℕ,b=0⊢ab=0
    \newcommand{\aInNaturalwbEqualsZeroImpabEqualsZero}[1]{\hyperref[aInNaturalwbEqualsZeroImpabEqualsZero]{\ensuremath{a\in\mathbb{N},b=0\vdash ab=0}(#1)}}

    % a∈ℕ,ab≠0⊢b≠0
    \newcommand{\aInNaturalwabNotEqualsZeroImpbNotEqualsZero}[1]{\hyperref[aInNaturalwabNotEqualsZeroImpbNotEqualsZero]{\ensuremath{a\in\mathbb{N},ab\neq 0\vdash b\neq 0}(#1)}}

    % a,b,c∈ℕ,a=b⊢ac=bc
    \newcommand{\aInNaturalwbInNaturalwcInNaturalLpaEqualsbImpacEqualsbcRp}[1]{\hyperref[aInNaturalwbInNaturalwcInNaturalLpaEqualsbImpacEqualsbcRp]{\ensuremath{a,b,c\in\mathbb{N},a=b\vdash ac=bc}(#1)}}

    % a,b,c∈ℕ,ac≠bc⊢a≠b
    \newcommand{\awbwcInNaturalwacNotEqualsbcImpaNotEqualsb}[1]{\hyperref[awbwcInNaturalwacNotEqualsbcImpaNotEqualsb]{\ensuremath{a,b,c\in\mathbb{N},ac\neq bc\vdash a\neq b}(#1)}}

    % a,b,c∈ℕ,a=b⊢ca=cb
    \newcommand{\awbwcInNaturalwaEqualsbImpcaEqualscb}[1]{\hyperref[awbwcInNaturalwaEqualsbImpcaEqualscb]{\ensuremath{a,b,c\in\mathbb{N},a=b\vdash ca=cb}(#1)}}

    % a,b,c∈ℕ,ca≠cb⊢a≠b
    \newcommand{\awbwcInNaturalwcaNotEqualscbImpaNotEqualsb}[1]{\hyperref[awbwcInNaturalwcaNotEqualscbImpaNotEqualsb]{\ensuremath{a,b,c\in\mathbb{N},ca\neq cb\vdash a\neq b}(#1)}}

    % Multiplikation und kleiner gleich Relation
    
    % a,b,c∈ℕ⊢a≤b⊢ac≤bc
    \newcommand{\awbwcInNaturalImpaLeqbImpacLeqbc}[1]{\hyperref[awbwcInNaturalImpaLeqbImpacLeqbc]{\ensuremath{a,b,c\in\mathbb{N}\vdash a\leq b\vdash ac\leq bc}(#1)}}

    % a,b,c∈ℕ⊢a≤b⊢ca≤cb
    \newcommand{\awbwcInNaturalImpaLeqbImpcaLeqcb}[1]{\hyperref[awbwcInNaturalImpaLeqbImpcaLeqcb]{\ensuremath{a,b,c\in\mathbb{N}\vdash a\leq b\vdash ca\leq cb}(#1)}}

    % a,b∈ℕ,b≠0⊢a≤ab
    \newcommand{\awbInNaturalwbNotEqualsZeroImpaLeqab}[1]{\hyperref[awbInNaturalwbNotEqualsZeroImpaLeqab]{\ensuremath{a,b\in\mathbb{N}, b\neq 0\vdash a\leq ab}(#1)}}

    % a,b∈ℕ,b≠0⊢a≤ba
    \newcommand{\awbInNaturalwbNotEqualsZeroImpaLeqba}[1]{\hyperref[awbInNaturalwbNotEqualsZeroImpaLeqba]{\ensuremath{a,b\in\mathbb{N}, b\neq 0\vdash a\leq ba}(#1)}}
    

    % a,b,c∈ℕ,c≤b⊢(b-c)a=ba-ca
    \newcommand{\awbwcInNaturalwcLeqbImpLpbMinuscRpaEqualsbaMinusca}[1]{\hyperref[awbwcInNaturalwcLeqbImpLpbMinuscRpaEqualsbaMinusca]{\ensuremath{a,b,c\in\mathbb{N},c\leq b\vdash (b-c)a=ba-ca}(#1)}}

    % a,b,c∈ℕ,c≤b⊢a(b-c)=ab-ac
    \newcommand{\awbwcInNaturalwcLeqbImpaLpbMinuscRpEqualsabMinusac}[1]{\hyperref[awbwcInNaturalwcLeqbImpaLpbMinuscRpEqualsabMinusac]{\ensuremath{a,b,c\in\mathbb{N},c\leq b\vdash a(b-c)=ab-ac}(#1)}}

    % a,b∈ℕ,a≠0,ab=0⊢b=0
    \newcommand{\awbInNaturalwaNotEqualsZerowabEqualsZeroImpbEqualsZero}[1]{\hyperref[awbInNaturalwaNotEqualsZerowabEqualsZeroImpbEqualsZero]{\ensuremath{a,b\in\mathbb{N},a\neq 0,ab=0\vdash b=0}(#1)}}

    % a,b∈ℕ,b≠0,ab=0⊢a=0
    \newcommand{\awbInNaturalwbNotEqualsZerowabEqualsZeroImpaEqualsZero}[1]{\hyperref[awbInNaturalwbNotEqualsZerowabEqualsZeroImpaEqualsZero]{\ensuremath{a,b\in\mathbb{N},b\neq 0,ab=0\vdash a=0}(#1)}}

    % a,b∈ℕ,a≠0,b≠0⊢ab≠0
    \newcommand{\awbInNaturalwaNotEqualsZerowbNotEqualsZeroImpabNotEqualsZero}[1]{\hyperref[awbInNaturalwaNotEqualsZerowbNotEqualsZeroImpabNotEqualsZero]{\ensuremath{a,b\in\mathbb{N},a\neq 0,b\neq 0\vdash ab\neq 0}(#1)}}

    % a,b,c∈ℕ,c≠0,ac=bc⊢a=b
    \newcommand{\awbwcInNaturalwcNotEqualsZerowacEqualsbcImpaEqualsb}[1]{\hyperref[awbwcInNaturalwcNotEqualsZerowacEqualsbcImpaEqualsb]{\ensuremath{a,b,c\in\mathbb{N},c\neq 0,ac=bc\vdash a=b}(#1)}}

    % a,b,c∈ℕ,c≠0,a≠b⊢ac≠bc
    \newcommand{\awbwcInNaturalwcNotEqualsZerowaNotEqualsbImpacNotEqualsbc}[1]{\hyperref[awbwcInNaturalwcNotEqualsZerowaNotEqualsbImpacNotEqualsbc]{\ensuremath{a,b,c\in\mathbb{N},c\neq 0,a\neq b\vdash ac\neq bc}(#1)}}

    % a,b,c∈ℕ,c≠0,ca=cb⊢a=b
    \newcommand{\awbwcInNaturalwcNotEqualsZerowcaEqualscbImpaEqualsb}[1]{\hyperref[awbwcInNaturalwcNotEqualsZerowcaEqualscbImpaEqualsb]{\ensuremath{a,b,c\in\mathbb{N},c\neq 0,ca=cb\vdash a=b}(#1)}}

    % a,b,c∈ℕ,c≠0,a≠b⊢ca≠cb
    \newcommand{\awbwcInNaturalwcNotEqualsZerowaNotEqualsbImpcaNotEqualscb}[1]{\hyperref[awbwcInNaturalwcNotEqualsZerowaNotEqualsbImpcaNotEqualscb]{\ensuremath{a,b,c\in\mathbb{N},c\neq 0,a\neq b\vdash ca\neq cb}(#1)}}


    % Ordnungsrelationen und Multiplikation in den natürlichen Zahlen

    % a,b,c∈ℕ,c≠0,ac≤bc⊢a≤b
    \newcommand{\awbwcInNaturalwcNotEqualsZerowacLeqbcImpaLeqb}[1]{\hyperref[awbwcInNaturalwcNotEqualsZerowacLeqbcImpaLeqb]{\ensuremath{a,b,c\in\mathbb{N},c\neq 0, ac\leq bc\vdash a\leq b}(#1)}}

    % a,b,c∈ℕ,c≠0,ca≤cb⊢a≤b
    \newcommand{\awbwcInNaturalwcNotEqualsZerowcaLeqcbImpaLeqb}[1]{\hyperref[awbwcInNaturalwcNotEqualsZerowcaLeqcbImpaLeqb]{\ensuremath{a,b,c\in\mathbb{N},c\neq 0, ca\leq cb\vdash a\leq b}(#1)}}

    % a,b,c∈ℕ,c≠0,ac<bc⊣⊢a<b
    \newcommand{\awbwcInNaturalwcNotEqualsZerowacLneqbcEqvaLneqb}[1]{\hyperref[awbwcInNaturalwcNotEqualsZerowacLneqbcEqvaLneqb]{\ensuremath{a,b,c\in\mathbb{N},c\neq 0, ac<bc\dashv\vdash a<b}(#1)}}


    % a,b,c∈ℕ,c≠0,ca<cb⊣⊢a<b
    \newcommand{\awbwcInNaturalwcNotEqualsZerowcaLneqcbEqvaLneqb}[1]{\hyperref[awbwcInNaturalwcNotEqualsZerowcaLneqcbEqvaLneqb]{\ensuremath{a,b,c\in\mathbb{N},c\neq 0, ca<cb\dashv\vdash a<b}(#1)}}

    % a,b∈ℕ,b>1⊢a<ba
    \newcommand{\awbInNaturalwbGneqOneImpaLneqba}[1]{\hyperref[awbInNaturalwbGneqOneImpaLneqba]{\ensuremath{a,b\in\mathbb{N},b>1\vdash a<ba}(#1)}}


    %Division natürlicher Zahlen

    % Regeln für die Teilbarkeit
    \newcommand{\rDivisibilityE}[1]{\hyperref[rule:rDivisibilityE]{\ensuremath{\mid}E(#1)}}

    \newcommand{\rDivisibilityI}[1]{\hyperref[rule:rDivisibilityI]{\ensuremath{\mid}I(#1)}}

    % ∃k_1,k_2∈ℕ(a=b·k_1∧a=b·k_2)⊢k_1=k_2
    \newcommand{\ExkSubOnewkSubTwoInNaturalLpaEqualsbMultkSubOneAndaEqualsbMultkSubTwoRpImpkSubOneEqualskSubTwo}[1]{\hyperref[ExkSubOnewkSubTwoInNaturalLpaEqualsbMultkSubOneAndaEqualsbMultkSubTwoRpImpkSubOneEqualskSubTwo]{\ensuremath{\exists k_1, k_2 \in \mathbb{N} (a = b \cdot k_1 \land a = b \cdot k_2) \vdash k_1 = k_2}(#1)}}

    % Regeln für die Division
    \newcommand{\rDivisionI}[1]{\hyperref[rule:rDivisionI]{\ensuremath{\div}I(#1)}}

    % a∈ℕ⊢1|a
    \newcommand{\aInNaturalImpOneMida}[1]{\hyperref[aInNaturalImpOneMida]{\ensuremath{a\in\mathbb{N}\vdash 1\mid a}(#1)}}

    % a∈ℕ⊢(a)/(1)=a
    \newcommand{\aInNaturalImpLpaRpDurchLpOneRpEqualsa}[1]{\hyperref[aInNaturalImpLpaRpDurchLpOneRpEqualsa]{\ensuremath{a\in\mathbb{N}\vdash \frac{a}{1}=a}(#1)}}

    % a∈ℕ,a≠0⊢(0)/(a)=0
    \newcommand{\aInNaturalwaNotEqualsZeroImpLpZeroRpDurchLpaRpEqualsZero}[1]{\hyperref[aInNaturalwaNotEqualsZeroImpLpZeroRpDurchLpaRpEqualsZero]{\ensuremath{a\in\mathbb{N},a\neq 0\vdash \frac{0}{a}=0}(#1)}}

    % a,b∈ℕ,b≠0,a|b⊢(b)/(a)≠0
    \newcommand{\awbInNaturalwbNotEqualsZerowaMidbImpLpbRpDurchLpaRpNotEqualsZero}[1]{\hyperref[awbInNaturalwbNotEqualsZerowaMidbImpLpbRpDurchLpaRpNotEqualsZero]{\ensuremath{a,b \in\mathbb{N}, b\neq 0, a\mid b\vdash \frac{b}{a}\neq 0}(#1)}}

    % a,b∈ℕ,b≠0,a|b⊢a≤b
    \newcommand{\awbInNaturalwbNotEqualsZerowaMidbImpaLeqb}[1]{\hyperref[awbInNaturalwbNotEqualsZerowaMidbImpaLeqb]{\ensuremath{a,b \in\mathbb{N}, b\neq 0, a\mid b\vdash a\leq b}(#1)}}


    % a,b,c∈ℕ,c|a,c|b⊢c|{a+b}
    \newcommand{\awbwcInNaturalwcMidawcMidbImpcMidLbaPlusbRb}[1]{\hyperref[awbwcInNaturalwcMidawcMidbImpcMidLbaPlusbRb]{\ensuremath{a,b,c\in\mathbb{N}, c\mid a, c\mid b\vdash c\mid{a+b}}(#1)}}

    % a∈ℕ,b∈ℕ,c∈ℕ,c|a,c|b⊢(a+b)/(c)=(a)/(c)+(b)/(c)
    \newcommand{\aInNaturalwbInNaturalwcInNaturalwcMidawcMidbImpLpaPlusbRpDurchLpcRpEqualsLpaRpDurchLpcRpPlusLpbRpDurchLpcRp}[1]{\hyperref[aInNaturalwbInNaturalwcInNaturalwcMidawcMidbImpLpaPlusbRpDurchLpcRpEqualsLpaRpDurchLpcRpPlusLpbRpDurchLpcRp]{\ensuremath{a\in\mathbb{N},b\in\mathbb{N}, c\in\mathbb{N}, c\mid a, c\mid b\vdash \frac{a+b}{c}=\frac{a}{c}+\frac{b}{c}}(#1)}}

    % a,b,c,d∈ℕ,c|a,d|b⊢cd|ab
    \newcommand{\awbwcwdInNaturalwcMidawdMidbImpcdMidab}[1]{\hyperref[awbwcwdInNaturalwcMidawdMidbImpcdMidab]{\ensuremath{a,b,c,d\in\mathbb{N},c\mid a, d\mid b\vdash cd\mid ab}(#1)}}

    % a,b,c,d∈ℕ,c|a,d|b⊢(ab)/(cd)=(a)/(c)·(b)/(d)
    \newcommand{\awbwcwdInNaturalwcMidawdMidbImpLpabRpDurchLpcdRpEqualsLpaRpDurchLpcRpMultLpbRpDurchLpdRp}[1]{\hyperref[awbwcwdInNaturalwcMidawdMidbImpLpabRpDurchLpcdRpEqualsLpaRpDurchLpcRpMultLpbRpDurchLpdRp]{\ensuremath{a,b,c,d\in\mathbb{N},c\mid a, d\mid b\vdash \frac{ab}{cd}=\frac{a}{c}\cdot\frac{b}{d}}(#1)}}

    % a∈ℕ,b∈ℕ,c∈ℕ,d∈ℕ,c|a,d|b⊢dc|ad+bc
    \newcommand{\aInNaturalwbInNaturalwcInNaturalwdInNaturalwcMidawdMidbImpdcMidadPlusbc}[1]{\hyperref[aInNaturalwbInNaturalwcInNaturalwdInNaturalwcMidawdMidbImpdcMidadPlusbc]{\ensuremath{a\in\mathbb{N},b\in\mathbb{N}, c\in\mathbb{N}, d\in\mathbb{N},c\mid a, d\mid b\vdash dc\mid ad+bc}(#1)}}

    % a∈ℕ,b∈ℕ,c∈ℕ,d∈ℕ,c|a,d|b⊢(ad+bc)/(dc)=(a)/(c)+(b)/(d)
    \newcommand{\aInNaturalwbInNaturalwcInNaturalwdInNaturalwcMidawdMidbImpLpadPlusbcRpDurchLpdcRpEqualsLpaRpDurchLpcRpPlusLpbRpDurchLpdRp}[1]{\hyperref[aInNaturalwbInNaturalwcInNaturalwdInNaturalwcMidawdMidbImpLpadPlusbcRpDurchLpdcRpEqualsLpaRpDurchLpcRpPlusLpbRpDurchLpdRp]{\ensuremath{a\in\mathbb{N},b\in\mathbb{N}, c\in\mathbb{N}, d\in\mathbb{N},c\mid a, d\mid b\vdash \frac{ad+bc}{dc}=\frac{a}{c}+\frac{b}{d}}(#1)}}

    % a,b∈ℕ,a|b,a|c,b≤c⊢(b)/(a)≤(c)/(a)
    \newcommand{\awbInNaturalwaMidbwaMidcwbLeqcImpLpbRpDurchLpaRpLeqLpcRpDurchLpaRp}[1]{\hyperref[awbInNaturalwaMidbwaMidcwbLeqcImpLpbRpDurchLpaRpLeqLpcRpDurchLpaRp]{\ensuremath{a,b\in\mathbb{N}, a\mid b, a\mid c, b\leq c\vdash \frac{b}{a}\leq \frac{c}{a}}(#1)}}

    % a∈ℕ,a≠0⊢a|{a}
    \newcommand{\aInNaturalwaNotEqualsZeroImpaMidLbaRb}[1]{\hyperref[aInNaturalwaNotEqualsZeroImpaMidLbaRb]{\ensuremath{a\in\mathbb{N}, a\neq 0 \vdash a\mid{a}}(#1)}}

    % a∈ℕ,a≠0⊢(a)/(a)=1
    \newcommand{\aInNaturalwaNotEqualsZeroImpLpaRpDurchLpaRpEqualsOne}[1]{\hyperref[aInNaturalwaNotEqualsZeroImpLpaRpDurchLpaRpEqualsOne]{\ensuremath{a\in\mathbb{N}, a\neq 0 \vdash \frac{a}{a}=1}(#1)}}

    % a,b∈ℕ,a|b,b|a⊢a=b
    \newcommand{\awbInNaturalwaMidbwbMidaImpaEqualsb}[1]{\hyperref[awbInNaturalwaMidbwbMidaImpaEqualsb]{\ensuremath{a,b\in\mathbb{N}, a\mid b, b\mid a \vdash a=b}(#1)}}

    % a,b,c∈ℕ,a|b,b|c⊢a|c
    \newcommand{\awbwcInNaturalwaMidbwbMidcImpaMidc}[1]{\hyperref[awbwcInNaturalwaMidbwbMidcImpaMidc]{\ensuremath{a,b,c\in\mathbb{N}, a\mid b, b\mid c \vdash a\mid c}(#1)}}

    % ⊢| ist eine Halbordnung auf ℕ
    \newcommand{\MidIsHalfOrderOnNaturalNumbers}[1]{\hyperref[MidIsHalfOrderOnNaturalNumbers]{\ensuremath{\mid \text{ ist eine Halbordnung auf } \mathbb{N} }(#1)}}

    % a,b,c∈ℕ,a|b,b|c⊢(a)/(c)=(b)/(a)·(c)/(b)
    \newcommand{\awbwcInNaturalwaMidbwbMidcImpLpaRpDurchLpcRpEqualsLpbRpDurchLpaRpMultLpcRpDurchLpbRp}[1]{\hyperref[awbwcInNaturalwaMidbwbMidcImpLpaRpDurchLpcRpEqualsLpbRpDurchLpaRpMultLpcRpDurchLpbRp]{\ensuremath{a,b,c\in\mathbb{N}, a\mid b, b\mid c \vdash \frac{a}{c}=\frac{b}{a}\cdot\frac{c}{b}}(#1)}}

    % a,b∈ℕ,ab=1⊢a=1∧b=1
    \newcommand{\awbInNaturalwabEqualsOneImpaEqualsOneAndbEqualsOne}[1]{\hyperref[awbInNaturalwabEqualsOneImpaEqualsOneAndbEqualsOne]{\ensuremath{a,b\in\mathbb{N},ab=1\vdash a=1\land b=1}(#1)}}

    % a,b∈ℕ,a≠1,a|b⊢a∤b+1
    \newcommand{\awbInNaturalwaNotEqualsOnewaMidbImpaNMidbPlusOne}[1]{\hyperref[awbInNaturalwaNotEqualsOnewaMidbImpaNMidbPlusOne]{\ensuremath{a,b\in\mathbb{N}, a\neq 1, a\mid b \vdash a\nmid b+1}(#1)}}

    % Existenz der Division mit Rest
    % ∈M:={r∈ℕ|∃q∈ℕ(r=a-bq)} I
    \newcommand{\tIMDefineEqualsLbrInNaturalMidExqInNaturalLprEqualsaMinusbqRpRb}[1]{\hyperref[tIMDefineEqualsLbrInNaturalMidExqInNaturalLprEqualsaMinusbqRpRb]{\ensuremath{\in M_I}(#1)}}

    % ∈M:={r∈ℕ|∃q∈ℕ(r=a-bq)} E
    \newcommand{\tEMDefineEqualsLbrInNaturalMidExqInNaturalLprEqualsaMinusbqRpRb}[1]{\hyperref[tEMDefineEqualsLbrInNaturalMidExqInNaturalLprEqualsaMinusbqRpRb]{\ensuremath{\in M_E}(#1)}}

    % M:={r∈ℕ|∃q∈ℕ(r=a-bq)}⊆ℕ
    \newcommand{\tMDefineEqualsLbrInNaturalMidExqInNaturalLprEqualsaMinusbqRpRbSubseteqNatural}[1]{\hyperref[tMDefineEqualsLbrInNaturalMidExqInNaturalLprEqualsaMinusbqRpRbSubseteqNatural]{\ensuremath{M\subseteq\mathbb{N}}(#1)}}

    % M:={r∈ℕ|∃q∈ℕ(r=a-bq)≠∅
    \newcommand{\tMDefineEqualsLbrInNaturalMidExqInNaturalLprEqualsaMinusbqRpNotEqualsEmptyset}[1]{\hyperref[tMDefineEqualsLbrInNaturalMidExqInNaturalLprEqualsaMinusbqRpNotEqualsEmptyset]{\ensuremath{M\neq\emptyset}(#1)}}

    % n∈M:={r∈ℕ|∃q∈ℕ(r=a-bq)},b≤n⊢n-b∈M
    \newcommand{\tnInMDefineEqualsLbrInNaturalMidExqInNaturalLprEqualsaMinusbqRpRbwbLeqnImpnMinusbInM}[1]{\hyperref[tnInMDefineEqualsLbrInNaturalMidExqInNaturalLprEqualsaMinusbqRpRbwbLeqnImpnMinusbInM]{\ensuremath{n\in M,b\leq n\vdash n-b\in M}(#1)}}

    % ∃n∈M:={r∈ℕ|∃q∈ℕ(r=a-bq)}(b=min(M)
    \newcommand{\tExnInMDefineEqualsLbrInNaturalMidExqInNaturalLprEqualsaMinusbqRpRbLpnEqualsMinLpMRp}[1]{\hyperref[tExnInMDefineEqualsLbrInNaturalMidExqInNaturalLprEqualsaMinusbqRpRbLpnEqualsMinLpMRp]{\ensuremath{\exists n\in M(n=\min(M)}(#1)}}

    % min(M:={r∈ℕ|∃q∈ℕ(r=a-bq)})∈ℕ
    \newcommand{\MinLpMDefineEqualsLbrInNaturalMidExqInNaturalLprEqualsaMinusbqRpRbRpInNatural}[1]{\hyperref[MinLpMDefineEqualsLbrInNaturalMidExqInNaturalLprEqualsaMinusbqRpRbRpInNatural]{\ensuremath{\min(M)\in\mathbb{N}}(#1)}}

    % b≠0⊢min(M:={r∈ℕ|∃q∈ℕ(r=a-bq)})<b
    \newcommand{\tbNotEqualsZeroImpMinLpMDefineEqualsLbrInNaturalMidExqInNaturalLprEqualsaMinusbqRpRbRpLneqb}[1]{\hyperref[tbNotEqualsZeroImpMinLpMDefineEqualsLbrInNaturalMidExqInNaturalLprEqualsaMinusbqRpRbRpLneqb]{\ensuremath{b\neq 0\vdash \min(M)<b}(#1)}}

    % ∃q∈ℕ(a=bq+min(M:={r∈ℕ|∃q∈ℕ(r=a-bq)}))
    \newcommand{\tExqInNaturalLpaEqualsbqPlusMinLpMDefineEqualsLbrInNaturalMidExqInNaturalLprEqualsaMinusbqRpRbRpRp}[1]{\hyperref[tExqInNaturalLpaEqualsbqPlusMinLpMDefineEqualsLbrInNaturalMidExqInNaturalLprEqualsaMinusbqRpRbRpRp]{\ensuremath{\exists q\in\mathbb{N}(a=bq+\min(M))}(#1)}}

    % b≠0⊢∃r∈ℕ(r<b∧∃q∈ℕ(a=bq+r))
    \newcommand{\bNotEqualsZeroImpExrInNaturalLprLneqbAndExqInNaturalLpaEqualsbqPlusrRpRp}[1]{\hyperref[bNotEqualsZeroImpExrInNaturalLprLneqbAndExqInNaturalLpaEqualsbqPlusrRpRp]{\ensuremath{b\neq 0\vdash \exists r\in\mathbb{N}(r<b\land \exists q\in\mathbb{N}(a=bq+r))}(#1)}}

        % a,b,c,d,f∈ℕ,a≤c,b≤d⊢f(c-a)+(d-b)=(fc+d)-(fa+b)
    \newcommand{\awbwcwdwfInNaturalwaLeqcwbLeqdImpfLpcMinusaRpPlusLpdMinusbRpEqualsLpfcPlusdRpMinusLpfaPlusbRp}[1]{\hyperref[awbwcwdwfInNaturalwaLeqcwbLeqdImpfLpcMinusaRpPlusLpdMinusbRpEqualsLpfcPlusdRpMinusLpfaPlusbRp]{\ensuremath{a,b,c,d,f\in\mathbb{N},a\leq c, b\leq d\vdash f(c-a)+(d-b)=(fc+d)-(fa+b)}(#1)}}

    % a,b,c,d,f∈ℕ,f≠0,f(c-a)=0∧(d-b)=0⊢c=a∧d=b
    \newcommand{\awbwcwdwfInNaturalwfNotEqualsZerowfLpcMinusaRpEqualsZeroAndLpdMinusbRpEqualsZeroImpcEqualsaAnddEqualsb}[1]{\hyperref[awbwcwdwfInNaturalwfNotEqualsZerowfLpcMinusaRpEqualsZeroAndLpdMinusbRpEqualsZeroImpcEqualsaAnddEqualsb]{\ensuremath{a,b,c,d,f\in\mathbb{N},f\neq 0, f(c-a)=0\land (d-b)=0 \vdash c=a\land d=b}(#1)}}

    % a,b,c,d,f∈ℕ,c≤a,b≤d,(fa+b=fc+d⊣⊢f(a-c)=d-b)
    \newcommand{\awbwcwdwfInNaturalwcLeqawbLeqdwLpfaPlusbEqualsfcPlusdEqvfLpaMinuscRpEqualsdMinusbRp}[1]{\hyperref[awbwcwdwfInNaturalwcLeqawbLeqdwLpfaPlusbEqualsfcPlusdEqvfLpaMinuscRpEqualsdMinusbRp]{\ensuremath{a,b,c,d,f\in\mathbb{N},c\leq a,b\leq d,(fa+b=fc+d \dashv\vdash f(a-c)=d-b)}(#1)}}

    % a,b∈ℕ,b≠0,∃q_1,r_1,q_2,r_2∈ℕ(a=b·q_1+r_1∧r_1<b∧a=b·q_2+r_2∧r_2<b)⊢q_1=q_2∧r_1=r_2
    \newcommand{\awbInNaturalwbNotEqualsZerowExqSubOnewrSubOnewqSubTwowrSubTwoInNaturalLpaEqualsbMultqSubOnePlusrSubOneAndrSubOneLneqbAndaEqualsbMultqSubTwoPlusrSubTwoAndrSubTwoLneqbRpImpqSubOneEqualsqSubTwoAndrSubOneEqualsrSubTwo}[1]{\hyperref[awbInNaturalwbNotEqualsZerowExqSubOnewrSubOnewqSubTwowrSubTwoInNaturalLpaEqualsbMultqSubOnePlusrSubOneAndrSubOneLneqbAndaEqualsbMultqSubTwoPlusrSubTwoAndrSubTwoLneqbRpImpqSubOneEqualsqSubTwoAndrSubOneEqualsrSubTwo]{\ensuremath{a,b\in\mathbb{N},b\neq 0,\exists q_1, r_1, q_2, r_2 \in \mathbb{N}(a = b \cdot q_1 + r_1 \land r_1 < b \land a = b \cdot q_2 + r_2 \land r_2 < b)\vdash q_1 = q_2 \land r_1 = r_2}(#1)}}

    % Regeln für die Division mit Rest
    \newcommand{\rDivisionWithRemainderI}[1]{\hyperref[rule:rDivisionWithRemainderI]{\ensuremath{\text{Def. - Division mit Rest}}
(#1)}}

    % a,b∈ℕ,b≠0,a<b⊢a_div_b=0
    \newcommand{\awbInNaturalwbNotEqualsZerowaLneqbImpaDivbEqualsZero}[1]{\hyperref[awbInNaturalwbNotEqualsZerowaLneqbImpaDivbEqualsZero]{\ensuremath{a,b\in\mathbb{N},b\neq 0, a<b\vdash a\div b=0}(#1)}}

    % a,b∈ℕ,b≠0,a<b⊢a=a_mod_b
    \newcommand{\awbInNaturalwbNotEqualsZerowaLneqbImpaEqualsaModb}[1]{\hyperref[awbInNaturalwbNotEqualsZerowaLneqbImpaEqualsaModb]{\ensuremath{a,b\in\mathbb{N},b\neq 0, a<b\vdash a=a\bmod b}(#1)}}


    % a,b∈ℕ,b≠0(a<b⊣⊢a_div_b=0∧a=a_mod_b)
    \newcommand{\awbInNaturalwbNotEqualsZeroLpaLneqbEqvaDivbEqualsZeroAndaEqualsaModbRp}[1]{\hyperref[awbInNaturalwbNotEqualsZeroLpaLneqbEqvaDivbEqualsZeroAndaEqualsaModbRp]{\ensuremath{a,b\in\mathbb{N},b\neq 0(a<b\dashv\vdash a\div b=0\land  a=a\bmod b)}(#1)}}

    % Ordnungsrelationen bei der Division
    % a,b∈ℕ,b≠0⊢a_div_b≤a
    \newcommand{\awbInNaturalwbNotEqualsZeroImpaDivbLeqa}[1]{\hyperref[awbInNaturalwbNotEqualsZeroImpaDivbLeqa]{\ensuremath{a,b\in\mathbb{N}, b\neq 0\vdash a\div b\leq a}(#1)}}

    % a,b∈ℕ,b>1⊢a_div_b<a
    \newcommand{\awbInNaturalwbGneqOneImpaDivbLneqa}[1]{\hyperref[awbInNaturalwbGneqOneImpaDivbLneqa]{\ensuremath{a,b\in\mathbb{N}, b > 1\vdash a\div b<a}(#1)}}


    % Teilbarkeit und Rest
    % a,b∈ℕ,a≠0,a|b⊣⊢b_div_a=(b)/(a)∧b_mod_a=0
    \newcommand{\awbInNaturalwaNotEqualsZerowaMidbEqvbDivaEqualsLpbRpDurchLpaRpAndbModaEqualsZero}[1]{\hyperref[awbInNaturalwaNotEqualsZerowaMidbEqvbDivaEqualsLpbRpDurchLpaRpAndbModaEqualsZero]{\ensuremath{a,b\in\mathbb{N},a\neq 0,a\mid b\dashv\vdash b\div a=\frac{b}{a}\land b\bmod a=0}(#1)}}

    % a,b∈ℕ,a|b⊢b_div_a=(b)/(a)
    \newcommand{\awbInNaturalwaMidbImpbDivaEqualsLpbRpDurchLpaRp}[1]{\hyperref[awbInNaturalwaMidbImpbDivaEqualsLpbRpDurchLpaRp]{\ensuremath{a,b\in\mathbb{N},a\mid b\vdash b\div a=\frac{b}{a}}(#1)}}

    % a,b∈ℕ,a|b⊢b_mod_a=0
    \newcommand{\awbInNaturalwaMidbImpbModaEqualsZero}[1]{\hyperref[awbInNaturalwaMidbImpbModaEqualsZero]{\ensuremath{a,b\in\mathbb{N},a\mid b\vdash b\bmod a=0}(#1)}}

    % a,b,k∈ℕ,b≠0⊢(a+bk)_mod_b=a_mod_b
    \newcommand{\awbwkInNaturalwbNotEqualsZeroImpLpaPlusbkRpModbEqualsaModb}[1]{\hyperref[awbwkInNaturalwbNotEqualsZeroImpLpaPlusbkRpModbEqualsaModb]{\ensuremath{a,b,k\in\mathbb{N},b\neq 0\vdash (a+bk)\bmod b=a\bmod b}(#1)}}

    % a,b,k∈ℕ,b≠0⊢(a+bk)_div_b=(a_div_b)+k

    % Endliche Summen natürlicher Zahlen
    % Linksdistributivität endlicher Summen
    \newcommand{\awbwkInNaturalwbNotEqualsZeroImpLpaPlusbkRpDivbEqualsLpaDivbRpPlusk}[1]{\hyperref[awbwkInNaturalwbNotEqualsZeroImpLpaPlusbkRpDivbEqualsLpaDivbRpPlusk]{\ensuremath{a,b,k\in\mathbb{N},b\neq 0\vdash (a+bk)\div b=(a\div b)+k}(#1)}}

    % a,s_{i_0}∈ℕ⊢a∑_{i=i_0}^{i_0}s_i=∑_{i=i_0}^{i_0}as_i
    \newcommand{\awsSubLbiSubZeroRbInNaturalImpaSumSubLbiEqualsiSubZeroRbPowerLbiSubZeroRbsSubiEqualsSumSubLbiEqualsiSubZeroRbPowerLbiSubZeroRbasSubi}[1]{\hyperref[awsSubLbiSubZeroRbInNaturalImpaSumSubLbiEqualsiSubZeroRbPowerLbiSubZeroRbsSubiEqualsSumSubLbiEqualsiSubZeroRbPowerLbiSubZeroRbasSubi]{\ensuremath{\text{IA}}(#1)}}

    % a,n,s_{i_0},...,s_{n+1}∈ℕ,a∑_{i=i_0}^{n}s_i=∑_{i=i_0}^{n}as_i⊢a∑_{i=i_0}^{n+1}s_i=∑_{i=i_0}^{n+1}as_i
    \newcommand{\awnwsSubLbiSubZeroRbwDotswsSubLbnPlusOneRbInNaturalwaSumSubLbiEqualsiSubZeroRbPowerLbnRbsSubiEqualsSumSubLbiEqualsiSubZeroRbPowerLbnRbasSubiImpaSumSubLbiEqualsiSubZeroRbPowerLbnPlusOneRbsSubiEqualsSumSubLbiEqualsiSubZeroRbPowerLbnPlusOneRbasSubi}[1]{\hyperref[awnwsSubLbiSubZeroRbwDotswsSubLbnPlusOneRbInNaturalwaSumSubLbiEqualsiSubZeroRbPowerLbnRbsSubiEqualsSumSubLbiEqualsiSubZeroRbPowerLbnRbasSubiImpaSumSubLbiEqualsiSubZeroRbPowerLbnPlusOneRbsSubiEqualsSumSubLbiEqualsiSubZeroRbPowerLbnPlusOneRbasSubi]{\ensuremath{\text{IS}}(#1)}}


    % a,n∈ℕ,s_{i_0},...,s_n∈ℕ⊢a∑_{i=i_0}^ns_i=∑_{i=i_0}^nas_i
    \newcommand{\awnInNaturalwsSubLbiSubZeroRbwDotswsSubnInNaturalImpaSumSubLbiEqualsiSubZeroRbPowernsSubiEqualsSumSubLbiEqualsiSubZeroRbPowernasSubi}[1]{\hyperref[awnInNaturalwsSubLbiSubZeroRbwDotswsSubnInNaturalImpaSumSubLbiEqualsiSubZeroRbPowernsSubiEqualsSumSubLbiEqualsiSubZeroRbPowernasSubi]{\ensuremath{a,n\in\mathbb{N},s_{i_0},\dots, s_n\in\mathbb{N}\vdash a\sum_{i=i_0}^n s_i=\sum_{i=i_0}^n as_i}(#1)}}

    % Rechtssdistributivität endlicher Summen

    % a,s_{i_0}∈ℕ⊢(∑_{i=i_0}^{i_0}s_i)a=∑_{i=i_0}^{i_0}s_ia
    \newcommand{\awsSubLbiSubZeroRbInNaturalImpLpSumSubLbiEqualsiSubZeroRbPowerLbiSubZeroRbsSubiRpaEqualsSumSubLbiEqualsiSubZeroRbPowerLbiSubZeroRbsSubia}[1]{\hyperref[awsSubLbiSubZeroRbInNaturalImpLpSumSubLbiEqualsiSubZeroRbPowerLbiSubZeroRbsSubiRpaEqualsSumSubLbiEqualsiSubZeroRbPowerLbiSubZeroRbsSubia]{\ensuremath{\text{IA}}(#1)}}

    % a,n,s_{i_0},...,s_{n+1}∈ℕ,(∑_{i=i_0}^{n}s_i)a=∑_{i=i_0}^{n}s_ia⊢(∑_{i=i_0}^{n+1}s_i)a=∑_{i=i_0}^{n+1}s_ia
    \newcommand{\awnwsSubLbiSubZeroRbwDotswsSubLbnPlusOneRbInNaturalwLpSumSubLbiEqualsiSubZeroRbPowerLbnRbsSubiRpaEqualsSumSubLbiEqualsiSubZeroRbPowerLbnRbsSubiaImpLpSumSubLbiEqualsiSubZeroRbPowerLbnPlusOneRbsSubiRpaEqualsSumSubLbiEqualsiSubZeroRbPowerLbnPlusOneRbsSubia}[1]{\hyperref[awnwsSubLbiSubZeroRbwDotswsSubLbnPlusOneRbInNaturalwLpSumSubLbiEqualsiSubZeroRbPowerLbnRbsSubiRpaEqualsSumSubLbiEqualsiSubZeroRbPowerLbnRbsSubiaImpLpSumSubLbiEqualsiSubZeroRbPowerLbnPlusOneRbsSubiRpaEqualsSumSubLbiEqualsiSubZeroRbPowerLbnPlusOneRbsSubia]{\ensuremath{\text{IS}}(#1)}}

    % a,n∈ℕ,s_{i_0},...,s_n∈ℕ⊢(∑_{i=i_0}^ns_i)a=∑_{i=i_0}^ns_ia
    \newcommand{\awnInNaturalwsSubLbiSubZeroRbwDotswsSubnInNaturalImpLpSumSubLbiEqualsiSubZeroRbPowernsSubiRpaEqualsSumSubLbiEqualsiSubZeroRbPowernsSubia}[1]{\hyperref[awnInNaturalwsSubLbiSubZeroRbwDotswsSubnInNaturalImpLpSumSubLbiEqualsiSubZeroRbPowernsSubiRpaEqualsSumSubLbiEqualsiSubZeroRbPowernsSubia]{\ensuremath{a,n\in\mathbb{N},s_{i_0},\dots, s_n\in\mathbb{N}\vdash (\sum_{i=i_0}^n s_i)a=\sum_{i=i_0}^n s_ia}(#1)}}

    % Rechtssdistributivität endlicher Summen mit Potenzen

    % a,s_{i_0}∈ℕ⊢(∑_{i=i_0}^{i_0}s_ia^i)a=∑_{i=i_0}^{i_0}s_ia^{i+1}
    \newcommand{\awsSubLbiSubZeroRbInNaturalImpLpSumSubLbiEqualsiSubZeroRbPowerLbiSubZeroRbsSubiaPoweriRpaEqualsSumSubLbiEqualsiSubZeroRbPowerLbiSubZeroRbsSubiaPowerLbiPlusOneRb}[1]{\hyperref[awsSubLbiSubZeroRbInNaturalImpLpSumSubLbiEqualsiSubZeroRbPowerLbiSubZeroRbsSubiaPoweriRpaEqualsSumSubLbiEqualsiSubZeroRbPowerLbiSubZeroRbsSubiaPowerLbiPlusOneRb]{\ensuremath{\text{IA}}(#1)}}

    % a,n,s_{i_0},...,s_{n+1}∈ℕ,(∑_{i=i_0}^{n}s_ia^i)a=∑_{i=i_0}^{n}s_ia^{i+1}⊢(∑_{i=i_0}^{n+1}s_ia^i)a=∑_{i=i_0}^{n+1}s_ia^{i+1}
    \newcommand{\awnwsSubLbiSubZeroRbwDotswsSubLbnPlusOneRbInNaturalwLpSumSubLbiEqualsiSubZeroRbPowerLbnRbsSubiaPoweriRpaEqualsSumSubLbiEqualsiSubZeroRbPowerLbnRbsSubiaPowerLbiPlusOneRbImpLpSumSubLbiEqualsiSubZeroRbPowerLbnPlusOneRbsSubiaPoweriRpaEqualsSumSubLbiEqualsiSubZeroRbPowerLbnPlusOneRbsSubiaPowerLbiPlusOneRb}[1]{\hyperref[awnwsSubLbiSubZeroRbwDotswsSubLbnPlusOneRbInNaturalwLpSumSubLbiEqualsiSubZeroRbPowerLbnRbsSubiaPoweriRpaEqualsSumSubLbiEqualsiSubZeroRbPowerLbnRbsSubiaPowerLbiPlusOneRbImpLpSumSubLbiEqualsiSubZeroRbPowerLbnPlusOneRbsSubiaPoweriRpaEqualsSumSubLbiEqualsiSubZeroRbPowerLbnPlusOneRbsSubiaPowerLbiPlusOneRb]{\ensuremath{\text{IS}}(#1)}}

    % a,n∈ℕ,s_{i_0},...,s_n∈ℕ⊢(∑_{i=i_0}^ns_ia^i)a=∑_{i=i_0}^ns_ia^{i+1}
    \newcommand{\awnInNaturalwsSubLbiSubZeroRbwDotswsSubnInNaturalImpLpSumSubLbiEqualsiSubZeroRbPowernsSubiaPoweriRpaEqualsSumSubLbiEqualsiSubZeroRbPowernsSubiaPowerLbiPlusOneRb}[1]{\hyperref[awnInNaturalwsSubLbiSubZeroRbwDotswsSubnInNaturalImpLpSumSubLbiEqualsiSubZeroRbPowernsSubiaPoweriRpaEqualsSumSubLbiEqualsiSubZeroRbPowernsSubiaPowerLbiPlusOneRb]{\ensuremath{a,n\in\mathbb{N},s_{i_0},\dots, s_n\in\mathbb{N}\vdash (\sum_{i=i_0}^n s_ia^i)a=\sum_{i=i_0}^n s_ia^{i+1}}(#1)}}

    % Linksdistributivität endlicher Summen mit Potenzen

    % a,n∈ℕ,s_{i_0},...,s_n∈ℕ⊢a∑_{i=i_0}^ns_ia^i=∑_{i=i_0}^ns_ia^{i+1}
    \newcommand{\awnInNaturalwsSubLbiSubZeroRbwDotswsSubnInNaturalImpaSumSubLbiEqualsiSubZeroRbPowernsSubiaPoweriEqualsSumSubLbiEqualsiSubZeroRbPowernsSubiaPowerLbiPlusOneRb}[1]{\hyperref[awnInNaturalwsSubLbiSubZeroRbwDotswsSubnInNaturalImpaSumSubLbiEqualsiSubZeroRbPowernsSubiaPoweriEqualsSumSubLbiEqualsiSubZeroRbPowernsSubiaPowerLbiPlusOneRb]{\ensuremath{a,n\in\mathbb{N},s_{i_0},\dots, s_n\in\mathbb{N}\vdash a\sum_{i=i_0}^n s_ia^i=\sum_{i=i_0}^n s_ia^{i+1}}(#1)}}


    % ∀n∈ℕ(n>1)→∃k∈ℕ∃a_0,a_1,...,a_k∈{0,1,...,n-1}(0=∑_{i=0}^ka_i·n^i)
    \newcommand{\FanInNaturalLpnGneqOneRpToExkInNaturalExaSubZerowaSubOnewDotswaSubkInLbZerowOnewDotswnMinusOneRbLpZeroEqualsSumSubLbiEqualsZeroRbPowerkaSubiMultnPoweriRp}[1]{\hyperref[FanInNaturalLpnGneqOneRpToExkInNaturalExaSubZerowaSubOnewDotswaSubkInLbZerowOnewDotswnMinusOneRbLpZeroEqualsSumSubLbiEqualsZeroRbPowerkaSubiMultnPoweriRp]{\ensuremath{\text{IA}}(#1)}}



    % ∃C∀f(f∈C↔f:{0,1,...,n-1}→A)
    \newcommand{\ExCFafLpfInCLrfDefineLbZerowOnewDotswnMinusOneRbToARp}[1]{\hyperref[ExCFafLpfInCLrfDefineLbZerowOnewDotswnMinusOneRbToARp]{\ensuremath{\exists C\forall f(f\in C\leftrightarrow f: \{0, 1, \dots, n-1\} \to A)}(#1)}}

    % ∀x(x∈E↔x:{0,1,...,n-1}→A)∧∀x(x∈F↔x:{0,1,...,n-1}→A)⊢E=F
    \newcommand{\FaxLpxInELrxDefineLbZerowOnewDotswnMinusOneRbToARpAndFaxLpxInFLrxDefineLbZerowOnewDotswnMinusOneRbToARpImpEEqualsF}[1]{\hyperref[FaxLpxInELrxDefineLbZerowOnewDotswnMinusOneRbToARpAndFaxLpxInFLrxDefineLbZerowOnewDotswnMinusOneRbToARpImpEEqualsF]{\ensuremath{\forall x  (x \in E \leftrightarrow x: \{0, 1, \dots, n-1\} \to A) \land \forall x  (x \in F \leftrightarrow x: \{0, 1, \dots, n-1\} \to A) \vdash E = F}(#1)}}

    % ∃C∀f(f∈C↔f:ℕ→A)
    \newcommand{\ExCFafLpfInCLrfDefineNaturalToARp}[1]{\hyperref[ExCFafLpfInCLrfDefineNaturalToARp]{\ensuremath{\exists C\forall f(f\in C\leftrightarrow f: \mathbb{N} \to A)}(#1)}}

    % ∀x(x∈E↔x:ℕ→A)∧∀x(x∈F↔x:ℕ→A)⊢E=F
    \newcommand{\FaxLpxInELrxDefineNaturalToARpAndFaxLpxInFLrxDefineNaturalToARpImpEEqualsF}[1]{\hyperref[FaxLpxInELrxDefineNaturalToARpAndFaxLpxInFLrxDefineNaturalToARpImpEEqualsF]{\ensuremath{    \forall x  (x \in E \leftrightarrow x: \mathbb{N} \to A) \land \forall x  (x \in F \leftrightarrow x:\mathbb{N} \to A) \vdash E = F}(#1)}}



    


