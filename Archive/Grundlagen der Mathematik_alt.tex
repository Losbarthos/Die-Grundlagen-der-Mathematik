\documentclass{book}
\usepackage{graphicx}
\usepackage[utf8]{inputenc}
\usepackage[ngerman]{babel}
\usepackage{amsmath,amssymb,amsthm}
\usepackage{mathtools}
\usepackage{microtype}
\usepackage{hyperref}
\usepackage{csquotes}
\usepackage{enumitem}
\usepackage{thmtools}
\usepackage{etoolbox}
\usepackage{longtable}
\usepackage{xparse}
\usepackage{imakeidx}

\usepackage{bussproofs}
\tolerance=9999
\MakeAutoQuote{„}{“}
\hypersetup{
    bookmarksopen=false, % Bookmarks standardmäßig zusammenklappen
    bookmarksnumbered=true % Kapitelnummerierung in Bookmarks
}




% ----------------------------------------------------------------------------
% Syntax-Abkürzungen für logische und mengentheoretische Operatoren
% ----------------------------------------------------------------------------
% Dieser Abschnitt definiert Kurzformen (Abkürzungen) für häufig verwendete logische
% und mengentheoretische Operatoren, die in diesem Dokumentin Form von Commands 
% verwendet werden. Diese Kurzformen sind entworfen, um die Lesbarkeit des Codes zu 
% verbessern und eine effiziente Bearbeitung zu ermöglichen.
%
% ----------------------------------------------------------------------------

    \newcommand{\rA}{\hyperref[rule:A]{\ensuremath{A}}}
    % Regeln zum Umgang mit dem ∧-Symbol
    \newcommand{\rAI}[1]{\hyperref[rule:AI]{\ensuremath{\land}I(#1)}}
    \newcommand{\rAEa}[1]{\hyperref[rule:AE1]{\ensuremath{\land}E1(#1)}}
    \newcommand{\rAEb}[1]{\hyperref[rule:AE2]{\ensuremath{\land}E2(#1)}}
    \newcommand{\rAEn}[1]{\hyperref[rule:AEn]{\ensuremath{\land}E(#1)}}	
	
    % Regeln zum Umgang mit dem ∨-Symbol
    \newcommand{\rOIa}[1]{\hyperref[rule:OI1]{\ensuremath{\lor}I1(#1)}}
    \newcommand{\rOIb}[1]{\hyperref[rule:OI2]{\ensuremath{\lor}I2(#1)}}
    \newcommand{\rOE}[1]{\hyperref[rule:OE]{\ensuremath{\lor}E(#1)}}	
    \newcommand{\rOEn}[1]{\hyperref[rule:OEn]{\ensuremath{\lor}E(#1)}}	
	
    % Regeln zum Umgang mit dem →-Symbol
    \newcommand{\rRI}[1]{\hyperref[rule:RI]{\ensuremath{\rightarrow}I(#1)}}	
    \newcommand{\rRE}[1]{\hyperref[rule:RE]{\ensuremath{\rightarrow}E(#1)}}	
	
    % Regeln zu Umgang mit dem ↔.Symbol
    \newcommand{\rLRI}[1]{\hyperref[rule:LRI]{\ensuremath{\leftrightarrow}I(#1)}}	
    \newcommand{\rLREa}[1]{\hyperref[rule:LRE1]
    {\ensuremath{\leftrightarrow}E1(#1)}}	
    \newcommand{\rLREb}[1]{\hyperref[rule:LRE2]{\ensuremath{\leftrightarrow}E2(#1)}}

    \newcommand{\rLRS}[1]{\hyperref[rule:LRSubst]{\ensuremath{\leftrightarrow}S(#1)}}

	
    % Regeln zum Umgang mit dem ∀-Symbol
    \newcommand{\rUI}[1]{\hyperref[rule:UI]{\ensuremath{\forall}I(#1)}}		
    \newcommand{\rUE}[1]{\hyperref[rule:UE]{\ensuremath{\forall}E(#1)}}	
	
	
    % Regeln zum Umgang mit dem ∃-Symbol	
    \newcommand{\rEI}[1]{\hyperref[rule:EI]{\ensuremath{\exists}I(#1)}}				
    \newcommand{\rEE}[1]{\hyperref[rule:EE]{\ensuremath{\exists}E(#1)}}	
	
    % Regeln zum Umgang mit dem ∃!-Symbol
    \newcommand{\UEI}[1]{\hyperref[rule:UEI]{\ensuremath{\exists!}I(#1)}}			
    \newcommand{\UEE}[1]{\hyperref[rule:UEE]{\ensuremath{\exists!}E(#1)}}	

    % Regeln zum Umgang mit dem =-Symbol
    \newcommand{\rII}{\hyperref[rule:II]{\ensuremath{=}I}}
    \newcommand{\rIE}[1]{\hyperref[rule:IE]{\ensuremath{=}E(#1)}}	

    % Regeln zum Umgang mit dem =-Symbol dreier Gleichheiten
    \newcommand{\rIIb}[1]{\hyperref[rule:rIIb]{\ensuremath{=}I(#1)}}
    \newcommand{\rIEb}[1]{\hyperref[rule:rIEb]{\ensuremath{=}E(#1)}}	
	
    \newcommand{\rNeq}{\hyperref[rule:Neq]{\ensuremath{\neq}}}	
	
    % Regeln zum Umgang mit dem ¬-Symbol als auch dem ⊥-Symbol	
    \newcommand{\rBI}[1]{\hyperref[rule:BI]{\ensuremath{\bot}I(#1)}}
    \newcommand{\rCI}[1]{\hyperref[rule:CI]{\ensuremath{\neg}I(#1)}}
    \newcommand{\rCE}[1]{\hyperref[rule:CE]{\ensuremath{\neg}E(#1)}}
    \newcommand{\rDN}[1]{\hyperref[rule:DN]{\ensuremath{}DN(#1)}}


    % Regel zur Kettennotation
    \newcommand{\rChain}[1]{\hyperref[rule:Chain]{\ensuremath{\mathsf{Tr.}}(#1)}}

    %Xor
    \newcommand*\lxor{\mathbin{\veebar}}
    
    \newcommand{\eqvdash}{\dashv\vdash}

    % Powerset
    \newcommand{\powerset}{\mathcal{P}}

    %Induktionsprinzip
    \newcommand{\rInduktion}[1]{\hyperref[rule:Induktion]{\ensuremath{\mathrm{Induktion}}(#1)}}

    




\makeindex[name=satz,title=Sätze und Definitionen zu diesem Kapitel]

\theoremstyle{plain}
\newtheorem{notation*}{Notation}
\newtheorem{theorem}{Theorem}
\newtheorem*{theorem*}{Theorem}
\newtheorem{corollary}[theorem]{Korollar}
\newtheorem*{lemma}{Lemma} % unnummerierte Fakten, mit dessen Hilfe komplexe Beweise geführt werden können
\theoremstyle{remark}
\newtheorem*{remark}{Bemerkung}
\newtheorem*{bemerkung}{Bemerkung}
\theoremstyle{definition}
\newtheorem{definition}{Definition}[section]
\newtheorem{hilfsdefinition}{Hilfsdefinition}[section]
\newtheorem*{tempdefinition}{Temporäre Definition}
\newtheorem*{example}{Beispiel}
\newtheorem*{hint}{Hinweis}

\title{Die Grundlagen der Mathematik}
\author{Martin Kunze}
\date{}



\begin{document}
	\selectlanguage{ngerman}
	\maketitle
	\tableofcontents
	\listoftheorems

\chapter{Aussagenlogik und Prädikatenlogik}

\section{Definitionen im Allgemeinen}

Eine Definition ist eine präzise Festlegung der Bedeutung eines Begriffs. Sie gibt an, welche Eigenschaften und Merkmale etwas besitzen muss, damit wir es eindeutig identifizieren können. In der Mathematik und Logik verwenden wir Definitionen, um Begriffe zu formalisieren und sicherzustellen, dass alle Beteiligten eine gemeinsame Grundlage haben, wenn sie über diese Begriffe sprechen.

Im weiteren Verlauf des Textes werden wir hauptsächlich \textbf{Nominaldefinitionen} verwenden. Eine Nominaldefinition legt fest, wie ein bestimmter Begriff oder Ausdruck innerhalb eines formalen Systems verwendet werden soll. Es handelt sich um sprachliche Festlegungen, die uns helfen, Begriffe klar und eindeutig zu definieren. Nominaldefinitionen bieten eine \textbf{informelle} oder \textbf{intuitive} Beschreibung eines Begriffs, um das Verständnis zu erleichtern, bevor wir ihn formal einführen.

Später im Text, insbesondere bei der Einführung formaler Konzepte wie in der Aussagenlogik oder der Mengenlehre, werden wir auf \textbf{explizite Definitionen} zurückgreifen. Explizite Definitionen liefern eine formale Grundlage für Begriffe, indem sie präzise Kriterien angeben, die erfüllt sein müssen, damit ein Objekt oder eine Aussage eindeutig als Instanz des definierten Begriffs gelten kann.

Da wir uns im nächsten Abschnitt mit der formalen Struktur von Aussagen und ihren Beziehungen befassen werden, ist es hilfreich, diesen allgemeinen Begriff der Definition im Hinterkopf zu behalten. In der Logik verwenden wir Definitionen, um präzise festzulegen, was wir unter einer Aussage und ihren Komponenten verstehen.

\section{Aussagenlogik}

Im Folgenden führen wir die grundlegenden Begriffe der Aussagenlogik ein, die zunächst in Form von \textbf{Nominaldefinitionen} erklärt werden. Diese Definitionen legen fest, wie wir die zentralen Begriffe wie Aussage, atomare Aussage, logische Operatoren und zusammengesetzte Aussage verwenden. Durch die Nominaldefinitionen erhalten wir eine sprachliche Grundlage für unser Verständnis.

In späteren Kapiteln werden diese Begriffe mithilfe von \textbf{expliziten Definitionen} formalisiert. Die expliziten Definitionen erlauben es uns, Aussagen und ihre logischen Verknüpfungen formal zu beschreiben, wodurch eine präzise mathematische Behandlung ermöglicht wird.

Die Aussagenlogik ist ein Bereich der Logik, der sich mit Aussagen und den logischen Beziehungen zwischen ihnen befasst.

\begin{definition}[Aussage]
Eine Aussage ist eine Behauptung, die entweder wahr oder falsch sein kann.
\end{definition}

In der Aussagenlogik verwenden wir Symbole, um Aussagen darzustellen. Zum Beispiel könnten wir die Symbole \(p\) und \(q\) verwenden, um zwei verschiedene Aussagen darzustellen.

\begin{definition}[Atomare Aussage]
Eine atomare Aussage ist eine Aussage, die nicht weiter zerlegt werden kann. Sie bildet die einfachste Einheit in der Aussagenlogik. Beispiele für atomare Aussagen sind \(p\) und \(q\).
\end{definition}


Um die Aussagenlogik genauer zu beschreiben, betrachten wir zwei zentrale Aspekte:

\subsection{Syntax}

Die \textbf{Syntax} der Aussagenlogik beschreibt die formalen Regeln, nach denen gültige Ausdrücke (Formeln) gebildet werden können. Sie legt fest, welche Kombinationen aus atomaren Aussagen, logischen Operatoren und Klammern zulässig sind.

Die wesentlichen syntaktischen Elemente sind:
\begin{itemize}
    \item \textbf{Atomare Aussagen:} Diese bilden die Grundbausteine und werden durch Symbole wie \(p, q, r, \dots\) dargestellt.
    \item \textbf{Logische Operatoren (Junktoren):} Hierzu zählen \(\neg\) (Negation), \(\land\) (Konjunktion), \(\lor\) (Disjunktion) und \(\rightarrow\) (Implikation).
    \item \textbf{Klammern:} Diese werden verwendet, um die Reihenfolge der Operationen in komplexen Ausdrücken eindeutig festzulegen.
\end{itemize}

Die Syntax bezieht sich nur auf die Struktur der Ausdrücke und trifft keine Aussage über deren Bedeutung. Dies ist Gegenstand der Semantik.

\subsection{Semantik}

Die \textbf{Semantik} der Aussagenlogik befasst sich mit der Bedeutung und der Bewertung von Aussagen. Insbesondere wird festgelegt, wie einer Aussage ein Wahrheitswert (\textit{wahr} oder \textit{falsch}) zugeordnet wird.

Für die Semantik der Aussagenlogik gelten folgende Prinzipien:
\begin{itemize}
    \item \textbf{Atomare Aussagen:} Jeder atomaren Aussage wird ein Wahrheitswert zugewiesen.
    \item \textbf{Logische Operatoren:} Die Bedeutung zusammengesetzter Aussagen wird durch die Wahrheitswerte ihrer Bestandteile und die Regeln der Operatoren festgelegt. Beispielsweise gilt:
    \begin{align*}
        \text{Wahrheitstabelle für \(\land\):} & \quad 
        \begin{array}{c|c|c}
            p & q & p \land q \\
            \hline
            \text{wahr} & \text{wahr} & \text{wahr} \\
            \text{wahr} & \text{falsch} & \text{falsch} \\
            \text{falsch} & \text{wahr} & \text{falsch} \\
            \text{falsch} & \text{falsch} & \text{falsch}
        \end{array}
    \end{align*}
\end{itemize}

\begin{example}[Unterschied zwischen Syntax und Semantik]
Betrachten wir die atomaren Aussagen \(a\) und \(b\) mit den Beispielen:
\begin{align*}
    a &: \text{\enquote{Es regnet.}}, \\
    b &: \text{\enquote{Die Straße ist nass.}}.
\end{align*}

Die zu zeigende Aussage lautet:
\begin{align*}
    \text{\enquote{Wenn \(a \land b\) gilt, dann gilt auch \(b \land a\)}}.
\end{align*}

\subsubsection*{Regeln des Kalküls des natürlichen Schließens}
Die folgenden Regeln werden verwendet, um diese Aussage in der Syntax herzuleiten:
\begin{itemize}
    \item \textbf{Regel der Annahme:} Eine Aussage darf als Annahme in den Beweis eingeführt werden, um von dieser ausgehend weitere Schlussfolgerungen zu ziehen.
    \item \textbf{Und-Elimination (Teil 1):} Aus einer Konjunktion \(a \land b\) kann die Teilaussage \(a\) abgeleitet werden.
    \item \textbf{Und-Elimination (Teil 2):} Aus einer Konjunktion \(a \land b\) kann die Teilaussage \(b\) abgeleitet werden.
    \item \textbf{Und-Einführung:} Wenn zwei Aussagen \(a\) und \(b\) hergeleitet wurden, kann die Konjunktion \(a \land b\) gebildet werden.
\end{itemize}

\subsubsection*{Beweis in der Syntax}
Die tabellarische Form eines Beweises im Kalkül des natürlichen Schließens wird verwendet, um Ableitungen übersichtlich darzustellen. Jede Zeile gibt an, welche Regel auf welche vorherigen Aussagen angewendet wurde, um die neue Aussage zu begründen.

Hier ist der Beweis für die Aussage:

\[
\begin{array}{lll}
    (1) & a \land b & \text{Regel der Annahme} \\
    (2) & a & \text{Und-Elimination (Teil 1) auf Zeile (1)} \\
    (3) & b & \text{Und-Elimination (Teil 2) auf Zeile (1)} \\
    (4) & b \land a & \text{Und-Einführung auf Zeilen (3) und (2)} \\
\end{array}
\]

Der Beweis zeigt, dass \(b \land a\) syntaktisch korrekt aus \(a \land b\) hergeleitet werden kann.

\subsubsection*{Wahrheitstabelle in der Semantik}
In der Semantik betrachten wir die Wahrheitswerte der atomaren Aussagen \(a\) und \(b\). Die Wahrheitswertetabelle für \(a \land b\) und \(b \land a\) zeigt die logische Äquivalenz der beiden Ausdrücke:

\[
\begin{array}{c|c|c|c}
    a & b & a \land b & b \land a \\
    \hline
    \text{wahr} & \text{wahr} & \text{wahr} & \text{wahr} \\
    \text{wahr} & \text{falsch} & \text{falsch} & \text{falsch} \\
    \text{falsch} & \text{wahr} & \text{falsch} & \text{falsch} \\
    \text{falsch} & \text{falsch} & \text{falsch} & \text{falsch} \\
\end{array}
\]

Die Semantik bestätigt, dass \(a \land b\) und \(b \land a\) dieselben Wahrheitswerte haben und somit logisch äquivalent sind.

\end{example}
\subsection{Logische Operatoren (Junktoren)}

In der Aussagenlogik verwenden wir logische Operatoren, auch Junktoren genannt, um aus atomaren Aussagen komplexere Aussagen zu bilden. Jeder Operator hat eine spezifische Bedeutung und bestimmt, wie die Wahrheitswerte der atomaren Aussagen miteinander verknüpft werden.

\begin{definition}[Logische Operatoren]
Die grundlegenden logischen Operatoren der Aussagenlogik sind:
\begin{itemize}
    \item Die \textbf{Konjunktion} (\(\land\)): "`und"'
    \item Die \textbf{Disjunktion} (\(\lor\)): "`oder"'
    \item Die \textbf{Implikation} (\(\rightarrow\)): "`impliziert"' oder "`führt zu"'
    \item Die \textbf{Negation} (\(\neg\)): "`nicht"'
\end{itemize}
Diese Operatoren werden verwendet, um Aussagen miteinander zu verknüpfen und komplexere Aussagen zu bilden.
\end{definition}

\begin{definition}[Zusammengesetzte Aussage]
Eine zusammengesetzte Aussage entsteht durch die Verknüpfung atomarer Aussagen mittels logischer Operatoren.
\end{definition}

\subsection{Notation zusammengesetzter Aussagen}

In der Aussagenlogik verwenden wir standardmäßig die \textbf{Infixnotation}, bei der die logischen Operatoren zwischen den atomaren Aussagen stehen. Um die Reihenfolge der Operationen in komplexen Ausdrücken eindeutig zu machen, können Klammern verwendet werden. In der reinen Infixnotation stehen die Operatoren jedoch einfach zwischen den Operanden ohne Klammern, wobei die Priorität der Operatoren die Reihenfolge der Auswertung bestimmt.

\begin{definition}[Infixnotation]
Die \textbf{Infixnotation} ist eine Notation, in der die Operatoren zwischen den atomaren Aussagen stehen. In einfachen Ausdrücken ist die Reihenfolge der Auswertung durch die Priorität der Operatoren festgelegt. Klammern werden verwendet, um die Reihenfolge explizit zu machen, wenn dies erforderlich ist, um Mehrdeutigkeit zu vermeiden.

\textbf{Beispiel ohne Klammern:} 
\[
p \land q \lor r
\]
Hier wird zuerst \(p \land q\) berechnet, da die Konjunktion eine höhere Priorität als die Disjunktion hat.

\textbf{Beispiel mit Klammern:} 
\[
(p \land q) \lor r
\]
Hier machen die Klammern die Reihenfolge explizit und verdeutlichen, dass \(p\) und \(q\) zuerst verknüpft werden, bevor das Ergebnis mit \(r\) durch \(\lor\) kombiniert wird.
\end{definition}

Die \textbf{reine Infixnotation ohne Klammern} funktioniert nur in Fällen, in denen die Priorität der Operatoren die Auswertung eindeutig bestimmt. Bei komplexeren Ausdrücken, bei denen die Reihenfolge nicht allein durch Prioritäten festgelegt werden kann, sind Klammern unerlässlich. Zum Beispiel kann der Ausdruck \((p \lor q) \land r\) in der reinen Infixnotation nicht korrekt dargestellt werden, da \(\land\) eine höhere Priorität hat als \(\lor\). Ohne Klammern würde er als \(p \lor (q \land r)\) interpretiert.

\subsection{Priorität der logischen Operatoren}

Auch ohne Klammern gibt es eine feste Reihenfolge, in der die logischen Operatoren ausgewertet werden. Diese Priorität bestimmt, welche Operatoren zuerst ausgewertet werden, wenn keine Klammern gesetzt sind. Die folgende Liste zeigt die Priorität der grundlegenden logischen Operatoren (von höchster zu niedrigster Priorität):

\begin{enumerate}
    \item \textbf{Negation} (\(\neg\)): Die Negation hat die höchste Priorität und wird vor allen anderen Operatoren ausgewertet. Zum Beispiel wird in \(\neg p \land q\) zuerst \(\neg p\) berechnet, dann \(p \land q\).
    \item \textbf{Konjunktion} (\(\land\)): Die Konjunktion hat eine höhere Priorität als die Disjunktion. In \(p \land q \lor r\) wird zuerst \(p \land q\) ausgewertet, dann \(q \lor r\).
    \item \textbf{Disjunktion} (\(\lor\)): Die Disjunktion hat eine geringere Priorität als die Konjunktion, wird jedoch vor der Implikation ausgewertet.
    \item \textbf{Implikation} (\(\rightarrow\)): Die Implikation wird erst nach den Konjunktionen und Disjunktionen ausgewertet.
\end{enumerate}

\textbf{Beispiele:}
\begin{example}
    \item In \(\neg (p \lor q)\) wird zuerst die Disjunktion \(p \lor q\) ausgewertet und dann die Negation darauf angewendet.
    \item In \(p \land (q \rightarrow r)\) wird zuerst \(q \rightarrow r\) ausgewertet und anschließend die Konjunktion mit \(p\) gebildet.
\end{example}

\subsection{Warum Klammern verwenden?}

Obwohl es feste Regeln für die Priorität der logischen Operatoren gibt, empfehlen wir die Verwendung von Klammern, um komplexe Ausdrücke klarer und lesbarer zu machen. Klammern helfen dabei, Missverständnisse zu vermeiden und sicherzustellen, dass der Ausdruck eindeutig interpretiert wird. Besonders bei langen oder verschachtelten Ausdrücken kann dies die Übersichtlichkeit erheblich verbessern.

\begin{remark}
In der reinen Infixnotation ohne Klammern kann es bei bestimmten Ausdrücken zu Mehrdeutigkeiten kommen, wenn die Priorität der Operatoren nicht ausreicht, um die Reihenfolge der Auswertung eindeutig festzulegen. Deshalb ist die Verwendung von Klammern in solchen Fällen unverzichtbar, um sicherzustellen, dass der Ausdruck korrekt interpretiert wird.
\end{remark}

\subsection{Weitere Notationen}

Neben der Infixnotation mit Klammern, die wir in diesem Skript als Standardnotation verwenden, gibt es in der Logik noch weitere Notationen, die zur Darstellung von logischen Ausdrücken verwendet werden. Diese Notationen unterscheiden sich darin, wie die Reihenfolge der Operationen dargestellt wird, und haben in verschiedenen Bereichen ihre Vorteile. Im Folgenden stellen wir zwei weitere gängige Notationen vor, die in der mathematischen Logik und der Informatik verwendet werden.

\begin{itemize}
    \item \textbf{Polnische Notation (Präfixnotation)}: In dieser Notation werden die Operatoren vor den Operanden geschrieben. Klammern sind hier überflüssig, da die Reihenfolge der Operationen durch die Position der Symbole eindeutig festgelegt wird.
    
    \textbf{Beispiel}: Anstelle von \((p \land q) \lor r\) wird \(\lor \land p q r\) geschrieben. In dieser Darstellung steht der Operator immer vor den Operanden, was die Klammerung unnötig macht.
    
    \item \textbf{Umgekehrte Polnische Notation (Postfixnotation)}: In dieser Notation, die häufig in der Informatik verwendet wird, stehen die Operatoren hinter den Operanden. Auch hier sind keine Klammern notwendig, da die Reihenfolge der Operationen durch die Position der Symbole eindeutig bestimmt wird.
    
    \textbf{Beispiel}: Anstelle von \((p \land q) \lor r\) wird \(\land p q \lor r\) geschrieben. Der Operator kommt nach den Operanden, was die Auswertung von links nach rechts ermöglicht.
\end{itemize}

\textbf{Vergleich der Notationen:} 
Die Wahl der Notation hängt vom Kontext und den Anforderungen der jeweiligen Anwendung ab. In der Logik und Mathematik ist die Infixnotation am gebräuchlichsten, da sie intuitiv und gut lesbar ist. Die polnische und die umgekehrte polnische Notation haben ihre Vorteile in der Informatik, insbesondere bei der Auswertung durch Computer, da sie keine Klammern benötigen und effizienter ausgewertet werden können.

In diesem Skript verwenden wir standardmäßig die \textbf{Infixnotation mit vollständiger Klammerung}, um die Struktur von Aussagen klar darzustellen.

\subsection{Wahrheitswerttabellen}

Ein zentrales Konzept der Aussagenlogik ist die Verwendung von sogenannten Wahrheitswerttabellen, um die Wahrheitsbedingungen einer zusammengesetzten Aussage zu analysieren. Eine Wahrheitswerttabelle legt für jede Kombination der Wahrheitswerte der beteiligten atomaren Aussagen den Wahrheitswert der zusammengesetzten Aussage fest. Auf diese Weise kann die Bedeutung komplexer Aussagen systematisch erfasst werden, ohne auf die inhaltliche Bedeutung der atomaren Aussagen einzugehen.

\subsection{Regeln der Aussagenlogik}

Jeder logische Operator hat bestimmte Regeln, wie er in Beweisen verwendet werden kann. Diese Regeln sind Teil des Kalküls des natürlichen Schließens, einem formalen System, das in der Aussagen- und Prädikatenlogik verwendet wird, um Beweise zu führen. Die spezifischen Regeln für jeden Operator werden in den nachstehenden Abschnitten detailliert beschrieben.

\section{Prädikatenlogik}
\subsection{Syntax der Prädikatenlogik}

Die Syntax der Prädikatenlogik definiert die Struktur und Formation von Aussagen oder Ausdrücken in der Prädikatenlogik. Im Gegensatz zur Aussagenlogik erlaubt die Prädikatenlogik, über Objekte in einer Domäne zu sprechen und Eigenschaften sowie Beziehungen zwischen diesen auszudrücken.

\begin{definition}[Domäne]
Die \textbf{Domäne} (auch \textbf{Grundmenge} genannt) ist die nicht-leere Menge von Objekten, über die in einer prädikatenlogischen Struktur gesprochen wird. Die Objekte der Domäne sind die möglichen Werte, die Variablen in prädikatenlogischen Ausdrücken annehmen können.
\end{definition}

\begin{definition}[Term]
Ein \textbf{Term} ist ein Ausdruck, der ein Objekt in der Domäne beschreibt. Terme sind die Grundbausteine prädikatenlogischer Ausdrücke und repräsentieren Objekte in der Domäne. Es gibt drei Hauptarten von Termen:

\begin{itemize}
    \item \textbf{Variablen}: Eine Variable steht für ein beliebiges Objekt in der Domäne. Beispiele für Variablen sind \(x, y, z\). Zum Beispiel beschreibt der Term \(x\) ein beliebiges Objekt in der Domäne.

    \item \textbf{Konstanten}: Eine Konstante bezeichnet ein bestimmtes, festes Objekt in der Domäne. Beispiele für Konstanten sind \(a, b, c\). Zum Beispiel repräsentiert \(a\) ein spezifisches Objekt in der Domäne, wie z.B. eine bestimmte Person oder Zahl.

    \item \textbf{Funktionssymbole}: Ein Funktionssymbol beschreibt eine Operation, die auf Termen ausgeführt wird und einen neuen Term ergibt. Ein Funktionssymbol wird auf eine feste Anzahl von Termen (den sogenannten Argumenten) angewendet. Beispiele für Funktionssymbole sind \(f, g, h\). Zum Beispiel könnte \(f(a)\) eine Funktion sein, die das Objekt \(a\) in ein anderes Objekt der Domäne abbildet, etwa \(f(a)\) als die Mutter von \(a\) in einer Domäne von Personen.
\end{itemize}

\textbf{Beispiel für Terme}:
\begin{itemize}
    \item Der Ausdruck \(x\) ist ein Term, der eine Variable darstellt.
    \item Der Ausdruck \(a\) ist ein Term, der eine Konstante darstellt.
    \item Der Ausdruck \(f(x)\) ist ein Term, der eine Funktion darstellt, die auf der Variable \(x\) basiert.
\end{itemize}

\end{definition}

\begin{definition}[Prädikat]
Ein \textbf{Prädikat} ist eine Funktion, die eine bestimmte Eigenschaft oder Beziehung zwischen Objekten der Domäne beschreibt. Prädikate nehmen eine feste Anzahl von Termen als Argumente und bilden daraus eine Aussage. Diese Aussagen können dann entweder wahr oder falsch sein, abhängig von der Interpretation der Prädikate und der Terme in der Domäne.

Prädikate werden häufig durch Großbuchstaben (wie \(P, Q, R\)) dargestellt. Es gibt eine Unterscheidung zwischen Aussagen mit Konstanten und Variablen:

\begin{itemize}
    \item Wenn ein Prädikat nur mit **Konstanten** als Argumenten verwendet wird, kann die Wahrheit der Aussage unmittelbar bestimmt werden. Beispiel: \(R(5, 3)\), wobei \(R(x, y)\) die Beziehung "größer als" beschreibt, ist wahr, da \(5 > 3\).
    
    \item Wenn ein Prädikat mit **Variablen** als Argumenten verwendet wird, hängt der Wahrheitswert von der Interpretation der Variablen ab. Das bedeutet, dass erst durch die Zuordnung von Objekten der Domäne zu den Variablen \(x\) und \(y\) festgelegt wird, ob die Aussage wahr oder falsch ist. Beispiel: Die Aussage \(R(x, y)\) (mit \(R(x, y)\) = "größer als") kann nicht als wahr oder falsch festgelegt werden, solange wir nicht wissen, welche Werte für \(x\) und \(y\) eingesetzt werden.
\end{itemize}

Prädikate können folgende Stelligkeit haben:
\begin{itemize}
\item 	\textbf{Einstellig (Unär)}: Ein einstelliges Prädikat beschreibt eine Eigenschaft eines einzelnen Objekts der Domäne. Zum Beispiel kann  ausdrücken, dass  eine bestimmte Eigenschaft besitzt, wie etwa \enquote{ist eine gerade Zahl}.
\item 	\textbf{Zweistellig (Binär)}: Ein zweistelliges Prädikat beschreibt eine Beziehung zwischen zwei Objekten der Domäne. Solche Prädikate werden auch als \textbf{binäre Relationen} bezeichnet, da sie eine Relation zwischen zwei Objekten herstellen. Zum Beispiel beschreibt  eine Beziehung zwischen  und , wie etwa \enquote{ist größer als}.

\item 	\textbf{Dreistellig (Ternär) oder höherer Stelligkeit}: Prädikate können auch drei oder mehr Argumente haben, um komplexere Beziehungen zwischen mehreren Objekten auszudrücken. Zum Beispiel beschreibt  eine dreistellige Relation zwischen  und , wie etwa \\enquote{liegt zwischen  und}.
\end{itemize}

\textbf{Beispiele für Prädikate}:
\begin{itemize}
    \item Das Prädikat \(P(x)\) könnte ausdrücken, dass \(x\) eine gerade Zahl ist. \(P(2)\) wäre wahr, da 2 eine gerade Zahl ist, aber \(P(3)\) wäre falsch.
    \item Das Prädikat \(R(x, y)\) könnte die Beziehung \enquote{ist größer als} ausdrücken. Zum Beispiel wäre \(R(5, 3)\) wahr, da \(5 > 3\), aber \(R(2, 4)\) wäre falsch, da \(2\) nicht größer als \(4\) ist.
    \item Ein ternäres Prädikat \(S(x, y, z)\) könnte ausdrücken: \(x\) \enquote{liegt zwischen} \(y\) und \(z\). Zum Beispiel wäre \(S(5, 3, 7)\) wahr, weil \(5\) zwischen \(3\) und \(7\) liegt.
    \item Ein Beispiel für ein Prädikat mit einer Funktionsanwendung wäre \(P(f(x))\), wobei \(f(x)\) eine Funktion ist, die das Doppelte von \(x\) zurückgibt. Wenn \(P(y)\) ausdrückt, dass \(y\) eine gerade Zahl ist, dann überprüft \(P(f(x))\), ob das Doppelte von \(x\) gerade ist.
    \item Ein weiteres Beispiel mit Konstanten: Das Prädikat \(Q(x)\) könnte ausdrücken, dass \(x\) größer als 10 ist. Wenn \(a = 7\), wäre \(Q(a)\) falsch, weil 7 nicht größer als 10 ist. Wenn \(b = 12\) ist, wäre \(Q(b)\) wahr.
\end{itemize}

Prädikate erlauben es uns, über Eigenschaften von Objekten in der Domäne und deren Beziehungen zu sprechen, was die Prädikatenlogik mächtiger und flexibler als die Aussagenlogik macht.
\end{definition}
\begin{remark}
In diesem Skript wird gelegentlich die Notation \(Px\) anstelle von \(P(x)\) verwendet, insbesondere wenn eindeutig ist, dass \(P\) ein Prädikat ist, das auf die Variable \(x\) angewendet wird. Ebenso wird nach dem Allquantor \(\forall\) manchmal die Klammer weggelassen, wenn klar erkennbar ist, dass das Prädikat zum Quantor gehört. Diese Vereinfachung dient der Übersichtlich-keit und wird nur dann verwendet, wenn keine Verwechslungsgefahr besteht.
\end{remark}

\begin{definition}[Gleichheit]
Das \textbf{Gleichheitszeichen} (\(=\)) ist ein spezielles binäres Prädikat in der Prädikatenlogik, das die Identität zweier Terme ausdrückt. Für zwei Terme \(s\) und \(t\) bedeutet \(s = t\), dass \(s\) und \(t\) auf dasselbe Objekt in der Domäne verweisen.

\textbf{Beispiele für Gleichheit}:
\begin{itemize}
    \item \(a = a\) ist immer wahr, da jede Konstante gleich sich selbst ist.
    \item Wenn \(f(x)\) eine Funktion ist, die \(x\) auf ein bestimmtes Objekt abbildet, dann \(f(a) = b\) bedeutet, dass das Ergebnis der Funktion \(f\) bei Eingabe \(a\) das Objekt \(b\) ist.
    \item In einer Domäne von Zahlen könnte \(x = y\) bedeuten, dass die Zahlen, die durch \(x\) und \(y\) repräsentiert werden, identisch sind.
\end{itemize}
\end{definition}


\begin{definition}[Quantoren]
Quantoren sind Symbole, die den Umfang einer Aussage in der Prädikatenlogik angeben, indem sie definieren, für welche Objekte in der Domäne die Aussage gilt. Es gibt zwei Haupttypen von Quantoren:

\begin{itemize}
    \item \textbf{Allquantor} (\(\forall\)): Der Allquantor drückt aus, dass eine Aussage für \textbf{alle} Objekte in der Domäne wahr ist. Beispiel: \(\forall x P(x)\) bedeutet, dass \(P(x)\) für jedes Objekt \(x\) in der Domäne gilt. Wenn \(P(x)\) bedeutet \enquote{\(x\) ist eine gerade Zahl}, würde \(\forall x P(x)\) bedeuten \enquote{Alle \(x\) in der Domäne sind gerade Zahlen}.
    
    \item \textbf{Existenzquantor} (\(\exists\)): Der Existenzquantor drückt aus, dass es \textbf{mindestens ein} Objekt in der Domäne gibt, für das die Aussage wahr ist. Beispiel: \(\exists x P(x)\) bedeutet, dass es mindestens ein Objekt \(x\) in der Domäne gibt, für das \(P(x)\) wahr ist. Wenn \(P(x)\) bedeutet \enquote{\(x\) ist eine gerade Zahl}, würde \(\exists x P(x)\) bedeuten \enquote{Es gibt mindestens ein \(x\) in der Domäne, das eine gerade Zahl ist}.
\end{itemize}
\end{definition}

\textbf{Beispiele für Quantoren}:
\begin{itemize}
    \item \(\forall x P(x)\): Diese Aussage besagt, dass die Eigenschaft \(P(x)\) für jedes \(x\) in der Domäne wahr ist. Beispiel: \(\forall x (x > 0)\) könnte ausdrücken, dass alle Objekte \(x\) in der Domäne größer als 0 sind.
    \item \(\exists x P(x)\): Diese Aussage besagt, dass es mindestens ein \(x\) in der Domäne gibt, für das \(P(x)\) wahr ist. Beispiel: \(\exists x (x > 10)\) könnte ausdrücken, dass es mindestens ein Objekt \(x\) gibt, das größer als 10 ist.
\end{itemize}

\begin{definition}[Gültigkeitsbereich eines Quantors]
Der \textbf{Gültigkeitsbereich eines Quantors} ist der Teil eines prädikatenlogischen Ausdrucks, auf den der Quantor wirkt. Wenn ein Quantor der Form \(\forall x\) oder \(\exists x\) eine Variable \(x\) einführt, dann gilt der Quantor für alle Vorkommen von \(x\) innerhalb einer bestimmten Teilformel des Ausdrucks, die seinen Gültigkeitsbereich darstellt. Dieser Bereich endet in der Regel an der nächsten Klammer oder am Ende des Ausdrucks.

\textbf{Beispiele:}
\begin{itemize}
    \item In der Aussage \(\forall x (P(x) \land Q(y))\) ist der Gültigkeitsbereich des Quantors \(\forall x\) die Teilformel \(P(x) \land Q(y)\). Innerhalb dieses Bereichs ist \(x\) gebunden, während \(y\) frei ist.
    \item In der Aussage \(\exists y (\forall x (R(x, y)) \rightarrow P(z))\) ist der Gültigkeitsbereich von \(\exists y\) die gesamte Aussage \(\forall x (R(x, y)) \rightarrow P(z)\). Innerhalb dieses Bereichs ist \(y\) gebunden, während \(z\) frei bleibt.
\end{itemize}
\end{definition}

\begin{definition}[Gebundene Variable]
Eine Variable in einem prädikatenlogischen Ausdruck ist \textbf{gebunden}, wenn sie im Gültigkeitsbereich eines Quantors steht, der diese Variable einführt. Das bedeutet, dass der Quantor die Werte der Variablen festlegt und somit beeinflusst, für welche Objekte in der Domäne die Aussage gilt.

\textbf{Beispiele:}
\begin{itemize}
    \item In der Aussage \(\forall x P(x)\) ist die Variable \(x\) gebunden, weil sie im Gültigkeitsbereich des Quantors \(\forall x\) steht.
    \item In der Aussage \(\exists y (Q(x) \land R(y))\) ist \(y\) gebunden durch \(\exists y\), während \(x\) eine freie Variable bleibt.
\end{itemize}
\end{definition}

\begin{definition}[Freie Variable]
Eine Variable in einem prädikatenlogischen Ausdruck ist \textbf{frei}, wenn sie nicht durch einen Quantor (\(\forall\) oder \(\exists\)) gebunden ist. Ein Ausdruck mit freien Variablen ist keine vollständige Aussage, da die Variablen nicht festgelegt sind.

\textbf{Beispiele:}
\begin{itemize}
    \item In dem Ausdruck \(P(x)\) ist die Variable \(x\) \textbf{frei}, da sie nicht durch einen Quantor gebunden ist. Solange \(x\) frei ist, kann der Wahrheitswert des Ausdrucks nicht bestimmt werden.
    
    \item In der Aussage \(\forall y (P(x) \land Q(y))\) ist die Variable \(x\) frei, während \(y\) durch den Allquantor \(\forall y\) gebunden ist.
\end{itemize}
\end{definition}

\begin{definition}[Symbol]
In der Prädikatenlogik ist ein \textbf{Symbol} ein grundlegendes Element der formalen Sprache, das verwendet wird, um Ausdrücke zu bilden. Symbole lassen sich in zwei Hauptkategorien unterteilen:

\begin{itemize}
    \item \textbf{Logische Symbole}: Diese umfassen die logischen Verbindungszeichen und Quantoren, die die Struktur der Aussagen bestimmen. Beispiele sind:
    \begin{itemize}
        \item \(\neg\) (Negation)
        \item \(\land\) (Konjunktion)
        \item \(\lor\) (Disjunktion)
        \item \(\rightarrow\) (Implikation)
        \item \(\leftrightarrow\) (Äquivalenz)
        \item \(\forall\) (Allquantor)
        \item \(\exists\) (Existenzquantor)
    \end{itemize}
    
    \item \textbf{Nicht-logische Symbole}: Diese beziehen sich auf die spezifische Domäne und umfassen:
    \begin{itemize}
        \item \textbf{Konstantensymbole}: Bezeichnen spezifische Objekte in der Domäne, z.B. \(a, b, c\).
        \item \textbf{Funktionssymbole}: Stellen Funktionen dar, die auf Objekte der Domäne angewendet werden, z.B. \(f, g, h\).
        \item \textbf{Prädikatssymbole}: Beschreiben Eigenschaften oder Relationen zwischen Objekten, z.B. \(P, Q, R\).
        \item \textbf{Variablen}: Stehen für beliebige Objekte in der Domäne, z.B. \(x, y, z\).
    \end{itemize}
\end{itemize}

Symbole sind die Bausteine, aus denen komplexere Ausdrücke und Aussagen der Prädikatenlogik aufgebaut werden.
\end{definition}

\begin{definition}[Atomare Aussage in der Prädikatenlogik]
Eine \textbf{atomare Aussage} in der Prädikatenlogik ist ein Ausdruck, der aus einem Prädikat besteht, das auf eine feste Anzahl von Termen angewendet wird, wobei alle darin vorkommenden Variablen entweder durch Konstanten ersetzt oder durch Quantoren gebunden sind. Nur unter dieser Bedingung kann die atomare Aussage einen festen Wahrheitswert (wahr oder falsch) annehmen.

\begin{example}
\[
P(a), \quad \forall x P(x), \quad \exists y R(y, a)
\]
Hier sind \(P(a)\) und \(\forall x P(x)\) atomare Aussagen, weil alle Variablen entweder durch Konstanten ersetzt oder durch Quantoren gebunden sind. Sie können daher als wahr oder falsch beurteilt werden.
\end{example}

\begin{hint}
Ein Ausdruck wie \(P(x)\), der eine freie Variable enthält, ist keine vollständige atomare Aussage, da er keinen festen Wahrheitswert hat, solange \(x\) nicht gebunden oder durch eine Konstante ersetzt ist.
\end{hint}
\end{definition}

\begin{definition}[Singuläre Aussage]
Eine \textbf{singuläre Aussage} ist eine atomare Aussage, bei der alle Variablen durch Konstanten ersetzt wurden. Diese Aussagen beziehen sich auf spezifische Objekte in der Domäne und können einen festen Wahrheitswert annehmen.

\begin{example}
    \(P(a)\), \quad \(R(a, b)\)
\end{example}

Hier sind \(a\) und \(b\) Konstanten, und die Aussagen \(P(a)\) und \(R(a, b)\) können wahr oder falsch sein, je nachdem, wie die Prädikate in der Domäne interpretiert werden.
\end{definition}

\begin{definition}[Universelle Aussage]
Eine \textbf{universelle Aussage} ist eine atomare Aussage, die durch einen \textbf{Allquantor} (\(\forall\)) eine Behauptung über alle Objekte in der Domäne trifft. Alle in der Aussage vorkommenden Variablen sind durch den Allquantor gebunden.

\begin{example}
    \(\forall x P(x)\), \quad \(\forall x \forall y R(x, y)\)
\end{example}

Hier gilt \(P(x)\) oder \(R(x, y)\) für alle Objekte in der Domäne. Der Wahrheitswert hängt davon ab, ob die Prädikate für jedes Objekt oder jedes Paar von Objekten in der Domäne zutreffen.
\end{definition}

\begin{definition}[Existenzielle Aussage]
Eine \textbf{existenzielle Aussage} ist eine atomare Aussage, die durch einen \textbf{Existenzquantor} (\(\exists\)) die Existenz von mindestens einem Objekt in der Domäne behauptet, für das die Aussage gilt. Alle in der Aussage vorkommenden Variablen sind durch den Existenzquantor gebunden.

\begin{example}
    \(\exists x P(x)\), \quad \(\exists x \exists y R(x, y)\)
\end{example}

Hier wird behauptet, dass es mindestens ein Objekt \(x\) oder ein Paar von Objekten \(x, y\) gibt, für das \(P(x)\) oder \(R(x, y)\) wahr ist.
\end{definition}

\begin{definition}[Zusammengesetzte Aussage in der Prädikatenlogik]
Eine \textbf{zusammengesetzte Aussage} in der Prädikatenlogik entsteht durch die Verknüpfung von atomaren Aussagen oder anderen zusammengesetzten Aussagen mittels logischer Operatoren. Diese Operatoren sind die Negation (\(\neg\)), die Konjunktion (\(\land\)), die Disjunktion (\(\lor\)), die Implikation (\(\rightarrow\)) und die Äquivalenz (\(\leftrightarrow\)).

\begin{example}
\[
\neg P(a), \quad P(a) \land Q(b), \quad \forall x (P(x) \rightarrow Q(x))
\]
Hier ist \(\neg P(a)\) eine zusammengesetzte Aussage, da sie eine Negation der atomaren Aussage \(P(a)\) ist. \(P(a) \land Q(b)\) ist eine Konjunktion zweier atomarer Aussagen, und \(\forall x (P(x) \rightarrow Q(x))\) ist eine universelle Aussage, die eine Implikation zwischen \(P(x)\) und \(Q(x)\) enthält.
\end{example}
\end{definition}

\subsection{Infixnotation in der Prädikatenlogik}

In der Prädikatenlogik verwenden wir, wie in der Aussagenlogik, standardmäßig die \textbf{Infixnotation}, bei der die logischen Operatoren zwischen den Aussagen und Prädikaten stehen. Zusätzlich zur Verknüpfung von Aussagen durch logische Operatoren verwenden wir Quantoren, um über Objekte in einer Domäne zu sprechen. 

\begin{definition}[Infixnotation in der Prädikatenlogik]
Die \textbf{Infixnotation} ist eine Notation, in der die logischen Operatoren zwischen den atomaren Aussagen und Prädikaten stehen. Die Struktur einer prädikatenlogischen Aussage wird durch die Verwendung von Quantoren, logischen Operatoren und eventuell Klammern eindeutig festgelegt. Klammern können verwendet werden, um die Reihenfolge der Operationen explizit zu machen.

\textbf{Beispiele für Infixnotation in der Prädikatenlogik:}
\begin{example}
    \[
    P(a) \land Q(b), \quad \forall x (P(x) \rightarrow Q(x)), \quad \exists y (R(x, y) \lor S(y, z))
    \]
    Im ersten Beispiel \(P(a) \land Q(b)\) stehen die Konjunktion (\(\land\)) zwischen zwei atomaren Aussagen \(P(a)\) und \(Q(b)\). Im zweiten Beispiel \(\forall x (P(x) \rightarrow Q(x))\) wird die universelle Quantifikation verwendet, um eine Aussage über alle \(x\) in der Domäne zu treffen. Im dritten Beispiel \(\exists y (R(x, y) \lor S(y, z))\) wird der Existenzquantor genutzt, um die Existenz eines Objekts \(y\) zu behaupten, für das die Disjunktion \(R(x, y) \lor S(y, z)\) wahr ist.
\end{example}

Die Reihenfolge der Operatoren richtet sich nach den gleichen Prioritäts-regeln wie in der Aussagenlogik, wobei Quantoren immer die höchste Priorität haben.
\end{definition}

\begin{remark}
Wie in der Aussagenlogik kann es in der Prädikatenlogik notwendig sein, Klammern zu verwenden, um Mehrdeutigkeiten zu vermeiden. Zum Beispiel wird der Ausdruck \(\forall x P(x) \land Q(x)\) anders interpretiert als \(\forall x (P(x) \land Q(x))\), da der Quantor \(\forall x\) im zweiten Fall auf die gesamte Konjunktion wirkt.
\end{remark}

\subsection{Infixnotation in der Prädikatenlogik}

Wie in der Aussagenlogik verwenden wir in der Prädikatenlogik standardmäßig die \textbf{Infixnotation}, bei der die logischen Operatoren zwischen den prädikaten-logischen Ausdrücken stehen.

\chapter{Beweistheorie}

\section{Einleitung zur Beweistheorie}
Beweistheorie ist ein zentraler Bestandteil der formalen Logik, der sich mit den Methoden und Prinzipien befasst, mit denen logische Aussagen bewiesen werden können. Durch die systematische Anwendung von Regeln können wir die Gültigkeit von Aussagen innerhalb eines logischen Systems feststellen. In diesem Kapitel werden wir das Kalkül des natürlichen Schließens als eine der grundlegenden Beweisformen kennenlernen und anwenden.

\section{Formale Sprache}

\begin{definition}[Formale Sprache]
Eine \textbf{formale Sprache} \(\mathcal{L}\) besteht aus zwei Hauptkomponenten:

\begin{enumerate}
    \item \textbf{Alphabet} \( \Sigma \): Eine endliche Menge von \textbf{Symbolen} (z.\,B.\ Buchstaben, Zeichen, Operatoren), die die Grundbausteine der Sprache bilden.
    \item \textbf{Syntaxregeln}: Regeln, die festlegen, wie die Symbole aus \( \Sigma \) kombiniert werden dürfen, um \textbf{wohlgeformte Formeln} (Wff) der Sprache zu bilden. Diese Regeln definieren die Struktur der zulässigen Ausdrücke in der Sprache.
\end{enumerate}

Eine \textbf{wohlgeformte Formel} ist eine endliche Zeichenkette aus Symbolen des Alphabets \( \Sigma \), die den Syntaxregeln der formalen Sprache \(\mathcal{L}\) entspricht.
\end{definition}

\begin{remark}
Die \textbf{Prädikatenlogik} und die \textbf{Aussagenlogik} sind zentrale Beispiele für formale Sprachen. Sie erfüllen die oben genannten Kriterien, indem sie jeweils ein spezifisches Alphabet und klare Syntaxregeln definieren, die die Konstruktion wohlgeformter Formeln ermöglichen.
\end{remark}

\subsection{Beispiele für Formale Sprachen}

Um die Definition einer formalen Sprache zu verdeutlichen, betrachten wir zwei grundlegende logische Systeme: die \textbf{Aussagenlogik} und die \textbf{Prädikatenlogik}.

\subsubsection{Aussagenlogik}

\begin{itemize}
    \item \textbf{Alphabet} \( \Sigma \): Besteht aus
    \begin{enumerate}
        \item \textbf{Aussagevariablen}: Symbole wie \( p, q, r, \ldots \), die atomare Aussagen repräsentieren.
        \item \textbf{Logische Operatoren}: 
        \begin{itemize}
            \item \( \neg \) (Nicht)
            \item \( \land \) (Und)
            \item \( \lor \) (Oder)
            \item \( \rightarrow \) (Impliziert)
            \item \( \leftrightarrow \) (Genau dann wenn)
        \end{itemize}
        \item \textbf{Klammern}: \( (, ) \), die zur Strukturierung von Ausdrücken verwendet werden.
    \end{enumerate}
    
    \item \textbf{Syntaxregeln}: Bestimmen, wie Aussagevariablen und Operatoren kombiniert werden dürfen, um \textbf{wohlgeformte Formeln} zu bilden. Beispiele hierfür sind:
    \begin{itemize}
        \item Eine Aussagevariable allein ist eine wohlgeformte Formel, z.\,B.\ \( p \).
        \item Wenn \( \phi \) eine wohlgeformte Formel ist, dann ist \( \neg \phi \) ebenfalls eine wohlgeformte Formel.
        \item Wenn \( \phi \) und \( \psi \) wohlgeformte Formeln sind, dann sind auch \( \phi \land \psi \), \( \phi \lor \psi \), \( \phi \rightarrow \psi \) und \( \phi \leftrightarrow \psi \) wohlgeformte Formeln.
        \item Klammern werden verwendet, um die Struktur und Priorität der Operatoren zu definieren, z.\,B.\ \( (p \land q) \rightarrow r \).
    \end{itemize}
\end{itemize}

\subsubsection{Prädikatenlogik}

\begin{itemize}
    \item \textbf{Alphabet} \( \Sigma \): Enthält die folgenden Komponenten:
    \begin{enumerate}
        \item \textbf{Individuenkonstanten}: Symbole, die spezifische Objekte in der Domäne repräsentieren, z.\,B.\ \( a, b, c, \ldots \).
        \item \textbf{Variablen}: Symbole wie \( x, y, z, \ldots \), die beliebige Elemente der Domäne darstellen können.
        \item \textbf{Funktionssymbole}: Symbole wie \( f, g, h, \ldots \), die Funktionen mit festgelegter Arity (Anzahl der Argumente) repräsentieren, z.\,B.\ \( f(x), g(x, y) \).
        \item \textbf{Prädikate}: Symbole wie \( P, Q, R, \ldots \), die Relationen oder Eigenschaften ausdrücken und ebenfalls eine festgelegte Arity haben, z.\,B.\ \( P(x), Q(x, y) \).
        \item \textbf{Quantoren}: \( \forall \) (Für alle) und \( \exists \) (Es existiert), die die Bereiche der Variablen einschränken.
        \item \textbf{Logische Operatoren}: Wie in der Aussagenlogik, einschließlich \( \neg \) (Nicht), \( \land \) (Und), \( \lor \) (Oder), \( \rightarrow \) (Impliziert), \( \leftrightarrow \) (Genau dann wenn).
        \item \textbf{Gleichheitszeichen}: \( = \), das die Gleichheit zweier Terme ausdrückt.
        \item \textbf{Klammern}: \( (, ) \), die zur Strukturierung von Ausdrücken verwendet werden.
    \end{enumerate}
    
    \item \textbf{Syntaxregeln}: Regeln, die festlegen, wie die Symbole aus \( \Sigma \) kombiniert werden dürfen, um \textbf{wohlgeformte Formeln} (Wff) der Sprache zu bilden. Diese Regeln definieren die Struktur der zulässigen Ausdrücke in der Sprache.
\end{itemize}

\section{Grundbegriffe: Kalkül und Beweis}

Bevor wir uns dem Kalkül des natürlichen Schließens zuwenden, ist es wichtig, die grundlegenden Begriffe \textit{Kalkül} und \textit{Beweis} präzise zu definieren.

\begin{definition}[Kalkül]
Ein \textbf{Kalkül} (auch \textit{formales Kalkül} oder \textit{formales System} genannt) ist ein strukturiertes System, das aus folgenden Komponenten besteht:

\begin{enumerate}
    \item \textbf{Formale Sprache} \(\mathcal{L}\): Bestehend aus einem Alphabet von Symbolen und Syntaxregeln, die bestimmen, wie wohlgeformte Formeln gebildet werden.
    \item \textbf{Axiome} und/oder \textbf{Annahmen} (Prämissen): Eine Menge von wohlgeformten Formeln in der Sprache \(\mathcal{L}\), die ohne Beweis als wahr angenommen werden.
    \item \textbf{Inferenzenregeln}: Regeln, die festlegen, wie aus gegebenen Formeln neue Formeln abgeleitet werden können.
\end{enumerate}

Die \textbf{Axiome} und \textbf{Annahmen} sind wohlgeformte Formeln, die als Ausgangspunkt dienen. Die \textbf{Inferenzenregeln} ermöglichen es, innerhalb des Kalküls systematisch neue Aussagen zu beweisen.
\end{definition}

\begin{definition}[Beweis]
Ein \textbf{Beweis} in einem Kalkül ist eine endliche Folge von Aussagen
\[
\phi_1, \phi_2, \ldots, \phi_n
\]
wobei jede Aussage \(\phi_i\) entweder eine \textbf{Annahme} (Prämisse), ein \textbf{Axiom} ist oder sich durch Anwendung einer Inferenzenregel auf vorhergehende Aussagen \(\phi_{j}\) (mit \(j < i\)) ableiten lässt. Die letzte Formel \(\phi_n\) des Beweises ist die zu beweisende Aussage.
\end{definition}

\begin{definition}[Argument]
Ein \textbf{Argument} in der Logik besteht aus einer Menge von Aussagen, den sogenannten \textbf{Prämissen}, und einer Aussage, der \textbf{Schlussfolgerung}. Die Prämissen sollen die Schlussfolgerung unterstützen oder begründen.

Formal kann ein Argument als geordnetes Paar \((\Gamma, \phi)\) definiert werden, wobei \(\Gamma\) eine Menge von Aussagen (die Prämissen) und \(\phi\) eine einzelne Aussage (die Schlussfolgerung) ist.

Ein Argument \((\Gamma, \phi)\) wird als \textbf{gültig} bezeichnet, wenn es eine Ableitung der Schlussfolgerung \(\phi\) aus den Prämissen in \(\Gamma\) gemäß den Regeln des Kalküls gibt.
\end{definition}

\section{Definition einer Theorie}

Eine \textbf{Theorie} in der Prädikatenlogik ist ein formales System, das aus einer Menge von Aussagen besteht, die als Axiome dienen. Aus diesen Axiomen können mittels logischer Schlussregeln weitere Aussagen bewiesen werden.

\begin{definition}[Theorie]
Eine \textbf{Theorie} \(\mathcal{T}\) in der Prädikatenlogik besteht aus einem \textbf{Axiomensystem} \(\Sigma\) und umfasst alle Aussagen \(\phi\) in der Sprache \(\mathcal{L}\), für die gilt:
\[
\Sigma \vdash \phi
\]
Das bedeutet, dass die Aussage \(\phi\) beweisbar ist aus den Axiomen \(\Sigma\) unter Verwendung der festgelegten Schlussregeln.
\end{definition}

\begin{remark}
Der wesentliche Unterschied zwischen einem \textbf{Kalkül} und einer \textbf{Theorie} besteht darin, dass der Kalkül das formale System mit seiner \textbf{Sprache}, \textbf{Axiomen} und \textbf{Inferenzenregeln} definiert, während eine Theorie eine spezifische Menge von Axiomen innerhalb dieses Kalküls ist und alle daraus ableitbaren Aussagen umfasst.
\end{remark}

\section{Eigenschaften einer Theorie}

Die Eigenschaften \textbf{Konsistenz} und \textbf{Vollständigkeit} sind zentrale Merkmale einer Theorie. Diese werden im Folgenden separat definiert und formal beschrieben.

\subsection{Konsistenz}

\begin{definition}[Konsistenz]
Eine \textbf{Theorie} \(\mathcal{T}\) mit Axiomensystem \(\Sigma\) ist \textbf{konsistent}, wenn es keine Aussage \(\phi\) gibt, für die sowohl \(\Sigma \vdash \phi\) als auch \(\Sigma \vdash \neg \phi\) gilt.
\end{definition}

\begin{remark}
Eine Theorie ist konsistent genau dann, wenn sie keine widersprüchlichen Aussagen enthält. Das bedeutet, es ist nicht möglich, sowohl eine Aussage als auch ihr Gegenteil aus den Axiomen abzuleiten.
\end{remark}

\subsection{Vollständigkeit}

\begin{definition}[Vollständigkeit]
Eine \textbf{Theorie} \(\mathcal{T}\) mit Axiomensystem \(\Sigma\) ist \textbf{vollständig}, wenn für jede Aussage \(\phi\) in der Sprache \(\mathcal{L}\) entweder \(\Sigma \vdash \phi\) oder \(\Sigma \vdash \neg \phi\) gilt.
\end{definition}

\begin{remark}
Eine Theorie ist vollständig genau dann, wenn für jede Aussage entweder die Aussage selbst oder ihre Negation aus den Axiomen ableitbar ist.
\end{remark}

\section{Arten von Definitionen}

\subsection{Explizite Definitionen}

\begin{definition}[Explizite Definition]
Sei \(\varphi(x_1,\dots,x_n)\) ein bereits bekannter, wohldefinierter Ausdruck (z.\,B.\ ein Prädikat oder Term) in einer formalen Sprache \(\mathcal{L}\) und \(Q(x_1,\dots,x_n)\) ein \emph{neues} Symbol, das in \(\mathcal{L}\) bisher nicht vorkam. Dann nennen wir
\[
\forall x_1,\dots,x_n \bigl( Q(x_1,\dots,x_n) \coloneqq \varphi(x_1,\dots,x_n) \bigr)
\]
eine \textbf{explizite Definition} von \(Q\). In dieser Definition heißt:
\begin{itemize}
    \item \(Q\) \textbf{Definiendum} (das neu eingeführte Symbol),
    \item \(\varphi\) \textbf{Definiens} (der bereits bekannte Ausdruck).
\end{itemize}

\noindent
\subsubsection{Wohldefiniertheit der expliziten Definition} 
\begin{enumerate}
    \item Alle in \(\varphi\) vorkommenden Symbole (Variablen, Konstanten, Funktions- oder Prädikatsymbole) sind bereits bekannt \emph{und} ihrerseits wohldefiniert.
    \item Das neue Symbol \(Q\) kommt in \(\varphi\) selbst nicht vor (keine Zirkularität).
    \item Für jede mögliche Belegung der Variablen \(x_1,\dots,x_n\) existiert genau ein \(y\), sodass \(\varphi(x_1,\dots,x_n) = y\). Formal ausgedrückt:
    \[
    \forall x_1,\dots,x_n \, \forall y \, \forall z \, \bigl( \varphi(x_1,\dots,x_n) = y \land \varphi(x_1,\dots,x_n) = z \rightarrow y = z \bigr).
    \]
    \item Für jede mögliche Belegung der Variablen \(x_1, \dots, x_n\) ist \(\varphi(x_1,\dots,x_n)\) widerspruchsfrei und erfüllt das Prinzip des ausgeschlossenen Dritten. Formal:
    \[
    \forall x_1,\dots,x_n \, \neg \bigl( \varphi(x_1,\dots,x_n) \land \neg \varphi(x_1,\dots,x_n) \bigr) \quad \text{(Konsistenz)},
    \]
    sowie
    \[
    \forall x_1,\dots,x_n \, \bigl( \varphi(x_1,\dots,x_n) \lor \neg \varphi(x_1,\dots,x_n) \bigr) \quad \text{(Vollständigkeit)}.
    \]
\end{enumerate}
\end{definition}

\subsection{Partielle Definitionen}
\begin{definition}[Partielle Definition]
Sei \(\varphi(x_1,\dots,x_n)\) ein bereits bekannter, wohldefinierter Ausdruck (z.\,B.\ ein Prädikat oder Term) in einer formalen Sprache \(\mathcal{L}\), \(C(x_1,\dots,x_n)\) eine Bedingung (ein weiteres Prädikat), und \(Q(x_1,\dots,x_n)\) ein \emph{neues} Symbol, das in \(\mathcal{L}\) bisher nicht vorkam. Dann nennen wir
\[
\forall x_1,\dots,x_n \bigl( C(x_1,\dots,x_n) \rightarrow ( Q(x_1,\dots,x_n) \coloneqq \varphi(x_1,\dots,x_n) ) \bigr)
\]
eine \textbf{partielle Definition} von \(Q\). In dieser Definition heißt:
\begin{itemize}
    \item \(Q\) \textbf{Definiendum} (das neu eingeführte Symbol),
    \item \(\varphi\) \textbf{Definiens} (der bereits bekannte Ausdruck),
    \item \(C\) \textbf{Bedingung} (das Prädikat, unter dem die Definition gilt).
\end{itemize}

\noindent
\subsubsection{Wohldefiniertheit der partiellen Definition} \emph{wohldefiniert}, wenn
\begin{enumerate}
    \item Alle in \(\varphi\) und \(C\) vorkommenden Symbole (Variablen, Konstanten, Funktions- oder Prädikatsymbole) sind bereits bekannt \emph{und} ihrerseits wohldefiniert.
    \item Das neue Symbol \(Q\) kommt weder in \(\varphi\) noch in \(C\) selbst vor (keine Zirkularität).
    \item Für jede mögliche Belegung der Variablen \(x_1,\dots,x_n\), für die \(C(x_1,\dots,x_n)\) wahr ist, existiert genau ein \(y\), sodass \(\varphi(x_1,\dots,x_n) = y\). Formal:
    \[
    \forall x_1,\dots,x_n \, \forall y \, \forall z \, \bigl( C(x_1,\dots,x_n) \rightarrow ( \varphi(x_1,\dots,x_n) = y \land \varphi(x_1,\dots,x_n) = z \rightarrow y = z ) \bigr).
    \]
    \item Für jede mögliche Belegung der Variablen \(x_1, \dots, x_n\), für die \(C(x_1,\dots,x_n)\) wahr ist, gilt, dass \(\varphi(x_1,\dots,x_n)\) widerspruchsfrei und vollständig ist. Formal:
    \[
    \forall x_1,\dots,x_n \, \bigl( C(x_1,\dots,x_n) \rightarrow \neg ( \varphi(x_1,\dots,x_n) \land \neg \varphi(x_1,\dots,x_n) ) \bigr) \quad \text{(Konsistenz)},
    \]
    sowie
    \[
    \forall x_1,\dots,x_n \, \bigl( C(x_1,\dots,x_n) \rightarrow ( \varphi(x_1,\dots,x_n) \lor \neg \varphi(x_1,\dots,x_n) ) \bigr) \quad \text{(Vollständigkeit)}.
    \]
\end{enumerate}
\end{definition}

\subsection{Rekursive Definition}

\begin{definition}[Rekursive Definition]
Sei \(\sigma(x_1, \ldots, x_n)\) ein \emph{neues} Symbol, das in einer formalen Sprache \(\mathcal{L}\) eingeführt werden soll. Eine \textbf{rekursive Definition} von \(\sigma\) besteht aus zwei Teilen:

\begin{itemize}
    \item \textbf{Basisfall:} Eine oder mehrere explizite Festlegungen von \(\sigma\) für spezielle Werte der Variablen, beispielsweise:
    \[
    \sigma(a_1, \ldots, a_k) \coloneqq b \quad \text{für feste Werte } a_1, \ldots, a_k \text{ und } b.
    \]
    Hierbei sind \(a_1, \ldots, a_k\) konkrete Werte aus der Domäne \(D\) und \(b\) ein bereits bekannter Term oder Wert in \(\mathcal{L}\).
    
    \item \textbf{Rekursionsvorschrift:} Eine oder mehrere Regeln, die \(\sigma(x_1, \ldots, x_n)\) für allgemeine Werte der Variablen definieren, indem sie auf zuvor definierte Werte von \(\sigma\) Bezug nehmen. Dies erfolgt in der Form:
    \[
    \sigma(x_1, \ldots, x_n) \coloneqq f\big(\sigma(y_1, \ldots, y_m), z_1, \ldots, z_p\big),
    \]
    wobei \(y_1, \ldots, y_m\) eine Teilmenge von \(x_1, \ldots, x_n\) ist, \(f\) eine bekannte Funktion in der Sprache \(\mathcal{L}\), und \(z_1, \ldots, z_p\) bekannte Terme in \(\mathcal{L}\) sind. Die Terme \(z_1, \ldots, z_p\) dürfen nicht rekursiv auf \(\sigma\) zurückgreifen, können jedoch von den Variablen \(x_1, \ldots, x_n\) abhängen oder andere bekannte Größen in \(\mathcal{L}\) enthalten.
\end{itemize}

In dieser Definition heißt:
\begin{itemize}
    \item \(\sigma\) \textbf{Definiendum} (das neu eingeführte Symbol),
    \item Die Basisfälle und Rekursionsvorschriften zusammen bilden das \textbf{Definiens}.
\end{itemize}

\noindent
\subsubsection{Wohldefiniertheit der rekursiven Definition}
\begin{enumerate}
    \item \textbf{Terminierung der Rekursion:} Für jede mögliche Belegung der Variablen \(x_1, \ldots, x_n\) wird die Rekursion nach endlich vielen Schritten beendet. Formal:
    \[
    \forall x_1, \ldots, x_n \, \exists k \in \mathbb{N} \, \bigl( \text{\(\sigma(x_1, \ldots, x_n)\) wird nach \(k\) Schritten berechnet} \bigr).
    \]
    \item \textbf{Eindeutigkeit:} Für jede Belegung der Variablen \(x_1, \ldots, x_n\) gibt es genau einen Wert \(y\), sodass die Rekursionsvorschrift für \(\sigma(x_1, \ldots, x_n)\) erfüllt ist. Formal:
    \[
    \forall x_1, \ldots, x_n \, \forall y \, \forall z \, \bigl( \sigma(x_1, \ldots, x_n) = y \land \sigma(x_1, \ldots, x_n) = z \rightarrow y = z \bigr).
    \]
    \item \textbf{Konsistenz der Rekursionsvorschrift:} Für jede mögliche Belegung der Variablen \(x_1, \ldots, x_n\) erzeugt die Rekursionsvorschrift keinen Widerspruch. Formal:
    \[
    \forall x_1, \ldots, x_n \, \neg \bigl( \sigma(x_1, \ldots, x_n) \land \neg \sigma(x_1, \ldots, x_n) \bigr).
    \]
    \item \textbf{Vollständigkeit der Rekursionsvorschrift:} Für jede mögliche Belegung der Variablen \(x_1, \ldots, x_n\) existiert ein definierter Wert. Formal:
    \[
    \forall x_1, \ldots, x_n \, \bigl( \sigma(x_1, \ldots, x_n) \lor \neg \sigma(x_1, \ldots, x_n) \bigr).
    \]
    \item \textbf{Monotonie der Rekursion:} Die Rekursionsvorschrift bezieht sich nur auf kleinere oder gleichwertige Argumente, die gemäß einer festen Ordnungsrelation \(R\) definiert sind. Formal:
    \[
    \forall x_1, \ldots, x_n \, \forall y_1, \ldots, y_m \, \bigl( R((y_1, \ldots, y_m), (x_1, \ldots, x_n)) \bigr),
    \]
    wobei \(R\) die Terminierung sicherstellt.
\end{enumerate}
\end{definition}


\subsection{Implizite Definition}

Die \textbf{implizite Definition} ist eine Methode, neue Symbole oder Konzepte in einer Theorie einzuführen, indem ihre Eigenschaften durch Axiome festgelegt werden, anstatt sie direkt zu definieren.

\begin{definition}[Implizite Definition]
Eine \textbf{implizite Definition} eines neuen Symbols \(\sigma\) in einer Theorie \(\mathcal{T}\) besteht aus einer Menge von \textbf{Aussagen} \(\Phi(\sigma)\), die die Eigenschaften von \(\sigma\) festlegen.
In formalen Kalkülen dürfen diese Axiome durch \textbf{äquivalente Schlussregeln} ersetzt werden, die dieselben Eigenschaften syntaktisch ausdrücken, um die Anwendbarkeit zu erleichtern.

\end{definition}

\subsubsection{Wohldefiniertheit einer Impliziten Definition}

\begin{definition}[Wohldefiniertheit einer impliziten Definition]
Eine implizite Definition \(\Phi(\sigma)\) ist \emph{wohldefiniert}, wenn sie die folgenden Eigenschaften erfüllt:
\begin{enumerate}
    \item \textbf{Konsistenz:} Die Menge der Aussagen \(\Phi(\sigma)\) darf keine widersprüchlichen Bedingungen enthalten. Formal:
    \[
    \neg \exists \phi, \psi \in \Phi(\sigma) \, \bigl( \phi \land \neg \phi \bigr).
    \]

    \item \textbf{Existenz:} Es muss mindestens eine Interpretation von \(\sigma\) geben, die alle Aussagen in \(\Phi(\sigma)\) erfüllt. Formal:
    \[
    \exists \sigma' \, \forall \phi \in \Phi(\sigma) \, \bigl( \phi[\sigma' / \sigma] \bigr),
    \]
    wobei \(\phi[\sigma' / \sigma]\) die Aussage \(\phi\) mit der Interpretation \(\sigma'\) für das Symbol \(\sigma\) bezeichnet.

    \item \textbf{Eindeutigkeit:} Es darf höchstens eine Interpretation von \(\sigma\) geben, die alle Aussagen in \(\Phi(\sigma)\) erfüllt. Formal:
    \[
    \forall \sigma', \sigma'' \, \bigl( (\forall \phi \in \Phi(\sigma) \, \phi[\sigma' / \sigma]) \land (\forall \phi \in \Phi(\sigma) \, \phi[\sigma'' / \sigma]) \rightarrow \sigma' = \sigma'' \bigr).
    \]
\end{enumerate}
\end{definition}

\subsection{Das Iota-Symbol}

In manchen Fällen möchten wir einen Begriff oder ein Symbol definieren, der oder das eindeutig durch eine bestimmte Eigenschaft charakterisiert ist. Hier kommt das \textbf{Iota-Symbol} \( \iota \) zum Einsatz, das in der formalen Logik verwendet wird, um auf ein eindeutig bestimmtes Objekt Bezug zu nehmen.

\begin{definition}[Iota-Symbol (\( \iota \))]
Sei \( P(x) \) eine Aussage oder Eigenschaft über \( x \). Falls es genau ein \( x \) gibt, für das \( P(x) \) gilt, schreiben wir:
\[
\iota x \, P(x)
\]
und meinen damit \enquote{das eindeutige \( x \), für das \( P(x) \) gilt}.

Formal bedeutet dies:
\[
\exists! x \, P(x) \quad \text{und} \quad \iota x \, P(x) \text{ ist dieses eindeutige } x.
\]
Dabei steht \( \exists! x \, P(x) \) für \enquote{Es existiert genau ein \( x \), für das \( P(x) \) gilt} und ist definiert als:
\[
\exists x \, P(x) \land \forall y \, (P(y) \rightarrow y = x).
\]
\end{definition}

\paragraph{Definition einer neuen Konstante mittels \(\iota\)}

\begin{definition}[Iota-Definition]
Sei \(P(x)\) eine bereits bekannte und wohldefinierte Eigenschaft in einer formalen Sprache \(\mathcal{L}\). Sei \(\sigma\) ein \emph{neues} Symbol (z.\,B.\ eine neue Konstante), das in \(\mathcal{L}\) bisher nicht vorkam. Dann definieren wir
\[
\sigma \coloneqq \iota x\,P(x)
\]
\emph{genau dann}, wenn \(\exists!\,x\,P(x)\) (es existiert \emph{genau ein} \(x\), für das \(P(x)\) gilt). 

\noindent
In dieser Definition heißt:
\begin{itemize}
    \item \(\sigma\) \textbf{Definiendum} (das neu eingeführte Symbol),
    \item \(\iota x\,P(x)\) \textbf{Definiens} (der bereits bekannte Ausdruck mit dem Iota-Operator).
\end{itemize}

\noindent
\textbf{Wohldefiniertheit der Iota-Definition} setzt voraus:
\begin{enumerate}
    \item \(\exists!\,x\,P(x)\), d.\,h.\ es existiert \emph{genau ein} \(x\), für das \(P(x)\) gilt.
    \item Das neue Symbol \(\sigma\) kommt in \(\iota x\,P(x)\) selbst nicht vor (keine Zirkularität).
\end{enumerate}
\end{definition}


\section{Grundlegende Beweisprinzipien}

In diesem Abschnitt werden weiterführende Konzepte vorgestellt, die im Rahmen von Beweisen und der Entwicklung formaler Theorien eine zentrale Rolle spielen.


\begin{definition}[Theorem]
Ein \textbf{Theorem} ist eine Aussage, die innerhalb eines Kalküls unter Verwendung der definierten Inferenzenregeln und gegebenen Annahmen bewiesen wurde. Theoreme sind zentrale Bestandteile mathematischer Theorien und dienen als Bausteine für weiterführende Beweise.
\end{definition}

\begin{definition}[Lemma]
Ein \textbf{Lemma} ist ein unterstützender Satz, der im Verlauf des Beweisens eines Theorems verwendet wird. Lemmata dienen dazu, komplexere Beweise in überschaubare Schritte zu gliedern.
\end{definition}

\begin{definition}[Korollar]
Ein \textbf{Korollar} ist eine Aussage, die sich unmittelbar aus einem bereits bewiesenen Theorem oder Lemma ableiten lässt. Korollare sind oft direkte Konsequenzen der zuvor bewiesenen Aussagen.
\end{definition}

\begin{definition}[Hilfsdefinition]
Eine \textbf{Hilfsdefinition} ist eine unterstützende Definition, die im Verlauf der Entwicklung einer formalen Theorie eingeführt wird, um eine komplexere Hauptdefinition vorzubereiten oder zu vereinfachen. Hilfsdefinitionen können als Zwischenstufen betrachtet werden, die es ermöglichen, komplexe Begriffe in mehreren Schritten aufzubauen.
\end{definition}

\begin{remark}
Hilfsdefinitionen dienen dazu, Definitionen modular und übersichtlich zu gestalten. Sie sind im Allgemeinen lokal auf den Kontext der Hauptdefinition beschränkt.
\end{remark}

\begin{definition}[Temporäre Definition]
Eine \textbf{temporäre Definition} ist eine Definition, die ausschließlich innerhalb eines Beweises eingeführt wird, um dessen Struktur zu vereinfachen und zu verdeutlichen. Diese Definition hat keine Bedeutung außerhalb des Beweises und wird nach Abschluss des Beweises nicht weiter verwendet.
\end{definition}

\begin{remark}
Temporäre Definitionen sind nützlich, um komplexe Beweisschritte zu modularisieren und zu vereinfachen. Sie werden nur im Rahmen eines einzelnen Beweises verwendet und danach verworfen.
\end{remark}

\section{Kalkül des natürlichen Schließens}

\subsection{Einführung}
Das Kalkül des natürlichen Schließens ist ein formales System, das in der Aussagen- und Prädikatenlogik verwendet wird, um Beweise zu führen. Es wurde entwickelt, um eine Methode zur Verfügung zu stellen, die der Art und Weise, wie Menschen intuitiv logische Schlussfolgerungen ziehen, so nahe wie möglich kommt.

In Bezug auf die oben definierte \textbf{Definition des Kalküls} stellt das natürliche Schließen ein spezifisches Kalkül dar, das \textbf{keine Axiome} verwendet, sondern sich vollständig auf eine festgelegte Menge von \textbf{Inferenzenregeln} stützt. Diese Inferenzenregeln sind die fundamentalen Bausteine des Kalküls und bestimmen, wie aus gegebenen Formeln neue Formeln abgeleitet werden können. Sie definieren die zulässigen Schritte innerhalb des Systems und gewährleisten die logische Konsistenz und Korrektheit der abgeleiteten Aussagen.

Ein \textbf{Beweis} im Kalkül des natürlichen Schließens ist somit eine endliche Folge von Formeln, bei der jede Formel entweder eine \textbf{Annahme} (Prämisse) ist oder sich durch Anwendung einer der Inferenzenregeln auf zuvor eingeführte Formeln ableiten lässt, gemäß der \textbf{Definition des Beweises}. Diese strukturierte Vorgehensweise stellt sicher, dass jeder Schritt des Beweises logisch korrekt ist und auf den festgelegten Regeln des Kalküls basiert.

\subsection{Beweise im Kalkül des natürlichen Schließens}

Im Kalkül des natürlichen Schließens werden Beweise oft in einer tabellarischen Form, der sogenannten \textbf{Beweistabelle}, dargestellt. Jede Zeile einer Beweistabelle stellt einen Ableitungsschritt dar, der eine logische Aussage zusammen mit einer Regel und den Referenzen auf die Zeilen enthält, auf die die Regel angewendet wird.

\subsubsection{Allgemeine Form einer Beweistabelle}

Die allgemeine Form einer Zeile in einer Beweistabelle ist:

\[
\begin{array}{llll}
	i_1, i_2, \ldots, i_k & (n) & P & \text{Regelname} \, j_1, j_2, \ldots, j_l \\
\end{array}
\]

Hier bedeutet:
\begin{itemize}
    \item \(i_1, i_2, \ldots, i_k\): Die Indizes der Annahmen, von denen die abgeleitete Aussage \(P\) abhängt.
    \item \((n)\): Die Zeilennummer, die den aktuellen Ableitungsschritt angibt.
    \item \(P\): Die abgeleitete Aussage in dieser Zeile.
    \item \textit{Regelname}: Der Name der angewendeten Inferenzregel oder das referenzierte Theorem/Definition.
    \item \(j_1, j_2, \ldots, j_l\): Die Zeilennummern der Aussagen, auf die die Regel angewendet wird.
\end{itemize}

\subsubsection{Beweistabelle mit Theoremen und Definitionen}

Um die Übersichtlichkeit zu erhöhen, können Theoreme und Definitionen als spezielle Inferenzregeln oder als referenzierte Schritte behandelt werden. Dies kann durch klare Kennzeichnung innerhalb der Beweistabelle erfolgen.

\begin{definition}[Beweistabelle]
Eine \textbf{Beweistabelle} ist eine endliche Sequenz von Zeilen \( \{Z_1, Z_2, \ldots, Z_n\} \), wobei jede Zeile \( Z_m \) (\(1 \leq m \leq n\)) durch die folgenden Bedingungen definiert ist:

\begin{enumerate}
    \item \textbf{Basisfall:} Die ersten Zeilen \( Z_1, Z_2, \ldots, Z_k \) entsprechen den Annahmen oder Axiomen des Kalküls.
    \item \textbf{Ableitungsschritte:} Für jede Zeile \( Z_m \) (\(m > k\)) gilt, dass \( Z_m \) durch Anwendung einer Inferenzregel, eines Theorems oder einer Definition auf eine oder mehrere vorhergehende Zeilen \( Z_{j_1}, Z_{j_2}, \ldots, Z_{j_l} \) (\(j_i < m\)) abgeleitet wird.
\end{enumerate}

Jede Zeile \( Z_m \) besteht aus:
\[
\begin{array}{llll}
	i_1, i_2, \ldots, i_k & (m) & P & \text{Regelname / Theorem / Definition} \, j_1, j_2, \ldots, j_l \\
\end{array}
\]
\end{definition}

\subsubsection{Beispiel für eine Beweistabelle}
Um das Prinzip zu veranschaulichen, betrachten wir einen einfachen Beweis für ein Argument der Aussagenlogik:

\[
P \rightarrow Q, P \vdash Q
\]

Die Beweistabelle für dieses Argument sieht folgendermaßen aus:

\[
\begin{array}{llll}
	1 & (1) & P \rightarrow Q & \rA \\
	2 & (2) & P & \rA \\
	1,2 & (3) & Q & \rRE{1,2} \\
\end{array}
\]

In dieser Tabelle:
\begin{itemize}
    \item Zeile 1: Die Annahme \(P \rightarrow Q\) wird eingeführt (\(\rA\), Annahme).
    \item Zeile 2: Die Annahme \(P\) wird eingeführt ( \(\rA \), Annahme).
    \item Zeile 3: Die Regel \(\rightarrow\)-Elimination (\(\rRE{}\)) wird angewendet, um \(Q\) aus den Zeilen 1 und 2 abzuleiten.
\end{itemize}



\subsection{Regeln für die Annahmeneinführung}
\label{rule:A}
Die Annahmeneinführung ist eine grundlegende Regel im Kalkül des natürlichen Schließens. Sie erlaubt es uns, eine Aussage als Annahme in den Beweis einzuführen. Diese Annahme kann dann in den folgenden Zeilen des Beweises verwendet werden, um weitere Aussagen abzuleiten.

Die Regel der Annahmeneinführung wird oft durch das Symbol \(A\) dargestellt. Es gibt keine spezifischen Bedingungen für die Anwendung dieser Regel, da sie einfach eine Aussage als Annahme in den Beweis einführt.

\[
\begin{array}{llll}	
i & (i) & P & \rA \\
\end{array}
\]

Hierbei repräsentiert \(i\) die Zeilennummer der Annahme, und \(P\) ist die Aussage, die als Annahme eingeführt wird.

Es ist wichtig zu beachten, dass Annahmen im Kalkül des natürlichen Schließens nicht unbedingt wahr sein müssen. Sie dienen lediglich als Ausgangspunkt für den Beweis. Wenn wir aus einer Annahme einen Widerspruch ableiten können, dann können wir schließen, dass die ursprüngliche Annahme falsch sein muss. Dies ist die Grundlage für den Beweis durch Widerspruch, eine wichtige Methode im Kalkül des natürlichen Schließens.

In einigen Fällen können wir eine Annahme auch wieder aufheben, wenn wir ihre Konsequenzen untersucht haben. Dies ist zum Beispiel die Grundlage für die Regel der Implikationseinführung, die es uns erlaubt, aus der Annahme \(P\) und der daraus abgeleiteten Aussage \(Q\) die Implikation \(P \rightarrow Q\) zu erzeugen. In diesem Fall sagen wir, dass die Annahme \(P\) für den Beweis der Implikation \(P \rightarrow Q\) entfernt oder aufgehoben wurde.


\subsection{Regeln für die Konjunktion}

\subsubsection{Einführung der Konjunktion}
\label{rule:AI}
Die Konjunktion, oft dargestellt durch das Symbol \(\land\), ist ein logischer Operator, der \enquote{und} bedeutet. Im Kalkül des natürlichen Schließens gibt es zwei grundlegende Regeln für die Konjunktion: die Einführungs- und die Eliminierungsregel.

\begin{definition}[Einführungsregel der Konjunktion]
Seien die Zeilen \( m \) und \( n \) einer Beweistabelle wie folgt definiert:

\textbf{Zeile \( m \):}
\[
\begin{array}{llll}
    i & (m) & P & \dots \\
\end{array}
\]

\textbf{Zeile \( n \):}
\[
\begin{array}{llll}
    j & (n) & Q & \dots \\
\end{array}
\]

Die Einführungsregel \(\land I\) erlaubt die Ableitung von \( P \land Q \) in einer neuen Zeile \( k \). Dies wird wie folgt notiert:
\[
\begin{array}{llll}
    i, j & (k) & P \land Q & \rAI{m,n} \\
\end{array}
\]

Hierbei gilt:
\begin{itemize}
    \item \(m, n < k\), wobei \(m\) die Zeile ist, in der \(P\) abgeleitet wurde und \(n\) ist die Zeile in der \(Q\) abgeleitet wurde.
    \item Die Indizes \( i \) und \( j \) geben die Listen der Annahmen an, die für die Ableitungen in den Zeilen \( m \) und \( n \) verwendet wurden. 
\end{itemize}

\end{definition}

\subsubsection{Eliminierung der Konjunktion}
\label{rule:AE1}\label{rule:AE2}
Die Konjunktion \(\land\) erlaubt es, aus einer zusammengesetzten Aussage die einzelnen Bestandteile zu isolieren. Im Kalkül des natürlichen Schließens werden hierfür zwei Eliminierungsregeln definiert: \(\land E1\) und \(\land E2\).

\begin{definition}[Eliminierungsregeln der Konjunktion]
Sei die Zeile \(m\) einer Beweistabelle wie folgt definiert:

\textbf{Zeile \(m\):}
\[
\begin{array}{llll}
    i & (m) & P \land Q & \dots \\
\end{array}
\]

Die Eliminierungsregeln \(\land E1\) und \(\land E2\) erlauben es, aus \(P \land Q\) jeweils \(P\) oder \(Q\) abzuleiten. Dies wird wie folgt notiert:

\textbf{Eliminierung \(\land E1\):}
\[
\begin{array}{llll}
    i & (k) & P & \rAEa{m} \\
\end{array}
\]

\textbf{Eliminierung \(\land E2\):}
\[
\begin{array}{llll}
    i & (n) & Q & \rAEb{m} \\
\end{array}
\]

Hierbei gilt:
\begin{itemize}
    \item \(m < k, n\), wobei \(m\) die Zeile ist, in der \(P \land Q\) abgeleitet wurde.
    \item Der Index \(i\) repräsentiert die Liste der Annahmen, die für die Ableitung von \(P \land Q\) verwendet wurden. Diese Liste bleibt für die Ableitungen von \(P\) und \(Q\) unverändert.
\end{itemize}

Die Eliminierungsregeln ermöglichen es, eine Konjunktion in ihre Bestandteile zu zerlegen und jeweils nur einen der beiden Teile weiter zu verwenden.
\end{definition}

\subsection{Regeln für die Disjunktion}

\subsubsection{Einführung der Disjunktion}
\label{rule:OI1}\label{rule:OI2}
Die Disjunktion, oft dargestellt durch das Symbol \(\lor\), ist ein logischer Operator, der \enquote{oder} bedeutet. Im Kalkül des natürlichen Schließens gibt es zwei Einführungsregeln für die Disjunktion: \(\lor I1\) und \(\lor I2\).

\begin{definition}[Einführungsregeln der Disjunktion]
Sei die Zeile \(m\) einer Beweistabelle wie folgt definiert:

\textbf{Einführung \(\lor I1\):}
\[
\begin{array}{l l l l}
    i & (m) & P & \dots \\
\end{array}
\]

Die Regel \(\lor I1\) erlaubt die Ableitung von \(P \lor Q\) in einer neuen Zeile \(k\). Dies wird wie folgt notiert:
\[
\begin{array}{l l l l}
    i & (k) & P \lor Q & \rOIa{m} \\
\end{array}
\]

\textbf{Einführung \(\lor I2\):}
\[
\begin{array}{l l l l}
    i & (m) & Q & \dots \\
\end{array}
\]

Die Regel \(\lor I2\) erlaubt die Ableitung von \(P \lor Q\) in einer neuen Zeile \(k\). Dies wird wie folgt notiert:
\[
\begin{array}{l l l l}
    i & (k) & P \lor Q & \rOIb{m} \\
\end{array}
\]

Hierbei gilt:
\begin{itemize}
    \item \(m < k\), wobei \(m\) die Zeile ist, in der \(P\) bzw. \(Q\) abgeleitet wurde.
    \item Der Index \(i\) gibt die Liste der Annahmen an, die für die Ableitung in Zeile \(m\) verwendet wurden.
\end{itemize}

\end{definition}

\subsubsection{Eliminierung der Disjunktion}
\label{rule:OE}
Die Disjunktion \(\lor\) erlaubt es, aus einer zusammengesetzten Aussage mit mehreren Alternativen auf eine gemeinsame Konsequenz zu schließen. Im Kalkül des natürlichen Schließens wird hierfür die Eliminierungsregel \(\lor E\) definiert.

\begin{definition}[Eliminierungsregel der Disjunktion]
Sei die Zeile \(m\) einer Beweistabelle wie folgt definiert:

\textbf{Zeile \(m\):}
\[
\begin{array}{l l l l}
    i & (m) & P \lor Q & \dots \\
\end{array}
\]

Die Eliminierung der Disjunktion erfolgt durch eine Fallunterscheidung mit zwei separaten Unterbeweistabellen, die wie folgt aufgebaut sind:

\paragraph{Fall 1: \(P\) führt zu \(R\)}
\[
\begin{array}{l l l l}
    n_1 & (n_1) & P & \rA \\
    j,n_1 & (k_1) & R & \dots \\
\end{array}
\]

\paragraph{Fall 2: \(Q\) führt zu \(R\)}
\[
\begin{array}{l l l l}
    n_2 & (n_2) & Q & \rA \\
    k,n_2 & (k_2) & R & \dots \\
\end{array}
\]

\textbf{Ableitung von \(R\):}
Nachdem beide Fälle betrachtet wurden, erlaubt die Regel \(\lor E\) die Ableitung von \(R\) in einer neuen Zeile \(k\):
\[
\begin{array}{l l l l}
    i,j,k & (k) & R & \rOE{m,n_1,k_1,n_2,k_2} \\
\end{array}
\]

Hierbei gilt:
\begin{itemize}
    \item \(m < n_1, n_1 < k_1, k_1 < k\).
    \item \(m < n_2, n_2 < k_2, k_2 < k\).
    \item Die Indizes \(i, j, k\) geben die Listen der Annahmen an, die für die Ableitungen in den Zeilen \(m, k_1, k_2\) verwendet wurden.
    \item Weder die Liste \(j\) der Annahmen der Zeile \(k_1\) die Annahme noch die Liste \(k\) der Annahmen der Zeile \(k_2\) enthalten die Annahmen \(n_1\) bzw. \(n_2\).
\end{itemize}

Die Eliminierungsregel ermöglicht es, eine Disjunktion aufzulösen, indem beide Fälle separat betrachtet und jeweils auf die gemeinsame Konsequenz \(R\) geschlossen wird.
\end{definition}

\begin{lemma}[Beispiel - Vertauschung der Disjunktion]
\( P \lor Q \vdash Q \lor P \)
\end{lemma}

\textbf{Erläuterung:} Dieses Theorem zeigt, dass die Reihenfolge in einer Disjunktion vertauscht werden kann. Der Beweis erfolgt durch Anwendung der Regel \(\lor E\) mit einer Fallunterscheidung. Hierzu wird jeweils der Fall \(P\) und der Fall \(Q\) separat betrachtet und in beiden Fällen gezeigt, dass \(Q \lor P\) gilt.

\begin{proof}
\[
\begin{array}{l l l l}
    1 & (1) & P \lor Q & \rA \\
    \multicolumn{4}{l}{\quad \textbf{Fall 1: \(P\vdash Q\lor P\) } } \\
    \quad 2 & (2) & P & \rA \\
    \quad 2 & (3) & Q \lor P & \rOIb{2} \\
    \multicolumn{4}{l}{\quad \textbf{Fall 2: \(Q\vdash Q\lor P\) }} \\
    \quad 4 & (4) & Q & \rA \\
    \quad 4 & (5) & Q \lor P & \rOIa{4} \\
    1 & (6) & Q \lor P & \rOE{1,2,3,4,5} \\
\end{array}
\]
\end{proof}

\subsection{Regeln für die Implikation}

Die Implikation, oft dargestellt durch das Symbol \(\rightarrow\), ist ein logischer Operator, der \enquote{impliziert} oder \enquote{führt zu} bedeutet. Im Kalkül des natürlichen Schließens gibt es zwei grundlegende Regeln für die Implikation: die Einführungs- und die Eliminierungsregel.

\subsubsection{Einführung der Implikation}
\label{rule:RI}

\begin{definition}[Einführungsregel der Implikation]
Seien die Zeilen \( m \) und \( n \) einer Beweistabelle wie folgt definiert:

\textbf{Zeile \( m \):}
\[
\begin{array}{llll}
    m & (m) & P & \rA \\
\end{array}
\]

\textbf{Zeile \( n \):}
\[
\begin{array}{llll}
    m,i & (n) & Q & \dots \\
\end{array}
\]

Die Einführungsregel \(\rightarrow I\) erlaubt die Ableitung von \( P \rightarrow Q \) in einer neuen Zeile \( k \). Dies wird wie folgt notiert:
\[
\begin{array}{llll}
    i & (k) & P \rightarrow Q & \rRI{m,n} \\
\end{array}
\]

Hierbei gilt:
\begin{itemize}
    \item \(m < n < k\), wobei \(m\) die Zeile ist, in der \(P\) abgeleitet wurde, und \(n\) ist die Zeile, in der \(Q\) abgeleitet wurde.
    \item Die Menge der Annahmen \(i\) darf die Annahme \(m\) nicht enthalten.
\end{itemize}
\end{definition}

\subsubsection{Eliminierung der Implikation}
\label{rule:RE}

\begin{definition}[Eliminierungsregel der Implikation]
Seien die Zeilen \( m \) und \(n\) einer Beweistabelle wie folgt definiert:

\textbf{Zeile \( m \):}
\[
\begin{array}{llll}
    i & (m) & P \rightarrow Q & \dots \\
\end{array}
\]

\textbf{Zeile \( n \):}
\[
\begin{array}{llll}
    j & (n) & P & \dots \\
\end{array}
\]

Die Eliminierungsregel \(\rightarrow E\) erlaubt es, aus \(P \rightarrow Q\) und \(P\) die Aussage \(Q\) in einer neuen Zeile \( k \) abzuleiten. Dies wird wie folgt notiert:
\[
\begin{array}{llll}
    i, j & (k) & Q & \rRE{m,n} \\
\end{array}
\]

Hierbei gilt:
\begin{itemize}
    \item \(m, n < k\), wobei \(m\) die Zeile ist, in der \(P \rightarrow Q\) abgeleitet wurde, und \(n\) die Zeile, in der \(P\) abgeleitet wurde.
    \item Die Indizes \(i\) und \(j\) geben die Listen der Annahmen an, die für die Ableitungen in den Zeilen \(m\) und \(n\) verwendet wurden.
\end{itemize}
\end{definition}

\subsection{Regeln für die Äquivalenz}

\subsubsection{Einführung der Äquivalenz}
\label{rule:LRI}
Die Äquivalenz, oft dargestellt durch das Symbol \(\leftrightarrow\), ist ein logischer Operator, der \enquote{genau dann, wenn} bedeutet. Im Kalkül des natürlichen Schließens gibt es zwei grundlegende Regeln für die Äquivalenz: die Einführungs- und die Eliminierungsregel.

\begin{definition}[Einführungsregel der Äquivalenz]
Seien die Zeilen \( m \) und \( n \) einer Beweistabelle wie folgt definiert:

\textbf{Zeile \( m \):}
\[
\begin{array}{llll}
    i & (m) & P \rightarrow Q & \dots \\
\end{array}
\]

\textbf{Zeile \( n \):}
\[
\begin{array}{llll}
    j & (n) & Q \rightarrow P & \dots \\
\end{array}
\]

Die Einführungsregel \(\leftrightarrow I\) erlaubt die Ableitung von \( P \leftrightarrow Q \) in einer neuen Zeile \( k \). Dies wird wie folgt notiert:
\[
\begin{array}{llll}
    i, j & (k) & P \leftrightarrow Q & \rLRI{m,n} \\
\end{array}
\]

Hierbei gilt:
\begin{itemize}
    \item \(m, n < k\), wobei \(m\) die Zeile ist, in der \(P \rightarrow Q\) abgeleitet wurde, und \(n\) die Zeile, in der \(Q \rightarrow P\) abgeleitet wurde.
    \item Die Indizes \( i \) und \( j \) geben die Listen der Annahmen an, die für die Ableitungen in den Zeilen \( m \) und \( n \) verwendet wurden.
\end{itemize}

\end{definition}

\subsubsection{Eliminierung der Äquivalenz}
\label{rule:LRE1}\label{rule:LRE2}
Die Äquivalenz \(\leftrightarrow\) erlaubt es, aus einer zusammengesetzten Aussage die beiden Richtungsimplikationen zu isolieren. Im Kalkül des natürlichen Schließens werden hierfür zwei Eliminierungsregeln definiert: \(\leftrightarrow E1\) und \(\leftrightarrow E2\).

\begin{definition}[Eliminierungsregeln der Äquivalenz]
Sei die Zeile \(m\) einer Beweistabelle wie folgt definiert:

\textbf{Zeile \(m\):}
\[
\begin{array}{llll}
    i & (m) & P \leftrightarrow Q & \dots \\
\end{array}
\]

Die Eliminierungsregeln \(\leftrightarrow E1\) und \(\leftrightarrow E2\) erlauben es, aus \(P \leftrightarrow Q\) jeweils \(P \rightarrow Q\) oder \(Q \rightarrow P\) abzuleiten. Dies wird wie folgt notiert:

\textbf{Eliminierung \(\leftrightarrow E1\):}
\[
\begin{array}{llll}
    i & (k) & P \rightarrow Q & \rLREa{m} \\
\end{array}
\]

\textbf{Eliminierung \(\leftrightarrow E2\):}
\[
\begin{array}{llll}
    i & (n) & Q \rightarrow P & \rLREb{m} \\
\end{array}
\]

Hierbei gilt:
\begin{itemize}
    \item \(m < k, n\), wobei \(m\) die Zeile ist, in der \(P \leftrightarrow Q\) abgeleitet wurde.
    \item Der Index \(i\) repräsentiert die Liste der Annahmen, die für die Ableitung von \(P \leftrightarrow Q\) verwendet wurden. Diese Liste bleibt für die Ableitungen von \(P \rightarrow Q\) und \(Q \rightarrow P\) unverändert.
\end{itemize}

Die Eliminierungsregeln ermöglichen es, eine Äquivalenz in ihre Bestandteile zu zerlegen und jeweils nur eine der beiden Richtungsimplikationen weiter zu verwenden.
\end{definition}

\subsection{Regeln für den Widerspruch und die Negation}

Der Widerspruch, dargestellt durch das Symbol \(\bot\), und die Negation, dargestellt durch das Symbol \(\neg\), sind zentrale Bestandteile des Kalküls des natürlichen Schließens. Es gibt drei grundlegende Regeln in diesem Zusammenhang: die Einführungsregel des Widerspruchs, die Einführungsregel der Negation und die Eliminierungsregel der Negation.

\subsection{Regeln für den Widerspruch und die Negation}

Der Widerspruch, dargestellt durch das Symbol \(\bot\), und die Negation, dargestellt durch das Symbol \(\neg\), sind zentrale Bestandteile des Kalküls des natürlichen Schließens. Es gibt drei grundlegende Regeln in diesem Zusammenhang: die Einführungsregel des Widerspruchs, die Einführungsregel der Negation und die Eliminierungsregel der Negation.

\subsubsection{Einführung des Widerspruchs}
\label{rule:BI}

\begin{definition}[Einführungsregel des Widerspruchs]
Seien die Zeilen einer Beweistabelle wie folgt definiert:

\textbf{Zeile \(m\):}
\[
\begin{array}{llll}
    i & (m) & P & \dots \\
\end{array}
\]

\textbf{Zeile \(n\):}
\[
\begin{array}{llll}
    j & (n) & \neg P & \dots \\
\end{array}
\]

Die Einführungsregel \(\bot I\) erlaubt die Ableitung von \(\bot\) in einer neuen Zeile \(k\). Dies wird wie folgt notiert:
\[
\begin{array}{llll}
    i, j & (k) & \bot & \rBI{m,n} \\
\end{array}
\]

Hierbei gilt:
\begin{itemize}
    \item \(m, n < k\), wobei \(m\) die Zeile ist, in der \(P\) abgeleitet wurde, und \(n\) die Zeile, in der \(\neg P\) abgeleitet wurde.
    \item Die Indizes \(i\) und \(j\) geben die Listen der Annahmen an, die für die Ableitungen in den Zeilen \(m\) und \(n\) verwendet wurden.
\end{itemize}
\end{definition}

\subsubsection{Einführung der Negation}
\label{rule:CI}

\begin{definition}[Einführungsregel der Negation]
Seien die Zeilen einer Beweistabelle wie folgt definiert:

\textbf{Zeile \(m\):}
\[
\begin{array}{llll}
    m & (m) & P & \rA \\
\end{array}
\]

\textbf{Zeile \(n\):}
\[
\begin{array}{llll}
    m,j & (n) & \bot & \dots \\
\end{array}
\]

Die Einführungsregel \(\neg I\) erlaubt die Ableitung von \(\neg P\) in einer neuen Zeile \(k\). Dies wird wie folgt notiert:
\[
\begin{array}{llll}
    m & (k) & \neg P & \rCI{m,n} \\
\end{array}
\]

Hierbei gilt:
\begin{itemize}
    \item \(m < n < k\), wobei \(m\) die Zeile ist, in der \(P\) angenommen wurde, und \(n\) die Zeile, in der ein Widerspruch \(\bot\) abgeleitet wurde.
    \item Die Menge der Annahmen \(j\) darf die Annahme aus Zeile \(m\) nicht enthalten.
\end{itemize}
\end{definition}

\subsubsection{Eliminierung der Negation}
\label{rule:CE}

\begin{definition}[Eliminierungsregel der Negation]
Seien die Zeilen einer Beweistabelle wie folgt definiert:

\textbf{Zeile \(m\):}
\[
\begin{array}{llll}
    m & (m) & \neg P & \rA \\
\end{array}
\]

\textbf{Zeile \(n\):}
\[
\begin{array}{llll}
    j & (n) & \bot & \dots \\
\end{array}
\]

Die Eliminierungsregel \(\neg E\) erlaubt die Ableitung von \(P\) in einer neuen Zeile \(k\). Dies wird wie folgt notiert:
\[
\begin{array}{llll}
    m, j & (k) & P & \rCE{m,n} \\
\end{array}
\]

Hierbei gilt:
\begin{itemize}
    \item \(m < n < k\), wobei \(m\) die Zeile ist, in der \(\neg P\) als Annahme definiert wurde, und \(n\) die Zeile, in der ein Widerspruch \(\bot\) festgestellt wurde.
    \item Der Index \(m\) repräsentiert die Annahme von \(\neg P\), und \(j\) gibt die Listen der Annahmen an, die für die Ableitung von \(\bot\) verwendet wurden.
\end{itemize}
\end{definition}

\subsection{Regeln für den Allquantor}

Die Allquantifikation, dargestellt durch das Symbol \(\forall\), ist ein logischer Operator, der \enquote{für alle} bedeutet. Im Kalkül des natürlichen Schließens gibt es zwei grundlegende Regeln für die Allquantifikation: die Einführungs- und die Eliminierungsregel.

\subsubsection{Einführungsreggel der  Allquantifikation}
\label{rule:UI}
\begin{definition}[Einführungsregel der Allquantifikation]
Seien die Zeilen einer Beweistabelle wie folgt definiert:

\textbf{Zeile \(m\):}
\[
\begin{array}{llll}
    i & (m) & P(x) & \dots \\
\end{array}
\]

Die Einführungsregel \(\forall I\) erlaubt die Ableitung von \(\forall x P(x)\) in einer neuen Zeile \(k\). Dies wird wie folgt notiert:
\[
\begin{array}{llll}
    i & (k) & \forall x P(x) & \rUI{m} \\
\end{array}
\]

Hierbei gilt:
\begin{itemize}
    \item \(m < k\), wobei \(m\) die Zeile ist, in der \(P(x)\) abgeleitet wurde.
    \item Die Variable \(x\) darf in den Annahmen der Zeile \(m\) nicht frei vorkommen.
    \item Der Index \(i\) repräsentiert die Liste der Annahmen, die für die Ableitung von \(P(x)\) verwendet wurden. Diese Liste bleibt für die Ableitung von \(\forall x P(x)\) unverändert.
\end{itemize}
\end{definition}

\subsubsection{Eliminierungsregel der  Allquantifikation}
\label{rule:UE}
\begin{definition}[Eliminierungsregel der Allquantifikation]
Seien die Zeilen einer Beweistabelle wie folgt definiert:

\textbf{Zeile \(m\):}
\[
\begin{array}{llll}
    i & (m) & \forall x P(x) & \dots \\
\end{array}
\]

Die Eliminierungsregel \(\forall E\) erlaubt die Ableitung von \(P(t)\) in einer neuen Zeile \(k\), wobei \(t\) ein beliebiger Term ist. Dies wird wie folgt notiert:
\[
\begin{array}{llll}
    i & (k) & P(t) & \rUE{m} \\
\end{array}
\]

Hierbei gilt:
\begin{itemize}
    \item \(m < k\), wobei \(m\) die Zeile ist, in der \(\forall x P(x)\) abgeleitet wurde.
    \item Der Term \(t\) kann eine Konstante oder ein komplexer Ausdruck aus Funktionssymbolen sein.
    \item Der Index \(i\) repräsentiert die Liste der Annahmen, die für die Ableitung von \(\forall x P(x)\) verwendet wurden. Diese Liste bleibt für die Ableitung von \(P(t)\) unverändert.
\end{itemize}
\end{definition}

\subsection{Regeln für den Existenzquantor}

Der Existenzquantor, oft dargestellt durch das Symbol \(\exists\), ist ein logischer Operator, der \enquote{es existiert} bedeutet. Im Kalkül des natürlichen Schließens gibt es zwei grundlegende Regeln für den Existenzquantor: die Einführungs- und die Eliminierungsregel.

\subsubsection{Einführungsregel des Existenzquantors}
\label{rule:EI}
\begin{definition}[Einführungsregel des Existenzquantors]
Seien die Zeilen einer Beweistabelle wie folgt definiert:

\textbf{Zeile \(m\):}
\[
\begin{array}{llll}
    i & (m) & P(t) & \dots \\
\end{array}
\]

Die Einführungsregel \(\exists I\) erlaubt die Ableitung von \(\exists x P(x)\) in einer neuen Zeile \(k\). Dies wird wie folgt notiert:
\[
\begin{array}{llll}
    i & (k) & \exists x P(x) & \rEI{m} \\
\end{array}
\]

Hierbei gilt:
\begin{itemize}
    \item \(m < k\), wobei \(m\) die Zeile ist, in der \(P(t)\) abgeleitet wurde.
    \item Der Term \(t\) darf eine Konstante oder ein komplexer Ausdruck aus Funktionssymbolen sein.
    \item Der Index \(i\) repräsentiert die Liste der Annahmen, die für die Ableitung von \(P(t)\) verwendet wurden. Diese Liste bleibt für die Ableitung von \(\exists x P(x)\) unverändert.
\end{itemize}
\end{definition}

\subsubsection{Eliminierungsregel des Existenzquantors}
\label{rule:EE}
\begin{definition}[Eliminierungsregel des Existenzquantors]
Seien die Zeilen einer Beweistabelle wie folgt definiert:

\textbf{Zeile \(m\):}
\[
\begin{array}{llll}
    i & (m) & \exists x P(x) & \dots \\
\end{array}
\]

\textbf{Zeile \(n\):}
\[
\begin{array}{llll}
    n & (n) & P(t) & \rA \\
\end{array}
\]

\textbf{Zeile \(p\):}
\[
\begin{array}{llll}
    i,n & (p) & Q & \dots \\
\end{array}
\]

Die Eliminierungsregel \(\exists E\) erlaubt die Ableitung von \(Q\) in einer neuen Zeile \(k\). Dies wird wie folgt notiert:
\[
\begin{array}{llll}
    i,n & (k) & Q & \rEE{m,n,p} \\
\end{array}
\]

Hierbei gilt:
\begin{itemize}
    \item \(m < n < p < k\).
    \item Der Term \(t\) ist eine neue Variable, die in den bisherigen Zeilen des Beweises nicht vorkommt.
    \item Die Aussage \(Q\) darf \(t\) nicht enthalten.
    \item Weder die Liste \(i\) der Annahmen der Zeile \(m\) noch der Index \(n\) der Annahme der Zeile \(n\) dürfen die Variable \(t\) enthalten.
    \item Die Indizes \(i\) und \(n\) repräsentieren:
    \begin{itemize}
        \item \(i\) ist die Liste der Annahmen, die zur Ableitung von \(\exists x P(x)\) in Zeile \(m\) verwendet wurden.
        \item \(n\) ist der Index der Annahme, die in Zeile \(n\) getroffen wurde.
    \end{itemize}
\end{itemize}
\end{definition}

\subsection{Regeln für das Identitätssymbol}


Das Identitätssymbol, oft dargestellt durch das Symbol \(=\), wird in der Prädikatenlogik verwendet, um die Gleichheit von Termen auszudrücken. Es stellt sicher, dass zwei Terme, die durch das Identitätssymbol verbunden sind, als dasselbe Objekt behandelt werden können. Im Kalkül des natürlichen Schließens gibt es zwei grundlegende Regeln für das Identitätssymbol: die Einführungs- und die Eliminierungsregel.

\subsubsection{Einführungsregel für das Identitätssymbol}
\label{rule:II} 
\begin{definition}[Einführungsregel für das Identitätssymbol]
Die Einführungsregel \(=I\) besagt, dass für jeden Term \(t\) die Aussage \(t = t\) wahr ist, da jedes Objekt mit sich selbst identisch ist. Dies wird wie folgt notiert:
\[
\begin{array}{lll}
    (m) & t = t & \rII \\
\end{array}
\]
Hierbei steht \(m\) für einen allgemeinen Zeilenindex.
\end{definition}

\subsubsection{Eliminierungsregel für das Identitätssymbol}
\label{rule:IE}
\begin{definition}[Eliminierungsregel für das Identitätssymbol]
Seien die Zeilen einer Beweistabelle wie folgt definiert:

\textbf{Zeile \(m\):}
\[
\begin{array}{llll}
    i & (m) & t = u & \dots \\
\end{array}
\]

\textbf{Zeile \(n\):}
\[
\begin{array}{llll}
    j & (n) & P(t) & \dots \\
\end{array}
\]

Die Eliminierungsregel \(=E\) erlaubt die Ableitung von \(P(u)\) in einer neuen Zeile \(k\). Dies wird wie folgt notiert:
\[
\begin{array}{llll}
    i,j & (k) & P(u) & \rIE{m,n} \\
\end{array}
\]

Hierbei gilt:
\begin{itemize}
    \item \(m, n < k\), wobei \(m\) die Zeile ist, in der \(t = u\) abgeleitet wurde, und \(n\) die Zeile, in der \(P(t)\) festgestellt wurde.
    \item Der Term \(t\) wird durch \(u\) ersetzt, wenn \(t = u\) gilt.
    \item Die Indizes \(i,j\) repräsentieren die Listen der Annahmen, die für die Ableitung von \(t = u\) und \(P(t)\) verwendet wurden. Diese Listen bleiben unverändert.
\end{itemize}
\end{definition}

\subsection{Regeln von Definitionen}

\subsubsection{Explizite Definition}
Wir betrachten den allgemeinen Fall, dass ein Symbol \(Q(x_1,\dots,x_n)\) explizit durch einen Ausdruck \(\varphi(x_1,\dots,x_n)\) definiert wird:

\begin{definition}[\(Q\) (explizit)]
\label{QLpxSubOnewDotswxSubnRpDefEqVarphiLpxSubOnewDotswxSubnRp}
\[
Q(x_1,\dots,x_n) \coloneqq \varphi(x_1,\dots,x_n)
\]    
\end{definition}

Hierbei ist \(Q(x_1,\dots,x_n)\) das neu eingeführte Symbol (Definiendum) und \(\varphi(x_1,\dots,x_n)\) der definierende Ausdruck (Definiens).

\paragraph{Eliminierungsregel für explizite Definitionen}
Sei die Zeile einer Beweistabelle wie folgt definiert:

\textbf{Zeile \(m\):}
\[
\begin{array}{llll}
   i & (m) & Q(x_1,\dots,x_n) & \dots \\
\end{array}
\]

Die Eliminierungsregel erlaubt die Ableitung des definierenden Ausdrucks \(\varphi(x_1,\dots,x_n)\) in einer neuen Zeile \(k\). Dies wird wie folgt notiert:
\[
\begin{array}{llll}
   i& (k) & \varphi(x_1,\dots,x_n) & \QLpxSubOnewDotswxSubnRpDefEqVarphiLpxSubOnewDotswxSubnRp{m} \\
\end{array}
\]

Hierbei gilt:
\begin{itemize}
    \item \(m\) verweist auf die Zeile, in der \(Q(x_1,\dots,x_n)\) eingeführt wurde.
    \item Der Index \(i\) repräsentiert eine Liste von der Annahmen.
\end{itemize}

\paragraph{Einführungsregel für explizite Definitionen}
Sei die Zeile einer Beweistabelle wie folgt definiert:

\textbf{Zeile \(m\):}
\[
\begin{array}{llll}
   i& (m) & \varphi(x_1,\dots,x_n) & \dots \\
\end{array}
\]

Die Einführungsregel erlaubt die Ableitung des definierten Symbols \(Q(x_1,\dots,x_n)\) in einer neuen Zeile \(k\). Dies wird wie folgt notiert:
\[
\begin{array}{llll}
   i& (k) & Q(x_1,\dots,x_n) & \QLpxSubOnewDotswxSubnRpDefEqVarphiLpxSubOnewDotswxSubnRp{m} \\
\end{array}
\]

Hierbei gilt:
\begin{itemize}
    \item \(m\) verweist auf die Zeile, in der \(\varphi(x_1,\dots,x_n)\) eingeführt wurde.
    \item Der Index \(i\) repräsentiert eine Liste von der Annahmen.
\end{itemize}

\subsubsection{Partielle Definition}
Wir betrachten nun den Fall einer \emph{partiellen Definition}, bei der ein neues Symbol \(Q(x_1,\dots,x_n)\) nur dann durch den Ausdruck \(\varphi(x_1,\dots,x_n)\) festgelegt wird, wenn eine Bedingung \(C(x_1,\dots,x_n)\) erfüllt ist. Formal:

\begin{definition}[\(Q\) (partiell)]
\label{CLpxSubOnewDotswxSubnRpToQLpxSubOnewDotswxSubnRpDefEqVarphiLpxSubOnewDotswxSubnRp}
\[
C(x_1,\dots,x_n)\rightarrow Q(x_1,\dots,x_n) \coloneqq \varphi(x_1,\dots,x_n)
\]
\end{definition}

Hierbei ist 
\begin{itemize}
    \item \(Q(x_1,\dots,x_n)\) das \emph{neu} eingeführte Symbol (\emph{Definiendum}), 
    \item \(\varphi(x_1,\dots,x_n)\) der bereits bekannte Ausdruck (\emph{Definiens}),
    \item \(C(x_1,\dots,x_n)\) die Bedingung, unter der die Definition gilt.
\end{itemize}

\paragraph{Eliminierungsregel für partielle Definitionen}
Wenn in einer Beweistabelle zwei Zeilen wie folgt vorliegen:

\[
\textbf{Zeile }m: 
\quad
\begin{array}{llll}
   i & (m) & Q(x_1,\dots,x_n) & \dots \\
\end{array}
\]

\[
\textbf{Zeile }n:
\quad
\begin{array}{llll}
   j & (n) & C(x_1,\dots,x_n) & \dots \\
\end{array}
\]

dann darf in einer neuen Zeile \(k\) \emph{(Elimination)} der definierende Ausdruck \(\varphi(x_1,\dots,x_n)\) abgeleitet werden. Wir notieren dies:

\[
\textbf{Zeile }k: 
\quad
\begin{array}{llll}
   i,j & (k) & \varphi(x_1,\dots,x_n) & \text{(part. Def.)}_{m,n} \\
\end{array}
\]

\paragraph{Einführungsregel für partielle Definitionen}
Analog dazu darf man in einer neuen Zeile \(k\) \emph{(Einf\"uhrung)} das definierte Symbol \(Q(x_1,\dots,x_n)\) ableiten, wenn man bereits \(\varphi(x_1,\dots,x_n)\) \emph{und} \(C(x_1,\dots,x_n)\) hergeleitet hat. Das hei\ss t, liegen folgende Zeilen vor:

\[
\textbf{Zeile }m: 
\quad
\begin{array}{llll}
   i & (m) & \varphi(x_1,\dots,x_n) & \dots \\
\end{array}
\]

\[
\textbf{Zeile }n: 
\quad
\begin{array}{llll}
   j & (n) & C(x_1,\dots,x_n) & \dots \\
\end{array}
\]

so kann in Zeile \(k\) gefolgert werden:

\[
\textbf{Zeile }k: 
\quad
\begin{array}{llll}
   i,j & (k) & Q(x_1,\dots,x_n) & \text{(part. Def.)}_{m,n} \\
\end{array}
\]

Dabei verweisen die Indizes \(m,n\) auf die Zeilen, in denen \(Q(x_1,\dots,x_n)\) bzw. \(\varphi(x_1,\dots,x_n)\) und \(C(x_1,\dots,x_n)\) abgeleitet wurden, während \(i,j\) für die jeweiligen Annahmelisten stehen.

\subsubsection{Rekursive Definition}
Wir betrachten den allgemeinen Fall, dass ein Symbol \(\sigma(x_1,\dots,x_n)\) rekursiv durch Basisfälle und eine Rekursionsvorschrift definiert wird:

\begin{definition}[\(\sigma\) (rekursiv)]
\label{dfSigmaGenRecursive}
\begin{itemize}
    \item \textbf{Basisfall:}
    \[
    \sigma(a_1, \ldots, a_k) \coloneqq b,
    \]
    wobei \(a_1, \ldots, a_k\) konkrete Werte und \(b\) ein bereits bekannter Ausdruck sind.
    \item \textbf{Rekursionsvorschrift:}
    \[
    \sigma(x_1, \ldots, x_n) \coloneqq f\big(\sigma(y_1, \ldots, y_m), z_1, \ldots, z_p\big),
    \]
    wobei \(y_1, \ldots, y_m\) Teilmengen von \(x_1, \ldots, x_n\) und \(f\) wohldefinierte Funktionen sind.
\end{itemize}
\end{definition}

\paragraph{Eliminierungsregel für den Basisfall}
Sei die Zeile einer Beweistabelle wie folgt definiert:

\textbf{Zeile \(m\):}
\[
\begin{array}{llll}
   i & (m) & \sigma(x_1, \ldots, x_n) & \dots \\
\end{array}
\]

Falls die Argumente von \(\sigma(x_1, \ldots, x_n)\) einem Basisfall entsprechen, kann der entsprechende Wert \(b\) abgeleitet werden. Dies wird wie folgt notiert:
\[
\begin{array}{llll}
   i & (k) & b & \dfSigmaGenRecursive{m} \\
\end{array}
\]

\paragraph{Eliminierungsregel für die Rekursionsvorschrift}
Sei die Zeile einer Beweistabelle wie folgt definiert:

\textbf{Zeile \(m\):}
\[
\begin{array}{llll}
   i & (m) & \sigma(x_1, \ldots, x_n) & \dots \\
\end{array}
\]

Falls die Argumente von \(\sigma(x_1, \ldots, x_n)\) der Rekursionsvorschrift entsprechen, kann der entsprechende Ausdruck abgeleitet werden. Dies wird wie folgt notiert:
\[
\begin{array}{llll}
   i & (k) & f\big(\sigma(y_1, \ldots, y_m), z_1, \ldots, z_p\big) & \dfSigmaGenRecursive{m} \\
\end{array}
\]

\paragraph{Einführungsregel für den Basisfall}
Sei die Zeile einer Beweistabelle wie folgt definiert:

\textbf{Zeile \(m\):}
\[
\begin{array}{llll}
   i & (m) & b & \dots \\
\end{array}
\]

Falls der Ausdruck \(b\) gegeben ist, kann das definierte Symbol \(\sigma(a_1, \ldots, a_k)\) abgeleitet werden. Dies wird wie folgt notiert:
\[
\begin{array}{llll}
   i & (k) & \sigma(a_1, \ldots, a_k) & \dfSigmaGenRecursive{m} \\
\end{array}
\]

\paragraph{Einführungsregel für die Rekursionsvorschrift}
Sei die Zeile einer Beweistabelle wie folgt definiert:

\textbf{Zeile \(m\):}
\[
\begin{array}{llll}
   i & (m) & f\big(\sigma(y_1, \ldots, y_m), z_1, \ldots, z_p\big) & \dots \\
\end{array}
\]

Falls der Ausdruck \(f\big(\sigma(y_1, \ldots, y_m), z_1, \ldots, z_p\big)\) gegeben ist, kann das definierte Symbol \(\sigma(x_1, \ldots, x_n)\) abgeleitet werden. Dies wird wie folgt notiert:
\[
\begin{array}{llll}
   i & (k) & \sigma(x_1, \ldots, x_n) & \dfSigmaGenRecursive{m} \\
\end{array}
\]

\paragraph{Zusatzregel: Vereinfachung der Rekursion}
Um die Übersichtlichkeit in Beweisen zu gewährleisten, können einzelne Zwischenschritte rekursiver Ableitungen ausgelassen werden. Dabei wird die Äquivalenz zwischen dem Endergebnis einer rekursiven Ableitung und der vollständigen Folge aller Zwischenschritte vorausgesetzt. Formal:
\[
\sigma(x_1, \ldots, x_n) \leftrightarrow 
\begin{cases} 
    b & \text{(nach endlicher Anwendung der Rekursion)} \\
    \sigma(y_1, \ldots, y_m) & \text{(Zwischenschritt der Rekursion)} \\
    f\big(\sigma(y_1, \ldots, y_m), z_1, \ldots, z_p\big) & \text{(Rekursionsschritt)}
\end{cases}
\]
Zwischenschritte können jedoch bei Bedarf in die Beweistabelle eingefügt werden, um den Nachvollzug zu erleichtern.


\subsubsection{Implizite Definition}
Wir betrachten nun den Fall, dass ein \emph{neues} Symbol \(\sigma\) nicht \emph{direkt} (explizit) durch eine Gleichung eingeführt wird, sondern \emph{implizit} über bestimmte Axiome bzw. Eigenschaften, die \(\sigma\) erfüllen soll. Formal sprechen wir auch von einem \textbf{Axiomatisierungsverfahren} oder einer \textbf{impliziten Definition} \(\Phi(\sigma)\).

\begin{definition}[\(\sigma\) (implizit)]
\label{dfSigmaGenImplicitAxiom}
Sei \(\sigma\) ein neues Symbol (z.\,B. ein Prädikat, eine Konstante, ein Funktionssymbol oder ein Operatorsymbol), das in der formalen Sprache \(\mathcal{L}\) bisher nicht vorkam. Weiter sei
\[
\Phi(\sigma) \;=\; \{\phi_1(\sigma), \phi_2(\sigma), \dots\}
\]
eine Menge von Axiomen oder Aussagen, die \(\sigma\) charakterisieren. Dann nennen wir die Forderung, dass alle \(\phi_i(\sigma)\) in unserem Theoriekontext \(\mathcal{T}\) gelten,
\[
\forall i \in I:\quad \phi_i(\sigma) \quad \text{(für eine Indexmenge }I\text{)},
\]
eine \textbf{implizite Definition} von \(\sigma\).
\end{definition}

\paragraph{Anmerkung zum Gebrauch im Beweissystem}
Eine implizite Definition kann in \emph{vielen} Fällen nicht direkt in einer einzelnen Zeile der Beweistabelle stehen. Stattdessen wird das Symbol \(\sigma\) bereits \emph{vor} oder \emph{zum} Beginn eines formalen Beweises (oder ganz zu Anfang eines Abschnitts) \emph{vorausgesetzt}. 

Eine Beispielformulierung dafür könnte lauten:
\[
\text{\glqq Sei }\sigma\text{ ein Symbol, das die Axiome }\Phi(\sigma)\text{ erfüllt.\grqq}
\]

\paragraph{Regel (Implizite Definition)}
Liegt eine Aussage \(\phi_i(\sigma)\) in der Menge \(\Phi(\sigma)\) vor, so kann sie als gegeben angenommen werden und in einer neuen Zeile der Beweistabelle wie folgt notiert werden:

\textbf{Zeile }k:
\[
\begin{array}{llll}
   \ & (k) & \phi_i(\sigma) & \dfSigmaGenImplicitAxiom{} \\
\end{array}
\]
\emph{Beachte:} Hierbei wird keine spezifische Ableitung durchgeführt, sondern \(\phi_i(\sigma)\) wird direkt als Axiom eingeführt.

\paragraph{Alternative Darstellung (Axiome als Schlussregeln)}
Axiome können in formalen Kalkülen des natürlichen Schließens oft in Form von Inferenzregeln formuliert werden, um die Anwendbarkeit in Beweisen zu erleichtern. Im Gegensatz zu „freien“ Axiomen, die ohne Prämissen gelten, können solche axiomatischen Regeln Abhängigkeiten (Annahmen) aufweisen, die in einem Beweis erfüllt sein müssen, bevor das Axiom angewendet werden kann.

\begin{definition}[Axiom-Nutzungsregel]
\label{dfSigmaGenImplicitRule}
Sei \(\Phi\) ein Axiom, das in einer formalen Sprache \(\mathcal{L}\) definiert ist und eine Aussage der Form 
\[
\Gamma \vdash \Delta
\]
besitzt, wobei \(\Gamma\) eine Menge von Prämissen und \(\Delta\) eine Konklusion ist. Die Axiom-Nutzungsregel erlaubt es, \(\Phi\) in einem Beweis wie folgt zu verwenden:

\paragraph{Annahmen}
\textbf{Zeile \(m_1\):}
\[
\begin{array}{llll}
   i_1& (m_1) & A_1 & \dots \\
\end{array}
\]

\textbf{Zeile \(m_2\):}
\[
\begin{array}{llll}
   i_2& (m_2) & A_2 & \dots \\
\end{array}
\]

\vdots

\textbf{Zeile \(m_k\):}
\[
\begin{array}{llll}
   i_k& (m_k) & A_k & \dots \\
\end{array}
\]

\paragraph{Schluss}
\[
\begin{array}{llll}
   i_1, i_2, \ldots, i_k & (n) & \Delta & \dfSigmaGenImplicitRule{m_1,m_2,\dots,m_k} \\
\end{array}
\]

Hierbei gilt:
\begin{itemize}
    \item \( \Gamma \) besteht aus den \( k \) Elementen \( \{A_1, A_2, \ldots, A_k\} \), die in separaten Zeilen dargestellt werden.
    \item \( m_1, m_2, \ldots, m_k \) sind die Zeilennummern, in denen die Prämissen \( \Gamma \) des Axioms \(\Phi\) erfüllt wurden.
    \item \( i_1, i_2, \ldots, i_k \) sind die Annahmen, von denen die Prämissen \( \Gamma \) abhängen.
    \item \( \Delta \) ist die Konklusion des Axioms \(\Phi\), die in der aktuellen Zeile \( n \) abgeleitet wird.
    \item \( n > m_k \) für alle \( k \), d. h. die Zeilennummer \( n \) der Konklusion ist größer als die Zeilennummern \( m_1, m_2, \ldots, m_k \) aller Annahmen.
\end{itemize}
\end{definition}

\begin{remark}
Die Axiom-Nutzungsregel betont die Anwendbarkeit von Axiomen in Beweisen, insbesondere wenn sie explizit Prämissen (\(\Gamma\)) voraussetzen. Solche Prämissen definieren den Kontext, in dem die Aussage des Axioms (\(\Delta\)) gültig ist. Diese Darstellung ist besonders hilfreich, wenn Axiome Äquivalenzen oder Bedingungen enthalten, die in spezifischen Situationen überprüft werden müssen.
\end{remark}


\subsubsection{Regel zur Iota-Definition}

Im Unterschied zu einer \emph{expliziten} oder \emph{impliziten} Definition wird die \(\iota\)-Konstante \(\sigma\) nicht erst \emph{im} Beweis selbst eingeführt, sondern bereits \emph{außerhalb} (meta-logisch) durch
\begin{definition}[\(\sigma\) (iota)]
\label{dfSigmaGenIota}
\[
\sigma \;\coloneqq\; \iota x \, P(x).
\]
\end{definition}
Dies setzt voraus, dass \(\exists!\,x\,P(x)\) (es existiert genau \emph{ein} \(x\), für das \(P(x)\) gilt). 

\paragraph{Iota-Regel (Schluss auf \(P(\sigma)\))}
Sobald die Konstante \(\sigma\) \textbf{durch Definition} als \(\sigma := \iota x \, P(x)\) feststeht, kann in der Beweistabelle direkt \(P(\sigma)\) notiert werden. Man benötigt dafür keine gesonderte Einführungs- oder Eliminierungszeile (\(\sigma\) steht ja bereits \emph{ausserhalb} des Beweises fest). 

Dies lässt sich als \emph{eine} Regel formulieren:
\[
\begin{array}{llll}
   & (k) & P(\sigma) & \dfSigmaGenIota{} \\
\end{array}
\]



\subsection{Regeln zur Nutzung von Theoremen}
\label{rule:TheoremUsage}

Im Kalkül des natürlichen Schließens ist es oft nützlich, bereits bewiesene Theoreme in weiteren Beweisen zu verwenden. Um dies systematisch zu ermöglichen, führen wir die \textbf{Theorem-Nutzungsregel} ein. Diese Regel erlaubt es, ein zuvor bewiesenes Theorem als eine zusätzliche Inferenzregel in einem Beweis zu verwenden.

\begin{definition}[Theorem-Nutzungsregel]
\label{theoremUsageRule}
Sei \( T \) ein bewiesenes Theorem, das eine Aussage der Form \( \Gamma \vdash \Delta \) hat, wobei \( \Gamma \) eine Menge von Prämissen und \( \Delta \) eine Konklusion ist. Die Theorem-Nutzungsregel erlaubt es, \( T \) in einem Beweis wie folgt zu verwenden:

\paragraph{Annahmen}
\textbf{Zeile \(m_1\):}
\[
\begin{array}{llll}
   i_1& (m_1) & A_1 & \dots \\
\end{array}
\]

\textbf{Zeile \(m_2\):}
\[
\begin{array}{llll}
   i_2& (m_2) & A_2 & \dots \\
\end{array}
\]

\vdots

\textbf{Zeile \(m_k\):}
\[
\begin{array}{llll}
   i_k& (m_k) & A_k & \dots \\
\end{array}
\]

\paragraph{Schluss}
\[
\begin{array}{llll}
   i_1, i_2, \ldots, i_k & (n) & \Delta & \theoremUsageRule{m_1,m_2,\dots,m_k} \\
\end{array}
\]

Hierbei gilt:
\begin{itemize}
    \item \( \Gamma \) besteht aus den \( k \) Elementen \( \{A_1, A_2, \ldots, A_k\} \), die in separaten Zeilen dargestellt werden.
    \item \( m_1, m_2, \ldots, m_k \) sind die Zeilennummern, in denen die Prämissen \( \Gamma \) des Theorems \( T \) abgeleitet wurden.
    \item \( i_1, i_2, \ldots, i_k \) sind die Annahmen, von denen die Prämissen \( \Gamma \) abhängen.
    \item \( \Delta \) ist die Konklusion des Theorems \( T \), die in der aktuellen Zeile \( n \) abgeleitet wird.
    \item \( n > m_k \) für alle \( k \), d. h. die Zeilennummer \( n \) der Konklusion ist größer als die Zeilennummern \( m_1, m_2, \ldots, m_k \) aller Annahmen.
\end{itemize}
\end{definition}

\begin{remark}
Die Theorem-Nutzungsregel ermöglicht es, komplexe Schlussfolgerungen modular zu behandeln, indem man sie in kleinere, bereits bewiesene Teile zerlegt. Dies fördert die Wiederverwendbarkeit von Beweisen und erhöht die Übersichtlichkeit und Struktur der Beweistabelle.
\end{remark}

\section{Verkettete Regelanwendung}
In den bisherigen Kapiteln haben wir eine Vielzahl von Inferenzregeln eingeführt, die jeweils genau \emph{einen} logischen Schritt ausführen. In der praktischen Beweisführung ist es jedoch oft wünschenswert, mehrere dieser Regeln \emph{hintereinander} in einem einzigen Schluss anzuwenden, um den Beweisfluss kompakter und übersichtlicher zu gestalten. Dieses Vorgehen wird im Folgenden als \emph{verkettete Regelanwendung} bezeichnet.

\subsection{Allgemeine Formulierung der verketteten Regelanwendung}

\begin{definition}[Verkettete Regelanwendung]
Sei eine Zeile \((m)\) einer Beweistabelle gegeben, in der bereits eine Aussage \(P\) steht. Angenommen, es existiere eine endliche Folge von Regeln \(R_1, R_2, \dots, R_k\), so dass man durch \emph{hintereinandergeschaltete Anwendung} dieser Regeln auf \(P\) eine Aussage \(Q\) gewinnt.

\paragraph{Annahmen}
\textbf{Zeile \(m\):}
\[
\begin{array}{llll}
   i & (m) & P & \dots \\
\end{array}
\]

\paragraph{Schluss}
\textbf{Zeile \(n\):}
\[
\begin{array}{llll}
   i & (n) & Q & R_1R_2\cdots R_k(m) \\
\end{array}
\]
oder präziser, falls Platz vorhanden ist:
\[
\begin{array}{llll}
   i 
   & (n) 
   & Q 
   & R_1 \,\circ\, R_2 \,\circ\,\cdots\,\circ\, R_k \;(m) \\
\end{array}
\]
Hierbei bedeutet \(R_1 \circ R_2 \circ \dots \circ R_k\), dass zuerst \(R_1\) auf Zeile \(m\) angewendet wird, dann \(R_2\) auf das \emph{Ergebnis} von \(R_1\) usw.
\end{definition}

\begin{remark}
Die verkettete Regelanwendung ist \emph{keine} neue logische Regel im Sinne eines Axioms oder einer Erweiterung des Kalküls, sondern eine reine \emph{Abkürzungsnotation}. Sie ändert nicht die \emph{Inhalte} eines Beweises, sondern \emph{lediglich} die Darstellung und Zusammenfassung mehrerer Einzelschritte in einer kompakten Zeile.
\end{remark}





\subsection{Kompakte Notation für Inferenzregeln}

Um die Einführung neuer Inferenzregeln zu vereinfachen und die Übersichtlichkeit zu erhöhen, verwenden wir im Folgenden eine standardisierte, in der mathematischen Logik übliche Darstellungsweise. Dabei werden die Prämissen über einem horizontalen Strich aufgelistet, und die sich daraus ergebende Konklusion darunter geschrieben. Der Name der Regel wird rechts oder links daneben angegeben. 

\subsubsection{Allgemeine Form einer Inferenzregel}
Eine Inferenzregel hat typischerweise die Form
\[
\frac{P_1 \quad P_2 \quad \dots \quad P_n}{Q}
\quad
\text{(Regelname)},
\]
wobei
\begin{itemize}
    \item $P_1, \dots, P_n$ die \emph{Prämissen} (Voraussetzungen) der Regel sind,
    \item $Q$ die \emph{Konklusion} (abgeleitete Aussage) ist,
    \item \textit{Regelname} den Namen der verwendeten Regel angibt.
\end{itemize}
Zusätzlich können \emph{Nebenbedingungen} gelten (z.\,B.\ ob eine Variable nicht in bestimmten Annahmen frei sein darf). Diese werden meist gesondert formuliert.

In manchen Regeln wird eine temporäre Annahme gemacht, um eine Aussage abzuleiten oder einen Widerspruch zu zeigen. Die allgemeine Darstellung dafür lautet:
\[
\frac{
\begin{array}{c}
\text{(Annahme }P\text{)} \\
\vdots \\
R
\end{array}
}{Q}
\]
Hierbei gilt:
\begin{itemize}
    \item \textbf{(Annahme $P$)}: $P$ wird vorübergehend als Annahme eingeführt.
    \item \textbf{$\vdots$}: Dies symbolisiert mehrere logische Schritte, die aus der Annahme $P$ gefolgert werden.
    \item \textbf{$R$}: Die Ableitung einer Aussage (z.\,B. ein Widerspruch $\bot$ oder eine Konklusion).
    \item \textbf{$Q$}: Das abschließende Resultat der Regel, nachdem die Annahme aufgehoben wurde.
\end{itemize}
Diese Notation ist in der Literatur weit verbreitet und bildet die Grundlage für das Kalkül des natürlichen Schließens. Ein konkreter Beispielfall wird im Verlauf des Abschnitts vorgestellt.

\subsubsection{Beispiele bereits bekannter Regeln}

\paragraph{Konjunktionseinführung ($\land I$)} 
\[
\frac{P \quad Q}{P \land Q}
\quad \land I
\]

\paragraph{Konjunktionseliminierung ($\land E1$, $\land E2$)} 
\[
\frac{P \land Q}{P}
\quad \land E1
\quad\quad
\frac{P \land Q}{Q}
\quad \land E2
\]

\paragraph{Implikationseinführung ($\to I$)}
\[
\frac{
\begin{array}{c}
\text{(Annahme }P\text{)} \\
\vdots \\
Q
\end{array}
}{P \rightarrow Q}
\quad \to I
\]
\emph{Nebenbedingung:} Die Annahme $P$ wird \emph{aufgehoben} (geschlossen). Das heißt, $P$ darf in der Schlusszeile nicht mehr als offene Annahme vorkommen.

\paragraph{Implikationelimierung ($\to E$)}
\[
\frac{P \quad P \to Q}{Q}
\quad \to E
\]

\paragraph{Negationseinführung ($\neg I$)}
\[
\frac{
\begin{array}{c}
\text{(Annahme }P\text{)} \\
\vdots \\
\bot
\end{array}
}{\neg P}
\quad \neg I
\]
\emph{Nebenbedingung:} Die Annahme $P$ wird aufgehoben.

\paragraph{Negationelimierung ($\neg E$)}
\[
\frac{\neg P \quad P}{\bot}
\quad \neg E
\]

\subsubsection{Regeln für Allquantor ($\forall$) und Existenzquantor ($\exists$)}

Auch für die Quantoren werden die Inferenzregeln in derselben Notation angegeben. Häufig formuliert man die \emph{Nebenbedingungen} (z.\,B.\ ob eine Variable nicht in bestimmten Annahmen frei sein darf) nur in Worten.

\paragraph{Allquantor-Einführung ($\forall I$)}
\[
\frac{P(x)}{\forall x\,P(x)}
\quad \forall I
\]
\emph{Nebenbedingung:} Die Variable $x$ darf in keinen offenen Annahmen frei vorkommen (sie muss \emph{beliebig} sein).

\paragraph{Allquantor-Eliminierung ($\forall E$)}
\[
\frac{\forall x\,P(x)}{P(t)}
\quad \forall E
\]
\emph{Nebenbedingung:} Der Term $t$ darf überall sinnvoll substituierbar sein (z.\,B.\ keine Kollisionsprobleme mit gebundenen Variablen).

\paragraph{Existenzquantor-Einführung ($\exists I$)}
\[
\frac{P(t)}{\exists x\,P(x)}
\quad \exists I
\]
Hier ist $t$ ein konkreter Term, für den $P(t)$ gilt.

\paragraph{Existenzquantor-Eliminierung ($\exists E$)}
\[
\frac{
\exists x\,P(x) 
\quad
\begin{array}{c}
\text{(Annahme }P(x)\text{)}\\
\vdots \\
Q
\end{array}
}{Q}
\quad \exists E
\]
\emph{Nebenbedingung:}
\begin{itemize}
    \item Die in der Annahme verwendete Variable $x$ darf nicht bereits in anderen offenen Annahmen frei sein.
    \item Die Annahme $P(x)$ wird nach der Ableitung von $Q$ wieder aufgehoben.
\end{itemize}













\section{Weiterführende Regeln}
\subsection{Die alternative Disjunktion}
\subsubsection{Definition}


Die alternative Disjunktion, oft dargestellt durch das Symbol \(\lxor\), ist ein logischer Operator, der "`entweder ... oder ..."' bedeutet. Sie unterscheidet sich von der klassischen Disjunktion (\(\lor\)), da die alternative Disjunktion impliziert, dass genau eine der beiden Aussagen wahr ist, während die andere falsch sein muss. Die alternative Disjunktion kann wie folgt definiert werden:

\begin{definition}[XOR]
\label{XOr}
\[
P \lxor Q \coloneqq (P \land \neg Q) \lor (\neg P \land Q)
\]
\end{definition}

\subsubsection{Regeln für die alternative Disjunktion}

\subsubsection{Einführung der exklusiven Disjunktion}
\label{rule:XOrI1}\label{rule:XOrI2}


\begin{definition}[Einführungsregeln der exklusiven Disjunktion]
Sei die Zeile \(m\) einer Beweistabelle wie folgt definiert:

\textbf{Einführung \(\lxor I1\):}
\[
\begin{array}{l l l l}
    i & (m) & P \land \neg Q & \dots \\
\end{array}
\]

Die Regel \(\lxor I1\) erlaubt die Ableitung von \(P \lxor Q\) in einer neuen Zeile \(k\). Dies wird wie folgt notiert:
\[
\begin{array}{l l l l}
    i & (k) & P \lxor Q & \rXOrIa{m} \\
\end{array}
\]

\textbf{Einführung \(\lxor I2\):}
\[
\begin{array}{l l l l}
    i & (m) & \neg P \land Q & \dots \\
\end{array}
\]

Die Regel \(\lxor I2\) erlaubt die Ableitung von \(P \lxor Q\) in einer neuen Zeile \(k\). Dies wird wie folgt notiert:
\[
\begin{array}{l l l l}
    i & (k) & P \lxor Q & \rXOrIb{m} \\
\end{array}
\]

Hierbei gilt:
\begin{itemize}
    \item \(m < k\), wobei \(m\) die Zeile ist, in der \(P \land \neg Q\) bzw. \(\neg P \land Q\) abgeleitet wurde.
    \item Der Index \(i\) gibt die Liste der Annahmen an, die für die Ableitung in Zeile \(m\) verwendet wurden.
\end{itemize}

\end{definition}

\subsubsection{Eliminierung der exklusiven Disjunktion}
\label{rule:XOrE} 
Die exklusive Disjunktion \(\lxor\) erlaubt es, aus einer Aussage der Form \(P \lxor Q\) auf eine gemeinsame Konsequenz zu schließen, indem beide möglichen Fälle betrachtet werden: \(P\) wahr und \(Q\) falsch oder \(P\) falsch und \(Q\) wahr. Im Kalkül des natürlichen Schließens wird hierfür die Eliminierungsregel \(\lxor E\) definiert.

\begin{definition}[Eliminierungsregel der exklusiven Disjunktion]
Sei die Zeile \(m\) einer Beweistabelle wie folgt definiert:

\textbf{Zeile \(m\):}
\[
\begin{array}{l l l l}
    i & (m) & P \lxor Q & \dots \\
\end{array}
\]

Die Eliminierung der exklusiven Disjunktion erfolgt durch eine Fallunterscheidung mit zwei separaten Unterbeweistabellen, die wie folgt aufgebaut sind:

\paragraph{Fall 1: \(P \land \neg Q\) führt zu \(R\)}
\[
\begin{array}{l l l l}
    n_1 & (n_1) & P & \rA \\
    n_2 & (n_2) & \neg Q & \rA \\
    j,n_1,n_2 & (k_1) & R & \dots \\
\end{array}
\]

\paragraph{Fall 2: \(\neg P \land Q\) führt zu \(R\)}
\[
\begin{array}{l l l l}
    n_3 & (n_3) & \neg P & \rA \\
    n_4 & (n_4) & Q & \rA \\
    j,n_3,n_4 & (k_2) & R & \dots \\
\end{array}
\]

\textbf{Ableitung von \(R\):}
Nachdem beide Fälle betrachtet wurden, erlaubt die Regel \(\lxor E\) die Ableitung von \(R\) in einer neuen Zeile \(k\):
\[
\begin{array}{l l l l}
    i,j,k & (k) & R & \rXOrE{m,n_1,n_2,k_1,n_3,n_4,k_2} \\
\end{array}
\]

Hierbei gilt:
\begin{itemize}
    \item \(m < n_1, n_2 < k_1, k_1 < k\).
    \item \(m < n_3, n_4 < k_2, k_2 < k\).
    \item Die Indizes \(i, j, k\) geben die Listen der Annahmen an, die für die Ableitungen in den Zeilen \(m, k_1, k_2\) verwendet wurden.
    \item Weder die Liste \(j\) der Annahmen der Zeile \(k_1\) noch die Liste \(k\) der Annahmen der Zeile \(k_2\) enthalten die Annahmen \(n_1, n_2, n_3\) bzw. \(n_4\).
\end{itemize}

\end{definition}

\section{Abkürzungen und Definitionen}
\subsection{Der Eindeutigkeitsquantor}
\label{rule:UEI}\label{rule:UEE}

Der Eindeutigkeitsquantor, oft dargestellt durch das Symbol \(\exists!\), ist ein logischer Operator, der  "`es existiert genau ein"' bedeutet. Der Eindeutigkeitsquantor kann wie folgt definiert werden:
\begin{definition}[\(\exists !\)]
\label{ExonlyonexLpPLpxRpRpLrExxLpPLpxRpAndFayLpPLpyRpToxEqualsyRpRp}
\[
\exists! x(P(x)) \coloneqq \exists x(P(x) \land \forall y(P(y) \rightarrow x = y))
\]
\end{definition}

\paragraph{Einführungsregel für \(\exists!\) (\(\exists! I\))}
Die Einführungsregel ermöglicht die Ableitung einer Aussage der Form \(\exists! x(P(x))\), wenn neben der Existenz \(\exists x(P(x))\) auch nachgewiesen werden kann, dass, wenn wir ein neues \(a\) und \(b\) (beides frische Variablen, die in den bisherigen Schritten des Beweises noch nicht verwendet wurden) mit \(P(a)\) bzw. \(P(b)\) voraussetzen, gezeigt werden kann, dass \(a = b\) gilt. Dabei darf \(b\) in den Annahmen \(j\), aus denen \(a = b\) abgeleitet wird, nicht vorkommen.

\[
\begin{array}{llll}
    i      & (1) & \exists x P(x) & ...\\        
    2      & (2) & P(a)            & \rA\\
    3      & (3) & P(b)            &  \rA\\        
    2,3,j  & (4) & a = b           & ... \\
    i,j    & (5) & \exists! x P(x) & \UEI{1,2,3,4} \\
\end{array}
\]

\paragraph{Eliminierungsregel für \(\exists!\) (\(\exists! E\))}
Die Eliminierungsregel erlaubt die Ableitung einer Aussage \(Q\) aus einer Aussage der Form \(\exists! x(P(x))\), wenn wir \(P(a)\) und \(\forall y (P(y) \rightarrow a = y)\) als Annahmen nutzen können, um \(Q\) abzuleiten. Nachdem \(Q\) aus diesen Annahmen geschlussfolgert wurde, hängt \(Q\) nur noch von der ursprünglichen Eindeutigkeitsaussage \(\exists! x P(x)\) und weiteren Annahmen \(j\) ab, jedoch nicht mehr von den spezifischen Annahmen \(P(a)\) und \(\forall y (P(y) \rightarrow a = y)\).

\[
\begin{array}{llll}
    i      & (1) & \exists! x P(x)                                  & ...\\
    2      & (2) & P(a)                                             & \rA\\
    3      & (3) & \forall y (P(y) \rightarrow a = y)               & \rA\\
    2,3,j  & (4) & Q                                                & ...\\
    i,j    & (5) & Q                                                & \UEE{1,4}\\
\end{array}
\]

In diesem Beweisschema sind \(P(a)\) und \(\forall y (P(y) \rightarrow a = y)\) nur temporäre Annahmen, die für die Ableitung von \(Q\) in Zeile 4 genutzt werden. Sobald \(Q\) in Zeile 4 bewiesen ist, hängt \(Q\) nicht mehr von den Annahmen 2 und 3 ab, sondern nur noch von der ursprünglichen Aussage \(\exists! x P(x)\) in Zeile 1 und den weiteren Annahmen \(j\), aus denen \(Q\) geschlussfolgert wurde.

\paragraph{Herleitung der Regeln für \(\exists! x P(x)\)}
\begin{proof}
\(\UEI{}:\)
\[
\begin{array}{lll p{4cm}}
	i       & (1) & \exists x P(x)                             & ... \\		
	2       & (2) & P(a)                                        &  \rA \\
	3       & (3) & P(b)                                        & \rA \\		
	2,3,j   & (4) & a = b                                       & ...\\
        2,j     & (5) & P(b)\rightarrow a = b                       & \rRE{4,3} \\
        2,j     & (6) & \forall y(P(y)\rightarrow a = y)            & \rUI{5} \\
        2,j     & (7) & P(a)\land \forall y(P(y)\rightarrow a = y)  & \rAI{2,6} \\
        i,j     & (8) & P(a)\land \forall y(P(y)\rightarrow a = y)  & \rEE{1,2,7} \\
        i,j     & (9) & \exists x(P(x)\land \forall y(P(y)\rightarrow x = y)) & \rEI{8} \\
        i,j     & (10) & \exists! x P(x) & \ExonlyonexLpPLpxRpRpLrExxLpPLpxRpAndFayLpPLpyRpToxEqualsyRpRp{9}\\
\end{array}
\]
\(\UEE{}:\)
\[
\begin{array}{lll p{4cm}}
	i       & (1) & \exists! x P(x)                                & ... \\
	i       & (2) & \exists x(P(x) \wedge \forall y (P(y) \rightarrow x = y))                                            & \ExonlyonexLpPLpxRpRpLrExxLpPLpxRpAndFayLpPLpyRpToxEqualsyRpRp{2} \\	
 	3       & (3) & P(a) \land \forall y (P(y) \rightarrow a = y)                                            & \rA \\	
   	3       & (4) & P(a)                                            & \rAEa{3} \\	
        3       & (5) & \forall y (P(y) \rightarrow a = y)              & \rAEb{3} \\	
	3,j     & (6) & Q                                               & ...\\
 	3,j     & (7) & Q                                               & \rEE{2,3,6}\\
\end{array}
\]
\end{proof}
\begin{remark}
    In der dargestellten Eliminierungsregel für den Eindeutigkeitsquantor (\(\exists! E\)) werden die Aussagen \( P(a) \) und \( \forall y (P(y) \rightarrow a = y) \) als Annahmen in den Zeilen (2) und (3) formuliert. Diese Einzelaussagen resultieren aus der Anwendung der \(\exists! E\)-Regel auf die Aussage \( \exists x (P(x) \land \forall y (P(y) \rightarrow x = y)) \). Die Zerlegung in einzelne Komponenten ist hilfreich, um mit den Einzelaussagen flexibler arbeiten zu können, da es oft notwendig ist, diese getrennt voneinander in verschiedenen Ableitungen zu verwenden, anstatt die gesamte Aussage \( P(a) \land \forall y (P(y) \rightarrow a = y) \) zu behalten.
\end{remark}



\subsubsection{Behandlung mehrfacher Gleichheiten}
In der Praxis können Gleichheitsausdrücke mehrere Terme involvieren, wie beispielsweise \(a = b = c\). Solche Ausdrücke sind jedoch nicht direkt in der Prädikatenlogik darstellbar. Stattdessen werden sie durch eine Verkettung von binären Gleichheiten ausgedrückt, die mittels der Konjunktion \(\land\) verbunden sind. Der Ausdruck \(a = b = c\) wird somit zu \(a = b \land b = c\).

\begin{definition}
\[
\forall a, b, c \left( a = b = c \right) := \left( a = b \land b = c \right)
\]
\end{definition}

\paragraph{Einführungsregel für das Identitätssymbol dreier Gleichheiten:}
\label{rule:rIIb}

\[
\begin{array}{llll}
	i & (1) & a = b \land b = c & \dots \\
 	i & (2) & a = b = c & \rIIb{1} \\
\end{array}
\]

\[
\begin{array}{llll}
	i & (1) & a = b & \dots \\
        i & (2) & b = c & \dots \\
 	i,j & (3) & a = b = c & \rIIb{1,2} \\
\end{array}
\]

\[
\begin{array}{llll}
	i & (1) & a = b & \dots \\
        i & (2) & a = c & \dots \\
 	i,j & (3) & a = b = c & \rIIb{1,2} \\
\end{array}
\]

\(i\) und \(j\) sind dabei Listen von Annahmen.


\paragraph{Eliminierungsregel für das Identitätssymbol dreier Gleichheiten:}
\label{rule:rIEb}

\[
\begin{array}{llll}
	i & (1) & a = b = c & \dots \\
        i & (2) & a = b & \rIEb{1} \\
 	i & (3) & a = c & \rIEb{1} \\
        i & (4) & b = c & \rIEb{1} \\
\end{array}
\]

\(i\) ist dabei eine Liste von Annahmen.

\subsubsection{Direkte Anwendung von Gleichheitsregeln auf Prädikate}

Falls eine Regel oder ein Theorem mit der Bezeichnung \textit{RegelName} die Gleichheit \(a = b\) ausdrückt, gilt automatisch, dass für jedes Prädikat \(P(x)\), das entweder \(P(a)\) oder \(P(b)\) beschreibt, direkt geschlossen werden kann, dass \(P(b)\) aus \(P(a)\) folgt (und umgekehrt).

\paragraph{Formulierung:}
Falls eine Regel oder ein Theorem die Gleichheit \(a = b\) ausdrückt, so kann sie ebenso in folgendem Kontext verwendet werden:

\[
\begin{array}{llll}
	i & (1) & a = b & \text{\textit{RegelName}} \\
	i & (2) & P(a) & \dots \\
	i & (3) & P(b) & \text{\textit{RegelName}} \\
\end{array}
\]

Analog dazu:

\[
\begin{array}{llll}
	i & (1) & a = b & \text{\textit{RegelName}} \\
	i & (2) & P(b) & \dots \\
	i & (3) & P(a) & \text{\textit{RegelName}} \\
\end{array}
\]

\paragraph{Vorteile:}
- Die explizite Verwendung der Eliminierungsregel (\(\rIE{}\)) entfällt.
- Gleichheiten können direkt in Beweisen genutzt werden, wodurch Beweise kompakter und übersichtlicher werden.



\subsection{Regeln für das Nicht-Gleichheitszeichen}
\label{rule:NII} \label{rule:NIE}
Das Nicht-Gleichheitszeichen, dargestellt durch das Symbol \(\neq\), ist ein logischer Operator, der die Ungleichheit zweier Terme ausdrückt. Wir führen zwei grundlegende Regeln für das Nicht-Gleichheitszeichen ein: die Einführungs- und die Eliminierungsregel.

\[
\begin{array}{llll}
	i & (1) & \neg t = u & \text{...} \\
	i & (2) & t \neq u & \rNII{1} \\
\end{array}
\]

\[
\begin{array}{llll}
	i & (1) & t \neq u & \text{...} \\
	i & (2) & \neg t = u & \rNIE{1} \\
\end{array}
\]

\subsubsection{Integration von Definitionen in die Identitätsregeln}

Definitionen mittels des Definitionszeichens \( \coloneqq \) können als spezielle Identitätsaussagen betrachtet werden. Insbesondere:

\begin{itemize}
	\item \textbf{Nominaldefinitionen}: Definieren ein neues Symbol vollständig durch ein bestehendes Prädikat oder einen bestehenden Term.
	\item \textbf{Partielle Definitionen}: Definieren ein neues Symbol nur unter bestimmten Bedingungen.
\end{itemize}

Diese Definitionen lassen sich formal als Gleichungen darstellen, die dann in den Identitätsregeln verwendet werden können.

\paragraph{Behandlung von Nominaldefinitionen als Identitätsaussagen}


Eine Nominaldefinition wie \( Q(x) \coloneqq P(x) \) kann als die Gleichung \( Q(x) = P(x) \) interpretiert werden. Dadurch können die Regeln für das Identitätssymbol unmittelbar angewendet werden.

\paragraph{Behandlung von Partiellen Definitionen als bedingte Identitätsaussagen}

Eine partielle Definition wie 
\[
\forall x \, [ C(x) \rightarrow ( Q(x) \coloneqq P(x) ) ]
\]
kann als
\[
\forall x \, [ C(x) \rightarrow ( Q(x) = P(x) ) ]
\]
interpretiert werden. Somit wird die Eliminierungsregel nur angewendet, wenn die Bedingung \( C(x) \) erfüllt ist.

\paragraph{Regel für Nominaldefinitionen als Identitätsaussagen:}
\label{rule:IIN}

Nominaldefinitionen, die eine Gleichheit zwischen \(Q(x)\) und \(P(x)\) ausdrücken, können direkt in den Identitätsregeln verwendet werden:

\[
\begin{array}{llll}
	  & (1) & Q(x) = P(x) & \rIIN{df(Q)} \\
\end{array}
\]

\paragraph{Regel für partielle Definitionen als bedingte Identitätsaussagen:}
\label{rule:IIP}

 Wenn \(C(x)\) erfüllt ist, kann die Identitätsaussage \(Q(x) = P(x)\) hergeleitet werden. Hier wird die Identitätsregel auf partielle Definitionen angewendet, jedoch nur unter der Voraussetzung, dass die Bedingung \(C(x)\) gegeben ist.

\[
\begin{array}{llll}
	1 & (1) & C(x) & \rA{} \\
	1 & (2) & Q(x)=P(x) & \rIIP{1,df(Q)} \\
\end{array}
\]

Diese Regel für partielle Definitionen stellt sicher, dass die Identitätsaussage \(Q(x) = P(x)\) nur dann gilt, wenn die Bedingung \(C(x)\) zutrifft, und erweitert somit die Anwendbarkeit der Identitätsregeln auf komplexere, bedingte Definitionen.

\paragraph{Regel für partielle Definitionen unter einer notwendigen Bedingung:}
\label{rule:IEP}
Sei \(C(x)\) eine notwendige Voraussetzung, um \(Q(x)\) als Annahme treffen zu können. Die Aussage \(Q(x)\) ist nur unter der Bedingung \(C(x)\) definiert und darf daher nur angenommen werden, wenn \(C(x)\) erfüllt ist. Diese Voraussetzung gilt für den gesamten Beweisabschnitt und muss nicht explizit in jeder Beweiszeile wiederholt werden.

\vspace{0.5em}
\textbf{Voraussetzung:} Sei \(C(x)\) gegeben, welches benötigt wird, um \(Q(x)\) zu definieren.

\[
\begin{array}{llll}
	1 & (1) & Q(x) & \rA \\
	1 & (2) & P(x) & \rIEP{1,df(Q)} \\
\end{array}
\]

Diese Regel stellt sicher, dass die Aussage \(P(x)\) korrekt aus der Annahme \(Q(x)\) abgeleitet wird, aber nur unter der Voraussetzung, dass \(C(x)\) gilt. Die Bedingung \(C(x)\) ist notwendig, um \(Q(x)\) zu definieren, wird jedoch nicht explizit in jeder Beweiszeile verwendet, sondern als globale Bedingung vorausgesetzt.

\paragraph{Regel für Ableitungen aus impliziten Definitionen:}
\label{rule:ImpliziteDefinition}
Sei \(\sigma\) ein Symbol, das implizit durch eine Menge von Aussagen \(\Phi(\sigma)\) in einer Theorie \(\mathcal{T}\) definiert ist. Wenn \(\sigma\) gegeben ist, dürfen Aussagen, die aus \(\Phi(\sigma)\) folgen, im Beweis verwendet werden, ohne dass die gesamte Definition explizit in jeder Beweiszeile angegeben werden muss.

\vspace{0.5em}
\textbf{Voraussetzung:} Sei \(\sigma\) gegeben, dann gilt:

\[
\begin{array}{llll}
	  & (1) & \Phi(\sigma) & \text{Regelbezeichnung für } \Phi(\sigma) \\
\end{array}
\]

Diese Regel stellt sicher, dass Aussagen über \(\sigma\) korrekt aus den impliziten Eigenschaften von \(\Phi(\sigma)\) abgeleitet werden können, vorausgesetzt, \(\sigma\) ist implizit durch \(\Phi(\sigma)\) definiert. Die Definition von \(\Phi(\sigma)\) muss nicht in jeder Beweiszeile wiederholt werden, sondern wird als globale Grundlage für die Ableitungen betrachtet.





\subsection{Vereinfachung für das Nicht-Gleichheitszeichen}
Das Symbol \(\neq\), das die Ungleichheit von Elementen darstellt, ist in seiner Bedeutung offensichtlich und direkt verständlich. Es bedeutet einfach, dass zwei Elemente nicht gleich sind. Aufgrund dieser direkten und intuitiven Bedeutung können wir in vielen Fällen auf separate Einführungs- und Eliminierungsregeln für \(\neq\) verzichten.

Die Notation \(t \neq u\) ist äquivalent zu \(\neg(t = u)\) und kann direkt verwendet werden, ohne dass eine formale Regel angewendet werden muss. Dies vereinfacht den Umgang mit der Ungleichheit in logischen Argumentationen und Beweisen.

Daher können wir in Zukunft die Verwendung von \(t \neq u\) als eine unmittelbare Folgerung von \(\neg(t = u)\) und umgekehrt ansehen, ohne explizit eine Regel anzuwenden.



\chapter{Beweise in der Logik}

\section{Grundlegende Beweise}

\subsection{Äquivalenz als Folgerung}
\label{PLrQwPImpQ}
\begin{theorem}[$P \leftrightarrow Q, P \vdash Q$]
\end{theorem}
\begin{proof}
	\[
	\begin{array}{llll}
		1 & (1) & P \leftrightarrow Q & \rA \\
		2 & (2) & P & \rA \\
		1 & (3) & P\rightarrow Q & \rLREa{2} \\
		1,2 & (4) & Q & \rLREb{2,3} \\
	\end{array}
	\]
\end{proof}


\label{PLrQwQImpP}
\begin{theorem}[\( P \leftrightarrow Q, Q \vdash P \)]
\end{theorem}
\begin{proof}
	\[
	\begin{array}{llll}
		1 & (1) & P \leftrightarrow Q & \rA \\
		2 & (2) & Q & \rA \\
		1 & (3) & Q\rightarrow P & \rLREb{2} \\
		1,2 & (4) & Q & \rLREb{2,3} \\
	\end{array}
	\]
\end{proof}

\subsection{Idempotenzen}

\label{POrPEqvP}
\begin{theorem}[\( P \lor P \dashv\vdash P \)]
\end{theorem}
\begin{proof}
	\(\vdash:\)
	\[
	\begin{array}{llll}
		1 & (1) & P \lor P & \rA \\
		2 & (2) & P & \rA \\
		1 & (3) & P & \rOE{1,2,2,2,2} \\
	\end{array}
	\]
	\(\dashv:\)
	\[
	\begin{array}{llll}
		1 & (1) & P & \rA \\
		2 & (2) & P\lor P & \rOIa{1,1} \\
	\end{array}
	\]
\end{proof}

\label{PAndPEqvP}
\begin{theorem}[\( P \land P \dashv\vdash P \)]
\end{theorem}
\begin{proof}
	\(\vdash:\)
	\[
	\begin{array}{llll}
		1 & (1) & P \land P & \rA \\
		2 & (2) & P & \rAEa{1} \\
	\end{array}
	\]
	\(\dashv:\)
	\[
	\begin{array}{llll}
		1 & (1) & P & \rA \\
		2 & (2) & P\land P & \rAI{1,1} \\
	\end{array}
	\]
\end{proof}

\subsection{Kommutativgesetze}

\label{POrQImpQOrP}
\begin{theorem}[\( P \lor Q \vdash Q\lor P \)]
\end{theorem}
\begin{proof}
	\[
	\begin{array}{llll}
		1 & (1) & P \lor Q & \rA \\
		2 & (2) & P & \rA \\
		2 & (3) & Q\lor P & \rOIb{2} \\
		4 & (4) & Q & \rA \\
		4 & (5) & Q\lor P & \rOIa{4} \\
		1 & (6) & Q\lor P & \rOE{1,2,3,4,5} \\		
	\end{array}
	\]
\end{proof}



\label{PAndQImpQAndP}
\begin{theorem}[\( P \land Q \vdash Q\land P \)]
\end{theorem}
\begin{proof}
	\[
	\begin{array}{llll}
		1 & (1) & P \land Q & \rA \\
		1 & (2) & P & \rAEa{1} \\
		1 & (3) & Q & \rAEb{1} \\
		1 & (4) & Q\land P & \rAI{3,2} \\
	\end{array}
	\]
\end{proof}	


\label{PLrQImpQLrP}
\begin{theorem}[\( P \leftrightarrow Q \vdash Q\leftrightarrow P \)]
\end{theorem}
\begin{proof}
	\[
	\begin{array}{llll}
		1 & (1) & P\leftrightarrow Q & \rA \\
		1 & (2) & P\rightarrow Q & \rLREa{1} \\
		1 & (3) & Q\rightarrow P & \rLREb{1} \\
		1 & (4) & P\land Q & \rLRI{3,2} \\
	\end{array}
	\]
\end{proof}	

\subsection{Transitivitäten}

\label{PToQwQToRwPImpR}
\begin{theorem}[\(P\rightarrow Q, Q\rightarrow R, P\vdash R\)]
\end{theorem}
\begin{proof}
	\[
	\begin{array}{llll}
		1 & (1) & P \rightarrow Q & \rA \\
		2 & (2) & Q \rightarrow R & \rA \\
		3 & (3) & P & \rA \\
		1,3 & (4) & Q & \rRE{1,3} \\
		1,2,3 & (5) & R & \rRE{2,4} \\
	\end{array}
	\]
\end{proof}

\label{PToQwQToRImpPToR}
\begin{theorem}[\(P\rightarrow Q, Q\rightarrow R\vdash P \rightarrow R\) (Implikationstransitivität)]
\end{theorem}
\begin{proof}
	\[
	\begin{array}{llll}
		1 & (1) & P \rightarrow Q & \rA \\
		2 & (2) & Q \rightarrow R & \rA \\
		3 & (3) & P & \rA \\
		1,3 & (4) & R & \PToQwQToRwPImpR{1,2,3} \\
		1,2 & (5) & P \rightarrow R & \rRI{3,5} \\
	\end{array}
	\]
\end{proof}

\label{PLrQwQLrRImpPToR}
\begin{theorem}[\(P \leftrightarrow Q, Q \leftrightarrow R \vdash P \rightarrow R\)]
\end{theorem}
\begin{proof}
	\[
	\begin{array}{llll}
		1 & (1) & P \leftrightarrow Q & \rA \\
		2 & (2) & Q \leftrightarrow R & \rA \\
		1 & (3) & P \rightarrow Q & \rLREa{1} \\
		2 & (4) & Q \rightarrow R & \rLREa{2} \\
		1,2 & (5) & P \rightarrow R & \PToQwQToRImpPToR{3,4} \\
	\end{array}
	\]
\end{proof}

\label{PLrQwQLrRImpRToP}
\begin{theorem}[\(P \leftrightarrow Q, Q \leftrightarrow R \vdash R \rightarrow P\)]
\end{theorem}
\begin{proof}
	\[
	\begin{array}{llll}
		1 & (1) & P \leftrightarrow Q & \rA \\
		2 & (2) & Q \leftrightarrow R & \rA \\
		1 & (3) & Q \rightarrow P & \rLREb{1} \\
		2 & (4) & R \rightarrow Q & \rLREb{2} \\
		1,2 & (5) & R \rightarrow P & \PToQwQToRImpPToR{4,3} \\
	\end{array}
	\]
\end{proof}

\label{PLrQwQLrRImpPLrR}
\begin{theorem}[\(P \leftrightarrow Q, Q \leftrightarrow R \vdash P \leftrightarrow R\) (Äquivalenztransitivität)]
\end{theorem}
\begin{proof}
	\[
	\begin{array}{llll}
		1   & (1) & P \leftrightarrow Q & \rA \\
		2   & (2) & Q \leftrightarrow R & \rA \\
		1,2 & (3) & P \rightarrow R & \PLrQwQLrRImpPToR{1,2} \\
		1,2 & (4) & R \rightarrow P & \PLrQwQLrRImpRToP{1,2} \\
		1,2 & (5) & P \leftrightarrow R & \rLRI{3,4} \\
	\end{array}
	\]
\end{proof}

\label{PLrQwRLrQImpPLrR}
\begin{theorem}[\(P \leftrightarrow Q, R \leftrightarrow Q \vdash P \leftrightarrow R\) (Äquivalenztransitivität)]
\end{theorem}
\begin{proof}
	\[
	\begin{array}{llll}
		1 & (1) & P \leftrightarrow Q & \rA \\
		2 & (2) & R \leftrightarrow Q & \rA \\
            2 & (3) & Q \leftrightarrow R & \PLrQImpQLrP{2} \\
            2 & (4) & P \leftrightarrow R & \PLrQwQLrRImpPLrR{3} \\
	\end{array}
	\]
\end{proof}

\subsection{Assoziativgesetze}

\label{POrLpQOrRRpImpLpPOrQRpOrQ}
\begin{theorem}[\( (P \lor (Q \lor R)) \dashv\vdash ((P \lor Q) \lor R) \)]
\end{theorem}	
\begin{proof}
	\(\vdash:\)
	\[
	\begin{array}{llll}
		1 & (1) & P \lor (Q \lor R) & \rA \\
		2 & (2) & P & \rA \\
		2 & (3) & P \lor Q & \rOIa{2} \\
		2 & (4) & (P \lor Q) \lor R & \rOIa{3}  \\
		5 & (5) & Q \lor R & \rA \\
		5 & (6) & Q & \rA \\
		6 & (7) & P\lor Q & \rOIb{6}  \\
		6 & (8) & (P\lor Q)\lor R & \rOIa{7}  \\
		9 & (9) & R & \rA \\
		9 & (10) & (P\lor Q)\lor R & \rOIb{9} \\
		5 & (11) & (P\lor Q)\lor R & \rOE{5,6,8,9,10}\\	
		1 & (12) & (P\lor Q)\lor R & \rOE{1,2,4,5,11}\\	
	\end{array}
	\]
	\(\dashv:\)
	\[
	\begin{array}{llll}
		1 & (1) & (P \lor Q) \lor R & \rA \\
		2 & (2) & P \lor Q & \rA \\
		2 & (3) & P & \rA \\
		3 & (4) & P \lor (Q \lor R) & \rOIa{3} \\
		5 & (5) & Q & \rA \\
		5 & (6) & Q \lor R & \rOIa{5} \\
		5 & (7) & P \lor (Q \lor R) & \rOIb{6} \\
		2 & (8) & P \lor (Q \lor R) & \rOE{2,3,4,5,7} \\
		9 & (9) & R & \rA \\
		9 & (10) & Q \lor R & \rOIb{9} \\
		9 & (11) & P \lor (Q \lor R) & \rOIb{10} \\
		1 & (12) & P \lor (Q \lor R) & \rOE{1,2,8,9,11} \\
	\end{array}
	\]
\end{proof}


\label{PAndLpQAndRRpImpLpPAndQRpAndQ}
\begin{theorem}[\( (P \land (Q \land R)) \dashv\vdash ((P \land Q) \land R) \)]
\end{theorem}

\begin{proof}
	\(\vdash:\)
	\[
	\begin{array}{llll}
		1 & (1) & P \land (Q \land R) & \rA \\
		1 & (2) & P & \rAEa{1} \\
		1 & (3) & Q \land R & \rAEb{1} \\
		1 & (4) & Q & \rAEa{3}  \\
		1 & (5) & R & \rAEb{3}  \\
		1 & (6) & P \land Q & \rAI{2,4}  \\
		1 & (7) & (P \land Q) \land R & \rAI{6,5} \\
	\end{array}
	\]
	\(\dashv:\)
	\[
	\begin{array}{llll}
		1 & (1) & (P \land Q) \land R & \rA \\
		1 & (2) & P \land Q & \rAEa{1} \\
		1 & (3) & R & \rAEb{1} \\
		1 & (4) & P & \rAEa{2}\\
		1 & (5) & Q & \rAEb{2} \\
		1 & (6) & Q \land R & \rAI{5,3} \\
		1 & (7) & P \land (Q \land R) & \rAI{4,6}\\
	\end{array}
	\]
\end{proof}

\label{PAndLpQAndRRpRpImpQ}
\begin{theorem}[\( (P \land (Q \land R))\vdash Q \)]
\end{theorem}
\begin{proof}
	\[
	\begin{array}{llll}
		1 & (1) & P \land (Q \land R) & \rA \\
		1 & (2) & Q\land R & \rAEb{1} \\
		1 & (3) & Q  & \rAEa{1} \\
	\end{array}
	\]
\end{proof}

\label{LpPAndLpQAndRRpRpImpR}
\begin{theorem}[\( (P \land (Q \land R))\vdash R \)]
\end{theorem}
\begin{proof}
	\[
	\begin{array}{llll}
		1 & (1) & P \land (Q \land R) & \rA \\
		1 & (2) & Q\land R & \rAEb{1} \\
		1 & (3) & R  & \rAEb{1} \\
	\end{array}
	\]
\end{proof}

\label{LpLpPAndQRpAndRRpImpP}
\begin{theorem}[\( ((P \land Q) \land R)\vdash P \)]
\end{theorem}
\begin{proof}
	\[
	\begin{array}{llll}
		1 & (1) & (P \land Q) \land R & \rA \\
		1 & (2) & P\land Q & \rAEa{1} \\
		1 & (3) & P  & \rAEa{1} \\
	\end{array}
	\]
\end{proof}

\label{LpLpPAndQRpAndRRpImpQ}
\begin{theorem}[\( ((P \land Q) \land R)\vdash Q \)]
\end{theorem}
\begin{proof}
	\[
	\begin{array}{llll}
		1 & (1) & (P \land Q) \land R & \rA \\
		1 & (2) & P\land Q & \rAEa{1} \\
		1 & (3) & Q  & \rAEb{1} \\
	\end{array}
	\]
\end{proof}


\label{PToQwnQImpnP}
\begin{theorem}[\(P \rightarrow Q, \neg Q \vdash \neg P\) (Modus Tollens)]
\end{theorem}
\begin{proof}
	\[
	\begin{array}{llll}
		1 & (1) & P \rightarrow Q & \rA \\
		2 & (2) & \neg Q & \rA \\
		3 & (3) & P & \rA \\
		1,3 & (4) & Q & \rRE{1,3} \\
		1,2,3 & (5) & \bot & \rBI{2,4} \\
		1,2 & (6) & \neg P & \rCI{3,5}\\
	\end{array}
	\]
\end{proof}

\label{nPTonQwQImpP}
\begin{theorem}[\(\neg P \rightarrow \neg Q, Q \vdash P\)]
\end{theorem}
\begin{proof}		
	\[
	\begin{array}{llll}
		1 & (1) & \neg P \rightarrow \neg Q & \rA \\
		2 & (2) & Q & \rA \\
		3 & (3) & \neg P & \rA \\
		1,3 & (4) & \neg Q & \rRE{1,3} \\
		1,2,3 & (5) & \bot & \rBI{2,4} \\
		1,2 & (6) & P & \rCE{3,5} \\
	\end{array}
	\]
\end{proof}


\label{PLrQwnQImpnP}
\begin{theorem}[\(P \leftrightarrow Q, \neg Q \vdash \neg P\)]
\end{theorem}
\begin{proof}		
	\[
	\begin{array}{llll}
		1 & (1) & P \leftrightarrow Q &\rA \\
		1 & (2) & P\rightarrow Q & \rLREa{1} \\
		3 & (3) & \neg Q & \rA \\
		1,3 & (4) & \neg P & \PToQwnQImpnP{2,3} \\
	\end{array}
	\]
\end{proof}

\label{PLrQwnPImpnQ}
\begin{theorem}[\(P \leftrightarrow Q, \neg P \vdash \neg Q\)]
\end{theorem}
\begin{proof}		
	\[
	\begin{array}{llll}
		1 & (1) & P \leftrightarrow Q & \rA \\
		1 & (2) & Q\rightarrow P & \rLREb{1} \\
		3 & (3) & \neg P & \rA \\
		1,3 & (4) & \neg Q & \PToQwnQImpnP{2,3} \\
	\end{array}
	\]
\end{proof}


\label{nPLrnQwQImpP}
\begin{theorem}[\(\neg P \leftrightarrow \neg Q, Q \vdash P\)]
\end{theorem}
\begin{proof}		
	\[
	\begin{array}{llll}
		1 & (1) & \neg P \leftrightarrow \neg Q & \rA \\
		1 & (2) & \neg P\rightarrow \neg Q & \rLREa{1} \\
		3 & (3) & Q & \rA \\
		1,3 & (4) & P & \nPTonQwQImpP{2,3} \\
	\end{array}
	\]
\end{proof}


\label{nPLrnQwPImpQ}
\begin{theorem}[\(\neg P \leftrightarrow \neg Q, P \vdash Q\)]
\end{theorem}
\begin{proof}		
	\[
	\begin{array}{llll}
		1 & (1) & \neg P \leftrightarrow \neg Q & \rA \\
		1 & (2) & \neg Q\rightarrow \neg P & \rLREb{1} \\
		3 & (3) & P & \rA \\
		1,3 & (4) & Q & \nPTonQwQImpP{2,3} \\
	\end{array}
	\]
\end{proof}


\label{PToQEqvnQTonP}
\begin{theorem}[\( P \rightarrow Q \dashv\vdash \neg Q \rightarrow \neg P \) (Kontraposition)]
\end{theorem}
\begin{proof}
	\(\vdash:\)
	\[
	\begin{array}{llll}
		1 & (1) & P \rightarrow Q & \rA \\
		2 & (2) & \neg Q & \rA \\
		1,2 & (3) & \neg P & \PToQwnQImpnP{1,2} \\
		1 & (4) & \neg Q\rightarrow \neg P & \rRI{2,3} \\
	\end{array}		
	\]
	
	\(\dashv:\)
	\[
	\begin{array}{llll}
		1 & (1) & \neg P \rightarrow \neg Q & \rA \\
		2 & (2) & Q & \rA \\
		1,2 & (3) & P & \nPTonQwQImpP{1,2} \\
		1 & (4) & Q\rightarrow P & \rRI{2,3} \\
	\end{array}		
	\]
\end{proof}


\label{LpPAndQRpToREqvPToLpQToRRp}
\begin{theorem}[\( (P \land Q) \rightarrow R \dashv\vdash P\rightarrow (Q\rightarrow R) \) (Exportation)]
\end{theorem}
\begin{proof}
	\(\vdash:\)
	\[
	\begin{array}{llll}
		1 & (1) & (P \land Q) \rightarrow R & \rA \\
		2 & (2) & P & \rA \\
		3 & (3) & Q & \rA \\
		2,3 & (4) & P\land Q & \rAI{2,3} \\
		1,2,3 & (5) & R & \rRE{1,4} \\
		1,2 & (6) & Q\rightarrow R & \rRI{3,5} \\
		1 & (7) & P\rightarrow (Q\rightarrow R) & \rRE{2,6} \\
	\end{array}		
	\]
	
	\(\dashv:\)
	\[
	\begin{array}{llll}
		1 & (1) & P\rightarrow (Q\rightarrow R) & \rA \\
		2 & (2) & P\land Q & \rA \\
		2 & (3) & P & \rAEa{2} \\
		2 & (4) & Q & \rAEb{2} \\
		1,2 & (5) & Q\rightarrow R & \rRE{1,3} \\
		1,2 & (6) & R & \rRE{5,4} \\
		1 & (7) & (P\land Q)\rightarrow R & \rRI{2,6} \\
	\end{array}		
	\]
\end{proof}

\label{LpPAndQRpToREqvQToLpPToRRp}
\begin{theorem}[\( (P \land Q) \rightarrow R \dashv\vdash Q\rightarrow (P\rightarrow R) \) (Exportation)]
\end{theorem}
\begin{proof}
	\(\vdash:\)
	\[
	\begin{array}{llll}
		1 & (1) & (P \land Q) \rightarrow R & \rA \\
		2 & (2) & P & \rA \\
		3 & (3) & Q & \rA \\
		2,3 & (4) & P\land Q & \rAI{2,3} \\
		1,2,3 & (5) & R & \rRE{1,4} \\
		1,3 & (6) & P\rightarrow R & \rRI{2,5} \\
		1 & (7) & Q\rightarrow (P\rightarrow R) & \rRE{3,6} \\
	\end{array}		
	\]
	
	\(\dashv:\)
	\[
	\begin{array}{llll}
		1 & (1) & Q\rightarrow (P\rightarrow R) & \rA \\
		2 & (2) & P\land Q & \rA \\
		2 & (3) & P & \rAEa{2} \\
		2 & (4) & Q & \rAEb{2} \\
		1,2 & (5) & P\rightarrow R & \rRE{1,4} \\
		1,2 & (6) & R & \rRE{5,3} \\
		1 & (7) & (P\land Q)\rightarrow R & \rRI{2,6} \\
	\end{array}		
	\]
\end{proof}


\label{PToLpQToRRpImpQToLpPToRRp}
\begin{theorem}[\(P\rightarrow (Q\rightarrow R)\vdash  Q\rightarrow (P\rightarrow R)\)]
\end{theorem}
\begin{proof}
	\(\vdash:\)
	\[
	\begin{array}{llll}
		1 & (1) & P\rightarrow (Q\rightarrow R) & \rA \\
		1 & (2) & (P \land Q) \rightarrow R & \LpPAndQRpToREqvPToLpQToRRp{1} \\
		1 & (3) & Q\rightarrow (P\rightarrow R) & \LpPAndQRpToREqvQToLpPToRRp{2} \\
	\end{array}		
	\]
\end{proof}

\label{PLrLpQAndRRpwQToRImpPLrQ}
\begin{theorem}[\(P\leftrightarrow (Q\land R), Q\rightarrow R\vdash P\leftrightarrow Q\)]
\end{theorem}
\begin{proof}
	\[
	\begin{array}{llll}
		1 & (1) & P\leftrightarrow (Q\land R) & \rA \\
		2 & (2) & Q\rightarrow R & \rA \\
		3 & (3) & P & \rA \\
		1,3 & (4) & Q\land R & \PLrQwPImpQ[1,3] \\
		1,3 & (5) & Q & \rAEa{4} \\
		1 & (6) & P\rightarrow Q & \rRI{3,5} \\
		7 & (7) & Q & \rA \\
		2,7 & (8) & R & \rRE{2,7} \\
		2,7 & (9) & Q\land R & \rAI{7,8} \\
		1,2,7 & (10) & P & \PLrQwPImpQ{1,9} \\
		1,2 & (11) & Q\rightarrow P & \rRI{7,10} \\
		1,2 & (12) & P\leftrightarrow Q & \rLRI{11,6} \\
	\end{array}		
	\]
\end{proof}


\label{rule:DN}
\begin{theorem}[\(P\dashv\vdash \neg\neg P\) (Regel der doppelten Negation)]
\end{theorem}
\begin{proof}
	\(\vdash:\)
	\[
	\begin{array}{llll}
		1 & (1) & P & \rA \\
		2 & (2) & \neg P & \rA \\
		1,2 & (3) & \bot & \rBI{1,2} \\
		1 & (4) & \neg\neg P & \rCI{2,3} \\
	\end{array}
	\]
	\(\dashv:\)
	\[
	\begin{array}{llll}
		1 & (1) & \neg\neg P & \rA \\
		2 & (2) & \neg P & \rA \\
		1,2 & (3) & \bot & \rBI{2,1} \\
		1 & (4) & P & \rCE{2,3} \\
	\end{array}
	\]
\end{proof}

\label{POrQEqvnnPOrnnQ}
\begin{theorem}[\(P\lor Q \dashv\vdash \neg\neg P\lor \neg\neg Q\)]
\end{theorem}
\begin{proof}
	\(\vdash:\)
	\[
	\begin{array}{llll}
		1 & (1) & P\lor Q & \rA \\
		2 & (2) & P & \rA \\
		2 & (3) & \neg\neg P & \rDN{2} \\
		2 & (4) & \neg\neg P\lor \neg\neg Q & \rOIa{3} \\
		5 & (5) &  Q & \rA \\
        5 & (6) & \neg\neg Q & \rDN{5} \\
	    5 & (7) & \neg\neg P\lor \neg\neg Q & \rOIb{6} \\
        1 & (8) & \neg\neg P\lor \neg\neg Q & \rOE{1,2,4,5,7} \\
    \end{array}
	\]
	\(\dashv:\)
	\[
	\begin{array}{llll}
		1 & (1) & \neg\neg P\lor \neg\neg Q & \rA \\
		2 & (2) & \neg\neg P & \rA \\
		2 & (3) & P & \rDN{2} \\
		2 & (4) & P\lor Q & \rOIa{3} \\
		5 & (5) & \neg\neg Q & \rA \\
        5 & (6) & Q & \rDN{5} \\
	    5 & (7) & P\lor Q & \rOIb{6} \\
        1 & (8) &  P\lor Q & \rOE{1,2,4,5,7} \	
    \end{array}
	\]
\end{proof}


\label{PAndQEqvnnPAndnnQ}
\begin{theorem}[\(P\land Q \dashv\vdash \neg\neg P\land \neg\neg Q\)]
\end{theorem}
\begin{proof}
	\(\vdash:\)
	\[
	\begin{array}{llll}
		1 & (1) & P\land Q & \rA \\
		1 & (2) & P & \rAEa{1}  \\
  		1 & (3) & Q & \rAEb{1}  \\
		1 & (4) & \neg\neg P & \rDN{2}  \\
		1 & (5) & \neg\neg Q & \rDN{3}  \\
		1 & (6) & \neg\neg P\land \neg\neg Q & \rAI{4,5}  \\
    \end{array}
	\]
	\(\dashv:\)
	\[
	\begin{array}{llll}
		1 & (1) & \neg\neg P\land \neg\neg Q & \rA \\
		1 & (2) & \neg\neg P & \rAEa{1}  \\
  		1 & (3) & \neg\neg Q & \rAEb{1}  \\
		1 & (4) & P & \rDN{2}  \\
		1 & (5) & Q & \rDN{3}  \\
		1 & (6) & P\land Q & \rAI{4,5}  \\
    \end{array}
	\]
\end{proof}



\label{ImpPOrnP}
\begin{theorem}[\(\vdash P \lor \neg P\) (Gesetz vom ausgeschlossenen dritten)]
\end{theorem}
\begin{proof}
	\[
	\begin{array}{llll}
		1 & (1) & \neg(P \lor \neg P) & \rA \\
		2 & (2) & P & \rA \\
		2 & (3) & P \lor \neg P & \rOIa{2} \\
		1,2 & (4) & \bot & \rBI{1,3} \\
		1 & (5) & \neg P & \rCI{2,4} \\
		5 & (6) & P \lor \neg P & \rOIb{5}\\
		1,5 & (7) & \bot & \rBI{1,6} \\
		& (8) & P \lor \neg P & \rCE{1,7} \\
	\end{array}
	\]
\end{proof}

\subsection{De-morgansche Gesetze}

\label{nLpPOrQRpEqvnPAndnQ}
\begin{theorem}[\(\neg(P \lor Q) \dashv \vdash \neg P \land \neg Q\) (De Morgan)]
\end{theorem}
\begin{proof}
	\(\vdash:\)
	\[
	\begin{array}{llll}
		1 & (1) & \neg(P \lor Q) & \rA \\
		2 & (2) & P & \rA \\
		2 & (3) & P \lor Q & \rOIa{2} \\
		1,2 & (4) & \bot & \rBI{1,3} \\
		1 & (5) & \neg P & \rCI{2,4} \\
		6 & (6) & Q & \rA \\
		6 & (7) & P \lor Q & \rOIb{6} \\
		1,6 & (8) & \bot & \rBI{1,7} \\
		1 & (9) & \neg Q & \rCI{6,8} \\
		1 & (10) & \neg P \land \neg Q & \rAI{5,9} \\
	\end{array}
	\]
	
	\(\dashv:\)
	\[
	\begin{array}{llll}
		1 & (1) & \neg P \land \neg Q & \rA \\
		1 & (2) & \neg P & \rAEa{1} \\
		1 & (3) & \neg Q & \rAEb{1} \\
		4 & (4) & P \lor Q & \rA \\
		5 & (5) & P & \rA \\
		1,5 & (6) & \bot & \rBI{2,5} \\
		1 & (7) & \neg (P \lor Q) & \rCI{4,6} \\
		8 & (8) & Q & \rA \\
		1,8 & (9) & \bot & \rBI{3,8} \\
		1 & (10) & \neg (P \lor Q) & \rCI{4,9} \\
		1 & (11) & \neg (P \lor Q) & \rOE{4,5,7,8,10} \\
	\end{array}
	\]
\end{proof}

\label{nLpPAndQRpEqvnPOrnQ}
\begin{theorem}[\(\neg(P \land Q) \dashv \vdash \neg P \lor \neg Q\) (De Morgan 2)]
\end{theorem}
\begin{proof}
	\(\vdash:\)
	\[
	\begin{array}{llll}
		1 & (1) & \neg(P \land Q) & \rA \\
		& (2) & P \lor \neg P & \ImpPOrnP \\
		3 & (3) & P & \rA \\
		4 & (4) & Q & \rA \\
		3,4 & (5) & P \land Q & \rAI{3,4} \\
		1,3,4 & (6) & \bot & \rBI{1,5} \\
		1,3 & (7) & \neg Q & \rCI{4,6} \\
		1,3 & (8) & \neg P \lor \neg Q & \rOIb{7} \\
		9 & (9) & \neg P & \rA \\
		9 & (10) & \neg P \lor \neg Q & \rOIa{9} \\
		1 & (11) & \neg P \lor \neg Q & \rOE{2,3,8,9,10} 
	\end{array}
	\]
	\(\dashv:\)
	\[
	\begin{array}{llll}
		1 & (1) & \neg P \lor \neg Q & \rA \\
		2 & (2) & P \land Q & \rA \\
		2 & (3) & P & \rAEa{2} \\
		2 & (4) & Q & \rAEb{2} \\
		5 & (5) & \neg P & \rA \\
		2,5 & (6) & \bot & \rBI{3,5} \\
		5 & (7) & \neg (P \land Q) & \rCI{2,6} \\
		8 & (8) & \neg Q & \rA \\
		2,8 & (9) & \bot & \rBI{4,8} \\
		8 & (10) & \neg (P \land Q) & \rCI{2,9} \\
		1 & (11) & \neg (P \land Q) & \rOE{1,5,7,8,10} 
	\end{array}
	\]
\end{proof}

\label{nLpnPAndnQRpEqvPOrQ}
\begin{theorem}[\(\neg(\neg P \land \neg Q) \dashv \vdash P \lor Q\)]
\end{theorem}
\begin{proof}
	\(\vdash:\)
	\[
	\begin{array}{llll}
		1 & (1) & \neg(\neg P \land \neg Q) & \rA \\
		1 & (2) & \neg\neg P \lor \neg\neg Q & \nLpPAndQRpEqvnPOrnQ{1} \\
		1 & (3) & P\lor Q & \POrQEqvnnPOrnnQ{2} \\
	\end{array}
	\]
	\(\dashv:\)
	\[
	\begin{array}{llll}
		1 & (1) & P \lor Q & \rA \\
		1 & (2) & \neg\neg P \lor \neg\neg Q & \POrQEqvnnPOrnnQ{1} \\
        1 & (3) & \neg(\neg P \land \neg Q) & \nLpPAndQRpEqvnPOrnQ{2} \\
	\end{array}
	\]
\end{proof}

\label{nLpnPOrnQRpEqvPAndQ}
\begin{theorem}[\(\neg(\neg P \lor \neg Q) \dashv \vdash P \land Q\)]
\end{theorem}
\begin{proof}
	\(\vdash:\)
	\[
	\begin{array}{llll}
		1 & (1) & \neg(\neg P \lor \neg Q) & \rA \\
		1 & (2) & \neg\neg P \land \neg\neg Q & \nLpPOrQRpEqvnPAndnQ{1} \\
		1 & (3) & P\land Q & \PAndQEqvnnPAndnnQ{2} \\
	\end{array}
	\]
	\(\dashv:\)
	\[
	\begin{array}{llll}
		1 & (1) & P \land Q & \rA \\
		1 & (2) & \neg\neg P \land \neg\neg Q & \PAndQEqvnnPAndnnQ{1} \\
        1 & (3) & \neg(\neg P \lor \neg Q) & \nLpPOrQRpEqvnPAndnQ{2} \\
	\end{array}
	\]
\end{proof}

\subsection{Distributivgesetze}

\label{PAndLpQOrRRpEqvLpPAndQRpOrLpPAndRRp}
\begin{theorem}[\(P \land (Q\lor R) \dashv \vdash  (P \land Q) \lor (P\land R)\) (Distributivgesetz)]
\end{theorem}
\begin{proof}
	\(\vdash:\)
	\[
	\begin{array}{llll}
		1 & (1) & P \land (Q\lor R) & \rA \\
		1 & (2) & P & \rAEa{1} \\
		1 & (3) & Q\lor R & \rAEb{1} \\
		4 & (4) & Q & \rA \\
		1,4 & (5) & P \land Q & \rAI{2,4} \\
		1,4 & (6) & (P \land Q) \lor (P\land R) & \rOIa{5} \\
		7 & (7) & R & \rA \\
		1,7 & (8) & P \land R & \rAI{2,7} \\
		1,7 & (9) & (P \land Q) \lor (P\land R) & \rOIb{8} \\
		1 & (10) & (P \land Q) \lor (P\land R) & \rOE{3,4,6,7,9} \\
	\end{array}
	\]
	\(\dashv:\)
	\[
	\begin{array}{llll}
		1 & (1) & (P \land Q) \lor (P\land R) & \rA \\
		2 & (2) & P \land Q & \rA \\
		2 & (3) & P & \rAEa{2} \\
		2 & (4) & Q & \rAEb{2} \\
		2 & (5) & Q \lor R & \rOIa{4} \\
		2 & (6) & P \land (Q \lor R) & \rAI{3,5} \\
		7 & (7) & P \land R & \rA \\
		7 & (8) & P & \rAEa{7} \\
		7 & (9) & R & \rAEb{7} \\
		7 & (10) & Q \lor R & \rOIb{9} \\
		7 & (11) & P \land (Q \lor R) & \rAI{8,10} \\
		1 & (12) & P \land (Q \lor R) & \rOE{1,2,6,7,11} 
	\end{array}
	\]
\end{proof}

\label{LpPOrQRpAndREqvLpPAndRRpOrLpQAndRRp}
\begin{theorem}[\((P\lor Q)\land R \dashv \vdash  (P \land R) \lor (Q\land R)\) (Distributivgesetz)]
\end{theorem}
\begin{proof}
	\(\vdash:\)
	\[
	\begin{array}{llll}
		1 & (1) & (P\lor Q)\land R & \rA \\
		1 & (2) & R & \rAEb{1} \\
		1 & (3) & P\lor Q & \rAEa{1} \\
		4 & (4) & P & \rA \\
		1,4 & (5) & P \land R & \rAI{4,2} \\
		1,4 & (6) & (P \land R) \lor (Q\land R) & \rOIa{5} \\
		7 & (7) & Q & \rA \\
		1,7 & (8) & Q \land R & \rAI{7,2} \\
		1,7 & (9) & (P \land Q) \lor (Q\land R) & \rOIb{8} \\
		1 & (10) & (P \land Q) \lor (P\land R) & \rOE{3,4,6,7,9} \\
	\end{array}
	\]
	
	\(\dashv:\)
	
	\[
	\begin{array}{llll}
		1 & (1) & (P \land R) \lor (Q\land R) & \rA \\
		2 & (2) & P \land R & \rA \\
		2 & (3) & P & \rAEa{2} \\
		2 & (4) & R & \rAEb{2} \\
		2 & (5) & P \lor Q & \rOIa{3} \\
		2 & (6) & (P \lor Q)\land R & \rAI{5,4} \\
		7 & (7) & Q \land R & \rA \\
		7 & (8) & Q & \rAEa{7} \\
		7 & (9) & R & \rAEb{7} \\
		7 & (10) & P \lor Q & \rOIb{8} \\
		7 & (11) & (P \lor Q)\land R & \rAI{10,9} \\
		1 & (12) & P \land (Q \lor R) & \rOE{1,2,6,7,11} 
	\end{array}
	\]
\end{proof}

\label{POrLpQAndRRpEqvLpPOrQRpAndLpPOrRRp}
\begin{theorem}[\(P \lor (Q\land R) \dashv \vdash  (P \lor Q) \land (P\lor R)\) (Distributivgesetz)]
\end{theorem}

\begin{proof}
	\(\vdash:\)
	\[
	\begin{array}{llll}
		1 & (1) & P \lor (Q \land R) & \rA \\
		2 & (2) & P & \rA \\
		2 & (3) & P \lor Q & \rOIa{2} \\
		2 & (4) & P \lor R & \rOIa{2} \\
		2 & (5) & (P \lor Q) \land (P \lor R) & \rAI{3,4} \\
		6 & (6) & Q \land R & \rA \\
		6 & (7) & Q & \rAEa{6} \\
		6 & (8) & R & \rAEb{6} \\
		6 & (9) & P \lor Q & \rOIb{7} \\
		6 & (10) & P \lor R & \rOIb{8} \\
		6 & (11) & (P \lor Q) \land (P \lor R) & \rAI{9,10} \\
		1 & (12) & (P \lor Q) \land (P \lor R) & \rOE{1,2,5,6,11} \\
	\end{array}
	\]
	
	\(\dashv:\)
	
	\[
	\begin{array}{llll}
		1 & (1) & (P \lor Q) \land (P \lor R) & \rA \\
		1 & (2) & P \lor Q & \rAEa{1} \\
		1 & (3) & P \lor R & \rAEb{1} \\
		4 & (4) & P & \rA \\
		4 & (5) & P \lor (Q \land R) & \rOIa{4} \\
		6 & (6) & Q & \rA \\
		7 & (7) & R & \rA \\
		6,7 & (8) & Q \land R & \rAI{6,7} \\
		6,7 & (9) & P \lor (Q \land R) & \rOIb{8} \\
		1,7 & (10) & P \lor (Q \land R) & \rOE{2,4,5,6,9} \\
		1 & (11) & P \lor (Q \land R) & \rOE{3,4,5,7,10} \\
	\end{array}
	\]
\end{proof}

\label{LpPAndQRpOrREqvLpPOrRRpAndLpQOrRRp}
\begin{theorem}[\((P\land Q)\lor R \dashv \vdash  (P \lor R) \land (Q\lor R)\) (Distributivgesetz)]
\end{theorem}

\begin{proof}
	\(\vdash:\)
	\[
	\begin{array}{llll}
		1 & (1) & (P \land Q)\lor R & \rA \\
		2 & (2) & R & \rA \\
		2 & (3) & P \lor R & \rOIb{2} \\
		2 & (4) & Q \lor R & \rOIb{2} \\
		2 & (5) & (P \lor R) \land (Q \lor R) & \rAI{3,4} \\
		6 & (6) & P \land Q & \rA \\
		6 & (7) & P & \rAEa{6} \\
		6 & (8) & Q & \rAEb{6} \\
		6 & (9) & P \lor R & \rOIa{7} \\
		6 & (10) & Q \lor R & \rOIa{8} \\
		6 & (11) & (P \lor Q) \land (P \lor R) & \rAI{9,10} \\
		1 & (12) & (P \lor Q) \land (P \lor R) & \rOE{1,2,5,6,11} \\
	\end{array}
	\]
	
	\(\dashv:\)
	
	\[
	\begin{array}{llll}
		1 & (1) & (P \lor R) \land (Q \lor R) & \rA \\
		1 & (2) & P \lor R & \rAEa{1} \\
		1 & (3) & Q \lor R & \rAEb{1} \\
		4 & (4) & R & \rA \\
		4 & (5) & (P \land Q)\lor R & \rOIb{4} \\
		6 & (6) & P & \rA \\
		7 & (7) & Q & \rA \\
		6,7 & (8) & P \land Q & \rAI{6,7} \\
		6,7 & (9) & (P \land Q)\lor R & \rOIa{8} \\
		1,7 & (10) & (P \land Q)\lor R & \rOE{2,4,5,6,9} \\
		1 & (11) & (P \land Q)\lor R & \rOE{3,4,5,7,10} \\
	\end{array}
	\]
\end{proof}

\label{PToQEqvnPOrQ}
\begin{theorem}[\(P \rightarrow Q \dashv \vdash \neg P \lor Q\) (Materielle Implikation)]
\end{theorem}
\label{rule:MI}
\begin{proof}
	\(\vdash:\)
	\[
	\begin{array}{llll}
		1 & (1) & P \rightarrow Q & \rA \\
		& (2) & \neg P \lor P & \ImpPOrnP \\
		3 & (3) & \neg P & \rA \\
		3 & (4) & \neg P \lor Q & \rOIa{3} \\
		5 & (5) & P & \rA \\
		1,5 & (6) & Q & \rRE{1,5} \\
		1,5 & (7) & \neg P \lor Q & \rOIb{6} \\
		1 & (8) & \neg P \lor Q & \rOE{2,3,4,5,7}\\
	\end{array}
	\]
	\(\dashv:\)
	\[
	\begin{array}{llll}
		1 & (1) & \neg P \lor Q & \rA \\
		2 & (2) & P & \rA \\
		3 & (3) & \neg Q & \rA \\
		2 & (4) & \neg\neg P & \rDN{2} \\
		2,3 & (5) & \neg\neg P \land \neg Q & \rAI{4,3} \\
		2,3 & (6) & \neg(\neg P \lor Q) & \nLpPOrQRpEqvnPAndnQ{5} \\
		1,2,3 & (6) & \bot & \rBI{1,6} \\
		1,2 & (7) & Q & \rCE{3,6} \\
		1 & (8) & P \rightarrow Q & \rRI{2,7}\\
	\end{array}
	\]		
\end{proof}

\label{POrQwnPImpQ}
\begin{theorem}[\(P \lor Q,\neg P \vdash Q\)]
\end{theorem}
\begin{proof}
	\[
	\begin{array}{llll}
		1 & (1) & P \lor Q & \rA \\
	    2 & (2) & \neg P & \rA \\  
        1 & (3) & \neg\neg P\lor\neg\neg Q & \POrQEqvnnPOrnnQ{1} \\ 
        1 & (4) & \neg P \rightarrow \neg\neg Q & \PToQEqvnPOrQ{3} \\
        1,2 & (6) & \neg\neg Q & \rRE{2,4} \\
        1,2 & (7) & Q & \rDN{6} \\
	\end{array}
	\]
\end{proof}

\label{POrQwnQImpP}
\begin{theorem}[\(P \lor Q,\neg Q \vdash P\)]
\end{theorem}
\begin{proof}
	\[
	\begin{array}{llll}
		1 & (1) & P \lor Q & \rA \\
	    2 & (2) & \neg Q & \rA \\
        1 & (3) & Q\lor P & \POrQImpQOrP{1} \\
        1,2 & (4) & P & \POrQwnPImpQ{3,2} \\
	\end{array}
	\]
\end{proof}

\label{nLpPToQRpEqvPAndnQ}
\begin{theorem}[\(\neg(P \rightarrow Q) \dashv \vdash P \land \neg Q\) (Implikation-Negations-Regel)]
\end{theorem}
\begin{proof}
	\(\vdash:\)
	\[
	\begin{array}{llll}
		1 & (1) & \neg(P \rightarrow Q) & \rA \\
		& (2) & P \rightarrow Q\leftrightarrow \neg P\lor Q & \PToQEqvnPOrQ{} \\
		1 & (3) & \neg(\neg P \lor Q) & \PLrQwnPImpnQ{1,2}\\
		1 & (4) & \neg\neg P \land \neg Q & \nLpPOrQRpEqvnPAndnQ{3}\\
		1 & (5) & \neg\neg P & \rAEa{4}\\  
		1 & (6) & P & \rDN{5}\\
		1 & (7) & Q & \rAEb{4}\\
		1 & (8) & P\land Q & \rAI{6,7}\\
	\end{array}
	\]
	\(\dashv:\)
	\[
	\begin{array}{llll}
		1 & (1) & P \wedge \neg Q & \rA \\
		1 & (2) & P  & \rAEa{1}\\
		1 & (3) & \neg Q  & \rAEb{1}\\
		1 & (4) & \neg\neg P  & \rDN{2}\\
		1 & (5) & \neg\neg P\land \neg Q  & \rAI{4,3}\\
		1 & (6) & \neg(\neg P\lor Q)  & \nLpPOrQRpEqvnPAndnQ{5}\\
		& (7) & P \rightarrow Q\leftrightarrow \neg P\lor Q & \PToQEqvnPOrQ{} \\
		1 & (8) & \neg (P \rightarrow Q) & \PLrQwnQImpnP{7,6}\\
	\end{array}
	\]		
\end{proof}

\label{QImpPToQ}
\begin{theorem}[\(Q \vdash P \rightarrow Q\)]
\end{theorem}
\begin{proof}
	\[
	\begin{array}{llll}
		1 & (1) & Q & \rA \\
		1 & (2) & \neg P \lor Q & \rOIa{1} \\
		1 & (3) & P\rightarrow Q & \PToQEqvnPOrQ{2} \\
	\end{array}
	\]
\end{proof}

\label{nPImpPToQ}
\begin{theorem}[\(\neg P \vdash P \rightarrow Q\)]
\end{theorem}
\begin{proof}
	\[
	\begin{array}{llll}
		1 & (1) & \neg P & \rA \\
		1 & (2) & \neg P \lor Q & \rOIb{1} \\
		1 & (3) & P\rightarrow Q & \PToQEqvnPOrQ{2} \\
	\end{array}
	\]
\end{proof}

\label{PLrQEqvnPLrnQ}
\begin{theorem}[\(P \leftrightarrow Q \dashv \vdash \neg P \leftrightarrow \neg Q\)]
\end{theorem}
\begin{proof}
	\(\vdash:\)
	\[
	\begin{array}{llll}
		1 & (1) & P \leftrightarrow Q & \rA \\
		1 & (2) & P \rightarrow Q & \rLREa{1} \\
		1 & (3) & Q \rightarrow P & \rLREb{1} \\
		1 & (4) & \neg Q \rightarrow \neg P & \PToQEqvnQTonP{2} \\
		1 & (5) & \neg P \rightarrow \neg Q & \PToQEqvnQTonP{3} \\
		1 & (6) & \neg P \leftrightarrow \neg Q & \rLRI{4,5} \\
	\end{array}
	\]
	
	\(\dashv:\)
	
	\[
	\begin{array}{llll}
		1 & (1) & \neg P \leftrightarrow \neg Q & \rA \\
		1 & (2) & \neg P \rightarrow \neg Q & \rLREa{1} \\
		1 & (3) & \neg Q \rightarrow \neg P & \rLREb{1} \\
		1 & (4) & Q \rightarrow P & \PToQEqvnQTonP{2} \\
		1 & (5) & P \rightarrow Q & \PToQEqvnQTonP{3} \\
		1 & (6) & P \leftrightarrow Q & \rLRI{4,5} \\
	\end{array}
	\]
\end{proof}


\label{FaxLpPLpxRpLrQLpxRpRpEqvFaxLpnPLpxRpLrnQLpxRpRp}
\begin{theorem}[\(\forall x(P(x) \leftrightarrow Q(x)) \dashv \vdash \forall x(\neg P(x) \leftrightarrow \neg Q(x))\)]
\end{theorem}
\begin{proof}
	\(\vdash:\)
	\[
	\begin{array}{llll}
		1 & (1) & \forall x (P(x) \leftrightarrow Q(x)) & \rA \\
		1 & (2) & P(a) \leftrightarrow Q(a) & \rUE{1} \\
		1 & (3) & \neg P(a) \leftrightarrow \neg Q(a) & Eq\PToQEqvnQTonP{2} \\
		1 & (4) & \forall x (\neg P(x) \leftrightarrow \neg Q(x)) & \rUI{3} \\
	\end{array}
	\]
	\(\dashv:\)
	\[
	\begin{array}{llll}
		1 & (1) & \forall x (\neg P(x) \leftrightarrow \neg Q(x)) & \rA \\
		1 & (2) & \neg P(a) \leftrightarrow \neg Q(a) & \rUE{1} \\
		1 & (3) & P(a) \leftrightarrow Q(a) & Eq\PToQEqvnQTonP{2} \\
		1 & (4) & \forall x (P(x) \leftrightarrow Q(x)) & \rUI{3} \\
	\end{array}
	\]
\end{proof}

\label{PLrQEqvLpPAndQRpOrLpnPAndnQRp}
\begin{theorem}[\(P \leftrightarrow Q \dashv \vdash  (P \land Q) \lor (\neg P\land \neg Q)\)]
\end{theorem}
\begin{proof}
	\(\vdash:\)
	\[
	\begin{array}{llll}
		1 & (1) & P \leftrightarrow Q & \rA \\
		1 & (2) & P \rightarrow Q & \rLREa{1} \\
		1 & (3) & Q \rightarrow P & \rLREb{1} \\
		& (4) & P \lor \neg P & \ImpPOrnP \\
		5 & (5) & P & \rA \\
		1,5 & (6) & Q & \rRE{2,5} \\
		1,5 & (7) & P \land Q & \rAI{5,6} \\
		1,5 & (8) & (P \land Q) \lor (\neg P\land \neg Q) & \rOIa{7} \\
		9 & (9) & \neg P & \rA \\
		1,9 & (10) & \neg Q & \PToQwnQImpnP{3,9} \\
		1,9 & (11) & \neg P \land \neg Q & \rAI{9,10} \\
		1,9 & (12) & (P \land Q) \lor (\neg P\land \neg Q) & \rOIb{11} \\
		1 & (13) & (P \land Q) \lor (\neg P\land \neg Q) & \rOE{4,5,8,9,12} \\
	\end{array}
	\]
	
	\(\dashv:\)
	
	\[
	\begin{array}{llll}
		1 & (1) & (P \land Q) \lor (\neg P\land \neg Q) & \rA \\
		2 & (2) & P \land Q & \rA \\
		2 & (3) & P & \rAEa{2} \\
		2 & (4) & Q & \rAEb{2} \\
		2 & (5) & P \rightarrow Q & \rRI{4} \\
		2 & (6) & Q \rightarrow P & \rRI{3} \\
		2 & (7) & P \leftrightarrow Q & \rLRI{5,6} \\
		8 & (8) & \neg P \land \neg Q & \rA \\
		8 & (9) & \neg P & \rAEa{8} \\
		8 & (10) & \neg Q & \rAEb{8} \\
		8 & (11) & P \rightarrow Q & \rRI{9} \\
		8 & (12) & Q \rightarrow P & \rRI{10} \\
		8 & (13) & P \leftrightarrow Q & \rLRI{11,12} \\
		1 & (14) & P \leftrightarrow Q & \rOE{1,2,7,8,13} \\
	\end{array}
	\]
\end{proof}

\label{PLrQEqvLpPToQRpAndLpQToPRp}
\begin{theorem}[\(P \leftrightarrow Q \dashv \vdash (P \rightarrow Q)\land (Q\rightarrow P)\)]
\end{theorem}
\begin{proof}
	\(\vdash:\)
	\[
	\begin{array}{llll}
		1 & (1) & P\leftrightarrow Q & \rA \\
		1 & (2) & P\rightarrow Q & \rLREa{1} \\
		1 & (3) & Q\rightarrow P & \rLREb{1} \\
		1 & (4) & (P\rightarrow Q)\land (Q\rightarrow P) & \rAI{2,3} \\
	\end{array}
	\]
	\(\dashv:\)
	\[
	\begin{array}{llll}
		1 & (1) & (P\rightarrow Q)\land (Q\rightarrow P) & \rA \\
		1 & (2) & P \rightarrow Q & \rAEa{1} \\
		1 & (3) & Q \rightarrow P & \rAEb{1} \\
		1 & (4) & P \leftrightarrow Q & \rLRI{2,3} \\
	\end{array}
	\]
\end{proof}

\label{FaxLpPLpxRpLrQLpxRpRpEqvFaxLpPLpxRpToQLpxRpRpAndFaxLpQLpxRpToPLpxRpRp}
\begin{theorem}[\(\forall x (P(x) \leftrightarrow Q(x)) \dashv \vdash \forall x (P(x) \rightarrow Q(x)) \land \forall x (Q(x) \rightarrow P(x))\)]
\end{theorem}

\begin{proof}
	\(\vdash:\)
	\[
	\begin{array}{llll}
		1 & (1) & \forall x (P(x) \leftrightarrow Q(x)) & \rA \\
		1 & (2) & P(a) \leftrightarrow Q(a) & \rUE{1} \\
		1 & (3) & P(b) \leftrightarrow Q(b) & \rUE{1} \\
		1 & (4) & P(a) \rightarrow Q(a) & \rLREa{2} \\
		1 & (5) & Q(b) \rightarrow P(b) & \rLREb{3} \\
		1 & (6) & \forall x (P(x) \rightarrow Q(x)) & \rUI{4} \\
		1 & (7) & \forall x (Q(x) \rightarrow P(x)) & \rUI{5} \\
		1 & (8) & \forall x (Q(x) \rightarrow P(x))\wedge \forall x (P(x) \rightarrow Q(x))  & \rAI{6,7} \\
	\end{array}
	\]
	
	\(\dashv:\)
	\[
	\begin{array}{llll}
		1 & (1) & \forall x (P(x) \rightarrow Q(x)) \wedge \forall x (Q(x) \rightarrow P(x)) & \rA \\
		1 & (2) & \forall x (P(x) \rightarrow Q(x)) & \rAEa{1} \\
		1 & (3) & \forall x (Q(x) \rightarrow P(x)) & \rAEb{1} \\
		1 & (4) & P(a) \rightarrow Q(a) & \rUE{2} \\
		1 & (5) & Q(a) \rightarrow P(a) & \rUE{3} \\
		1 & (6) & P(a) \leftrightarrow Q(a) & \rLRI{4,5} \\
		1 & (7) & \forall x (P(x) \leftrightarrow Q(x)) & \rUI{6} \\
	\end{array}
	\]
\end{proof}

\label{nLpPLrQRpEqvLpnPAndQRpOrLpPAndnQRp}
\begin{theorem}[\(\neg (P\leftrightarrow Q)\dashv\vdash (\neg P\land Q)\lor (P\land \neg Q)\)]
\end{theorem}
\begin{proof}
	\(\vdash:\)
	\[
	\begin{array}{llll}
		1 & (1) & \neg(P\leftrightarrow Q) & \rA \\
		& (2) & (P\leftrightarrow Q)\leftrightarrow ((P\rightarrow Q)\land (Q\rightarrow P)) & \PLrQEqvLpPToQRpAndLpQToPRp{} \\		
		1 & (3) & \neg((P\rightarrow Q)\land (Q\rightarrow P)) & \PLrQwnPImpnQ{1,2} \\
		1 & (4) & \neg(P\rightarrow Q)\lor \neg(Q\rightarrow P) & \nLpPAndQRpEqvnPOrnQ{3} \\
		5 & (5) & \neg(P\rightarrow Q) & \rA \\
		5 & (6) & P\land \neg Q & \nLpPToQRpEqvPAndnQ{3} \\
		5 & (7) & (\neg P\land Q)\lor (P\land \neg Q) & \rOIb{6} \\
		8 & (8) & \neg(Q\rightarrow P) & \rA \\
		8 & (9) & Q\land \neg P & \nLpPToQRpEqvPAndnQ{8} \\
		8 & (10) & \neg P \land Q & \PAndQImpQAndP{9} \\
		8 & (11) & (\neg P\land Q)\lor (P\land \neg Q) & \rOIa{10} \\
		1 & (12) & (\neg P\land Q)\lor (P\land \neg Q) & \rOE{4,5,7,8,11} \\
	\end{array}
	\]
	\(\dashv:\)
	\[
	\begin{array}{llll}
		1 & (1) & (\neg P\land Q)\lor (P\land \neg Q) & \rA \\
		2 & (2) & (\neg P\land Q) & \rA \\
		2 & (3) & (Q\land \neg P) & \PAndQImpQAndP{2} \\
		2 & (4) & \neg(Q\rightarrow P) & \nLpPToQRpEqvPAndnQ{3} \\
		2 & (5) & \neg(P\rightarrow Q)\lor \neg(Q\rightarrow P) & \rOIb{4} \\
		6 & (6) & (P\land \neg Q) & \rA \\
		6 & (7) & \neg (P\rightarrow Q) & \nLpPToQRpEqvPAndnQ{6} \\
		6 & (8) & \neg(P\rightarrow Q)\lor \neg(Q\rightarrow P) & \rOIa{7} \\
		1 & (9) & \neg(P\rightarrow Q)\lor \neg(Q\rightarrow P) & \rOE{1,2,5,6,8}\\
		1 & (10) & \neg ((P\rightarrow Q)\land (Q\rightarrow P)) & \nLpPAndQRpEqvnPOrnQ{9} \\
		& (11) & (P\leftrightarrow Q)\leftrightarrow ((P\rightarrow Q)\land (Q\rightarrow P)) & \PLrQEqvLpPToQRpAndLpQToPRp{} \\	
		1 & (12) & \neg (P\leftrightarrow Q) & \PLrQwnQImpnP{11,10} \\	
	\end{array}
	\]
\end{proof}

\label{ExxLpPLpxRpRpEqvnFaxLpnLpPLpxRpRp}
\begin{theorem}[\(\exists x(P(x)) \dashv\vdash \neg\forall x (\neg(P(x))\)]	
\end{theorem}
\begin{proof}
	\(\vdash:\)
	\[
	\begin{array}{llll}
		1 & (1) & \exists x(P(x)) & \rA \\
		2 & (2) & P(a) & \rA \\
		3 & (3) & \forall x (\neg(P(x))) & \rA \\
		3 & (4) & \neg(P(a) & \rUE{3} \\
		2,3 & (5) & \bot & \rBI{2,4} \\
		2 & (6) & \neg\forall x(\neg P(x)) & \rCI{3,5} \\
		1 & (7) & \neg\forall x(\neg P(x)) & \rEE{1,2,6} \\
	\end{array}
	\]
	\(\dashv:\)
	\[
	\begin{array}{llll}
		1 & (1) & \neg\forall x(\neg P(x)) & \rA \\
		2 & (2) & \neg\exists x(P(x)) & \rA \\
		3 & (3) & P(a) & \rA \\
		3 & (4) & \exists x(P(x)) & \rEI{3} \\
		2,3 & (5) & \bot & \rBI{2,4} \\
		2 & (6) & \neg P(a) & \rCI{3,5} \\
		2 & (7) & \forall x(\neg P(x)) & \rUI{6} \\
		1,2 & (8) & \bot & \rBI{1,7} \\
		1 & (9) & \exists x(P(x)) & \rCE{2,8} \\
	\end{array}
	\]
\end{proof}

\label{ExxLpnPLpxRpRpEqvnFaxLpPLpxRpRp}
\begin{theorem}[\(\exists x(\neg P(x)) \dashv\vdash \neg\forall x (P(x))\)]	
\end{theorem}
\begin{proof}
	\(\vdash:\)
	\[
	\begin{array}{llll}
		1 & (1) & \exists x(\neg P(x)) & \rA \\
		2 & (2) & \neg P(a) & \rA \\
		3 & (3) & \forall x (P(x)) & \rA \\
		3 & (4) & P(a) & \rUE{3} \\
		2,3 & (5) & \bot & \rBI{2,4} \\
		2 & (6) & \neg\forall x(P(x)) & \rCI{3,5} \\
		1 & (7) & \neg\forall x(P(x)) & \rEE{1,2,6} \\
	\end{array}
	\]
	\(\dashv:\)
	\[
	\begin{array}{llll}
		1 & (1) & \neg\forall x(P(x)) & \rA \\
		2 & (2) & \neg \exists x(\neg P(x)) & \rA \\
		3 & (3) & \neg P(a) & \rA \\
		3 & (4) & \exists x(\neg P(x)) & \rEI{3} \\
		2,3 & (5) & \bot & \rBI{2,4} \\
		2 & (6) & P(a) & \rCE{3,5} \\
		3 & (7) & \forall x(P(x)) & \rUI{6} \\
		1,3 & (8) & \bot & \rBI{1,7} \\
		1 & (9) & \exists x(P(x)) & \rCE{2,8} \\
	\end{array}
	\]
\end{proof}

\label{LpnFaxLpPLpxRpTonQLpxRpRpEqvExxLpPLpxRpAndQLpxRpRp}
\begin{theorem}[\(\neg\forall x(P(x)\rightarrow \neg Q(x)) \dashv\vdash \exists x (P(x)\land  Q(x))\)]
\end{theorem}
\begin{proof}
	\(\vdash:\)
	\[
	\begin{array}{llll}
		1 & (1) & \neg\forall x(P(x)\rightarrow \neg Q(x)) & \rA \\
		1 & (2) & \exists x \neg (P(x)\rightarrow \neg Q(x)) & \ExxLpnPLpxRpRpEqvnFaxLpPLpxRpRp{1} \\
		3 & (3) & \neg (P(a)\rightarrow \neg Q(a)) & \rA \\
		3 & (4) & P(a)\land \neg\neg Q(a) &  \nLpPToQRpEqvPAndnQ{3} \\
		3 & (5) & P(a) & \rAEa{4} \\
		3 & (6) & \neg\neg Q(a) & \rAEb{4} \\
		3 & (7) & Q(a) & \rDN{6} \\
		3 & (8) & P(a)\land Q(a) & \rAI{5,7} \\
		3 & (9) & \exists x(P(x)\land Q(x)) & \rEI{8} \\
		1 & (10) & \exists x(P(x)\land Q(x)) & \rEI{2,3,9} \\
	\end{array}
	\]
	\(\dashv:\)
	\[
	\begin{array}{llll}
		1 & (1) & \exists x (P(x)\wedge Q(x)) & \rA \\
		2 & (2) & P(a)\wedge Q(a) & \rA \\
		2 & (3) & P(a) & \rAEa{2} \\
		2 & (4) & Q(a) & \rAEb{2} \\
		2 & (5) & \neg\neg Q(a) & \rDN{4} \\
		2 & (6) & P(a)\land \neg\neg Q(a) & \rAI{3,5} \\
		2 & (7) & \neg(P(a)\rightarrow \neg Q(a)) & \nLpPToQRpEqvPAndnQ{6} \\
		2 & (8) & \exists x(\neg(P(x)\rightarrow \neg Q(x))) & \rEI{7} \\
		2 & (9) & \neg\forall x(P(x)\rightarrow \neg Q(x)) & \ExxLpnPLpxRpRpEqvnFaxLpPLpxRpRp{8} \\
		1 & (10) & \neg\forall x(P(x)\rightarrow \neg Q(x)) & \rEI{1,2,9} \\
	\end{array}
	\]
\end{proof}

\label{nFaxLpPLpxRpToQLpxRpRpEqvExxLpPLpxRpAndnQLpxRpRp}
\begin{theorem}[\(\neg\forall x(P(x)\rightarrow Q(x)) \dashv\vdash \exists x (P(x)\land  \neg Q(x))\)]
\end{theorem}
\begin{proof}
	\(\vdash:\)
	\[
	\begin{array}{llll}
		1 & (1) & \neg\forall x(P(x)\rightarrow Q(x)) & \rA \\
		1 & (2) & \exists x \neg (P(x)\rightarrow Q(x)) & \ExxLpnPLpxRpRpEqvnFaxLpPLpxRpRp{1} \\
		3 & (3) & \neg (P(a)\rightarrow Q(a)) & \rA \\
		3 & (4) & P(a)\land \neg Q(a) &  \nLpPToQRpEqvPAndnQ{3} \\
		3 & (5) & \exists x(P(x)\land \neg Q(x)) & \rEI{8} \\
		1 & (6) & \exists x(P(x)\land \neg Q(x)) & \rEI{2,3,5} \\
	\end{array}
	\]
	\(\dashv:\)
	\[
	\begin{array}{llll}
		1 & (1) & \exists x (P(x)\wedge \neg Q(x)) & \rA \\
		2 & (2) & P(a)\wedge \neg Q(a) & \rA \\
		2 & (3) & \neg(P(a)\rightarrow Q(a)) & \nLpPToQRpEqvPAndnQ{2} \\
		2 & (4) & \exists x(\neg(P(x)\rightarrow \neg Q(x))) & \rEI{3} \\
		2 & (5) & \neg\forall x(P(x)\rightarrow \neg Q(x)) & \ExxLpnPLpxRpRpEqvnFaxLpPLpxRpRp{8} \\
		1 & (6) & \neg\forall x(P(x)\rightarrow \neg Q(x)) & \rEI{1,2,5} \\
	\end{array}
	\]
\end{proof}

\label{FaxLpPLpxRpRpEqvnExxnLpPLpxRpRp}
\begin{theorem}[\(\forall x(P(x)) \dashv\vdash \neg\exists x \neg(P(x)) \)]
\end{theorem}
\begin{proof}
	\(\vdash:\)
	\[
	\begin{array}{llll}
		1 & (1) & \forall x(P(x)) & \rA \\
		2 & (2) & \exists x(\neg P(x)) & \rA \\
		3 & (3) & \neg P(a) & \rA \\
		1 & (4) & P(a) & \rUE{1} \\
		1,3 & (5) & \bot & \rBI{3,4} \\
		3 & (6) & \neg\forall x(P(x)) & \rCI{1,5} \\
		2 & (7) & \neg\forall x(P(x)) & \rEE{2,3,6} \\
		1,2 & (8) & \bot & \rBI{1,7} \\
		1 & (7) & \neg\exists x(\neg P(x)) & \rCI{2,8} \\
	\end{array}
	\]
	\(\dashv:\)
	\[
	\begin{array}{llll}
		1 & (1) & \neg\exists x(\neg P(x)) & \rA \\
		2 & (2) & \neg P(a) & \rA \\
		2 & (3) & \exists x(\neg P(x)) & \rEI{2} \\
		1,2 & (4) & \bot & \rBI{1,3} \\
		1 & (5) & P(a) & \rCE{2,4} \\
		1 & (6) & \forall x(P(x)) & \rUI{5} \\
	\end{array}
	\]
\end{proof}

\label{FaxLpnPLpxRpRpEqvnExxLpPLpxRpRp}
\begin{theorem}[\(\forall x(\neg P(x)) \dashv\vdash \neg\exists x (P(x))\)]
\end{theorem}

\begin{proof}
	\(\vdash:\)
	\[
	\begin{array}{llll}
		1 & (1) & \forall x(\neg P(x)) & \rA \\
		2 & (2) & \exists x(P(x)) & \rA \\
		3 & (3) & P(a) & \rA \\
		1 & (4) & \neg P(a) & \rUE{1} \\
		1,3 & (5) & \bot & \rBI{3,4} \\
		3 & (6) & \neg\exists x(P(x)) & \rCI{2,5} \\
		1 & (7) & \neg\exists x(P(x)) & \rUE{1,6} \\
	\end{array}
	\]
	\(\dashv:\)
	\[
	\begin{array}{llll}
		1 & (1) & \neg\exists x(P(x)) & \rA \\
		2 & (2) & P(a) & \rA \\
		2 & (3) & \exists x(P(x)) & \rEI{2} \\
		1,2 & (4) & \bot & \rBI{1,3} \\
		1 & (5) & \neg P(a) & \rCE{2,4} \\
		1 & (6) & \forall x(\neg P(x)) & \rUI{5} \\
	\end{array}
	\]
\end{proof}

\label{FaxLpPLpxRpToQLpxRpRpEqvnExxLpPLpxRpAndnQLpxRpRp}
\begin{theorem}[\(\forall x(P(x)\rightarrow Q(x)) \dashv\vdash \neg\exists x (P(x)\land \neg Q(x))\)]
\end{theorem}
\begin{proof}
	\(\vdash:\)
	\[
	\begin{array}{llll}
		1 & (1) & \forall x(P(x)\rightarrow Q(x)) & \rA \\
		2 & (2) & \exists x(P(x)\wedge \neg Q(x)) & \rA \\
		2 & (3) & P(a)\wedge \neg Q(a) & \rEE{2} \\
		2 & (4) & P(a) & \rAEa{3} \\
		2 & (5) & \neg Q(a) & \rAEb{3} \\
		1 & (6) & P(a)\rightarrow Q(a) & \rUE{1} \\
		1,2 & (7) & Q(a) & \rRE{4,6} \\
		1,2 & (8) & \bot & \rBI{5,7} \\
		1 & (9) & \neg\exists x(P(x)\wedge \neg Q(x)) & \rCI{2,9} \\
	\end{array}
	\]
	\(\dashv:\)
	\[
	\begin{array}{llll}
		1 & (1) & \neg\exists x(P(x)\wedge \neg Q(x)) & \rA \\
		1 & (2) & \forall x(\neg(P(x)\wedge \neg Q(x))) & \FaxLpnPLpxRpRpEqvnExxLpPLpxRpRp{1} \\
		1 & (3) & \neg(P(a)\wedge \neg Q(a)) & \rUE{2} \\
		1 & (4) & \neg P(a)\vee \neg \neg Q(a) & \nLpPAndQRpEqvnPOrnQ{3} \\
		5 & (5) & \neg P(a)& \rA \\
		5 & (6) & \neg P(a)\vee  Q(a) & \rOIa{5} \\
		7 & (7) & \neg \neg Q(a)& \rA \\
		7 & (8) & Q(a)& \rDN{7} \\
		7 & (9) & \neg P(a)\vee  Q(a) & \rOIb{7} \\
		1 & (10) & \neg P(a)\vee  Q(a) & \rOE{4,5,6,7,9} \\
		1 & (11) & P(a)\rightarrow  Q(a) & \PToQEqvnPOrQ{10} \\
		1 & (12) & \forall x(P(x)\rightarrow  Q(x)) & \rUI{11} \\
	\end{array}
	\]
\end{proof}

\label{FaxLpPLpxRpTonQLpxRpRpEqvnExxLpPLpxRpAndQLpxRpRp}
\begin{theorem}[\(\forall x(P(x)\rightarrow \neg Q(x)) \dashv\vdash \neg\exists x (P(x)\land Q(x))\)]
\end{theorem}
\begin{proof}
	\(\vdash:\)
	\[
	\begin{array}{llll}
		1 & (1) & \forall x(P(x)\rightarrow \neg Q(x)) & \rA \\
		2 & (2) & \exists x(P(x)\wedge Q(x)) & \rA \\
		2 & (3) & P(a)\wedge Q(a) & \rEE{2} \\
		2 & (4) & P(a) & \rAEa{3} \\
		2 & (5) & Q(a) & \rAEb{3} \\
		1 & (6) & P(a)\rightarrow \neg Q(a) & \rUE{1} \\
		1,2 & (7) & \neg Q(a) & \rRE{4,6} \\
		1,2 & (8) & \bot & \rBI{5,7} \\
		1 & (9) & \neg\exists x(P(x)\wedge Q(x)) & \rCI{2,9} \\
	\end{array}
	\]
	\(\dashv:\)
	\[
	\begin{array}{llll}
		1 & (1) & \neg\exists x(P(x)\wedge Q(x)) & \rA \\
		1 & (2) & \forall x(\neg(P(x)\wedge Q(x))) & \FaxLpnPLpxRpRpEqvnExxLpPLpxRpRp{1} \\
		1 & (3) & \neg(P(a)\wedge Q(a)) & \rUE{2} \\
		1 & (4) & \neg P(a)\vee \neg Q(a) & \nLpPAndQRpEqvnPOrnQ{3} \\
		1 & (5) &  P(a)\rightarrow  Q(a) & \PToQEqvnPOrQ{4} \\
		1 & (6) & \forall x(P(x)\rightarrow  Q(x)) & \rUI{5} \\
	\end{array}
	\]
\end{proof}

\label{FaxLpPLpxRpLrQLpxRpRpEqvnExxLpPLpxRpAndnQLpxRpRpAndnExxLpQLpxRpAndnPLpxRpRp}
\begin{theorem}[\(\forall x(P(x)\leftrightarrow Q(x)) \dashv\vdash \neg\exists x (P(x)\land \neg Q(x))\land \neg\exists x (Q(x)\land \neg P(x))\)]
\end{theorem}

\begin{proof}
	\(\vdash:\)
	\[
	\begin{array}{lll p{7cm}}
		1 & (1) & \forall x(P(x)\leftrightarrow Q(x)) & \rA \\
            1 & (2) & \forall x(P(x)\rightarrow Q(x))\land \forall x(Q(x)\rightarrow P(x))  & \FaxLpPLpxRpLrQLpxRpRpEqvFaxLpPLpxRpToQLpxRpRpAndFaxLpQLpxRpToPLpxRpRp{1} \\
            1 & (3) & \forall x(P(x)\rightarrow Q(x)) & \rAEa{2} \\
            1 & (4) & \forall x(Q(x)\rightarrow P(x)) & \rAEb{2} \\
            1 & (5) & \neg\exists x (P(x)\land \neg Q(x)) & \FaxLpPLpxRpToQLpxRpRpEqvnExxLpPLpxRpAndnQLpxRpRp{3} \\
            1 & (6) & \neg\exists x (Q(x)\land \neg P(x)) & \FaxLpPLpxRpToQLpxRpRpEqvnExxLpPLpxRpAndnQLpxRpRp{4} \\
            1 & (7) &  \neg\exists x (P(x)\land \neg Q(x))\land \neg\exists x (Q(x)\land \neg P(x)) & \rAI{5,6} \\
	\end{array}
	\]
	\(\dashv:\)
	\[
	\begin{array}{lll p{7cm}}
		1 & (1) & \neg\exists x (P(x)\land \neg Q(x))\land \neg\exists x (Q(x)\land \neg P(x)) & \rA \\
            1 & (2) & \neg\exists x (P(x)\land \neg Q(x))  & \rAEa{1} \\
            1 & (3) & \neg\exists x (Q(x)\land \neg P(x)) & \rAEb{1} \\
            1 & (4) & \forall x(P(x)\rightarrow Q(x)) & \FaxLpPLpxRpToQLpxRpRpEqvnExxLpPLpxRpAndnQLpxRpRp{2} \\
            1 & (5) & \forall x(Q(x)\rightarrow P(x)) & \FaxLpPLpxRpToQLpxRpRpEqvnExxLpPLpxRpAndnQLpxRpRp{3} \\
            1 & (6) & \forall x(P(x)\rightarrow Q(x))\land \forall x(Q(x)\rightarrow P(x)) & \rAI{5} \\
            1 & (7) &  \neg\exists x (P(x)\land \neg Q(x))\land \neg\exists x (Q(x)\land \neg P(x)) & \FaxLpPLpxRpLrQLpxRpRpEqvFaxLpPLpxRpToQLpxRpRpAndFaxLpQLpxRpToPLpxRpRp{6} \\
	\end{array}
	\]
\end{proof}

\label{nFaxLpPLpxRpLrQLpxRpRpEqvExxLpPLpxRpAndnQLpxRpRpOrExxLpQLpxRpAndnPLpxRpRp}
\begin{theorem}[\(\neg\forall x(P(x)\leftrightarrow Q(x)) \dashv\vdash \exists x (P(x)\land \neg Q(x))\lor \exists x (Q(x)\land \neg P(x))\)]
\end{theorem}

\begin{proof}
	\(\vdash:\)
	\[
	\begin{array}{lll p{4cm}}
		1 & (1) & \neg\forall x(P(x)\leftrightarrow Q(x)) & \rA \\
            1 & (2) & \forall x(P(x)\leftrightarrow Q(x)) \leftrightarrow \neg\exists x (P(x)\land \neg Q(x))\land \neg\exists x (Q(x)\land \neg P(x))  & \FaxLpPLpxRpLrQLpxRpRpEqvnExxLpPLpxRpAndnQLpxRpRpAndnExxLpQLpxRpAndnPLpxRpRp{} \\
            1 & (3) & \neg(\neg\exists x (P(x)\land \neg Q(x))\land \neg\exists x (Q(x)\land \neg P(x))) & \PLrQwnPImpnQ{1,2} \\
            1 & (4) & \exists x (P(x)\land \neg Q(x))\lor \neg\exists x (Q(x)\land \neg P(x))) & \nLpnPAndnQRpEqvPOrQ{3} \\
	\end{array}
	\]
	\(\dashv:\)
	\[
	\begin{array}{lll p{4cm}}
            1 & (1) & \exists x (P(x)\land \neg Q(x))\lor \neg\exists x (Q(x)\land \neg P(x))) & \rA \\
		1 & (2) & \neg(\neg\exists x (P(x)\land \neg Q(x))\land \neg\exists x (Q(x)\land \neg P(x))) & \nLpnPAndnQRpEqvPOrQ{1} \\
	    1 & (3) & \forall x(P(x)\leftrightarrow Q(x)) \leftrightarrow \neg\exists x (P(x)\land \neg Q(x))\land \neg\exists x (Q(x)\land \neg P(x))  & \FaxLpPLpxRpLrQLpxRpRpEqvnExxLpPLpxRpAndnQLpxRpRpAndnExxLpQLpxRpAndnPLpxRpRp{} \\
            1 & (4) & \neg\forall x(P(x)\leftrightarrow Q(x)) & \PLrQwnQImpnP{2,3} \\
     
        \end{array}
	\]
\end{proof}


\subsection{Theoreme zur Quantorendistribution}

\label{PLpaRpwFaxLpPLpxRpLrQLpxRpRpImpQLpaRp}
\begin{theorem}[\(P(a), \forall x(P(x) \leftrightarrow Q(x)) \vdash Q(a)\)]
\end{theorem}
\begin{proof}
	\[
	\begin{array}{llll}
		1 & (1) & P(a) & \rA \\
		2 & (2) & \forall x(P(x) \leftrightarrow Q(x)) & \rA \\
		2 & (3) & P(a) \leftrightarrow Q(a) & \rUE{2} \\
		1,2 & (4) & Q(a) & \rLREa{1,3} \\
	\end{array}
	\]
\end{proof}

\label{QLpaRpwFaxLpPLpxRpLrQLpxRpRpImpPLpaRp}
\begin{theorem}[\(Q(a), \forall x(P(x) \leftrightarrow Q(x)) \vdash P(a)\)]
\end{theorem}
\begin{proof}
	\[
	\begin{array}{llll}
		1 & (1) & Q(a) & \rA \\
		2 & (2) & \forall x(P(x) \leftrightarrow Q(x)) & \rA \\
		2 & (3) & P(a) \leftrightarrow Q(a) & \rUE{2} \\
		1,2 & (4) & Q(a) & \rLREb{1,3} \\
	\end{array}
	\]
\end{proof}

\label{FaxLpPToFLpxRpRpEqvPToFaxLpFLpxRpRp}
\begin{theorem}[\(\forall x(P \rightarrow F(x)) \dashv \vdash P \rightarrow \forall x(F(x))\)]
\end{theorem}
\begin{proof}
	\(\vdash:\)
	\[
	\begin{array}{llll}
		1 & (1) & \forall x(P \rightarrow F(x)) & \rA \\
		2 & (2) & P & \rA \\
		1,2 & (3) & P \rightarrow F(x) & \rUE{1} \\
		1,2 & (4) & F(x) & \rRE{2,3} \\
		2 & (5) & \forall x(F(x)) & \rUI{4} \\
		1 & (6) & P \rightarrow \forall x(F(x)) & \rRI{2,5} \\
	\end{array}
	\]
	\(\dashv:\)
	\[
	\begin{array}{llll}
		1 & (1) & P \rightarrow \forall x(F(x)) & \rA \\
		2 & (2) & P & \rA \\
		1 & (3) & \forall x(F(x)) & \rRE{1,2} \\
		1,2 & (4) & F(x) & \rUE{3} \\
		2 & (5) & P \rightarrow F(x) & \rRI{2,4} \\
		1 & (6) & \forall x(P \rightarrow F(x)) & \rUI{5} \\
	\end{array}
	\]
\end{proof}


\label{FaxLpPAndFLpxRpRpEqvPAndFaxLpFLpxRpRp}
\begin{theorem}[\(\forall x(P \land F(x)) \dashv \vdash P \land \forall x(F(x))\)]
\end{theorem}
\begin{proof}
	\(\vdash:\)
	\[
	\begin{array}{llll}
		1 & (1) & \forall x(P \land F(x)) & \rA \\
		1 & (2) & P \land F(x) & \rUE{1} \\
		1 & (3) & P & \rAEa{2} \\
		1 & (4) & F(x) & \rAEb{2} \\
		1 & (5) & \forall x(F(x)) & \rUI{4} \\
		1 & (6) & P \land \forall x(F(x)) & \rAI{3,5} \\
	\end{array}
	\]
	\(\dashv:\)
	\[
	\begin{array}{llll}
		1 & (1) & P \land \forall x(F(x)) & \rA \\
		1 & (2) & P & \rAEa{1} \\
		1 & (3) & \forall x(F(x)) & \rAEb{1} \\
		1 & (4) & F(x) & \rUE{3} \\
		1 & (5) & P \land F(x) & \rAI{2,4} \\
		1 & (6) & \forall x(P \land F(x)) & \rUI{5} \\
	\end{array}
	\]
\end{proof}

\label{ExxLpPAndFLpxRpRpEqvPAndExxLpFLpxRpRp}
\begin{theorem}[\(\exists x(P \land F(x)) \dashv \vdash P \land \exists x(F(x))\)]
\end{theorem}
\begin{proof}
	\(\vdash:\)
	\[
	\begin{array}{llll}
		1 & (1) & \exists x(P \land F(x)) & \rA \\
		1 & (2) & P \land F(x) & \rEE{1} \\
		1 & (3) & P & \rAEa{2} \\
		1 & (4) & F(x) & \rAEb{2} \\
		1 & (5) & \exists x(F(x)) & \rEI{4} \\
		1 & (6) & P \land \exists x(F(x)) & \rAI{3,5} \\
	\end{array}
	\]
	\(\dashv:\)
	\[
	\begin{array}{llll}
		1 & (1) & P \land \exists x(F(x)) & \rA \\
		1 & (2) & P & \rAEa{1} \\
		1 & (3) & \exists x(F(x)) & \rAEb{1} \\
		1 & (4) & F(x) & \rEE{3} \\
		1 & (5) & P \land F(x) & \rAI{2,4} \\
		1 & (6) & \exists x(P \land F(x)) & \rEI{5} \\
	\end{array}
	\]
\end{proof}

\label{FaxLpPOrFLpxRpRpEqvPOrFaxLpFLpxRpRp}
\begin{theorem}[\(\forall x(P \lor F(x)) \dashv \vdash P \lor \forall x(F(x))\)]
\end{theorem}
\begin{proof}
	\(\vdash:\)
	\[
	\begin{array}{llll}
		1 & (1) & \forall x(P \lor F(x)) & \rA \\
		2 & (2) & \neg(P \lor \forall x(F(x))) & \rA \\
		2 & (3) & \neg P \land \neg \forall x(F(x)) & \nLpPOrQRpEqvnPAndnQ{2} \\			
		2 & (4) & \neg P & \rAEa{3} \\
		2 & (5) & \neg \forall x(F(x)) & \rAEb{3} \\
		2 & (6) & \exists x(\neg F(x)) & Df. \exists\, (5) \\
		2 & (7) & \neg F(a) & \rEE{6} \\
		1 & (8) & P\lor F(a) & \rUE{7} \\
		9 & (9) & P & \rA \\
		2,9 & (10) & \bot & \rBI{4,9} \\
		9 & (11) & P \lor \forall x(F(x) & \rCE{2,10} \\
		12 & (12) & F(a) & \rA \\
		2,12 & (13) & \bot & \rBI{12,7} \\
		12 & (14) & P \lor \forall x(F(x) & \rCE{2,13} \\
		1 & (15) & P \lor \forall x(F(x) & \rOE{8,9,11,12,14} \\
	\end{array}
	\]
	\(\dashv:\)
	\[
	\begin{array}{llll}
		1 & (1) & P \lor \forall x(F(x)) & \rA \\
		2 & (2) & P & \rA \\
		2 & (3) & P\lor F(x) & \rOIa{2} \\
		2 & (4) & \forall x(P\lor F(x)) & \rUI{3} \\
		5 & (5) & \forall x(F(x)) & \rA \\
		5 & (6) & F(x) & \rUE{4} \\
		5 & (7) & P\lor F(x) & \rOIb{6} \\
		5 & (8) & \forall x(P\lor F(x)) & \rUI{7} \\
		5 & (9) & \forall x(P\lor F(x)) & \rOE{1,2,4,5,8} \\
	\end{array}
	\]
\end{proof}

\label{ExxLpPOrFLpxRpRpEqvPOrExxLpFLpxRpRp}
\begin{theorem}[\(\exists x(P \lor F(x)) \dashv \vdash P \lor \exists x(F(x))\)]
\end{theorem}
\begin{proof}
	\(\vdash:\)
	\[
	\begin{array}{llll}
		1 & (1) & \exists x(P \lor F(x)) & \rA \\
		2 & (2) & P \lor F(a) & \rA \\
		3 & (3) & P & \rA \\
		1,3 & (4) & P \lor \exists x(F(x)) & \rOIa{3} \\
		5 & (5) & F(a) & \rA \\
		1,5 & (6) & \exists x(F(x)) & \rEI{5} \\
		1,5 & (7) & P \lor \exists x(F(x)) & \rOIb{6} \\
		2 & (8) & P \lor \exists x(F(x)) & \rOE{2,3,4,5,7} \\
		1 & (9) & P \lor \exists x(F(x)) & \rEE{1,2,8} \\
	\end{array}
	\]
	\(\dashv:\)
	\[
	\begin{array}{llll}
		1 & (1) & P \lor \exists x(F(x)) & \rA \\
		2 & (2) & P & \rA \\
		2 & (3) & P\lor F(x) & \rOIa{2} \\
		2 & (4) & \exists x(P \lor F(x)) & \rEI{3} \\
		5 & (5) & \exists x(F(x)) & \rA \\
		5 & (6) & F(a) & \rEE{5} \\
		5 & (7) & P \lor F(a) & \rOIb{6} \\
		5 & (8) & \exists x(P \lor F(x)) & \rEI{7} \\
		1 & (9) & \exists x(P \lor F(x)) & \rOE{1,2,4,5,8} \\
	\end{array}
	\]
\end{proof}

\label{ExxLpFLpxRpToPRpEqvFaxLpFLpxRpRpToP}
\begin{theorem}[\(\exists x(F(x) \rightarrow P) \dashv \vdash \forall x(F(x)) \rightarrow P\)]
\end{theorem}
\begin{proof}
	\(\vdash:\)
	\[
	\begin{array}{llll}
		1 & (1) & \exists x(F(x) \rightarrow P) & \rA \\
		2 & (2) & F(a) \rightarrow P & \rA \\
		3 & (3) & \forall x(F(x)) & \rA \\
		3 & (4) & F(a) & \rUE{3} \\
		2,3 & (5) & P & \rRE{2,4} \\
		2 & (6) & \forall x(F(x)) \rightarrow P & \rRI{3,5} \\
		1 & (7) & \forall x(F(x)) \rightarrow P & \rEE{1,2,6} \\
	\end{array}
	\]
	\(\dashv:\)
	\[
	\begin{array}{llll}
		1 & (1) & \forall x(F(x)) \rightarrow P & \rA \\
		2 & (2) & \forall x(F(x)) & \rA \\
		1 & (3) & P & \rRE{1,2} \\
		1 & (4) & F(a) \rightarrow P & \rRI{3} \\
		1 & (5) & \exists x(F(x) \rightarrow P) & \rEI{4} \\
	\end{array}
	\]
\end{proof}

\label{FaxLpFLpxRpAndGLpxRpRpEqvFaxLpFLpxRpRpAndFaxLpGLpxRpRp}
\begin{theorem}[\(\forall x(F(x)\land G(x))\dashv\vdash \forall x(F(x))\land \forall x(G(x))\)]
\end{theorem}
\begin{proof}
	\(\vdash:\)
	\[
	\begin{array}{llll}
		1 & (1) & \forall x(F(x)\land G(x)) & \rA \\
		1 & (2) &  F(x)\land G(x) & \rUE{1} \\	
		1 & (3) &  F(x) & \rAEa{2} \\
		1 & (4) &  \forall x(F(x)) & \rUI{3} \\	
		1 & (5) &  G(x) & \rAEb{2} \\
		1 & (6) &  \forall x(G(x)) & \rUI{5} \\
		1 & (7) &  \forall x(F(x))\land \forall x(G(x)) & \rAI{3,5} \\
	\end{array}
	\]
	\(\dashv:\)
	\[
	\begin{array}{llll}
		1 & (1) & \forall x(F(x))\land \forall x(G(x)) & \rA \\
		1 & (2) & \forall x(F(x)) & \rAEa{1} \\
		1 & (3) & F(x) & \rUE{2} \\
		1 & (4) & \forall x(G(x)) & \rAEb{1} \\
		1 & (5) & G(x) & \rUE{4} \\
		1 & (6) & F(x)\land G(x) & \rAI{3,5} \\
		1 & (7) & \forall x(F(x)\land G(x)) & \rUI{6} \\
	\end{array}
	\]
\end{proof}

\label{ExxLpFLpxRpOrGLpxRpRpEqvExxLpFLpxRpRpOrExxLpGLpxRpRp}
\begin{theorem}[\(\exists x(F(x)\lor G(x))\dashv\vdash \exists x(F(x))\lor \exists x(G(x))\)]
\end{theorem}
\begin{proof}
	\(\vdash:\)
	\[
	\begin{array}{llll}
		1 & (1) & \exists x(F(x)\lor G(x)) & \rA \\
		2 & (2) & F(a)\lor G(a) & \rA \\
		3 & (3) & F(a) & \rA \\	
		3 & (4) & \exists x(F(x)) & \rEI{3} \\	
		3 & (5) & \exists x(F(x))\lor \exists x(G(x)) & \rOIa{4} \\
		6 & (6) & G(a) & \rA \\
		6 & (7) & \exists x(G(x)) & \rEI{6} \\
		6 & (8) & \exists x(F(x))\lor \exists x(G(x)) & \rOIb{7} \\
		2 & (9) & \exists x(F(x))\lor \exists x(G(x)) & \rOE{2,3,5,6,8} \\
		1 & (10) & \exists x(F(x))\lor \exists x(G(x)) & \rEE{1,2,9} \\
	\end{array}
	\]
	\(\dashv:\)
	\[
	\begin{array}{llll}
		1 & (1) & \exists x(F(x))\lor \exists x(G(x)) & \rA \\
		2 & (2) & \exists x(F(x)) & \rA \\		7 & (11) & \exists x(F(x)\lor G(x)) & \rEE{7,8,9} \\
		3 & (3) & F(a) & \rA \\
		3 & (4) & F(a)\lor G(a) & \rOIa{3} \\
		3 & (5) & \exists x(F(x)\lor G(x)) & \rEI{4} \\
		2 & (6) & \exists x(F(x)\lor G(x)) & \rEI{2,3,5} \\
		7 & (7) & \exists x(G(x)) & \rA \\
		8 & (8) & G(a) & \rA \\
		8 & (9) & F(a)\lor G(a) & \rOIb{8} \\
		8 & (10) & \exists x(F(x)\lor G(x)) & \rEI{9} \\
		7 & (11) & \exists x(F(x)\lor G(x)) & \rEE{7,8,9} \\
		1 & (12) & \exists x(F(x)\lor G(x)) & \rEE{1,2,6,7,11} \\
	\end{array}
	\]
\end{proof}

\label{ExxExyFLpxwyRpImpExyExxFLpxwyRp}
\begin{theorem}[\(\exists x\exists yF(x,y)\vdash \exists y\exists x F(x,y)\)]
\end{theorem}
\begin{proof}
	\[
	\begin{array}{llll}
		1 & (1) & \exists x\exists y F(x,y) & \rA \\
		2 & (2) & \exists y F(a,y) & \rA \\
		3 & (3) & F(a,b) & \rA \\
		3 & (4) & \exists xF(x,b) & \rEI{3} \\
		3 & (5) & \exists y\exists xF(x,y) & \rEI{4} \\
		2 & (6) & \exists y\exists xF(x,y) & \rEE{2,3,5} \\
		1 & (7) & \exists y\exists xF(x,y) & \rEE{1,2,6} \\
	\end{array}
	\]
\end{proof}

\begin{theorem}[\(\forall x\forall yF(x,y)\vdash \forall y\forall x F(x,y)\)]
\end{theorem}
\begin{proof}
	\[
	\begin{array}{llll}
		1 & (1) & \forall x\forall y F(x,y) & \rA \\
		1 & (2) & \forall y F(a,y) & \rUE{1}\\
		1 & (3) & F(a,b) & \rUE{2}\\
		1 & (4) & \forall x F(x,b) & \rUI{3}\\
		1 & (5) & \forall y\forall x F(x,y) & \rUI{4}\\
	\end{array}
	\]
\end{proof}


\subsection{Theoreme mit dem Identitätssymbol}

\label{FaEqvExxLpxIdaAndFxRp}
\begin{theorem}[\(Fa\dashv\vdash \exists x(x=a\land Fx)\)]
\end{theorem}
\begin{proof}
	\(\vdash:\)
	\[
	\begin{array}{llll}
		1 & (1) & Fa & \rA \\
		& (2) & a=a & \rII\\
		1 & (3) & a=a\land Fa & \rAI{2,1} \\	
		1 & (4) & \exists x(x=a\land Fx) & \rEI{3} \\	
	\end{array}
	\]
	\(\dashv:\)
	\[
	\begin{array}{llll}
		1 & (1) & \exists x(x=a\land Fx) & \rA \\
		2 & (2) & b=a\land Fb & \rA \\		
		2 & (3) & b=a & \rAEa{2} \\	
		2 & (4) & Fb & \rAEb{2} \\	
		2 & (5) & Fa & \rIE{3,4} \\	
		1 & (6) & Fa & \rEE{1,2,5} \\	
	\end{array}
	\]
\end{proof}

\label{aIdbImpbIda}
\begin{theorem}[\(a=b\vdash b=a\)]
\end{theorem}
\begin{proof}
	\[
	\begin{array}{llll}
		1 & (1) & a=b & \rA \\
		& (2) & a=a & \rII \\
		1 & (3) & b=a & \rIE{1,2} \\
	\end{array}
	\]
\end{proof}

\label{aIdbwbIdcImpaIdc}
\begin{theorem}[\(a=b,b=c\vdash a=c\) (Transitivität)]
\end{theorem}
\begin{proof}
	\[
	\begin{array}{llll}
		1 & (1) & a=b & \rA \\
		2 & (2) & b=c & \rA \\
		1,2 & (3) & a=c & \rIE{1,2} \\
	\end{array}
	\]
\end{proof}

\label{aIdbwcIdbImpaIdc}
\begin{theorem}[\(a=b,c=b\vdash a=c\) ]
\end{theorem}
\begin{proof}
	\[
	\begin{array}{llll}
		1 & (1) & a=b & \rA \\
		2 & (2) & c=b & \rA \\
		1,2 & (3) & a=c & \rIE{1,2} \\
	\end{array}
	\]
\end{proof}



\section{Gleichheitsketten-Notation}
\label{Gleichheitsketten-Notation}
In mathematischen Beweisen ist es oft hilfreich, Gleichheitsbeziehungen in kompakter Form zu notieren, insbesondere wenn diese durch eine Folge von Regeln begründet werden. Dazu führen wir eine Notation ein, die es erlaubt, jede Gleichheit in einer Gleichungskette durch die entsprechende Regel zu begründen. Diese basiert auf der Transitivität der Gleichheit \(a=b, b=c \vdash a=c\).

\subsection*{Definition der Notation}

Eine Gleichheitskette ist eine Darstellung der Form:
\[
a = b \stackrel{\text{Regel}_1}{=} c \stackrel{\text{Regel}_2}{=} d,
\]
wobei jeder Übergang \(x \stackrel{\text{Regel}}{=} y\) durch die angegebene Regel \(\text{Regel}\) gerechtfertigt ist. 

Sofern die Regelnamen sehr lang oder nicht offensichtlich sind und das zu einer unübersichtlichen Darstellung in der Gleichheitskette führen würde, nutzen wir alternativ die tabellarische Notation:
\[
\begin{array}{llclll}
	1 & (1) & a & = & b & \rA \\
	1 & (2) &   & = & c & \text{Regel}_1 \\
	1 & (3) &   & = & d & \text{Regel}_2 \\
        1 & (4) &   & = & d & \text{Transitivität} \\
\end{array}
\]

Beide Notationen stellen eine Abkürzung der klassischen Beweistabellen dar, beispielsweise:
\[
\begin{array}{llll}
	1 & (1) & a=b & \rA \\
	1 & (2) & b=c & \text{Regel}_1 \\
	1 & (3) & c=d & \text{Regel}_2 \\
	1 & (4) & a=d & \text{Transitivität} \\
\end{array}
\]

\subsection*{Bemerkung zur Effizienz der Notation}

Die hier eingeführten Gleichheitsketten-Notationen erlauben es, Beweise übersichtlicher darzustellen, ohne an mathematischer Präzision einzubüßen. Die klassische inline-Notation eignet sich besonders für kurze Transformationen, während die tabellarische Darstellung eine strukturiertere Übersicht bietet, insbesondere bei komplexeren Beweisen mit längeren Regelnamen.


\subsection{Theoreme mit dem Nicht-Gleichheitszeichen}

\label{aNotEqualsbImpbNotEqualsa}
\begin{theorem}[\(a\neq b\vdash b\neq a\)]
\end{theorem}
\begin{proof}
	\[
	\begin{array}{llll}
		1 & (1) & a\neq b & \rA \\
		1 & (2) & \neg a=b & \rNIE{1} \\
            3 & (3) & b=a & \rA \\
            3 & (4) & a=b & \aIdbImpbIda{3} \\
            1,3 & (5) & \bot & \rBI{4,2} \\
            1 & (6) & \neg b=a & \rCI{3,5} \\
            1 & (7) & b\neq a & \rNII{6} \\
	\end{array}
	\]
\end{proof}

\chapter{Mengenlehre}

Die Mengenlehre ist ein fundamentaler Teil der Mathematik, der die Grundlage für viele andere Bereiche bildet. In diesem Kapitel werden wir die Zermelo-Fraenkel (ZF) Axiome der Mengenlehre einführen und diskutieren. Dabei bezeichnen \( A \), \( B \), \( C \) und \( D \) stets Mengen, es sei denn, es wird ausdrücklich etwas anderes angegeben. Alle Variablen, die als Mengen bezeichnet werden, sind implizit durch Allquantoren gebunden, es sei denn, es wird ein anderer Quantor verwendet. Das bedeutet, dass Aussagen wie „\( A = B \)“ oder „\( A \neq B \)“ für alle Mengen \( A \) und \( B \) gelten, ohne dass dies explizit angegeben werden muss.

\section{Grundlagen der Mengenlehre}

\begin{definition}[Elementzugehörigkeit]
    \textbf{Domäne}: Die Domäne besteht aus allen Mengen, wie sie durch die Zermelo-Fraenkel-Axiome weiter unten definiert sind.
    
    Die \emph{Elementzugehörigkeit} \( \in \) ist eine binäre Relation zwischen Mengen. Für Mengen \( x \) und \( M \) bedeutet \( x \in M \), dass \( x \) ein Element von \( M \) ist.
\end{definition}

\subsection{Allquantor und Existenzquantor in Mengen}

Die Regeln für \(\forall x \in A(P(x))\) und \(\exists x \in A(P(x))\) können aus den allgemeinen Regeln für den Allquantor und den Existenzquantor abgeleitet werden, indem wir die folgenden Definitionen verwenden:

\[
\forall x \in A(P(x)) \coloneqq \forall x(x \in A \rightarrow P(x))
\]
\[
\exists x \in A(P(x)) \coloneqq \exists x(x \in A \wedge P(x))
\]

\begin{definition}[Def. des Allquantors für Mengen ($Df. Set(\forall)$ )]
	Der Ausdruck \(\forall x \in A(P(x))\) wird definiert und abgekürzt als:
	\[
	\forall x \in A(P(x)) \coloneqq\forall x(x \in A \rightarrow P(x))
	\]
\end{definition}
\label{rule:rSetU} 

\subsubsection{Regeln für den Allquantor für Mengen}
\label{rule:rSetUI} \label{rule:rSetUE}

% Einführungsregel für den Allquantor für Mengen (\(\forall_{Set} I\))
Die Einführungsregel für den Allquantor in Bezug auf Mengen (\(\forall_{Set} I\)) lässt sich wie folgt formulieren:
Wenn wir eine Aussage \(P(x)\) für ein beliebiges, aber fest gewähltes Element \(x\) aus einer Menge \(A\) beweisen können, und \(x\) kommt in den Annahmen, aus denen \(P(x)\) abgeleitet wird, nicht vor, dann können wir die allgemeine Aussage \(\forall x \in A(P(x))\) ableiten.

\[
\begin{array}{llll}
	1 & (1) & x\in A & \rA \\
	1,i & (2) & P(x) & ... \\
	i & (3) & \forall x \in A(P(x)) & \rSetUIa{1,2} \\
\end{array}
\]

\[
\begin{array}{llll}
	i & (1) & \forall x(x\in A\rightarrow P(x)) & ... \\
	i & (3) & \forall x \in A(P(x)) & \rSetUIb{1,2} \\
\end{array}
\]

% Eliminierungsregel für den Allquantor für Mengen (\(\forall_{Set} E\))
Die Eliminierungsregel für den Allquantor in Bezug auf Mengen (\(\forall_{Set} E\)) formuliert sich folgendermaßen:
Aus einer allgemeinen Aussage der Form \(\forall x \in A(P(x))\) können wir eine spezielle Aussage \(P(t)\) für ein beliebiges Element \(t\) aus der Menge \(A\) ableiten.

\[
\begin{array}{llll}
	i & (1) & \forall x \in A(P(x)) & ... \\
	i & (2) & \forall x(x\in A\rightarrow P(x)) & \rSetUEa{1} \\
\end{array}
\]

\[
\begin{array}{llll}
	i & (1) & \forall x \in A(P(x)) & ... \\
	i & (2) & t\in A\rightarrow P(t) & \rSetUEb{1} \\
\end{array}
\]

\[
\begin{array}{llll}
	i & (1) & \forall x \in A(P(x)) & ... \\
        j & (2) & t \in A & ... \\
	i,j & (3) & P(t) & \rSetUEc{1,2} \\
\end{array}
\]

Hierbei bezieht sich \(i\) auf eine Liste von Annahmen und \(t\) ist ein spezifisches Element der Menge \(A\).

\begin{definition}[Def. des Existenzquantors für Mengen ($Df. Set(\exists)$)]
	Der Ausdruck \(\exists x \in A(P(x))\) wird definiert und abgekürzt als:
	\[
	\exists x \in A(P(x)) \coloneqq \exists x(x \in A \wedge P(x))
	\]
\end{definition}
\label{rule:rSetE} 

\subsubsection{Definition und Regeln für den Existenzquantor für Mengen}
\label{rule:rSetEI} \label{rule:rSetEE}

% Einführungsregel für den Existenzquantor für Mengen (\(\exists_{Set} I\))
Die Einführungsregel für den Existenzquantor in Bezug auf Mengen (\(\exists_{Set} I\)) lässt sich wie folgt formulieren:
Wenn wir eine Aussage \(P(a)\) für ein spezifisches Element \(a\) aus einer Menge \(A\) beweisen können, wobei \(a\) tatsächlich ein Element von \(A\) ist, dann können wir daraus die Existenzaussage \(\exists x \in A(P(x))\) ableiten.

\[
\begin{array}{llll}
	i & (1) & a\in A & ... \\
	i,j & (2) & P(a) & ... \\
	i,j & (3) & \exists x \in A(P(x)) & \rSetEIa{1,2} \\
\end{array}
\]

\[
\begin{array}{llll}
	i & (1) & \exists x(x\in A\land P(x)) & ... \\
	i & (3) & \exists x \in A(P(x)) & \rSetEIb{1,2} \\
\end{array}
\]

% Eliminierungsregel für den Existenzquantor für Mengen (\(\exists_{Set} E\))
Die Eliminierungsregel für den Existenzquantor in Bezug auf Mengen (\(\exists_{Set} E\)) formuliert sich folgendermaßen:
Aus einer Existenzaussage der Form \(\exists x \in A(P(x))\) können wir, unter gewissen Bedingungen, eine spezielle Aussage \(P(t)\) für ein Element \(t\) aus der Menge \(A\) ableiten. Die Variable \(t\) wurde dabei in den bisherigen Schritten des Beweises noch nicht verwendet.

\[
\begin{array}{llll}
	i & (1) & \exists x \in A(P(x)) & ... \\
	i & (2) & \exists x(x\in A\land P(x)) & \rSetEEa{1} \\
\end{array}
\]

\[
\begin{array}{llll}
	i & (1) & \exists x \in A(P(x)) & ... \\
	i & (2) & t\in A\land P(t) & \rSetEEa{1} \\
\end{array}
\]

\[
\begin{array}{llll}
	i & (1) & \exists x \in A(P(x)) & ... \\
	2 & (2) & t\in A\land P(t) & \rA \\
	2,j & (3) & Q & ... \\
	i,j & (4) & t\in A\land P(t) & \rSetEEb{1,2,3} \\
\end{array}
\]

\textbf{Aufteilung der Tabelle bei nicht expliziter Darstellung von \(t \in A\):}

Wenn die Zugehörigkeit \(t \in A\) für den Beweis nicht explizit verwendet werden muss, kann die Tabelle in vereinfachter Form aufgeteilt werden:

\[
\begin{array}{llll}
	i & (1) & \exists x \in A(P(x)) & \dots \\
\end{array}
\]

Wähle ein spezifisches Element \(t \in A\) so, dass:

\[
\begin{array}{llll}
	2 & (3) & P(t) & \rA \\
      2,j & (4) & Q & \text{weitere Schritte} \\
\end{array}
\]


\(i, j\) sind dabei Listen von Annahmen. 

\begin{definition}[Def. des Eindeutigkeitsquantors für Mengen (\(Df. Set(\exists !)\))]
	Der Ausdruck \(\exists ! x \in A(P(x))\) wird definiert und abgekürzt als:
	\[
	\exists ! x \in A(P(x)) \coloneqq \exists! x (x \in A \wedge P(x))
	\]
\end{definition}

\subsubsection{Definition und Regeln für den Eindeutigkeitsquantor für Mengen}
\label{rule:rSetUEI} \label{rule:rSetUEE}

% Regel für die Überführung von \(\exists ! x \in A(P(x))\) zu \(\exists !x(x\in A\land P(x))\) und umgekehrt

Die Regel, die zeigt, dass aus der Aussage \(\exists ! x \in A(P(x))\) folgt, dass genau ein \(x\) existiert, das \(x \in A \land P(x)\) erfüllt, kann wie folgt ausgedrückt werden:
\[
\begin{array}{llll}
	i & (1) & \exists ! x \in A(P(x)) & ... \\
	i & (2) & \exists !x(x\in A\land P(x)) & \rSetUEE{1} \\
\end{array}
\]

Umgekehrt formulieren wir nun die Regel, die zeigt, dass aus der formalen Aussage \(\exists !x(x\in A\land P(x))\) folgt, dass genau ein \(x\) in der Menge \(A\) mit der Eigenschaft \(P(x)\) existiert, ist wie folgt:
\[
\begin{array}{llll}
	i & (1) & \exists !x(x\in A\land P(x)) & ... \\
	i & (2) & \exists ! x \in A(P(x)) & \rSetUEI{1} \\
\end{array}
\]

\(i\) ist dabei eine Liste von Annahmen.

\begin{definition}[Quantoren mit Mengen für mehrere Variablen]
    Sei \( M \) eine Menge, \( k, n \in \mathbb{N} \) mit \( k \leq n \), und \( P(x_k, x_{k+1}, \ldots, x_n) \) ein Prädikat mit den Variablen \( x_k, x_{k+1}, \ldots, x_n \). Dann definieren wir:
    \begin{enumerate}
        \item \( \forall x_k, x_{k+1}, \ldots, x_n \in M \, (P(x_k, x_{k+1}, \ldots, x_n)) \)\,\(\coloneqq\)\,
        \[
            \forall x_k \in M \, \forall x_{k+1} \in M \, \ldots \, \forall x_n \in M \, (P(x_k, x_{k+1}, \ldots, x_n))
        \]
        
        \item \( \exists x_k, x_{k+1}, \ldots, x_n \in M \, (P(x_k, x_{k+1}, \ldots, x_n)) \)\,\(\coloneqq\)\,
        \[
            \exists x_k \in M \, \exists x_{k+1} \in M \, \ldots \, \exists x_n \in M \, (P(x_k, x_{k+1}, \ldots, x_n))
        \]
    \end{enumerate}
\end{definition}

\begin{remark}
\textbf{Hinweis zur Variablenbezeichnung:}\\
Die Indizes \(k, k+1, \ldots, n\) (oder entsprechend andere Buchstaben/Indizes, z.\,B.\ \(a, b, c,\ldots\)) dienen nur als Platzhalter. Entscheidend ist, dass \emph{dieselbe Anzahl} von Variablen eingeführt wird und sie \emph{konsistent} verwendet werden. Man kann die Indizes also frei anpassen, solange klar bleibt, dass sich hierdurch weder die Anzahl der Variablen noch deren logische Rolle ändert. 

Ebenso wichtig ist, dass die gewählten Variablen \emph{neu} sind, d.\,h.\ sie dürfen in vorherigen Schritten des Beweises nicht bereits als freie oder gebundene Variablen aufgetreten sein, und dass keine Kollision mit existierenden Namen entsteht.
\end{remark}

\subsubsection{Regeln für Quantoren mit Mengen und mehreren Variablen}

\paragraph{Allquantor mit mehreren Variablen}
\label{rule:rSetUm} 
\[
\begin{array}{llll}
    i & (1) & \forall x_k, x_{k+1}, \ldots, x_n \in M \, (P(x_k, x_{k+1}, \ldots, x_n)) & \dots \\
    i & (2) & \forall x_k \in M \, \forall x_{k+1} \in M \, \ldots \, \forall x_n \in M \, (P(x_k, x_{k+1}, \ldots, x_n)) 
            & \rSetUm{1} \\
\end{array}
\]

\[
\begin{array}{llll}
    i & (1) & \forall x_k \in M \, \forall x_{k+1} \in M \, \ldots \, \forall x_n \in M \, (P(x_k, x_{k+1}, \ldots, x_n)) & \dots \\
    i & (2) & \forall x_k, x_{k+1}, \ldots, x_n \in M\, (P(x_k, x_{k+1}, \ldots, x_n)) & \rSetUm{1} \\
\end{array}
\]

\paragraph{Existenzquantor mit mehreren Variablen}
\label{rule:rSetEm}

\[
\begin{array}{llll}
    i & (1) & \exists x_k, x_{k+1}, \ldots, x_n \in M \, (P(x_k, x_{k+1}, \ldots, x_n)) & \dots \\
    i & (2) & \exists x_k \in M \, \exists x_{k+1} \in M \, \ldots \, \exists x_n \in M \, (P(x_k, x_{k+1}, \ldots, x_n)) 
            & \rSetEm{1} \\
\end{array}
\]

\[
\begin{array}{llll}
    i & (1) & \exists x_k \in M \, \exists x_{k+1} \in M \, \ldots \, \exists x_n \in M \, (P(x_k, x_{k+1}, \ldots, x_n)) & \dots \\
    i & (2) & \exists x_k, x_{k+1}, \ldots, x_n \in M \, (P(x_k, x_{k+1}, \ldots, x_n)) & \rSetEm{1} \\
\end{array}
\]

\noindent
\textbf{Aufteilung der Tabelle bei nicht expliziter Darstellung von \(x_k, \ldots, x_n \in M\):}

Wenn die Zugehörigkeit der neuen Variablen \(a_k, \ldots, a_n \in M\) für den Beweis nicht explizit dargestellt werden muss, kann die Tabelle in vereinfachter Form aufgeteilt werden. Dabei gilt:

\textit{Die Variablen \(a_k, \ldots, a_n\) müssen neu sein, d.\,h.\ sie dürfen in den vorhergehenden Schritten des Beweises weder als freie noch als gebundene Variablen aufgetreten sein.}

\[
\begin{array}{llll}
    i & (1) & \exists x_k \in M \, \exists x_{k+1} \in M \, \ldots \, \exists x_n \in M \, (P(x_k, x_{k+1}, \ldots, x_n)) & \dots \\
    \multicolumn{4}{l}{\text{Wähle \(a_k, \ldots, a_n \in M\), sodass:}}\\
    2 & (2) & P(a_k, a_{k+1}, \ldots, a_n) & \rA \\
    i,j,2 & (3) & Q & \dots \\
    i,j & (4) & Q & \rSetEm{1,2,3} \\
\end{array}
\]

\bigskip

\noindent
\textbf{Kompakte Einführungsregel für \(\exists x_k, \ldots, x_n \in M\) (Alternative Darstellung):}
\[
\begin{array}{llll}
    i & (1) & a_k, a_{k+1}, \ldots, a_n \in M & \dots \\
    i,j & (2) & P(a_k, a_{k+1}, \ldots, a_n) & \dots \\
    i,j & (3) & \exists x_k, x_{k+1}, \ldots, x_n \in M \, (P(x_k, x_{k+1}, \ldots, x_n)) & \rSetEm{1,2} \\
\end{array}
\]

\noindent
Hier wird kompakt vorausgesetzt, dass alle \(a_k, \ldots, a_n\) in \(M\) liegen (Zeile (1)) und dass die Aussage \(P(a_k,\ldots,a_n)\) für diese Elemente gilt (Zeile (2)). Anschließend folgt (Zeile (3)) die Existenzaussage.

\paragraph{Eindeutigkeitsoperator mit mehreren Variablen}
\label{rule:rSetEEm} 
\[
\begin{array}{llll}
    i & (1) & \exists! x_k, x_{k+1}, \ldots, x_n \in M \, (P(x_k, x_{k+1}, \ldots, x_n)) & \dots \\
    i & (2) & \exists! x_k \in M \, \exists! x_{k+1} \in M \, \ldots \, \exists! x_n \in M \, (P(x_k, x_{k+1}, \ldots, x_n)) 
            & \rSetEEm{1} \\
\end{array}
\]

\[
\begin{array}{llll}
    i & (1) & \exists! x_k \in M \, \exists! x_{k+1} \in M \, \ldots \, \exists! x_n \in M \, (P(x_k, x_{k+1}, \ldots, x_n)) & \dots \\
    i & (2) & \exists! x_k, x_{k+1}, \ldots, x_n \in M\, (P(x_k, x_{k+1}, \ldots, x_n)) & \rSetEEm{1} \\
\end{array}
\]

\noindent
\textbf{Auch für den Eindeutigkeitsoperator gelten alle obigen Hinweise analog:} 
Die jeweiligen Indizes bzw.\ Variablen müssen \emph{neu} sein und dürfen weder als freie noch als gebundene Variablen 
zuvor im Beweis verwendet worden sein. Die konkrete Wahl der Indizes oder Buchstaben ist frei, 
sofern sie klar in Bezug auf die Menge \(M\) und die Aussage \(P(\dots)\) definiert werden. 

\section{Einführung der Zermelo-Fraenkel-Axiome}

Nachdem wir die grundlegenden Begriffe und Notationen eingeführt haben, wenden wir uns nun den Zermelo-Fraenkel-Axiomen zu, die das Fundament der modernen Mengenlehre bilden. Diese Axiome definieren, wie Mengen gebildet werden können und welche Eigenschaften sie besitzen.

\begin{definition}[Begriff der Menge]
    Der \textbf{Begriff der Menge} wird durch das \textbf{Element-Symbol} \(\in\) \textbf{implizit definiert}. Das Symbol \(\in\) ist ein binäres Prädikat, das die Mitgliedschaft zwischen einem Element und einer Menge ausdrückt, also \(x \in y\) bedeutet, dass \(x\) ein Element von \(y\) ist. Die Eigenschaften von \(\in\) werden durch die folgenden Axiome der Zermelo-Fraenkel-Mengenlehre festgelegt, welche zusammen die Menge von Aussagen \(\Phi(\in)\) bilden:

    \begin{itemize}
        \item \textbf{Axiom der Extensionalität}: 
        \[
        \forall A \forall B \left( \forall x \, (x \in A \leftrightarrow x \in B) \leftrightarrow A = B \right).
        \]
        \item \textbf{Axiom der leeren Menge}:
        \[
        \exists \emptyset \, \forall x \, (x \notin \emptyset).
        \]
        Die \textbf{leere Menge} \(\emptyset\) ist eindeutig bestimmt durch das Axiom der Extensionalität (vgl. \(\ExOFaxLpxNotinORpImpOEqualsEmptyset{}\)).
        
        \item \textbf{Das Aussonderungsaxiom}:
        \[
        \forall A \forall P \, \exists B \, \forall x \, \left( x \in B \leftrightarrow (x \in A \land P(x)) \right).
        \]
        Eine \textbf{ausgesonderte Menge} \(\{ x \in A \mid P(x) \}\) wird durch dieses Axiom aus einer bestehenden Menge \(A\) und einer Eigenschaft \(P\) gebildet und ist eindeutig durch das Axiom der Extensionalität bestimmt (vgl. \(\FaxLpxInBLrPLpxRpRpAndFaxLpxInCLrPLpxRpRpImpBEqualsC{}\)).
        
        \item \textbf{Axiom der Paarmenge}: 
        \[
        \forall A \forall B \, \exists \{A, B\} \, \forall x \, \left( x \in \{A, B\} \leftrightarrow (x = A \lor x = B) \right).
        \]
        Die \textbf{Paarmenge} \(\{A, B\}\) enthält genau die Elemente \(A\) und \(B\) und ist somit eindeutig durch das Axiom der Extensionalität bestimmt (vgl. \(\FaxLpxInCLrLpxEqualsAOrxEqualsBRpRpAndFaxLpxInDLrLpxEqualsAOrxEqualsBRpRpImpCEqualsD{} \)).
        
        \item \textbf{Axiom der Vereinigung}:
        \[
        \forall A \, \exists \bigcup A \, \forall x \, \left( x \in \bigcup A \leftrightarrow \exists B \, (B \in A \land x \in B) \right).
        \]
        Die \textbf{Vereinigungsmenge} \(\bigcup A\) besteht aus allen Elementen der Elemente von \(A\) und ist eindeutig durch das Axiom der Extensionalität bestimmt (vgl. \(\FaxLpxInBLrExDLpDInAAndxInDRpRpAndFaxLpxInCLrExDLpDInAAndxInDRpRpImpBEqualsC{}\)).
        
        \item \textbf{Axiom der Potenzmenge}: 
        \[
        \forall A \, \exists \mathcal{P}(A) \, \forall x \, \left( x \in \mathcal{P}(A) \leftrightarrow x \subseteq A \right).
        \]
        Die \textbf{Potenzmenge} \(\mathcal{P}(A)\) umfasst alle Teilmengen von \(A\) und ist somit eindeutig durch das Axiom der Extensionalität bestimmt (vgl. \(\FaxLpxInELrLpExaInAExbInBLpxEqualsLpawbRpRpRpRpAndFaxLpxInFLrLpExaInAExbInBLpxEqualsLpawbRpRpRpRpImpEEqualsF{}\)).
        
        \item \textbf{Axiom der Unendlichkeit}:
        \[
        \exists A (\emptyset \in A \land \forall x\in A (x \cup \{x\} \in A))
        \]
        
        \item \textbf{Axiom der Regularität}:
        \[
        \forall A \, \left( A \neq \emptyset \rightarrow \exists x \in A \, (x \cap A = \emptyset) \right).
        \]
        Dieses Axiom stellt sicher, dass jede nicht-leere Menge ein \textbf{Minimalelement} enthält, wodurch zyklische Mitgliedschaften verhindert werden.
        
        \item \textbf{Axiom der Ersetzung}:
        \[
        \forall A \, \forall F \, \exists B \, \forall y \, \left( y \in B \leftrightarrow \exists x \, (x \in A \land y = F(x)) \right).
        \]
        Dieses Axiom ermöglicht die Bildung neuer Mengen durch die Anwendung von Funktionen auf bestehende Mengen und garantiert die Eindeutigkeit der resultierenden Menge durch das Axiom der Extensionalität.
        
        \item \textbf{Axiom des Auswahlaxioms (AC)}:
        \[
        \forall \mathcal{A} \, \left( \mathcal{A} \neq \varnothing \land \forall A \in \mathcal{A} \, (A \neq \varnothing) \right) \rightarrow \exists f \, \left( \text{Funktion}(f) \land \text{Wahlfunktion}(f, \mathcal{A}) \right).
        \]
        Das \textbf{Auswahlaxiom} garantiert die Existenz von Wahlfunktionen für nicht-leere Familien von nicht-leeren Mengen und ist somit eindeutig durch das Axiom der Extensionalität bestimmt.
    \end{itemize}
\end{definition}






\section{Das Extensionalitätsaxiom}

Das erste Axiom, das wir betrachten, ist das Extensionalitätsaxiom. Es besagt, dass zwei Mengen gleich sind, wenn sie genau die gleichen Elemente haben. Formal ausgedrückt sagt man, dass für alle Mengen $A$ und $B$ gilt

\label{AIdBEqvFaxLpxInALrxInBRp}
\[
A = B \dashv\vdash\forall x (x \in A \leftrightarrow x \in B)
\]

Dieses Axiom ist fundamental für die Definition von Gleichheit in der Mengenlehre.

\label{ANotIdBEqvExxLpxNotInAAndxInBRpOrExxLpxInAAndxNotInBRp}
\begin{theorem}[\(A \neq B \dashv\vdash \exists x (x \notin A\land x\in B) \lor  \exists x (x \in A\land x\notin B)\)]
\end{theorem}
\begin{proof}
	\(\vdash:\)
	\[
	\begin{array}{lll p{7cm}}
		1 & (1) & A \neq B & \rA \\
		& (2) & A=B\leftrightarrow \forall x(x\in A\leftrightarrow x\in B) & \AIdBEqvFaxLpxInALrxInBRp{}  \\
            1 & (3) & \neg\forall x(x\in A\leftrightarrow x\in B) & \PLrQwnPImpnQ{2,1}  \\	
		1 & (4) & \exists x(x \notin A\land x\in B) \lor \exists x(x \in A\land x\notin B) & \nFaxLpPLpxRpLrQLpxRpRpEqvExxLpPLpxRpAndnQLpxRpRpOrExxLpQLpxRpAndnPLpxRpRp{3}  \\
	\end{array}
	\]
	\(\dashv:\)
	\[
	\begin{array}{lll p{7cm}}
		1 & (1) & \exists x (x \notin A\land x\in B) \lor  \exists x (x \in A\land x\notin B) & \rA \\
		1 & (2) & \neg\forall x(x\in A\leftrightarrow x\in B) & \nFaxLpPLpxRpLrQLpxRpRpEqvExxLpPLpxRpAndnQLpxRpRpOrExxLpQLpxRpAndnPLpxRpRp{1} \\
  		& (3) & A=B\leftrightarrow \forall x(x\in A\leftrightarrow x\in B) & \AIdBEqvFaxLpxInALrxInBRp{}  \\
            1 & (4) & A \neq B & \PLrQwnQImpnP{3,2} \\
	\end{array}
	\]
\end{proof}

\subsubsection{Regeln für die Gleichheit von Mengen}
\label{rule:rIEaSet} \label{rule:rIEbSet} \label{rule:rIISet}

Das Extensionalitätsaxiom, oft dargestellt durch das Symbol \(=\), definiert die Gleichheit von Mengen. Basierend auf diesem Axiom können wir zwei grundlegende Regeln für die Gleichheit von Mengen formulieren: die Einführungs- und die Eliminierungsregel.

Die Einführungsregel für die Gleichheit von Mengen (\(= I\)) besagt, dass wenn wir für jedes Element \(x\) zeigen können, dass \(x\) in beiden Mengen \(A\) und \(B\) enthalten ist oder nicht enthalten ist, dann können wir daraus schließen, dass \(A\) und \(B\) gleich sind.

\[
\begin{array}{llll}
	1,i & (1) & \forall x(x \in A \leftrightarrow x \in B) & ... \\
	i & (2) & A = B & \rIISet{1} \\
\end{array}
\]

Die Eliminierungsregel für die Gleichheit von Mengen (\(= E\)) besagt, dass wenn wir wissen, dass \(A\) und \(B\) gleich sind und wir ein Element \(x\) in \(A\) haben, wir daraus schließen können, dass \(x\) auch in \(B\) enthalten ist, und umgekehrt.

\[
\begin{array}{llll}
	i & (1) & A = B & ... \\
	j & (2) & x \in A & ... \\
	i,j & (3) & x \in B & \rIEaSet{1,2} \\
\end{array}
\]

\[
\begin{array}{llll}
	i & (1) & A = B & ... \\
	j & (2) & x \in B & ... \\
	i,j & (3) & x \in A & \rIEbSet{1,2} \\
\end{array}
\]

\(i,j\) sind dabei Listen von Annahmen.

\subsubsection{Regeln für die Ungleichheit von Mengen}
\label{rule:rNIIaSet} \label{rule:rNIIbSet} \label{rule:rNIESet}

Das Theorem über die Ungleichheit von Mengen definiert die Ungleichheit von Mengen. Basierend auf diesem Theorem können wir zwei grundlegende Regeln formulieren: die Einführungs- und die Eliminierungsregel.

Die Einführungsregel für die Ungleichheit von Mengen ($\neq I$) besagt, dass wenn wir ein Element \(x\) finden können, das entweder in \(A\) aber nicht in \(B\), oder in \(B\) aber nicht in \(A\) enthalten ist, wir daraus schließen können, dass \(A\) und \(B\) ungleich sind.

\[
\begin{array}{llll}
	i & (1) & x \notin A \land x \in B & ... \\
	i & (2) & A \neq B & \rNIIaSet{1} \\
\end{array}
\]

\[
\begin{array}{llll}
	i & (1) & x \in A \land x \notin B & ... \\
	i & (2) & A \neq B & \rNIIbSet{1} \\
\end{array}
\]

Die Eliminationsregel für die Ungleichheit von Mengen \(\neg E\) besagt, dass wenn wir wissen, dass \(A\) und \(B\) ungleich sind, wir daraus schließen können, dass es mindestens ein Element \(x\) gibt, das entweder nur in \(A\) oder nur in \(B\) enthalten ist.

\[
\begin{array}{llll}
	i & (1) & A \neq B & \rA \\
	i & (2) & \exists x (x \notin A \land x \in B) \lor \exists x (x \in A \land x \notin B) & \rNIESet{1} \\
\end{array}
\]




\section{Definition der Teilmenge und Beweis der Äquivalenz}

\begin{definition}[Teilmenge]
	Für zwei Mengen A und B, sagen wir, dass A eine Teilmenge von B ist (geschrieben als \(A \subseteq B\)), wenn jedes Element von A auch ein Element von B ist. Formal ausgedrückt: 
	\[
	\forall A\forall B(A \subseteq B \leftrightarrow \forall x (x \in A \rightarrow x \in B)).
	\]
\end{definition}

\subsubsection{Regeln für die Teilmenge}
\label{rule:SubsetE} \label{rule:SubsetI}

Die Teilmenge, oft dargestellt durch das Symbol \(\subseteq\), ist eine Beziehung zwischen zwei Mengen. Basierend auf unserer Definition für die Teilmenge, können wir zwei grundlegende Regeln für die Teilmenge einfügen: die Einführungs- und die Eliminierungsregel.

Die Einführungsregel für die Teilmenge (\(\subseteq I\)) besagt, dass wenn wir für jedes Element \(x\) aus einer Menge \(A\) zeigen können, dass \(x\) auch in einer Menge \(B\) enthalten ist, dann können wir daraus schließen, dass \(A\) eine Teilmenge von \(B\) ist. Dabei darf x in keiner der Annahmen aus i vorkommen.

\[
\begin{array}{llll}
	1   & (1) & x \in A & \rA \\
	1,i & (2) & x \in B & ... \\
	i   & (3) & A \subseteq B & \subseteqIa{1,2} \\
\end{array}
\]

oder

\[
\begin{array}{llll}
	i & (1) & x\in A\rightarrow x \in B & ... \\
	i & (2) & A \subseteq B & \subseteqIb{1} \\
\end{array}
\]

Die Eliminierungsregel für die Teilmenge (\(\subseteq E\)) besagt, dass wenn wir wissen, dass \(A\) eine Teilmenge von \(B\) ist und wir ein Element \(x\) in \(A\) haben, wir daraus schließen können, dass \(x\) auch in \(B\) enthalten ist.

\[
\begin{array}{llll}
	i & (1) & x \in A & ... \\
	j & (2) & A \subseteq B & ... \\
	i,j & (3) & x \in B & \subseteqE{1,2} \\
\end{array}
\]

\(i,j\) sind dabei Listen von Annahmen.


\begin{definition}[Keine Teilmenge]
	Für zwei Mengen \(A\) und \(B\), sagen wir, dass \(A\) keine Teilmenge von \(B\) ist (geschrieben als \(A \nsubseteq B\)), wenn es mindestens ein Element in \(A\) gibt, das nicht in \(B\) ist. Formal ausgedrückt:
	\[
	A \nsubseteq B \leftrightarrow \neg(A\subseteq B).
	\]
\end{definition}



Unter Verwendung dieser Definition und der gegebenen Regeln der Aussagenlogik und Prädikatenlogik können wir das folgende Argument beweisen: 
\begin{theorem}[Theorem über die Gleichheit von Mengen]
	Unter Verwendung des Extensionalitätsaxioms gilt für alle Mengen $A$ und $B$:
	\[
	A \subseteq B\land B \subseteq A\dashv\vdash A = B 
	\]
\end{theorem}
\begin{proof}
	\(\vdash:\)
	\[
	\begin{array}{llll}
		1 & (1) & A \subseteq B\land B\subseteq A & \rA \\
		1 & (2) & A \subseteq B & \rAEa{1} \\
		1 & (3) & B \subseteq A & \rAEb{2} \\
		4 & (4) & x\in A & \rA \\
		1,4 & (5) & x\in B & \subseteqE{4,2} \\
		1 & (6) & x\in A\rightarrow x\in B & \rRI{4,5} \\
		7 & (7) & x\in B & \rA \\
		1,7 & (8) & x\in A & \subseteqE{7,3} \\
		1 & (8) & x\in B\rightarrow x\in A & \rRI{7,8} \\
		1 & (9) & x\in A\leftrightarrow x\in B & \rLRI{6,8} \\
		1 & (10) & \forall x(x\in A\leftrightarrow x\in B) & \rAI{9} \\		
		1 & (11) & A=B & \rIISet{10} \\
	\end{array}
	\]
	\(\dashv:\)
	\[
	\begin{array}{llll}
		1 & (1) & A = B & \rA \\
		2 & (2) & x\in A & \rA \\		
		1,2 & (3) & x\in B & \rIEaSet{1,2} \\
		1 & (3) & A\subseteq B & \subseteqIa{1,2} \\
		4 & (4) & x\in B & \rA \\		
		1,4 & (5) & x\in A & \rIEbSet{1,4} \\
		1 & (6) & B\subseteq A & \subseteqIa{4,5} \\
		1 & (7) & A\subseteq B\land B\subseteq A & \rAI{3,6} \\
	\end{array}
	\]
\end{proof}

\subsubsection{Regeln basierend auf der Definition der Mengenidentität}
\label{rule:rIIaSet} \label{rule:subsetEqI}

Das Theorem über die Gleichheit von Mengen stellt einen Zusammenhang zwischen der Gleichheit von Mengen und der Teilmenge her. Basierend auf diesem Theorem können wir Regeln für die Einführung und Eliminierung der Gleichheit von Mengen sowie der Teilmenge formulieren.

Die Einführungsregel für die Gleichheit von Mengen basierend auf der Teilmenge (\(= I\)) besagt, dass wenn wir zeigen können, dass \(A\) eine Teilmenge von \(B\) und \(B\) eine Teilmenge von \(A\) ist, dann können wir daraus schließen, dass \(A\) und \(B\) gleich sind.

\[
\begin{array}{llll}
	i & (1) & A \subseteq B & ... \\
	j & (2) & B \subseteq A & ... \\
	i,j & (3) & A = B & \rIIaSet{1,2} \\
\end{array}
\]

Die Eliminierungsregel für die Gleichheit von Mengen basierend auf der Teilmenge (\(= E\)) besagt, dass wenn wir wissen, dass \(A\) und \(B\) gleich sind, dann können wir daraus schließen, dass \(A\) eine Teilmenge von \(B\) und \(B\) eine Teilmenge von \(A\) ist.

\[
\begin{array}{llll}
	i & (1) & A = B & ... \\
	i & (2) & A \subseteq B & \subsetEqIa{1} \\
	i & (3) & B \subseteq A & \subsetEqIb{1} \\
\end{array}
\]

\(i\) sind dabei Listen von Annahmen.

\label{ASubseteqBwBSubseteqCImpASubseteqC}
\begin{theorem}[\(A \subseteq B, B \subseteq C \vdash A \subseteq C\)]
	Für alle Mengen \(A\), \(B\) und \(C\) gilt: Wenn \(A \subseteq B\) und \(B \subseteq C\), dann \(A \subseteq C\). Die Teilmengenbeziehung ist also Transitiv.
\end{theorem}
\begin{proof}
	\[
	\begin{array}{llll}
		1 & (1) & A \subseteq B & \rA \\
		2 & (2) & B \subseteq C & \rA \\
		3 & (3) & x \in A & \rA \\
		1,3 & (5) & x \in B & \subseteqE{3,1} \\
		1,2,3 & (6) & x \in C & \subseteqE{5,2} \\
		1,2 & (7) & A\subseteq C & \subseteqIa{3,6} \\
	\end{array}
	\]
\end{proof}

\section{Das Axiom der leeren Menge}


Das Axiom der leeren Menge ($\emptyset$) in der Zermelo-Fraenkel Mengenlehre besagt, dass es eine Menge gibt, die keine Elemente enthält. Formal ausgedrückt:
\label{ImpExOFaxLpxNotinORp}
\[
\exists \emptyset \forall x (x \notin \emptyset)
\]
Dieses Axiom garantiert die Existenz der leeren Menge, die wir oft mit $\emptyset$ bezeichnen.

\subsubsection{Einführungsregel für die leere Menge}
\label{rule:remptyset}
Die leere Menge \(\emptyset\) ist ein spezielles Objekt in der Mengenlehre, das durch das Axiom der leeren Menge definiert wird. Dieses Axiom besagt, dass es eine Menge gibt, die keine Elemente enthält. Die Einführungsregel für die leere Menge leitet sich aus dem Axiom der leeren Menge ab und wird wie folgt formuliert:

\[
\begin{array}{llll}
	& (1) & x\notin\emptyset & \remptyset \\
\end{array}
\]

\subsubsection{Regeln für das Nicht-Enthaltensein}
\label{rule:notinE} \label{rule:notinI}
Das Nicht-Enthaltensein, dargestellt durch das Symbol \(\notin\), ist ein logischer Operator, der bedeutet, dass ein Element nicht zu einer Menge gehört. Wir führen zwei grundlegende Regeln für das Nicht-Enthaltensein ein: die Einführungs- und die Eliminierungsregel.

\[
\begin{array}{llll}
	i & (1) & \neg(a\in A) & ... \\
	i & (2) & a \notin A & \notinI{1} \\
\end{array}
\]

\[
\begin{array}{llll}
	i & (1) & a \notin A & ... \\
	i & (2) & \neg(a\in A) & \notinE{1} \\
\end{array}
\]
\subsubsection{Vereinfachung für das Nicht-Enthaltensein}
Das Symbol \(\notin\), das das Nicht-Enthaltensein darstellt, ist in seiner Bedeutung offensichtlich und direkt verständlich. Es bedeutet einfach, dass ein bestimmtes Element nicht zu einer Menge gehört. Aufgrund dieser direkten und intuitiven Bedeutung können wir in vielen Fällen auf separate Einführungs- und Eliminierungsregeln für \(\notin\) verzichten.

Die Notation \(a \notin A\) ist äquivalent zu \(\neg(a \in A)\) und kann direkt verwendet werden, ohne dass eine formale Regel angewendet werden muss. Dies vereinfacht den Umgang mit dem Nicht-Enthaltensein in logischen Argumentationen und Beweisen.

Daher können wir in Zukunft die Verwendung von \(a \notin A\) als eine unmittelbare Folgerung von \(\neg(a \in A)\) und umgekehrt ansehen, ohne explizit eine Regel anzuwenden.

\subsection{Die Eindeutigkeit der leeren Menge}

Um zu zeigen, dass die leere Menge eindeutig bestimmt ist, können wir das Extensionalitätsaxiom verwenden. Angenommen, es gibt zwei leere Mengen $O_1$ und $O_2$. Nach dem Extensionalitätsaxiom sind diese beiden Mengen gleich, wenn sie die gleichen Elemente haben. Da beide Mengen keine Elemente haben, müssen sie gleich sein. Daher ist die leere Menge eindeutig.

\label{ExOFaxLpxNotinORpImpOEqualsEmptyset}
\begin{theorem}[\(\exists O\forall x (x \notin O)\vdash O = \emptyset\) (Eindeutigkeit der leeren Menge)]
\end{theorem}
\begin{proof}
	\[
	\begin{array}{llll}
		1 & (1) & \exists O\forall x (x \notin O) & \rA \\
		2 & (2) & \forall x (x \notin O) & \rA \\
		2 & (3) & x \notin O & \rUE{2} \\
		& (4) & x \notin \emptyset & \remptyset \\
		2 & (5) & x \notin O\land x\notin\emptyset & \rAI{3,4} \\
		2 & (6) & (x \in O\land x\in\emptyset)\lor(x \notin O\land x\notin\emptyset) & \rOIb{5} \\
		2 & (7) & (x \in O\leftrightarrow x \in \emptyset) & \PLrQEqvLpPAndQRpOrLpnPAndnQRp{6} \\
		2 & (8) & \forall x(x \in O\leftrightarrow x \in \emptyset) & \rUI{7} \\
		2 & (9) & O=\emptyset & \rIISet{8} \\
		1 & (9) & O=\emptyset & \rEI{1,2,8} \\
	\end{array}
	\]
\end{proof}

\subsubsection{Regeln für die leere Menge}


\paragraph{Existenzregel für die leere Menge.}
\label{rule:remptysetIsSet}
Die Existenzregel stellt sicher, dass die leere Menge \(\emptyset\) tatsächlich als Menge existiert.
\[
\begin{array}{llll}
	& (1) & \emptyset\text{ ist eine Menge} & \remptysetIsSet{} \\
\end{array}
\]


\paragraph{Eliminationsregel für die leere Menge.}
\label{rule:remptysetb}
Die Eliminationsregel für die leere Menge besagt, dass eine Menge ohne Elemente identisch mit der leeren Menge \(\emptyset\) ist. Wenn also für alle \(x\) gilt, dass \(x \notin A\), dann folgt daraus, dass \(A = \emptyset\). Formal wird die Regel wie folgt angewendet:

\[
\begin{array}{llll}
	& (1) & \forall x(x\notin A) & \\
	& (2) & A = \emptyset & \remptysetb{1} \\
\end{array}
\]

\label{ImpFaALpEmptysetSubseteqARp}
\begin{theorem}[\(\vdash\forall A (\emptyset\subseteq A)\)]
	Für jede Menge \( A \), ist die leere Menge \( \emptyset \) eine Teilmenge von \( A \).
\end{theorem}
\begin{proof}
	\[
	\begin{array}{llll}
		& (1) & x\notin\emptyset & \remptyset \\
		& (2) & x\notin A\rightarrow x\notin\emptyset & \QImpPToQ{1}\\  
		& (3) & x\in \emptyset\rightarrow x\in A & \PToQEqvnQTonP{2}\\ 
		& (4) & \emptyset\subseteq A & \subseteqIb{3}\\ 			
	\end{array}
	\]
\end{proof}

\label{ASubseteqBwFaxInBLpxNotinARpImpAEqualsEmptyset}
\begin{theorem}[\(A\subseteq B, \forall x\in B(x\notin A)\vdash A=\emptyset\)]
\end{theorem}
\begin{proof}
	\[
	\begin{array}{llll}
		1 & (1) & A\subseteq B & \rA \\
		2 & (2) & \forall x\in B(x\notin A) & \rA\\  
        3 & (3) & x\in A & \rA\\ 
        1,3 & (4) & x\in B & \subseteqE{1,3}\\ 
        2 & (5) & x\in B\rightarrow x\notin A & \rSetUEb{2}\\ 
        2 & (6) & x\in A\rightarrow x\notin B & \PToQEqvnQTonP{5}\\ 
        2 & (7) & x\notin B & \rRE{3,6}\\ 
        1,2,3 & (8) & \bot & \rBI{4,7}\\ 
        1,2 & (9) & x\notin A & \rBI{4,7}\\ 
        1,2 & (10) & \forall x(x\notin A) & \rUI{9}\\ 	
        1,2 & (11) & A=\emptyset & \remptysetb{10}\\ 
	\end{array}
	\]
\end{proof}

\label{ExxInSImpSNotEqualsEmptyset}
\begin{theorem}[\(\exists x(x\in S) \vdash S\neq\emptyset\)]
\end{theorem}
\begin{proof}
	\[
	\begin{array}{llll}
		1 & (1) & \exists x\in S & \rA \\
		2 & (2) & a\in S & \rA \\
		& (3) & a\notin \emptyset & \remptyset \\
		2 & (4) & a\in S\land a\notin \emptyset & \rAI{2,3} \\
		2 & (5) & S\neq\emptyset & \rNIIbSet{4} \\
		1 & (6) & S\neq\emptyset & \rEE{1,2,7} \\
	\end{array}
	\]
\end{proof}
\begin{remark}
	Das Symbol \(\emptyset\) wird oft verwendet, um die leere Menge zu repräsentieren. Es stammt aus der skandinavischen Schreibweise des Buchstabens "`O"' und wurde von den Mathematikern André Weil und Bourbaki eingeführt. Es ist wichtig zu beachten, dass \(\emptyset\) nur ein Symbol ist und nicht als eine Zahl oder ein anderes mathematisches Objekt betrachtet werden sollte. In der Mengenlehre repräsentiert es speziell eine Menge, die keine Elemente enthält.
\end{remark}

\section{Das Aussonderungsaxiom}
Das Aussonderungsaxiom (Auss) ist ein weiteres grundlegendes Axiom in der Zermelo-Fraenkel Mengenlehre. Es ermöglicht die Konstruktion einer Menge, die alle Elemente einer gegebenen Menge enthält, die eine bestimmte Eigenschaft erfüllen. Formal ausgedrückt:

\label{ImpFaAFaPExBFaxLpxInBLrLpxInAAndPLpxRpRpRp}
\[
\vdash\forall A \forall P \exists B \forall x (x \in B \leftrightarrow (x \in A \land P(x)))
\]

Dieses Axiom garantiert, dass für jede Menge \( A \) und jede Eigenschaft \( P \), es eine Menge \( B \) gibt, die alle Elemente von \( A \) enthält, die \( P \) erfüllen.

\label{ExALpFaxLpPLpxRpToxInARpRpImpExBFaxLpxInBLrPLpxRpRp}
\begin{theorem}[\(\exists A(\forall x(P(x)\rightarrow x\in A))\vdash \exists B \forall x (x \in B \leftrightarrow P(x))\)]
\end{theorem}
\begin{proof}
	\[
	\begin{array}{llll}
		& (1) & \exists B \forall x (x \in B \leftrightarrow (x \in A \land P(x))) & \ImpFaAFaPExBFaxLpxInBLrLpxInAAndPLpxRpRpRp{} \\
		2 & (2) & \exists A(\forall x(P(x)\rightarrow x\in A)) & \rA \\
		3 & (3) & \forall x (x \in B \leftrightarrow (x \in A \land P(x))) & \rA \\
		4 & (4) & \forall x(P(x)\rightarrow x\in A) & \rA \\
		3 & (5) & x \in B \leftrightarrow (x \in A \land P(x)) & \rUE{3} \\
		4 & (6) & P(x)\rightarrow x\in A & \rUE{4} \\
		3,4 & (7) & x\in A\leftrightarrow P(x) & \PLrLpQAndRRpwQToRImpPLrQ{5,6} \\
		3,4 & (8) & \forall x(x\in B\leftrightarrow P(x))& \rUI{7} \\  
		2,3 & (9) & \forall x(x\in B\leftrightarrow P(x))& \rEE{2,4,8} \\
		2 & (10) & \forall x(x\in B\leftrightarrow P(x))& \rEE{1,3,9} \\
		2 & (11) & \exists B\forall x(x\in B\leftrightarrow P(x))& \rEI{10} \\		  
	\end{array}
	\]
\end{proof}

\subsection{Element-Symbol}
\label{rule:inE} \label{rule:inI}
Das Elementsymbol, oft dargestellt durch das Symbol \(\in\), wird in der Mengenlehre verwendet, um die Zugehörigkeit eines Elements zu einer Menge auszudrücken. Wir führen nun zwei grundlegende Regeln für das Elementsymbol ein: die Einführungs- und die Eliminierungsregel.

Die Einführungsregel für das Elementsymbol (\(\in I\)) besagt, dass wenn wir eine Aussage der Form \(P(a)\) haben, wir daraus die Aussage \(a \in \{x\in A | P(x)\}\) ableiten können.

\[
\begin{array}{llll}
	i & (1) & P(a) & ... \\
	j & (2) & a \in A & ... \\
	i,j & (3) & a \in \{x \in A| P(x)\} & \inI{1,2} \\
\end{array}
\]

\[
\begin{array}{llll}
	i & (1) & a\in A\wedge P(a) & ... \\
	i & (2) & a \in \{x \in A| P(x)\} & \inI{1} \\
\end{array}
\]

Die Eliminierungsregel für das Elementsymbol (\(\in E\)) besagt, dass wenn wir eine Aussage der Form \(a \in \{x\in A | P(x)\}\) haben, wir daraus die Aussage \(P(a)\) ableiten können, analog die Aussage \( a\in A \).

\[
\begin{array}{llll}
	i & (1) & a \in \{x\in A | P(x)\} & ... \\
	i & (2) & P(a) & \inE{1} \\
\end{array}
\]

\[
\begin{array}{llll}
	i & (1) & a \in \{x\in A | P(x)\} & ... \\
	i & (2) & a\in A & \inE{1} \\
\end{array}
\]

\[
\begin{array}{llll}
	i & (1) & a \in \{x\in A | P(x)\} & ... \\
	i & (2) & a\in A\wedge P(a) & \inE{1} \\
\end{array}
\]

\(i\) und \(j\) sind dabei Listen von Annahmen.	

\label{FaxLpxNotinLbxInAMidPLpxRpRbLrLpxNotinAOrnPLpxRpRpRp}
\begin{theorem}[\(\forall x(x \notin \{x \in A \mid P(x)\} \leftrightarrow (x \notin A \lor \neg P(x)))\)]
\end{theorem}
\begin{proof}
	\[
	\begin{array}{llll}
		1 & (1) & \neg(x \notin \{x \in A \mid P(x)\}) & \rA \\
		1 & (2) & x \in \{x \in A \mid P(x)\} & \rDN{1} \\
		1 & (3) & x \in A\land P(a) & \inE{2} \\
		1 & (4) & \neg(x \notin A\lor \neg P(a)) & \nLpnPOrnQRpEqvPAndQ{3} \\
		& (5) & \neg (x \notin \{x \in A \mid P(x)\})\rightarrow \neg(x \notin A\lor \neg P(a)) & \rRI{1,4} \\
		& (6) & x \notin A\lor \neg P(a)\rightarrow x \notin \{x \in A \mid P(x)\}  & \PToQEqvnQTonP{5} \\
		7 & (7) & \neg(x\notin A\lor \neg P(a)) & \rA \\
		7 & (8) & x\in A\land P(a) & \nLpnPOrnQRpEqvPAndQ{7} \\
		7 & (9) & x\in \{x \in A \mid P(x)\} & \inI{8} \\
		7 & (10) & \neg(x\notin \{x \in A \mid P(x)\}) & \rDN{9} \\			  
		& (11) & \neg(x\notin A\lor \neg P(a))\rightarrow \neg(x\notin \{x \in A \mid P(x)\}) & \rRI{10} \\	
		& (12) & x\notin \{x \in A \mid P(x)\}\rightarrow x\notin A\lor \neg P(a) & \PToQEqvnQTonP{11} \\		
		& (13) & x\notin \{x \in A \mid P(x)\}\leftrightarrow x\notin A\lor \neg P(a) & \rLRI{6,12} \\		  
	\end{array}
	\]
\end{proof}

\subsection{Nicht-Element-Symbol}
\label{rule:notinEa} \label{rule:notinIa}
Das Nicht-Element-Symbol, oft dargestellt durch das Symbol \(\notin\), wird in der Mengenlehre verwendet, um die Nichtzugehörigkeit eines Elements zu einer Menge auszudrücken. Wir führen nun zwei grundlegende Regeln für das Nicht-Element-Symbol ein: die Einführungs- und die Eliminierungsregel.

Die Einführungsregel für das Nicht-Element-Symbol (\(\notin I\)) besagt, dass wenn wir eine Aussage der Form \(\neg P(a)\) oder \(a \notin A\) haben, wir daraus die Aussage \(a \notin \{x\in A | P(x)\}\) ableiten können.

\[
\begin{array}{llll}
	i & (1) & \neg P(a) & ... \\
	i & (2) & a \notin \{x \in A| P(x)\} & \notinIa{1} \\
\end{array}
\]

\[
\begin{array}{llll}
	i & (1) & a \notin A & ... \\
	i & (2) & a \notin \{x \in A| P(x)\} & \notinIa{1} \\
\end{array}
\]

Die Eliminierungsregel für das Nicht-Element-Symbol (\(\notin E\)) besagt, dass wenn wir eine Aussage der Form \(a \notin \{x\in A | P(x)\}\) haben, wir daraus die Aussage \(\neg P(a)\) oder \(a \notin A\) ableiten können.

\[
\begin{array}{llll}
	i & (1) & a \notin \{x\in A | P(x)\} & ... \\
	i & (2) & a\notin A\lor \neg P(a) & \notinEa{1} \\
\end{array}
\]

\[
\begin{array}{llll}
	i & (1) & a \notin \{x\in A | P(x)\} & ... \\
	j & (2) & a\in A & ... \\
        i,j & (3) & \neg P(x) & \notinEa{1,2} \\
\end{array}
\]

\[
\begin{array}{llll}
	i & (1) & a \notin \{x\in A | \neg P(x)\} & ... \\
	j & (2) & a\in A & ... \\
        i,j & (3) & P(x) & \notinEa{1,2} \\
\end{array}
\]

\[
\begin{array}{llll}
	i & (1) & a \notin \{x\in A | P(x)\} & ... \\
	j & (2) & P(a) & ... \\
        i,j & (3) & a\notin A & \notinEa{1} \\
\end{array}
\]

\(i,j\) sind dabei Liste von Annahmen.

\subsection{Eigenschaften des Aussonderungsaxioms}

Die Menge \( B \) enthält alle Elemente der Menge \( A \), die die Eigenschaft \( P \) erfüllen.

\label{FaxLpxInBLrPLpxRpRpAndFaxLpxInCLrPLpxRpRpImpBEqualsC}
\begin{theorem}[\(\forall x (x \in B \leftrightarrow P(x)) \land \forall x (x \in C \leftrightarrow P(x)) \vdash B = C\)]
	Seien \( B \) und \( C \) Mengen, deren Elemente alle  die Eigenschaft \( P \) erfüllen, dann gilt unter Verwendung des Extensionalitätsaxioms \(B = C\).
\end{theorem}
\begin{proof}
	\[
	\begin{array}{llll}
		1 & (1) & \forall x (x \in B \leftrightarrow P(x)) \land \forall x (x \in C \leftrightarrow P(x)) & \rA \\
		1 & (2) & \forall x (x \in B \leftrightarrow P(x)) & \rAEa{1} \\
		1 & (3) & \forall x (x \in C \leftrightarrow P(x)) & \rAEb{1} \\
		1 & (4) & x \in B \leftrightarrow P(x) & \rUE{2} \\
		1 & (5) & x \in C \leftrightarrow P(x) & \rUE{3} \\
		1 & (6) & x \in B \leftrightarrow x \in C & \PLrQwRLrQImpPLrR{4,5} \\
		1 & (7) & \forall x (x \in B \leftrightarrow x \in C) & \rUI{6} \\
		1 & (8) & B=C & \rIISet{6} \\
	\end{array}
	\]
\end{proof}

\label{FaxLpxInBLrxInAAndPLpxRpRpImpBEqualsLbxInAMidPLpxRpRb}
\begin{theorem}[\(\forall x(x\in B\leftrightarrow x\in A\land P(x))\vdash B=\{x\in A \mid P(x)\}\) (Eindeutigkeit der ausgesonderten Menge)]
	Seien \( B \) und \( C \) Mengen, die alle Elemente einer Menge \( A \) enthalten, die die Eigenschaft \( P \) erfüllen, dann gilt unter Verwendung des Extensionalitätsaxioms \(B = C\).
\end{theorem}

\begin{proof}
	\[
	\begin{array}{llll}
		1 & (1) & \forall x(x\in B\leftrightarrow x\in A\land P(x)) & \rA \\
		1 & (2) & x\in B\leftrightarrow x\in A\land P(x) & \rUE{1} \\
		3 & (3) & x\in B & \rA \\
		1,3 & (4) & x\in A\land P(x) & \PLrQwPImpQ{2,3} \\
		1,3 & (5) & x\in \{x\in A|P(x)\} & \inI{4} \\
		1 & (6) & x\in B\rightarrow x\in \{x\in A|P(x)\} & \rRI{3,5} \\
		7 & (7) & x\in \{x\in A|P(x)\} & \rA \\
		7 & (8) & x\in A\land P(x) & \inE{7} \\
		1,7 & (9) & x\in B & \PLrQwPImpQ{2,8} \\
		1 & (10) & x\in \{x\in A|P(x)\}\rightarrow x\in B & \PLrQwPImpQ{2,8} \\			
		1 & (11) & x\in B\leftrightarrow x\in \{x\in A|P(x)\} & \rLRI{6,10} \\						
		1 & (12) & \forall x(x\in B\leftrightarrow x\in \{x\in A|P(x)\}) & \rUI{11} \\	
		1 & (13) & B=\{x\in A|P(x)\} & \rIISet{12} \\	
	\end{array}
	\]
\end{proof}

\begin{remark}
	Das Aussonderungsaxiom ist ein Schlüsselkonzept in der Mengenlehre und ermöglicht die Konstruktion von Mengen, die bestimmte Eigenschaften erfüllen. Es dient als Grundlage für die Definition von Teilmengen und ist ein wichtiger Baustein für die Entwicklung vieler anderer Konzepte in der Mengenlehre. Die ausgesonderte Menge einer Menge \( A \) mit einer Eigenschaft \( P \) wird oft als \( \{ x \in A | P(x) \} \) oder \( \{ x | x \in A          \cap P(x) \} \) bezeichnet. Diese Notation stellt sicher, dass die ausgesonderte Menge eindeutig ist, da sie durch das Aussonderungsaxiom und das Extensionalitätsaxiom eindeutig bestimmt ist.
\end{remark}

\paragraph{Existenzregel für die ausgesonderte Menge.}
\label{rule:comprehensionSetExists}
Die Existenzregel stellt sicher, dass für jede Menge \(A\) und jede Eigenschaft \(P(x)\) die ausgesonderte Menge \(\{x \in A \mid P(x)\}\) existiert.

\[
\begin{array}{llll}
	i & (1) & A \text{ ist eine Menge} & \dots  \\
	i & (2) & \{x \in A \mid P(x)\}\text{ ist eine Menge} & \comprehensionSetExists{1} \\
\end{array}
\]

\(i\) ist dabei eine Liste von Annahmen.

\label{ImpLbxInAMidPLpxRpRbSubseteqA}
\begin{theorem}[\(\vdash \{ x \in A \mid P(x) \} \subseteq A\)]
\end{theorem}

\begin{proof}
	\[
	\begin{array}{llll}
		1 & (1) & x\in \{ x \in A \mid P(x) \} & \rA \\
		1 & (2) & x\in A\land P(x) & \inE{1}\\
		1 & (3) & x\in A & \rAEa{2}\\
		1 & (4) & \{ x \in A \mid P(A) \}\subseteq A & \subseteqIa{1,3}\\
	\end{array}
	\]
\end{proof}

\label{FaxInALpPLpxRpRpwyInAImpPLpyRp}
\begin{theorem}[\(\forall x \in A(P(x)), y\in A \vdash P(y)\)]
\end{theorem}
\begin{proof}
	\[
	\begin{array}{llll}
		1 & (1) & \forall x \in A(P(x)) & \rA \\
		2 & (2) & y\in A & \rA \\
		1 & (3) & \forall x(x\in A\rightarrow P(x)) &  \rSetUEa{1} \\
		1 & (4) & y\in A\rightarrow P(y) &  \rUE{3} \\
		1,2 & (5) & P(y) &  \rRE{4,2}\\
	\end{array}
	\]
\end{proof}

\label{FaxInMLpPLpxRpRpEqvMEqualsLbxInMMidPLpxRpRb}
\begin{theorem}[\(\forall x \in M(P(x)) \dashv\vdash M = \{x \in M \mid P(x)\}\))]
\end{theorem}
\begin{proof}
	\(\vdash:\)
	\[
	\begin{array}{llll}
		1   & (1) & \forall x \in M(P(x)) & \rA \\
		& (2) & \{x\in M\mid P(x)\}\subseteq M & \ImpLbxInAMidPLpxRpRbSubseteqA{} \\		
		1   & (3) & \forall x(x\in M\rightarrow P(x)) & \rSetUEa{1} \\
		4   & (4) & y\in M & \rA \\
		1   & (5) & y\in M\rightarrow P(y) & \rUE{3} \\
		1,4 & (6) & P(y) & \rRE{5,4} \\
		1,4 & (7) & y\in\{x\in M\mid P(x)\}  & \inI{4,6} \\
		1   & (8) & M\subseteq \{x\in M\mid P(x)\}  & \subseteqIa{4,7} \\
		1   & (9) & M=\{x\in M\mid P(x)\}  & \rIIaSet{2,8} \\
	\end{array}
	\]
	\(\dashv:\)
	\[
	\begin{array}{llll}
		1   & (1) & M = \{x \in M \mid P(x)\} & \rA \\
		2   & (2) & y\in M  & \rA \\
		1,2 & (3) & y\in \{x \in M \mid P(x)\}  & \rIEaSet{2} \\
		1,2 & (4) & P(y)  & \inE{3} \\
		1   & (5) & y\in M\rightarrow P(y)  & \rRI{2,4} \\
		1   & (6) & \forall y(y\in M\rightarrow P(y))  & \rUI{5} \\
		1   & (7) & \forall y\in M(P(y))  & \rSetUIb{6} \\		
	\end{array}
	\]
\end{proof}

\subsubsection{Regeln zur Quantorenumkehr}
\subsubsection{Einstellige Quantorenumkehr}

\label{FaxInALpPLpxRpRpEqvnExxInALpnPLpxRpRp}
\begin{theorem}[\(\forall x \in A(P(x)) \dashv\vdash \neg\exists x \in A(\neg P(x))\)]
\end{theorem}
\begin{proof}
	\(\vdash:\)
	\[
	\begin{array}{llll}
		1 & (1) & \forall x \in A(P(x)) & \rA \\
		1 & (2) & \forall x(x \in A \rightarrow P(x)) & \rSetUEa{1} \\
		1 & (3) & \neg\exists x(x \in A \wedge \neg P(x)) &  \FaxLpPLpxRpToQLpxRpRpEqvnExxLpPLpxRpAndnQLpxRpRp{2} \\
		& (4) & \exists x \in A(\neg P(x)) \leftrightarrow \exists x(x \in A \wedge \neg P(x)) & \rSetE \\
		& (5) & \neg \exists x \in A(\neg P(x)) \leftrightarrow \neg \exists x(x \in A \wedge \neg P(x)) & \PLrQEqvnPLrnQ{4} \\
		1 & (6) & \neg \exists x \in A(\neg P(x)) & \PLrQwPImpQ{3,6} \\
	\end{array}
	\]
	\(\dashv:\)
	\[
	\begin{array}{llll}
		1 & (1) & \neg \exists x \in A(\neg P(x)) & \rA \\
		& (2) & \exists x \in A(\neg P(x)) \leftrightarrow \exists x(x \in A \wedge \neg P(x)) & \rSetE \\
		& (3) & \neg \exists x \in A(\neg P(x)) \leftrightarrow \neg \exists x(x \in A \wedge \neg P(x)) & \PLrQEqvnPLrnQ{2} \\
		1 & (4) & \neg \exists x(x \in A \wedge \neg P(x)) & \PLrQwPImpQ{1,3} \\
		1 & (5) & \forall x(x\in A\rightarrow P(x)) & \FaxLpPLpxRpToQLpxRpRpEqvnExxLpPLpxRpAndnQLpxRpRp{4} \\
		1 & (6) & \forall x\in A(P(x)) & \rSetUIb{5}\\
	\end{array}
	\]
\end{proof}

\label{FaxInALpnPLpxRpRpEqvnExxInALpPLpxRpRp}
\begin{theorem}[\(\forall x \in A(\neg P(x)) \dashv\vdash \neg\exists x \in A(P(x))\)]
\end{theorem}
\begin{proof}
	\(\vdash:\)
	\[
	\begin{array}{llll}
		1 & (1) & \forall x \in A(\neg P(x)) & \rA \\
		1 & (2) & \forall x(x \in A \rightarrow \neg P(x)) & \rSetUEa{1} \\
		1 & (3) & \neg\exists x(x \in A \wedge P(x)) &  \FaxLpPLpxRpTonQLpxRpRpEqvnExxLpPLpxRpAndQLpxRpRp{2} \\
		& (4) & \exists x \in A(P(x)) \leftrightarrow \exists x(x \in A \wedge P(x)) & \rSetE \\
		& (5) & \neg \exists x \in A(P(x)) \leftrightarrow \neg \exists x(x \in A \wedge P(x)) & \PLrQEqvnPLrnQ{4} \\
		1 & (6) & \neg \exists x \in A(P(x)) & \PLrQwPImpQ{3,6} \\
	\end{array}
	\]
	\(\dashv:\)
	\[
	\begin{array}{llll}
		1 & (1) & \neg \exists x \in A(P(x)) & \rA \\
		& (2) & \exists x \in A(P(x)) \leftrightarrow \exists x(x \in A \wedge P(x)) & \rSetE \\
		& (3) & \neg \exists x \in A(P(x)) \leftrightarrow \neg \exists x(x \in A \wedge P(x)) & \PLrQEqvnPLrnQ{2} \\
		1 & (4) & \neg \exists x(x \in A \wedge P(x)) & \PLrQwPImpQ{1,3} \\
		1 & (5) & \forall x(x\in A\rightarrow \neg P(x)) & \FaxLpPLpxRpTonQLpxRpRpEqvnExxLpPLpxRpAndQLpxRpRp{4} \\
		1 & (6) & \forall x\in A(P(x)) & \rSetUIb{5}\\
	\end{array}
	\]
\end{proof}

\label{ExxInALpPLpxRpRpEqvnFaxInALpnPLpxRpRp}
\begin{theorem}[\(\exists x \in A(P(x)) \dashv\vdash \neg\forall x \in A(\neg P(x))\)]
\end{theorem}
\begin{proof}
	\(\vdash:\)
	\[
	\begin{array}{llll}
		1 & (1) & \exists x \in A(P(x)) & \rA \\
		1 & (2) & \exists x (x \in A \wedge P(x)) & \rSetEEa{1} \\
		1 & (3) & \neg\forall x(x \in A \rightarrow \neg P(x)) & \LpnFaxLpPLpxRpTonQLpxRpRpEqvExxLpPLpxRpAndQLpxRpRp{2} \\
		& (4) & \forall x \in A(\neg P(x)) \leftrightarrow \forall x(x \in A \rightarrow \neg P(x)) & \rSetU \\
		& (5) & \neg \forall x \in A(\neg P(x)) \leftrightarrow \neg\forall x(x \in A \rightarrow \neg P(x)) & \PLrQEqvnPLrnQ{4} \\
		& (6) & \neg \forall x(x \in A \rightarrow \neg P(x))\rightarrow \neg \forall x \in A(\neg P(x)) & \rLREb{5} \\
		1 & (7) & \neg \forall x \in A(\neg P(x)) & \rRE{3,6} \\
	\end{array}
	\]
	\(\dashv:\)
	\[
	\begin{array}{llll}
		1 & (1) & \neg \forall x \in A(\neg P(x)) & \rA \\
		& (2) & \forall x \in A(\neg P(x)) \leftrightarrow \forall x(x \in A \rightarrow \neg P(x)) & \rSetU \\
		& (3) & \neg \forall x \in A(\neg P(x)) \leftrightarrow \neg\forall x(x \in A \rightarrow \neg P(x)) & \PLrQEqvnPLrnQ{2} \\
		1 & (4) & \neg\forall x(x \in A \rightarrow \neg P(x)) & \PLrQwPImpQ{1,3} \\
		1 & (5) & \exists x(x \in A \wedge P(x)) & \LpnFaxLpPLpxRpTonQLpxRpRpEqvExxLpPLpxRpAndQLpxRpRp{4} \\
		1 & (6) & \exists x \in A(P(x)) & \rSetEIb{5} \\
	\end{array}
	\]
\end{proof}

\label{ExxInALpnPLpxRpRpEqvnFaxInALpPLpxRpRp}
\begin{theorem}[\(\exists x \in A(\neg P(x)) \dashv\vdash \neg\forall x \in A(P(x))\)]
\end{theorem}
\begin{proof}
	\(\vdash:\)
	\[
	\begin{array}{llll}
		1 & (1) & \exists x \in A(\neg P(x)) & \rA \\
		1 & (2) & \exists x (x \in A \wedge \neg P(x)) & \rSetEEa{1} \\
		1 & (3) & \neg\forall x(x \in A \rightarrow P(x)) & \nFaxLpPLpxRpToQLpxRpRpEqvExxLpPLpxRpAndnQLpxRpRp{2} \\
		& (4) & \forall x \in A(P(x)) \leftrightarrow \forall x(x \in A \rightarrow P(x)) & \rSetU \\
		& (5) & \neg \forall x \in A(P(x)) \leftrightarrow \neg\forall x(x \in A \rightarrow P(x)) & \PLrQEqvnPLrnQ{4} \\
		& (6) & \neg \forall x(x \in A \rightarrow P(x))\rightarrow \neg \forall x \in A(P(x)) & \rLREb{5} \\
		1 & (7) & \neg \forall x \in A(P(x)) & \rRE{3,6} \\
	\end{array}
	\]
	\(\dashv:\)
	\[
	\begin{array}{llll}
		1 & (1) & \neg \forall x \in A(P(x)) & \rA \\
		& (2) & \forall x \in A(P(x)) \leftrightarrow \forall x(x \in A \rightarrow P(x)) & \rSetU \\
		& (3) & \neg \forall x \in A(P(x)) \leftrightarrow \neg\forall x(x \in A \rightarrow P(x)) & \PLrQEqvnPLrnQ{2} \\
		1 & (4) & \neg\forall x(x \in A \rightarrow P(x)) & \PLrQwPImpQ{1,3} \\
		1 & (5) & \exists x(x \in A \wedge \neg P(x)) & \nFaxLpPLpxRpToQLpxRpRpEqvExxLpPLpxRpAndnQLpxRpRp{4} \\
		1 & (6) & \exists x \in A(\neg P(x)) & \rSetEIb{5} \\
	\end{array}
	\]
\end{proof}

\subsubsection{Zweistellige Quantorenumkehr}

\label{nLpFaxInAExyInBLpPLpxwyRpRpRpEqvExxInAFayInBLpnPLpxwyRpRp}
\begin{theorem}[\(\neg(\forall x\in A\exists y\in B(P(x,y))) \dashv\vdash \exists x\in A\forall y\in B(\neg P(x,y))\)]
\end{theorem}
\begin{proof}
	\(\vdash:\)
	\[
	\begin{array}{llll}
		1 & (1) & \neg(\forall x\in A\exists y\in B(P(x,y))) & \rA \\
		1 & (2) & \exists x\in A\neg(\exists y\in B(P(x,y))) & \ExxInALpnPLpxRpRpEqvnFaxInALpPLpxRpRp{1} \\
		3 & (3) & x\in A\land \neg(\exists y\in B(P(a,y))) & \rA \\
        3 & (4) & x\in A & \rAEa{3} \\
        3 & (5) & \neg(\exists y\in B(P(a,y))) & \rAEb{3} \\
        3 & (6) & \forall y\in B(\neg P(a,y)) & \FaxInALpnPLpxRpRpEqvnExxInALpPLpxRpRp{5} \\
		3 & (7) & \exists x\in A\forall y\in B(\neg P(a,y)) & \rEI{4,6} \\
        1 & (8) & \exists x\in A\forall y\in B(\neg P(a,y)) & \rEE{2,3,7} \\
	\end{array}
	\]
	\(\dashv:\)
	\[
	\begin{array}{llll}
		1 & (1) & \exists x\in A\forall y\in B(\neg P(a,y)) & \rA \\
		2 & (2) & x\in A\land \forall y\in B(\neg P(a,y)) & \rA \\
        2 & (3) & x\in A & \rAEa{2} \\
        2 & (4) & \forall y\in B(\neg P(a,y)) & \rAEb{2} \\ 
		2 & (5) & \neg(\exists y\in B(P(a,y))) & \FaxInALpnPLpxRpRpEqvnExxInALpPLpxRpRp{4} \\
		2 & (6) & \exists x\in A\neg(\exists y\in B(P(a,y))) & \rEI{3,5} \\
		2 & (7) & \neg \forall x\in A\exists y\in B(P(a,y)) & \ExxInALpnPLpxRpRpEqvnFaxInALpPLpxRpRp{6} \\
		1 & (8) & \neg \forall x\in A\exists y\in B(P(a,y)) & \rEE{1,2,7} \\
	\end{array}
	\]
\end{proof}


\section{Definition des Schnitts}

\begin{definition}[Schnitt]
	Der Schnitt von zwei Mengen \( A \) und \( B \) ist die Menge aller Elemente, die sowohl in \( A \) als auch in \( B \) enthalten sind. Dies wird als \( A \cap B \) bezeichnet und kann als eine spezielle Anwendung des Aussonderungsaxioms formuliert werden:
	\[
	A \cap B = \{ x \in A \mid x \in B \}
	\]
	Formal ausgedrückt unter Verwendung des Aussonderungsaxioms:
	\[
	\forall A \forall B \exists C \forall x (x \in C \leftrightarrow (x \in A \land x \in B))
	\]
\end{definition}

\subsubsection{Regeln für den Schnitt}
\label{rule:capI} \label{rule:capEa} \label{rule:capEb} \label{rule:capE}
Der Schnitt, oft dargestellt durch das Symbol \(\cap\), ist eine Operation auf zwei Mengen, die die Menge aller Elemente, die in beiden Mengen enthalten sind, bildet. Wir können auf Basis der vorherigen Definition nun Einführungs- und Eliminationsregeln für den Schnitt definieren.

Die Einführungsregel für den Schnitt (\(\cap I\)) ermöglicht es uns, aus der Zugehörigkeit eines Elements zu zwei Mengen \(A\) und \(B\) die Zugehörigkeit desselben Elements zum Schnitt \(A \cap B\) zu schließen:

\[
\begin{array}{llll}
	i & (1) \, x \in A & ... & \\
	j & (2) \, x \in B & ... & \\
	i,j & (3) \, x \in A \cap B & \capI{1,2} \\
\end{array}
\]

\[
\begin{array}{llll}
	i & (1) \, x \in A\land x\in B & ... & \\
	i & (2) \, x \in A\cap B & \capI{1} \\
\end{array}
\]

Die Eliminierungsregel für den Schnitt (\(\cap E1\) und \(\cap E2\)) besagt, dass wenn wir ein Element haben, das zum Schnitt \(A \cap B\) gehört, wir daraus schließen können, dass dieses Element sowohl zu \(A\) als auch zu \(B\) gehört:

\[
\begin{array}{llll}
	i & (1) \, x \in A \cap B & ... & \\
	i & (2) \, x \in A\land x\in B & \capE{1} \\
\end{array}
\]


\[
\begin{array}{llll}
	i & (1) \, x \in A \cap B & ... & \\
	i & (2) \, x \in A & \capEa{1} \\
	i & (3) \, x \in B & \capEb{2} \\
\end{array}
\]

\(i,j\) sind dabei Listen von Annahmen.

\paragraph{Existenzregel für den Schnitt}
\label{rule:capSetExists}
Die Existenzregel stellt sicher, dass für alle Mengen \(A\) und \(B\) die Schnittmenge \(A\cap B\) ebenfalls eine Menge ist.

\[
\begin{array}{llll}
	i & (1) & A \text{ ist eine Menge} & \dots  \\
        j & (2) & B \text{ ist eine Menge} & \dots  \\
	i,j & (3) & A\cap B\text{ ist eine Menge} & \capSetExists{1} \\
\end{array}
\]

\(i\) und \(j\) sind dabei Listen von Annahmen.
\subsection{Eigenschaften des Schnitts}

\label{AEqualsAcaA}
\begin{theorem}[\(A = A \cap A\) (Idempotenz des Schnitts)]
\end{theorem}

\begin{proof}
	\[
	\begin{array}{llll}
		1 & (1) & x \in A \cap A  & \rA \\
		1 & (2) & x \in A & \capEa{1} \\
		& (3) & A\cap A\subseteq A & \subseteqIa{1,2}\\
		4 & (4) & x \in A  & \rA \\
		4 & (5) & x \in A\cap A  & \capI{4,4} \\
		& (6) & A\subseteq A\cap A  & \subseteqIa{4,5} \\			
		& (7) & A = A\cap A  & \rIIaSet{3,6} \\						  
	\end{array}
	\]
\end{proof}

\label{AcaBEqualsBcaA}
\begin{theorem}[\(A \cap B = B \cap A\) (Kommutativität des Schnitts)]
\end{theorem}
\begin{proof}
	\[
	\begin{array}{llll}
		1 & (1) & x \in A \cap B & \rA \\
		1 & (2) & x \in A & \capEa{1} \\
		1 & (3) & x \in B & \capEb{1} \\
		1 & (4) & x \in B \cap A & \capI{3,2} \\
		& (5) & A\cap B\subseteq B \cap A & \subseteqIa{1,4} \\
		6 & (6) & x \in B \cap A & \rA \\
		6 & (7) & x \in B & \capEa{6} \\
		6 & (8) & x \in A & \capEb{6} \\
		6 & (9) & x \in A \cap B & \capI{8,7} \\
		& (10) & B\cap A\subseteq A \cap B & \subseteqIa{6,9} \\
		& (11) & A\cap B=B\cap A & \rIIaSet{5,10} \\
	\end{array}
	\]
\end{proof}

\subsubsection{Regeln zur Kommutativität des Schnitts zweier Mengen}
\label{rule:kommCap}

% Regel für die Kommutativität des Schnitts zweier Mengen (Überführung)
Die Regel für die Überführung von \(A \cap B\) zu \(B \cap A\) basiert auf der Kommutativität des Schnitts und kann wie folgt ausgedrückt werden:
\[
\begin{array}{llll}
	i & (1) & A \cap B & \rA \\
	i & (2) & B \cap A & \kommCap{1} \\
\end{array}
\]

% Regel für die Zugehörigkeit eines Elements zu A\cap B und damit zu B\cap A
Die Regel, die zeigt, dass wenn ein Element \(z\) in \(A \cap B\) enthalten ist, es auch in \(B \cap A\) enthalten ist, nutzt ebenfalls die Kommutativität des Schnitts:
\[
\begin{array}{llll}
	i & (1) & z \in A \cap B & \rA \\
	i & (2) & z \in B \cap A & \kommCap{1} \\
\end{array}
\]

Dabei ist \(i\) eine Liste von Annahmen. 

\label{LpAcaBRpcaCEqualsAcaLpBcaCRp}
\begin{theorem}[\((A \cap B) \cap C = A \cap (B \cap C)\) (Assoziativität des Schnitts)]
\end{theorem}
\begin{proof}
	\[
	\begin{array}{llll}
		1 & (1) & x \in (A \cap B) \cap C & \rA \\
		1 & (2) & x \in (A \cap B) & \capEa{1} \\
		1 & (3) & x \in C & \capEb{1} \\
		1 & (4) & x \in A & \capEa{2} \\
		1 & (5) & x \in B & \capEb{2} \\
		1 & (6) & x \in B\cap C & \capI{5,3} \\
		1 & (7) & x \in A\cap (B\cap C) & \capI{4,6} \\			
		& (8) & (A\cap B)\cap C\subseteq A\cap (B\cap C) & \capI{1,7} \\						
		9 & (9) & x \in A \cap (B \cap C) & \rA \\
		9 & (10) & x \in (B \cap C) & \capEb{9} \\
		9 & (11) & x \in A & \capEa{9} \\
		9 & (12) & x \in B & \capEa{10} \\
		9 & (13) & x \in C & \capEb{10} \\
		9 & (14) & x \in A\cap B & \capI{11,12} \\
		9 & (15) & x \in (A\cap B)\cap C & \capI{14,13} \\			
		& (16) & A\cap (B\cap C)\subseteq (A\cap B)\cap C & \capI{9,15} \\									
		& (17) & (A\cap B)\cap C=A\cap (B\cap C) & \rIIaSet{8,16} \\									
	\end{array}
	\]
\end{proof}

\label{ImpnExUFaALpAInURp}
\begin{theorem}[\(\vdash \neg \exists U \forall A (A \in U)\) (Nichtexistenz einer universellen Menge)]
	Angenommen, es gibt eine universelle Menge \( U \) in der ZF-Mengenlehre, dann führt dies unter Verwendung des Aussonderungsaxioms zu einem Widerspruch durch das Russell'sche Paradoxon. 
\end{theorem}

\begin{proof}
	\[
	\begin{array}{llll}
		1 & (1) & \exists U \forall A (A \in U) & \rA \\
		2 & (2) & \forall A (A \in U) & \rA \\
		& (3) & R:=\{x\in U\mid x\notin x\} & := \\
		4 & (4) & R\in R & \rA \\
		4 & (5) & R\notin R & \inE{4} \\  
		4 & (6) & \bot & \rBI{4,5} \\  
		& (7) & R\notin R & \rCI{4,6}  \\  
		& (8) & R\notin U\lor R\in R & \notinE{7}  \\
		& (9) & R\in R\lor R\notin U &  \POrQImpQOrP{8} \\
		& (10) & R\notin R\rightarrow R\notin U & \PToQEqvnPOrQ{9}  \\
		& (11) & R\notin U & \rRE{7,10} \\
		2 & (12) & R\in U & \rUE{2} \\
		2 & (13) & \bot & \rBI{12,11} \\	
		1 & (14) & \bot & \rEE{1,2,13} \\	
		& (15) & \neg(\exists U \forall A (A \in U)) & \rCI{1,14} \\									  			  
	\end{array}
	\]
\end{proof}

\label{AcaBSubseteqA}
\begin{theorem}[\(A \cap B \subseteq A\)]
\end{theorem}
\begin{proof}
    \[
        \begin{array}{llll}
		1 & (1) & x \in A \cap B & \rA \\
		1 & (2) & x \in A & \capEa{1} \\
		& (3) & A \cap B \subseteq A & \subseteqIa{1,2} \\
	\end{array}
    \]
\end{proof}

\label{AcaBSubseteqB}
\begin{theorem}[\(A \cap B \subseteq B\)]
\end{theorem}
\begin{proof}
    \[
	\begin{array}{llll}
	   1 & (1) & x \in A \cap B & \rA \\
	   1 & (2) & x \in B & \capEa{1} \\
	    & (3) & A \cap B \subseteq B & \subseteqIa{1,2} \\
	\end{array}
    \]
\end{proof}

\label{ASubseteqBEqvAcaBEqualsA}
\begin{theorem}[\(A \subseteq B \dashv\vdash A \cap B = A\)]
\end{theorem}
\begin{proof}
    \(\vdash:\)
    \[
        \begin{array}{llll}
	   1 & (1) & A \subseteq B & \rA \\
	    & (2) & A\cap B\subseteq A & \AcaBSubseteqA{} \\
	   3 & (3) & x\in A & \rA \\		
	1,3 & (4) & x\in B & \subseteqE{3,1} \\		
	1,3 & (5) & x\in A\cap B & \capI{3,4} \\	
	   1 & (6) & A\subseteq A\cap B & \subseteqIa{3,5} \\			
	   1 & (7) & A \cap B = A & \rIIaSet{3,6} \\
        \end{array}
    \]
    \(\dashv:\)
    \[
        \begin{array}{llll}
	   1 & (1) & A\cap B = A & \rA \\
	   1 & (2) & A\subseteq A\cap B & \subsetEqIa{1} \\			
	   1 & (3) & A\cap B\subseteq B & \AcaBSubseteqB{} \\	
	   1 & (4) & A\subseteq B & \ASubseteqBwBSubseteqCImpASubseteqC{2,3} \\
	\end{array}
    \]
\end{proof}

\label{BSubseteqAEqvAcaBEqualsB}
\begin{theorem}[\(B \subseteq A \dashv\vdash A \cap B = B\)]
\end{theorem}
\begin{proof}
    \(\vdash:\)
    \[
        \begin{array}{llll}
	   1 & (1) & B \subseteq A & \rA \\
	    & (2) & A\cap B\subseteq B & \AcaBSubseteqB{} \\
	   3 & (3) & x\in B & \rA \\		
	1,3 & (4) & x\in A & \subseteqE{3,1} \\		
	1,3 & (5) & x\in A\cap B & \capI{3,4} \\	
	   1 & (6) & B\subseteq A\cap B & \subseteqIa{3,5} \\			
	   1 & (7) & A \cap B = B & \rIIaSet{3,6} \\
        \end{array}
    \]
    \(\dashv:\)
    \[
        \begin{array}{llll}
	   1 & (1) & A\cap B = B & \rA \\
	   1 & (2) & B\subseteq A\cap B & \subsetEqIa{1} \\			
	   1 & (3) & A\cap B\subseteq A & \AcaBSubseteqA{} \\	
	   1 & (4) & B\subseteq A & \ASubseteqBwBSubseteqCImpASubseteqC{2,3} \\
        \end{array}
    \]
\end{proof}


\label{EmptysetcaAEqualsEmptyset}
\begin{theorem}[\(\emptyset\cap A = \emptyset\)]
\end{theorem}
\begin{proof}
    \[
	\begin{array}{llll}
	   & (1) & \emptyset\subseteq A & \ImpFaALpEmptysetSubseteqARp{} \\
	   & (2) & \emptyset\cap A  = \emptyset & \ASubseteqBEqvAcaBEqualsA{1} \\
	\end{array}
    \]
\end{proof}

\label{AcaEmptysetEqualsEmptyset}
\begin{theorem}[\(A\cap\emptyset = \emptyset\)]
\end{theorem}
\begin{proof}
    \[
	\begin{array}{llll}
	   & (1) & \emptyset\subseteq A & \ImpFaALpEmptysetSubseteqARp{} \\
	   & (2) & A\cap\emptyset = \emptyset & \BSubseteqAEqvAcaBEqualsB{1} \\
	\end{array}
    \]
\end{proof}

\label{AcaBEqualsEmptysetwxInAImpxNotinB}
\begin{theorem}[\(A\cap B = \emptyset, x\in A\vdash x\notin B\)]
\end{theorem}
\begin{proof}
    \[
	\begin{array}{llll}
            1       & (1) & A\cap B = \emptyset & \rA \\
	    2       & (2) & x\in A & \rA \\
            3       & (3) & x\in B & \rA \\
            2,3     & (4) & x\in A\cap B & \capI{2,3} \\
            1,2,3   & (5) & x\in \emptyset & \rIE{1,4} \\
                    & (6) & x\notin \emptyset & \remptyset \\
                    & (7) & \bot & \rBI{5,6} \\           
            1,2     & (8) & x\notin B & \rCE{1,2} \\  
	\end{array}
    \]
\end{proof}

\label{AcaBEqualsEmptysetwxInBImpxNotinA}
\begin{theorem}[\(A\cap B = \emptyset, x\in B\vdash x\notin A\)]
\end{theorem}
\begin{proof}
    \[
	\begin{array}{llll}
            1 & (1) & A\cap B = \emptyset & \rA \\
	    2 & (2) & x\in B & \rA \\
              & (3) & B \cap A = A \cap B & \AcaBEqualsBcaA{} \\
            1 & (4) & B \cap A = \emptyset & \rIE{1,3} \\
            1,2 & (5) & x\notin B & \AcaBEqualsEmptysetwxInAImpxNotinB{5,2} \\
	\end{array}
    \]
\end{proof}

\subsection{Der unendliche Schnitt}

\label{LpPLpCRpwPLpDRpImpLbxInCMidFaALpPLpARpToxInARpRbEqualsLbxInDMidFaALpPLpARpToxInARpRb}
\begin{theorem}[\(P(C),P(D)\vdash \{ x \in C \mid \forall A (P(A) \rightarrow x \in A)\} = \{ x \in D \mid \forall A (P(A) \rightarrow x \in A) \}\)]
\end{theorem}

\begin{proof}
	\[
	\begin{array}{llll}
		1 & (1) & P(C) & \rA \\	
		2 & (2) & P(D) & \rA \\				
		3 & (3) & x\in \{x\in C\mid \forall A(P(A)\rightarrow x\in A)\} & \rA \\
		3 & (4) & \forall A(P(A)\rightarrow x\in A) & \inE{3} \\	
		3 & (5) & P(D)\rightarrow x\in D & \rUE{4} \\										
		2, 3 & (6) & x\in D & \rRE{2,4} \\
		2, 3 & (7) & x\in \{x\in D\mid \forall A(P(A)\rightarrow x\in A)\} & \inI{4,6} \\	
		2 & (8) & \{x\in C\mid \forall A(P(A)\rightarrow x\in A)\}\subseteq \{x\in D\mid \forall A(P(A)\rightarrow x\in A)\} & \subseteqIa{3,7} \\		
		9 & (9) & x\in \{x\in D\mid \forall A(P(A)\rightarrow x\in A)\} & \rA \\
		9 & (10) & \forall A(P(A)\rightarrow x\in A) & \inE{9} \\	
		9 & (11) & P(C)\rightarrow x\in C & \rUE{10} \\										
		1, 9 & (12) & x\in C & \rRE{1,10} \\
		1, 9 & (13) & x\in \{x\in C\mid \forall A(P(A)\rightarrow x\in A)\} & \inI{10,12} \\	
		1 & (14) & \{x\in D\mid \forall A(P(A)\rightarrow x\in A)\}\subseteq \{x\in C\mid \forall A(P(A)\rightarrow x\in A)\} & \subseteqIa{9,13} \\	
		1, 2 & (15) & \{x\in C\mid \forall A(P(A)\rightarrow x\in A)\}= \{x\in D\mid \forall A(P(A)\rightarrow x\in A)\} & \subseteqE{8,14} \\									
	\end{array}
	\]
	
\end{proof}

\begin{definition}[\(\exists C(P(C)\rightarrow \bigcap_{P(A)} A:=\bigcap \{ A \mid P(A) \} := \{ x\in C \mid \forall A(P(A)\rightarrow x\in A) \})\) (Der unendliche Schnitt)]
\end{definition}
\begin{remark}
	Aufgrund des vorherigen Theorems zur Unabhängigkeit des unendlichen Schnitts von der Ursprungsmenge, können wie die Bezeichnung \((Df. \bigcap \{A\mid P(A)\})\) so wählen, dass sie unabhängig von der Ursprungsmenge $C$ aus $\{ x\in C \mid \forall A(P(A)\rightarrow x\in A) \}$ ist.
\end{remark}

\subsubsection{Regeln für den unendlichen Schnitt}
\label{rule:bigcapI} \label{rule:bigcapE}

Der unendliche Schnitt, dargestellt durch das Symbol \(\bigcap\), ist eine Operation über eine Menge von Mengen, die durch eine Eigenschaft definiert sind. Basierend auf unserer Definition für den unendlichen Schnitt können wir zwei grundlegende Regeln formulieren: die Einführungs- und die Eliminierungsregel.

Die Einführungsregel für den unendlichen Schnitt (\(\bigcap I\)) besagt, dass wenn wir zeigen können, dass ein Element \(x\) in jeder Menge enthalten ist, die eine bestimmte Eigenschaft \(P\) erfüllt, dann ist \(x\) im unendlichen Schnitt aller Mengen mit dieser Eigenschaft enthalten.

\[
\begin{array}{llll}
	i & (1) & P(C) & ... \\
	j & (2) & \forall A(P(A)\rightarrow x\in A) & ... \\
	i,j & (3) & x\in \bigcap \{ A \mid P(A) \} & \bigcapI{1,2} \\
\end{array}
\]

Die Eliminierungsregel für den unendlichen Schnitt (\( \bigcap E\)) besagt, dass wenn wir wissen, dass ein Element \(x\) im unendlichen Schnitt aller Mengen mit einer bestimmten Eigenschaft \(P\) enthalten ist und eine bestimmte Menge \(B\) diese Eigenschaft \(P\) erfüllt, dann ist \(x\) auch in \(B\) enthalten.

\[
\begin{array}{llll}
	i & (1) & x\in \bigcap \{ A \mid P(A) \} & ... \\
	i & (2) & P(B)\land x\in B & \bigcapE{1} \\
	i & (3) & \forall A(P(A)\rightarrow x\in A) & \bigcapE{1} \\
\end{array}
\]

\(i,j\) sind dabei Listen von Annahmen.


\label{PLpCRpImpBigcapSubLbPLpARpRbASubseteqC}
\begin{theorem}[\(P(C)\vdash \bigcap_{P(A)} A\subseteq C\)]
\end{theorem}

\begin{proof}
	\[
	\begin{array}{llll}
		1 & (1) & P(C) & \rA \\
		2 & (2) & x\in\bigcap_{P(A)} A & \rA \\
		2 & (3) & \forall A(P(A)\rightarrow x\in A) & \bigcapE{2} \\
		2 & (4) & P(C)\rightarrow x\in C & \rUE{3} \\		
		1,2 & (5) & x\in C & \rRE{1,4} \\	
		1 & (6) & x\in C & \subseteqIa{2,5} \\
	\end{array}
	\]
\end{proof}



\section{Das Axiom der Paarmenge}

Das Axiom der Paarmenge (Pair) ist ein weiteres grundlegendes Axiom in der Zermelo-Fraenkel Mengenlehre. Es ermöglicht die Konstruktion von Mengen, die aus genau zwei Elementen bestehen. Formal ausgedrückt:

\label{FaAFaBExCFaxLpxInCLrLpxEqualsAOrxEqualsBRpRp}
\[
\forall A \forall B \exists C \forall x (x \in C \leftrightarrow (x = A \lor x = B))
\]

Dieses Axiom garantiert, dass für jede zwei Mengen \( A \) und \( B \), es eine Menge \( C \) gibt, die genau \( A \) und \( B \) als Elemente enthält.

\label{FaxLpxInCLrLpxEqualsAOrxEqualsBRpRpAndFaxLpxInDLrLpxEqualsAOrxEqualsBRpRpImpCEqualsD}
\begin{theorem}[\(\forall x (x \in C \leftrightarrow (x = A \lor x = B)) \land \forall x (x \in D \leftrightarrow (x = A \lor x = B)) \vdash C = D\) (Eindeutigkeit der Paarmenge)]
    Seien C und D zwei Paarmengen, die beide die Elemente A und B enthalten, dann sind diese unter Verwendung des Extensionalitätsaxioms gleich.
\end{theorem}

\begin{tempdefinition}
\[\forall x(Q(x):= x=A\lor x=B)\]
\end{tempdefinition}
\begin{proof}
	\[
	\begin{array}{ll p{5cm} p{5cm}}
		1 & (1) & \ensuremath{\forall x (x \in C \leftrightarrow (x=A\lor x=B)) \land \forall x (x \in D \leftrightarrow (x=A\lor x=B))} & \rA \\
		1 & (2) & \ensuremath{\forall x (x \in C \leftrightarrow Q(x)) \land \forall x (x \in D \leftrightarrow Q(x))} & \rIE{df(Q),1} \\
		1 & (3) & \ensuremath{C=D} & \FaxLpxInBLrPLpxRpRpAndFaxLpxInCLrPLpxRpRpImpBEqualsC{2} \\
	\end{array}
	\]
\end{proof}



\subsubsection{Regeln für die Paarmenge}
\label{rule:pairI} \label{rule:pairE} \label{rule:pairIb}
Die Einführungsregel für die Paarmenge besagt, dass aus der Existenz von zwei Elementen \(A\) und \(B\) eine Menge \(C\) konstruiert werden kann, die genau diese Elemente enthält.
\[
\begin{array}{llll}
	i & (1) & x=A\lor x=B & ... \\
	i & (2) & x\in \{A,B\} & \pairI{1} \\
\end{array}
\]

\[
\begin{array}{llll}
	& (1) & A\in \{A,B\} & \pairIb \\
\end{array}
\]

\[
\begin{array}{llll}
	& (1) & B\in \{A,B\} & \pairIb \\
\end{array}
\]


Die Eliminierungsregel für die Paarmenge ermöglicht es uns, aus der Existenz einer Paarmenge \(C\) zu schließen, dass wenn ein Element \(x\) in \(C\) enthalten ist, dieses Element entweder \(A\) oder \(B\) sein muss.
\[
\begin{array}{llll}
	i & (1) & x\in \{A,B\} & ... \\
	i & (2) & x = A\lor x = B & \pairE{1} \\
\end{array}
\]

\(i\) ist dabei eine Liste von Annahmen.

\subsubsection{Regeln für die Nicht-Zugehörigkeit zur Paarmenge}
\label{rule:nPairI} \label{rule:nPairE}
Die Nicht-Einführungsregel für die Paarmenge (\(\nPairI{...}\)) besagt, dass aus der Tatsache, dass ein Element \(x\) weder gleich \(A\) noch gleich \(B\) ist, folgt, dass \(x\) nicht in der Menge \(\{A, B\}\) enthalten ist.
\[
\begin{array}{llll}
	i & (1) & x \neq A \land x \neq B & ... \\
	i & (2) & x \notin \{A,B\} & \nPairI{1} \\
\end{array}
\]

\[
\begin{array}{llll}
	i & (1) & x \neq A  & ... \\
	j & (2) & x \neq B  & ... \\
	i,j & (3) & x \notin \{A,B\} & \nPairI{1,2} \\
\end{array}
\]

Die Nicht-Eliminierungsregel für die Paarmenge (\(\nPairE{...}\)) erlaubt den Schluss, dass wenn ein Element \(x\) nicht in der Menge \(\{A, B\}\) enthalten ist, \(x\) weder \(A\) noch \(B\) sein kann.
\[
\begin{array}{llll}
	i & (1) & x \notin \{A,B\} & ... \\
	i & (2) & x \neq A \land x \neq B & \nPairE{1} \\
\end{array} 
\]

\[
\begin{array}{llll}
	i & (1) & x \notin \{A,B\} & ... \\
	i & (2) & x \neq A  & \nPairE{1} \\
\end{array}
\]

\[
\begin{array}{llll}
	i & (1) & x \notin \{A,B\} & ... \\
	i & (2) & x \neq B  & \nPairE{1} \\
\end{array}
\]


\(i,j\) ist dabei eine Liste von Annahmen.

\paragraph{Existenzregel der Paarmenge}
\label{rule:pairSetExists}
Die Existenzregel stellt sicher, dass für alle Mengen \(A\) und \(B\) die Paarmenge \(\{A, B\}\) ebenfalls eine Menge ist.

\[
\begin{array}{llll}
	i & (1) & A \text{ ist eine Menge} & \dots  \\
        j & (2) & A \text{ ist eine Menge} & \dots  \\
	i,j & (3) & \{A, B\}\text{ ist eine Menge} & \pairSetExists{1} \\
\end{array}
\]

\(i\) und \(j\) sind dabei Listen von Annahmen.


\subsection{Eigenschaften der Paarmenge}

\label{ImpLbawbRbNotEqualsEmptyset}
\begin{theorem}[\(\vdash \{a,b\}\neq\emptyset\)]
	Jede Paarmenge ist keine leere Menge.
\end{theorem}
\begin{proof}
	\[
	\begin{array}{llll}
		& (1) & a\in \{a,b\} & \pairIb \\
		& (2) & \exists x(x\in \{a,b\}) & \rEI{1}  \\
		& (3) & \{a,b\}\neq\emptyset & \ExxInSImpSNotEqualsEmptyset{2}  \\
	\end{array}
	\]	
\end{proof}


\label{AEqualsBwaInLbAwBRbwbInLbAwBRbImpaEqualsb}
\begin{theorem}[\(A = B, a\in \{A,B\}, b\in \{A,B\}\vdash a=b\)]
	Wenn \( A = B \), dann enthält die Paarmenge \( C \) nur ein einzigartiges Element, nämlich \( A \).
\end{theorem}
\begin{proof}
	\[
	\begin{array}{llll}
		1 & (1) & A = B & \rA \\
		2 & (2) & a\in \{A,B\} & \rA \\
		3 & (3) & b\in \{A,B\} & \rA \\
		1,2 & (4) & a\in \{A,A\} & \rIE{1,2}\\
		1,3 & (5) & b\in \{A,A\} & \rIE{1,3}\\
		1,2 & (6) & a=A\lor a=A & \pairE{4}\\
		1,2 & (7) & a=A & \POrPEqvP{6}\\
		1,3 & (8) & b=A\lor b=A & \pairE{5}\\
		1,3 & (9) & b=A & \POrPEqvP{8}\\			
		1,2,3 & (10) & a=b & \aIdbwcIdbImpaIdc{7,9}\\			
	\end{array}
	\]
\end{proof}

\subsubsection{Regeln für die Einführung und Eliminierung von Elementen in einer Einermenge}
\label{rule:unitSetI} \label{rule:unitSetE} \label{rule:unitSetIb}

% Regel für die Einführung eines Elementes in einer Menge
Die Einführungsregel für Elemente in einer Menge (\( \in I \)) besagt, dass aus der Gleichheit von \( a \) und \( A \) gefolgert werden kann, dass \( a \) in der Menge \{A\} enthalten ist.
\[
\begin{array}{llll}
	i & (1) & a = A & ... \\
	i & (2) & a \in \{A\} & \unitSetI{1} \\
\end{array}
\]

\[
\begin{array}{llll}
	& (1) & a \in \{a\} & \unitSetIb \\
\end{array}
\]

% Regel für die Eliminierung eines Elementes aus einer Menge
Die Eliminierungsregel für Elemente aus einer Menge (\( \in E \)) ermöglicht es, aus der Tatsache, dass \( a \) in der Menge \{A\} enthalten ist, zu schließen, dass \( a = A \).
\[
\begin{array}{llll}
	i & (1) & a \in \{A\} & ... \\
	i & (2) & a = A & \unitSetE{1} \\
\end{array}
\]

\(i\) ist dabei eine Liste von Annahmen.

\subsubsection{Regeln für die Nicht-Zugehörigkeit zu einer Einermenge}
\label{rule:nUnitSetI} \label{rule:nUnitSetE}

% Regel für die Nicht-Einführung eines Elementes in einer Menge
Die Nicht-Einführungsregel für Elemente in einer Einermenge (\(\nUnitSetI{...}\)) besagt, dass aus der Ungleichheit von \(a\) und \(b\) gefolgert werden kann, dass \(a\) nicht in der Menge \(\{b\}\) und umgekehrt enthalten ist.
\[
\begin{array}{llll}
	i & (1) & a \neq b & ... \\
	i & (2) & a \notin \{b\} & \nUnitSetI{1} \\
\end{array}
\]

\[
\begin{array}{llll}
	i & (1) & a \neq b & ... \\
	i & (2) & b \notin \{a\} & \nUnitSetI{1} \\
\end{array}
\]

% Regel für die Nicht-Eliminierung eines Elementes aus einer Menge
Die Nicht-Eliminierungsregel für Elemente aus einer Einermenge (\(\nUnitSetE{...}\)) ermöglicht es, aus der Tatsache, dass \(a\) nicht in der Menge \(\{b\}\) enthalten ist, zu schließen, dass \(a\) nicht gleich \(b\) und umgekehrt ist.
\[
\begin{array}{llll}
	i & (1) & a \notin \{b\} & ... \\
	i & (2) & a \neq b & \nUnitSetE{1} \\
\end{array}
\]

\[
\begin{array}{llll}
	i & (1) & a \notin \{b\} & ... \\
	i & (2) & b \neq a & \nUnitSetE{1} \\
\end{array}
\]


\(i\) ist dabei eine Liste von Annahmen.	

\begin{remark}
	Die Paarmenge ist ein wichtiges Konzept in der Mengenlehre und dient als Grundlage für die Definition von geordneten Paaren, Relationen und Funktionen. Es ermöglicht auch die Konstruktion komplexerer Mengenstrukturen. Die eindeutige Paarmenge, die die Elemente \( A \) und \( B \) enthält, wird oft als \( \{ A, B \} \) bezeichnet. Dies ist eine kompakte Notation, die in der Mengenlehre weit verbreitet ist. Wenn \( A = B \), dann wird die Paarmenge einfach als \( \{ A \} \) oder \( \{ B \} \) bezeichnet, da in diesem Fall die Menge nur ein einzigartiges Element enthält.
\end{remark}

\label{aInAImpLbaRbSubseteqA}
\begin{theorem}[\(a\in A\vdash \{a\}\subseteq A\)]
\end{theorem}
\begin{proof}
	\[
	\begin{array}{llll}
		1 & (1) & x\in \{a\} & \rA \\
		2 & (2) & a\in A & \rA \\
		1 & (3) & x=a & \unitSetE{1} \\		
		1,2 & (4) & x\in A & \rIE{3,2} \\	
		2 & (4) & \{a\}\subseteq A & \subseteqIa{1,4} \\	
	\end{array}
	\]
\end{proof}

\section{Definition der Differenz}

\begin{definition}[\(A \setminus B := \{ x \in A \mid x \notin B \}\) (Differenz)]
	Die Differenz von zwei Mengen \(A\) und \(B\), bezeichnet mit \(A \setminus B\) oder \(A - B\), ist die Menge aller Elemente, die in \(A\) enthalten sind, aber nicht in \(B\). 
	Formal ausgedrückt unter Verwendung des Aussonderungsaxioms:
	\[
	\forall A \forall B \exists C \forall x (x \in C \leftrightarrow (x \in A \land x \notin B))
	\]
\end{definition}

\subsubsection{Regeln für die Differenz}
\label{rule:diffI} \label{rule:diffEa} \label{rule:diffEb}

Die Differenz, oft dargestellt durch das Symbol \(\setminus\) oder \(-\), ist eine Operation auf zwei Mengen, die die Menge aller Elemente, die in \(A\) enthalten sind, aber nicht in \(B\), bildet. Wir können auf Basis der vorherigen Definition nun Einführungs- und Eliminationsregeln für die Differenz definieren.

Die Einführungsregel für die Differenz (\( \setminus I \)) ermöglicht es uns, aus der Zugehörigkeit eines Elements zu \(A\) und der Nicht-Zugehörigkeit desselben Elements zu \(B\) die Zugehörigkeit dieses Elements zur Differenz \(A \setminus B\) zu schließen:

\[
\begin{array}{llll}
	i & (1) \, x \in A & ... & \\
	j & (2) \, x \notin B & ... & \\
	i,j & (3) \, x \in A \setminus B & \diffI{1,2} \\
\end{array}
\]

Die Eliminierungsregel für die Differenz (\(\setminus E\)) besagt, dass wenn wir ein Element haben, das zur Differenz \(A \setminus B\) gehört, wir daraus schließen können, dass dieses Element zu \(A\) gehört und nicht zu \(B\):

\[
\begin{array}{llll}
	i & (1) \, x \in A \setminus B & ... & \\
	i & (2) \, x \in A & \diffEa{1} \\
	i & (3) \, x \notin B & \diffEb{1} \\
\end{array}
\]

\(i,j\) sind dabei Listen von Annahmen.

\paragraph{Existenzregel der Differenzmenge}
\label{rule:differenceSetExists}
Die Existenzregel stellt sicher, dass für alle Mengen \(A\) und \(B\) die Paarmenge \(A\setminus B\) ebenfalls eine Menge ist.

\[
\begin{array}{llll}
	i & (1) & A \text{ ist eine Menge} & \dots  \\
        j & (2) & B \text{ ist eine Menge} & \dots  \\
	i,j & (3) & A\setminus B\text{ ist eine Menge} & \differenceSetExists{1} \\
\end{array}
\]

\(i\) und \(j\) sind dabei Listen von Annahmen.

\label{cInASetminusLbaRbImpcNotEqualsa}
\begin{theorem}[\( c\in A\setminus\{a\}\vdash c\neq a \)]
\end{theorem}
\begin{proof}
    \[
	\begin{array}{llll}
	    1 & (1) & c\in A\setminus\{a\} & \rA \\
            1 & (2) & c\notin \{a\} & \diffEb{1} \\
		1 & (3) & c\neq a & \nUnitSetE{2} \\
	\end{array}
    \]
\end{proof}

\label{cInASetminusLbawbRbImpcNotEqualsa}
\begin{theorem}[\( c\in A\setminus\{a,b\}\vdash c\neq a \)]
\end{theorem}
\begin{proof}
    \[
	\begin{array}{llll}
		1 & (1) & c\in A\setminus\{a,b\} & \rA \\
		1 & (2) & c\notin \{a,b\} & \diffEb{1} \\
		1 & (3) & c\neq a & \nPairE{2} \\
	\end{array}
    \]
\end{proof}

\label{cInASetminusLbawbRbImpcNotEqualsb}
\begin{theorem}[\( c\in A\setminus\{a,b\}\vdash c\neq b \)]
\end{theorem}
\begin{proof}
    \[
	\begin{array}{llll}
		1 & (1) & c\in A\setminus\{a,b\} & \rA \\
		1 & (2) & c\notin \{a,b\} & \diffEb{1} \\
		1 & (3) & c\neq b & \nPairE{2} \\
	\end{array}
    \]
\end{proof}

\label{cInASetminusBwbInBImpcNotEqualsb}
\begin{theorem}[\( c\in A\setminus B, b\in B\vdash c\neq b \)]
\end{theorem}
\begin{proof}
    \[
	\begin{array}{llll}
		1 & (1) & c\in A\setminus B & \rA \\
		2 & (2) & b\in B & \rA \\
		1 & (3) & c\notin B & \diffEb{1} \\
		4 & (4) & c=b & \rA \\
		1,4 & (5) & b\notin B & \rIE{4,3} \\
		1,2,4 & (6) & \bot & \rBI{2,5} \\
		1,2 & (6) & c\neq b & \rCI{4,6} \\
	\end{array}
    \]
\end{proof}

\label{aNotinASetminusLbaRb}
\begin{theorem}[\(a\notin A\setminus\{a\}\)]
\end{theorem}
\begin{proof}
    \[
	\begin{array}{llll}
		1 & (1) & a\in A\setminus\{a\} & \rA \\
		& (2) & a=a & \rII \\
		1 & (3) & a\neq a & \cInASetminusLbaRbImpcNotEqualsa{1} \\
		1 & (4) & \bot & \rBI{2,3} \\
		& (5) & a\notin A\setminus\{a\} & \rCI{1,4} \\
	\end{array}
    \]
\end{proof}

\label{aNotinASetminusLbawbRb}
\begin{theorem}[\(a\notin A\setminus\{a,b\}\)]
\end{theorem}
\begin{proof}
    \[
	\begin{array}{llll}
		1 & (1) & a\in A\setminus\{a,b\} & \rA \\
		& (2) & a=a & \rII \\
		1 & (3) & a\neq a & \cInASetminusLbawbRbImpcNotEqualsa{1} \\
		1 & (4) & \bot & \rBI{2,3} \\
		& (5) & a\notin A\setminus\{a,b\} & \rCI{1,4} \\
	\end{array}
    \]
\end{proof}

\label{bNotinASetminusLbawbRb}
\begin{theorem}[\(b\notin A\setminus\{a,b\}\)]
\end{theorem}
\begin{proof}
    \[
	\begin{array}{llll}
		1 & (1) & b\in A\setminus\{a,b\} & \rA \\
		& (2) & b=b & \rII \\
		1 & (3) & b\neq b & \cInASetminusLbawbRbImpcNotEqualsb{1} \\
		1 & (4) & \bot & \rBI{2,3} \\
		& (5) & b\notin A\setminus\{a,b\} & \rCI{1,4} \\
	\end{array}
    \]
\end{proof}

\label{aInAwaNotEqualsbImpaInASetminusLbbRb}
\begin{theorem}[\(a\in A, a\neq b\vdash a\in A\setminus\{b\}\)]
\end{theorem}
\begin{proof}
    \[
	\begin{array}{llll}
		1 & (1) & a\in A & \rA \\
		2 & (2) & a\neq b & \rII \\
		2 & (3) & a\notin \{b\} & \nUnitSetI{2} \\
		1,2 & (4) & a\in A\setminus \{b\} & \diffI{1,2} \\
	\end{array}
    \]
\end{proof}

\label{aInAwbNotEqualsaImpaInASetminusLbbRb}
\begin{theorem}[\(a\in A, b\neq a\vdash a\in A\setminus\{b\}\)]
\end{theorem}
\begin{proof}
    \[
	\begin{array}{llll}
		1 & (1) & a\in A & \rA \\
		2 & (2) & b\neq a & \rII \\
            2 & (3) & a\neq b & \aNotEqualsbImpbNotEqualsa{2} \\
		1,2 & (4) & a\in A\setminus \{b\} & \aInAwaNotEqualsbImpaInASetminusLbbRb{1,3} \\
	\end{array}
    \]
\end{proof}



\section{Das Axiom der Vereinigung}

Das Axiom der Vereinigung (Union) ist ein weiteres grundlegendes Axiom in der Zermelo-Fraenkel Mengenlehre. Es ermöglicht die Konstruktion einer Menge, die alle Elemente einer gegebenen Menge von Mengen enthält. Formal ausgedrückt:

\label{FaAExBFaxLpxInBLrExCLpCInAAndxInCRpRp}
\[
\forall A \exists B \forall x (x \in B \leftrightarrow \exists C (C \in A \land x \in C))
\]

Dieses Axiom garantiert, dass für jede Menge \( A \), es eine Menge \( B \) gibt, die alle Elemente der Mengen in \( A \) enthält.

\begin{remark}
	Das Axiom der Vereinigung ist ein Schlüsselkonzept in der Mengenlehre und ermöglicht die Konstruktion komplexerer Mengenstrukturen. Es dient als Grundlage für die Definition von Vereinigungen von Mengen und ist ein wichtiger Baustein für die Entwicklung der Theorie der Kardinalität, Ordinalität und viele andere Konzepte in der Mengenlehre. Die eindeutige Vereinigung einer Menge \( A \) wird oft als \( \bigcup A \) oder \( \bigcup_{x \in A} x \) bezeichnet. Diese Notation stellt sicher, dass die Vereinigung eindeutig ist, da sie alle Elemente der Mengen in \( A \) enthält und durch das Axiom der Vereinigung und das Extensionalitätsaxiom eindeutig bestimmt ist.
\end{remark}

\label{FaxLpxInBLrExDLpDInAAndxInDRpRpAndFaxLpxInCLrExDLpDInAAndxInDRpRpImpBEqualsC}
\begin{theorem}[\(\forall x (x \in B \leftrightarrow \exists D (D \in A \land x \in D)) \land \forall x (x \in C \leftrightarrow \exists D (D \in A \land x \in D)) \vdash B = C\) (Eindeutigkeit der Vereinigung)]
	Seien \(A, B, C\), wobei B und C alle Elemente der Mengen in A enthalten, dann sind B und C unter Verwendung des Extensionalitätsaxioms gleich.
\end{theorem}
\begin{tempdefinition}
    \[\forall x(Q(x):=\exists D (D \in A \land x \in D))\]
\end{tempdefinition}
\begin{proof}
	\[
	\begin{array}{ll p{8cm} p{4cm}}
		1 & (1) & \ensuremath{\forall x (x \in B \leftrightarrow \exists D (D \in A \land x \in D)) \land \forall x (x \in C \leftrightarrow \exists D (D \in A \land x \in D))} & \rA \\
		1 & (2) & \ensuremath{\forall x (x \in B \leftrightarrow Q(x)) \land \forall x (x \in C \leftrightarrow Q(x))} & \rIE{df(Q),1} \\
		1 & (3) & \ensuremath{B=C} & \FaxLpxInBLrPLpxRpRpAndFaxLpxInCLrPLpxRpRpImpBEqualsC{2} \\
	\end{array}
	\]
\end{proof}

\subsubsection{Regeln für die Vereinigung von Mengen}
\label{rule:bigcupI} \label{rule:bigcupE}

% Regel für die Einführung einer Vereinigungsmenge
Die Einführungsregel für die Vereinigung von Mengen (\( \bigcup I \)) besagt, dass aus der Existenz einer Menge von Mengen \( A \) eine Menge \( B \) konstruiert werden kann, die alle Elemente der Mengen in \( A \) enthält.
\[
\begin{array}{llll}
	i & (1) & x \in C & ... \\
	j & (2) & C \in A & ... \\
	i,j & (2) & x \in \bigcup A & \bigcupI{1,2} \\
\end{array}
\]

% Regel für die Eliminierung einer Vereinigungsmenge
Die Eliminierungsregel für die Vereinigung von Mengen (\( \bigcup E \)) ermöglicht es, aus der Existenz einer Vereinigungsmenge \( B \) zu schließen, dass wenn ein Element \( x \) in \( B \) enthalten ist, dieses Element in mindestens einer der Mengen in der Menge \( A \) enthalten sein muss.
\[
\begin{array}{llll}
	i & (1) & x \in \bigcup A & ... \\
	i & (2) & \exists C (C \in A \land x \in C) & \bigcupE{1} \\
\end{array}
\]

\(i,j\) sind dabei Listen von Annahmen.

\subsection{Eigenschaften der Vereinigung}

\label{ASubseteqBImpBigcupASubseteqBigcupB}
\begin{theorem}[\(A \subseteq B \vdash \bigcup A \subseteq \bigcup B\)]
\end{theorem}
\begin{proof}
	\[
	\begin{array}{llll}
		1 & (1) & A \subseteq B & \rA \\
		2 & (2) & x\in\bigcup A & \rA \\
		2 & (3) & \exists X(X\in A\land x\in X)A & \bigcupE{2} \\
		4 & (4) & C\in A\land x\in C & \rA \\
		4 & (5) & C\in A & \rAEa{4} \\
		4 & (6) & x\in C & \rAEb{4} \\
		1,4 & (7) & C\in B & \subseteqE{5,1} \\
		1,4 & (8) & x\in\bigcup B & \bigcupI{7,6} \\
		1,2 & (9) & x\in\bigcup B & \rEE{3,4,8} \\
		1 & (10) & \bigcup A\subseteq \bigcup B & \subseteqIa{2,9} \\
	\end{array}
	\]
\end{proof}

\label{AcuBEqualsBigcupLbAwBRb}
\section{Definition der Vereinigung zweier Mengen}
\begin{definition}[Vereinigung zweier Mengen]
	Die Vereinigung von zwei Mengen \( A \) und \( B \) ist die Menge aller Elemente, die entweder in \( A \) oder in \( B \) oder in beiden enthalten sind. Dies wird als \( A \cup B \) bezeichnet und kann als eine spezielle Anwendung des Axioms der Vereinigung formuliert werden:
	\[
	A \cup B = \bigcup \{ A, B \}
	\]
\end{definition}

\label{zInAcuBEqvLpzInAOrzInBRp}
\begin{theorem}[\(z \in A \cup B \dashv\vdash (z \in A \lor z \in B)\)]
\end{theorem}	
\begin{proof}
	\(\vdash:\)
	\[
	\begin{array}{llll}
		1 & (1) & z\in A \cup B & \rA \\
		1 & (2) & z\in \bigcup\{A,B\} & \AcuBEqualsBigcupLbAwBRb{1} \\
		1 & (3) & \exists C (C\in \{A,B\}\wedge z\in C) & \bigcupE{2} \\
		1 & (4) & C\in \{A,B\}\wedge z\in C & \rEE{3} \\
		1 & (5) & C\in \{A,B\} & \rAEa{4} \\
		1 & (6) & C=A\lor C=B & \pairE{5} \\
		7 & (7) & C=A & \rA \\
		1 & (8) & z\in C & \rAEb{4} \\
		1,7 & (9) & z\in A & \rIE{7,8} \\			
		1,7 & (10) & z\in A\lor z\in B & \rOIa{9} \\
		11 & (11) & C=B & \rA \\
		1,11 & (12) & z\in B & \rIE{11,9} \\			
		1,11 & (13) & z\in A\lor z\in B & \rOIb{9} \\
		1 & (14) & z\in A\lor z\in B & \rOE{6,7,10,11,13}				
	\end{array}
	\]
	\(\dashv:\)
	\[
	\begin{array}{llll}
		1 & (1) & z\in A\lor z\in B & \rA\\			
		2 & (2) & z\in A & \rA\\
		& (3) & A = A & \rII\\
		& (4) & A = A\lor A = B & \rOIa{3}\\
		& (5) & A\in\{A,B\} & \pairI{4}\\
		2 & (6) & z\in\bigcup \{A,B\} & \bigcupI{5,2}\\
		2 & (7) & z\in A\cup B & Df. Vereinigung\\
		8 & (8) & z\in B & \rA\\
		& (9) & B = B & \rII\\
		& (10) & A = B\lor B = B & \rOIb{9}\\
		& (11) & B\in\{A,B\} & \pairI{10}\\
		8 & (12) & z\in \bigcup{A,B} & \bigcupI{11,8}\\
		8 & (13) & z\in A\cup B & Df. Vereinigung\\
		1 & (14) & z\in A\cup B & \rOE{1,2,7,8,13}\\
	\end{array}
	\]
\end{proof}


\subsubsection{Regeln für die Vereinigung zweier Mengen}
\label{rule:cupI} \label{rule:cupE}

% Regel für die Einführung der Vereinigung zweier Mengen
Die Einführungsregel für die Vereinigung zweier Mengen (\( \cup I \)) besagt, dass, wenn ein Element \( x \) in der Menge \( A \) oder in der Menge \( B \) enthalten ist, es auch in der Vereinigung \( A \cup B \) enthalten ist.
\[
\begin{array}{llll}
	i & (1) & x \in A & ... \\
	i & (2) & x \in A \cup B & \cupI{1} \\
\end{array}
\]

\[
\begin{array}{llll}
	i & (1) & x \in B & ... \\
	i & (2) & x \in A \cup B & \cupI{1} \\
\end{array}
\]

\[
\begin{array}{llll}
	i & (1) & x\in A\lor x \in B & ... \\
	i & (2) & x \in A \cup B & \cupI{1} \\
\end{array}
\]



% Regel für die Eliminierung der Vereinigung zweier Mengen
Die Eliminierungsregel für die Vereinigung zweier Mengen (\( \cup E \)) ermöglicht es, aus der Existenz eines Elements \( x \) in der Vereinigung \( A \cup B \) zu schließen, dass dieses Element entweder in \( A \) oder in \( B \) enthalten sein muss.
\[
\begin{array}{llll}
	i & (1) & x \in A \cup B & ... \\
	& (2) & x \in A \lor x \in B & \cupE{1} \\
\end{array}
\]

\(i\) ist dabei eine Liste von Annahmen.

\paragraph{Existenzregel der Vereinigungsmenge}
\label{rule:cupSetExists}
Die Existenzregel stellt sicher, dass für alle Mengen \(A\) und \(B\) die Paarmenge \(A\cup B\) ebenfalls eine Menge ist.

\[
\begin{array}{llll}
	i & (1) & A \text{ ist eine Menge} & \dots  \\
        j & (2) & B \text{ ist eine Menge} & \dots  \\
	i,j & (3) & A\cup B\text{ ist eine Menge} & \cupSetExists{1,2} \\
\end{array}
\]

\(i\) und \(j\) sind dabei Listen von Annahmen.


\label{LbawbRbEqualsLbaRbcuLbbRb}
\begin{theorem}[\(\{a,b\}=\{a\}\cup \{b\}\)]
\end{theorem}	
\begin{proof}
	\[
	\begin{array}{llll}
		1 & (1) & x\in \{a,b\} & \rA \\
		1 & (2) & x=a\lor x=b & \pairE{1} \\
		1 & (3) & x=a\lor x=b & \pairE{1} \\
		4 & (4) & x=a & \rA \\
		4 & (5) & x\in \{a\} & \unitSetI{4} \\
		4 & (6) & x\in \{a\}\lor x\in \{b\} & \rOIa{5} \\
		4 & (7) & x\in \{a\}\cup \{b\} & \cupI{6} \\
		8 & (8) & x=b & \rA \\
		8 & (9) & x\in \{b\} & \unitSetI{8} \\
		8 & (10) & x\in \{a\}\lor x\in \{b\} & \rOIb{9} \\
		8 & (11) & x\in \{a\}\cup \{b\} & \cupI{10} \\
		1 & (12) & x\in \{a\}\cup \{b\} & \rOE{3,4,7,8,11} \\
		& (13) & \{a,b\}\subseteq \{a\}\cup \{b\} & \subseteqIa{1,12} \\
		14& (14) & x\in \{a\}\cup \{b\} & \rA \\
		14& (15) & x\in \{a\}\lor x\in \{b\} & \cupE{14} \\
		14& (16) & x\in \{a\}\lor x\in \{b\} & \cupE{14} \\
		17& (17) & x\in \{a\} & \rA \\
		17& (18) & x = \{a\} & \unitSetE{17} \\
		17& (19) & x = \{a\}\lor x = \{b\} & \rOIa{18} \\
		17& (20) & x \in \{a,b\} & \pairI{19} \\
		21& (21) & x\in \{b\} & \rA \\
		21& (22) & x = \{b\} & \unitSetE{21} \\
		21& (23) & x = \{a\}\lor x = \{b\} & \rOIb{22} \\
		21& (24) & x \in \{a,b\} & \pairI{23} \\
		14& (25) & x \in \{a,b\} & \rOE{16,17,20,21,24} \\
		& (26) & \{a\}\cup \{b\}\subseteq \{a,b\} & \subseteqIa{14,25} \\
		& (27) & \{a,b\} = \{a\}\cup \{b\} & \rIIaSet{13,26} \\
	\end{array}
	\]
\end{proof}

\label{aInAcuLbaRb}
\begin{theorem}[\(a\in A\cup\{a\}\)]
\end{theorem}
\begin{proof}
    \[
	\begin{array}{llll}
		& (1) & a\in\{a\} & \unitSetIb \\
		& (1) & a\in A\cup\{a\} & \cupI{1} \\
	\end{array}
    \]
\end{proof}

\label{aInLbaRbcuA}
\begin{theorem}[\(a\in \{a\}\cup A\)]
\end{theorem}
\begin{proof}
    \[
	\begin{array}{llll}
		& (1) & a\in\{a\} & \unitSetIb \\
		& (1) & a\in\{a\}\cup A & \cupI{1} \\
	\end{array}
    \]
\end{proof}

\label{aNotinBwAEqualsBcuLbaRbImpANsubseteqB}
\begin{theorem}[\(a\notin B, A=B\cup \{a\}\vdash A\nsubseteq B\)]
\end{theorem}
\begin{proof}
	\[
	\begin{array}{llll}
		1 & (1) & a\notin B & \rA \\
		2 & (2) & A=B\cup \{a\} & \rA \\
		3 & (3) & A\subseteq B & \rA \\
		& (4) & a\in B\cup \{a\} & \aInAcuLbaRb{} \\
		2 & (5) & a\in A & \rIE{2,4} \\
		2,3 & (6) & a\in B & \subseteqE{5,3} \\
		2,3 & (7) & a\in B & \subseteqE{5,3} \\
		1,2,3 & (8) & \bot & \rBI{1,7} \\
		1,2 & (9) & A\nsubseteq B & \rBI{1,7} \\
	\end{array}
	\]
\end{proof}

\label{AEqualsAcuA}
\begin{theorem}[\(A = A \cup A\)]
\end{theorem}
\begin{proof}
	\[
	\begin{array}{llll}
		1 & (1) & x \in A \cup A & \rA \\
		1 & (2) & x \in A \lor x \in A & \cupE{1} \\
		1 & (3) & x \in A & \POrPEqvP{2} \\
		& (4) & A\cup A\subseteq A & \subseteqIa{1,3} \\
		5 & (5) & x\in A & \rA \\
		5 & (6) & x\in A\cup A & \cupI{5} \\
		& (7) & A\subseteq A\cup A & \subseteqIa{5,6} \\
		& (8) & A=A\cup A& \rIIaSet{4,7} \\
	\end{array}
	\]
\end{proof}

\label{AcuBEqualsBcuA}
\begin{theorem}[\(A \cup B = B \cup A\)]
\end{theorem}
\begin{proof}
	\[
	\begin{array}{llll}
		1 & (1) & x \in A \cup B & \rA \\
		1 & (2) & x \in A\lor x\in B & \cupE{1} \\
		1 & (3) & x \in B\lor x\in A & \POrQImpQOrP{2} \\
		1 & (4) & x \in B\cup A & \cupI{3} \\
		& (5) & A\cup B\subseteq B\cup A & \subseteqIa{1,4} \\
		6 & (6) & x \in B\cup A & \rA \\
		6 & (7) & x \in B\lor x\in A & \cupE{6} \\
		6 & (8) & x \in A\lor x\in B & \POrQImpQOrP{7} \\
		6 & (9) & x \in A\cup B & \cupI{8} \\
		& (10) & B\cup A\subseteq A\cup B & \subseteqIa{6,9} \\		
		& (11) & A\cup B = B\cup A & \rIIaSet{5,10} \\		
	\end{array}
	\]
\end{proof}

\subsubsection{Regeln zur Kommutativität der Vereinigung zweier Mengen}
\label{rule:kommCup}

% Regel für die Kommutativität der Vereinigung zweier Mengen (Überführung)
Die Regel für die Überführung von \(A \cup B\) zu \(B \cup A\) basiert auf der Kommutativität der Vereinigung und kann wie folgt ausgedrückt werden:
\[
\begin{array}{llll}
	i & (1) & A \cup B & \rA \\
	i & (2) & B \cup A & \kommCup{1} \\
\end{array}
\]

% Regel für die Zugehörigkeit eines Elements zu A\cup B und damit zu B\cup A
Die Regel, die zeigt, dass wenn ein Element \(z\) in \(A \cup B\) enthalten ist, es auch in \(B \cup A\) enthalten ist, nutzt ebenfalls die Kommutativität der Vereinigung:
\[
\begin{array}{llll}
	i & (1) & z \in A \cup B & \rA \\
	i & (2) & z \in B \cup A & \kommCup{1} \\
\end{array}
\]

\(i\) ist dabei eine Liste von Annahmen. 


\label{AEqualsAcuEmptyset}
\begin{theorem}[\(A=A\cup\emptyset\)]
\end{theorem}	
\begin{proof}
	\[
	\begin{array}{llll}
		1 & (1) & x\in A & \rA \\
		  1 & (2) & x\in A\cup\emptyset & \cupI{1} \\
        1 & (3) & A\subseteq A\cup\emptyset & \subseteqIa{1,2} \\
    	4 & (4) & x\in A\cup\emptyset & \rA \\
        4 & (5) & x\in A\lor x\in\emptyset & \cupE{4} \\
        4 & (6) & x\notin \emptyset & \remptyset{} \\
        4 & (7) & x\in A & \POrQwnQImpP{5,6} \\
        4 & (8) & A\cup\emptyset\subseteq A & \subseteqIa{4,7} \\        
          & (9) & A=A\cup\emptyset & \rIIaSet{4,8} \\        
	\end{array}
	\]
\end{proof}

\label{AEqualsEmptysetcuA}
\begin{theorem}[\(A=\emptyset\cup A\)]
\end{theorem}
\begin{proof}
	\[
	\begin{array}{llll}
		   & (1) & A=A\cup\emptyset & \AEqualsAcuEmptyset{} \\
          & (2) & A\cup\emptyset = \emptyset\cup A & \AcuBEqualsBcuA{} \\
          & (3) & A = \emptyset\cup A & \rIE{2,1} \\
  
	\end{array}
	\]
\end{proof}

\label{LbEmptysetRbEqualsAcuLbARbEqvAEqualsEmptyset}
\begin{theorem}[\(\{\emptyset\} = A \cup \{A\} \dashv\vdash A = \emptyset\)]
\end{theorem}
\begin{proof}
\(\vdash:\)
\[
\begin{array}{llll}
    1       & (1) & \{\emptyset\} = A \cup \{A\} & \rA \\
            & (2) & A\in A \cup \{A\} & \aInAcuLbaRb{} \\
    1       & (3) & A\in\{\emptyset\} & \rIEaSet{1,2} \\
    1       & (4) & A=\emptyset & \unitSetE{3} \\
\end{array}
\]
\(\dashv:\)
\[
\begin{array}{llll}
    1 & (1) & A = \emptyset & \rA \\
      & (2) & \{\emptyset\} = \emptyset\cup\{\emptyset\} & \AEqualsEmptysetcuA{} \\
    1 & (3) & \{\emptyset\} = A\cup\{A\} & \rIE{1,2} \\
\end{array}
\]

\end{proof}


\label{zInAcuBEqvzNotinATozInB}
\begin{theorem}[\(z\in A \cup B \dashv\vdash z\notin A \rightarrow z\in B\)]
\end{theorem}
\begin{proof}
    \(\vdash:\)
	\[
	\begin{array}{llll}
		1 & (1) & z \in A \cup B & \rA \\
		1 & (2) & z\in A\lor z\in B & \cupEb{1} \\
		3 & (3) & z\in A & \rA \\
		3 & (4) & \neg(z\notin A) & \rDN{3}\\
		3 & (5) & \neg(z\notin A)\lor z\in B & \rOIa{4}\\
		6 & (6) & z\in B & \rA\\
		6 & (7) & \neg(z\notin A)\lor z\in B & \rOIb{6}\\
		1 & (8) & \neg(z\notin A)\lor z\in B & \rOE{2,3,5,6,8}\\
		1 & (9) & z\notin A\rightarrow z\in B & \PToQEqvnPOrQ{8}\\
	\end{array}
	\]	
	\(\dashv:\)
	\[
	\begin{array}{llll}
		1 & (1) & z\notin A\rightarrow z\in B & \rA \\
		1 & (2) & \neg(z\notin A)\lor z\in B & \PToQEqvnPOrQ{1} \\
		3 & (3) & \neg(z\notin A) & \rA \\
		3 & (4) & z\in A & \rDN{3} \\
		3 & (5) & z\in A\lor z\in B & \rOIa{4} \\
		6 & (6) & z\in B & \rA \\
		6 & (7) & z\in A\lor z\in B & \rOIb{6} \\
		1 & (8) & z\in A\lor z\in B & \rOE{2,3,5,6,7} \\
		1 & (9) & z\in A\cup B & \cupIb{8} \\
	\end{array}
	\]	
\end{proof}

\label{zInAcuBEqvzNotinBTozInA}
\begin{theorem}[\(z\in A \cup B \dashv\vdash z\notin B \rightarrow z\in A\)]
\end{theorem}
\begin{proof}
		\(\vdash:\)
		\[
		\begin{array}{llll}
			1 & (1) & z \in A \cup B & \rA \\
			1 & (2) & z\in B\cup A & \kommCup{1}  \\
			1 & (3) & z\notin B\rightarrow z\in A & \zInAcuBEqvzNotinATozInB{2}  \\
		\end{array}
		\]	
		\(\dashv:\)
		\[
		\begin{array}{llll}
			1 & (1) & z\notin A\rightarrow z\in B & \rA \\
			1 & (2) & z\in B\cup A & \zInAcuBEqvzNotinATozInB{1} \\
			1 & (3) & z\in A\cup B & \kommCup{2} \\
		\end{array}
		\]		
\end{proof}

\label{zInLpAcuBRpcuCEqvLpzInAOrzInBRpOrzInC}
\begin{theorem}[\(z \in (A \cup B)\cup C \dashv\vdash (z \in A \lor z \in B)\lor z\in C\)]
\end{theorem}	
\begin{proof}
	\(\vdash:\)
	\[
	\begin{array}{llll}
		1 & (1) & z\in (A \cup B)\cup C & \rA \\
		1 & (2) & z\in (A \cup B)\lor z\in C & \cupE{1} \\
		3 & (3) & z\in (A \cup B) & \rA \\
		3 & (4) & z\in A\lor z\in B & \cupE{3} \\
		3 & (5) & (z\in A\lor z\in B)\lor z\in C & \rOIa{4} \\
		6 & (6) & z\in C & \rA \\
		6 & (7) & (z\in A\lor z\in B)\lor z\in C & \rOIb{6}\\
		1 & (8) & (z\in A\lor z\in B)\lor z\in C & \rOE{2,3,5,6,7}\\		
	\end{array}
	\]
	\(\dashv:\)
	\[
	\begin{array}{llll}
		1 & (1) & (z\in A\lor z\in B)\lor z\in C & \rA\\			
		2 & (2) & z\in A\lor z\in B & \rA\\			
		2 & (3) & z\in (A\cup B) & \cupI{2}\\			
		2 & (4) & z\in (A\cup B)\cup C & \cupI{2}\\			
		5 & (5) & z\in C & \rA\\		
		5 & (6) & z\in (A\cup B)\cup C & \cupI{5}\\		
		1 & (7) & z\in (A\cup B)\cup C & \rOE{1,2,4,5,6}\\				
	\end{array}
	\]
\end{proof}

\label{zInAcuLpBcuCRpEqvzInAOrLpzInBOrzInCRp}
\begin{theorem}[\(z \in A \cup (B\cup C) \dashv\vdash z \in A \lor (z \in B\lor z\in C)\)]
\end{theorem}	
\begin{proof}
	\(\vdash:\)
	\[
	\begin{array}{llll}
		1 & (1) & z\in A \cup (B\cup C) & \rA \\
		1 & (2) & z\in A\lor z\in (B\cup C) & \cupE{1} \\
		3 & (3) & z\in (B \cup C) & \rA \\
		3 & (4) & z\in B\lor z\in C & \cupE{3} \\
		3 & (5) & z\in A\lor (z\in B\lor z\in C) & \rOIa{4} \\
		6 & (6) & z\in A & \rA \\
		6 & (7) & z\in A\lor (z\in B\lor z\in C) & \rOIb{6}\\
		1 & (8) & z\in A\lor (z\in B\lor z\in C) & \rOE{2,3,5,6,7}\\		
	\end{array}
	\]
	\(\dashv:\)
	\[
	\begin{array}{llll}
		1 & (1) & z\in A\lor (z\in B\lor z\in C) & \rA\\			
		2 & (2) & z\in B\lor z\in C & \rA\\			
		2 & (3) & z\in (B\cup C) & \cupI{2}\\			
		2 & (4) & z\in A\cup (B\cup C) & \cupI{2}\\			
		5 & (5) & z\in A & \rA\\		
		5 & (6) & z\in A\cup (B\cup C) & \cupI{5}\\		
		1 & (7) & z\in A\cup (B\cup C) & \rOE{1,2,4,5,6}\\				
	\end{array}
	\]
\end{proof}

\subsubsection{Regeln für die Vereinigung von drei Mengen}
\label{rule:cupIb} \label{rule:cupEb}

% Regel für die Einführung der Vereinigung von drei Mengen
Die Einführungsregel für die Vereinigung von drei Mengen (\( \cup Ib \)) besagt, dass, wenn ein Element \( z \) in der Menge \( A \), \( B \) oder \( C \) enthalten ist, es auch in der Vereinigung \( A \cup (B \cup C) \) enthalten ist.
\[
\begin{array}{llll}
	i & (1) & z \in A\lor (z\in B\lor z\in C) & ... \\
	i & (2) & z \in A \cup (B \cup C) & \cupIb{1} \\
\end{array}
\]

\[
\begin{array}{llll}
	i & (1) & (z \in A\lor z\in B)\lor z\in C & ... \\
	i & (2) & z \in (A \cup B) \cup C & \cupIb{1} \\
\end{array}
\]


% Regel für die Eliminierung der Vereinigung von drei Mengen
Die Eliminierungsregel für die Vereinigung von drei Mengen (\( \cup E_{3} \)) ermöglicht es, aus der Existenz eines Elements \( z \) in der Vereinigung \( A \cup (B \cup C) \) zu schließen, dass dieses Element entweder in \( A \) oder in \( B \) oder in \( C \) enthalten sein muss.
\[
\begin{array}{llll}
	i & (1) & z \in A \cup (B \cup C) & ... \\
	i & (2) & z \in A \lor (z \in B \lor z \in C) & \cupEb{1} \\
\end{array}
\]

\[
\begin{array}{llll}
	i & (1) & z \in (A \cup B) \cup C & ... \\
	i & (2) & (z \in A \lor z \in B) \lor z \in C & \cupEb{1} \\
\end{array}
\]

\(i\) ist dabei eine Liste von Annahmen.

\label{LpAcuBRpcuCEqualsAcuLpBcuCRp}
\begin{theorem}[\((A \cup B) \cup C = A \cup (B \cup C)\) (Assoziativität der Vereinigung)]
\end{theorem}

\begin{proof}
	\[
	\begin{array}{llll}
		1 & (1) & x \in (A \cup B) \cup C & \rA \\
		1 & (2) & (x \in A\lor x\in B)\lor x\in C & \cupEb{1} \\
		1 & (3) & x \in A\lor (x\in B\lor x\in C) &  \POrLpQOrRRpImpLpPOrQRpOrQ{2} \\
		1 & (4) & x\in A\cup (B\cup C) &  \cupIb{3} \\
		& (5) & (A\cup B)\cup C\subseteq A\cup (B\cup C) &  \subseteqIa{1,4} \\
		6 & (6) & x \in A \cup (B \cup C) & \rA \\
		6 & (7) & x \in A\lor (x\in B\lor x\in C) & \cupEb{6} \\
		6 & (8) & (x \in A\lor x\in B)\lor x\in C &  \POrLpQOrRRpImpLpPOrQRpOrQ{7} \\
		6 & (9) & x\in (A\cup B)\cup C &  \cupIb{8} \\
		& (10)& A\cup (B\cup C)\subseteq (A\cup B)\cup C &  \subseteqIa{6,9} \\
		& (11)& (A\cup B)\cup C = A\cup (B\cup C) &  \rIIaSet{5,10} \\
	\end{array}
	\]
\end{proof}

\label{zInAcuLpBcaCRpEqvzInAOrLpzInBAndzInCRp}
\begin{theorem}[\(z \in A \cup (B\cap C) \dashv\vdash z \in A \lor (z \in B\land z\in C)\)]
\end{theorem}	
\begin{proof}
	\(\vdash:\)
	\[
	\begin{array}{llll}
		1 & (1) & z \in A \cup (B\cap C) & \rA \\
		1 & (2) & z\in A\lor z\in (B\cap C) & \cupE{1} \\
		3 & (3) & z\in A & \rA \\
		3 & (4) & z\in A\lor (z\in B\land z\in C) & \rOIa{3} \\
		5 & (5) & z\in (B\cap C) & \rA \\
		5 & (6) & z\in B & \capEa{5} \\
		5 & (7) & z\in C & \capEb{5} \\
		5 & (8) & z\in B\land z\in C & \rAI{6,7} \\
		5 & (9) & z\in A\lor (z\in B\land z\in C) & \rOIb{8} \\
		1 & (10) & z\in A\lor (z\in B\land z\in C) & \rOE{2,3,4,5,9} \\
	\end{array}
	\]
	\(\dashv:\)
	\[
	\begin{array}{llll}
		1 & (1) & z\in A\lor (z\in B\land z\in C) & \rA\\			
		2 & (2) & z\in A & \rA\\			
		2 & (3) & z\in (A\cup (B\cap C)) & \cupI{2} \\			
		4 & (4) & z\in B\land z\in C & \rA \\	
		4 & (5) & z\in B\cap C & \capI{4} \\
		4 & (6) & z\in A\cup (B\cap C) & \cupI{5} \\
		1 & (7) & z\in A\cup (B\cap C) & \rOE{1,2,3,4,6} \\
	\end{array}
	\]
\end{proof}

\label{zInLpAcaBRpcuCEqvLpzInAAndzInBRpOrzInC}
\begin{theorem}[\(z \in (A\cap B)\cup C \dashv\vdash (z \in A\land z\in B)\lor z\in C\)]
\end{theorem}	

\begin{proof}
	\(\vdash:\)
	\[
	\begin{array}{llll}
		1 & (1) & z \in (A\cap B)\cup C & \rA \\
		1 & (2) & z\in (A\cap B)\lor z\in C & \cupE{1} \\
		3 & (3) & z\in C & \rA \\
		3 & (4) & (z\in A\land z\in B)\lor z\in C & \rOIb{3} \\
		5 & (5) & z\in (A\cap B) & \rA \\
		5 & (6) & z\in A & \capEa{5} \\
		5 & (7) & z\in B & \capEb{5} \\
		5 & (8) & z\in A\land z\in B & \rAI{6,7} \\
		5 & (9) & (z\in A\land z\in B)\lor z\in C & \rOIa{8} \\
		1 & (10) & (z\in A\land z\in B)\lor z\in C & \rOE{2,3,4,5,9} \\
	\end{array}
	\]
	\(\dashv:\)
	\[
	\begin{array}{llll}
		1 & (1) & (z\in A\land z\in B)\lor z\in C & \rA\\			
		2 & (2) & z\in C & \rA\\			
		2 & (3) & z\in (A\cap B)\cup C & \cupI{2} \\			
		4 & (4) & z\in A\land z\in B & \rA \\	
		4 & (5) & z\in A\cap B & \capI{4} \\
		4 & (6) & z\in (A\cap B)\cup C & \cupI{5} \\
		1 & (7) & z\in (A\cap B)\cup C & \rOE{1,2,3,4,6} \\
	\end{array}
	\]
\end{proof}

\label{zInAcaLpBcuCRpEqvzInAAndLpzInBOrzInCRp}
\begin{theorem}[\(z \in A\cap (B\cup C) \dashv\vdash z \in A\land (z\in B\lor z\in C)\)]
\end{theorem}	
\begin{proof}
	\(\vdash:\)
	\[
	\begin{array}{llll}
		1 & (1) & z \in A\cap (B\cup C) & \rA \\
		1 & (2) & z\in A & \capEa{1} \\
		1 & (3) & z\in (B\cup C) & \capEb{1} \\
		1 & (4) & z\in B\lor z\in C & \cupE{1} \\
		1 & (5) & z\in A\land(z\in B\lor z\in C) & \rAI{2,4} \\
	\end{array}
	\]
	\(\dashv:\)
	\[
	\begin{array}{llll}
		1 & (1) & z\in A\land (z\in B\lor z\in C) & \rA\\			
		1 & (2) & z\in A & \rAEa{1}\\	
		1 & (3) & z\in B\lor z\in C & \rAEb{1}\\	
		1 & (4) & z\in (B\cup C) & \cupI{3}\\	
		1 & (5) & z\in A\cap(B\cup C) & \capI{2,4}\\	
	\end{array}
	\]
\end{proof}

\label{zInLpAcuBRpcaLpCcuDRpEqvLpzInAOrzInBRpAndLpzInCOrzInDRp}
\begin{theorem}[\(z \in (A\cup B)\cap (C\cup D) \dashv\vdash (z \in A\lor z\in B)\land (z\in C\lor z\in D)\)]
\end{theorem}	

\begin{proof}
	\(\vdash:\)
	\[
	\begin{array}{llll}
		1 & (1) & z \in (A\cup B)\cap (C\cup D) & \rA \\
		1 & (2) & z \in (A\cup B) & \capEa{1} \\
		1 & (3) & z \in (C\cup D) & \capEb{1} \\
		1 & (4) & z \in A\lor z\in B & \cupE{2} \\
		1 & (5) & z \in C\lor z\in D & \cupE{3} \\
		1 & (6) & (z \in A\lor z\in B) \land (z \in C\lor z\in D) & \rAI{4,5} \\
	\end{array}
	\]
	\(\dashv:\)
	\[
	\begin{array}{llll}
		1 & (1) & (z \in A\lor z\in B) \land (z \in C\lor z\in D) & \rA\\
		1 & (2) & z \in A\lor z\in B & \rAEa{1}\\
		1 & (3) & z \in C\lor z\in D & \rAEb{1}\\
		1 & (4) & z \in (A\cup B) & \cupI{2}\\
		1 & (5) & z \in (C\cup D) & \cupI{3}\\
		1 & (6) & z \in (A\cup B)\cap (C\cup D) & \capI{4,5}\\
	\end{array}
	\]
\end{proof}

\label{zInLpAcaBRpcuLpCcaDRpEqvLpzInAAndzInBRpOrLpzInCAndzInDRp}
\begin{theorem}[\(z \in (A\cap B)\cup (C\cap D) \dashv\vdash (z \in A\land z\in B)\lor (z\in C\land z\in D)\)]
\end{theorem}	

\begin{proof}
	\(\vdash:\)
	\[
	\begin{array}{llll}
		1 & (1) & z \in (A\cap B)\cup (C\cap D) & \rA \\
		1 & (2) & z \in (A\cap B)\lor z\in (C\cap D) & \cupE{1} \\
		3 & (3) & z \in (A\cap B) & \rA \\
		3 & (4) & z \in A\land z\in B & \capE{3} \\
		3 & (5) & (z \in A\land z\in B)\lor (z\in C\land z\in D) & \rOIa{4} \\
		6 & (6) &  z \in (C\cap D) & \rA \\
		6 & (7) &  z \in C\land z\in D & \capE{6} \\
		6 & (8) &  (z \in A\land z\in B)\lor (z\in C\land z\in D) & \rOIb{7} \\		
		1 & (9) &  (z \in A\land z\in B)\lor (z\in C\land z\in D) & \rOE{2,3,5,6,8} \\	
	\end{array}
	\]
	\(\dashv:\)
	\[
	\begin{array}{llll}
		1 & (1) & (z \in A\land z\in B)\lor (z\in C\land z\in D) & \rA \\	
		2 & (2) & (z \in A\land z\in B) & \rA \\	
		2 & (3) & z\in (A\cap B) & \capI{2} \\
		2 & (4) & z\in (A\cap B)\cup(C\cap D) & \cupI{3} \\
		5 & (5) & (z\in C\land z\in D) & \rA \\		
		5 & (6) & z\in (C\cap D) & \capI{5} \\
		5 & (7) & z\in (A\cap B)\cup(C\cap D) & \cupI{6} \\
		1 & (8) & z\in (A\cap B)\cup(C\cap D) & \rOE{1,2,4,5,7} \\
	\end{array}
	\]
\end{proof}

\subsubsection{Regeln für die Vereinigung und den Durchschnitt von Mengen}
\label{rule:capcupI} \label{rule:capcupE}

% Regel für die Einführung der Vereinigung und des Durchschnitts von Mengen
Die Einführungsregel für die Vereinigung und den Durchschnitt von Mengen (\(\cup\cap I\)) ermöglicht es, aus der Tatsache, dass ein Element \( z \) in einer spezifischen Kombination von Mengen enthalten ist, zu schließen, dass es auch in der entsprechenden kombinierten Menge enthalten ist. Sie sind wie folgt definiert:

\[
\begin{array}{llll}
	i & (1) & z \in A \land (z \in B \lor z \in C) & ... \\
	i & (2) & z \in A \cap (B \cup C) & \capcupI{1} \\
\end{array}
\]

\[
\begin{array}{llll}
	i & (1) &  (z \in A \lor z \in B)\land z\in C & ... \\
	i & (2) & z\in (A \cup B)\cap C & \capcupI{1} \\
\end{array}
\]

\[
\begin{array}{llll}
	i & (1) & z \in A \lor (z \in B \land z \in C) & ... \\
	i & (2) & z \in A \cup (B \cap C) & \capcupI{1} \\
\end{array}
\]

\[
\begin{array}{llll}
	i & (1) &  (z \in A \land z \in B)\lor z\in C & ... \\
	i & (2) & z\in (A \cap B)\cup C & \capcupI{1} \\
\end{array}
\]

\[
\begin{array}{llll}
	i & (1) & (z \in A \lor z \in B) \land (z \in C \lor z \in D) & ... \\
	i & (2) & z \in (A \cup B) \cap (C \cup D) & \capcupI{1} \\
\end{array}
\]

\[
\begin{array}{llll}
	i & (1) & (z \in A \land z \in B) \lor (z \in C \land z \in D) & ... \\
	i & (2) & z \in (A \cap B) \cup (C \cap D) & \capcupI{1} \\
\end{array}
\]

% Regel für die Eliminierung der Vereinigung und des Durchschnitts von Mengen
Die Eliminierungsregel für die Vereinigung und den Durchschnitt von Mengen (\(\cup\cap E\)) ermöglicht es, aus der Existenz eines Elements \( z \) in einer kombinierten Menge zu schließen, dass dieses Element spezifischen Bedingungen in Bezug auf die ursprünglichen Mengen entspricht. Sie sind wie folgt definiert:

\[
\begin{array}{llll}
	i & (1) & z \in A \cap (B \cup C) & ... \\
	i & (2) & z \in A \land (z \in B \lor z \in C) & \capcupE{1} \\
	
\end{array}
\]

\[
\begin{array}{llll}
	i & (1) & z\in (A \cup B)\cap C & ... \\
	i & (2) &  (z \in A \lor z \in B)\land z\in C & \capcupE{1} \\
\end{array}
\]

\[
\begin{array}{llll}
	i & (1) & z \in A \cup (B \cap C) & ... \\
	i & (2) & z \in A \lor (z \in B \land z \in C) & \capcupE{1} \\
	
\end{array}
\]

\[
\begin{array}{llll}
	i & (1) & z\in (A \cap B)\cup C & ... \\
	i & (2) &  (z \in A \land z \in B)\lor z\in C & \capcupE{1} \\
\end{array}
\]

\[
\begin{array}{llll}
	i & (1) & z \in (A \cup B) \cap (C \cup D) & ... \\
	i & (2) & (z \in A \lor z \in B) \land (z \in C \lor z \in D) & \capcupE{1} \\
\end{array}
\]

\[
\begin{array}{llll}
	i & (1) & z \in (A \cap B) \cup (C \cap D) & ... \\
	i & (2) & (z \in A \land z \in B) \lor (z \in C \land z \in D) & \capcupE{1} \\
\end{array}
\]

\label{AcuLpBcaCRpEqualsLpAcuBRpcaLpAcuCRp}
\begin{theorem}[\(A \cup (B \cap C) = (A \cup B) \cap (A \cup C)\)]
\end{theorem}
\begin{proof}
	\[
	\begin{array}{llll}
		1 & (1) & x \in A \cup (B \cap C) & \rA \\
		1 & (2) & x \in A\lor (x \in B\land x\in C) & \capcupE{1} \\
		1 & (3) & (x \in A\lor x \in B)\land (x\in A\lor x\in C) & \POrLpQAndRRpEqvLpPOrQRpAndLpPOrRRp{2} \\
		1 & (4) & x\in (A\cup B)\cap (A\cup C) & \capcupI{3} \\		
		& (5) & A \cup (B \cap C)\subseteq (A\cup B)\cap (A\cup C) & \subseteqIa{1,4} \\	
		6 & (6) & x\in (A\cup B)\cap (A\cup C) & \rA \\	
		6 & (7) & (x \in A\lor x \in B)\land (x\in A\lor x\in C) & \capcupE{6} \\	
		6 & (8) & x \in A\lor (x \in B\land x\in C) & \POrLpQAndRRpEqvLpPOrQRpAndLpPOrRRp{7} \\
		6 & (9) & x \in A \cup (B \cap C) & \capcupI{8} \\
		& (10) & (A\cup B)\cap (A\cup C)\subseteq A \cup (B \cap C) & \subseteqIa{6,9} \\
		& (11) & A \cup (B \cap C) = (A\cup B)\cap (A\cup C) & \rIIaSet{5,10} \\
	\end{array}
	\]
\end{proof}

\label{LpAcaBRpcuCEqualsLpAcuCRpcaLpBcuCRp}
\begin{theorem}[\((A \cap B)\cup C = (A \cup C) \cap (B \cup C)\)]
\end{theorem}

\begin{proof}
	\[
	\begin{array}{llll}
		1 & (1) & x \in (A \cap B)\cup C & \rA \\
		1 & (2) & (x \in A\land x\in B)\lor x\in C  & \capcupE{1} \\
		1 & (3) & (x \in A\lor x \in C)\land (x\in B\lor x\in C) & \LpPAndQRpOrREqvLpPOrRRpAndLpQOrRRp{2} \\
		1 & (4) & x\in (A\cup C)\cap (B\cup C) & \capcupI{3} \\		
		& (5) & (A \cap B)\cup C\subseteq (A\cup C)\cap (B\cup C) & \subseteqIa{1,4} \\	
		6 & (6) & x\in (A\cup C)\cap (B\cup C) & \rA \\	
		6 & (7) & (x \in A\lor x \in C)\land (x\in B\lor x\in C) & \capcupE{6} \\	
		6 & (8) & (x \in A\land x\in B)\lor x\in C & \LpPAndQRpOrREqvLpPOrRRpAndLpQOrRRp{7} \\
		6 & (9) & x \in (A \cap B)\cup C & \capcupI{8} \\
		& (10) & (A\cup C)\cap (B\cup C)\subseteq (A \cap B)\cup C & \subseteqIa{6,9} \\
		& (11) & (A \cap B)\cup C = (A\cup C)\cap (B\cup C) & \rIIaSet{5,10} \\
	\end{array}
	\]
\end{proof}

\label{AcaLpBcuCRpEqualsLpAcaBRpcuLpAcaCRp}
\begin{theorem}[\(A \cap (B \cup C) = (A \cap B) \cup (A \cap C)\)]
\end{theorem}

\begin{proof}
	\[
	\begin{array}{llll}
		1 & (1) & x \in A \cap (B \cup C) & \rA \\
		1 & (2) & x \in A\land (x \in B\lor x\in C) & \capcupE{1} \\
		1 & (3) & (x \in A\land x \in B)\lor (x\in A\land x\in C) & \PAndLpQOrRRpEqvLpPAndQRpOrLpPAndRRp{2} \\
		1 & (4) & x\in (A\cap B)\cup (A\cap C) & \capcupI{3} \\		
		& (5) & A \cap (B \cup C)\subseteq (A\cap B)\cup (A\cap C) & \subseteqIa{1,4} \\	
		6 & (6) & x\in (A\cap B)\cup (A\cap C) & \rA \\	
		6 & (7) & (x \in A\land x \in B)\lor (x\in A\land x\in C) & \capcupE{6} \\	
		6 & (8) & x \in A\land (x \in B\lor x\in C) & \PAndLpQOrRRpEqvLpPAndQRpOrLpPAndRRp{7} \\
		6 & (9) & x \in A \cap (B \cup C) & \capcupI{8} \\
		& (10) & (A\cap B)\cup (A\cap C)\subseteq A \cap (B \cup C) & \subseteqIa{6,9} \\
		& (11) & A \cap (B \cup C) = (A\cap B)\cup (A\cap C) & \rIIaSet{5,10} \\
	\end{array}
	\]
\end{proof}

\label{LpAcuBRpcaCEqualsLpAcaCRpcuLpBcaCRp}
\begin{theorem}[\((A \cup B)\cap C = (A \cap C) \cup (B \cap C)\)]
\end{theorem}

\begin{proof}
	\[
	\begin{array}{llll}
		1 & (1) & x \in (A \cap B)\cup C & \rA \\
		1 & (2) & (x \in A\land x\in B)\lor x\in C  & \capcupE{1} \\
		1 & (3) & (x \in A\lor x \in C)\land (x\in B\lor x\in C) & \LpPOrQRpAndREqvLpPAndRRpOrLpQAndRRp{2} \\
		1 & (4) & x\in (A\cup C)\cap (B\cup C) & \capcupI{3} \\		
		& (5) & (A \cup B)\cap C\subseteq (A\cap C)\cup (B\cap C) & \subseteqIa{1,4} \\	
		6 & (6) & x\in (A\cap C)\cup (B\cap C) & \rA \\	
		6 & (7) & (x \in A\land x \in C)\lor (x\in B\land x\in C) & \capcupE{6} \\	
		6 & (8) & (x \in A\lor x\in B)\land x\in C & \LpPOrQRpAndREqvLpPAndRRpOrLpQAndRRp{7} \\
		6 & (9) & x \in (A \cup B)\cap C & \capcupI{8} \\
		& (10) & (A\cap C)\cup (B\cap C)\subseteq (A \cup B)\cap C & \subseteqIa{6,9} \\
		& (11) & (A \cup B)\cap C = (A\cap C)\cup (B\cap C) & \rIIaSet{5,10} \\
	\end{array}
	\]
\end{proof}

\subsection{Eigenschaften von Teilmengen in Bezug auf Vereinigung und Durchschnitt}
Die folgenden Regeln kürzen wir im folgenden mit \(R(\subseteq,\cap/\cup)\) oder \(R(\subseteq,\cup/\cap)\)ab.

\label{ASubseteqCwBSubseteqCImpAcuBSubseteqC}
\begin{theorem}[\(A \subseteq C, B \subseteq C \vdash A \cup B \subseteq C\)]
\end{theorem}
\begin{proof}
	\[
	\begin{array}{llll}
		1 & (1) & A \subseteq C & \rA \\
		2 & (2) & B \subseteq C & \rA \\
		3 & (3) & x \in A \cup B & \rA \\
		3 & (4) & x \in A \lor x \in B & \cupE{3} \\
		5 & (5) & x \in A & \rA \\
		1,5 & (6) & x \in C & \subseteqE{5,1}\\		
		7 & (7) & x \in B & \rA\\		
		2,7 & (8) & x \in C & \subseteqE{7,2}\\	
		1,2,3 & (9) & x \in C & \rOE{3,5,6,7,8}\\	
		1,2 & (10) & A\cup B\subseteq C & \subseteqIa{3.9}\\		
	\end{array}
	\]
\end{proof}

\label{ASubseteqBImpAcuCSubseteqBcuC}
\begin{theorem}[\(A \subseteq B \vdash A \cup C \subseteq B \cup C\)]
\end{theorem}
\begin{proof}
	\[
	\begin{array}{llll}
		1 & (1) & A \subseteq B & \rA \\
		2 & (2) & x \in A \cup C & \rA \\
		2 & (3) & x \in A \lor x \in C & \cupE{2}\\
		4 & (4) & x \in A & \rA \\
		1,4 & (5) & x\in B & \subseteqE{4,1}\\
		1,4 & (6) & x\in B\cup C & \cupI{5}\\		
		7 & (7) & x \in C & \rA \\
		7 & (8) & x \in B\cup C & \cupI{6}\\
		1,2 & (9) & x \in B\cup C & \rOE{3,4,6,7,8}\\
		1 & (10) & A\cup C\subseteq B\cup C & \subseteqIa{2,9}\\		
	\end{array}
	\]
\end{proof}

\label{ASubseteqBImpAcaCSubseteqBcaC}
\begin{theorem}[\(A \subseteq B \vdash A \cap C \subseteq B \cap C\)]
\end{theorem}
\begin{proof}
	\[
	\begin{array}{llll}
		1 & (1) & A \subseteq B & \rA \\
		2 & (2) & x \in A \cap C & \rA \\
		2 & (3) & x \in A & \capEa{2} \\
		2 & (4) & x \in C & \capEb{2} \\
		1,2 & (5) & x \in B & \subseteqE{3,1} \\		
		1,2 & (6) & x \in B\cap C & \capI{5,4} \\	
		1 & (7) & A\cap C\subseteq B\cap C & \subseteqIa{2,6} \\
	\end{array}
	\]
\end{proof}

\label{ASubseteqBwCSubseteqDImpAcuCSubseteqBcuD}
\begin{theorem}[\(A \subseteq B, C \subseteq D \vdash A \cup C \subseteq B \cup D\)]
\end{theorem}
\begin{proof}
	\[
	\begin{array}{llll}
		1 & (1) & A \subseteq B & \rA \\
		1 & (2) & C \subseteq D & \rA \\
		3 & (3) & x \in A \cup C & \rA \\
		3 & (4) & x \in A \lor x \in C & \cupE{3}\\		
		5 & (5) & x \in A & \rA \\
		1,5 & (6) & x \in B & \subseteqE{5,1} \\
		1,5 & (7) & x \in B\cup C & \cupI{6} \\	
		8 & (8) & x \in C & \rA \\			
		8 & (9) & x \in D & \subseteqE{8,2} \\	
		2,8 & (10) & x \in B\cup D & \cupI{9} \\	
		1,2,3 & (11) & x \in B\cup D & \rOE{4,5,7,8,10} \\	
		1,2 & (12) & A \cup C\subseteq B\cup D & \subseteqIa{3,11} \\			
	\end{array}
	\]
\end{proof}

\label{ASubseteqBwCSubseteqDImpAcaCSubseteqBcaD}
\begin{theorem}[\(A \subseteq B, C \subseteq D \vdash A \cap C \subseteq B \cap D\)]
\end{theorem}
\begin{proof}
	\[
	\begin{array}{llll}
		1 & (1) & A \subseteq B & \rA \\
		2 & (2) & C \subseteq D & \rA \\
		3 & (3) & x \in A \cap C & \rA \\
		3 & (4) & x \in A & \capE{3}\\
		3 & (5) & x \in C & \capE{3}\\
		1,3 & (6) & x \in B & \subseteqE{4,1}\\
		2,3 & (7) & x \in D & \subseteqE{5,2}\\
		1,2,3 & (8) & x\in B\cap D & \capI{6,7}\\	
		1,2 & (9) & A\cap C\subseteq B\cap D & \subseteqE{3,8}\\
	\end{array}
	\]
\end{proof}

\label{aInAwbInBImpLbawbRbSubseteqAcuB}
\begin{theorem}[\(a\in A, b\in B\vdash \{a,b\}\subseteq A\cup B\)]
\end{theorem}	
\begin{proof}
	\[
	\begin{array}{llll}
		1 & (1) & a\in A & \rA \\
		2 & (2) & b\in B & \rA \\
		1 & (3) & \{a\}\subseteq A & \aInAImpLbaRbSubseteqA{1} \\
		2 & (4) & \{b\}\subseteq B & \aInAImpLbaRbSubseteqA{2} \\
		1,2 & (5) & \{a\}\cup \{b\}\subseteq A\cup B & \ASubseteqBwCSubseteqDImpAcuCSubseteqBcuD{3,4} \\
		& (6) & \{a,b\} = \{a\}\cup \{b\}  & \LbawbRbEqualsLbaRbcuLbbRb{} \\
		& (7) & \{a,b\}\subseteq A\cup B & \rIE{6,5} \\
	\end{array}
	\]
\end{proof}


\section{Das Axiom der Potenzmenge}

Das Axiom der Potenzmenge ist ein weiteres grundlegendes Axiom in der Zermelo-Fraenkel Mengenlehre. Es ermöglicht die Konstruktion einer Menge, die alle Teilmengen einer gegebenen Menge enthält. Formal ausgedrückt:

\label{FaAExBFaxLpxInBLrxSubseteqARp}
\[
\forall A \exists B \forall x (x \in B \leftrightarrow x \subseteq A)
\]

Dieses Axiom garantiert, dass für jede Menge \( A \), es eine Menge \( B \) gibt, die alle Teilmengen von \( A \) enthält. Die Menge \( B \) heißt Potenzmenge von \( A \).

\subsection{Eigenschaften der Potenzmenge}

Die Potenzmenge \( B \) enthält alle Teilmengen der Menge \( A \).

\label{FaxLpxInBLrxSubseteqARpAndFaxLpxInCLrxSubseteqARpImpBEqualsC}
\begin{theorem}[\(\forall x (x \in B \leftrightarrow x \subseteq A) \land \forall x (x \in C \leftrightarrow x \subseteq A) \vdash B = C\) (Eindeutigkeit der Potenzmenge)]
	Seien \(A, B\) und \(C\) Mengen, wobei $B$ und $C$ Potenzmengen von A sind. Unter Verwendung des Extensionalitätsaxioms gilt \(B = C\).
\end{theorem}
\begin{tempdefinition}
    \[\forall x(Q(x):=x\subseteq A)\]
\end{tempdefinition}
\begin{proof}
	\[
	\begin{array}{lll p{4cm}}
		1 & (1) & \forall x (x \in B \leftrightarrow x \subseteq A) \land \forall x (x \in C \leftrightarrow x \subseteq A) & \rA \\
		1 & (2) & \forall x (x \in B \leftrightarrow Q(x)) \land \forall x (x \in C \leftrightarrow Q(x)) & \rIE{df(Q), 1} \\
		1 & (3) & B=C &  \FaxLpxInBLrPLpxRpRpAndFaxLpxInCLrPLpxRpRpImpBEqualsC{2}\\
	\end{array}
	\]
\end{proof}

\begin{remark}
	Das Axiom der Potenzmenge ist ein Schlüsselkonzept in der Mengenlehre und ermöglicht die Konstruktion komplexerer Mengenstrukturen. Es dient als Grundlage für die Definition von Potenzmengen und ist ein wichtiger Baustein für die Entwicklung der Theorie der Kardinalität, Ordinalität und viele andere Konzepte in der Mengenlehre. Die eindeutige Potenzmenge einer Menge \( A \) wird oft als \( \mathcal{P}(A) \) bezeichnet. Diese Notation stellt sicher, dass die Potenzmenge eindeutig ist, da sie alle Teilmengen der Menge \( A \) enthält und durch das Axiom der Potenzmenge und das Extensionalitätsaxiom eindeutig bestimmt ist.
\end{remark}

\subsubsection{Regeln für die Potenzmenge}
\label{rule:powersetI} \label{rule:powersetE}

% Regel für die Einführung der Potenzmenge
Die Einführungsregel für die Potenzmenge (\( \mathcal{P}I \)) besagt, dass, wenn eine Menge \( x \) eine Teilmenge der Menge \( A \) ist, \( x \) auch in der Potenzmenge \( \mathcal{P}(A) \) enthalten ist.
\[
\begin{array}{llll}
	i & (1) & x \subseteq A & ... \\
	i & (2) & x \in \mathcal{P}(A) & \powersetI{1} \\
\end{array}
\]

% Regel für die Eliminierung der Potenzmenge
Die Eliminierungsregel für die Potenzmenge (\( \mathcal{P}E \)) ermöglicht es, aus der Existenz einer Menge \( x \) in der Potenzmenge \( \mathcal{P}(A) \) zu schließen, dass \( x \) eine Teilmenge von \( A \) ist.
\[
\begin{array}{llll}
	i & (1) & x \in \mathcal{P}(A) & ... \\
	i & (2) & x \subseteq A & \powersetE{1} \\
\end{array}
\]

\(i\) ist dabei eine Liste von Annahmen.

\paragraph{Existenzregel der Potezmenge}
\label{rule:powerSetExists}
Die Existenzregel stellt sicher, dass für alle Mengen \(A\) die Potenzmenge \(\mathcal{P}(A)\) ebenfalls eine Menge ist.

\[
\begin{array}{llll}
	i & (1) & A \text{ ist eine Menge} & \dots  \\
	i & (2) & \mathcal{P}(A)\text{ ist eine Menge} & \powerSetExists{1} \\
\end{array}
\]

\(i\) sind dabei Listen von Annahmen.

\label{ASubseteqBImpPowersetLpARpSubseteqPowersetLpBRp}
\begin{theorem}[\(A \subseteq B \vdash \mathcal{P}(A) \subseteq \mathcal{P}(B)\)]
\end{theorem}
\begin{proof}
	\[
	\begin{array}{llll}
		1 & (1) & A \subseteq B & \rA \\
		2 & (2) & x\in\mathcal{P}(A) & \rA \\
		2 & (3) & x\subseteq A & \powersetE{2} \\
		1,2 & (4) & x\subseteq B & \ASubseteqBwBSubseteqCImpASubseteqC{3,1} \\
		1,2 & (5) & x\in \mathcal{P}(B) & \powersetI{4} \\
		1 & (6) & \mathcal{P}(A)\subseteq \mathcal{P}(B) & \subseteqIa{2,5} \\
	\end{array}
	\]
\end{proof}

\label{aInAImpLbaRbInPowersetLpARp}
\begin{theorem}[\(a\in A \vdash \{a\}\in\mathcal{P}(A)\)]
\end{theorem}

\begin{proof}
	\[
	\begin{array}{llll}
		1 & (1) & a \in A & \rA \\
		1 & (2) & \{a\}\subseteq A & \aInAImpLbaRbSubseteqA{1} \\
		1 & (3) & a\in\mathcal{P}(A) & \powersetI{2} \\
	\end{array}
	\]
\end{proof}

\label{aInPowersetLpARpImpFaBLpaInPowersetLpAcuBRpRp}
\begin{theorem}[\(a\in \mathcal{P}(A) \vdash \forall B(a\in\mathcal{P}(A\cup B))\)]
\end{theorem}
\begin{proof}
	\[
	\begin{array}{llll}
		1 & (1) & a\in \mathcal{P}(A) & \rA \\
		1 & (2) & a\subseteq A & \powersetE{1} \\
		1 & (2) & \{a\}\subseteq A & \aInAImpLbaRbSubseteqA{1} \\
		1 & (3) & a\in\mathcal{P}(A) & \powersetI{2} \\
	\end{array}
	\]
\end{proof}


\section{Das kartesische Produkt}
\begin{definition}[Geordnetes Paar]
	Für alle \( a \) und \( b \) definieren wir das geordnete Paar \( (a,b) \), als folgende Paarmenge:
	\[
	(a, b) := \{ \{ a \}, \{ a, b \} \}
	\]
\end{definition}
\begin{remark}
	Ein geordnetes Paar ist eine Paarmenge und damit eindeutig bestimmt.
\end{remark}

\subsubsection{Regeln für geordnete Paare}
\label{rule:orderedPairI} \label{rule:orderedPairE}

% Regel für die Einführung des geordneten Paares
Die Einführungsregel für das geordnete Paar (\( \mathcal{O}I \)) besagt, dass aus zwei gegebenen Elemente \( a \) und \( b \) und der Aussage \(P(\{ \{ a \}, \{ a, b \} \})\) die Aussage \(P((a,b))\) abgeleitet werden kann.
\[
\begin{array}{llll}
	i & (1) & P(\{x,\{x,y\}\}) & ... \\
	i & (2) & P((x,y)) & \orderedPairI{1} \\
\end{array}
\]

% Regel für die Eliminierung des geordneten Paares
Die Einführungsregel für das geordnete Paar (\( \mathcal{O}I \)) besagt, dass aus zwei gegebenen Elemente \( a \) und \( b \) und der Aussage \(P((a,b))\) die Aussage \(P(\{ \{ a \}, \{ a, b \} \})\) abgeleitet werden kann.
\[
\begin{array}{llll}
	i & (1) & P((x,y)) & ... \\
	i & (2) & P(\{x,\{x,y\}\}) & \orderedPairE{1} \\
\end{array}
\]

\(i\) ist dabei eine Liste von Annahmen.

\paragraph{Existenzregel des karthesischen Produktes}
\label{rule:cartesianSetExists}
Die Existenzregel stellt sicher, dass für alle Mengen \(A\) und \(B\) das karthesische Produkt \(A\times B\) ebenfalls eine Menge ist.

\[
\begin{array}{llll}
	i & (1) & A \text{ ist eine Menge} & \dots  \\
        j & (2) & B \text{ ist eine Menge} & \dots  \\
	i,j & (3) & A\times B\text{ ist eine Menge} & \cartesianSetExists{1} \\
\end{array}
\]

\(i\) und \(j\) sind dabei Listen von Annahmen.


\label{aInAwbInBImpLpawbRpInPowersetLpPowersetLpAcuBRpRp}
\begin{theorem}[\(a\in A, b\in B \vdash (a,b)\in \mathcal{P}(\mathcal{P}(A \cup B))\)]
    Seien \( a \) und \( b \) Elemente von Mengen \( A \) und \( B \) respektive. Dann ist das geordnete Paar \( (a, b) \) ein Element der Menge \(\mathcal{P}(\mathcal{P}(A \cup B))\). 
\end{theorem}
\begin{proof}
	\[
	\begin{array}{llll}
		1 & (1) & a\in A & \rA \\
		2 & (2) & b\in B & \rA \\
		1,2 & (3) & \{a,b\}\subseteq A\cup B & \aInAwbInBImpLbawbRbSubseteqAcuB{1,2} \\
		1,2 & (4) & \{a,b\}\in \mathcal{P}(A\cup B) & \powersetI{3} \\
		1 & (5) & a\in A\cup B & \cupI{1} \\
		1 & (6) & \{a\}\in \mathcal{P}(A\cup B) & \powersetI{5} \\
		1,2 & (7) & \{\{a\},\{a,b\}\}\subseteq \mathcal{P}(A\cup B) & \aInAwbInBImpLbawbRbSubseteqAcuB{6,4} \\
		1,2 & (8) & \{\{a\},\{a,b\}\}\in \mathcal{P}(\mathcal{P}(A\cup B)) & \powersetI{7} \\
		1,2 & (9) & (a,b)\in \mathcal{P}(\mathcal{P}(A\cup B)) & \orderedPairI{8} \\
	\end{array}
	\]
\end{proof}

\label{ExCFaxLpxInCLrLpExaInAExbInBLpxEqualsLpawbRpRpRpRp}
\begin{theorem}[\(\exists C\forall x (x \in C \leftrightarrow (\exists a \in A \exists b \in B (x = (a,b))))\) (Existenz des kartesischen Produkts)]
	Seien \( A \) und \( B \) Mengen. Dann existiert eine Menge \( C \) so, dass für alle Elemente \( x \) gilt:
	\[
	x \in C \leftrightarrow (\exists a \in A \exists b \in B (x = (a,b)))
	\]
\end{theorem}
\begin{proof}
	\[
	\begin{array}{lll p{3cm}}
		1 & (1) & \exists a\in A\exists b\in B(x=(a,b)) & \rA \\
		1 & (2) & \exists a (a\in A\land \exists b\in B(x=(a,b))) & \rSetEEa{1} \\
		3 & (3) & m\in A\land \exists b\in B(x=(m,b)) & \rA \\
		3 & (4) & m\in A & \rAEa{3} \\
		3 & (5) & \exists b\in B(x=(m,b) & \rAEb{3} \\
		3 & (6) & \exists b(b\in B\land x=(m,b)) & \rSetEEa{5} \\
		7 & (7) & n\in B\land x=(m,n) & \rA \\
		7 & (8) & n\in B & \rAEa{7} \\
		7 & (9) & x=(m,n) & \rAEb{7} \\
		3,7 & (10) & (m,n)\in \mathcal{P}(\mathcal{P}(A\cup B)) & \aInAwbInBImpLpawbRpInPowersetLpPowersetLpAcuBRpRp{4,8} \\
		3,7 & (11) & x\in \mathcal{P}(\mathcal{P}(A\cup B)) & \rIE{9,10} \\
		3 & (12) & x\in \mathcal{P}(\mathcal{P}(A\cup B)) & \rEE{6,7,11} \\
		1 & (13) & x\in \mathcal{P}(\mathcal{P}(A\cup B)) & \rEE{2,3,12} \\
		& (14) & \exists a\in A\exists b\in B(x=(a,b))\rightarrow x\in \mathcal{P}(\mathcal{P}(A\cup B)) & \rRI{1,13} \\
		& (15) & \forall x(\exists a\in A\exists b\in B(x=(a,b))\rightarrow x\in \mathcal{P}(\mathcal{P}(A\cup B))) & \rUI{14} \\
		& (16) & \exists D(\forall x(\exists a\in A\exists b\in B(x=(a,b))\rightarrow x\in D))) & \rEI{15} \\
		& (17) & 	\exists C\forall x (x \in C \leftrightarrow (\exists a \in A \exists b \in B (x = (a,b))))
		& \ExALpFaxLpPLpxRpToxInARpRpImpExBFaxLpxInBLrPLpxRpRp{16} \\
	\end{array}
	\]
\end{proof}

\label{FaxLpxInELrLpExaInAExbInBLpxEqualsLpawbRpRpRpRpAndFaxLpxInFLrLpExaInAExbInBLpxEqualsLpawbRpRpRpRpImpEEqualsF}
\begin{theorem}[Eindeutigkeit des kartesischen Produkts]
	Seien \( E \) und \( F \) zwei Mengen, die beide das kartesische Produkt von \( A \) und \( B \) darstellen. Dann sind diese unter Verwendung des Extensionalitätsaxioms gleich, das heißt:
 
	\(
	\forall x (x \in E \leftrightarrow (\exists a \in A \exists b \in B (x = (a,b)))) \land \forall x (x \in F \leftrightarrow (\exists a \in A \exists b \in B (x = (a,b)))) \vdash E = F 
	\)
\end{theorem}
\begin{tempdefinition}
    \[\forall x(Q(x):= \exists a \in A \exists b \in B (x = (a,b)))\]
\end{tempdefinition}
\begin{proof}
    \[
    \begin{array}{ll p{6cm} p{4cm}}
        1 & (1) & \ensuremath{\forall x (x \in E \leftrightarrow (\exists a \in A \exists b \in B (x = (a,b)))) \land \forall x (x \in F \leftrightarrow (\exists a \in A \exists b \in B (x = (a,b))))} &  \rA \\
        1 & (2) & \ensuremath{\forall x (x \in E \leftrightarrow Q(x)) \land \forall x (x \in C \leftrightarrow Q(x))} & \rIE{df(Q), 1} \\
        1 & (3) & \ensuremath{E=F} &  \FaxLpxInBLrPLpxRpRpAndFaxLpxInCLrPLpxRpRpImpBEqualsC{3} \\
    \end{array}
    \]
\end{proof}

\begin{definition}[Kartesisches Produkt ($A\times B$)]
	Das kartesische Produkt von zwei Mengen \( A \) und \( B \) ist die Menge aller geordneten Paare \( (a, b) \), wobei \( a \) ein Element von \( A \) und \( b \) ein Element von \( B \) ist. Dies wird als \( A \times B \) bezeichnet und kann als eine spezielle Anwendung des Axioms der Potenzmengen formuliert werden:
	\[
	A \times B = \{ x \mid \exists a \in A \exists b \in B (x = (a, b)) \}
	\]
\end{definition}
\begin{remark}
	Das kartesische Produkt ist gemäß des vorherigen Theorems eine Menge.
\end{remark}

\subsubsection*{Einführungsregel für das kartesische Produkt ($\times I$)}
\label{rule:timesI}
Die Einführungsregel für das kartesische Produkt ermöglicht es, aus der Zugehörigkeit zu den Mengen $A$ und $B$ die Zugehörigkeit eines geordneten Paares zum kartesischen Produkt $A \times B$ abzuleiten.
\[
\begin{array}{llll}
	i & (1) & a \in A & ... \\
	j & (2) & b \in B & ... \\
	i,j & (3) & (a, b) \in A \times B & \timesI{1,2} \\
\end{array}
\]

\(i,j\) sind dabei Listen von Annahmen.

\subsubsection*{Eliminierungsregel für das kartesische Produkt ($\times E$)}
\label{rule:timesEa}\label{rule:timesEb}
Die Eliminierungsregel für das kartesische Produkt ermöglicht es, aus der Zugehörigkeit eines geordneten Paares zum kartesischen Produkt $A \times B$ die Zugehörigkeit der Elemente zu ihren jeweiligen Mengen abzuleiten.
\[
\begin{array}{llll}
	i & (1) & (a, b) \in A \times B & ... \\
	i & (2) & a \in A & \timesEa{1} \\
	i & (3) & b \in B & \timesEb{1} \\
\end{array}
\]
\(i,j\) sind dabei Listen von Annahmen.

\section{Das Ersetzungsaxiom}

Das Ersetzungsaxiom ist ein fundamentales Axiom der Zermelo-Fraenkel-Mengenlehre, das die Bildung einer Menge ermöglicht, indem auf die Elemente einer gegebenen Menge eine Funktion angewendet wird. Es sichert die Existenz einer Menge, die genau die Bilder der Elemente der ursprünglichen Menge unter einer Funktion enthält. Formal lässt sich das Axiom wie folgt ausdrücken:

\label{Ersetzungsaxiom}
\[
\forall A \, \forall F \, \exists B \, \forall y \, \left( y \in B \leftrightarrow \exists x \, (x \in A \land y = F(x)) \right)
\]

Das Ersetzungsaxiom garantiert, dass für jede Menge \( A \) und jede Funktion \( F \), die eine Abbildung auf den Elementen von \( A \) definiert, eine Menge \( B \) existiert, die alle Bilder der Elemente von \( A \) unter \( F \) enthält.

\subsection{Das Axiom der Ersetzung}

Durch das Ersetzungsaxiom lässt sich eine Menge \( B \) konstruieren, die die Bilder aller Elemente einer gegebenen Menge \( A \) unter der Abbildung \( F \) umfasst.

\label{Ersetzungsaxiom_Eindeutigkeit}
\begin{theorem}[\(\forall y (y \in B \leftrightarrow \exists x \, (x \in A \land y = F(x))) \land \newline \forall y (y \in C \leftrightarrow \exists x \, (x \in A \land y = F(x))) \vdash B = C\) (Eindeutigkeit der Menge der Bilder)]
	Seien \(A\), \(B\) und \(C\) Mengen und \(F\) eine Funktion, die jedes Element \(x \in A\) eindeutig auf ein Bild \(y = F(x)\) abbildet. Unter Verwendung des Extensionalitätsaxioms gilt \(B = C\).
\end{theorem}
\begin{tempdefinition}
    \[\forall y (Q(y) := \exists x \, (x \in A \land y = F(x)))\]
\end{tempdefinition}
\begin{proof}
	\[
	\begin{array}{ll p{6.5cm} p{4cm}}
		1 & (1) & \ensuremath{\forall y (y \in B \leftrightarrow \exists x \, (x \in A \land y = F(x))) \land \forall y (y \in C \leftrightarrow \exists x \, (x \in A \land y = F(x)))} & \rA \\
		1 & (2) & \ensuremath{\forall y (y \in B \leftrightarrow Q(y)) \land \forall y (y \in C \leftrightarrow Q(y))} & \rIE{df(Q), 1} \\
		1 & (3) & \ensuremath{B = C} &  \FaxLpxInBLrPLpxRpRpAndFaxLpxInCLrPLpxRpRpImpBEqualsC{2} \\
	\end{array}
	\]
\end{proof}

\begin{remark}
	Das Ersetzungsaxiom erlaubt es, neue Mengen durch die Anwendung einer Abbildung auf bestehende Mengen zu bilden. Die dabei entstehende Menge \( B \), die alle Bilder der Elemente von \( A \) unter \( F \) umfasst, wird als \textbf{Bildmenge} oder \textbf{Ersetzungsklasse} bezeichnet. Häufig wird sie mit \( F(A) \), \( \operatorname{Im}(F) \) oder \(\{F(x)\mid x\in A\}\)  notiert. Dieses Konzept ist ein zentraler Bestandteil der Mengenlehre und besonders nützlich bei der Konstruktion von Mengen in komplexeren Kontexten, wie der Definition von Ordinalzahlen, Kardinalzahlen und anderen Konzepten, die auf der Idee beruhen, eine Menge durch eine Vorschrift oder Funktion zu transformieren.
\end{remark}

\subsubsection{Regeln für das Ersetzungsaxiom}
\label{rule:replacementI} \label{rule:replacementE}

% Einführungsregel für das Ersetzungsaxiom
Die Einführungsregel für das Ersetzungsaxiom (\(\mathcal{R}I\)) besagt, dass, wenn eine Funktion \( F \) auf die Elemente einer Menge \( A \) angewendet wird, eine Menge \( B \) existiert, die genau die Werte \( F(x) \) für jedes \( x \in A \) enthält.
\[
\begin{array}{llll}
	i & (1) & y\in \{F(x)\mid x\in A\} & ... \\
	i & (2) & \exists x\in A(y = F(x)) & \replacementE{1} \\
\end{array}
\]

% Eliminierungsregel für das Ersetzungsaxiom
Die Eliminierungsregel für das Ersetzungsaxiom (\(\mathcal{R}E\)) ermöglicht es, aus der Existenz einer Menge \( B \), die die Bilder von \( A \) unter \( F \) enthält, auf die Beziehung zwischen den Elementen von \( A \) und ihren Bildern zu schließen.
\[
\begin{array}{llll}
	i & (1) & y=F(x) & ... \\
        i,j & (2) & x\in A & ... \\
	i & (2) & y\in \{F(n)\mid n\in A\} & \replacementI{1,2} \\
\end{array}
\]

\(i\) und \(j\) sind dabei Listen von Annahmen.

\label{xInBLrExiInILpxInASubiRp}
\begin{theorem}[Existenz einer Menge, die alle Elemente indizierter Mengen enthält]
    Sei \( I \) eine Menge und \( A_i \) für jedes \( i \in I \) eine Menge, die durch ein Element \( i \) aus der Indexmenge \( I \) indiziert ist. Dann existiert eine Menge \( B \) mit der Eigenschaft, dass für jedes Element \( x \) gilt:
    \[
    x \in B \leftrightarrow \exists i \in I(x \in A_i).
    \]
\end{theorem}

\begin{tempdefinition}
    \[
    \forall i \in I \, (F(i) := A_i)
    \]
    \[
    B := \bigcup \{ A_i \mid i \in I \}
    \]
\end{tempdefinition}

\begin{proof}
    Wir wissen bereits, dass \(B\) eine Menge ist. Wir zeigen nun noch, dass das so gewählte B die im Theorem aufgestellte Eigenschaft erfüllt.

    
    \(\vdash\):
    \[
	\begin{array}{llll}
		1 &  (1) & x\in B & \rA \\
		1 &  (2) & x\in\bigcup\{A_n\mid n\in I\} & \rIE{df(B), 1} \\			
		1 &  (3) & \exists C(C\in \{A_n\mid n\in I\}\land x\in C)  & \bigcupE{2} \\
		4 &  (4) & C\in \{A_n\mid n\in I\}\land x\in C & \rA \\	
  	4 &  (5) & C\in \{A_n\mid n\in I\} & \rAEa{4} \\	
        4 &  (6) & x\in C & \rAEb{4} \\
        4 &  (7) & \exists j\in I(C = F(j)) & \replacementE{5} \\
        8 &  (8) & j\in I\land C = F(j) & \rA \\
        8 &  (9) & j\in I & \rAEa{8} \\
        8 &  (10) & C=F(j) & \rAEb{8} \\
        8 &  (11) & A_j = F(j) & \rIE{df(F), 10} \\
        8 &  (12) & A_j=C & \rIE{10,11} \\
        4,8 &  (13) & x\in A_j & \rIE{12,6} \\
        4,8 &  (14) & \exists i\in I(x\in A_i) & \rSetEIa{9,13} \\
        4 &  (15) & \exists i\in I(x\in A_i) & \rEE{7,8,14} \\
        1 &  (16) & \exists i\in I(x\in A_i) & \rEE{3,4,15} \\
	\end{array}
    \]

    \(\dashv\):
    \[
	\begin{array}{llll}
		1 &  (1) & \exists i\in I(x\in A_i) & \rA \\
		2 &  (2) & i\in I\land x\in A_i & \rA \\	
        2 &  (3) & i\in I & \rAEa{2} \\
        2 &  (4) & x\in A_i & \rAEb{2} \\
        2 &  (5) & F(i)=A_i & \rIE{df(F), 4} \\	
        2 &  (6) & A_i\in\{A_n\mid n\in I\} & \replacementI{5,3} \\	
        2 &  (7) & x\in\bigcup \{A_n\mid n\in I\} & \bigcupI{3,5} \\	
        2 &  (8) & x\in B & \rIE{df(B), 7} \\	
	\end{array}
    \]
\end{proof}

\label{FaxLpxInBLrExiInILpxInASubiRpRpAndFaxLpxInCLrExiInILpxInASubiRpRpImpBEqualsC}
\begin{theorem}[\(\forall x (x \in B \leftrightarrow \exists i \in I (x \in A_i)) \land \forall x (x \in C \leftrightarrow \exists i \in I (x \in A_i)) \vdash B = C\) (Eindeutigkeit der Menge, die alle indizierten Mengen umfasst)]
    Seien \( I \) eine Menge, \( B \) und \( C \) Mengen, und sei \( A_i \) für jedes \( i \in I \) eine Menge, die durch ein Element \( i \) aus \( I \) indiziert ist. Dann gilt:
    \[
    \forall x (x \in B \leftrightarrow \exists i \in I \, (x \in A_i)) \land \forall x (x \in C \leftrightarrow \exists i \in I \, (x \in A_i)) \vdash B = C
    \]
\end{theorem}

\begin{tempdefinition}
    \[
    \forall x \, (Q(x) := \exists i \in I \, (x \in A_i))
    \]
\end{tempdefinition}

\begin{proof}
    \[
    \begin{array}{ll p{5cm} p{4cm}}
        1 & (1) & \ensuremath{\forall x (x \in B \leftrightarrow \exists i \in I \, (x \in A_i)) \land \forall x (x \in C \leftrightarrow \exists i \in I \, (x \in A_i))} & \rA \\
        1 & (2) & \ensuremath{\forall x (x \in B \leftrightarrow Q(x)) \land \forall x (x \in C \leftrightarrow Q(x))} & \rIE{df(Q), 1} \\
        1 & (3) & \ensuremath{B = C} & \FaxLpxInBLrPLpxRpRpAndFaxLpxInCLrPLpxRpRpImpBEqualsC{2} \\
    \end{array}
    \]
\end{proof}

\begin{definition}[Vereinigung über eine Indexmenge]
    Sei \( I \) eine Menge und \( A_i \) für jedes \( i \in I \) eine Menge, die durch das Element \( i \) aus \( I \) indiziert ist. Die Vereinigung der Mengen \( A_i \) über alle \( i \in I \) ist definiert als die Menge
    \[
    \bigcup_{i \in I} A_i := \{ x \mid \exists i \in I \, (x \in A_i) \}.
    \]
\end{definition}

\subsubsection{Regeln für die Vereinigung}
\label{rule:bigunionI} \label{rule:bigunionE}

% Einführungsregel für die Vereinigung
Die Einführungsregel für die Vereinigung (\(\bigcup I\)) besagt, dass, wenn ein Element \( x \) zu einer der Mengen \( A_i \) gehört, wobei \( i \in I \), dann gehört \( x \) auch zur Vereinigung \(\bigcup_{i \in I} A_i\).
\[
\begin{array}{llll}
    i & (1) & x \in A_i & ... \\
    i & (2) & i \in I & ... \\
    i & (3) & x \in \bigcup_{i \in I} A_i & \bigunionI{1,2} \\
\end{array}
\]

% Eliminierungsregel für die Vereinigung
Die Eliminierungsregel für die Vereinigung (\(\bigcup E\)) besagt, dass, wenn ein Element \( x \) in der Vereinigung \(\bigcup_{i \in I} A_i\) liegt, es ein \( i \in I \) gibt, sodass \( x \in A_i \) ist.
\[
\begin{array}{llll}
    i & (1) & x \in \bigcup_{i \in I} A_i & ... \\
    i & (2) & \exists i \in I \, (x \in A_i) & \bigunionE{1} \\
\end{array}
\]

\(i\) ist dabei die Liste der Annahmen.

\section{Das Axiom der Unendlichkeit}

Das Axiom der Unendlichkeit (\(Inf.\)) ist ein weiteres grundlegendes Axiom in der Zermelo-Fraenkel Mengenlehre. Es garantiert die Existenz mindestens einer unendlichen Menge und ist somit die Grundlage für die Entwicklung der Theorie der natürlichen Zahlen in der Mengenlehre. Formal ausgedrückt:

\label{ExALpEmptysetInAAndFaxInALpxcuLbxRbInARpRp}
\[
\exists A (\emptyset \in A \land \forall x\in A (x \cup \{x\} \in A))
\]

Dieses Axiom garantiert, dass es mindestens eine Menge \( A \) gibt, die das leere Set enthält und für jedes Element \( x \) in \( A \), das Set \( x \cup \{x\} \) ebenfalls in \( A \) ist.

\section{Das Axiom der Regularität}
\label{rule:Regularity}
Das Axiom der Regularität (Reg.), auch als Axiom der Fundierung bekannt, ist ein weiteres grundlegendes Axiom in der Zermelo-Fraenkel Mengenlehre. Es wird verwendet, um das Induktionsprinzip zu beweisen. Formal ausgedrückt:

\label{FaALpANotEqualsEmptysetToExxInALpxcaAEqualsEmptysetRpRp}
\[
\forall A (A \neq \emptyset \rightarrow \exists x\in A (x \cap A = \emptyset))
\]

\subsubsection{Regel der Regularität}
\label{rule:RegI}

% Regel für die Einführung der Regularität
Wir führen im Folgenden die Regel der Regularität ein. Diese Regel ermöglicht es, aus der Annahme, dass eine Menge \(A\) nicht leer ist (\(A \neq \emptyset\)), die Existenz eines Elements \(x\) in \(A\) zu schlussfolgern, für das gilt, dass \(x\) und \(A\) disjunkt sind (\(x \cap A = \emptyset\)):

\[
\begin{array}{llll}
	i & (1) & A \neq \emptyset & \rA \\
	i & (2) & \exists x \in A (x \cap A = \emptyset) & \RegI{1} \\
\end{array}
\]

\(i\) ist dabei eine Liste von Annahmen.

\label{aInbImpbNotina}
\begin{theorem}[\(a\in b \vdash b\notin a\)]
\end{theorem}
\begin{proof}
	\[
	\begin{array}{llll}
		1 &  (1) & a\in b & \rA \\
		2 &  (2) & b\in a & \rA \\
		&  (3) & \{a,b\}\neq\emptyset & \ImpLbawbRbNotEqualsEmptyset{} \\
		&  (4) & \exists x\in\{a,b\}(x\cap \{a,b\} = \emptyset) & \RegI{3} \\
		&  (5) & c\in\{a,b\}\land c\cap \{a,b\} = \emptyset & \rEE{4} \\
		&  (6) & c\in\{a,b\} & \rAEa{5} \\
		&  (7) & c\cap \{a,b\} = \emptyset & \rAEb{5} \\		
		&  (8) & c=a\lor c=b & \pairE{6}  \\
		9 & (9) & c=a & \rA \\
		9 & (10) & a\cap \{a,b\} = \emptyset & \rIE{9,7} \\
		& (11) & a\in \{a,b\}  &  \pairIb \\
		& (12) & b\in \{a,b\}  &  \pairIb \\
		2 & (13) & b\in a\land b\in \{a,b\}  &  \rAI{2,11} \\
		2 & (14) & b\in a\cap \{a,b\}  &  \capI{13} \\
		2 & (15) & \exists x(x\in a\cap \{a,b\})  &  \rEI{14} \\
		2 & (16) & a\cap \{a,b\}\neq\emptyset  &  \ExxInSImpSNotEqualsEmptyset{15} \\
		2,9 & (17) & \bot  &  \rBI{10,16} \\
		9 & (18) & b\notin a  &  \rCI{2,17} \\
		9 & (19) & a\notin b\lor b\notin a  &  \rOIa{18} \\		
		20 & (20) & c=b & \rA \\
		20 & (21) & b\cap \{a,b\} = \emptyset & \rIE{20,7} \\
		1 & (22) & a\in b\land a\in \{a,b\}  &  \rAI{1,12} \\
		1 & (23) & a\in b\cap \{a,b\}  &  \capI{22} \\
		1 & (24) & \exists x(x\in b\cap \{a,b\})  &  \rEI{23} \\
		1 & (25) & b\cap \{a,b\}\neq\emptyset  &  \ExxInSImpSNotEqualsEmptyset{24} \\
		1,20 & (26) & \bot  &  \rBI{21,25} \\
		20 & (27) & a\notin b  &  \rCI{1,26} \\
		20 & (28) & a\notin b\lor b\notin a  &  \rOIb{27} \\
		& (29) & a\notin b\lor b\notin a  &  \rOIb{27} \\
		& (30) & a\in b\rightarrow b\notin a  & \PToQEqvnPOrQ{29} \\
		1 & (31) & b\notin a  & \rRE{1,29} \\			
	\end{array}
	\]
\end{proof}

\chapter{Die Menge der natürlichen Zahlen}

\begin{definition}[Existenz der natürlichen Zahlen in ZF ($Df.\mathbb{N}$)]
	Unter Verwendung des Axioms der Unendlichkeit und des Aussonderungsaxioms kann die Menge der natürlichen Zahlen \( \mathbb{N} \) in der ZF-Mengenlehre definiert werden als:
	\[
	\mathbb{N} := \bigcap \{A\mid \emptyset \in A\land \forall x\in A(x\cup \{x\}\in A) \}
	\]
\end{definition}

\subsubsection{Regeln für die Zugehörigkeit zu den natürlichen Zahlen}
\label{rule:NaturalI} \label{rule:NaturalE}

Basierend auf unserer Definition für die natürlichen Zahlen können wir zwei grundlegende Regeln formulieren: die Einführungs- und die Eliminierungsregel. Diese basieren auf den Regeln für den unendlichen Schnitt (siehe z.B. $\bigcapI{...}$).

Die Einführungsregel für die natürlichen Zahlen (\( \NaturalI{...}\)) besagt, dass wenn wir zeigen können, dass ein Element \(x\) in jeder Menge enthalten ist, die das Axiom der Unendlichkeit erfüllt, dann ist \(x\) ein Element der natürlichen Zahlen.

\[
\begin{array}{llll}
	i & (1) & \forall A(\emptyset\in A\land \forall y\in A(y\cup \{y\}\in A)\rightarrow x\in A) & ... \\
	i & (2) & x\in \mathbb{N} & \NaturalI{1} \\
\end{array}
\]

Die Eliminierungsregel für den unendlichen Schnitt (\( \NaturalE{...}\)) besagt, dass wenn wir wissen, dass ein Element \(x\) ein Element von $\mathbb{N}$ ist, dann diese Eigenschaft \(P\) erfüllt, dann ist \(x\) auch in \(B\) enthalten.

\[
\begin{array}{llll}
	i & (1) & x\in \mathbb{N} & ... \\
	i & (2) & \forall A(\emptyset\in A\land \forall x\in A(x\cup \{x\}\in A)\rightarrow x\in A) & \NaturalE{1} \\
\end{array}
\]

\(i\) sind dabei Listen von Annahmen.

\subsubsection{Regeln für die Nichtzugehörigkeit zu den natürlichen Zahlen}
\label{rule:NotNaturalI} \label{rule:NotNaturalE}

Um das Konzept der natürlichen Zahlen weiter zu verfeinern, führen wir Regeln für den Umgang mit der Negation, d.h., der Nichtzugehörigkeit zu den natürlichen Zahlen, ein. Diese Regeln sind essentiell für Beweise, die die Nichtzugehörigkeit einer Zahl zu den natürlichen Zahlen behandeln.

Die Einführungsregel für \(x \notin \mathbb{N}\) (\(\NotNaturalI{...}\)) besagt, dass, wenn gezeigt werden kann, dass ein Element \(x\) in einer Menge \(A\) nicht enthalten ist, die das Axiom der Unendlichkeit erfüllt, dann ist \(x\) kein Element der natürlichen Zahlen.

\[
\begin{array}{llll}
	i & (1) & \emptyset\in A\land \forall y\in A(y\cup\{y\}\in A)\land x\notin A) & ... \\
        j & (2) &  \forall y\in A(y\cup\{y\}\in A) & ... \\
        k & (3) &  x\notin A & ... \\
	i,j,k & (4) & x \notin \mathbb{N} & \NotNaturalI{1,2,3} \\
\end{array}
\]

Die Eliminierungsregel für \(x \notin \mathbb{N}\) (\(\NotNaturalE{...}\)) ermöglicht es uns, aus der Tatsache, dass \(x\) nicht zu \(\mathbb{N}\) gehört, die Existenz einer Menge zu folgern, welche das Axiom der Unendlichkeit erfüllt und x nicht enthält.

\[
\begin{array}{llll}
	i & (1) & x \notin \mathbb{N} & ... \\
	i & (2) & \exists A(\emptyset\in A\land \forall y\in A(y\cup\{y\}\in A)\land x\notin A) & \NotNaturalE{1} \\
\end{array}
\]

Hierbei ist \(i\) eine Liste von Annahmen.


\label{EmptysetInNatural}
\begin{theorem}[\(\emptyset\in\mathbb{N}\)]
\end{theorem}
\begin{proof}
	\[
	\begin{array}{llll}
		1 & (1) & \emptyset\in A\land \forall x\in A(x\cup \{x\}\in A & \rA \\
		1 & (2) & \emptyset\in A & \rAEa{1} \\
		& (3) & \emptyset\in A\land \forall x\in A(x\cup \{x\}\in A\rightarrow \emptyset\in A & \rII{1,2} \\
		& (4) & \emptyset\in\mathbb{N} & \NaturalI{3} \\
	\end{array}
	\]
\end{proof}

\label{nInNaturalImpncuLbnRbInNatural}
\begin{theorem}[\(n\in\mathbb{N}\vdash n\cup\{n\}\in\mathbb{N}\)]
\end{theorem}
\begin{proof}
	\[
	\begin{array}{llll}
		1 & (1) & n\in\mathbb{N} & \rA \\
		2 & (2) & \emptyset\in A\land \forall x\in A(x\cup \{x\}\in A) & \rA \\
		1 & (3) & \forall X(\emptyset\in X\land \forall y\in X(y\cup \{y\}\in X)\rightarrow n\in X) & \NaturalE{1} \\
		1 & (4) & \emptyset\in A\land \forall y\in A(y\cup \{y\}\in A)\rightarrow n\in A & \NaturalE{1} \\
		1,2 & (5) & n\in A & \rRE{4,2} \\
		2 & (6) & \forall x\in A(x\cup \{x\}\in A) & \rRE{4,2} \\
		1,2 & (7) & n\cup \{n\}\in A & \FaxInALpPLpxRpRpwyInAImpPLpyRp{5,6} \\
		1 & (8) & \emptyset\in A\land \forall x\in A(x\cup \{x\}\in A)\rightarrow n\cup \{n\}\in A & \rRI{2,7} \\
		1 & (9) & \forall x(\emptyset\in X\land \forall x\in X(x\cup \{x\}\in X)\rightarrow n\cup \{n\}\in X) & \rRI{8} \\	
		1 & (10) & n\cup \{n\}\in \mathbb{N} & \NaturalI{9} \\
	\end{array}
	\]
\end{proof}

\begin{definition}[\(0:=\emptyset\) (Definition der Zahl \(0\))]
\end{definition}


\begin{definition}[\(1 := 0\cup \{0\}\) (Definition der Zahl \(1\))]
\end{definition}

\begin{definition}[\(n\in\mathbb{N}\rightarrow n+1:=n\cup\{n\}\) (Definition des Nachfolgers)]
\end{definition}

\subsubsection{Regeln für die Definition der natürlichen Zahlen}

\label{rule:zeroSetDefinition} \label{rule:oneSetDefinition} \label{rule:successorSetDefinition}

% Regel für die Definition der Zahl \(0\)
\[
\begin{array}{llll}
	& (1) & 0 = \emptyset & \zeroSetDefinition \\
\end{array}
\]

% Regel für die Definition des Nachfolgers einer natürlichen Zahl
\[
\begin{array}{llll}
	i & (1) & n \in \mathbb{N} & \rA \\
	i & (2) & n + 1 = n \cup \{n\} & \successorSetDefinition{1} \\
\end{array}
\]

% Regel für die Definition der Zahl \(1\)
\[
\begin{array}{llll}
	& (1) & 1 = 0 \cup \{0\} & \oneSetDefinition \\
 	& (1) & 1 = 0+1 & \oneSetDefinition \\
\end{array}
\]


\label{OneEqualsLbEmptysetRb}
\begin{theorem}[\(1=\{\emptyset\}\)]
\end{theorem}
\begin{proof}
	\[
	\begin{array}{llll}
		      & (1) & 1=0\cup\{0\} & \oneSetDefinition{}\\
		    & (2) & 0=\emptyset  & \zeroSetDefinition{}\\
		    & (3) & 1=\emptyset\cup\{\emptyset\} & \rIE{2,1} \\		
            & (4) & 1=\{\emptyset\} & \AEqualsEmptysetcuA{3} \\		
	\end{array}
	\]
\end{proof}

\label{nInNaturalImpnInnPlusOne}
\begin{theorem}[\(n\in\mathbb{N}\vdash n\in n+1\)]
\end{theorem}
\begin{proof}
	\[
	\begin{array}{llll}
		  1   & (1) & n\in\mathbb{N} & \rA\\
		1   & (2) & n+1=n\cup\{n\}  & \successorSetDefinition{1}\\
            & (3) & n\in n\cup\{n\}  & \aInAcuLbaRb{}\\
        1   & (4) & n\in n+1  & \rIE{2,3}
	\end{array}
	\]
\end{proof}

\label{nInNaturalImpnPlusOneNotEqualsZero}
\begin{theorem}[\(n\in\mathbb{N}\vdash n+1\neq 0\)]
\end{theorem}
\begin{proof}
	\[
	\begin{array}{llll}
		  1   & (1) & n\in\mathbb{N} & \rA\\
		1   & (2) & n\in n+1  & \OneEqualsLbEmptysetRb{1}\\
  		1   & (3) & \exists x(x\in n+1)  & \rEI{2}\\
        1   & (4) & n+1\neq \emptyset  & \ExxInSImpSNotEqualsEmptyset{3}\\
            & (5) & 0=\emptyset  & \zeroSetDefinition{}\\
        1   & (6) & n+1\neq 0  & \rIE{5,4}\\
	\end{array}
	\]
\end{proof}

\label{ImpOneNotEqualsZero}
\begin{theorem}[\(\vdash 1\neq 0\)]
\end{theorem}
\begin{proof}
	\[
	\begin{array}{llll}
		      & (1) & 1=0+1 & \oneSetDefinition{}\\
		    & (2) & 0+1\neq 0  & \nInNaturalImpnPlusOneNotEqualsZero{}\\
  		    & (3) & 1\neq 0  & \rEI{2}\\
	\end{array}
	\]
\end{proof}




\subsubsection{Regeln für die Zugehörigkeit zu natürlichen Zahlen}
\label{rule:zeroIsNaturalNumber} \label{rule:oneIsNaturalNumber} \label{rule:successorIsNaturalNumber}

% Regel für die Einführung einer natürlichen Zahl
Wir führen im Folgenden die Regeln der Elementzugehörigkeit von \(0 \in \mathbb{N}\), \(1 \in \mathbb{N}\) und \(n+1 \in \mathbb{N}\) unter der Voraussetzung, dass \(n \in \mathbb{N}\) ist, ein:

% Regel, dass 0 eine natürliche Zahl ist
\[
\begin{array}{llll}
	& (1) & 0\in\mathbb{N} & \zeroIsNaturalNumber \\
\end{array}
\]
\textbf{Daraus folgt:} \(0 \in \mathbb{N}\).

% Regel, dass 1 eine natürliche Zahl ist
\[
\begin{array}{llll}
	& (1) & 1\in\mathbb{N} & \oneIsNaturalNumber \\
\end{array}
\]

% Regel für den Nachfolger einer natürlichen Zahl
\[
\begin{array}{llll}
	i & (1) & n \in \mathbb{N} & \rA \\
	i & (2) & n+1\in\mathbb{N} & \successorIsNaturalNumber{1} \\
\end{array}
\]

\(i\) ist dabei eine Liste von Annahmen.

Diese Regeln und Definitionen bilden die Grundlage für das Verständnis der Konstruktion der natürlichen Zahlen und ihrer Eigenschaften gemäß den Peano-Axiomen.


\label{mInNaturalwmNotEqualsZeroImpExxInNaturalLpxPlusOneEqualsmRp}
\begin{theorem}[\(m\in\mathbb{N}, m\neq 0\vdash\exists x\in\mathbb{N}(x+1=m)\) (Existenz des Vorgängers)]
\end{theorem}
\begin{proof}
	\[
	\begin{array}{llll}
		1  & (1) & m\in\mathbb{N} & \rA\\
		2  & (2) & m\neq 0 & \rA\\
		3  & (3) & \forall x\in\mathbb{N}(x+1\neq m) & \rA \\		
		& (4) & m\notin \mathbb{N}\setminus\{m\} & \aNotinASetminusLbaRb{} \\
		& (5) & 0\in\mathbb{N} & \zeroIsNaturalNumber \\
		2  & (6) & 0\in\mathbb{N}\setminus\{m\} & \aInAwbNotEqualsaImpaInASetminusLbbRb{5,2} \\
		7 & (7) & y\in\mathbb{N}\setminus\{m\} & \rA \\
		7 & (8) & y\in\mathbb{N} & \diffEa{7} \\
  		3,7 & (9) & y+1\neq m & \FaxInALpPLpxRpRpwyInAImpPLpyRp{3,8} \\
		7  & (10) & y+1\in\mathbb{N} & \successorIsNaturalNumber{8}  \\
		3,7 & (11) & y+1\in\mathbb{N}\setminus\{m\} & \aInAwaNotEqualsbImpaInASetminusLbbRb{10,9}  \\
		3 & (12) & \forall y\in\mathbb{N}\setminus\{m\}(y+1\in\mathbb{N}\setminus\{m\})  & \rSetUIa{7,11}  \\
           2,3 & (13) & m\notin\mathbb{N}  & \NotNaturalI{6,12,4}  \\ 
		1,2,3 & (14) & \bot  & \rBI{1,13} \\	
		1,2 & (15) & \neg(\forall x\in\mathbb{N}(x+1\neq m))  & \rCI{3,14} \\	
		1,2 & (16) & \exists x\in\mathbb{N}(x+1= m)  & \ExxInALpPLpxRpRpEqvnFaxInALpnPLpxRpRp{15} \\
	\end{array}
	\]
\end{proof}




\label{mInNaturalwmNotEqualsZeroImpExonlyonexInNaturalLpxPlusOneEqualsmRp}
\begin{theorem}[\(m\in\mathbb{N}, m\neq 0\vdash\exists! x\in\mathbb{N}(x+1=m)\) (Eindeutigkeit des Vorgängers in $\mathbb{N}$)]
\end{theorem}
\begin{proof}
	\[
	\begin{array}{lll p{4cm}}
		1  & (1) & m\in\mathbb{N} & \rA\\
		2  & (2) & m\neq 0 & \rA\\
		1,2  & (3) & \exists x\in\mathbb{N}(x+1=m) & \mInNaturalwmNotEqualsZeroImpExxInNaturalLpxPlusOneEqualsmRp{1,2}\\
		4  & (4) & a\in\mathbb{N}\land a+1=m  & \rA\\
		5  & (5) & b\in\mathbb{N}\land b+1=m  & \rA\\
		4  & (6) & a+1=m & \rAEb{4}\\
		5  & (7) & b+1=m & \rAEb{5}\\
		4  & (8) & a\cup \{a\}=m & \successorSetDefinition{6} \\
		5  & (9) & b\cup \{b\}=m & \successorSetDefinition{7} \\	
		4,5 & (10) & a\cup\{a\}=b\cup \{b\} & \aIdbwcIdbImpaIdc{8,9} \\
		& (11) & b\in b\cup\{b\} & \aInAcuLbaRb{} \\
		& (12) & a\in a\cup\{a\} & \aInAcuLbaRb{} \\
		4,5 & (13) & b\in a\cup\{a\} & \rIE{10,11} \\
		4,5 & (14) & a\in b\cup\{b\} & \rIE{10,12} \\
		4,5 & (15) & b\notin \{a\}\rightarrow b\in a & \zInAcuBEqvzNotinBTozInA{13} \\
		4,5 & (16) & a\notin \{b\}\rightarrow a\in b & \zInAcuBEqvzNotinBTozInA{14} \\
		17 & (17) & a\neq b & \rA \\
		17 & (18) & a\notin \{b\} & \nUnitSetE{17} \\
		17 & (19) & b\notin \{a\} & \nUnitSetE{17} \\
		4,5,17 & (20) & b\in a & \rIE{19,15} \\
		4,5,17 & (21) & a\in b & \rIE{18,16} \\
		4,5,17 & (22) & b\notin a & \aInbImpbNotina{21} \\
		4,5,17 & (23) & \bot & \rBI{20,22} \\		
		4,5 & (23) & a=b & \rCE{17,23} \\	
		1,2 & (24) & \exists!x(x\in \mathbb{N}\land x+1=m) & \UEI{3,4,5,23} \\	
	\end{array}
	\]
\end{proof}

\label{def:predecessor}
\begin{definition}[Definition des Vorgängers]
Für eine natürliche Zahl \( n \in \mathbb{N} \) und \( n \neq 0 \) definieren wir den Vorgänger \( n-1 \) von \( n \) als die natürliche Zahl \( x \), die die Gleichung \( x + 1 = n \) erfüllt:
\[
n-1 := x, \text{ wobei } x \in \mathbb{N} \text{ und } x + 1 = n.
\]
\end{definition}

\subsubsection{Regeln für die Einführung und Elimination des Vorgängers in den  natürlichen Zahlen}
\label{rule:rPredecessorI} \label{rule:rPredecessorEa} \label{rule:rPredecessorEb} \label{rule:rPredecessorEc}

% Regel für die Einführung eines Vorgängers in den natürlichen Zahlen
Wir führen im Folgenden die Regeln für den Vorgänger \(n-1\) einer natürlichen Zahl \(n\) unter der Voraussetzung, dass \(n \in \mathbb{N}\) und \(n \neq 0\) ist, ein:

% Regel für die Zugehörigkeit des Vorgängers zu den natürlichen Zahlen
\[
\begin{array}{llll}
	i & (1) & n \in \mathbb{N} & ... \\
	j & (2) & n \neq 0 & ... \\
	i,j & (3) & n-1 \in \mathbb{N} & \rPredecessorI{1, 2} \\
        i,j & (4) & n = (n-1)+1 & \rPredecessorI{1,2} \\
        i,j & (5) & n = (n+1)-1 & \rPredecessorI{1,2} \\
\end{array}
\]

\(i, j\) sind dabei Listen von Annahmen.

% Regel für die Eliminierung des Vorgängers
Wir führen im Folgenden die Regel für die Eliminierung des Vorgängers ein, welche es erlaubt, aus der Annahme \(n-1\in\mathbb{N}\) zu folgern, dass \(n\) eine natürliche Zahl größer als \(0\) ist:

\[
\begin{array}{llll}
	i & (1) & n-1 \in \mathbb{N} & ... \\
	i & (2) & n \in \mathbb{N} & \rPredecessorEa{1} \\
    i & (3) & n = (n-1)+1 & \rPredecessorEa{1} \\
    i & (4) & n = (n+1)-1 & \rPredecessorEc{2} \\
	i & (5) & n \neq 0 & \rPredecessorEb{1} \\
\end{array}
\]

\(i\) ist dabei eine Liste von Annahmen.

\subsubsection{Regel für die Eindeutigkeit des Vorgängers}
\label{rule:rPredecessorUniqueness}

Die Regel für die Eindeutigkeit des Vorgängers besagt, dass wenn zwei natürliche Zahlen \(m\) und \(n\) gleich sind, auch ihre Vorgänger gleich sind. Formal lässt sich dies so ausdrücken:

\[
m = n \rightarrow (m-1 = n-1)
\]

Im Kalkül des natürlichen Schließens wird dies folgendermaßen formalisiert:

\[
\begin{array}{llll}
i & (1) & m\in\mathbb{N} & ... \\
j & (2) & n\in\mathbb{N} & ... \\
k & (3) & m\neq 0 & ... \\
l & (1) & m = n & ... \\
i,j,k,l & (2) & m - 1 = n - 1 & \rPredecessorUniqueness{1,2,3,4}
\end{array}
\]

alternativ kann folgende Regel verwendet werden:

\[
\begin{array}{llll}
i & (1) & m\in\mathbb{N} & ... \\
j & (2) & n\in\mathbb{N} & ... \\
k & (3) & n\neq 0 & ... \\
l & (1) & m = n & ... \\
i,j,k,l & (2) & m - 1 = n - 1 & \rPredecessorUniqueness{1,2,3,4}
\end{array}
\]

\(i,j,k\) und \(l\) ist dabei eine Listen von Annahmen.

\label{nInNaturalwnNotEqualsZeroImpnMinusOneInn}
\begin{theorem}[\(n\in\mathbb{N}, n\neq 0\vdash n-1\in n\)]
\end{theorem}
\begin{proof}
	\[
	\begin{array}{llll}
		1 & (1) & n\in\mathbb{N} & \rA \\
  		2 & (2) & n\neq 0 & \rA \\
  		1,2 & (3) & n-1\in\mathbb{N} & \rPredecessorI{1,2} \\
            1,2 & (4) & (n-1)+1 = (n-1)\cup \{n-1\} & \successorSetDefinition{3} \\
            1,2 & (5) & n = (n-1)+1 & \rPredecessorEa{3} \\
            1,2 & (6) & n = (n-1)\cup \{n-1\} & \rIE{5,4} \\
                & (7) & n-1\in (n-1)\cup \{n-1\} & \aInAcuLbaRb{} \\
            1,2 & (8) & n-1\in n & \rIE{6,7} \\
	\end{array}
	\]
\end{proof}


\chapter{Das Prinzip der vollständigen Induktion}

\begin{tempdefinition}
    \[S:=\{n\in\mathbb{N} \mid \neg P(n)\}\]
\end{tempdefinition}

\label{PLpZeroRpwFanInNaturalLpPLpnRpToPLpnPlusOneRpRpImpFanInNaturalPLpnRpLo}
\begin{lemma}[1]
\[s\in S\vdash s\in\mathbb{N}\]
\end{lemma}
\begin{proof}
	\[
	\begin{array}{llll}
		1 &  (1) & s\in S & \rA \\
		1 &  (2) & s\in \{n\in\mathbb{N} \mid \neg P(n)\} & \rIE{df(S),1} \\	
		1 &  (3) & s\in\mathbb{N} & \inE{2} \\	
	\end{array}
	\]
\end{proof}

\label{PLpZeroRpwFanInNaturalLpPLpnRpToPLpnPlusOneRpRpImpFanInNaturalPLpnRpLoo}
\begin{lemma}[2]
\[s\in S\vdash \neg P(s)\]
\end{lemma}
\begin{proof}
	\[
	\begin{array}{llll}
		% Annahme: Induktionsprinzip ist falsch
		1 &  (1) & s\in S & \rA \\
		1 &  (2) & s\in \{n\in\mathbb{N} \mid \neg P(n)\} & \rIE{df(S),1} \\	
		1 &  (3) & \neg P(s) & \inE{2} \\	
	\end{array}
	\]
\end{proof}

\label{PLpZeroRpwFanInNaturalLpPLpnRpToPLpnPlusOneRpRpImpFanInNaturalPLpnRpLooo}
\begin{lemma}[3]
\[s\notin S, s\in \mathbb{N}\vdash P(s)\]
\end{lemma}
\begin{proof}
	\[
	\begin{array}{llll}
		1 &  (1) & s\notin S & \rA \\
            2 &  (2) & s\in \mathbb{N} & \rA \\
            1 &  (3) & s\notin \{n\in\mathbb{N} \mid \neg P(n)\} & \rIE{df(S), 1} \\
		1,2 &  (4) & P(s) & \notinEa{3,2} \\	
	\end{array}
	\]
\end{proof}

\label{PLpZeroRpwFanInNaturalLpPLpnRpToPLpnPlusOneRpRpImpFanInNaturalPLpnRpLoooo}
\begin{lemma}[4]
\[\neg (\forall n \in \mathbb{N} \, P(n))\vdash S\neq\emptyset\]
\end{lemma}
\begin{proof}
	\[
	\begin{array}{llll}
		% Annahme: Induktionsprinzip ist falsch
		1 &  (1) & \neg (\forall n \in \mathbb{N} \, P(n)) & \rA \\
		1 &  (2) & \exists n \in \mathbb{N} \neg P(n) & \ExxInALpPLpxRpRpEqvnFaxInALpnPLpxRpRp{1} \\
		1 &  (3) & m\in\mathbb{N}\land \neg P(m) & \rSetEEa{2} \\
		1 &  (4) & m\in \{n\in\mathbb{N}\mid \neg P(n)\} & \inI{4} \\
		1 &  (5) & m\in S & \rIE{df(S),4} \\
		1 &  (6) & S\neq\emptyset & \ExxInSImpSNotEqualsEmptyset{5} \\	
	\end{array}
	\]
\end{proof}

\label{PLpZeroRpwFanInNaturalLpPLpnRpToPLpnPlusOneRpRpImpFanInNaturalPLpnRpLooooo}
\begin{lemma}[5]
\[S\neq\emptyset, P(0) \vdash \exists x-1\in \mathbb{N} (P(x-1)\land P(x))\]
\end{lemma}
\begin{proof}
	\[
	\begin{array}{llll}
		1 &  (1) & S\neq\emptyset & \rA \\
		2 &  (2) & P(0) & \rA \\	
		1 &  (3) & \exists x\in S(x\cap S=\emptyset) & \RegI{2} \\	
		1 & (4) & s\in S\land s\cap S=\emptyset & \rSetEEa{3} \\
		1 & (5) & s\in S & \rAEa{4} \\
		1 & (6) & s\cap S=\emptyset & \rAEb{4} \\
		1 & (7) & s\in\mathbb{N} & \PLpZeroRpwFanInNaturalLpPLpnRpToPLpnPlusOneRpRpImpFanInNaturalPLpnRpLo{5} \\
		1 & (8) & \neg P(s) & \PLpZeroRpwFanInNaturalLpPLpnRpToPLpnPlusOneRpRpImpFanInNaturalPLpnRpLoo{5} \\
            9 & (9) & s=0 & \rA \\
            1,9 & (10) & \neg P(0) & \rIE{9,8} \\
		1,2,9 & (11) & \bot & \rBI{2,10} \\
		1,2 & (12) & s\neq 0 & \rCI{9,11} \\
            1,2 & (13) & s-1\in s & \nInNaturalwnNotEqualsZeroImpnMinusOneInn{7,12}  \\
            1,2 & (14) & s-1\notin S & \AcaBEqualsEmptysetwxInAImpxNotinB{6,13}  \\
            1,2 & (15) & s-1\in \mathbb{N} & \rPredecessorI{7,12}  \\
            1,2 & (16) & P(s-1)& \PLpZeroRpwFanInNaturalLpPLpnRpToPLpnPlusOneRpRpImpFanInNaturalPLpnRpLooo{14,15}  \\
            1,2 & (16) & P(s-1)\land \neg P(s)& \rAI{16,8}  \\
            1,2 & (17) & \exists x-1\in\mathbb{N} (P(x-1)\land \neg P(x))& \rSetEIa{15,16}  \\           
	\end{array}
	\]
\end{proof}
%ISE
\label{PLpZeroRpwFanInNaturalLpPLpnRpToPLpnPlusOneRpRpImpFanInNaturalPLpnRp}
\begin{theorem}[\(P(0), \forall n \in \mathbb{N} (P(n) \rightarrow P(n+1)) \vdash \forall n \in \mathbb{N} P(n)\) (Induktionsprinzip)]
\end{theorem}
\begin{proof}
	\[
	\begin{array}{llll}
		% Annahme: Induktionsprinzip ist falsch
		1 &  (1) & \neg (\forall n \in \mathbb{N} \, P(n)) & \rA \\
		2 &  (2) & P(0) & \rA \\			
		3 &  (3) & \forall n \in \mathbb{N}(P(n)\rightarrow P(n+1)) & \rA \\
		1 &  (4) & S\neq\emptyset & \PLpZeroRpwFanInNaturalLpPLpnRpToPLpnPlusOneRpRpImpFanInNaturalPLpnRpLo{1} \\	
            1,2 & (5) & \exists x-1\in\mathbb{N}(P(x-1)\land\neg P(x))& \PLpZeroRpwFanInNaturalLpPLpnRpToPLpnPlusOneRpRpImpFanInNaturalPLpnRpLooooo{4,2}  \\
            1,2 & (6) & s-1\in\mathbb{N}\land (P(s-1)\land\neg P(s))& \rSetEEa{5}  \\
            1,2 & (7) & s-1\in\mathbb{N}& \rAEa{6}  \\
            1,2 & (8) & P(s-1)\land\neg P(s)& \rAEb{6}  \\
            1,2 & (9) & P(s-1)& \rAEa{8}  \\
            1,2 & (10) & \neg P(s)& \rAEb{8}  \\
            3 & (11) & s-1\in\mathbb{N}\rightarrow (P(s-1)\rightarrow P(s))& \rUE{3}  \\
            1,2,3 & (12) & P(s-1)\rightarrow P(s)& \rRE{7,11}  \\
            1,2,3 & (13) & P(s)& \rRE{9,12}  \\
            1,2,3 & (14) & \bot & \rBI{10,13}  \\
            2,3 & (15) & \forall n \in \mathbb{N} \, P(n) & \rCE{1,14}  \\
	\end{array}
	\]
\end{proof}

\section{Regel der vollständigen Induktion über den natürlichen Zahlen}
\label{rule:rInductionN}

Die Regel der vollständigen Induktion über den natürlichen Zahlen (\(\rInductionN{}\)) erlaubt es, nach der Herleitung von \(P(0)\), der Annahme \(P(n)\) für ein beliebiges \(n \in \mathbb{N}\), und der Herleitung von \(P(n+1)\) aus diesen Annahmen, direkt auf die Aussage \(\forall n \in \mathbb{N} P(n)\) zu schließen. Es ist somit nicht mehr notwendig, die Zwischenschritte \(\forall n \in \mathbb{N} (P(n) \rightarrow P(n+1))\) explizit aufzuschreiben.

\[
\begin{array}{llll}
    i & (1) & P(0) & ... \\
    2 & (2) & P(n) & \rA \\
    3 & (3) & n \in \mathbb{N} & \rA \\
    2,3,j & (4) & P(n+1) & ... \\
    i,j & (5) & \forall n \in \mathbb{N} P(n) & \rInductionN{1,2,3,4}
\end{array}
\]

\[
\begin{array}{llll}
    i & (1) & P(0) & ... \\
    2 & (2) & P(n) & \rA \\
    3 & (3) & n \in \mathbb{N} & \rA \\
    2,3,j & (4) & P(n+1) & ... \\
    3,i,j & (5) & P(n) & \rInductionN{1,2,3,4}
\end{array}
\]

Hierbei beziehen sich \(n\) und \(P(n)\) auf ein beliebiges Element und eine beliebige Aussage über \(\mathbb{N}\). Der Ausdruck \(\rInductionN{1,4}\) zeigt an, dass die Regel der vollständigen Induktion angewendet wurde, um die allgemeine Aussage \(\forall n \in \mathbb{N} P(n)\) abzuleiten.

\(i\) und \(j\) sind dabei Listen von Annahmen, und \(n\) kommt in keiner der Annahmen \(i\) und \(j\) vor, aus denen \(P(0)\) und \(P(n+1)\) abgeleitet werden.






\subsection{Induktive Definition von Tupel-Mengen}

\begin{definition}
Sei $A$ eine Menge und $n \in \mathbb{N}$. Dann definieren wir die n-Tupel-Menge $A^n$ induktiv als:
\begin{itemize}
    \item $A^0 := \emptyset$, die leere Menge.
    \item $A^{n+1} := A^n \times A$.
\end{itemize}
\end{definition}

\paragraph{Regeln für \( A^n \).}
\label{rule:zeroPowerSet} \label{rule:nextPowerSet}
\[
\begin{array}{llll}
	& (1) & A^0=\emptyset & \zeroPowerSet{} \\
\end{array}
\]

\[
\begin{array}{llll}
	& (1) & A^{n+1}=A^n\times A & \nextPowerSet{} \\
\end{array}
\]

\begin{theorem}[(\(n\in\mathbb{N}, A\) ist eine Menge \(\vdash A^n\) ist eine Menge.]
\end{theorem}
\begin{proof}
	\[
	\begin{array}{llll}
		  1 &  (1) & A \text{ ist eine Menge} & \rA{} \\
            &  (2) & A^0=\emptyset & \zeroPowerSet{} \\
            &  (3) & \emptyset \text{ ist eine Menge} & \remptysetIsSet{} \\
            &  (4) & A^0 \text{ ist eine Menge} & \rIE{1,2} \\
            5 &  (5) & n\in\mathbb{N} & \rA \\
            6 &  (6) & A^n \text{ ist eine Menge} & \rA \\
              &  (7) & A^{n+1}=A^n\times A & \nextPowerSet{}  \\
            1,6  &  (8) & A^n\times A \text{ ist eine Menge} & \cartesianSetExists{6,1}  \\
            1,6  &  (9) & A^{n+1} \text{ ist eine Menge} & \rIE{7,8}  \\
            1,5  &  (10) & A^{n} \text{ ist eine Menge} & \rInductionN{4,5,6,9}  \\
	\end{array}
	\]
\end{proof}

\paragraph{Existenzregel von Tupel-Mengen}
\label{rule:PowerSetExists}
Die Existenzregel stellt sicher, dass für alle Mengen \(A\) und alle \(n\in\mathbb{N}\) \(A^{n}\) ebenfalls eine Menge ist.

\[
\begin{array}{llll}
	i & (1) & A \text{ ist eine Menge} & \dots  \\
        j & (2) & n\in\mathbb{N} & \dots  \\
	i,j & (3) & A^n\text{ ist eine Menge} & \PowerSetExists{1,2} \\
\end{array}
\]

\(i\) und \(j\) sind dabei Listen von Annahmen.




\chapter{Abbildungen}

\begin{definition}[Abbildung]
    Der \textbf{Begriff der Abbildung (oder Funktion)} \( f: A \to B \) wird \textbf{implizit definiert}. Das Symbol \( f \) ist eine Relation zwischen den Mengen \( A \) und \( B \), die die folgenden Eigenschaften erfüllt:
    
    \begin{itemize}
        \item \textbf{Definitionsbereich (Domain)}: Die Menge \( A \), aus der die Elemente stammen, die durch die Abbildung \( f \) abgebildet werden, heißt \textbf{Definitionsbereich} oder \textbf{Domäne} der Abbildung. Es gilt:
        \[
        \text{Dom}(f) = A.
        \]
        
        \item \textbf{Wertebereich (Codomain)}: Die Menge \( B \), in die die Elemente abgebildet werden, heißt \textbf{Wertebereich} oder \textbf{Codomäne} der Abbildung. Es gilt:
        \[
        \text{Cod}(f) = B.
        \]
        
        \item \textbf{Bild (Image)}: Die Menge der tatsächlich angenommenen Werte der Abbildung heißt \textbf{Bildbereich} oder \textbf{Bild} der Abbildung. Es gilt:
        \[
        f(A) = \{ f(a) \mid a \in A \} \subseteq B.
        \]
        
        \item \textbf{Teilmenge des kartesischen Produkts}: Die Abbildung \( f \) ist eine Teilmenge des kartesischen Produkts \( A \times B \):
        \[
        f \subseteq A \times B.
        \]
        
        \item \textbf{Eindeutigkeit}: Für jedes \( a \in A \) existiert genau ein \( b \in B \), sodass \( (a, b) \in f \) ist. Dieses eindeutige \( b \) wird als \( f(a) \) bezeichnet:
        \[
        \forall a \in A \, \exists! b \in B \, \big( (a, b) \in f \big) \quad \text{und} \quad b = f(a).
        \]
    \end{itemize}
\end{definition}
\begin{remark}
     Für den Bildbereich \( f(A) = \{ f(a) \mid a \in A \} \) existiert nach dem Ersetzungsaxiom eine Menge, die alle durch \( f \) definierten Bilder umfasst. Somit ist \( f(A) \subseteq B \) als Menge definiert und durch das Ersetzungsaxiom in ZFC immer gegeben.
\end{remark}

Die Definition von Abbildungen verwendet das \textit{Axiom der Potenzmenge}, welches die Existenz des kartesischen Produkts \( A \times B \) sicherstellt, sowie das \textit{Aussonderungsaxiom}, welches die Auswahl derjenigen Paare erlaubt, die die Bedingung der Eindeutigkeit erfüllen.


\subsubsection*{Einführungsregel für Funktionen}
\label{rule:toI}
Die Einführungsregel für Funktionen \( f: A \to B \) ermöglicht es, eine Funktion zu definieren, indem für jedes Element \( a \in A \) ein eindeutiges Element \( b \in B \) bestimmt wird, sodass \( f(a) = b \) gilt. Diese Regel basiert auf der Definition der Abbildung und der Eindeutigkeit der Zuordnung.

\[
\begin{array}{llll}
    i       & (1) & f\subseteq A\times B & ... \\
    j       & (2) & \forall a \in A \, \exists! b \in B \, ((a, b) \in f) & ... \\
    i,j     & (3) & f:A\rightarrow B & \toI{1,2}
\end{array}
\]

\(i\) und \(j\) sind dabei Listen von Annahmen.

\subsubsection*{Eliminierungsregel für Funktionen}
\label{rule:toE}
Die Eliminierungsregel für Funktionen \( f: A \to B \) erlaubt es, aus der Tatsache, dass eine Funktion \( f \) existiert und ein Element \( a \in A \) auf ein \( b \in B \) abbildet, die Zugehörigkeit von \( a \) zu \( A \) und \( b \) zu \( B \) abzuleiten.

\[
\begin{array}{llll}
    i       & (1) & f:A\rightarrow B & ... \\
    i       & (2) & f\subseteq A\times B & \toE{1} \\
    i       & (3) & \forall a \in A \, \exists! b \in B \, ((a, b) \in f) & \toE{1} \\
\end{array}
\]

\[
\begin{array}{llll}
    i       & (1) & \forall a\in A(f(a)\in B) & ... \\
    i       & (2) & \forall a \in A \, \exists! b \in B \, ((a, b) \in f) & \toE{1} \\
\end{array}
\]

\(i\) ist dabei die Liste der Annahmen.

\begin{definition}[Binäre Operation]
    Eine \textbf{binäre Operation} auf einer Menge \( A \) ist eine Abbildung \( \cdot : A \times A \to A \), die zwei Elemente \( a, b \in A \) nimmt und das Ergebnis \( a \cdot b \in A \) liefert. Diese Operation wird definiert durch:
    
    \[
    \forall a,b\in A(a \cdot b := \cdot(a, b)) 
    \]
\end{definition}

\subsubsection*{Einführungsregel für binäre Operationen}
\label{rule:cdotI}

Die Einführungsregel besagt, dass für alle \( a, b \in A \), das Ergebnis der binären Operation \( a \cdot b \) ebenfalls in \( A \) liegt:

\[
\begin{array}{llll}
    (1) & a \in A & & \\
    (2) & b \in A & & \\
    (3) & a \cdot b \in A & \cdotI{1,2} & 
\end{array}
\]


\begin{theorem}[\(\forall a\in A(f(a)\in B)\vdash f:A\rightarrow B\)]
\end{theorem}
\begin{proof}
	\[
	\begin{array}{llll}
		1   &  (1) & \forall a\in A(f(a)\in B) & \rA \\
            1   &  (2) & a\in A\rightarrow f(a)\in B & \rSetUEb{1} \\
            3   &  (3) & a\in A & \rA \\
            1,3 &  (4) & f(a)\in B & \rRE{2,3} \\
            1,3 &  (5) & (a,f(a))\in A\times B & \timesI{3,4} \\
	\end{array}
	\]
\end{proof}

\begin{definition}[Injektivität]
    Sei \( f: A \to B \) eine Abbildung. Die Abbildung \( f \) heißt \textbf{injektiv} (oder \textbf{eineindeutig}), wenn:
    \[
    \forall a_1,a_2\in A(f(a_1) = f(a_2) \rightarrow a_1 = a_2)
    \]
    Eine injektive Abbildung ordnet also unterschiedlichen Elementen des Definitionsbereichs unterschiedliche Elemente im Wertebereich zu.
\end{definition}

\subsubsection*{Einführungsregel für Injektivität}
\label{rule:InjI}

Die Einführungsregel für die Injektivität einer Funktion \( f: A \to B \) ermöglicht es, die Injektivität zu zeigen, indem man beweist, dass für beliebige \( a_1, a_2 \in A \) die Gleichheit \( f(a_1) = f(a_2) \) impliziert, dass \( a_1 = a_2 \) gilt.

\[
\begin{array}{llll}
    i       & (1) & \forall a_1, a_2 \in A \, (f(a_1) = f(a_2) \rightarrow a_1 = a_2) & ... \\
    i       & (2) & f \text{ ist injektiv} & \InjI{1}
\end{array}
\]

Diese Regel definiert, dass wenn die Bedingung erfüllt ist, \( f \) als injektiv angenommen werden kann.

\(i\) ist dabei dabei eine Liste von Annahmen.

\subsubsection*{Eliminierungsregel für Injektivität}
\label{rule:InjE}

Die Eliminierungsregel für Injektivität erlaubt es, aus der Injektivität einer Funktion und der Tatsache, dass \( f(a_1) = f(a_2) \), darauf zu schließen, dass \( a_1 = a_2 \) gilt.

\[
\begin{array}{llll}
    i       & (1) & f \text{ ist injektiv} & ... \\
    2       & (2) & f(a_1) = f(a_2) & \rA \\
    i,2     & (3) & a_1 = a_2 & \InjE{1,2}
\end{array}
\]

Hierbei zeigt die Eliminationsregel, dass die Gleichheit der Funktionswerte \( f(a_1) = f(a_2) \) aufgrund der Injektivität der Funktion zur Gleichheit der Argumente \( a_1 = a_2 \) führt.

\(i\) ist dabei dabei eine Liste von Annahmen.

\begin{definition}[Surjektivität]
    Sei \( f: A \to B \) eine Abbildung. Die Abbildung \( f \) heißt \textbf{surjektiv} (oder \textbf{auf}), wenn: 
    \[
    \forall b\in B\exists a\in A(f(a) = b).
    \]
\end{definition}

\subsubsection*{Einführungsregel für Surjektivität}
\label{rule:SurjI}

Die Einführungsregel für die Surjektivität einer Funktion \( f: A \to B \) ermöglicht es, die Surjektivität zu zeigen, indem man für jedes \( b \in B \) ein entsprechendes \( a \in A \) findet, sodass \( f(a) = b \) gilt.

\[
\begin{array}{llll}
    i       & (1) & \forall b \in B \, \exists a \in A \, (f(a) = b) & ... \\
    i       & (2) & f \text{ ist surjektiv} & \SurjI{1}
\end{array}
\]

Diese Regel definiert, dass wenn die Bedingung erfüllt ist, \( f \) als surjektiv angenommen werden kann.

\(i\) ist dabei eine Liste von Annahmen.

\subsubsection*{Eliminierungsregel für Surjektivität}
\label{rule:SurjE}

Die Eliminierungsregel für Surjektivität erlaubt es, aus der Surjektivität einer Funktion zu folgern, dass für jedes \( b \in B \) ein Element \( a \in A \) existiert, sodass \( f(a) = b \) gilt.

\[
\begin{array}{llll}
    i       & (1) & f \text{ ist surjektiv} & ... \\
    i     & (2) & \forall b\in A\exists a \in A \, (f(a) = b) & \SurjE{1}
\end{array}
\]

Hierbei zeigt die Eliminationsregel, dass für jedes \( b \in B \), welches im Wertebereich liegt, ein \( a \in A \) existiert, sodass \( f(a) = b \) gilt.

\(i\) ist dabei eine Liste von Annahmen.

\begin{definition}[Bijektivität]
    Sei \( f: A \to B \) eine Abbildung. Die Abbildung \( f \) heißt \textbf{bijektiv} (oder \textbf{umkehrbar eindeutig}), wenn sie sowohl injektiv als auch surjektiv ist. 
\end{definition}

\subsubsection*{Einführungsregel für Bijektivität}
\label{rule:BijectionI}

Die Einführungsregel für die Bijektivität einer Funktion \( f: A \to B \) ermöglicht es, die Bijektivität zu zeigen, indem man sowohl die Injektivität als auch die Surjektivität der Funktion nachweist.

\[
\begin{array}{llll}
    i       & (1) & f \text{ ist injektiv} & ... \\
    j       & (2) & f \text{ ist surjektiv} & ... \\
    i,j     & (3) & f \text{ ist bijektiv} & \BijectionI{1,2}
\end{array}
\]

Diese Regel legt fest, dass \( f \) als bijektiv angenommen werden kann, wenn sowohl die Injektivitäts- als auch die Surjektivitätsbedingungen erfüllt sind.

\(i\) und \(j\) sind dabei Listen von Annahmen.

\subsubsection*{Eliminierungsregel für Bijektivität}
\label{rule:BijectionE}

Die Eliminierungsregel für Bijektivität erlaubt es, aus der Bijektivität einer Funktion \( f \) abzuleiten, dass \( f \) sowohl injektiv als auch surjektiv ist.

\[
\begin{array}{llll}
    i       & (1) & f \text{ ist bijektiv} & ... \\
    i       & (2) & f \text{ ist injektiv} & \BijectionE{1}\\
    i       & (2) & f \text{ ist surjektiv} & \BijectionE{1}
\end{array}
\]

Hierbei zeigt die Eliminierungsregel, dass die Bijektivität von \( f \) sowohl die Injektivität als auch die Surjektivität impliziert.

\(i\) ist dabei eine Liste von Annahmen.

\chapter{Einführung in algebraische Strukturen}

\section{Halbgruppen}

\begin{definition}[Halbgruppe]
    Der \textbf{Begriff der Halbgruppe} wird durch das \textbf{Symbol} \((S, \cdot)\) \textbf{implizit definiert}. Dabei gelten die folgenden Axiome:
    
    \begin{itemize}
     
        \item \textbf{Existenz der binären Operation \(\cdot\)}:
        \[
        \cdot \colon S \times S \to S \text{ ist eine binäre Operation auf } S.
        \]
        
        \item \textbf{Assoziativität}: 
        \[
        \forall a, b, c \in S \colon (a \cdot b) \cdot c = a \cdot (b \cdot c).
        \]
    \end{itemize}
\end{definition}
\begin{remark}
    In Halbgruppen wird der Operator \(\cdot\) häufig weggelassen, sodass die Operation zwischen zwei Elementen \(a\) und \(b\) einfach als \(ab:=a\cdot b\) geschrieben wird. Dies dient der Vereinfachung der Notation, insbesondere in formalen Ausdrücken.
\end{remark}

\subsubsection*{Einführungsregel für Halbgruppen (HG\textsubscript{I})}
\label{rule:rSemigroupI}
Die Einführungsregel für Halbgruppen \((S, \cdot)\) ermöglicht es, eine Halbgruppe zu definieren, indem die Menge \(S\), die binäre Operation \(\cdot\) und die Assoziativität der Operation gezeigt werden. Diese Regel basiert auf den Axiomen der Halbgruppe.

\[
\begin{array}{llll}
    i       & (1) & \cdot \colon S \times S \to S & \dots \\
    j       & (2) & \forall a, b, c \in S ((a \cdot b) \cdot c = a \cdot (b \cdot c)) & \dots \\
    i,j     & (3) & (S, \cdot) \text{ ist eine Halbgruppe} & \rSemigroupI{1,2}
\end{array}
\]

\(i\) und \(j\) sind dabei Listen von Annahmen.

\subsubsection*{Eliminierungsregel für Halbgruppen (HG\textsubscript{E})}
\label{rule:rSemigroupE}
Die Eliminierungsregel für Halbgruppen \((S, \cdot)\) erlaubt es, aus der Tatsache, dass \((S, \cdot)\) eine Halbgruppe ist, die Existenz der Menge \(S\), der binären Operation \(\cdot\) sowie die Assoziativität der Operation abzuleiten.

\[
\begin{array}{llll}
    i       & (1) & (S, \cdot) \text{ ist eine Halbgruppe.} & \dots \\
    i       & (2) & \cdot \colon S \times S \to S& \rSemigroupE{1} \\
    i       & (3) & \forall a, b, c \in S \colon (a \cdot b) \cdot c = a \cdot (b \cdot c). & \rSemigroupE{1}
\end{array}
\]

\(i\) ist dabei die Liste der Annahmen.

\subsubsection*{Regel der Assoziativität}
\label{rule:rAssociativityHG}
Die Regel der Assoziativität ermöglicht es, aus den Elementen einer Halbgruppe die Assoziativität der binären Operation abzuleiten. Dies ist eine direkte Anwendung des Assoziativitätsaxioms der Halbgruppe.

\[
\begin{array}{llll}
        & (1) & (ab)c = a(bc) & \rAssociativityHG{}
\end{array}
\]


\subsection{Abelsche Halbgruppen}

\begin{definition}[Abelsche Halbgruppe]
    Eine \textbf{abelsche Halbgruppe} ist eine Halbgruppe \((S, \cdot)\), in der zusätzlich zur Assoziativität die Kommutativität der binären Operation \(\cdot\) gilt. Das bedeutet, dass für alle Elemente \(a, b \in S\) gilt:
    
    \begin{itemize}
        \item \textbf{Kommutativität}:
        \[
        \forall a, b \in S(a \cdot b = b \cdot a).
        \]
    \end{itemize}
\end{definition}

\begin{remark}
    In abelschen Halbgruppen wird der Operator \(\cdot\) häufig weggelassen, sodass die Operation zwischen zwei Elementen \(a\) und \(b\) einfach als \(ab := a \cdot b\) geschrieben wird. Dies dient der Vereinfachung der Notation, insbesondere in formalen Ausdrücken.
\end{remark}

\subsubsection*{Einführungsregel für abelsche Halbgruppen (AHG\textsubscript{I})}
\label{rule:rAbelianSemigroupI}
Die Einführungsregel für abelsche Halbgruppen ermöglicht es, eine abelsche Halbgruppe zu definieren, indem die Menge \(S\), die binäre Operation \(\cdot\), die Assoziativität und die Kommutativität der Operation gezeigt werden.


\[
\begin{array}{llll}
    i       & (1) &  (S, \cdot) \text{ ist eine Halbgruppe.} & \dots \\
    j       & (2) & \forall a, b \in S (ab = ba) & \dots \\
    i,j   & (3) & (S, \cdot) \text{ ist eine abelsche Halbgruppe.} & \rAbelianSemigroupI{1,2}
\end{array}
\]



\subsubsection*{Eliminierungsregel für abelsche Halbgruppen (AHG\textsubscript{E})}
\label{rule:rAbelianSemigroupE}
Die Eliminierungsregel für abelsche Halbgruppen ermöglicht es, aus der Tatsache, dass \((S, \cdot)\) eine abelsche Halbgruppe ist, die Assoziativität und Kommutativität der binären Operation \(\cdot\) sowie die Halbgruppenstruktur von \(S\) abzuleiten.

\[
\begin{array}{llll}
    i       & (1) &  (S, \cdot) \text{ ist eine abelsche Halbgruppe.} & \dots \\
    i       & (2) & \forall a, b \in S(ab = ba) & \rAbelianSemigroupE{1} \\
    i       & (3) & (S, \cdot) \text{ ist eine Halbgruppe.} & \rAbelianSemigroupE{1}
\end{array}
\]

\(i\) ist dabei eine Liste von Annahmen.

\subsubsection*{Kommutativität der abelschen Halbgruppe}
\label{rule:rCommutativeSemigroup}
Die Kommutativität der abelschen Halbgruppe ermöglicht es, in einer abelschen Halbgruppe \((S, \cdot)\) die Reihenfolge der Operanden der binären Operation \(\cdot\) zu vertauschen. Dies folgt direkt aus der Definition einer abelschen Halbgruppe, bei der die Operation \(\cdot\) sowohl assoziativ als auch kommutativ ist.

\[
\begin{array}{llll}
          & (1) & ab = ba & \rCommutativeSemigroup{} \\
\end{array}
\]


Die Regel der Kommutativität erlaubt es, die Argumente \(a\) und \(b\) der Operation zu vertauschen, ohne dass sich das Ergebnis ändert, was eine grundlegende Eigenschaft von abelschen Halbgruppen ist.

\section{Monoide}

\begin{definition}[Monoid]
    Der \textbf{Begriff des Monoids} wird durch das \textbf{Symbol} \((M, \cdot)\) \textbf{implizit definiert}. Dabei gelten die folgenden Axiome:
    
    \begin{itemize}
        \item \textbf{Existenz der binären Operation \(\cdot\)}:
        \[
        \cdot \colon M \times M \to M \text{ ist eine binäre Operation auf } M.
        \]
        
        \item \textbf{Assoziativität}: 
        \[
        \forall a, b, c \in M \colon (a \cdot b) \cdot c = a \cdot (b \cdot c).
        \]
        
        \item \textbf{Existenz eines neutralen Elements}:
        \[
        \exists e \in M\forall a \in M \colon e \cdot a = a \cdot e = a.
        \]
    \end{itemize}
\end{definition}

\begin{remark}
    In Monoiden wird der Operator \(\cdot\) häufig weggelassen, sodass die Operation zwischen zwei Elementen \(a\) und \(b\) einfach als \(ab := a \cdot b\) geschrieben wird. Dies dient der Vereinfachung der Notation, insbesondere in formalen Ausdrücken.
\end{remark}

\subsubsection*{Einführungsregel für Monoide (M\textsubscript{I})}
\label{rule:rMonoidI}
Die Einführungsregel für Monoide \((M, \cdot)\) ermöglicht es, ein Monoid zu definieren, indem die Menge \(M\), die binäre Operation \(\cdot\), die Assoziativität der Operation und die Existenz eines neutralen Elements gezeigt werden. Diese Regel basiert auf den Axiomen des Monoids.

\[
\begin{array}{llll}
    i       & (1) & \cdot \colon M \times M \to M & \dots \\
    j       & (2) & \forall a, b, c \in M ((ab)c = a(bc)) & \dots \\
    k       & (3) & \exists e \in M \forall a \in M(e \cdot a = a \cdot e = a) & \dots \\
    i,j,k   & (4) & (M, \cdot, e) \text{ ist ein Monoid.} & \rMonoidI{1,2,3}
\end{array}
\]

\(i\), \(j\) und \(k\) sind dabei Listen von Annahmen.

\subsubsection*{Eliminierungsregel für Monoide (M\textsubscript{E})}
\label{rule:rMonoidE}
Die Eliminierungsregel für Monoide \((M, \cdot)\) erlaubt es, aus der Tatsache, dass \((M, \cdot)\) ein Monoid ist, die Existenz der Menge \(M\), der binären Operation \(\cdot\), die Assoziativität der Operation sowie die Existenz eines neutralen Elements abzuleiten.

\[
\begin{array}{llll}
    i       & (1) & (M, \cdot, e) \text{ ist ein Monoid.} & \dots \\
    i       & (2) & \cdot \colon M \times M \to M & \rMonoidE{1} \\
    i       & (3) & \forall a, b, c \in M \colon (ab)c = a(bc) & \rMonoidE{1} \\
    i       & (4) & \exists e \in M \forall a \in M \colon e \cdot a = a \cdot e = a & \rMonoidE{1}
\end{array}
\]

\(i\) ist dabei die Liste der Annahmen.

\subsubsection*{Regel der Assoziativität}
\label{rule:rAssociativityMonoid}
Die Regel der Assoziativität ermöglicht es, aus den Elementen einer Halbgruppe die Assoziativität der binären Operation abzuleiten. Dies ist eine direkte Anwendung des Assoziativitätsaxioms der Halbgruppe.

\[
\begin{array}{llll}
         & (1) & (ab)c = a(bc) & \rAssociativityMonoid{}
\end{array}
\]

\(i\) ist dabei eine Liste von Annahmen.

\subsubsection*{Regel des neutralen Elements}
\label{rule:rNeutralElementMonoid}
Die Regel des neutralen Elements ermöglicht es, aus der Tatsache, dass \((M, \cdot, e)\) ein Monoid ist, das neutrale Element \(e \in M\) und dessen Eigenschaften abzuleiten. Dies ist eine direkte Anwendung des Axioms des neutralen Elements im Monoid. Für \(a\in M\) gilt somit:

\[
\begin{array}{llll}
        & (1) & e \cdot a = a \cdot e = a & \rNeutralElementMonoid{} \\
\end{array}
\]

\(i\) ist dabei eine Liste von Annahmen.


\subsubsection*{Eindeutigkeit des neutralen Elements}
\label{ExeweApostropheInMLpFaaInMLpeMultaEqualsaMulteEqualsaAndeApostropheMultaEqualsaMulteApostropheEqualsaRpRpImpeEqualseApostrophe}
\begin{theorem}[\(\exists e, e' \in M (\forall a \in M (e \cdot a = a \cdot e = a \land e' \cdot a = a \cdot e' = a)) \vdash e = e'\) (Eindeutigkeit des neutralen Elements)]
Sei \((M, \cdot)\) ein Monoid, das durch die Menge \(M\), die binäre Operation \(\cdot\) und ein neutrales Element \(e \in M\) definiert ist, sodass gilt:
\[
\forall a \in M \colon e \cdot a = a \cdot e = a.
\]
Wenn es ein weiteres Element \(e' \in M\) gibt, das ebenfalls die neutrale Eigenschaft besitzt, d. h.,
\[
\forall a \in M \colon e' \cdot a = a \cdot e' = a,
\]
dann folgt, dass \(e = e'\).
\end{theorem}
\begin{proof}
    Seien \(e,e'\in M\) neutrale Elemente von M. Dann gilt:
    \[
	\begin{array}{llll}
		1 &  (1) & \forall a \in M(e \cdot a = a \cdot e = a \land e' \cdot a = a \cdot e' = a)  & \rA \\
		1 &  (2) & e \cdot e = e \cdot e = e \land e' \cdot e = e \cdot e' = e & \rUE{1} \\	
            1 &  (3) & e' \cdot e = e \cdot e' = e & \rAEb{2} \\	
            1 &  (4) & e' \cdot e = e & \rIEb{3} \\	
		1 &  (5) & e \cdot e' = e' \cdot e = e' \land e' \cdot e' = e' \cdot e' = e' & \rUE{1} \\	
		1 &  (6) & e \cdot e' = e' \cdot e = e' & \rAEa{5} \\	
            1 &  (7) & e' \cdot e = e' & \rIEb{6} \\	
            1 &  (8) & e = e' & \rIE{4,7} \\	
	\end{array}
	\]
\end{proof}



\label{LpMwMultweRpInMonoidImpLpMwMultRpInSemiGroup}
\begin{theorem}[\((M,\cdot, e) \text{ ist ein Monoid} \vdash (M,\cdot) \text{ ist eine Halbgruppe}\)]
\end{theorem}
\begin{proof}
	\[
	\begin{array}{llll}
		1 &  (1) & (M, \cdot, e) \text{ ist ein Monoid.} & \rA \\
		1 &  (2) & \cdot \colon M \times M \to M & \rMonoidE{1} \\			
		1 &  (3) & \forall a, b, c \in M \colon (ab)c = a(bc) & \rMonoidE{1} \\
		1 &  (4) & (M, \cdot) \text{ ist eine Halbgruppe.} & \rSemigroupI{2,3} \\	
	\end{array}
	\]
\end{proof}

\subsection{Abelsche Monoide}

\begin{definition}[Abelsches Monoid]
    Ein \textbf{abelsches Monoid} ist ein Monoid \((M, \cdot, e)\), in dem zusätzlich zur Assoziativität und der Existenz eines neutralen Elements die Kommutativität der binären Operation \(\cdot\) gilt. Das bedeutet, dass für alle Elemente \(a, b \in M\) gilt:
    
    \begin{itemize}
        \item \textbf{Kommutativität}:
        \[
        \forall a, b \in M \colon a \cdot b = b \cdot a.
        \]
    \end{itemize}
\end{definition}

\begin{remark}
    In abelschen Monoiden wird der Operator \(\cdot\) häufig weggelassen, sodass die Operation zwischen zwei Elementen \(a\) und \(b\) einfach als \(ab := a \cdot b\) geschrieben wird. Dies dient der Vereinfachung der Notation, insbesondere in formalen Ausdrücken.
\end{remark}

\subsubsection*{Einführungsregel für abelsche Monoide (AM\textsubscript{I})}
\label{rule:rAbelianMonoidE}
Die Einführungsregel für abelsche Monoide ermöglicht es, aus der Tatsache, dass \((M, \cdot, e)\) ein abelsches Monoid ist, sowohl die Kommutativität der binären Operation \(\cdot\) als auch die Monoidstruktur abzuleiten. Das bedeutet, dass sowohl die Assoziativität als auch die Existenz eines neutralen Elements erfüllt sein müssen.

\[
\begin{array}{llll}
    i       & (1) & (M, \cdot, e) \text{ ist ein abelsches Monoid.} & \dots \\
    i       & (2) & \forall a, b \in M (ab = ba) & \rAbelianMonoidE{1}\\
    i       & (3) & (M, \cdot, e) \text{ ist ein  Monoid.} & \rAbelianMonoidE{1}
\end{array}
\]

\(i\) ist dabei eine Liste von Annahmen.


\subsubsection*{Eliminierungsregel für abelsche Monoide (AM\textsubscript{E})}
\label{rule:rAbelianMonoidI}
Die Eliminierungsregel für abelsche Monoide ermöglicht es, aus der Tatsache, dass \((M, \cdot, e)\) ein Monoid ist und die Kommutativität der Operation gezeigt wird, die abelsche Monoidstruktur abzuleiten.

\[
\begin{array}{llll}
    i       & (1) & (M, \cdot, e) \text{ ist ein Monoid.} & \dots \\
    j       & (2) & \forall a, b \in M (ab = ba) & \dots \\
    i,j     & (3) & (M, \cdot, e) \text{ ist ein abelscher Monoid.} & \rAbelianMonoidI{1,2}
\end{array}
\]

\(i\) und \(j\) sind dabei Listen von Annahmen.

\subsubsection*{Kommutativität des Monoids}
\label{rule:rCommutativeMonoid}
Die Kommutativität des Monoids ermöglicht es, in einem abelschen Monoid \((M, \cdot, e)\) die Reihenfolge der Operanden der binären Operation \(\cdot\) zu vertauschen. Dies folgt direkt aus der Definition eines abelschen Monoids, bei dem die Operation \(\cdot\) sowohl assoziativ als auch kommutativ ist.


\[
\begin{array}{llll}
          & (1) & ab = ba & \rCommutativeMonoid{} \\
\end{array}
\]

\label{LpMwMultweRpInAbelMonoidImpLpMwMultRpInAbelSemiGroup}
\begin{theorem}[\((M,\cdot, e) \text{ ist ein abelscher Monoid} \vdash (M,\cdot) \text{ ist eine abelsche Halbgruppe}\)]
\end{theorem}
\begin{proof}
	\[
	\begin{array}{llll}
		1 &  (1) & (M, \cdot, e) \text{ ist ein abelscher Monoid.} & \rA \\
		1 &  (2) & (M, \cdot, e) \text{ ist ein Monoid.} & \rAbelianMonoidE{1} \\	
		1 &  (3) & \forall a, b \in M (ab = ba) & \rAbelianMonoidE{1} \\
		1 &  (4) & (M, \cdot) \text{ ist eine Halbgruppe.} & \LpMwMultweRpInMonoidImpLpMwMultRpInSemiGroup{1} \\	
  		1 &  (5) & (M, \cdot) \text{ ist eine abelsche Halbgruppe.} & \rAbelianSemigroupI{4,3} \\	
	\end{array}
	\]
\end{proof}

\subsection{Erweiterte Vertauschungsgesetze abelscher Monoide}
Im folgenden sei \((M, \cdot, e)\) ein abelscher Monoid mit der binären Operation \(\cdot :M\times M\rightarrow M\).

\label{aInMwbInMwcInMImpLpaPlusbRpPluscEqualsLpaPluscRpPlusb}
\begin{theorem}[\(a\in M, b\in M, c\in M \vdash (ab)c=(ac)b\)]
\end{theorem}
\begin{proof}
        Seien \(a,b,c\in M\), dann gilt:
        \[
	\begin{array}{lllcll}
		  &  (1) & (ab)c &=& a(bc) & \rAssociativityMonoid{} \\
            &  (2) &             &=& a(cb) & \rCommutativeMonoid{} \\
    	&  (3) & &=&(ac)b & \rAssociativityMonoid{} \\
	\end{array}
	\]
\end{proof}

\label{MInAbelMonoidwawbwcInMImpLpabRpcEqualsLpcaRpb}
\begin{theorem}[\(a,b,c\in M \vdash (ab)c=(ca)b\)]
\end{theorem}
\begin{proof}
        Seien \(a,b,c\in M\), dann gilt:
        \[
	\begin{array}{lllcll}
		  &  (1) & (ab)c &=& (ac)b & \aInMwbInMwcInMImpLpaPlusbRpPluscEqualsLpaPluscRpPlusb{} \\
            &  (2) &             &=& (ca)b & \rCommutativeMonoid{} \\
	\end{array}
	\]
\end{proof}

\label{aInMwbInMwcInMImpaPlusLpbPluscRpEqualsLpaPluscRpPlusb}
\begin{theorem}[\(a\in M, b\in M, c\in M \vdash a(bc)=(ac)b\)]
\end{theorem}
\begin{proof}
        Seien \(a,b,c\in M\), dann gilt:
        \[
	\begin{array}{lllcll}
		  &  (1) & a(bc) &=& (ab)c & \rAssociativityMonoid{} \\
            &  (2) &             &=& (ac)b & \aInMwbInMwcInMImpLpaPlusbRpPluscEqualsLpaPluscRpPlusb{} \\
	\end{array}
	\]
\end{proof}

\label{aInMwbInMwcInMwdInMImpLpaPlusbRpPlusLpcPlusdRpEqualsLpaPluscRpPlusLpbPlusdRp}
\begin{theorem}[\(a\in M, b\in M, c\in M, d\in M \vdash (ab)(cd)=(ac)(bd)\)]
\end{theorem}
\begin{proof}
    Seien \(a,b,c,d\in M\), dann gilt:
	\[
	\begin{array}{lllcll}
		  &  (1) & (ab)(cd) &=&((ab)c)d & \rAssociativityMonoid{} \\
            &  (2) &             &=& ((ac)b)d & \aInMwbInMwcInMImpLpaPlusbRpPluscEqualsLpaPluscRpPlusb{1,2,3} \\
    	&  (3) & &=&(ac)(bd) & \rAssociativityMonoid{} \\
	\end{array}
	\]
\end{proof}


\section{Halbringe}

\begin{definition}[Halbring]
    Ein \textbf{Halbring} ist eine algebraische Struktur \((R, +, \cdot)\), wobei \(R\) eine Menge ist und \(+\) und \(\cdot\) zwei binäre Operationen auf \(R\) sind, die die folgenden Axiome erfüllen:
    
    \begin{itemize}
        \item \((R, +)\) ist eine abelsche Halbgruppe.        
        \item \((R, \cdot)\) ist eine Halbgruppe.
        \item \textbf{Linksdistributivität der Multiplikation über Addition}:
        \[
        \forall a, b, c \in R (a \cdot (b + c) = (a \cdot b) + (a \cdot c)).
        \]
        \item \textbf{Rechtsdistributivität der Multiplikation über Addition}:
        \[
        \forall a, b, c \in R ((a + b) \cdot c = (a \cdot c) + (b \cdot c)).
        \]
    \end{itemize}
\end{definition}

\begin{remark}
    In Halbringen wird die Operation \(\cdot\) oft weggelassen, sodass \(a \cdot b\) als \(ab\) geschrieben wird. Diese Vereinfachung dient der Notation und Übersichtlichkeit.
\end{remark}

\subsubsection*{Einführungsregel für Halbringe (HR\textsubscript{I})}
\label{rule:rSemiringI}
Die Einführungsregel für Halbringe \((R, +, \cdot)\) ermöglicht es, einen Halbring zu definieren, indem die Menge \(R\), die Operationen \(+\) und \(\cdot\) sowie die entsprechenden Axiome (abelsche Halbgruppe, Halbgruppe und Distributivität) gezeigt werden.

\[
\begin{array}{llll}
    i       & (1) & (R, +) \text{ ist eine abelsche Halbgruppe.} & \dots \\
    j       & (2) & (R, \cdot) \text{ ist eine Halbgruppe.} & \dots \\
    k       & (3) & \forall a, b, c \in R (a(b + c)=ab+ac) & \dots \\
    l       & (4) & \forall a, b, c \in R ((a + b)c=ac+bc) & \dots \\
    i,j,k,l   & (5) & (R, +, \cdot) \text{ ist ein Halbring.} & \rSemiringI{1,2,3,4}
\end{array}
\]

\(i\), \(j\), \(k\) und \(l\) sind Listen von Annahmen.

\subsubsection*{Eliminierungsregel für Halbringe (HR\textsubscript{E})}
\label{rule:rSemiringE}
Die Eliminierungsregel für Halbringe erlaubt es, aus der Tatsache, dass \((R, +, \cdot)\) ein Halbring ist, die abelsche Halbgruppe \((R, +)\), die Halbgruppe \((R, \cdot)\) sowie die Distributivität der Multiplikation über Addition abzuleiten.

\[
\begin{array}{llll}
    i       & (1) & (R, +, \cdot) \text{ ist ein Halbring.} & \dots \\
    i       & (2) & (R, +) \text{ ist eine abelsche Halbgruppe.} & \rSemiringE{1} \\
    i       & (3) & (R, \cdot) \text{ ist eine Halbgruppe.} & \rSemiringE{1} \\
    i       & (4) & \forall a, b, c \in R (a(b + c)=ab+ac) & \rSemiringE{1} \\
    i       & (5) & \forall a, b, c \in R ((a + b)c=ac+bc) & \rSemiringE{1}
\end{array}
\]

\(i\) ist dabei die Liste der Annahmen.

\subsubsection*{Regel der Linksdistributivität}
\label{rule:rLeftDistributiveSemigroup}
Die Regel der Linksdistributivität ermöglicht es, die Eigenschaft der Linksdistributivität der Multiplikation über Addition in Halbringen zu nutzen. Diese Regel basiert auf dem Axiom der Linksdistributivität in Halbringen. Für Elemente \(a,b,c\in R\) gilt somit:

\[
\begin{array}{llll}
     & (1) & a(b + c) = ab+ac & \rLeftDistributiveSemigroup{}
\end{array}
\]


\subsubsection*{Regel der Rechtsdistributivität}
\label{rule:rRightDistributiveSemigroup}
Die Regel der Rechtsdistributivität ermöglicht es, die Eigenschaft der Rechtsdistributivität der Multiplikation über Addition in Halbringen zu nutzen. Diese Regel basiert auf dem Axiom der Rechtsdistributivität in Halbringen.

\[
\begin{array}{llll}
      & (1) & (a + b)c = ac+bc & \rRightDistributiveSemigroup{}
\end{array}
\]


\subsection{Abelsche Halbringe}

\begin{definition}[Abelscher Halbring]
    Ein \textbf{abelscher Halbring} ist ein Halbring \((R, +, \cdot)\), wobei die folgenden Bedingungen erfüllt sind:
    
    \begin{itemize}
        \item \((R, +)\) ist eine abelsche Halbgruppe.
        \item \((R, \cdot)\) ist eine abelsche Halbgruppe.
        \item \textbf{Linksdistributivität der Multiplikation über Addition}:
        \[
        \forall a, b, c \in R \colon a \cdot (b + c) = (a \cdot b) + (a \cdot c).
        \]
        \item \textbf{Rechtsdistributivität der Multiplikation über Addition}:
        \[
        \forall a, b, c \in R \colon (a + b) \cdot c = (a \cdot c) + (b \cdot c).
        \]
    \end{itemize}
\end{definition}

\begin{remark}
    In abelschen Halbringen wird der Operator \(\cdot\) häufig weggelassen, sodass die Multiplikation zwischen zwei Elementen \(a\) und \(b\) einfach als \(ab := a \cdot b\) geschrieben wird. Diese Vereinfachung dient der Notation und Übersichtlichkeit.
\end{remark}

\subsubsection*{Einführungsregel für abelsche Halbringe (AHR\textsubscript{I})}
\label{rule:rAbelianSemiringI}
Die Einführungsregel für abelsche Halbringe \((R, +, \cdot)\) ermöglicht es, einen abelschen Halbring zu definieren, indem die Menge \(R\), die Operationen \(+\) und \(\cdot\), die Distributivität sowie die abelschen Halbgruppenstrukturen gezeigt werden.

\[
\begin{array}{llll}
    i       & (1) & (R, +) \text{ ist eine abelsche Halbgruppe.} & \dots \\
    j       & (2) & (R, \cdot) \text{ ist eine abelsche Halbgruppe.} & \dots \\
    k       & (3) & \forall a, b, c \in R \colon a(b + c) = ab + ac & \dots \\
    l       & (4) & \forall a, b, c \in R \colon (a + b)c = ac + bc & \dots \\
    i,j,k,l & (5) & (R, +, \cdot) \text{ ist ein abelscher Halbring.} & \rAbelianSemiringI{1,2,3,4}
\end{array}
\]

\subsubsection*{Eliminierungsregel für abelsche Halbringe (AHR\textsubscript{E})}
\label{rule:rAbelianSemiringE}
Die Eliminierungsregel für abelsche Halbringe erlaubt es, aus der Tatsache, dass \((R, +, \cdot)\) ein abelscher Halbring ist, die abelschen Halbgruppenstrukturen für \(+\) und \(\cdot\) sowie die Distributivität der Multiplikation über Addition abzuleiten.

\[
\begin{array}{llll}
    i       & (1) & (R, +, \cdot) \text{ ist ein abelscher Halbring.} & \dots \\
    i       & (2) & (R, +) \text{ ist eine abelsche Halbgruppe.} & \rAbelianSemiringE{1} \\
    i       & (3) & (R, \cdot) \text{ ist eine abelsche Halbgruppe.} & \rAbelianSemiringE{1} \\
    i       & (4) & \forall a, b, c \in R \colon a(b + c) = ab + ac & \rAbelianSemiringE{1} \\
    i       & (5) & \forall a, b, c \in R \colon (a + b)c = ac + bc & \rAbelianSemiringE{1}
\end{array}
\]

\subsubsection*{Regel der Linksdistributivität}
\label{rule:rLeftDistributiveAbelianSemigroup}
Die Regel der Linksdistributivität ermöglicht es, die Eigenschaft der Linksdistributivität der Multiplikation über Addition in abelschen Halbringen zu nutzen. Diese Regel basiert auf dem Axiom der Linksdistributivität in abelschen Halbringen.

\[
\begin{array}{llll}
      & (1) & a(b + c) = ab+ac & \rLeftDistributiveAbelianSemigroup{}
\end{array}
\]
\subsubsection*{Regel der Rechtsdistributivität}
\label{rule:rRightDistributiveAbelianSemigroup}
Die Regel der Rechtsdistributivität ermöglicht es, die Eigenschaft der Rechtsdistributivität der Multiplikation über Addition in abelschen Halbringen zu nutzen. Diese Regel basiert auf dem Axiom der Rechtsdistributivität in abelschen Halbringen.

\[
\begin{array}{llll}
       & (1) & (a + b)c = ac+bc & \rRightDistributiveAbelianSemigroup{}
\end{array}
\]



\subsection{Potenzen in Halbgruppen und Monoiden}

\begin{definition}[Potenz eines Elements in einer Halbgruppe]
    Sei \((S, \cdot)\) eine Halbgruppe mit einer \textbf{multiplikativen} Operation \(\cdot\), und \(a \in S\) ein beliebiges Element. Für \(n \in \mathbb{N}\) definieren wir die \textbf{Potenz} \(a^n\) des Elements \(a\) induktiv durch:
    \[
    a^0 := 0,
    a^1 := a,
    \]
    und für \(n > 1\):
    \[
    a^{n+1} := a^n \cdot a.
    \]
\end{definition}

\begin{remark}
    Die Potenzierung \(a^n\) setzt voraus, dass die Operation \(\cdot\) assoziativ ist, wie es in Halbgruppen und Monoiden gefordert ist. Die Definition lässt sich in Monoiden erweitern, indem man \(a^0 := e\) setzt, wobei \(e\) das neutrale Element des Monoids ist. In abelschen Strukturen wie abelschen Monoiden oder abelschen Gruppen ist die Reihenfolge der Multiplikation irrelevant, da die Operation kommutativ ist.
\end{remark}

\subsubsection*{Einführungsregel für Potenzen}
\label{rule:rPowerI}
Die Einführungsregel für Potenzen eines Elements in einer Halbgruppe erlaubt es, Potenzen \(a^n\) induktiv zu definieren, indem die Basis \(a\) und die rekursive Definition der Potenzierung verwendet werden.

\[
\begin{array}{llll}
    i   & (1) & a \in S & \dots \\
    i   & (2) & a^n = a^{n-1} \cdot a & \rPowerI{1} \\
    i   & (3) & a^n\in S & \rPowerI{1} \\
\end{array}
\]

\[
\begin{array}{llll}
    i   & (1) & a \in S & \dots \\
    i   & (2) & 0 = a^{0} & \rPowerI{1} \\
    i   & (3) & a^{0}\in S & \rPowerI{1} \\
\end{array}
\]

\[
\begin{array}{llll}
    i   & (1) & a \in S & \dots \\
    i   & (2) & a = a^{1} & \rPowerI{1} \\
    i   & (3) & a^{1}\in S & \rPowerI{1} \\
\end{array}
\]

\(i\) ist die Liste der Annahmen.

\chapter{Einführung in Relationen}

\section{Binäre Relationen}

\begin{definition}[Binäre Relation]
    Eine \textbf{binäre Relation} auf einer Menge \(S\) ist eine Teilmenge \(R \subseteq S \times S\). Man schreibt \(a \, R \, b\), wenn \((a,b) \in R\).
\end{definition}

\begin{remark}
    Eine binäre Relation kann als eine Verknüpfung zwischen Elementen einer Menge \(S\) betrachtet werden. Es wird geschrieben \(a \, R \, b\), wenn \(a\) in Relation zu \(b\) steht.
\end{remark}

\subsubsection*{Einführungsregel für binäre Relationen}
\label{rule:rBinaryRelationI}
Die Einführungsregel für eine binäre Relation \(R\) ermöglicht es, eine Relation auf einer Menge \(S\) zu definieren, indem \(R\) als Teilmenge von \(S \times S\) spezifiziert wird.

\[
\begin{array}{llll}
    i       & (1) & R \subseteq S \times S & \dots \\
    i       & (2) & R \text{ ist eine binäre Relation auf } S & \rBinaryRelationI{1}
\end{array}
\]

\(i\) ist dabei die Liste der Annahmen.

\subsubsection*{Eliminierungsregel für binäre Relationen}
\label{rule:rBinaryRelationE}
Die Eliminierungsregel für binäre Relationen ermöglicht es, aus der Tatsache, dass \(R \subseteq S \times S\) eine Relation auf \(S\) ist, abzuleiten, dass für \(a, b \in S\), wenn \((a,b) \in R\), dann steht \(a\) in Relation zu \(b\).

\[
\begin{array}{llll}
    i       & (1) & R \text{ ist eine binäre Relation auf } S & \dots \\
    i,j     & (2) & R \subseteq S \times S & \rBinaryRelationE{1}
\end{array}
\]

\(i\) und \(j\) sind dabei Listen von Annahmen.



\section{Äquivalenzrelationen}

\begin{definition}[Äquivalenzrelation]
    Sei \(S\) eine Menge. Der \textbf{Begriff der Äquivalenzrelation} auf \(S\) wird durch das \textbf{Symbol} \(\sim\) \textbf{implizit definiert}. Dabei gelten die folgenden Axiome:
    
    \begin{itemize}
        \item \textbf{Reflexivität}:
        \[
        \forall a \in S (a \sim a).
        \]
        
        \item \textbf{Symmetrie}: 
        \[
        \forall a, b \in S (a \sim b \rightarrow b \sim a).
        \]
        
        \item \textbf{Transitivität}: 
        \[
        \forall a, b, c \in S ((a \sim b \land b \sim c) \rightarrow a \sim c).
        \]
    \end{itemize}
\end{definition}

\begin{remark}
    Äquivalenzrelationen auf einer Menge \(S\) teilen die Menge in disjunkte Äquivalenzklassen auf, wobei jede Klasse durch die Relation \(\sim\) definiert wird.
\end{remark}

\subsubsection*{Einführungsregel für Äquivalenzrelationen}
\label{rule:rEquivalenceRelationI}
Die Einführungsregel für Äquivalenzrelationen \(\sim\) ermöglicht es, eine Relation als Äquivalenzrelation zu definieren, indem Reflexivität, Symmetrie und Transitivität nachgewiesen werden. Diese Regel basiert auf den Axiomen der Äquivalenzrelation.

\[
\begin{array}{llll}
    i       & (1) & \forall a \in S (a \sim a) & \dots \\
    j       & (2) & \forall a, b \in S (a \sim b \rightarrow b \sim a) & \dots \\
    k       & (3) & \forall a, b, c \in S ((a \sim b \land b \sim c) \rightarrow a \sim c) & \dots \\
    i,j,k   & (4) & \sim \text{ ist eine Äquivalenzrelation auf } S & \rEquivalenceRelationI{1,2,3}
\end{array}
\]

\(i\), \(j\), und \(k\) sind dabei Listen von Annahmen.

\subsubsection*{Eliminierungsregel für Äquivalenzrelationen}
\label{rule:rEquivalenceRelationE}
Die Eliminierungsregel für Äquivalenzrelationen \(\sim\) erlaubt es, aus der Tatsache, dass \(\sim\) eine Äquivalenzrelation ist, die Reflexivität, Symmetrie und Transitivität abzuleiten.

\[
\begin{array}{llll}
    i       & (1) & \sim \text{ ist eine Äquivalenzrelation auf } S & \dots \\
    i       & (2) & \forall a \in S (a \sim a) & \rEquivalenceRelationE{1} \\
    i       & (3) & \forall a, b \in S (a \sim b \rightarrow b \sim a) & \rEquivalenceRelationE{1} \\
    i       & (4) & \forall a, b, c \in S ((a \sim b \land b \sim c) \rightarrow a \sim c) & \rEquivalenceRelationE{1}
\end{array}
\]

\(i\) ist dabei die Liste der Annahmen.

\subsubsection*{Einführungsregel für Reflexivität}
\label{rule:rReflexivityEqRI}
Die Einführungsregel für Reflexivität ermöglicht es, die Reflexivität einer Relation \(\sim\) zu zeigen, indem bewiesen wird, dass \(a \sim a\) für alle \(a \in S\) gilt.

\[
\begin{array}{llll}
      & (1) & a \sim a & \rReflexivityEqRI{}
\end{array}
\]

\subsubsection*{Einführungsregel für Symmetrie}
\label{rule:rSymmetryEqRI}
Die Einführungsregel für Symmetrie ermöglicht es, die Symmetrie einer Relation \(\sim\) zu zeigen, indem bewiesen wird, dass \(a \sim b \rightarrow b \sim a\) für alle \(a, b \in S\) gilt.

\[
\begin{array}{llll}
    i       & (1) & a \sim b & \dots \\
    i       & (2) & b \sim a & \rSymmetryEqRI{1}
\end{array}
\]

\(i\) ist dabei die Liste der Annahmen.

\subsubsection*{Einführungsregel für Transitivität}
\label{rule:rTransitivityEqRI}
Die Einführungsregel für Transitivität ermöglicht es, die Transitivität einer Relation \(\sim\) zu zeigen, indem bewiesen wird, dass \(a \sim b \land b \sim c \rightarrow a \sim c\) für alle \(a, b, c \in S\).

\[
\begin{array}{llll}
    i           & (1) & a \sim b & \dots \\
    j           & (2) & b \sim c & \dots \\
    i,j         & (3) & a \sim c & \rTransitivityEqRI{1,2}
\end{array}
\]

\(i\) und \(j\) sind dabei Listen von Annahmen.

\label{FaSLpEqualsInEquivalencerelationSRp}
\begin{theorem}[\(\forall S(= \text{ ist eine Äquivalenzrelation auf } S)\)]
\end{theorem}
\begin{proof}
    Sei \(S\) eine beliebige Menge. Dann gilt für alle \(a,b,c\in S\):
	\[
	\begin{array}{llll}
		    & (1) & a \sim a & \rII{}\\
                & (2) & a=b\rightarrow b=a & \aIdbImpbIda{}\\
		    & (3) & a=b\land b=c\rightarrow a=c & \aIdbwbIdcImpaIdc{}\\
                & (4) & = \text{ ist eine Äquivalenzrelation auf } S & \rEquivalenceRelationI{1,2,3}\\
	\end{array}
	\]
\end{proof}


\subsection{Kettennotation für Äquivalenzrelationen}

Die zuvor eingeführte Gleichheitsketten-Notation (siehe Abschnitt \ref{Gleichheitsketten-Notation}) lässt sich analog auf Äquivalenzrelationen anwenden. Dabei wird das Symbol \(=\) durch \(\sim\) ersetzt, und die Kettennotation basiert auf der Transitivität der Äquivalenzrelation:
\[
a \sim b, \, b \sim c \vdash a \sim c.
\]

Für die Darstellung verwenden wir die gleiche Struktur, wobei jede Transformation in der Kette durch die entsprechende Regel begründet wird:
\[
a \sim b \stackrel{\text{Regel}_1}{\sim} c \stackrel{\text{Regel}_2}{\sim} d.
\]

Ebenso ist die tabellarische Notation möglich:
\[
\begin{array}{llclll}
	1 & (1) & a & \sim & b & \rA \\
	1 & (2) &   & \sim & c & \text{Regel}_1 \\
	1 & (3) &   & \sim & d & \text{Regel}_2 \\
        1 & (4) & a & \sim & d & \rTransitivityEqRI{1,3} \\
\end{array}
\]

Aufgrund \(\rSymmetryEqRI{}\) können wir auch eine tabellarische Notation der folgenden Form anwenden:
\[
\begin{array}{llclll}
	1 & (1) & a & \sim & b & \rA \\
	1 & (2) &   & \sim & c & \text{Regel}_1 \\
	1 & (3) &   & \sim & d & \text{Regel}_2 \\
        1 & (4) & d & \sim & a & \rTransitivityEqRI{1,3} \\
\end{array}
\]

\begin{remark}
    Durch die Anwendung der bereits beschriebenen Kettennotation auf Äquivalenzrelationen können Beweise übersichtlicher dargestellt werden, ohne die mathematische Präzision zu verlieren.
\end{remark}

\section{Ordnungsrelationen}

\subsection{Halbordnungen}

\begin{definition}[Halbordnung]
    Eine \textbf{Halbordnung} auf einer Menge \( S \) ist eine binäre Relation, die die folgenden Axiome erfüllt:

    \begin{itemize}
        \item \textbf{Reflexivität}:
        \[
        \forall a \in S \, (a \leq a).
        \]
        
        \item \textbf{Antisymmetrie}: 
        \[
        \forall a, b \in S \, \big( (a \leq b \land b \leq a) \rightarrow a = b \big).
        \]
        
        \item \textbf{Transitivität}: 
        \[
        \forall a, b, c \in S \, \big( (a \leq b \land b \leq c) \rightarrow a \leq c \big).
        \]
    \end{itemize}

    \(\geq\) wird als durch \(\leq\) wie folgt definiert:
    \[
    a \geq b \coloneqq b \leq a.
    \]
\end{definition}

\begin{remark}
    Halbordnungen auf einer Menge \( S \) erlauben die Vergleichbarkeit der Elemente dieser Menge nach bestimmten Kriterien, wobei die genannten Axiome die Grundlage für diesen Vergleich bilden. Im Gegensatz zu Totalordnungen ist es bei Halbordnungen nicht erforderlich, dass jedes Paar von Elementen vergleichbar ist. Die Relation \(\geq\) bietet eine alternative Perspektive auf dieselbe Halbordnung, indem sie die Richtung der Vergleichsrelation umkehrt.
\end{remark}

\subsubsection*{Einführungsregel für Halbordnungen}
\label{rule:rPartialOrderRelationI}
Die Einführungsregel für Halbordnungen \(\leq\) ermöglicht es, eine Relation als Halbordnung zu definieren, indem Reflexivität, Antisymmetrie und Transitivität nachgewiesen werden. Diese Regel basiert auf den Axiomen der Halbordnung.

\[
\begin{array}{llll}
    i       & (1) & \forall a \in S (a \leq a) & \dots \\
    j       & (2) & \forall a, b \in S ((a \leq b \land b \leq a) \rightarrow a = b) & \dots \\
    k       & (3) & \forall a, b, c \in S ((a \leq b \land b \leq c) \rightarrow a \leq c) & \dots \\
    i,j,k   & (4) & \leq \text{ ist eine Halbordnung auf } S & \rPartialOrderRelationI{1,2,3}
\end{array}
\]

\(i\), \(j\), und \(k\) sind dabei Listen von Annahmen.

\subsubsection*{Eliminierungsregel für Halbordnungen}
\label{rule:rPartialOrderRelationE}
Die Eliminierungsregel für Halbordnungen \(\leq\) erlaubt es, aus der Tatsache, dass \(\leq\) eine Halbordnung ist, die Reflexivität, Antisymmetrie und Transitivität abzuleiten.

\[
\begin{array}{llll}
    i       & (1) & \leq \text{ ist eine Halbordnung auf } S & \dots \\
    i       & (2) & \forall a \in S (a \leq a) & \rPartialOrderRelationE{1} \\
    i       & (3) & \forall a, b \in S ((a \leq b \land b \leq a) \rightarrow a = b) & \rPartialOrderRelationE{1} \\
    i       & (4) & \forall a, b, c \in S ((a \leq b \land b \leq c) \rightarrow a \leq c) & \rPartialOrderRelationE{1}
\end{array}
\]

\(i\) ist dabei die Liste der Annahmen.

\subsubsection*{Einführungsregel für Reflexivität}
\label{rule:rReflexivityOrdRI}
Die Einführungsregel für Reflexivität ermöglicht es, die Reflexivität einer Halbordnung \(\leq\) zu zeigen, indem bewiesen wird, dass \(a \leq a\) für alle \(a \in S\) gilt.

\[
\begin{array}{llll}
          & (1) & a \leq a & \rReflexivityOrdRI{}
\end{array}
\]

\subsubsection*{Einführungsregel für Antisymmetrie}
\label{rule:rAntisymmetryOrdRI}
Die Einführungsregel für Antisymmetrie ermöglicht es, die Antisymmetrie einer Halbordnung \(\leq\) zu zeigen, indem bewiesen wird, dass \(a \leq b \land b \leq a \rightarrow a = b\) für alle \(a, b \in S\).

\[
\begin{array}{llll}
    i       & (1) & a \leq b & \dots \\
    j       & (2) & b \leq a & \dots \\
    i,j     & (3) & a = b & \rAntisymmetryOrdRI{1,2}
\end{array}
\]

\(i,j,k\) und \(l\) ist dabei die Liste der Annahmen.

\subsubsection*{Einführungsregel für Transitivität}
\label{rule:rTransitivityOrdRI}
Die Einführungsregel für Transitivität ermöglicht es, die Transitivität einer Halbordnung \(\leq\) zu zeigen, indem bewiesen wird, dass \(a \leq b \land b \leq c \rightarrow a \leq c\) für alle \(a, b, c \in S\).

\[
\begin{array}{llll}
    i           & (1) & a \leq b & \dots \\
    j           & (2) & b \leq c & \dots \\
    i,j         & (3) & a \leq c & \rTransitivityOrdRI{1,2}
\end{array}
\]

\(i\) und \(j\) sind dabei Listen von Annahmen.

\paragraph{Einführungsregel für \(\geq\)}
\label{rule:rgeqI}
Die Einführungsregel für \(\geq\) (\(\geq I\)) besagt, dass, wenn \(b \leq a\) bewiesen ist, daraus direkt \(a \geq b\) gefolgert werden kann. Dies ergibt sich unmittelbar aus der Definition von \(\geq\).

\[
\begin{array}{llll}
    i & (1) & b \leq a & \text{...} \\
    i & (2) & a \geq b & \rgeqI{1} \\
\end{array}
\]

\paragraph{Eliminierungsregel für \(\geq\) }
\label{rule:rgeqE}
Die Eliminierungsregel für \(\geq\) (\(\geq E\)) besagt, dass, wenn \(a \geq b\) bewiesen ist, daraus \(b \leq a\) gefolgert werden kann. Auch dies ergibt sich direkt aus der Definition von \(\geq\).

\[
\begin{array}{llll}
    i & (1) & a \geq b & \text{...} \\
    i & (2) & b \leq a & \rgeqE{1} \\
\end{array}
\]
\(i\) ist dabei eine Liste von Annahmen.

\paragraph{Bemerkung}

Die Regeln \(\geq I\) und \(\geq E\) basieren auf der Definition von \(\geq\) und nutzen die Umkehrung der Relation \(\leq\). Sie können in Beweisen verwendet werden, um die Äquivalenz dieser beiden Relationen auszudrücken.

\subsection{Totale Ordnung}

\begin{definition}[Totale Ordnung]
    Eine \textbf{totale Ordnung} auf einer Menge \(S\) ist eine Halbordnung \(\leq\) auf \(S\), die zusätzlich das Axiom der Totalität erfüllt:
    
    \[
    \forall a, b \in S (a \leq b \lor b \leq a).
    \]
\end{definition}

\begin{remark}
    Eine totale Ordnung stellt sicher, dass jedes Paar von Elementen der Menge \(S\) vergleichbar ist. Das bedeutet, dass für alle \(a, b \in S\) entweder \(a \leq b\) oder \(b \leq a\) gilt. Sie erweitert damit den Begriff der Ordnungsrelation um das Axiom der Totalität.
\end{remark}

\subsubsection*{Einführungsregel für totale Ordnung}
\label{rule:rTotalOrderI}
Die Einführungsregel für totale Ordnungen \(\leq\) besagt, dass eine Relation als totale Ordnung definiert werden kann, wenn gezeigt wird, dass es sich um eine Ordnungsrelation handelt und das Axiom der Totalität erfüllt ist.

\[
\begin{array}{llll}
    i       & (1) & \leq \text{ ist eine Halbordnung auf } S & \dots \\
    j       & (2) & \forall a, b \in S (a \leq b \lor b \leq a) & \dots \\
    i,j     & (3) & \leq \text{ ist eine totale Ordnung auf } S & \rTotalOrderI{1,2}
\end{array}
\]

\(i\) und \(j\) sind dabei Listen von Annahmen.

\subsubsection*{Eliminierungsregel für totale Ordnung}
\label{rule:rTotalOrderE}
Die Eliminierungsregel für totale Ordnungen \(\leq\) besagt, dass aus der Annahme, dass \(\leq\) eine totale Ordnung ist, das Axiom der Totalität abgeleitet werden kann, zusammen mit den Eigenschaften der zugrunde liegenden Ordnungsrelation.

\[
\begin{array}{llll}
    i       & (1) & \leq \text{ ist eine totale Ordnung auf } S & \dots \\
    i       & (2) & \leq \text{ ist eine Halbordnung auf } S & \rTotalOrderE{1} \\
    i       & (3) & \forall a, b \in S (a \leq b \lor b \leq a) & \rTotalOrderE{1}
\end{array}
\]

\(i\) ist dabei die Liste der Annahmen.

\subsubsection*{Einführungsregel für Totalität}
\label{rule:rTotalityOrdRI}
Die Einführungsregel für Totalität ermöglicht es, die Totalität einer Ordnungsrelation \(\leq\) zu zeigen, indem bewiesen wird, dass für alle \(a, b \in S\) entweder \(a \leq b\) oder \(b \leq a\) gilt.

\[
\begin{array}{llll}
           & (1) & a \leq b \lor b \leq a & \rTotalityOrdRI{}
\end{array}
\]

\subsection{Strikte Ordnung}

\begin{definition}[Strikte Ordnung]
    Eine \textbf{strikte Ordnung} auf einer Menge \( S \) ist eine binäre Relation, die die folgenden Axiome erfüllt:
    
    \begin{itemize}
        \item \textbf{Irreflexivität}:
        \[
        \forall a \in S \, \neg (a < a).
        \]
        
        \item \textbf{Transitivität}: 
        \[
        \forall a, b, c \in S \, \big( (a < b \land b < c) \rightarrow a < c \big).
        \]
    \end{itemize}

    Die \textbf{geordnete Relation} \(>\) wird als die \textbf{duale Relation} zu \(<\) definiert durch:
    \[
    a > b \coloneqq b < a.
    \]
\end{definition}

\begin{remark}
    Strikte Ordnungen auf einer Menge \( S \) erlauben eine strikte Vergleichbarkeit der Elemente dieser Menge, wobei die Irreflexivität sicherstellt, dass kein Element sich selbst in Relation steht. Strikte Ordnungen bilden die Grundlage für viele Anwendungen in der algebraischen Strukturtheorie. Die Relation \(>\) bietet eine alternative Perspektive auf dieselbe strikte Ordnung, indem sie die Richtung der Vergleichsrelation umkehrt.
\end{remark}

\subsubsection*{Einführungsregel für Strikte Ordnungen}
\label{rule:rStrictOrderRelationI}
Die Einführungsregel für strikte Ordnungen \(<\) ermöglicht es, eine Relation als strikte Ordnung zu definieren, indem Irreflexivität, Transitivität und Totalität nachgewiesen werden.

\[
\begin{array}{llll}
    i       & (1) & \forall a \in S \, \neg (a < a) & \dots \\
    j       & (2) & \forall a, b, c \in S ((a < b \land b < c) \rightarrow a < c) & \dots \\
    i,j     & (3) & < \text{ ist eine strikte Ordnung auf } S & \rStrictOrderRelationI{1,2}
\end{array}
\]

\(i\) und \(j\) sind dabei Listen von Annahmen.

\subsubsection*{Eliminierungsregel für Strikte Ordnungen}
\label{rule:rStrictOrderRelationE}
Die Eliminierungsregel für strikte Ordnungen \(<\) erlaubt es, aus der Tatsache, dass \(<\) eine strikte Ordnung ist, die Irreflexivität, Transitivität und Totalität abzuleiten.

\[
\begin{array}{llll}
    i       & (1) & < \text{ ist eine strikte Ordnung auf } S & \dots \\
    i       & (2) & \forall a \in S \, \neg (a < a) & \rStrictOrderRelationE{1} \\
    i       & (3) & \forall a, b, c \in S ((a < b \land b < c) \rightarrow a < c) & \rStrictOrderRelationE{1} \\
\end{array}
\]

\(i\) ist dabei die Liste der Annahmen.

\subsubsection*{Einführungsregel für Irreflexivität}
\label{rule:rIrreflexivityStrictRI}
Die Einführungsregel für Irreflexivität ermöglicht es, die Irreflexivität einer strikten Ordnung \(<\) zu zeigen, indem bewiesen wird, dass \( \neg (a < a) \) für alle \(a \in S\) gilt.

\[
\begin{array}{llll}
    i       & (1) & < \text{ ist eine strikte Ordnung auf } S & \dots \\
    j       & (2) & a \in S & \dots \\
    i,j     & (3) & \neg (a < a) & \rIrreflexivityStrictRI{1,2}
\end{array}
\]

\(i\) und \(j\) ist dabei die Liste der Annahmen.

\subsubsection*{Einführungsregel für Transitivität}
\label{rule:rTransitivityStrictRI}
Die Einführungsregel für Transitivität ermöglicht es, die Transitivität einer strikten Ordnung \(<\) zu zeigen, indem bewiesen wird, dass \(a < b \land b < c \rightarrow a < c\) für alle \(a, b, c \in S\).

\[
\begin{array}{llll}
    i           & (1) & a < b & \dots \\
    j           & (2) & b < c & \dots \\
    i,j         & (3) & a < c & \rTransitivityStrictRI{1,2}
\end{array}
\]

\(i\) und \(j\) sind dabei Listen von Annahmen.

\paragraph{Einführungsregel für \(>\)}
\label{rule:rgtI}
Die Einführungsregel für \(>\) (\(> I\)) besagt, dass, wenn \(b < a\) bewiesen ist, daraus direkt \(a > b\) gefolgert werden kann. Dies ergibt sich unmittelbar aus der Definition von \(>\).

\[
\begin{array}{llll}
    i & (1) & b < a & \text{...} \\
    i & (2) & a > b & \rgtI{1} \\
\end{array}
\]

Hierbei ist \(i\) eine Liste von Annahmen, die zur Herleitung der entsprechenden Zeilen verwendet wird.

\paragraph{Eliminierungsregel für \(>\)}
\label{rule:rgtE}
Die Eliminierungsregel für \(>\) (\(> E\)) besagt, dass, wenn \(a > b\) bewiesen ist, daraus \(b < a\) gefolgert werden kann. Auch dies ergibt sich direkt aus der Definition von \(>\).

\[
\begin{array}{llll}
    i & (1) & a > b & \text{...} \\
    i & (2) & b < a & \rgtE{1} \\
\end{array}
\]

\(i\) ist dabei eine Liste von Annahmen.

\label{aGneqbImpaGeqb}
\begin{theorem}[\(a > b\vdash a\geq b\)]
Sei \(<\) eine strikte Ordnung auf \(S\) und \(a,b\in  S\), dann gilt:
\[a<b\vdash \neg(b<a)\]
\end{theorem}
\begin{proof}
    Sei \(<\) eine strikte Ordnung auf \(S\) und \(a,b\in  S\), dann gilt:
	\[
	\begin{array}{llll}
		1   & (1) & a > b & \rA\\
            1   & (2) & b<a & \rgtE{1}\\
		1   & (3) & b\leq a & \InducedStrictOrderE{2}\\
            1   & (4) & a\geq b & \rgeqI{3}\\
	\end{array}
	\]
\end{proof}



\label{aLneqbImpnLpbLneqaRp}
\begin{theorem}[\(a<b\vdash \neg(b<a)\) (Asymmetrische strikte Ordnungen)]
Sei \(<\) eine strikte Ordnung auf \(S\) und \(a,b\in  S\), dann gilt:
\[a<b\vdash \neg(b<a)\]
\end{theorem}
\begin{proof}
    Sei \(<\) eine strikte Ordnung auf \(S\) und \(a,b\in  S\), dann gilt:
	\[
	\begin{array}{llll}
		1   & (1) & a<b & \rA\\
        2   & (2) & b<a & \rA\\
		1,2 & (3) & a<a & \rTransitivityStrictRI{1,2}\\
            & (4) & a<a & \rIrreflexivityStrictRI{}\\
        1,2 & (5) & \bot & \rBI{3,4}\\
        1   & (6) & \neg(b<a) & \rCI{2,5}\\
	\end{array}
	\]
\end{proof}

\label{aEqualsbImpnLpaLneqbRp}
\begin{theorem}[\(a=b\vdash \neg(a<b)\)]
Sei \(<\) eine strikte Ordnung auf \(S\) und \(a,b\in  S\), dann gilt:
\[a=b\vdash \neg(a<b)\]
\end{theorem}
\begin{proof}
    Sei \(<\) eine strikte Ordnung auf \(S\) und \(a,b\in  S\), dann gilt:
	\[
	\begin{array}{llll}
		1   & (1) & a=b & \rA\\
            & (2) & \neg(a<a) & \rReflexivityOrdRI{}\\
		1   & (3) & \neg(a<b) & \rIE{1,2}\\
	\end{array}
	\]
\end{proof}

\label{aEqualsbImpnLpbLneqaRp}
\begin{theorem}[\(a=b\vdash \neg(b<a)\)]
Sei \(<\) eine strikte Ordnung auf \(S\) und \(a,b\in  S\), dann gilt:
\[a=b\vdash \neg(b<a)\]
\end{theorem}
\begin{proof}
    Sei \(<\) eine strikte Ordnung auf \(S\) und \(a,b\in  S\), dann gilt:
	\[
	\begin{array}{llll}
		1   & (1) & a=b & \rA\\
            & (2) & \neg(a<a) & \rReflexivityOrdRI{}\\
		1   & (3) & \neg(b<a) & \rIE{1,2}\\
	\end{array}
	\]
\end{proof}

\subsection{Totale Strikte Ordnung}

\begin{definition}[Totale Strikte Ordnung]
    Eine \textbf{totale strikte Ordnung} ist eine strikte Ordnung \(<\) auf einer Menge \(S\), die zusätzlich das Axiom der Totalität erfüllt:
    
    \[
    \forall a, b \in S (a < b \lor b < a).
    \]
\end{definition}

\begin{remark}
    Eine totale strikte Ordnung stellt sicher, dass jedes Paar von Elementen der Menge \(S\) strikt vergleichbar ist. Das bedeutet, dass für alle \(a, b \in S\) entweder \(a < b\) oder \(b < a\) gilt. Sie erweitert den Begriff der strikten Ordnung um das Axiom der Totalität.
\end{remark}

\subsubsection*{Einführungsregel für totale strikte Ordnung}
\label{rule:rTotalStrictOrderI}
Die Einführungsregel für totale strikte Ordnungen \(<\) besagt, dass eine Relation als totale strikte Ordnung definiert werden kann, wenn gezeigt wird, dass es sich um eine strikte Ordnung handelt und das Axiom der Totalität erfüllt ist.

\[
\begin{array}{llll}
    i       & (1) & < \text{ ist eine strikte Ordnung} & \dots \\
    j       & (2) & \forall a, b \in S (a < b \lor b < a) & \dots \\
    i,j     & (3) & < \text{ ist eine totale strikte Ordnung.} & \rTotalStrictOrderI{1,2}
\end{array}
\]

\(i\) und \(j\) sind dabei Listen von Annahmen.

\subsubsection*{Eliminierungsregel für totale strikte Ordnung}
\label{rule:rTotalStrictOrderE}
Die Eliminierungsregel für totale strikte Ordnungen \(<\) besagt, dass aus der Annahme, dass \(<\) eine totale strikte Ordnung ist, das Axiom der Totalität abgeleitet werden kann, zusammen mit den Eigenschaften der zugrunde liegenden strikten Ordnung.

\[
\begin{array}{llll}
    i       & (1) & < \text{ ist eine totale strikte Ordnung auf } S & \dots \\
    i       & (2) & \forall a, b \in S (a < b \lor b < a) & \rTotalStrictOrderE{1} \\
    i       & (3) & < \text{ ist eine strikte Ordnung auf }S & \rTotalStrictOrderE{1}
\end{array}
\]

\(i\) ist dabei die Liste der Annahmen.

\subsubsection*{Einführungsregel für Totalität}
\label{rule:rTotalityStrictRI}
Die Einführungsregel für Totalität ermöglicht es, die Totalität einer totalen strikten Ordnung \(<\) zu zeigen, indem bewiesen wird, dass für alle \(a, b \in S\) entweder \(a < b\) oder \(b < a\) gilt.

\[
\begin{array}{llll}
       & (1) & a < b \lor b < a & \rTotalityStrictRI{}
\end{array}
\]

\subsection{Induzierte strikte Ordnung aus einer Halbordnung}

\begin{definition}[Induzierte Ordnung aus totaler Ordnung]
    Sei \(\leq\) eine Halbordnung auf einer Menge \(S\). Dann definieren wir die zugehörige Ordnung \(<\) auf \(S\) durch:
    \[
    \forall a, b \in S \, (a < b \coloneqq a \leq b \land a \neq b).
    \]
\end{definition}

\begin{remark}
    Die Ordnung \(<\) wird auch als die von der Halbordnung \(\leq\) induzierte  Ordnung bezeichnet. Sie verfeinert die Relation \(\leq\), indem sie die Gleichheit ausschließt.
\end{remark}

\subsubsection{Regeln für die induzierte Ordnung}
\label{rule:InducedStrictOrderE} \label{rule:InducedStrictOrderI}

Die induzierte Ordnung, dargestellt durch das Symbol \(<\), ist eine Beziehung zwischen zwei Elementen einer Menge \(S\). Basierend auf unserer Definition für die induzierte strikte Ordnung, können wir zwei grundlegende Regeln einführen: die Einführungs- und die Eliminierungsregel.

\paragraph{Einführungsregel für die induzierte Ordnung (\(<\))}
Die Einführungsregel für die induzierte Ordnung (\(< I\)) besagt, dass wenn wir zeigen können, dass \(a \leq b\) und gleichzeitig \(a \neq b\) gilt, dann können wir daraus schließen, dass \(a < b\) gilt. Dabei dürfen \(a\) und \(b\) aus keiner der Annahmen eliminiert worden sein.

\[
\begin{array}{llll}
    i   & (1) & a \leq b & \dots \\
    j   & (2) & a \neq b & \dots \\
    i,j & (3) & a < b & \InducedStrictOrderI{1,2}
\end{array}
\]

bzw.

\[
\begin{array}{llll}
    i   & (1) & a \leq b & \dots \\
    j   & (2) & b \neq a & \dots \\
    i,j & (3) & a < b & \InducedStrictOrderI{1,2}
\end{array}
\]

\paragraph{Eliminierungsregel für die induzierte Ordnung \(<\)}
Die Eliminierungsregel für die induzierte Ordnung (\(< E\)) besagt, dass wenn wir wissen, dass \(a <_{\leq} b\) gilt, dann können wir daraus schließen, dass sowohl \(a \leq b\) als auch \(a \neq b\) gelten.

\[
\begin{array}{llll}
    i & (1) & a < b & \dots \\
    i & (2) & a \leq b & \InducedStrictOrderE{1} \\
    i & (3) & a \neq b & \InducedStrictOrderE{1}
\end{array}
\]

\(i\) und \(j\) sind dabei Listen von Annahmen.

\label{InducedStrictOrderImpnLpaLneqaRp}
\begin{theorem}[Irreflexivität der induzierten Ordnung]
    Sei \(<\) die aus der Halbordnung \(\leq\) induzierte Ordnung auf \(S\) und \(a\in S\). Dann gilt:
    \[
    \vdash \neg (a < a).
    \]
\end{theorem}
\begin{proof}
	\[
	\begin{array}{llll}
		1   & (1) & a<a & \rA\\
		1   & (2) & a\neq a & \InducedStrictOrderE{1}\\
		  & (3) & a=a & \rII{}\\
        1   & (4) & \bot & \rBI{2,3}\\
        1   & (5) & \neg(a<a) & \rCI{1,4}\\
	\end{array}
	\]    
\end{proof}

\label{InducedStrictOrderaLneqbwbLneqcImpaLneqc}
\begin{theorem}[Irreflexivität der induzierten Ordnung]
    Sei \(<\) die aus der Halbordnung \(\leq\) induzierte Ordnung auf \(S\) und \(a,b,c\in S\). Dann gilt:
    \[
    a<b,b<c\vdash a < c
    \]
\end{theorem}
\begin{proof}
	\[
	\begin{array}{llll}
		1     & (1) & a<b & \rA\\
		2     & (2) & b<c & \rA\\
		1     & (3) & a\leq b & \InducedStrictOrderE{1}\\
        1     & (4) & a\neq b & \InducedStrictOrderE{1}\\
        2     & (5) & b\leq c & \InducedStrictOrderE{2}\\
        2     & (6) & b\neq c & \InducedStrictOrderE{2}\\
        1,2   & (7) & a\leq c & \rTransitivityOrdRI{3,5}\\
        8     & (8) & a=c & \rA\\
        2,8   & (9) & b\leq a & \rIE{8,5}\\
        1,2,8 & (10) & a=b & \rAntisymmetryOrdRI{3,9}\\
        1,2,8 & (11) & \bot & \rBI{3,9}\\
        1,2   & (12) & a\neq b & \rCI{8,11}\\
	\end{array}
	\]
\end{proof}

\label{InducedStrictOrderFromHalfOrder}
\begin{theorem}[Die aus einer Halbordnung induzierte Ordnung \(<\) ist eine strikte Ordnung.]
    Sei \(<\) die aus der Halbordnung \(\leq\) induzierte Ordnung auf \(S\). Dann gilt:
    \[
    \vdash \text{ } < \text{ ist eine strikte Ordnung auf }S
    \]
\end{theorem}
\begin{proof}
	\[
        \begin{array}{llll}
          & (1) & \forall a \in S \, \neg (a < a) & \InducedStrictOrderImpnLpaLneqaRp{} \\
          & (2) & \forall a, b, c \in S ((a < b \land b < c) \rightarrow a < c) & \InducedStrictOrderaLneqbwbLneqcImpaLneqc{} \\
          & (3) & < \text{ ist eine strikte Ordnung auf } S & \rStrictOrderRelationI{1,2}
        \end{array}
	\]
\end{proof}

\label{ImpLpaLneqbOrbLneqaRpOraEqualsb}
\begin{theorem}[\(\vdash (a<b\lor b<a)\lor a=b\)]
    Sei \(<\) die aus der totalen Halbordnung \(\leq\) induzierte Ordnung auf \(S\). Dann gilt für alle \(a,b\in S\):
    \[
    \vdash (a<b\lor b<a)\lor a=b
    \]
\end{theorem}
\begin{proof}
        Seien \(a,b\in S\), dann gilt:
	\[
        \begin{array}{llll}
            & (1) & a\leq b\lor b\leq a & \rTotalityOrdRI{} \\
            & (2) & a=b\lor a\neq b & \ImpPOrnP{} \\
          3 & (3) & a=b & \rA \\
          3 & (4) & (a<b\lor b<a)\lor a=b & \rOIb{3} \\
          5 & (5) & a\neq b & \rA \\
          6 & (6) & a\leq b & \rA \\
          5,6 & (7) & a<b & \InducedStrictOrderI{5,6} \\
          5,6 & (8) & a<b\lor b<a  & \rOIa{7} \\
          5,6 & (9) & (a<b\lor b<a)\lor a=b  & \rOIa{8} \\
          10 & (10) & b\leq a  & \rA \\
          5,10 & (11) & b<a  & \InducedStrictOrderI{5,10} \\
          5,10 & (12) & a<b\lor b<a  & \rOIb{11} \\
          5,10 & (13) & (a<b\lor b<a)\lor a=b  & \rOIa{12} \\
          5    & (14) & (a<b\lor b<a)\lor a=b  & \rOE{1,6,9,10,13} \\
               & (15) & (a<b\lor b<a)\lor a=b  & \rOE{2,3,4,5,14} \\
        \end{array}
	\]
\end{proof}

\label{aLneqbOrLpbLneqaOraEqualsbRp}
\begin{theorem}[\(\vdash a<b\lor (b<a\lor a=b)\)]
    Sei \(<\) die aus der totalen Halbordnung \(\leq\) induzierte Ordnung auf \(S\). Dann gilt für alle \(a,b\in S\):
    \[
    \vdash a<b\lor (b<a\lor a=b)
    \]
\end{theorem}
\begin{proof}
        Seien \(a,b\in S\), dann gilt:
	\[
        \begin{array}{llll}
            & (1) & (a<b\lor b<a)\lor a=b & \ImpLpaLneqbOrbLneqaRpOraEqualsb{} \\
            & (2) & a<b\lor (b<a\lor a=b) & \POrLpQOrRRpImpLpPOrQRpOrQ{} \\
        \end{array}
	\]
\end{proof}

\label{LpaLeqbRpEqvaLneqbOraEqualsb}
\begin{theorem}[\((a\leq b)\dashv\vdash a<b\lor a=b\)]
    Sei \(<\) die aus der Halbordnung \(\leq\) induzierte Ordnung auf \(S\). Dann gilt für alle \(a,b\in S\):
    \[
    (a\leq b)\dashv\vdash a<b\lor a=b
    \]
\end{theorem}
\begin{proof}
        Seien \(a,b\in S\), dann gilt:
\(\vdash\):
	\[
        \begin{array}{llll}
            1   &   (1) &   a\leq b             &   \rA                         \\
                &   (2) &   a=b\lor a\neq b     &   \ImpPOrnP{}                 \\
            3   &   (3) &   a=b                 &   \rA                         \\
            3   &   (4) &   a<b\lor a=b         &   \rOIb{3}                    \\
            5   &   (5) &   a\neq b             &   \rA                         \\
            1,5 &   (6) &   a<b                 &   \InducedStrictOrderI{1,5}   \\   
            1,5 &   (7) &   a<b\lor a=b         &   \rOIa{6}                    \\
            1   &   (8) &   a<b\lor a=b         &   \rOE{2,3,4,5,7}             \\
        \end{array}
	\]
 \(\dashv\):
 	\[
        \begin{array}{llll}
            1   &   (1) &   a<b\lor a=b         &   \rA                         \\
            2   &   (2) &   a<b                 &   \rA                         \\
            3   &   (3) &   a\leq b             &   \InducedStrictOrderE{2}     \\
            4   &   (4) &   a=b                 &   \rA                         \\
                &   (5) &   a\leq a             &   \rReflexivityOrdRI{}        \\
            4   &   (6) &   a\leq b             &   \rIE{4,5}                   \\
            1   &   (7) &   a\leq b             &   \rOE{1,2,3,4,6}             \\
        \end{array}
	\]
\end{proof}

\label{nLpaLneqbRpEqvbLeqa}
\begin{theorem}[\(\neg(a<b)\dashv\vdash b\leq a\)]
    Sei \(<\) die aus der totalen Halbordnung \(\leq\) induzierte Ordnung auf \(S\). Dann gilt für alle \(a,b\in S\):
    \[
    \neg(a<b)\dashv\vdash b\leq a
    \]
\end{theorem}
\begin{proof}
        Seien \(a,b\in S\), dann gilt:
        
\(\vdash\):
	\[
        \begin{array}{llll}
          1     &   (1)     &   \neg(a<b)                   & \rA                               \\
                &   (2)     &   a<b\lor (b<a\lor a=b)       & \aLneqbOrLpbLneqaOraEqualsbRp{}   \\
          1     &   (3)     &   b<a\lor a=b                 & \POrQwnPImpQ{2,1}                 \\
          1     &   (4)     &   b\leq a                     & \LpaLeqbRpEqvaLneqbOraEqualsb{3}  \\
        \end{array}
	\]
\(\dashv\):
 	\[
        \begin{array}{llll}
            1   &   (1) &   b\leq               &   \rA                                \\
            1   &   (2) &   b<a\lor a=b         &   \LpaLeqbRpEqvaLneqbOraEqualsb{1}   \\
            3   &   (3) &   b<a                 &   \rA                                \\
            3   &   (4) &   \neg(a<b)           &   \aLneqbImpnLpbLneqaRp{3}           \\
            5   &   (5) &   a=b                 &   \rA                                \\
                &   (6) &   \neg(b<a)           &   \aEqualsbImpnLpbLneqaRp{5}         \\
            1   &   (7) &   \neg(b<a)           & \rOE{2,3,4,5,6}                      \\
        \end{array}
	\]
\end{proof}

\label{aLneqbEqvnLpbLeqaRp}
\begin{theorem}[\(a<b\dashv\vdash \neg(b\leq a)\)]
    Sei \(<\) die aus der totalen Halbordnung \(\leq\) induzierte Ordnung auf \(S\). Dann gilt für alle \(a,b\in S\):
    \[
    a<b\dashv\vdash \neg(b\leq a)
    \]
\end{theorem}
\begin{proof}
        Seien \(a,b\in S\), dann gilt:
        
\(\vdash\):
	\[
        \begin{array}{llll}
          1 & (1) &   a<b  & \rA\\
            & (2) &  \neg(a<b)\leftrightarrow b\leq a  & \nLpaLneqbRpEqvbLeqa{}\\
          1 & (3) &  \neg(b\leq a)  & \PLrQwnQImpnP{1}\\
        \end{array}
	\]
\(\dashv\):
 	\[
        \begin{array}{llll}
            1 & (1) &   \neg(b\leq a)  & \rA\\        
              & (2) &   \neg(a<b)\leftrightarrow b\leq a  &   \nLpaLneqbRpEqvbLeqa{1}   \\
            1 & (3) &   b<a                 &   \PLrQwnPImpnQ{1,2}\\
        \end{array}
	\]
\end{proof}


\subsection{Kettennotation für Relationen mit der Eigenschaft der Transititvität}

In mathematischen Beweisen ist es oft hilfreich, Beziehungen, die die Eigenschaft der Transitivität erfüllen, in kompakter Form zu notieren. Dazu führen wir eine Kettennotation ein, die für beliebige Relationen \(\mathrel{R}\) gilt, sofern die Transitivität 
\[
a \mathrel{R} b \land b \mathrel{R} c \vdash a \mathrel{R} c
\]
erfüllt ist. Beispiele solcher Relationen sind:
\begin{itemize}
    \item Äquivalenzrelationen (\(\sim\)),
    \item Halbordnungen (\(\leq\)),
    \item strikte Ordnungen (\(<\)),
    \item totale Ordnungen (\(\leq\) und \(<\)).
\end{itemize}

\subsubsection*{Definition der Kettennotation}

Eine Kettennotation für Relationen mit Transitivität ist eine Darstellung der Form:
\[
a \stackrel{\text{Regel}_1}{\mathrel{R}} b \stackrel{\text{Regel}_2}{\mathrel{R}} c \stackrel{\text{Regel}_3}{\mathrel{R}} d,
\]
wobei jeder Übergang \(x \stackrel{\text{Regel}}{\mathrel{R}} y\) durch die angegebene Regel \(\text{Regel}\) gerechtfertigt ist.

Sofern die Regelnamen sehr lang oder nicht offensichtlich sind und dies zu einer unübersichtlichen Darstellung in der Kette führen würde, nutzen wir alternativ die tabellarische Notation:
\[
\begin{array}{llclll}
	1 & (1) & a & \mathrel{R} & b & \rA \\
	1 & (2) &   & \mathrel{R} & c & \text{Regel}_1 \\
	1 & (3) &   & \mathrel{R} & d & \text{Regel}_2 \\
        1 & (4) &  a & \mathrel{R} & d & \rTransitivityOrdRI{1,3} \\
\end{array}
\]

Diese Notationen vereinfachen Beweise, indem sie die Struktur hervorheben und die Transitivität klar nutzen.

\paragraph{Anmerkung:}  
In die Kettennotation kann auch die Gleichheit \(=\) integriert werden, da sie im Kalkül des natürlichen Schließens eine fundamentale Rolle spielt und das Resultat der Relation nicht verändert. 

Die Gleichheit wurde als Identitätssymbol mit spezifischen Einführungs- und Eliminierungsregeln eingeführt, die garantieren, dass \(a = b\) bedeutet, dass \(a\) und \(b\) in jedem Kontext austauschbar sind. Darüber hinaus wurde gezeigt, dass die Gleichheit auf jeder beliebigen Menge eine Äquivalenzrelation ist, da sie die Reflexivität, Symmetrie und Transitivität erfüllt. Insbesondere gilt die Transitivität:
\[
a = b \land b = c \vdash a = c,
\]
wodurch sich die Gleichheit nahtlos in die Kettennotation einfügt.

Zusätzlich ist die Gleichheit mit anderen Relationen wie \(\leq\) oder \(\sim\) kompatibel. Dies erlaubt die Darstellung gemischter Relationen in der Kettennotation, wie etwa:
\[
a = b \leq c = d.
\]
Hierbei bleibt die logische Konsistenz der Kette erhalten, da die Gleichheit \(=\) keine neuen Einschränkungen einführt und die Ergebnisse der Relation \(\mathrel{R}\) nicht beeinflusst.

Die Integration der Gleichheit in die Kettennotation macht Beweise übersichtlicher und nutzt die Einführungs- und Eliminierungsregeln des Identitätssymbols effizient, um Beweisstrukturen kompakt darzustellen.

\paragraph{Verallgemeinerung durch Symmetrie:}  
Da die Gleichheit \(=\) im Kalkül des natürlichen Schließens auch die Eigenschaft der Symmetrie erfüllt, können Gleichungen in der Kettennotation beliebig vertauscht werden, ohne die logische Konsistenz der Kette zu beeinflussen. Dies bedeutet, dass in einer Kette wie:
\[
a \stackrel{\text{Regel}_1}{=} b \stackrel{\text{Regel}_2}{=} c \stackrel{\text{Regel}_3}{=} d
\]
die Reihenfolge der Gleichheitsrelationen verändert werden kann, um beispielsweise:
\[
d \stackrel{\text{Regel}_3}{=} c \stackrel{\text{Regel}_2}{=} b \stackrel{\text{Regel}_1}{=} a
\]
zu erhalten, wobei die Gültigkeit der Kette erhalten bleibt.

Diese Eigenschaft unterscheidet die Gleichheit von anderen Relationen wie \(\leq\), die keine Symmetrie besitzen. Während in gemischten Relationen wie \(a = b \leq c = d\) die Gleichungen vertauscht werden können, bleibt die Anordnung der nicht-symmetrischen Relationen wie \(\leq\) unverändert. 

Diese Flexibilität macht die Gleichheit in der Kettennotation besonders nützlich und hebt ihre Rolle in mathematischen Beweisen hervor.

\subsubsection*{Bemerkung zur Verallgemeinerung}

Die hier beschriebene Kettennotation ist universell anwendbar auf alle Relationen mit der Eigenschaft der Transitivität. Für spezifische Relationen wie \(\sim\), \(\leq\) oder \(<\) kann das jeweilige Symbol \(\mathrel{R}\) entsprechend ersetzt werden. Dies macht die Notation flexibel und für verschiedenste mathematische Strukturen geeignet.

\subsection{Geschachtelte Ordnungen}

\begin{definition}[Geschachtelte Ordnungen]
    Sei \(\leq\) eine Halbordnung auf \(S\) und \(<\) die aus \(\leq\) induzierte strikte Ordnung, definiert durch \( a < b \coloneqq (a \leq b) \land (a \neq b) \). Für Elemente \(a, b, c \in S\) definieren wir Ausdrücke der Form \(a \leq b < c\) als Kombination aus \(\leq\)- und \(<\)-Relationen:
    \[
    a \leq b < c \coloneqq (a \leq b) \land (b < c).
    \]
    Analog gelten:
    \begin{align*}
        a < b \leq c &\coloneqq (a < b) \land (b \leq c), \\
        a < b < c &\coloneqq (a < b) \land (b < c), \\
        a \leq b \leq c &\coloneqq (a \leq b) \land (b \leq c).
    \end{align*}
\end{definition}

\begin{remark}
    Die Schreibweise geschachtelter Ordnungen wie \(a \leq b < c\) ermöglicht eine kompakte Darstellung kombinierter Ordnungsrelationen und wird häufig in analytischen und algebraischen Strukturen verwendet.
\end{remark}

\subsubsection{Regeln für geschachtelte Ordnungen}
\label{rule:rLeqLltE} \label{rule:rLeqLltI}
\label{rule:rLltLeqE} \label{rule:rLltLeqI}
\label{rule:rLltLltE} \label{rule:rLltLltI}
\label{rule:rLeqLeqE} \label{rule:rLeqLeqI}

Die geschachtelten Ordnungen erlauben, kombinierte Ordnungsrelationen wie \(a \leq b < c\) formal abzuleiten. Die folgenden Regeln stellen die Einführungs- und Eliminierungsregeln für die vier möglichen geschachtelten Ordnungen dar.

\paragraph{Einführungsregel für \(a \leq b < c\)}
Die Einführungsregel für \(a \leq b < c\) besagt, dass wenn sowohl \(a \leq b\) als auch \(b < c\) gilt, dann \(a \leq b < c\) folgt. Dabei sind \(i\) und \(j\) Listen von Annahmen.
\[
\begin{array}{llll}
    i   & (1) & a \leq b & \dots \\
    j   & (2) & b < c & \dots \\
    i,j & (3) & a \leq b < c & \rLeqLltI{1,2}
\end{array}
\]

\paragraph{Eliminierungsregel für \(a \leq b < c\)}
Die Eliminierungsregel für \(a \leq b < c\) erlaubt es, aus \(a \leq b < c\) die beiden Aussagen \(a \leq b\) und \(b < c\) abzuleiten.
\[
\begin{array}{llll}
    i & (1) & a \leq b < c & \dots \\
    i & (2) & a \leq b & \rLeqLltE{1} \\
    i & (3) & b < c & \rLeqLltE{1}
\end{array}
\]

\paragraph{Einführungsregel für \(a < b \leq c\)}
Die Einführungsregel für \(a < b \leq c\) besagt, dass wenn sowohl \(a < b\) als auch \(b \leq c\) gilt, dann \(a < b \leq c\) folgt.
\[
\begin{array}{llll}
    i   & (1) & a < b & \dots \\
    j   & (2) & b \leq c & \dots \\
    i,j & (3) & a < b \leq c & \rLltLeqI{1,2}
\end{array}
\]

\paragraph{Eliminierungsregel für \(a < b \leq c\)}
Die Eliminierungsregel für \(a < b \leq c\) erlaubt es, aus \(a < b \leq c\) die beiden Aussagen \(a < b\) und \(b \leq c\) abzuleiten.
\[
\begin{array}{llll}
    i & (1) & a < b \leq c & \dots \\
    i & (2) & a < b & \rLltLeqE{1} \\
    i & (3) & b \leq c & \rLltLeqE{1}
\end{array}
\]

\paragraph{Einführungsregel für \(a < b < c\)}
Die Einführungsregel für \(a < b < c\) besagt, dass wenn sowohl \(a < b\) als auch \(b < c\) gilt, dann \(a < b < c\) folgt.
\[
\begin{array}{llll}
    i   & (1) & a < b & \dots \\
    j   & (2) & b < c & \dots \\
    i,j & (3) & a < b < c & \rLltLltI{1,2}
\end{array}
\]

\paragraph{Eliminierungsregel für \(a < b < c\)}
Die Eliminierungsregel für \(a < b < c\) erlaubt es, aus \(a < b < c\) die beiden Aussagen \(a < b\) und \(b < c\) abzuleiten.
\[
\begin{array}{llll}
    i & (1) & a < b < c & \dots \\
    i & (2) & a < b & \rLltLltE{1} \\
    i & (3) & b < c & \rLltLltE{1}
\end{array}
\]

\paragraph{Einführungsregel für \(a \leq b \leq c\)}
Die Einführungsregel für \(a \leq b \leq c\) besagt, dass wenn sowohl \(a \leq b\) als auch \(b \leq c\) gilt, dann \(a \leq b \leq c\) folgt.
\[
\begin{array}{llll}
    i   & (1) & a \leq b & \dots \\
    j   & (2) & b \leq c & \dots \\
    i,j & (3) & a \leq b \leq c & \rLeqLeqI{1,2}
\end{array}
\]

\paragraph{Eliminierungsregel für \(a \leq b \leq c\)}
Die Eliminierungsregel für \(a \leq b \leq c\) erlaubt es, aus \(a \leq b \leq c\) die beiden Aussagen \(a \leq b\) und \(b \leq c\) abzuleiten.
\[
\begin{array}{llll}
    i & (1) & a \leq b \leq c & \dots \\
    i & (2) & a \leq b & \rLeqLeqE{1} \\
    i & (3) & b \leq c & \rLeqLeqE{1}
\end{array}
\]

\(i\) und \(j\) sind dabei Listen von Annahmen.

\subsection{Intervalle}

\begin{definition}[Intervalle]
    Sei \(S\) eine Menge mit einer Halbordnung \(\leq\) und \(<\) die induzierte strikte Ordnung aus \(\leq\), definiert durch \( a < b \coloneqq (a \leq b) \land (a \neq b) \). Ein \textbf{Intervall} in \(S\) ist eine Teilmenge von \(S\), die alle Elemente zwischen zwei gegebenen Grenzen enthält. Wir definieren die verschiedenen Intervalltypen wie folgt:
    \begin{align*}
        [a, b] &\coloneqq \{ x \in S \mid a \leq x \leq b \}, \\
        (a, b) &\coloneqq \{ x \in S \mid a < x < b \}, \\
        [a, b) &\coloneqq \{ x \in S \mid a \leq x < b \}, \\
        (a, b] &\coloneqq \{ x \in S \mid a < x \leq b \}.
    \end{align*}
\end{definition}

\begin{remark}
    Intervalle stellen wichtige Teilmengen einer geordneten Menge dar und erlauben es, Teilbereiche basierend auf den Ordnungsrelationen zu untersuchen.
\end{remark}

\subsubsection{Einführungsregeln für Intervalle}
\label{rule:rClosedIntervalI} \label{rule:rOpenIntervalI} \label{rule:rClosedOpenIntervalI} \label{rule:rOpenClosedIntervalI}

Die Einführungsregeln für Intervalle ermöglichen es, die Zugehörigkeit eines Elements zu einem bestimmten Intervall zu zeigen, wenn die jeweiligen Ordnungsrelationen zwischen dem Element und den Intervallgrenzen erfüllt sind.

\paragraph{Einführungsregel für das abgeschlossene Intervall \([a, b]\)}
Die Einführungsregel für das abgeschlossene Intervall \([a, b]\) besagt, dass ein Element \(x \in S\) in \([a, b]\) liegt, wenn \(a \leq x \leq b\) gilt.
\[
\begin{array}{llll}
    i   & (1) & a \leq x & \dots \\
    j   & (2) & x \leq b & \dots \\
    i,j & (3) & x \in [a, b] & \rClosedIntervalI{1,2}
\end{array}
\]

\paragraph{Einführungsregel für das offene Intervall \((a, b)\)}
Die Einführungsregel für das offene Intervall \((a, b)\) besagt, dass ein Element \(x \in S\) in \((a, b)\) liegt, wenn \(a < x < b\) gilt.
\[
\begin{array}{llll}
    i   & (1) & a < x & \dots \\
    j   & (2) & x < b & \dots \\
    i,j & (3) & x \in (a, b) & \rOpenIntervalI{1,2}
\end{array}
\]

\paragraph{Einführungsregel für das halboffene Intervall \([a, b)\)}
Die Einführungsregel für das halboffene Intervall \([a, b)\) besagt, dass ein Element \(x \in S\) in \([a, b)\) liegt, wenn \(a \leq x < b\) gilt.
\[
\begin{array}{llll}
    i   & (1) & a \leq x & \dots \\
    j   & (2) & x < b & \dots \\
    i,j & (3) & x \in [a, b) & \rClosedOpenIntervalI{1,2}
\end{array}
\]

\paragraph{Einführungsregel für das halboffene Intervall \((a, b]\)}
Die Einführungsregel für das halboffene Intervall \((a, b]\) besagt, dass ein Element \(x \in S\) in \((a, b]\) liegt, wenn \(a < x \leq b\) gilt.
\[
\begin{array}{llll}
    i   & (1) & a < x & \dots \\
    j   & (2) & x \leq b & \dots \\
    i,j & (3) & x \in (a, b] & \rOpenClosedIntervalI{1,2}
\end{array}
\]

\subsubsection{Eliminationsregeln für Intervalle}
\label{rule:rClosedIntervalE} \label{rule:rOpenIntervalE} \label{rule:rClosedOpenIntervalE} \label{rule:rOpenClosedIntervalE}

Die Eliminationsregeln für Intervalle ermöglichen es, aus der Tatsache, dass ein Element in einem bestimmten Intervall liegt, die entsprechenden Ordnungsrelationen zwischen dem Element und den Intervallgrenzen abzuleiten.

\paragraph{Eliminationsregel für das abgeschlossene Intervall \([a, b]\)}
Die Eliminationsregel für das abgeschlossene Intervall \([a, b]\) besagt, dass wenn \(x \in [a, b]\) gilt, dann sowohl \(a \leq x\) als auch \(x \leq b\) folgt.
\[
\begin{array}{llll}
    i & (1) & x \in [a, b] & \dots \\
    i & (2) & a \leq x & \rClosedIntervalE{1} \\
    i & (3) & x \leq b & \rClosedIntervalE{1}
\end{array}
\]

\paragraph{Eliminationsregel für das offene Intervall \((a, b)\)}
Die Eliminationsregel für das offene Intervall \((a, b)\) besagt, dass wenn \(x \in (a, b)\) gilt, dann sowohl \(a < x\) als auch \(x < b\) folgt.
\[
\begin{array}{llll}
    i & (1) & x \in (a, b) & \dots \\
    i & (2) & a < x & \rOpenIntervalE{1} \\
    i & (3) & x < b & \rOpenIntervalE{1}
\end{array}
\]

\paragraph{Eliminationsregel für das halboffene Intervall \([a, b)\)}
Die Eliminationsregel für das halboffene Intervall \([a, b)\) besagt, dass wenn \(x \in [a, b)\) gilt, dann sowohl \(a \leq x\) als auch \(x < b\) folgt.
\[
\begin{array}{llll}
    i & (1) & x \in [a, b) & \dots \\
    i & (2) & a \leq x & \rClosedOpenIntervalE{1} \\
    i & (3) & x < b & \rClosedOpenIntervalE{1}
\end{array}
\]

\paragraph{Eliminationsregel für das halboffene Intervall \((a, b]\)}
Die Eliminationsregel für das halboffene Intervall \((a, b]\) besagt, dass wenn \(x \in (a, b]\) gilt, dann sowohl \(a < x\) als auch \(x \leq b\) folgt.
\[
\begin{array}{llll}
    i & (1) & x \in (a, b] & \dots \\
    i & (2) & a < x & \rOpenClosedIntervalE{1} \\
    i & (3) & x \leq b & \rOpenClosedIntervalE{1}
\end{array}
\]

\(i\) und \(j\) sind dabei Listen von Annahmen.

\subsection{Extremale Elemente und Schranken in halbgeordneten Mengen}


\label{MInSwNInSwFatInSLptLeqMRpwFatInSLptLeqNRpImpMEqualsN}
\begin{theorem}[Eindeutigkeit des Maximums]
    Sei \( S \) eine Halbordnung und seien \( M \) und \( N \) zwei maximale Elemente in \( S \). Dann gilt:
    \[
    M\in S, N\in S, \forall t \in S(t \leq M), \forall t \in S(t \leq N) \vdash M = N.
    \]
\end{theorem}

\begin{proof}
    Sei \(S\) eine Halbordnung, dann gilt:
    \[
    \begin{array}{lll p{4cm}}
        1 & (1) & M\in S & \rA \\
        2 & (2) & N\in S & \rA \\
        3 & (3) & \forall t \in S(t \leq M) & \rA \\
        4 & (4) & \forall t \in S(t \leq N) & \rA \\
        2,3 & (5) & N\leq M & \rSetUEc{2,3}  \\
        1,4 & (6) & M\leq N & \rSetUEc{1,4} \\
        1,2,3,4 & (7) & M = N & \rAntisymmetryOrdRI{5,6} \\
    \end{array}
    \]
\end{proof}

\label{PInSwQInSwFatInSLpPLeqtRpwFatInSLpQLeqtRpImpPEqualsQ}
\begin{theorem}[Eindeutigkeit des Minimums]
    Sei \( S \) eine Halbordnung und seien \( P \) und \( Q \) zwei minimale Elemente in \( S \). Dann gilt:
    \[
    P \in S, Q \in S, \forall t \in S(P \leq t), \forall t \in S(Q \leq t) \vdash P = Q.
    \]
\end{theorem}

\begin{proof}
    Sei \(S\) eine Halbordnung, dann gilt:
    \[
    \begin{array}{lll p{4cm}}
        1 & (1) & P \in S & \rA \\
        2 & (2) & Q \in S & \rA \\
        3 & (3) & \forall t \in S (P \leq t) & \rA \\
        4 & (4) & \forall t \in S (Q \leq t) & \rA \\
        2,3 & (5) & Q \leq P & \rSetUEc{2,3}  \\
        1,4 & (6) & P \leq Q & \rSetUEc{1,4} \\
        1,2,3,4 & (7) & P = Q & \rAntisymmetryOrdRI{5,6} \\
    \end{array}
    \]
\end{proof}

\begin{definition}[Maximum]
    Sei \( S \) eine halbgeordnete Menge. Das neue Symbol \(\max(S)\) wird durch folgende partielle Definition eingeführt:
    \[
    \max(S) \coloneqq \iota M \, \big(M \in S \land \forall t \in S \, (t \leq M)\big).
    \]
    Dabei ist \(\max(S)\) das \textbf{Definiendum} und \( M \) das \textbf{Definiens}. Diese Definition stellt sicher, dass \(\max(S)\) genau dann definiert ist, wenn ein eindeutiges maximales Element in \( S \) existiert.
\end{definition}

\begin{definition}[Minimum]
    Sei \( S \) eine halbgeordnete Menge. Das neue Symbol \(\min(S)\) wird durch folgende partielle Definition eingeführt:
    \[
    \min(S) \coloneqq \iota N \, \big(N \in S \land \forall t \in S \, (N \leq t)\big).
    \]
    Dabei ist \(\min(S)\) das \textbf{Definiendum} und \( N \) das \textbf{Definiens}. Diese Definition stellt sicher, dass \(\min(S)\) genau dann definiert ist, wenn ein eindeutiges minimales Element in \( S \) existiert.
\end{definition}

\subsubsection{Einführungs- und Eliminationsregeln für Maximum und Minimum}
\label{rule:rMaxI} \label{rule:rMaxE} \label{rule:rMinI} \label{rule:rMinE}

Die Regeln zur Einführung und Elimination von Maximum und Minimum ermöglichen es, die Existenz und Eigenschaften der maximalen und minimalen Elemente in halbgeordneten Mengen zu formalisieren.

\paragraph{Einführungsregel für das Minimum}
Die Einführungsregel für das Minimum \(\min(S)\) besagt, dass ein Element \(N\) das Minimum einer halbgeordneten Menge \(S\) ist, wenn \(N \in S\) und für alle \(t \in S\) gilt: \(N \leq t\).
\[
\begin{array}{llll}
    i   & (1) & N \in S & \dots \\
    j   & (2) & \forall t \in S \, (N \leq t) & \dots \\
    i,j & (3) & \min(S) = N & \rMinI{1,2}
\end{array}
\]

\paragraph{Eliminationsregel für das Minimum}
Die Eliminationsregel für das Minimum \(\min(S)\) besagt, dass, wenn \(N = \min(S)\), dann für jedes \(t \in S\) die Relation \(N \leq t\) gilt. Ebenso ist \(N \in S\).
\[
\begin{array}{llll}
    i & (1) & N = \min(S) & \dots \\
    i & (2) & \forall t \in S \, (N \leq t) & \rMinE{1} \\
    i & (3) & N \in S & \rMinE{1}\\
    j & (4) & j \in S & \hdots\\
    i,j & (5) & N\leq j & \rMinE{1,4}\\
\end{array}
\]

\[
\begin{array}{llll}
    i & (1) & \exists N\in S(N = \min(S)) & \dots \\
    i & (2) & \forall t \in S \, (\min(S) \leq t) & \rMinE{1} \\
    i & (3) & \min(S) \in S & \rMinE{1}\\
    j & (4) & j \in S & \hdots\\
    i,j & (5) & \min(S)\leq j & \rMinE{1,4}\\
\end{array}
\]

\paragraph{Einführungsregel für das Maximum}
Die Einführungsregel für das Maximum \(\max(S)\) besagt, dass ein Element \(M\) das Maximum einer halbgeordneten Menge \(S\) ist, wenn \(M \in S\) und für alle \(t \in S\) gilt: \(t \leq M\).
\[
\begin{array}{llll}
    i   & (1) & M \in S & \dots \\
    j   & (2) & \forall t \in S \, (t \leq M) & \dots \\
    i,j & (3) & \max(S) = M & \rMaxI{1,2}
\end{array}
\]

\paragraph{Eliminationsregel für das Maximum \(\max(S)\)}
Die Eliminationsregel für das Maximum \(\max(S)\) besagt, dass, wenn \(M = \max(S)\), dann für jedes \(t \in S\) die Relation \(t \leq M\) gilt. Ebenso ist \(M \in S\).
\[
\begin{array}{llll}
    i & (1) & M = \max(S) & \dots \\
    i & (2) & \forall t \in S \, (t \leq M) & \rMaxE{1} \\
    i & (3) & M \in S & \rMaxE{1}
\end{array}
\]
\(i\) und \(j\) sind dabei Listen von Annahmen.

\begin{definition}[Obere Schranke]
    Sei \( S \) eine halbgeordnete Menge und \( T \subseteq S \) eine Teilmenge. Ein Element \( u \in S \) heißt \textbf{obere Schranke} von \( T \), wenn gilt:
    \[
    \forall t \in T \, (t \leq u).
    \]
    Das Prädikat \(\text{ObereSchranke}(u, T)\) ist wahr genau dann, wenn \( u \) eine obere Schranke von \( T \) ist.
\end{definition}

\begin{definition}[Untere Schranke]
    Sei \( S \) eine halbgeordnete Menge und \( T \subseteq S \) eine Teilmenge. Ein Element \( l \in S \) heißt \textbf{untere Schranke} von \( T \), wenn gilt:
    \[
    \forall t \in T \, (l \leq t).
    \]
    Das Prädikat \(\text{UntereSchranke}(l, T)\) ist wahr genau dann, wenn \( l \) eine untere Schranke von \( T \) ist.
\end{definition}

\begin{definition}[Menge der oberen Schranken]
    Sei \( S \) eine halbgeordnete Menge und \( T \subseteq S \) eine Teilmenge. Die Menge der \textbf{oberen Schranken} von \( T \) in \( S \) wird definiert als:
    \[
    \text{UB}_S(T) := \{ u \in S \mid \forall t \in T \, (t \leq u) \}.
    \]
    Diese Menge enthält alle Elemente in \( S \), die eine obere Schranke von \( T \) sind.
\end{definition}

\begin{definition}[Menge der unteren Schranken]
    Sei \( S \) eine halbgeordnete Menge und \( T \subseteq S \) eine Teilmenge. Die Menge der \textbf{unteren Schranken} von \( T \) in \( S \) wird definiert als:
    \[
    \text{LB}_S(T) := \{ l \in S \mid \forall t \in T \, (l \leq t) \}.
    \]
    Diese Menge enthält alle Elemente in \( S \), die eine untere Schranke von \( T \) sind.
\end{definition}
\begin{remark}
    In diesem Skript wird in der Definition der Mengen der oberen und unteren Schranken die übergeordnete Menge \( S \) in der Notation oft weggelassen, sodass nur \(\text{UB}(T)\) und \(\text{LB}(T)\) geschrieben wird. Diese Konvention dient der Übersichtlichkeit und Kürze der Notation und ist sinnvoll, solange der Kontext klar ist und keine Verwechslungsgefahr besteht. Die Schranken beziehen sich dabei stets auf die Halbordnung der zugrunde liegenden Menge. Falls jedoch mehrere Ordnungen oder verschiedene übergeordnete Mengen betrachtet werden, kann es hilfreich sein, \( S \) explizit anzugeben, indem \(\text{UB}_S(T)\) und \(\text{LB}_S(T)\) verwendet werden.
\end{remark}

\subsubsection{Einführungs- und Eliminationsregeln für die Menge der oberen und unteren Schranken}
\label{rule:rUBSI} \label{rule:rUBSE} \label{rule:rLBSI} \label{rule:rLBSE}

Die Regeln zur Einführung und Elimination der Mengen der oberen und unteren Schranken ermöglichen es, die Existenz und Eigenschaften von oberen und unteren Schranken in halbgeordneten Mengen formal zu nutzen.

\paragraph{Einführungsregel für die Menge der oberen Schranken \(\text{UB}_S(T)\)}
Die Einführungsregel für die Menge der oberen Schranken \(\text{UB}_S(T)\) besagt, dass ein Element \(u\) eine obere Schranke der Teilmenge \(T\) in \( S \) ist, wenn \(u \in S\) und für alle \(t \in T\) gilt: \(t \leq u\).
\[
\begin{array}{llll}
    i   & (1) & u \in S & \dots \\
    j   & (2) & \forall t \in T \, (t \leq u) & \dots \\
    i,j & (3) & u \in \text{UB}_S(T) & \rUBSI{1,2}
\end{array}
\]

\paragraph{Eliminationsregel für die Menge der oberen Schranken \(\text{UB}_S(T)\)}
Die Eliminationsregel für die Menge der oberen Schranken \(\text{UB}_S(T)\) besagt, dass, wenn \(u \in \text{UB}_S(T)\), dann für jedes \(t \in T\) die Relation \(t \leq u\) gilt. Ebenso ist \(u \in S\).
\[
\begin{array}{llll}
    i & (1) & u \in \text{UB}_S(T) & \dots \\
    i & (2) & \forall t \in T \, (t \leq u) & \rUBSE{1} \\
    i & (3) & u \in S & \rUBSE{1}
\end{array}
\]

\paragraph{Einführungsregel für die Menge der unteren Schranken \(\text{LB}_S(T)\)}
Die Einführungsregel für die Menge der unteren Schranken \(\text{LB}_S(T)\) besagt, dass ein Element \(l\) eine untere Schranke der Teilmenge \(T\) in \( S \) ist, wenn \(l \in S\) und für alle \(t \in T\) gilt: \(l \leq t\).
\[
\begin{array}{llll}
    i   & (1) & l \in S & \dots \\
    j   & (2) & \forall t \in T \, (l \leq t) & \dots \\
    i,j & (3) & l \in \text{LB}_S(T) & \rLBSI{1,2}
\end{array}
\]

\paragraph{Eliminationsregel für die Menge der unteren Schranken \(\text{LB}_S(T)\)}
Die Eliminationsregel für die Menge der unteren Schranken \(\text{LB}_S(T)\) besagt, dass, wenn \(l \in \text{LB}_S(T)\), dann für jedes \(t \in T\) die Relation \(l \leq t\) gilt. Ebenso ist \(l \in S\).
\[
\begin{array}{llll}
    i & (1) & l \in \text{LB}_S(T) & \dots \\
    i & (2) & \forall t \in T \, (l \leq t) & \rLBSE{1} \\
    i & (3) & l \in S & \rLBSE{1}
\end{array}
\]

\(i\) und \(j\) sind dabei Listen von Annahmen.

\label{uInUBwvInUBwFawInUBLpuLeqwRpwFawInUBLpvLeqwRpImpuEqualsv}
\begin{theorem}[Eindeutigkeit des kleinsten Elements der oberen Schranken]
    Sei \( S \) eine Halbordnung und \( T \subseteq S \) eine Teilmenge. Seien \( u \) und \( v \) zwei Elemente in \( S \), die beide die folgenden Bedingungen erfüllen:
    \[
    u \in \text{UB}_S(T), v \in \text{UB}_S(T), \forall w \in \text{UB}_S(T) (u \leq w), \forall w \in \text{UB}_S(T) (v \leq w) \vdash u = v.
    \]
\end{theorem}

\begin{proof}
    Sei \( S \) eine Halbordnung und seien \( u, v \in S \) zwei Elemente, die die Bedingungen erfüllen. Dann gilt:
    \[
    \begin{array}{lll p{4cm}}
        1 & (1) & u \in \text{UB}_S(T) & \rA \\
        2 & (2) & v \in \text{UB}_S(T) & \rA \\
        3 & (3) & \forall w \in \text{UB}_S(T) \, (u \leq w) & \rA \\
        4 & (4) & \forall w \in \text{UB}_S(T) \, (v \leq w) & \rA \\
        2,3 & (5) & u \leq v & \rSetUEc{2,3}  \\
        1,4 & (6) & v \leq u & \rSetUEc{1,4} \\
        1,2,3,4 & (7) & u = v & \rAntisymmetryOrdRI{5,6} \\
    \end{array}
    \]
    Damit ist gezeigt, dass \( u = v \), und somit ist das kleinste Element der oberen Schranken, falls es existiert, eindeutig.
\end{proof}

\label{lInLBwmInLBwFanInLBLpnLeqlRpwFanInLBLpnLeqmRpImplEqualsm}
\begin{theorem}[Eindeutigkeit des größten Elements der unteren Schranken]
    Sei \( S \) eine Halbordnung und \( T \subseteq S \) eine Teilmenge. Seien \( l \) und \( m \) zwei Elemente in \( S \), die beide die folgenden Bedingungen erfüllen:
    \[
    l \in \text{LB}_S(T), m \in \text{LB}_S(T), \forall n \in \text{LB}_S(T) (n \leq l), \forall n \in \text{LB}_S(T) (n \leq m) \vdash l = m.
    \]
\end{theorem}

\begin{proof}
    Sei \( S \) eine Halbordnung und seien \( l, m \in S \) zwei Elemente, die die Bedingungen erfüllen. Dann gilt:
    \[
    \begin{array}{lll p{4cm}}
        1 & (1) & l \in \text{LB}_S(T) & \rA \\
        2 & (2) & m \in \text{LB}_S(T) & \rA \\
        3 & (3) & \forall n \in \text{LB}_S(T) \, (n \leq l) & \rA \\
        4 & (4) & \forall n \in \text{LB}_S(T) \, (n \leq m) & \rA \\
        2,3 & (5) & m \leq l & \rSetUEc{2,3}  \\
        1,4 & (6) & l \leq m & \rSetUEc{1,4} \\
        1,2,3,4 & (7) & l = m & \rAntisymmetryOrdRI{5,6} \\
    \end{array}
    \]
    Damit ist gezeigt, dass \( l = m \), und somit ist das größte Element der unteren Schranken, falls es existiert, eindeutig.
\end{proof}


\begin{definition}[Supremum]
    Sei \( S \) eine halbgeordnete Menge und \( T \subseteq S \) eine Teilmenge. Angenommen, \(\text{UB}_S(T)\), die Menge aller oberen Schranken von \( T \) in \( S \), ist nicht leer. Sei \( C(u) \) die Bedingung, dass \( u \) das kleinste Element von \(\text{UB}_S(T)\) ist, d.\,h.,
    \[
    C(u) := u \in \text{UB}_S(T) \land \forall v \in \text{UB}_S(T) \, (u \leq v).
    \]
    Eine partielle Definition erlaubt es, das neue Symbol \(\sup_S(T)\) einzuführen und es als \( u \) zu definieren, wenn die Bedingung \( C(u) \) erfüllt ist:
    \[
    \exists u \in S \, [ C(u) \rightarrow (\sup_S(T) \coloneqq u) ].
    \]
    Dabei ist \(\sup_S(T)\) das \textbf{Definiendum} und \( u \) das \textbf{Definiens}. Diese Definition stellt sicher, dass \(\sup_S(T)\) genau dann definiert ist, wenn eine kleinste obere Schranke von \( T \) in \( S \) existiert.
\end{definition}

\begin{definition}[Infimum]
    Sei \( S \) eine halbgeordnete Menge und \( T \subseteq S \) eine Teilmenge. Angenommen, \(\text{LB}_S(T)\), die Menge aller unteren Schranken von \( T \) in \( S \), ist nicht leer. Sei \( D(l) \) die Bedingung, dass \( l \) das größte Element von \(\text{LB}_S(T)\) ist, d.\,h.,
    \[
    D(l) := l \in \text{LB}_S(T) \land \forall m \in \text{LB}_S(T) \, (m \leq l).
    \]
    Eine partielle Definition erlaubt es, das neue Symbol \(\inf_S(T)\) einzuführen und es als \( l \) zu definieren, wenn die Bedingung \( D(l) \) erfüllt ist:
    \[
    \exists l \in S \, [ D(l) \rightarrow (\inf_S(T) \coloneqq l) ].
    \]
    Dabei ist \(\inf_S(T)\) das \textbf{Definiendum} und \( l \) das \textbf{Definiens}. Diese Definition stellt sicher, dass \(\inf_S(T)\) genau dann definiert ist, wenn eine größte untere Schranke von \( T \) in \( S \) existiert.
\end{definition}

\subsubsection{Einführungs- und Eliminationsregeln für Supremum und Infimum}
\label{rule:rSupSI} \label{rule:rSupSE} \label{rule:rInfSI} \label{rule:rInfSE}

Die Regeln zur Einführung und Elimination von Supremum und Infimum ermöglichen es, die Existenz und Eigenschaften des kleinsten Elements der oberen Schranken und des größten Elements der unteren Schranken in halbgeordneten Mengen formal zu nutzen.

\paragraph{Einführungsregel für das Supremum \(\sup_S(T)\)}
Die Einführungsregel für das Supremum \(\sup_S(T)\) besagt, dass ein Element \( u \) das Supremum der Teilmenge \( T \) in \( S \) ist, wenn \( u \in \text{UB}_S(T) \) und für alle \( v \in \text{UB}_S(T) \) gilt: \( u \leq v \).
\[
\begin{array}{llll}
    i   & (1) & u \in \text{UB}_S(T) & \dots \\
    j   & (2) & \forall v \in \text{UB}_S(T) \, (u \leq v) & \dots \\
    i,j & (3) & u = \sup_S(T) & \rSupSI{1,2}
\end{array}
\]

\paragraph{Eliminationsregel für das Supremum \(\sup_S(T)\)}
Die Eliminationsregel für das Supremum \(\sup_S(T)\) besagt, dass, wenn \( u = \sup_S(T) \), dann für jedes \( t \in T \) die Relation \( t \leq u \) gilt. Ebenso ist \( u \in S \).
\[
\begin{array}{llll}
    i & (1) & u = \sup_S(T) & \dots \\
    i & (2) & \forall t \in \text{UB}_S(T) \, (u \leq t) & \rSupSE{1} \\
    i & (3) & u \in \text{UB}_S(T) & \rSupSE{1}
\end{array}
\]

\paragraph{Einführungsregel für das Infimum \(\inf_S(T)\)}
Die Einführungsregel für das Infimum \(\inf_S(T)\) besagt, dass ein Element \( l \) das Infimum der Teilmenge \( T \) in \( S \) ist, wenn \( l \in \text{LB}_S(T) \) und für alle \( m \in \text{LB}_S(T) \) gilt: \( m \leq l \).
\[
\begin{array}{llll}
    i   & (1) & l \in \text{LB}_S(T) & \dots \\
    j   & (2) & \forall m \in \text{LB}_S(T) \, (m \leq l) & \dots \\
    i,j & (3) & l = \inf_S(T) & \rInfSI{1,2}
\end{array}
\]

\paragraph{Eliminationsregel für das Infimum \(\inf_S(T)\)}
Die Eliminationsregel für das Infimum \(\inf_S(T)\) besagt, dass, wenn \( l = \inf_S(T) \), dann für jedes \( t \in T \) die Relation \( l \leq t \) gilt. Ebenso ist \( l \in S \).
\[
\begin{array}{llll}
    i & (1) & l = \inf_S(T) & \dots \\
    i & (2) & \forall t \in \text{LB}_S(T) \, (t \leq l) & \rInfSE{1} \\
    i & (3) & l \in \text{LB}_S(T) & \rInfSE{1}
\end{array}
\]

\(i\) und \(j\) sind dabei Listen von Annahmen.

\begin{definition}[Vollständigkeit einer halbgeordneten Menge]
    Eine halbgeordnete Menge \( S \) heißt \textbf{vollständig}, wenn:
    \begin{itemize}
        \item \(\forall T((T\subseteq S\land T \neq \emptyset\land \exists u \in \text{UB}_S(T)) \rightarrow \exists u = \sup_S(T))\)
        \item \(\forall T((T\subseteq S\land T \neq \emptyset\land \exists l \in \text{LB}_S(T)) \rightarrow \exists l = \inf_S(T))\)
    \end{itemize}
    Das Prädikat \(\text{Vollständig}(S)\) ist wahr, wenn \( S \) vollständig ist.
\end{definition}

\subsubsection{Einführungs- und Eliminationsregeln für die Vollständigkeit einer halbgeordneten Menge}
\label{rule:rCompleteI} \label{rule:rCompleteE}

\paragraph{Einführungsregel für die Vollständigkeit \(\text{Vollständig}(S)\)}
Die Einführungsregel für die Vollständigkeit besagt, dass \( S \) eine vollständige halbgeordnete Menge ist, wenn:

\[
\begin{array}{llll}
    i   & (1) & \forall T((T\subseteq S\land T \neq \emptyset\land \exists u \in \text{UB}_S(T)) \rightarrow \exists u = \sup_S(T)) & \dots \\
    j   & (2) & \forall T((T\subseteq S\land T \neq \emptyset\land \exists l \in \text{LB}_S(T)) \rightarrow \exists l = \inf_S(T)) & \dots \\
    i,j & (3) & S \text{ ist Vollständig} & \rCompleteI{1}
\end{array}
\]

\paragraph{Eliminationsregel für die Vollständigkeit \(\text{Vollständig}(S)\)}
Die Eliminationsregel für die Vollständigkeit besagt, dass, wenn \( S \) vollständig ist:

\[
\begin{array}{llll}
    i & (1) & \text{Vollständig}(S) & \dots \\
    i & (2) & \forall T((T\subseteq S\land T \neq \emptyset\land \exists u \in \text{UB}_S(T)) \rightarrow \exists u = \sup_S(T)) & \rCompleteE{1} \\
    i & (3) & \forall T((T\subseteq S\land T \neq \emptyset\land \exists l \in \text{LB}_S(T)) \rightarrow \exists l = \inf_S(T)) & \rCompleteE{1}
\end{array}
\]

\(i\) und \(j\) sind dabei Listen von Annahmen.


\chapter{Eigenschaften rekursiver Definitionen}
\section{Konventionen im Kapitel}
Im folgenden Kapitel vereinbaren wir, dass die Menge der natürlichen Zahlen mit \(\mathbb{N}\) bezeichnet wird und \(<\) sei die induzierter strikte Ordnung auf \(\mathbb{N}\). Ferner bezeichnen wir mit \(A, D\) Mengen und \(p\in\mathbb{N}\) und \(R\) als eine Relation auf \(A\). Ebenso sei \(r:A^p\rightarrow A\) eine Abbildung. Ferner sei ein rekursiv definiertes Symbol wie folgt durch \(f\) festgelegt:
\textbf{Basisfall}:
\[
    \sigma(a_1, \ldots, a_k) \coloneqq b \quad \text{für feste Werte } a_1, \ldots, a_k \text{ aus } D \text{ und } b\in D\text{.}
\]
\textbf{Rekursionsschritt}:
Für ein bereits definiertes \(\sigma(y_1, \ldots, y_m)\) sei
\[
    \sigma(x_1, \ldots, x_n) \coloneqq r\big(\sigma(y_1, \ldots, y_m), z_1, \ldots, z_{p-1}\big), \text{ mit } z_1,\ldots z_{p-1}\in A\text{.}
\]


\section{Maßfunktionen}
\begin{definition}[Maßfunktion]
 Eine \textbf{Maßfunktion} für die Relation \(R\subseteq A\times A\) ist eine spezielle Abbildung \(m:A\rightarrow\mathbb{N}\), die folgende Eigenschaft erfüllt: 
    \begin{itemize}
        \item \textbf{Strikte Abnahme bei rekursiven Aufrufen:} \newline
        \[\forall x,y\in A(m(y)<m(x)\rightarrow R(x,y))\]
    \end{itemize}
\end{definition}
\begin{remark}
Die Eigenschaften der Maßfunktion garantieren, dass es keine unendlichen absteigenden Ketten gibt und jede Rekursion nach endlich vielen Schritten terminiert. Dies folgt aus der Tatsache, dass das Bild der Maßfunktion \(m\) in \(\mathbb{N}\) wohlgeordnet ist.
\end{remark}
Aus der Definition der Maßfunktion leitet sich die folgende Schlussregel ab 
\[x,y\in A,m(y)<m(x)\vdash R(x,y)\]

\begin{definition}[Terminierung einer rekursiven Definition]
    Eine rekursive Definition \(\sigma(x_1, \ldots, x_n)\) \emph{terminiert}, wenn es eine Maßfunktion \(m: A \rightarrow \mathbb{N}\) gibt, sodass für jede Anwendung der Rekursionsvorschrift \(\sigma(x_1, \ldots, x_n) \coloneqq r\big(\sigma(y_1, \ldots, y_m), z_1, \ldots, z_{p-1}\big)\) die folgende Bedingung erfüllt ist:
    \[
        m(\sigma(y_1, \ldots, y_m)) < m(\sigma(x_1, \ldots, x_n))
    \]
\end{definition}
\begin{remark}
    Aufgrund der wohlgeordneten Struktur der natürlichen Zahlen \(\mathbb{N}\) und der strikten Abnahme der Maßfunktion \(m\) bei jedem rekursiven Aufruf kann keine unendliche absteigende Kette existieren. Somit terminiert die Rekursion nach endlich vielen Schritten.
\end{remark}




\begin{definition}[Rekursive Definition über \(\mathbb{N}\)]
Sei \(\sigma(n)\) ein \emph{neues} Symbol, das in einer formalen Sprache \(\mathcal{L}\) eingeführt werden soll und dessen Definitionsbereich die Menge der natürlichen Zahlen \(\mathbb{N}\) ist. Eine \textbf{rekursive Definition} von \(\sigma\) \textbf{mit genau einem Startwert} besteht analog zum Induktionsprinzip aus zwei Teilen:

\begin{enumerate}
    \item \textbf{Basisfall (Startwert):} Für den kleinsten Wert \(n=0\) (oder einen anderen festgelegten Wert in \(\mathbb{N}\)) wird \(\sigma(n)\) explizit festgelegt:
    \[
    \sigma(0) \coloneqq a,
    \]
    wobei \(a\) ein bereits bekannter Term oder Wert in \(\mathcal{L}\) ist.
    \item \textbf{Induktionsschritt (rekursive Vorschrift):} Für alle \(n \in \mathbb{N}\) definiert man
    \[
    \sigma(n+1) \coloneqq f\bigl(\sigma(n), n\bigr),
    \]
    wobei \(f\) eine in \(\mathcal{L}\) bereits definierte Funktion ist. Die Form \(f(\sigma(n), n)\) verdeutlicht, dass der Wert von \(\sigma\) an der Stelle \((n+1)\) durch den vorherigen Wert \(\sigma(n)\) bestimmt wird (und zusätzlich evtl. die Zahl \(n\) selbst verwenden kann).
\end{enumerate}

\noindent
In dieser Definition heißt:
\begin{itemize}
    \item \(\sigma\) \textbf{Definiendum} (das neu eingeführte Symbol),
    \item Die Basisfestlegung und die Rekursionsvorschrift zusammen bilden das \textbf{Definiens}.
\end{itemize}
\end{definition}

\subsubsection{Wohldefiniertheit der rekursiven Definition (ein Startwert)}
Damit eine solche rekursive Definition über \(\mathbb{N}\) wohldefiniert ist, muss sie einige Bedingungen erfüllen, die sich am Prinzip der vollständigen Induktion orientieren:

\begin{enumerate}
    \item \textbf{Terminierung:} Für jede natürliche Zahl \(n\) ist sichergestellt, dass \(\sigma(n)\) nach endlich vielen (induktiven) Schritten berechnet werden kann.\\
    \(\forall n \in \mathbb{N}: \text{ Der Wert } \sigma(n) \text{ wird in endlich vielen Schritten aus } \sigma(0) \text{ abgeleitet.}\)

    \item \textbf{Eindeutigkeit:} Für jede natürliche Zahl \(n\) darf es genau einen Wert \(\sigma(n)\) geben. Damit wird ausgeschlossen, dass derselbe Wert \(\sigma(n)\) auf zwei verschiedene Arten widersprüchlich definiert wird.\\
    \(\forall n \in \mathbb{N} \, \forall y, z : \bigl(\sigma(n) = y \land \sigma(n) = z \implies y = z\bigr).\)

    \item \textbf{Konsistenz:} Die Rekursionsvorschrift darf keinen Widerspruch hervorrufen, das heißt, es darf nicht geschehen, dass \(\sigma(n)\) und \(\neg \sigma(n)\) zugleich vorliegen.\\
    \(\forall n \in \mathbb{N} : \neg \bigl(\sigma(n) \land \neg \sigma(n)\bigr).\)

    \item \textbf{Vollständigkeit:} Für jede natürliche Zahl \(n\) muss \(\sigma(n)\) tatsächlich definiert sein oder sich mittels der Rekursionsvorschrift bestimmen lassen.\\
    \(\forall n \in \mathbb{N} : \sigma(n) \lor \neg \sigma(n) \quad \text{(wobei \(\neg\sigma(n)\) hier nur symbolisch für „nicht definiert“ stehen kann).}\)

    \item \textbf{Monotonie bzw. „schrittweise“ Rekursion:} Die Bestimmung von \(\sigma(n+1)\) bezieht sich nur auf „kleinere“ Argumente (hier \(\sigma(n)\)), sodass ein Abbruch der Definitionskette (Rückgriff bis \(\sigma(0)\)) immer möglich ist. Diese Bedingung stellt sicher, dass die Rekursion nicht unendlich rückwärts fortgesetzt wird.
\end{enumerate}

\begin{remark}
Die in Punkt (5) genannte Monotonie schließt nahtlos an das Prinzip der vollständigen Induktion an: Man überprüft zunächst den Basisfall (\(\sigma(0)\)), um danach sicherzustellen, dass \(\sigma(n)\) stets aus einem bereits definierten Wert (\(\sigma(n-1)\) oder allgemein \(\sigma(k)\) mit \(k<n\)) hervorgeht. Dadurch ist gewährleistet, dass jedes \(\sigma(n)\) in \(\mathbb{N}\) in endlich vielen Schritten definiert werden kann.
\end{remark}


\chapter{Addition von natürlichen Zahlen}

\textbf{Index der Sätze und Definitionen:}


\begin{definition}[Addition]
    Die Addition von zwei natürlichen Zahlen \( a \) und \( b \) ist eine binäre Operation \( +: \mathbb{N} \times \mathbb{N} \to \mathbb{N} \), die rekursiv wie folgt definiert wird:
    
    \begin{itemize}
        \item \textbf{Basisfall}: Für jede natürliche Zahl \( a \) gilt:
        \[
        a + 0 := a.
        \]
        
        \item \textbf{Rekursionsschritt}: Für jede natürliche Zahl \( b \) gilt:
        \[
        a + (b+1) := (a + b) + 1.
        \]
    \end{itemize}
\end{definition}

\begin{remark}
In der Definition von \( a + (b+1) \) sorgt die Klammer um \( b+1 \) dafür, dass \( b \) zuerst um 1 erhöht wird, bevor diese Summe zu \( a \) addiert wird. Es ist wichtig zu beachten, dass die Klammer in \( (a + b) + 1 \) verwendet wird, um die Reihenfolge der Operationen zu verdeutlichen: Zuerst wird \( a + b \) berechnet, und dann wird 1 hinzugefügt. Wird jedoch der gesamte Ausdruck \( (a + b) + 1 \) durch eine Variable \( y \) ersetzt, also \( y = (a + b) + 1 \), dann kann die Klammer um \( y \) weggelassen werden, da \( y \) eine einzelne Zahl darstellt und keine Mehrdeutigkeit besteht.
\end{remark}

\paragraph{Beweisregeln für die Addition}
\label{rule:rAddI} 
Basierend auf diesen Definitionen können wir folgende Regel für die Addition formulieren. 

\[
\begin{array}{llll}
	i & (1) & a \in \mathbb{N} & ... \\
	j & (2) & b \in \mathbb{N} & ... \\
        i & (3) & a = a + 0 & \rAddI{1} \\
	i,j & (4) & a + (b+1) = (a+b)+1 & \rAddI{1,2} \\
            & (5) & +:\mathbb{N}\times\mathbb{N}\rightarrow\mathbb{N} & \rAddI{} \\
\end{array}
\]

\(i\) und \(j\) sind dabei Listen von Annahmen.

\label{aInNaturalwbInNaturalImpaPlusbInNatural}
\begin{theorem}[\(a\in\mathbb{N}, b\in\mathbb{N}\vdash a+b\in\mathbb{N}\) (Wohldefiniertheit der Addition)]
\end{theorem}
\begin{proof}
	\[
	\begin{array}{llll}
		1   &  (1) & a\in\mathbb{N} & \rA \\
            2   &  (2) & b\in\mathbb{N} & \rA \\
            1   &  (3) & a=a+0 & \rAddI{1} \\
            1   &  (4) & a+0\in\mathbb{N} & \rIE{3,1} \\
            5   &  (5) & n\in \mathbb{N} & \rA \\
            6   &  (6) & a+n\in \mathbb{N} & \rA \\
            5,6 &  (7) & a+(n+1)=(a+n)+1 & \rAddI{5,6} \\
            6   &  (8) & (a+n)+1\in\mathbb{N} & \successorIsNaturalNumber{6} \\
            5,6 &  (9) & a+(n+1)\in\mathbb{N} & \rIE{7,8} \\
            1   &  (10) & \forall n\in\mathbb{N}(a+n\in\mathbb{N}) & \rInductionN{4,5,6,9} \\
            1   &  (11) & b\in\mathbb{N}\rightarrow a+b\in\mathbb{N} & \rSetUEb{10} \\
            1,2 &  (12) & a+b\in\mathbb{N} & \rRE{11,2} \\
	\end{array}
	\]
\end{proof}

\label{aInNaturalImpaEqualsZeroPlusa}
\begin{theorem}[\(a\in\mathbb{N}\vdash a=0+a\) (Neutrales Element)]
\end{theorem}
\begin{proof}
        \[
	\begin{array}{llll}
            1   &  (1) & a\in\mathbb{N} & \rA \\
                &  (2) & 0\in\mathbb{N} & \zeroIsNaturalNumber \\
                &  (3) & 0=0+0 & \rAddI{2} \\
            4   &  (4) & n\in\mathbb{N} & \rA \\
            5   &  (5) & n=0+n & \rA \\
            4   &  (6) & 0+(n+1)=(0+n)+1 & \rAddI{2,4} \\
            4,5 &  (7) & 0+(n+1)=n+1 & \rIE{5,6} \\
            4,5 &  (8) & n+1=0+(n+1) & \aIdbImpbIda{7} \\
                &  (9) & \forall n\in\mathbb{N}(n=0+n) & \rInductionN{3,4,5,8} \\
                &  (10) & a\in\mathbb{N}\rightarrow a=0+a & \rSetUEb{9} \\
            1   &  (11) & a=0+a & \rRE{1,10} \\
	\end{array}
	\]
\end{proof}

\label{aInNaturalImpaPlusZeroEqualsZeroPlusa}
\begin{theorem}[\(a\in\mathbb{N}\vdash a+0=0+a\)]
\end{theorem}
\begin{proof}
        \[
	\begin{array}{llll}
            1   &  (1) & a\in\mathbb{N} & \rA \\
            1   &  (2) & a = a+0 & \rAddI{1} \\
            1   &  (3) & a = 0+a & \aInNaturalImpaEqualsZeroPlusa{1} \\
            1   &  (4) & a+0 = 0+a & \rIE{2,3} \\
	\end{array}
	\]
\end{proof}


\label{aInNaturalImpZeroPlusaEqualsaPlusZeroEqualsa}
\begin{theorem}[\(a\in\mathbb{N}\vdash 0+a=a+0=a\) (Neutrales Element)]
\end{theorem}
\begin{proof}
        \[
	\begin{array}{llll}
            1   &  (1) & a\in\mathbb{N} & \rA \\
            1   &  (2) & 0+a=a+0 & \aInNaturalImpaPlusZeroEqualsZeroPlusa{1} \\
            1   &  (3) & a+0=0+a & \aIdbImpbIda{2} \\
            1   &  (4) & a=0+a & \aInNaturalImpaEqualsZeroPlusa{1} \\
            1   &  (5) & 0+a=a & \aIdbImpbIda{4} \\
            1   &  (6) & a+0=a & \rIE{5,3} \\
            1   &  (7) & a+0=0+a\land a+0=a & \rAI{3,6} \\
            1   &  (8) & a+0=0+a=a & \rIIb{7} \\
	\end{array}
	\]
\end{proof}


\label{aInNaturalImpOnePlusaEqualsaPlusOne}
\begin{theorem}[\(a\in\mathbb{N}\vdash 1+a=a+1\)]
\end{theorem}
\begin{proof}
        \[
	\begin{array}{llll}
            1   &  (1) & a\in\mathbb{N} & \rA \\
                &  (2) & 1\in\mathbb{N} & \oneIsNaturalNumber{} \\
                &  (3) & 1+0=0+1 & \aInNaturalImpaPlusZeroEqualsZeroPlusa{2} \\
            4   &  (4) & n\in\mathbb{N} & \rA \\
            5   &  (5) & 1+n=n+1 & \rA \\
            4   &  (6) & 1+(n+1)=(1+n)+1 & \rAddI{4,2} \\
            4,5 &  (7) & 1+(n+1)=(n+1)+1 & \rEE{4,5} \\
                &  (8) & \forall n\in\mathbb{N}(1+n=n+1) & \rInductionN{3,4,5,7} \\
                &  (9) & a\in\mathbb{N}\rightarrow 1+a=a+1 & \rSetUEb{8} \\
            1   &  (10)&1+a=a+1 & \rRE{1,9} \\
	\end{array}
	\]
\end{proof}

\label{aInNaturalwbInNaturalwcInNaturalImpaPlusLpbPluscRpEqualsLpaPlusbRpPlusc}
\begin{theorem}[\(a\in\mathbb{N}, b\in\mathbb{N}, c\in\mathbb{N}\vdash a+(b+c)=(a+b)+c\) (Assoziativgesetz)]
\end{theorem}
\begin{proof}
    \[
	\begin{array}{llll}
            1   &  (1) & a\in\mathbb{N} & \rA \\
            2   &  (2) & b\in\mathbb{N} & \rA \\
            3   &  (3) & c\in\mathbb{N} & \rA \\
            2   &  (4) & b = b + 0 & \aInNaturalImpZeroPlusaEqualsaPlusZeroEqualsa{2}  \\
            1,2 &  (5) & a+b\in\mathbb{N} & \aInNaturalwbInNaturalImpaPlusbInNatural{1,2}  \\
            1,2  &  (6) & a+b = (a+b) + 0 & \aInNaturalImpZeroPlusaEqualsaPlusZeroEqualsa{5}  \\
            1,2  &  (7) & a+(b+0) = (a+b) + 0 & \rIE{4,6}  \\
            8  &  (8) & n\in\mathbb{N} & \rA  \\
            9  &  (9) & a+(b+n)=(a+b)+n & \rA  \\
            1,2,8  &  (10) & (a+b)+(n+1) = ((a+b)+n)+1 & \rAddI{5,8}  \\
            1,2,8,9  &  (11) & (a+b)+(n+1) = (a+(b+n))+1 & \rIE{9,10}  \\
            2,8  &  (12) & b+n\in\mathbb{N} & \aInNaturalwbInNaturalImpaPlusbInNatural{2,8}  \\
            1,2,8  &  (13) & a+((b+n)+1)=(a+(b+n))+1 & \rAddI{1,12}  \\
            1,2,8,9  &  (14) & (a+b)+(n+1)=a+((b+n)+1) & \rIE{13,11}  \\
            2,8    &  (15) & b+(n+1)=(b+n)+1 & \rAddI{2,8}  \\
            1,2,8,9    &  (16) & (a+b)+(n+1)=a+(b+(n+1)) & \rIE{15,14}  \\
            1,2,8,9    &  (17) & a+(b+(n+1))=(a+b)+(n+1) & \aIdbImpbIda{16}  \\
            1,2    &  (18) & \forall n\in\mathbb{N}(a+(b+n)=(a+b)+n) & \rInductionN{7,8,9,17}  \\
            1,2    &  (19) & c\in\mathbb{N}\rightarrow (a+(b+c)=(a+b)+c) & \rSetUEb{18}  \\
            1,2,3   &  (20) & (a+(b+c)=(a+b)+c) & \rRE{3,19}  \\
	\end{array}
	\]
\end{proof}

\label{aInNaturalwbInNaturalImpaPlusbEqualsbPlusa}
\begin{theorem}[\(a\in\mathbb{N}, b\in\mathbb{N}\vdash a+b=b+a\) (Kommutativgesetz)]
\end{theorem}
\begin{proof}
        \[
	\begin{array}{llll}
            1       &  (1) & a\in\mathbb{N} & \rA \\
            2       &  (2) & b\in\mathbb{N} & \rA \\
            1       &  (3) & a+0=0+a & \aInNaturalImpaPlusZeroEqualsZeroPlusa{1} \\
            4       &  (4) & n\in\mathbb{N} & \rA \\
            5       &  (5) & a+n=n+a & \rA \\
            1,4     &  (6) & a+(n+1)=(a+n)+1 & \rAddI{1,4} \\
            1,4,5   &  (7) & a+(n+1)=(n+a)+1 & \rIE{5,6} \\
            1,4     &  (8) & n+a\in\mathbb{N} & \aInNaturalwbInNaturalImpaPlusbInNatural{4,1} \\
            1,4     &  (9) & 1+(n+a)=(n+a)+1 & \aInNaturalImpOnePlusaEqualsaPlusOne{8} \\
            1,4,5   &  (10) & a+(n+1)=1+(n+a) & \rIE{9,7} \\
                    &  (11) & 1\in\mathbb{N} & \oneIsNaturalNumber{} \\
            1,4     &  (12) & 1+(n+a)=(1+n)+a & \aInNaturalwbInNaturalwcInNaturalImpaPlusLpbPluscRpEqualsLpaPlusbRpPlusc{11,4,1} \\
            1,4,5   &  (12) & a+(n+1)=(1+n)+a & \rIE{12,10} \\
            4       &  (13) & 1+n=n+1 & \aInNaturalImpOnePlusaEqualsaPlusOne{4} \\
            1,4,5   &  (14) & a+(n+1)=(n+1)+a & \rIE{13,12} \\
            1       &  (15) & \forall n\in\mathbb{N}(a+n=n+a) & \rInductionN{3,4,5,14} \\
            1       &  (16) & b\in\mathbb{N}\rightarrow a+b=b+a & \rSetUEb{15} \\
            1,2     &  (17) & a+b=b+a & \rRE{2,16} \\
	\end{array}
	\]
\end{proof}

\label{ImpLpNaturalwPluswZeroRpInMonoid}
\begin{theorem}[\(\vdash (\mathbb{N},+,0) \text{ ist ein Monoid.}\)]
\end{theorem}
\begin{proof}
        \[
	\begin{array}{llll}
                &  (1) & +:\mathbb{N}\times\mathbb{N}\rightarrow\mathbb{N} & \rAddI{} \\
                &  (2) & \forall a\in\mathbb{N}\forall b\in\mathbb{N}\forall c\in\mathbb{N}(a+(b+c)=(a+b)+c) & \aInNaturalwbInNaturalwcInNaturalImpaPlusLpbPluscRpEqualsLpaPlusbRpPlusc{} \\
                &  (3) & \forall a\in\mathbb{N}(a+0=0+a=a) & \aInNaturalImpZeroPlusaEqualsaPlusZeroEqualsa{} \\
                &  (4) & (\mathbb{N},+,0) \text{ ist ein Monoid.} & \rMonoidI{1,2,3} \\
	\end{array}
	\]
\end{proof}

\label{ImpLpNaturalwPluswZeroRpInAbelMonoid}
\begin{theorem}[\(\vdash (\mathbb{N},+,0) \text{ ist ein abelscher Monoid.}\)]
\end{theorem}
\begin{proof}
        \[
	\begin{array}{llll}
                &  (1) & (\mathbb{N},+,0) \text{ ist ein Monoid.} & \ImpLpNaturalwPluswZeroRpInMonoid{} \\
                &  (2) & \forall a\in\mathbb{N}\forall b\in\mathbb{N}(a+b=b+a) & \aInNaturalwbInNaturalwcInNaturalImpaPlusLpbPluscRpEqualsLpaPlusbRpPlusc{} \\
                &  (3) & (\mathbb{N},+,0) \text{ ist ein abelscher Monoid.} & \rAbelianMonoidI{1,2} \\
	\end{array}
	\]
\end{proof}

\section{Erweiterte Vertauschungsgesetze der Addition}

\label{aInNaturalwbInNaturalwcInNaturalImpLpaPlusbRpPluscEqualsLpaPluscRpPlusb}
\begin{theorem}[\(a\in\mathbb{N}, b\in\mathbb{N}, c\in\mathbb{N} \vdash (a+b)+c=(a+c)+b\)]
\end{theorem}
\begin{proof}
	\[
	\begin{array}{lll p{5cm}}
		1         &  (1) & a\in\mathbb{N} & \rA \\
		2         &  (2) & b\in\mathbb{N} & \rA \\
		3         &  (3) & c\in\mathbb{N} & \rA \\
		1,2,3     &  (4) & a+(b+c)=(a+b)+c & \aInNaturalwbInNaturalwcInNaturalImpaPlusLpbPluscRpEqualsLpaPlusbRpPlusc{1,2,3} \\
		2,3       &  (5) & b+c=c+b & \aInNaturalwbInNaturalImpaPlusbEqualsbPlusa{2,3} \\        
		             &  (6) & a+(b+c) = a+(b+c) & \rII{} \\       
        2,3       &  (7) & a+(b+c) = a+(c+b) & \rIE{5,6} \\       
        1,2,3     &  (8) & a+(c+b)=(a+c)+b & \aInNaturalwbInNaturalwcInNaturalImpaPlusLpbPluscRpEqualsLpaPlusbRpPlusc{1,3,2} \\       
        1,2,3     &  (9) & a+(b+c)=a+(c+b) & \rIE{7,6} \\       
        1,2,3     &  (10) & a+(b+c)=(a+c)+b & \rIE{8,9} \\       
        1,2,3     &  (11) & (a+b)+c=(a+c)+b & \rIE{6,10} \\       
	\end{array}
	\]
\end{proof}

\label{aInNaturalwbInNaturalwcInNaturalwdInNaturalImpLpaPlusbRpPlusLpcPlusdRpEqualsLpaPluscRpPlusLpbPlusdRp}
\begin{theorem}[\(a\in\mathbb{N}, b\in\mathbb{N}, c\in\mathbb{N}, d\in\mathbb{N} \vdash (a+b)+(c+d)=(a+c)+(b+d)\)]
\end{theorem}
\begin{proof}
	\[
	\begin{array}{lll p{5cm}}
		1         &  (1) & a\in\mathbb{N} & \rA \\
		2         &  (2) & b\in\mathbb{N} & \rA \\
		3         &  (3) & c\in\mathbb{N} & \rA \\
		4         &  (4) & d\in\mathbb{N} & \rA \\
		1,2       &  (5) & a+b \in \mathbb{N} & \aInNaturalwbInNaturalImpaPlusbInNatural{1,2} \\
		1,3       &  (6) & a+c \in \mathbb{N} & \aInNaturalwbInNaturalImpaPlusbInNatural{1,3} \\
		1,2,3,4   &  (7) & (a+b)+(c+d)=((a+b)+c)+d & \aInNaturalwbInNaturalwcInNaturalImpaPlusLpbPluscRpEqualsLpaPlusbRpPlusc{5,3,4} \\
		  1,2,3     &  (8) & (a+b)+c = (a+c)+b & \aInNaturalwbInNaturalwcInNaturalImpLpaPlusbRpPluscEqualsLpaPluscRpPlusb{1,2,3} \\
    	1,2,3     &  (9) & (a+b)+(c+d)=((a+c)+b)+d & \rIE{8,7} \\
        1,2,3     & (10) & ((a+c)+b)+d=(a+c)+(b+d) &  \aInNaturalwbInNaturalwcInNaturalImpaPlusLpbPluscRpEqualsLpaPlusbRpPlusc{6,2,4}\\
        1,2,3     & (11) & (a+b)+(c+d)=(a+c)+(b+d) &  \rIE{10,9}\\
	\end{array}
	\]
\end{proof}

\section{Weitere Eigenschaften der Addition}

\label{aNotEqualsZeroImpaPlusbNotEqualsZero}
\begin{theorem}[\(a\neq 0\vdash a+b\neq 0\)]
Seien \(a,b\in\mathbb{N}\), dann gilt:
\[a\neq 0\vdash a+b\neq 0\]
\end{theorem}
\begin{proof}
Im Beweis nutzen wir die Eigenschaft, dass \((\mathbb{N},+,0)\) ein Monoid ist. Wir führen eine Induktion über \(n\in\mathbb{N}\):
        \[
	\begin{array}{llll}
            1       &  (1)  & a\neq 0 & \rA \\
                    &  (2)  & a+0=a & \rNeutralElementMonoid{} \\
            1       &  (3)  & a+0\neq 0 & \rIE{2,1} \\
            4       &  (4)  & a+n\neq 0 & \rA \\
                    &  (5)  & (a+n)+1\neq 0 & \nInNaturalImpnPlusOneNotEqualsZero{} \\       
                    &  (6)  & (a+n)+1=a+(n+1) & \rAssociativityMonoid{} \\      
                    &  (7)  & a+(n+1)\neq 0 & \rIE{6,5} \\     
            1       &  (8)  & \forall n\in\mathbb{N}(a+n\neq 0) & \rInductionN{3,4,7} \\  
            1       &  (9)  & a+b\neq 0 & \rSetUEc{8} \\  
	\end{array}
	\]
\end{proof}

\label{bNotEqualsZeroImpaPlusbNotEqualsZero}
\begin{theorem}[\(b\neq 0\vdash a+b\neq 0\)]
Seien \(a,b\in\mathbb{N}\), dann gilt:
\[b\neq 0\vdash a+b\neq 0\]
\end{theorem}
\begin{proof}
Im Beweis nutzen wir die Eigenschaft, dass \((\mathbb{N},+,0)\) ein abelscher Monoid ist. 
        \[
	\begin{array}{llll}
            1       &  (1)  & b\neq 0 & \rA \\
            1       &  (2)  & b+a\neq 0 & \aNotEqualsZeroImpaPlusbNotEqualsZero{1} \\
                    &  (3)  & a+b=b+a & \rCommutativeMonoid{} \\
            1       &  (4)  & a+b\neq 0 & \rIE{3,2} \\
	\end{array}
	\]
\end{proof}




\label{aEqualsbEqvaPluscEqualsbPlusc}
\begin{theorem}[\( a=b\dashv \vdash a+c=b+c\)]
Seien \(a,b,c\in\mathbb{N}\), dann gilt:
\[a=b\dashv\vdash a+c=b+c\]
\end{theorem}
\begin{proof}
Seien \(a,b,c\in\mathbb{N}\).
\(\vdash:\)
        \[
	\begin{array}{llll}
            1       &  (1)  & a=b & \rA \\
                    &  (2)  & a+c=a+c & \rII{} \\
            1       &  (3)  & a+c=b+c & \rIE{1,2} \\       
	\end{array}
	\]
 \(\dashv:\)
        Sei \(n\in\mathbb{N}\) eine natürliche Zahl, über die wir die Induktion führen, dann gilt:
        \[
	\begin{array}{llll}
            1       &  (1)  & a+c=b+c & \rA \\
            2       &  (2)  & a+0=b+0 & \rA \\
                    &  (3)  & a+0=a & \rNeutralElementMonoid{} \\
                    &  (4)  & b+0=b & \rNeutralElementMonoid{}  \\
            2       &  (5)  & a=b+0 & \rIE{3,2} \\       
            2       &  (6)  & a=b & \rIE{4,5} \\   
                    &  (7)  & a+0=b+0\rightarrow a=b & \rRI{2,6} \\
            8       &  (8)  & a+n=b+n\rightarrow a=b & \rA \\  
            9       &  (9)  & a+(n+1)=b+(n+1) & \rA \\  
                    &  (10)  & a+(n+1)=(a+n)+1 & \rAssociativityMonoid{} \\
                    &  (11)  & b+(n+1)=(b+n)+1 & \rAssociativityMonoid{} \\
            9       &  (12)  & (a+n)+1=b+(n+1) & \rIE{10,9} \\
            9       &  (13)  & ((a+n)+1)-1=((b+n)+1)-1 & \rPredecessorUniqueness{13}\\ 
                    &  (14)  & a+n=((a+n)+1)-1 & \rPredecessorEc{}\\
                    &  (15)  & b+n=((b+n)+1)-1 & \rPredecessorEc{}\\
            9       &  (16)  & a+n=((b+n)+1)-1 & \rIE{14,13}\\ 
            9       &  (17)  & a+n=b+n & \rIE{15,13}\\ 
            8,9     &  (18)  & a=b & \rRE{8,17}\\ 
            8       &  (19)  & a+(n+1)=b+(n+1)\rightarrow a=b & \rRI{9,18}\\ 
                    &  (20)  &  \forall n\in\mathbb{N}(a+n=b+n\rightarrow a=b) & \rInductionN{7,8,19}\\ 
                    &  (21)  &  a+c=b+c\rightarrow a=b & \rSetUEc{}\\
            1       &  (22)  &   a=b & \rRE{1,21}\\ 
	\end{array}
	\]
\end{proof}


\label{aNotEqualsbEqvaPluscNotEqualsbPlusc}
\begin{theorem}[\(a\neq b\dashv \vdash a+c\neq b+c\)]
Seien \(a,b,c\in\mathbb{N}\), dann gilt:
\[a\neq b\dashv\vdash a+c\neq b+c\]
\end{theorem}
\begin{proof}
Seien \(a,b,c\in\mathbb{N}\).
\(\vdash:\)
        \[
	\begin{array}{llll}
            1       &  (1)  & a\neq b & \rA \\
                    &  (2)  & a=b\leftrightarrow a+c=b+c & \aEqualsbEqvaPluscEqualsbPlusc{} \\
            1       &  (3)  & a+c=b+c & \PLrQwnPImpnQ{2,1} \\       
	\end{array}
	\]
 \(\dashv:\)
        \[
	\begin{array}{llll}
            1       &  (1)  & a+c\neq b+c & \rA \\
                    &  (2)  & a=b\leftrightarrow a+c=b+c & \aEqualsbEqvaPluscEqualsbPlusc{} \\
            1       &  (3)  & a\neq b & \PLrQwnQImpnP{2,1} \\    
	\end{array}
	\]
\end{proof}

\label{aEqualsbEqvcPlusaEqualscPlusb}
\begin{theorem}[\(a=b\dashv \vdash c+a=c+b\)]
Seien \(a,b,c\in\mathbb{N}\), dann gilt:
\[a=b\dashv\vdash c+a=c+b\]
\end{theorem}
\begin{proof}
Im Beweis nutzen wir \(\ImpLpNaturalwPluswZeroRpInAbelMonoid{}\).

\(\vdash:\)
    \[
	\begin{array}{llll}
            1       &  (1)  & a=b & \rA \\
            1       &  (2)  & a+c=b+c & \aEqualsbEqvaPluscEqualsbPlusc{1} \\
                    &  (3)  & a+c=c+a & \rCommutativeMonoid{} \\     
                    &  (4)  & b+c=c+b & \rCommutativeMonoid{} \\ 
            1       &  (5)  & c+a=b+c & \rIE{3,2} \\
            1       &  (6)  & c+a=c+b & \rIE{4,5} \\        
	\end{array}
	\]
 \(\dashv:\)
    \[
	\begin{array}{llll}
            1       &  (1)  & c+a=c+b & \rA \\
                    &  (2)  & a+c=c+a & \rCommutativeMonoid{} \\     
                    &  (3)  & b+c=c+b & \rCommutativeMonoid{} \\ 
            1       &  (4)  & a+c=c+b & \rIE{2,1} \\
            1       &  (5)  & a+c=b+c & \rIE{3,4} \\        
            1       &  (6)  & a=b & \aEqualsbEqvaPluscEqualsbPlusc{5} \\        
	\end{array}
	\]
\end{proof}

\label{aNotEqualsbEqvcPlusaNotEqualscPlusb}
\begin{theorem}[\(a\neq b\dashv \vdash c+a\neq c+b\)]
Seien \(a,b,c\in\mathbb{N}\), dann gilt:
\[a\neq b\dashv\vdash c+a\neq c+b\]
\end{theorem}
\begin{proof}
Seien \(a,b,c\in\mathbb{N}\).
\(\vdash:\)
        \[
	\begin{array}{llll}
            1       &  (1)  & a\neq b & \rA \\
                    &  (2)  & a=b\leftrightarrow c+a=c+b & \aEqualsbEqvcPlusaEqualscPlusb{} \\
            1       &  (3)  & c+a=c+b & \PLrQwnPImpnQ{2,1} \\       
	\end{array}
	\]
 \(\dashv:\)
        \[
	\begin{array}{llll}
            1       &  (1)  & a+c\neq b+c & \rA \\
                    &  (2)  & a=b\leftrightarrow c+a=c+b & \aEqualsbEqvcPlusaEqualscPlusb{} \\
            1       &  (3)  & a\neq b & \PLrQwnQImpnP{2,1} \\    
	\end{array}
	\]
\end{proof}

\label{aInNaturalwbInNaturalwaPlusbEqualsaImpbEqualsZero}
\begin{theorem}[\(a\in\mathbb{N}, b\in\mathbb{N}, a+b=a\vdash b=0\)]
\end{theorem}
\begin{proof}
        \[
	\begin{array}{llll}
            1       &  (1)  & a\in\mathbb{N} & \rA \\
            2       &  (2)  & b\in\mathbb{N} & \rA \\
            3       &  (3)  & a+b=a & \rA \\
            4       &  (4)  & 0+b=0 & \rA \\
            2       &  (5)  & b=0+b & \aInNaturalImpaEqualsZeroPlusa{2} \\
            2,4     &  (6)  & b=0 & \rIE{5,4} \\
            2       &  (7)  & 0+b=0\rightarrow b=0 & \rRI{4,6} \\
            8       &  (8)  & n+b=n\rightarrow b=0 & \rA \\
            9       &  (9)  & n\in\mathbb{N} & \rA \\
            10      &  (10)  & (n+1)+b=n+1 & \rA \\
                    &  (11)  & 1\in\mathbb{N} & \oneIsNaturalNumber{} \\
            2,9     &  (12) & (n+1)+b=(n+b)+1 & \aInNaturalwbInNaturalwcInNaturalImpLpaPlusbRpPluscEqualsLpaPluscRpPlusb{9,11,2} \\
            2,9     &  (13) & n+b\in\mathbb{N} & \aInNaturalwbInNaturalImpaPlusbInNatural{9,2} \\
            2,9     &  (14) & n+b=((n+b)+1)-1 & \rPredecessorEa{13} \\
            2,9     &  (15) & n=(n+1)-1 & \rPredecessorEc{9} \\
            9       &  (16) & n+1\in\mathbb{N} & \aInNaturalwbInNaturalImpaPlusbInNatural{9,11} \\
            2,9     &  (17) & (n+1)+b\in\mathbb{N} & \aInNaturalwbInNaturalImpaPlusbInNatural{16,2} \\
            9       &  (18) & n+1\neq 0 & \nInNaturalImpnPlusOneNotEqualsZero{9} \\
            2,9,10  &  (19) & ((n+1)+b)-1=(n+1)-1 & \rPredecessorUniqueness{16,17,18,10} \\
            2,9,10  &  (20) & ((n+1)+b)-1=n & \rIE{15,19} \\
            2,9,10  &  (21) & ((n+b)+1)-1=n & \rIE{12,20} \\
            2,9,10  &  (22) & n+b=n & \rIE{14,21} \\
            2,8,9,10&  (23) & b=0 & \rRE{22,8} \\
            2,8,9   &  (24) & (n+1)+b=n+1\rightarrow b=0 & \rRI{10,23} \\
            2       &  (25) & \forall n\in\mathbb{N}(n+b=n\rightarrow b=0) & \rInductionN{7,9,8,24} \\
            2       &  (26) & a\in\mathbb{N}\rightarrow (n+b=n\rightarrow b=0) & \rSetUEb{25} \\
            1,2     &  (27) & a+b=a\rightarrow b=0 & \rRE{1,26} \\
            28      &  (28) & a+b=a & \rA \\
            1,2,28  &  (29) & b=0 & \rRE{28,27} \\
	\end{array}
	\]
\end{proof}

\label{aInNaturalwbInNaturalwaPlusbEqualsZeroImpaEqualsZeroAndbEqualsZero}
\begin{theorem}[\(a\in\mathbb{N}, b\in\mathbb{N}, a+b=0\vdash a=0\land b=0\)]
\end{theorem}
\begin{proof}
        \[
	\begin{array}{llll}
            1       &  (1)  & a\in\mathbb{N} & \rA \\
            2       &  (2)  & b\in\mathbb{N} & \rA \\
            3       &  (3)  & a+b=0 & \rA \\
            4       &  (4)  & \neg(a=0\land b=0) & \rA \\
            4       &  (5)  & a\neq 0\lor b\neq 0 & \nLpPAndQRpEqvnPOrnQ{4} \\
            6       &  (6)  & a\neq 0 & \rA \\
            1,6     &  (7)  & \exists x\in\mathbb{N}(x+1=a) & \mInNaturalwmNotEqualsZeroImpExxInNaturalLpxPlusOneEqualsmRp{1,6} \\
            1,6     &  (8)  & n\in\mathbb{N}\land n+1=a & \rSetEEa{7} \\
            1,6     &  (9)  & n\in\mathbb{N} & \rAEa{8} \\
            1,6     &  (10) & n+1=a & \rAEb{8} \\
            1,3,6   &  (11) & (n+1)+b=0 & \rIE{10,3} \\
                    &  (12) & 1\in\mathbb{N} & \oneIsNaturalNumber{} \\
            1,3,6   &  (13) & (n+1)+b=(n+b)+1 & \aInNaturalwbInNaturalwcInNaturalImpLpaPlusbRpPluscEqualsLpaPluscRpPlusb{1,12,2} \\
            1,2,6   &  (14) & n+b\in\mathbb{N} & \aInNaturalwbInNaturalImpaPlusbInNatural{9,2} \\
            1,2,6   &  (15) & (n+b)+1\neq 0 & \nInNaturalImpnPlusOneNotEqualsZero{14} \\
            1,2,3,6 &  (16) & (n+1)+b\neq 0\in & \rIE{13,15} \\
            1,2,3,6 &  (17) & a+b\neq 0 & \rIE{10,16} \\
            1,2,3,6 &  (18) & \bot & \rBI{3,17} \\
            1,2,3   &  (19) & a=0 & \rCE{6,18} \\
            1,2,3   &  (20) & 0+b=0 & \rIE{19,3} \\
            1,2,3   &  (21) & b=0+b & \aInNaturalImpaEqualsZeroPlusa{2} \\
            1,2,3   &  (22) & b=0 & \rIE{21,20} \\
            1,2,3   &  (23) & a=0\land b=0 & \rAI{19,22} \\
	\end{array}
	\]
\end{proof}

\chapter{Ordnungsrelationen und Differenzen für natürliche Zahlen}

\begin{definition}[Kleiner-gleich (\( \leq \))]
Für natürliche Zahlen \( a, b \in \mathbb{N} \) gilt:

\[a\leq b:=\exists c\in\mathbb{N}(a+c=b)\]
\end{definition}

\label{rule:rLeqNI} \label{rule:rLeqNE}
\paragraph{Beweisregeln für \( \leq \)}
Basierend auf der Definition der Kleiner-gleich-Relation (\( \leq \)) können wir die folgenden Regeln formulieren:

\[
\begin{array}{llll}
	i   & (1)      & a \in \mathbb{N}                  & ... \\
	j   & (2)      & b \in \mathbb{N}                  & ... \\ 
	k   & (3)      & \exists c\in\mathbb{N}(a + c = b) & ... \\
	i,j,k & (4)    & a \leq b                          & \rLeqNI{1,2,3} \\
\end{array}
\]

\[
\begin{array}{llll}
	i   & (1)      & a \in \mathbb{N}                  & ... \\
	j   & (2)      & b \in \mathbb{N}                  & ... \\ 
        k   & (3)      & c \in \mathbb{N}                  & ... \\ 
	l   & (3)      & a + c = b & ... \\
	i,j,k,l & (4)    & a \leq b                          & \rLeqNI{1,2,3,4} \\
\end{array}
\]

\[
\begin{array}{llll}
	i      & (1)    & a \in \mathbb{N}                  & ... \\
	j      & (2)    & b \in \mathbb{N}                  & ... \\ 
	k      & (3)    & a\leq b & ... \\
	i,j,k  & (4)    & \exists c\in\mathbb{N}(a + c = b) & \rLeqNE{1,2,3} \\
\end{array}
\]
\( i \), \( j \), \( k \) und \(l\) sind dabei Listen von Annahmen.

\section{Eindeutigkeit der Ordnungsrelation}

\label{ExcSubOnewcSubTwoInNaturalLpaPluscSubOneEqualsbAndaPluscSubTwoEqualsbRpImpcSubOneEqualscSubTwo}
\begin{theorem}[\(\exists c_1, c_2 \in \mathbb{N} (a + c_1 = b \land a + c_2 = b) \vdash c_1 = c_2\) (Eindeutigkeit der Differenz)]
    Seien \( a, b \in \mathbb{N} \) natürliche Zahlen, und es gelte \( a + c_1 = b \) sowie \( a + c_2 = b \) für \( c_1, c_2 \in \mathbb{N} \). Dann folgt, dass \( c_1 = c_2 \).
\end{theorem}
\begin{proof}
    Seien \(a,b\in\mathbb{N}\), dann gilt:
        \[
	\begin{array}{llll}
            1   &  (1) & \exists c_1,c_2\in \mathbb{N}(a+c_1=b\land a+c_2=b) & \rA \\
            1   &  (2) & \exists c_1\exists c_2\in \mathbb{N}(a+c_1=b\land a+c_2=b) & \rSetEEm{1} \\
            3   &  (3) & c_1\in\mathbb{N}\land \exists c_2\in \mathbb{N}(a+c_1=b\land a+c_2=b) & \rA \\
            3   &  (4) & c_1\in\mathbb{N} & \rAEa{3} \\
            3   &  (5) & \exists c_2\in \mathbb{N}(a+c_1=b\land a+c_2=b) & \rAEb{3} \\
            6   &  (6) & c_2\in\mathbb{N}\land (a+c_1=b\land a+c_2=b) & \rA \\
            6   &  (7) & c_2\in\mathbb{N} & \rAEa{6} \\
            6   &  (8) & a+c_1=b\land a+c_2=b & \rAEb{6} \\
            6   &  (9) & a+c_1=b & \rAEa{8} \\
            6   &  (10) & a+c_2=b & \rAEb{8} \\
            6   &  (11) & a+c_1=a+c_2 & \rIE{10,9} \\
            6   &  (12) & c_1=c_2 & \aEqualsbEqvcPlusaEqualscPlusb{11} \\
            3   &  (13) & c_1=c_2 & \rEE{5,6,12} \\
            1   &  (14) & c_1=c_2 & \rEE{2,3,13} \\
    \end{array}
	\]
\end{proof}

\section{Definition der Differenz}

\begin{definition}[Differenz (\( b - a \))]
    Sei \( a, b \in \mathbb{N} \). Eine partielle Definition erlaubt es, die Differenz \( b - a \) wie folgt einzuführen:
    \[
    \forall a, b \in \mathbb{N} \, [ a \leq b \rightarrow ( b - a \coloneqq \iota c \, (c \in \mathbb{N} \land a + c = b) ) ].
    \]
\end{definition}

\subsubsection*{Einführungsregel für die Differenz}
\label{rule:minusI}

\[
\begin{array}{llll}
    i & (1) & a\in\mathbb{N} & ... \\
    j & (2) & b\in\mathbb{N} & ... \\
    k & (3) & a\leq b & ... \\
    i,j,k & (4) & b-a\in\mathbb{N} & \minusI{1,2,3} \\
    i,j,k & (5) & b-a=b-a & \minusI{1,2,3} \\
    i,j,k & (6) & a+(b-a)=b & \minusI{1,2,3} \\
\end{array}
\]

\[
\begin{array}{llll}
    i & (1) & a\in\mathbb{N} & ... \\
    j & (2) & b\in\mathbb{N} & ... \\
    k & (3) & c\in\mathbb{N} & ... \\
    l & (4) & a+c=b & ... \\
    i,j,k,l & (5) & (b-a)=c & \minusI{1,2,3} \\
    i,j,k,l & (6) & c=(b-a) & \minusI{1,2,3} \\
\end{array}
\]
\(i, j\) und \(k\) sind dabei Listen von Annahmen.

\subsubsection*{Eliminierungsregel für die Differenz}
\label{rule:minusE}

\[
\begin{array}{llll}
    i & (1) & a\in\mathbb{N} & ... \\
    j & (2) & b\in\mathbb{N} & ... \\
    k & (3) & a-b\in\mathbb{N} & ... \\
    i,j,k & (4) & a\leq b & \minusE{1,2,3} \\
    i,j,k & (4) & a+(b-a)=b & \minusE{1,2,3} \\
\end{array}
\]

\(i, j\) und \(k\) sind dabei Listen von Annahmen.


\begin{definition}[Subtraktion als Funktion]
    Wir definieren die Subtraktionsfunktion \( - : \{ (a, b) \in \mathbb{N} \times \mathbb{N} \mid b \leq a \} \to \mathbb{N} \) mit
    durch
    \[
    -(a, b) := a-b
    \]
\end{definition}

\section{Eigenschaften von Ordnungsrelation und Differenz}

\label{awbInNaturalwaEqualsbImpaLeqb}
\begin{theorem}[\(a,b\in\mathbb{N},a=b\vdash a\leq b\)]
\end{theorem}
\begin{proof}
    Seien \(a,b\in\mathbb{N}\), \(\ImpLpNaturalwPluswZeroRpInAbelMonoid{}\) und daraus folgt:
        \[
	\begin{array}{lllcll}
                  &  (1) & a+0&=&a & \rNeutralElementMonoid{} \\
                2 &  (2) & &=&b & \rA \\
                2 &  (3) & a+0&=&b & \rTransitivityEqRI{1,2} \\
                2 &  (4) & \multicolumn{3}{l}{a\leq b} & \rLeqNI{3} \\
    \end{array}
	\]
\end{proof}

\label{awbInNaturalwaEqualsbImpbLeqa}
\begin{theorem}[\(a,b\in\mathbb{N},a=b\vdash b\leq a\)]
\end{theorem}
\begin{proof}
    Seien \(a,b\in\mathbb{N}\), \(\ImpLpNaturalwPluswZeroRpInAbelMonoid{}\) und daraus folgt:
        \[
	\begin{array}{lllcll}
                1 &  (1) & \multicolumn{3}{l}{a=b} & \rA \\
                1 &  (2) & \multicolumn{3}{l}{b=a} & \rSymmetryEqRI{1} \\
                1 &  (3) & \multicolumn{3}{l}{b\leq a}& \awbInNaturalwaEqualsbImpaLeqb{2} \\
    \end{array}
	\]
\end{proof}


\label{awbInNaturalaLeqbImpbMinusaLeqb}
\begin{theorem}[\(a,b\in\mathbb{N},a\leq b\vdash b-a\leq b\)]
\end{theorem}
\begin{proof}
    Seien \(a,b\in\mathbb{N}\), \(\ImpLpNaturalwPluswZeroRpInAbelMonoid{}\) und daraus folgt:
        \[
	\begin{array}{lllcll}
                1 &  (1) & \multicolumn{3}{l}{a\leq b} & \rA \\
                  &  (2) & (b-a)+a&=&a+(b-a) & \rCommutativeMonoid{} \\
                1 &  (3) & &=&b & \minusE{1} \\
                1 &  (4) & (b-a)+a&=&b & \rTransitivityEqRI{2,3} \\
                1 &  (5) & \multicolumn{3}{l}{(b-a)\leq b} & \rLeqNI{4} \\
    \end{array}
	\]
\end{proof}


\subsection{Differenz zu Nachfolger und Vorgänger}

\label{aInNaturalImpLpaPlusOneRpMinusaEqualsOne}
\begin{theorem}[\(a\in\mathbb{N}\vdash (a+1)-a=1\)]
\end{theorem}
\begin{proof}
    Seien \(a\in\mathbb{N}\), dann gilt:
        \[
	\begin{array}{llll}
                &  (1) & a=a & \rII{} \\
                &  (2) & a+1=a+1 & \aEqualsbEqvaPluscEqualsbPlusc{1} \\
                &  (3) & (a+1)-a=1 & \minusI{2} \\
    \end{array}
	\]
\end{proof}

\label{aInNaturalwaNotEqualsZeroImpaMinusLpaMinusOneRpEqualsOne}
\begin{theorem}[\(a\in\mathbb{N},a\neq 0\vdash a-(a-1)=1\)]
Seien \(a\in\mathbb{N}\), dann gilt:
\[a\neq 0\vdash a-(a-1)=1\]
\end{theorem}
\begin{proof}
    Sei \(a\in\mathbb{N}\), dann gilt:
        \[
	\begin{array}{llll}
            1   &  (1) & a\neq 0 & \rA \\
            1   &  (2) & a=(a-1)+1 & \rPredecessorI{1} \\
            1   &  (3) & a-(a-1)=1 & \minusI{3} \\
    \end{array}
	\]
\end{proof}

\subsection{Ordnungsrelation mit Nachfolger und Vorgänger}

\label{aInNaturalImpaLeqaPlusOne}
\begin{theorem}[\(a\in\mathbb{N}\vdash a\leq a+1\)]
\end{theorem}
\begin{proof}
    Seien \(a\in\mathbb{N}\), dann gilt:
        \[
	\begin{array}{llll}
                &  (1) & (a+1)-a=1 & \aInNaturalImpLpaPlusOneRpMinusaEqualsOne{} \\
                &  (2) & a\leq a+1 & \minusE{1} \\
    \end{array}
	\]
\end{proof}

\label{aInNaturalwbInNaturalwaEqualsbImpaLeqbPlusOne}
\begin{theorem}[\(a\in\mathbb{N},b\in\mathbb{N},a=b\vdash a\leq b+1\)]
Seien \(a,b\in\mathbb{N}\), dann gilt:
\[a=b\vdash a\leq b+1\]
\end{theorem}
\begin{proof}
    Seien \(a,b\in\mathbb{N}\), dann gilt:
        \[
	\begin{array}{llll}
            1   &  (1) & a=b & \rA \\
                &  (2) & a\leq a+1 & \aInNaturalImpaLeqaPlusOne{} \\
            1   &  (3) &  a\leq b+1 & \rIE{1,2} \\
    \end{array}
	\]
\end{proof}

\label{aInNaturalwbInNaturalwaEqualsbImpbLeqaPlusOne}
\begin{theorem}[\(a\in\mathbb{N},b\in\mathbb{N},a=b\vdash b\leq a+1\)]
Seien \(a,b\in\mathbb{N}\), dann gilt:
\[a=b\vdash b\leq a+1\]
\end{theorem}
\begin{proof}
    Seien \(a,b\in\mathbb{N}\), dann gilt:
        \[
	\begin{array}{llll}
            1   &  (1) & a=b & \rA \\
            1   &  (2) & b=a & \aIdbImpbIda{1} \\
            1   &  (3) & b\leq a+1 & \aInNaturalwbInNaturalwaEqualsbImpaLeqbPlusOne{2} \\
    \end{array}
	\]
\end{proof}

\label{aInNaturalwaNotEqualsZeroImpLpaMinusOneRpLeqa}
\begin{theorem}[\(a\in\mathbb{N},a\neq 0\vdash (a-1)\leq a\)]
Seien \(a\in\mathbb{N}\), dann gilt:
\[(a-1)\leq a\]
\end{theorem}
\begin{proof}
    Sei \(a\in\mathbb{N}\), dann gilt:
        \[
	\begin{array}{llll}
            1   &  (1) & a\neq 0 & \rA \\
            1   &  (2) & a-(a-1)=1 & \rPredecessorI{1} \\
            1   &  (3) & (a-1)\leq a & \minusI{3} \\
    \end{array}
	\]
\end{proof}

\label{aInNaturalwbInNaturalwaNotEqualsZerowaEqualsbImpaMinusOneLeqb}
\begin{theorem}[\(a\in\mathbb{N},b\in\mathbb{N},a\neq 0, a=b\vdash a-1\leq b\)]
Seien \(a,b\in\mathbb{N}\), dann gilt:
\[a\neq 0, a=b\vdash a-1\leq b\]
\end{theorem}
\begin{proof}
    Seien \(a,b\in\mathbb{N}\), dann gilt:
        \[
	\begin{array}{llll}
            1     &  (1) & a\neq 0 & \rA \\
            2     &  (2) & a=b & \rA \\
            1   &  (3) & a-1\leq a\in\mathbb{N} & \aInNaturalwaNotEqualsZeroImpLpaMinusOneRpLeqa{1} \\
            1,2   &  (4) & a-1\leq b & \rIE{2,3} \\
    \end{array}
	\]
\end{proof}

\label{aInNaturalwbInNaturalwaNotEqualsZerowaEqualsbImpbMinusOneLeqa}
\begin{theorem}[\(a\in\mathbb{N},b\in\mathbb{N},a\neq 0, a=b\vdash b-1\leq a\)]
Seien \(a,b\in\mathbb{N}\), dann gilt:
\[a\neq 0, a=b\vdash b-1\leq a\]
\end{theorem}
\begin{proof}
    Seien \(a,b\in\mathbb{N}\), dann gilt:
        \[
	\begin{array}{llll}
            1     &  (1) & a\neq 0 & \rA \\
            2     &  (2) & a=b & \rA \\
            2     &  (3) & b=a & \aIdbImpbIda{2} \\
            2     &  (4) & b\neq 0 & \rIE{2,1} \\
            1,2   &  (5) & b-1\leq a & \aInNaturalwbInNaturalwaNotEqualsZerowaEqualsbImpaMinusOneLeqb{5,4} \\
    \end{array}
	\]
\end{proof}

\subsection{Eigenschaften der Differenz in Bezug auf Null}

\label{aInNaturalImpaMinusZeroEqualsa}
\begin{theorem}[\(a\in\mathbb{N}\vdash a-0=a\)]
\end{theorem}
\begin{proof}
    Sei \(a\in\mathbb{N}\). \(\ImpLpNaturalwPluswZeroRpInAbelMonoid{}\) und daraus folgt:
        \[
	\begin{array}{llll}
            &  (1) & 0\leq a & \rLeqNI{1} \\
            &  (2) & 0+(a-0)=a & \rLeqNI{1} \\
            &  (3) & 0+(a-0)=a-0 & \rNeutralElementMonoid{} \\
            &  (4) & a-0=a & \rIE{3,2} \\
    \end{array}
	\]
\end{proof}



\label{aInNaturalImpaMinusaEqualsZero}
\begin{theorem}[\(a\in\mathbb{N}\vdash a-a=0\)]
\end{theorem}
\begin{proof}
        Sei \(a\in\mathbb{N}\). Wir wissen: \(\LeqIsTotalOrderOnNaturalNumbers{}\), woraus folgt:
        \[
	\begin{array}{llll}
                &  (1) & a\leq a & \rReflexivityOrdRI{} \\
                &  (2) & a+(a-a)=a & \minusI{1} \\
                &  (3) & a-a=0 & \aInNaturalwbInNaturalwaPlusbEqualsaImpbEqualsZero{2} \\
    \end{array}
	\]
\end{proof}

\label{awbInNaturalLpaEqualsbEqvaMinusbEqualsZeroRp}
\begin{theorem}[\(a=b\dashv\vdash a-b=0\)]
Seien \(a,b\in\mathbb{N}\), dann gilt:
\[a=b\dashv\vdash a-b=0\]
\end{theorem}
\begin{proof}
        Seien \(a,b\in\mathbb{N}\), dann gilt:
\(\vdash:\)
	\[
	\begin{array}{llll}
		1 & (1) & a=b & \rA \\
		   & (2) & a-a=0 & \aInNaturalImpaMinusaEqualsZero{1} \\
            1 & (3) & a-b=0 & \rIE{1,2} \\
	\end{array}
	\]
	\(\dashv:\)
    \(\ImpLpNaturalwPluswZeroRpInMonoid{}\) und daraus folgt:
	\[
	\begin{array}{llll}
		1 & (1) & a-b = 0 & \rA \\
            1 & (2) & a+(a-b)=b & \minusE{1} \\
            1 & (3) & a+0=b & \rIE{1,2} \\
              & (4) & a+0=a & \rNeutralElementMonoid{} \\
            1 & (5) & a=b & \rIE{4,3} \\
	\end{array}
	\]
\end{proof}

\label{awbInNaturalLpbEqualsaEqvaMinusbEqualsZeroRp}
\begin{theorem}[\(b=a\dashv\vdash a-b=0\)]
Seien \(a,b\in\mathbb{N}\), dann gilt:
\[a=b\dashv\vdash a-b=0\]
\end{theorem}
\begin{proof}
        Seien \(a,b\in\mathbb{N}\), dann gilt:
\(\vdash:\)
	\[
	\begin{array}{llll}
		1 & (1) & b=a & \rA \\
		  1 & (2) & a=b & \rSymmetryEqRI{1} \\
            1 & (3) & a-b=0 & \awbInNaturalLpaEqualsbEqvaMinusbEqualsZeroRp{1,2} \\
	\end{array}
	\]
	\(\dashv:\)
	\[
	\begin{array}{llll}
		1 & (1) & a-b = 0 & \rA \\
            1 & (2) & a=b & \awbInNaturalLpaEqualsbEqvaMinusbEqualsZeroRp{} \\
            1 & (3) & b=a & \rSymmetryEqRI{2} \\
	\end{array}
	\]
\end{proof}

\label{aNotEqualsbEqvaMinusbNotEqualsZero}
\begin{theorem}[\(a\neq b\dashv\vdash a-b\neq 0\)]
Seien \(a,b\in\mathbb{N}\), dann gilt:
\[a\neq b\dashv\vdash a-b\neq 0\]
\end{theorem}
\begin{proof}
        Seien \(a,b\in\mathbb{N}\), dann gilt:
        \[
	\begin{array}{llll}
		   & (1) & a=b\leftrightarrow a-b=0 & \awbInNaturalLpaEqualsbEqvaMinusbEqualsZeroRp{} \\
		   & (2) & a\neq b\leftrightarrow a-b\neq 0 & \PLrQEqvnPLrnQ{1} \\
	\end{array}
	\]
\end{proof}



\label{bNotEqualsaEqvaMinusbNotEqualsZero}
\begin{theorem}[\(b\neq a\dashv\vdash a-b\neq 0\)]
Seien \(a,b\in\mathbb{N}\), dann gilt:
\[b\neq a\dashv\vdash b-a\neq 0\]
\end{theorem}
\begin{proof}
        Seien \(a,b\in\mathbb{N}\), dann gilt:
        \[
	\begin{array}{llll}
		   & (1) & b=a\leftrightarrow a-b=0 & \awbInNaturalLpaEqualsbEqvaMinusbEqualsZeroRp{} \\
		   & (2) & b\neq a\leftrightarrow a-b\neq 0 & \PLrQEqvnPLrnQ{1} \\
	\end{array}
	\]
\end{proof}

\label{awbInNaturalLpaLeqbEqvZeroLeqbMinusaRp}
\begin{theorem}[\(a\leq b\dashv\vdash 0\leq b-a\)]
Seien \(a,b\in\mathbb{N}\), dann gilt:
\[a\leq b\dashv\vdash 0\leq b-a\]
\end{theorem}
\begin{proof}
        Seien \(a,b\in\mathbb{N}\), dann gilt:
\(\vdash:\)
	\[
	\begin{array}{llll}
		1 & (1) & a\leq b & \rA \\
		  1 & (2) & b-a\in\mathbb{N} & \minusI{1} \\
            1 & (3) & 0\leq b-a & \ImpZeroLeqa{2} \\
	\end{array}
	\]
	\(\dashv:\)
	\[
	\begin{array}{llll}
		1 & (1) & 0\leq b-a & \rA \\
            1 & (2) & ((b-a)-0)\in\mathbb{N} & \minusI{1} \\
            1 & (3) & (b-a)-0=b-a & \aInNaturalImpaMinusZeroEqualsa{2} \\
            1 & (4) & (b-a)\in\mathbb{N} & \rIE{3,2} \\
            1 & (3) & a\leq b & \minusE{4} \\
	\end{array}
	\]
\end{proof}


\label{ImpZeroLeqa}
\begin{theorem}[\(0\leq a\)]
Sei \(a\in\mathbb{N}\), dann gilt:
\[\vdash 0\leq a\]
\end{theorem}
\begin{proof}
    Seien \(a\in\mathbb{N}\). \(\ImpLpNaturalwPluswZeroRpInAbelMonoid{}\) und daraus folgt:
        \[
	\begin{array}{llll}
            &  (1) & 0+a=a & \rNeutralElementMonoid{} \\
            &  (2) & 0\leq a & \rLeqNI{1} \\
    \end{array}
	\]
\end{proof}

\label{aInNaturalwaNotEqualsZerowOneLeqa}
\begin{theorem}[\(a\in\mathbb{N}, a\neq 0, 1\leq a\)]
\end{theorem}
\begin{proof}
    Seien \(a\in\mathbb{N}\). \(\ImpLpNaturalwPluswZeroRpInAbelMonoid{}\) und daraus folgt:
        \[
	\begin{array}{llclll}
         1   &  (1) & \multicolumn{3}{l}{a\neq 0} & \rA \\
         1   &  (2) & a&=&(a-1)+1 & \rPredecessorI{1} \\
             &  (3) & &=&1+(a-1) & \rCommutativeMonoid{} \\
         1   &  (4) & a&=&1+(a-1) & \rTransitivityEqRI{2,3} \\
         1   &  (5) & \multicolumn{3}{l}{1\leq a} & \rLeqNI{4} \\
    \end{array}
	\]
\end{proof}

\label{aInNaturalwaLeqZeroImpaEqualsZero}
\label{aLeqZeroImpaEqualsZero}
\begin{theorem}[\(a\in\mathbb{N},a\leq 0\vdash a=0\)]
Sei \(a\in\mathbb{N}\), dann gilt:
\[a\leq 0\vdash a=0\]
\end{theorem}
\begin{proof}
    Seien \(a\in\mathbb{N}\), dann gilt:
        \[
	\begin{array}{llll}
           1 &  (1) & a\leq 0 & \rA{} \\
             &  (2) & 0\leq a & \ImpZeroLeqa{} \\
           1 &  (3) & a=0 & \rAntisymmetryOrdRI{1,2} \\
    \end{array}
	\]
\end{proof}



\subsection{Eigenschaften der Halbordnung}

\label{aInNaturalwbInNaturalwaLeqbwaNotEqualsbImpaPlusOneLeqb}
\begin{theorem}[\(a\in\mathbb{N},b\in\mathbb{N}, a\leq b,a\neq b\vdash a+1\leq b\)]
Seien \(a,b\in\mathbb{N}\), dann gilt:
\[a\leq b,a\neq b\vdash a+1\leq b\]
\end{theorem}
\begin{proof}
    Seien \(a,b\in\mathbb{N}\). \(\ImpLpNaturalwPluswZeroRpInAbelMonoid{}\) und daraus folgt:
        \[
	\begin{array}{llll}
            1     &  (1) & a\leq b & \rA \\
            2     &  (2) & a\neq b & \rA \\
            1     &  (3) & a+(b-a)=b & \minusI{1} \\
            4     &  (4) & b-a=0 & \rA \\
            1,4   &  (5) & a+0=b & \rIE{4,3} \\
                  &  (6) & a+0=a & \rNeutralElementMonoid{} \\
            1,4   &  (7) & a=b & \rIE{6,5} \\
            1,2,4 &  (8) & \bot & \rBI{2,7} \\
            1,2   &  (9) & b-a\neq 0 & \rCI{4,8} \\
            1,2   &  (10) & b-a=((b-a)-1)+1 & \rPredecessorI{4,8} \\
                  &  (11) & a+(((b-a)-1)+1)=(a+1)+((b-a)-1) & \aInMwbInMwcInMImpaPlusLpbPluscRpEqualsLpaPluscRpPlusb{} \\
            1,2   &  (12) & a+(((b-a)-1)+1)=b & \rIE{2,10} \\
            1,2   &  (13) & (a+1)+((b-a)-1)=b & \rIE{11,12} \\
            1,2   &  (14) & (a+1)\leq b & \rLeqNI{13} \\
    \end{array}
	\]
\end{proof}

\label{aInNaturalImpaLeqa}
\begin{theorem}[\(a\in\mathbb{N}\vdash a\leq a\) (Reflexivität)]
\end{theorem}
\begin{proof}
        Sei \(a\in\mathbb{N}\). \(\ImpLpNaturalwPluswZeroRpInAbelMonoid{}\) und daraus folgt:
        \[
	\begin{array}{llll}
                &  (1) & a+0=a & \rNeutralElementMonoid{} \\
                &  (2) & a\leq a & \rLeqNI{1} \\
    \end{array}
	\]
\end{proof}

\label{aInNaturalwbInNaturalwaLeqbwbLeqaImpaEqualsb}
\begin{theorem}[\(a\in\mathbb{N},b\in\mathbb{N},a\leq b,b\leq a\vdash a = b\) (Antisymmetrie)]
\end{theorem}
\begin{proof}
        Seien \(a,b\in\mathbb{N}\). \(\ImpLpNaturalwPluswZeroRpInAbelMonoid{}\) und daraus folgt:
        \[
	\begin{array}{llll}
            1       &  (1) & a\leq b & \rA \\
            2       &  (2) & b\leq a & \rA \\
            1       &  (3) & a+(b-a)=b & \minusI{1} \\
            2       &  (4) & b+(a-b)=a & \minusI{2} \\
            1,2     &  (5) & (a+(b-a))+(a-b)=a & \rIE{3,4} \\
                    &  (6) & (a+(b-a))+(a-b)=a+((b-a)+(a-b)) & \rAssociativityMonoid{} \\
            1,2     &  (7) & a+((b-a)+(a-b))=a & \rIE{6,5} \\
            1,2     &  (8) & (b-a)+(a-b)=0 & \aInNaturalwbInNaturalwaPlusbEqualsaImpbEqualsZero{7} \\     
            1,2     &  (9) & (b-a)=0\land (a-b)=0 & \aInNaturalwbInNaturalwaPlusbEqualsZeroImpaEqualsZeroAndbEqualsZero{8} \\
            1,2     &  (10) & (b-a)=0 & \rAEa{9} \\
            1,2     &  (11) & a+0=b & \rIE{10,3} \\
            1,2     &  (12) & a+0=b & \rIE{10,3} \\
                    &  (13) & a+0=a & \rNeutralElementMonoid{} \\
            1,2     &  (14) & a=b & \rIE{13,12} \\
    \end{array}
	\]
\end{proof}

\label{aInNaturalwbInNaturalwaLeqbwbLeqcImpaLeqc}
\begin{theorem}[Transitivität der Ordnungsrelation auf \(\mathbb{N}\)]
Seien \(a,b,c\in\mathbb{N}\), dann gilt:
\[a\leq b,b\leq c\vdash a\leq c\]
\end{theorem}
\begin{proof}
    Seien \(a,b,c\in\mathbb{N}\). \(\ImpLpNaturalwPluswZeroRpInAbelMonoid{}\) und daraus folgt:
        \[
	\begin{array}{llll}
            1       &  (1) & a\leq b & \rA \\
            2       &  (2) & b\leq c & \rA \\
            1       &  (3) & a+(b-a)=b & \minusI{1} \\
            2       &  (4) & b+(c-b)=c & \minusI{1} \\
            1,2     &  (5) & (a+(b-a))+(c-b)=c & \rIE{3,4} \\
                    &  (6) & (a+(b-a))+(c-b)=a+((b-a)+(c-b)) & \rAssociativityMonoid{} \\
            1,2     &  (7) & a+((b-a)+(c-b))=c & \rIE{6,5} \\
            1,2     &  (8) & a\leq c & \rLeqNI{7} \\
    \end{array}
	\]
\end{proof}

\label{LeqIsHalfOrderOnNaturalNumbers}
\begin{theorem}[\(\leq\) ist eine Halbordnung auf \(\mathbb{N}\) ]
\end{theorem}
\begin{proof}
        \[
	\begin{array}{llll}
                    & (1) & \forall a \in \mathbb{N}  (a \leq a) & \aInNaturalImpaLeqa{} \\
                    & (2) & \forall a, b \in \mathbb{N}  ((a \leq b \land b \leq a) \rightarrow a = b) & \aInNaturalwbInNaturalwaLeqbwbLeqaImpaEqualsb{} \\
                    & (3) & \forall a, b, c \in \mathbb{N}  ((a \leq b \land b \leq c) \rightarrow a \leq c) & \aInNaturalwbInNaturalwaLeqbwbLeqcImpaLeqc{} \\
                    & (4) & \leq \text{ ist eine Halbordnung auf } \mathbb{N} &  \rPartialOrderRelationI{1,2,3}
    \end{array}
	\]
\end{proof}

\subsection{Invarianz der Ordnungsrelation gegenüber Addition}

\label{awbwcInNaturalLpaLeqbEqvaPluscLeqbPluscRp}
\begin{theorem}[\(a\leq b\dashv\vdash a+c\leq b+c\)]
Seien \(a,b,c\in\mathbb{N}\), dann gilt:
\[a\leq b\dashv\vdash a+c\leq b+c\]
\end{theorem}
\begin{proof}
Wir beweisen die Äquivalenz in drei Schritten:
\begin{enumerate}
    \item Beweis von Lemma 1: \(a+(b-c)=b \vdash (a+c)+(b-c)=b+c\).
    \item Beweis von Lemma 2: \((a+c)+((b+c)-(a+c))=b+c \vdash a+((b+c)-(a+c))=b\).
    \item Beweis der Äquivalenz \(a\leq b \dashv\vdash a+c\leq b+c\) unter Verwendung von Lemma 1 und Lemma 2.
\end{enumerate}

Seien \(a, b, c \in \mathbb{N}\). Im Beweis nutzen wir die Eigenschaft \(\ImpLpNaturalwPluswZeroRpInMonoid{}\).

\begin{longtable}{llclll}
\multicolumn{6}{l}{\textbf{Lemma 1:} \quad \(a + (b - c) = b \vdash (a + c) + (b - c) = b + c\)} \\
1     &  (1) & \(a+(b-c)\) & = & \(b\) & \rA \\
      &  (2) & \((a+c)+(b-c)\) & = & \(a+((b-c)+c)\) & \aInMwbInMwcInMImpaPlusLpbPluscRpEqualsLpaPluscRpPlusb{} \\
      &  (3) &               & = & \((a+(b-c))+c\) & \rAssociativityMonoid{2} \\
1     &  (4) &               & = & \(b+c\) & \rIE{3,1} \\
\\
\multicolumn{6}{l}{\textbf{Lemma 2:} \quad \((a + c) + ((b + c) - (a + c)) = b + c \vdash a + ((b + c) - (a + c)) = b\)} \\
\multicolumn{6}{l}{\text{Im Beweis verwenden wir indirekt die Regel } \aIdbImpbIda{}.} \\
1     &  (1) & \(b+c\) & = & \((a+c)+((b+c)-(a+c))\) & \rA{} \\
1     &  (2) &         & = & \(a+(((b+c)-(a+c))+c)\) & \aInMwbInMwcInMImpaPlusLpbPluscRpEqualsLpaPluscRpPlusb{1} \\
1     &  (3) &         & = & \((a+((b+c)-(a+c)))+c\) & \rAssociativityMonoid{2} \\
1     &  (4) & \(b\)   & = & \(a+((b+c)-(a+c))\) & \aEqualsbEqvaPluscEqualsbPlusc{3} \\
\\
\multicolumn{6}{l}{\textbf{Beweis der Äquivalenz:} \quad \(a \leq b \dashv\vdash a + c \leq b + c\)} \\
\multicolumn{6}{l}{\(\vdash:\)} \\
1     &  (1) & \multicolumn{3}{l}{\(a \leq b\)} & \rA \\
1     &  (2) & \multicolumn{3}{l}{\(a + (b - a) = b\)} & \minusI{1} \\
1     &  (3) & \multicolumn{3}{l}{\((a + c) + (b - c) = b + c\)} & \text{Lemma 1(2)} \\
1     &  (4) & \multicolumn{3}{l}{\(a + c \leq b + c\)} & \rLeqNI{3} \\
\multicolumn{6}{l}{\(\dashv:\)} \\
1     &  (1) & \multicolumn{3}{l}{\(a + c \leq b + c\)} & \rA \\
1     &  (2) & \multicolumn{3}{l}{\(a + c + ((b + c) - (a + c)) = b + c\)} & \minusI{1} \\
1     &  (3) & \multicolumn{3}{l}{\(a + ((b + c) - (a + c)) = b\)} & \text{Lemma 2(2)} \\
1     &  (4) & \multicolumn{3}{l}{\(a \leq b\)} & \rLeqNI{3} \\
\end{longtable}

\end{proof}

\label{awbwcInNaturalLpaLeqbEqvcPlusaLeqcPlusbRp}
\begin{theorem}[\(a\leq b\dashv\vdash c+a\leq c+b\)]
Seien \(a,b,c\in\mathbb{N}\), dann gilt:
\[a\leq b\dashv\vdash c+a\leq c+b\]
\end{theorem}
\begin{proof}
Seien \(a, b, c \in \mathbb{N}\). Im Beweis nutzen wir die Eigenschaft \(\ImpLpNaturalwPluswZeroRpInMonoid{}\).
\(\vdash\):
        \[
	\begin{array}{llclll}
            1     &  (1) & \multicolumn{3}{l}{a\leq b} & \rA \\
                  &  (2) & c+a&=&a+c & \rCommutativeMonoid{} \\
            1     &  (3) &    &\leq& c+b & \awbwcInNaturalLpaLeqbEqvaPluscLeqbPluscRp{1} \\
            1     &  (4) &    &\leq& b+c & \rCommutativeMonoid{} \\
    \end{array}
	\]
\(\dashv\):
        \[
	\begin{array}{llclll}
                  &  (1) & a+c&=&c+a & \rCommutativeMonoid{} \\
            2     &  (2) &    &\leq& c+b & \rA \\
            2     &  (3) &    &\leq& c+b & \rCommutativeMonoid{} \\
            2     &  (4) &   \multicolumn{3}{l}{a\leq b} & \awbwcInNaturalLpaLeqbEqvaPluscLeqbPluscRp{3} \\
        \end{array}
	\]
\end{proof}

\label{awbwcwdInNaturalwaLeqbwcLeqdImpaPluscLeqbPlusd}
\begin{theorem}[\(a,b,c,d\in\mathbb{N},a\leq b, c\leq d\vdash a+c\leq b+d\)]
\end{theorem}
\begin{proof}
Seien \(a, b, c, d \in \mathbb{N}\). Im Beweis nutzen wir die Eigenschaft \(\ImpLpNaturalwPluswZeroRpInMonoid{}\).
        \[
	\begin{array}{llclll}
            1     &  (1) & \multicolumn{3}{l}{a\leq b} & \rA \\
            2     &  (2) & \multicolumn{3}{l}{c\leq d} & \rA \\
            1     &  (3) & a+c&\leq &b+c & \awbwcInNaturalLpaLeqbEqvaPluscLeqbPluscRp{1} \\
            2     &  (4) &    &\leq &b+d & \awbwcInNaturalLpaLeqbEqvcPlusaLeqcPlusbRp{2} \\
            1,2   &  (5) & a+c&\leq &b+d & \rTransitivityOrdRI{3,4} \\
    \end{array}
	\]
\end{proof}



\label{aInNaturalwbInNaturalwcInNaturalwaLeqbImpaLeqbPlusc}
\begin{theorem}[\(a\in\mathbb{N},b\in\mathbb{N},c\in\mathbb{N}, a\leq b\vdash a\leq b+c\)]
Seien \(a,b,c\in\mathbb{N}\), dann gilt:
\[a\leq b\vdash a\leq b+c\]
\end{theorem}
\begin{proof}
Im Beweis nehmen wir an, dass \(a,b,c\in\mathbb{N}\). Der Beweis besteht aus zwei Teilen:
\begin{enumerate}
    \item Wir beweisen ein Lemma, das im weiteren Verlauf verwendet wird.
    \item Wir zeigen, dass aus \(a \leq b\) folgt, dass \(a \leq b+c\).
\end{enumerate}

\paragraph{Lemma: \((a+c)+((b+c)-(a+c))=b+c \vdash a+(c+((b+c)-(a+c)))=b+c\)}

Im Beweis verwenden wir indirekt die Regel \(\aIdbImpbIda{}\).  \(\ImpLpNaturalwPluswZeroRpInMonoid{}\) und daher gilt:\\
        \[
	\begin{array}{llclll}
            1     &  (1) & b+c&=&(a+c)+((b+c)-(a+c)) & \rA \\
            1     &  (2) & &=&a+(c+((b+c)-(a+c)))=b+c & \rAssociativityMonoid{} \\
    \end{array}
	\]
    \\
\paragraph{Beweis des Theorems: \(a \leq b \vdash a \leq b+c\)}
        \[
	\begin{array}{llll}
            1     &  (1) & a\leq b & \rA \\
            1     &  (2) & a+c\leq b+c & \aInNaturalwbInNaturalwaNotEqualsZerowaEqualsbImpbMinusOneLeqa{1} \\
            1     &  (3) & (a+c)+((b+c)-(a+c))=b+c & \minusI{2} \\
            1     &  (4) & a+(c+((b+c)-(a+c)))=b+c & \text{Lemma(3)} \\
            1     &  (5) & a\leq b+c & \rLeqNI{4} \\
    \end{array}
	\]
\end{proof}

\label{aInNaturalwbInNaturalwcInNaturalwaLeqbImpaLeqcPlusb}
\begin{theorem}[\(a\in\mathbb{N},b\in\mathbb{N},c\in\mathbb{N}, a\leq b\vdash a\leq c+b\)]
\end{theorem}
\begin{proof}
Seien \(a, b, c \in \mathbb{N}\). Im Beweis nutzen wir die Eigenschaft \(\ImpLpNaturalwPluswZeroRpInMonoid{}\).
        \[
	\begin{array}{llclll}
            1     &  (1) & \multicolumn{3}{l}{a\leq b} & \rA \\
            1     &  (2) & a&\leq& b+c & \aInNaturalwbInNaturalwcInNaturalwaLeqbImpaLeqbPlusc{1} \\
            1     &  (3) &  &\leq& c+b & \rCommutativeMonoid{} \\
        \end{array}
	\]
\end{proof}


\label{awbInNaturalImpaLeqaPlusb}
\begin{theorem}[\(a,b\in\mathbb{N}\vdash a\leq a+b\)]
\end{theorem}
\begin{proof}
Seien \(a, b \in \mathbb{N}\). Im Beweis nutzen wir die Eigenschaft \(\LeqIsHalfOrderOnNaturalNumbers{}\).
        \[
	\begin{array}{llll}
                  &  (1) & a\leq a & \rReflexivityOrdRI{} \\
                  &  (2) & a\leq a+b & \aInNaturalwbInNaturalwcInNaturalwaLeqbImpaLeqbPlusc{1} \\
        \end{array}
	\]
\end{proof}

\label{awbInNaturalImpaLeqbPlusa}
\begin{theorem}[\(a,b\in\mathbb{N}\vdash a\leq b+a\)]
\end{theorem}
\begin{proof}
Seien \(a, b \in \mathbb{N}\). Im Beweis nutzen wir die Eigenschaft \(\LeqIsHalfOrderOnNaturalNumbers{}\).
        \[
	\begin{array}{llll}
                  &  (1) & a\leq a & \rReflexivityOrdRI{} \\
                  &  (2) & a\leq b+a & \aInNaturalwbInNaturalwcInNaturalwaLeqbImpaLeqcPlusb{1} \\
        \end{array}
	\]
\end{proof}

\label{awbwcInNaturalLpaPlusbLeqcImpaLeqcRp}
\begin{theorem}[\(a,b,c\in\mathbb{N}(a+b\leq c\vdash a\leq c)\)]
\end{theorem}
\begin{proof}
Seien \(a, b, c \in \mathbb{N}\). Im Beweis nutzen wir die Eigenschaft \(\LeqIsHalfOrderOnNaturalNumbers{}\).
        \[
	\begin{array}{llclll}
                  &  (1) & \multicolumn{3}{l}{a\leq a+b} & \awbInNaturalImpaLeqaPlusb{} \\
            2     &  (2) &  &\leq& c & \rA \\
        \end{array}
	\]
\end{proof}



\label{awbwcInNaturalLpaPlusbLeqcImpbLeqcRp}
\begin{theorem}[\(a,b,c\in\mathbb{N}(a+b\leq c\vdash b\leq c)\)]
\end{theorem}
\begin{proof}
Seien \(a, b, c \in \mathbb{N}\). Im Beweis nutzen wir die Eigenschaft \(\LeqIsHalfOrderOnNaturalNumbers{}\).
        \[
	\begin{array}{llclll}
                  &  (1) & \multicolumn{3}{l}{b\leq a+b} & \awbInNaturalImpaLeqbPlusa{} \\
            2     &  (2) &  &\leq& c & \rA \\
        \end{array}
	\]
\end{proof}

\label{bLeqcwaLeqbImpbMinuscLeqc}
\begin{theorem}[\(b\leq c, a\leq b\vdash b-a\leq c\)]
\end{theorem}
\begin{proof}
Seien \(a, b, c \in \mathbb{N}\). 
\[
\begin{array}{llclll}
          1  & (1) & \multicolumn{3}{l}{a\leq b}  & \rA \\
          1  & (2) & \multicolumn{3}{l}{b-a\in\mathbb{N}}  & \minusI{1} \\
          1  & (3) & b-a&\leq& a+(b-a)  & \awbInNaturalImpaLeqaPlusb{2} \\
          1  & (4) & &=& b  & \minusI{1} \\
          5  & (5) & &\leq & c  & \rA \\
          1,5& (6) & b-a&\leq & c  & \rTransitivityOrdRI{3,5} \\
\end{array}
\]
\end{proof}

\subsection{Totalität}

\label{ImpaLeqbOrbLeqa}
\begin{theorem}[Totalität der Ordnungsrelation auf \(\mathbb{N}\)]
Seien \(a,b\in\mathbb{N}\), dann gilt:
\[\vdash a\leq b\lor b\leq a\]
\end{theorem}
\begin{proof}
    Seien \(a,b\in\mathbb{N}\), dann gilt:
        \[
	\begin{array}{llll}
                    &  (1) & 0\leq b & \ImpZeroLeqa{} \\
                    &  (2) & 0\leq b\lor b\leq 0 & \rOIa{1} \\
        3           &  (3) & n\in\mathbb{N} & \rA \\           
        4           &  (4) & n\leq b\lor b\leq n & \rA \\    
        5           &  (5) & n\leq b & \rA \\  
                    &  (6) & n=b\lor n\neq b & \ImpPOrnP{} \\  
        7           &  (7) & n=b & \rA \\  
        7           &  (8) & b\leq n+1 & \aInNaturalwbInNaturalwaEqualsbImpbLeqaPlusOne{7} \\  
        7           &  (9) & n+1\leq b\lor b\leq n+1 & \rOIb{8} \\ 
        10          &  (10) & n\neq b & \rA \\ 
        5,10        &  (11) & n+1\leq b & \aInNaturalwbInNaturalwaLeqbwaNotEqualsbImpaPlusOneLeqb{5,10} \\ 
        5,10        &  (12) & n+1\leq b\lor b\leq n+1 & \rOIa{11} \\ 
        5           &  (13) & n+1\leq b\lor b\leq n+1 & \rOE{6,7,9,10,12} \\ 
        14          &  (14) & b\leq n & \rA \\  
        14          &  (15) & b\leq n+1 & \aInNaturalwbInNaturalwcInNaturalwaLeqbImpaLeqbPlusc{14} \\  
        14          &  (16) & n+1\leq b\lor b\leq n+1 & \rOIb{15} \\ 
         4          &  (17) & n+1\leq b\lor b\leq n+1 & \rOE{4,5,13,14,16} \\ 
                    &  (18) & \forall n\in\mathbb{N}(n\leq b\lor b\leq n) & \rInductionN{2,3,4,17} \\ 
                    &  (19) & a\leq b\lor b\leq a & \rSetUEa{32} \\ 
    \end{array}
	\]
\end{proof}




\label{LeqIsTotalOrderOnNaturalNumbers}
\begin{theorem}[\(\leq\) ist eine totale Ordnung auf \(\mathbb{N}\) ]
\end{theorem}
\begin{proof}
\[
\begin{array}{llll}
            & (1) & \leq \text{ ist eine Halbordnung auf } \mathbb{N}  & \LeqIsHalfOrderOnNaturalNumbers{} \\
            & (2) & \forall a, b \in \mathbb{N}  (a \leq b \lor b \leq a) & \ImpaLeqbOrbLeqa{} \\
          & (3) & \leq \text{ ist eine totale Ordnung auf } \mathbb{N} & \rTotalOrderI{1,2}
\end{array}
\]
\end{proof}

\subsection{Zusammenhänge zwischen Differenzen und Summen}

\label{awbwcInNaturalwcLeqaImpLpaPlusbRpMinuscEqualsbPlusLpaMinuscRp}
\begin{theorem}[\(a,b,c\in\mathbb{N},c\leq a\vdash (a+b)-c=b+(a-c)\)]
\end{theorem}
\begin{proof}
Seien \(a, b, c \in \mathbb{N}\). Im Beweis nutzen wir die Eigenschaft \(\ImpLpNaturalwPluswZeroRpInMonoid{}\).
\[
\begin{array}{llclll}
          1  & (1) & \multicolumn{3}{l}{c\leq a}  & \rA \\
          1  & (2) & \multicolumn{3}{l}{c\leq a+b}  & \aInNaturalwbInNaturalwcInNaturalwaLeqbImpaLeqbPlusc{1} \\
          1  & (3) & \multicolumn{3}{l}{c+((a+b)-c)=a+b} & \minusI{2} \\
          1  & (4) & \multicolumn{3}{l}{c+(a-c)=a} & \minusI{1} \\
          1  & (5) & \multicolumn{3}{l}{c+((a+b)-c)=c+(a-c)+b} & \rIE{4,3} \\
          1  & (6) & (a+b)-c&=&(a-c)+b & \aEqualsbEqvcPlusaEqualscPlusb{5} \\
          1  & (7) & &=&b+(a-c) & \rCommutativeMonoid{6} \\
\end{array}
\]
\end{proof}

\label{awbwcInNaturalwcLeqaImpLpaPlusbRpMinuscEqualsLpaMinuscRpPlusb}
\begin{theorem}[\(a,b,c\in\mathbb{N},c\leq a\vdash (a+b)-c=(a-c)+b\)]
\end{theorem}
\begin{proof}
Seien \(a, b, c \in \mathbb{N}\). Im Beweis nutzen wir die Eigenschaft \(\ImpLpNaturalwPluswZeroRpInAbelMonoid{}\).
\[
\begin{array}{llclll}
          1  & (1) & \multicolumn{3}{l}{c\leq a}  & \rA \\
          1  & (2) & (a+b)-c&=&b+(a-c) & \awbwcInNaturalwcLeqaImpLpaPlusbRpMinuscEqualsbPlusLpaMinuscRp{1} \\
          1  & (3) & &=&(a-c)+b & \rCommutativeMonoid{} \\
\end{array}
\]
\end{proof}

\label{awbwcInNaturalwcLeqbImpLpaPlusbRpMinuscEqualsaPlusLpbMinuscRp}
\begin{theorem}[\(a,b,c\in\mathbb{N}, c\leq b\vdash (a+b)-c=a+(b-c)\)]
\end{theorem}
\begin{proof}
Seien \(a, b, c \in \mathbb{N}\). Im Beweis nutzen wir die Eigenschaft \(\ImpLpNaturalwPluswZeroRpInMonoid{}\).
\[
\begin{array}{llcll p{5cm}}
          1  & (1) & \multicolumn{3}{l}{c\leq b}  & \rA \\
             & (2) & (a+b)-c&=&(b+a)-c & \rCommutativeMonoid{} \\
          1  & (3) & &=&a+(b-c) & \awbwcInNaturalwcLeqaImpLpaPlusbRpMinuscEqualsLpaMinuscRpPlusb{1} \\
\end{array}
\]
\end{proof}

\label{awbwcInNaturalwcLeqbImpLpaPlusbRpMinuscEqualsLpbMinuscRpPlusa}
\begin{theorem}[\(a,b,c\in\mathbb{N}, c\leq b\vdash (a+b)-c=(b-c)+a\)]
\end{theorem}
\begin{proof}
Seien \(a, b, c \in \mathbb{N}\). Im Beweis nutzen wir die Eigenschaft \(\ImpLpNaturalwPluswZeroRpInAbelMonoid{}\).
\[
\begin{array}{llclll}
          1  & (1) & \multicolumn{3}{l}{c\leq b}  & \rA \\
          1  & (2) & (a+b)-c&=&a+(b-c) & \awbwcInNaturalwcLeqbImpLpaPlusbRpMinuscEqualsaPlusLpbMinuscRp{5} \\
          1  & (3) & &=&(a-c)+a & \rCommutativeMonoid{} \\
\end{array}
\]
\end{proof}

\label{awbInNaturalImpLpaPlusbRpMinusbEqualsaPlusLpbMinusbRpEqualsa}
\begin{theorem}[\(a,b\in\mathbb{N}\vdash (a+b)-b=a+(b-b)=a\)]
\end{theorem}
\begin{proof}
Seien \(a, b \in \mathbb{N}\). Im Beweis nutzen wir die Eigenschaft \(\ImpLpNaturalwPluswZeroRpInMonoid{}\).
\[
\begin{array}{llclll}
             & (1) & \multicolumn{3}{l}{b\leq b}  & \rReflexivityOrdRI{} \\
             & (2) & \multicolumn{3}{l}{b\leq a+b}  & \aInNaturalwbInNaturalwcInNaturalwaLeqbImpaLeqcPlusb{1} \\
             & (3) & (a+b)-b&=&a+(b-b)  & \aInNaturalwbInNaturalwcInNaturalwaLeqbImpaLeqcPlusb{1,2} \\
             & (4) &  &=&a+0  & \aInNaturalImpaMinusaEqualsZero{3} \\
             & (5) &  &=&a  & \rNeutralElementMonoid{4} \\
\end{array}
\]
\end{proof}

\label{awbInNaturalImpLpaPlusbRpMinusaEqualsLpaMinusaRpPlusbEqualsb}
\begin{theorem}[\(a,b\in\mathbb{N}\vdash (a+b)-a=(a-a)+b=b\)]
\end{theorem}
\begin{proof}
Seien \(a, b \in \mathbb{N}\). Im Beweis nutzen wir die Eigenschaft \(\ImpLpNaturalwPluswZeroRpInMonoid{}\).
\[
\begin{array}{llclll}
             & (1) & (a+b)-a&=&(b+a)-a  & \rCommutativeMonoid{} \\
             & (2) & &=& b+(a-a)  & \awbInNaturalImpLpaPlusbRpMinusbEqualsaPlusLpbMinusbRpEqualsa{} \\
             & (3) & &=& (a-a)+b  & \rCommutativeMonoid{} \\
             & (4) & &=& b+(a-a)  & \rCommutativeMonoid{} \\
             & (5) & &=& b  & \awbInNaturalImpLpaPlusbRpMinusbEqualsaPlusLpbMinusbRpEqualsa{} \\
             & (6) & \multicolumn{3}{l}{(a+b)-a=(a-a)+b=b}  & \rTransitivityEqRI{1,3,5} \\
\end{array}
\]
\end{proof}

\label{awbInNaturalwbLeqaImpLpaMinusbRpPlusbEqualsa}
\begin{theorem}[\(a,b\in\mathbb{N},b\leq a \vdash (a-b)+b=a\)]
\end{theorem}
\begin{proof}
    Seien \(a,b\in\mathbb{N}\). Daraus folgt:
        \[
	\begin{array}{llclll}
             1 &  (1) & \multicolumn{3}{l}{b\leq a} & \rA \\
             1 &  (2) & (a-b)+b&=&(a+b)-b & \awbwcInNaturalwcLeqaImpLpaPlusbRpMinuscEqualsLpaMinuscRpPlusb{1} \\
             1 &  (3) & &=&a & \awbInNaturalImpLpaPlusbRpMinusbEqualsaPlusLpbMinusbRpEqualsa{2} \\
             1 &  (4) & (a-b)+b&=&a & \rTransitivityEqRI{2,3} \\
    \end{array}
	\]
\end{proof}

\label{awbInNaturalwaMinusbEqualsaImpbEqualsZero}
\begin{theorem}[\(a,b\in\mathbb{N}, a-b=a\vdash b=0\)]
\end{theorem}
\begin{proof}
    Seien \(a,b\in\mathbb{N}\). \(\ImpLpNaturalwPluswZeroRpInAbelMonoid{}\) und daraus folgt:
        \[
	\begin{array}{llclll}
             1 &  (1) & \multicolumn{3}{l}{a-b=a} & \rA \\
             1 &  (2) & \multicolumn{3}{l}{(a-b)+b=a+b} & \aEqualsbEqvaPluscEqualsbPlusc{1} \\
             1 &  (3) & \multicolumn{3}{l}{(a-b)+b=a} & \awbInNaturalwbLeqaImpLpaMinusbRpPlusbEqualsa{1} \\
             1 &  (4) & \multicolumn{3}{l}{a=a+b} & \rIE{3,2} \\
             1 &  (5) & \multicolumn{3}{l}{b=0} & \aInNaturalwbInNaturalwaPlusbEqualsaImpbEqualsZero{4} \\
    \end{array}
	\]
\end{proof}

\label{awbwcInNaturalwcLeqaLpaEqualsbEqvaMinuscEqualsbMinuscRp}
\begin{theorem}[\(a=b\dashv\vdash a-c=b-c\)]
Seien \(a,b,c\in\mathbb{N}\) und \(c\leq a\), dann gilt:
\[a=b\dashv\vdash a-c=b-c\]
\end{theorem}
\begin{proof}
Seien \(a,b,c\in\mathbb{N}\).

	\begin{longtable}{llclll}
               \multicolumn{6}{l}{\(\vdash\):}\\
               1 &  (1) & \multicolumn{3}{l}{\(a=b\)} & \rA \\
               2 &  (2) & \multicolumn{3}{l}{\(c\leq a\)} & \rA \\
               2 &  (3) & \multicolumn{3}{l}{\(a-c=a-c\)} & \minusI{2} \\
               1,2 &  (4) & \multicolumn{3}{l}{\(a-c=b-c\)} & \rIE{1,3} \\
               \multicolumn{6}{l}{\(\dashv\):}\\
               1 &  (1) & \multicolumn{3}{l}{\(c\leq a\)} & \rA \\
               2 &  (2) & \multicolumn{3}{l}{\(a-c=b-c\)} & \rA \\
               1 &  (3) & \(a\)&=&\((a-c)+c\) & \awbInNaturalwbLeqaImpLpaMinusbRpPlusbEqualsa{1} \\
             1,2 &  (4) & &=&\((b-c)+c\) & \rIE{2,3} \\
             1,2 &  (5) & &=&\(b\) & \rIE{1,3} \\
             1,2 &  (6) & \multicolumn{3}{l}{\(a=b\)} & \rTransitivityEqRI{3,5} \\
        \end{longtable}
\end{proof}

\label{awbwcInNaturalwcLeqaLpaNotEqualsbEqvaMinuscNotEqualsbMinuscRp}
\begin{theorem}[\(a\neq b\dashv\vdash a-c\neq b-c\)]
Seien \(a,b,c\in\mathbb{N}\) und \(c\leq a\), dann gilt:
\[a\neq b\dashv\vdash a-c\neq b-c\]
\end{theorem}
\begin{proof}
        Seien \(a,b,c\in\mathbb{N}\) und \(c\leq a\), dann gilt:
        \[
	\begin{array}{llll}
		   & (1) & a=b\leftrightarrow a-c=b-c & \awbInNaturalLpaEqualsbEqvaMinusbEqualsZeroRp{} \\
		   & (2) & a\neq b\leftrightarrow a-c\neq b-c & \PLrQEqvnPLrnQ{1} \\
	\end{array}
	\]
\end{proof}

\label{awbwcInNaturalwaLeqbwbLeqcImpLpbMinusaRpPlusLpcMinusbRpEqualsLpcMinusaRp}
\begin{theorem}[\(a,b,c \in \mathbb{N},a\leq b, b\leq c\vdash (b-a)+(c-b)=(c-a)\)]
\end{theorem}
\begin{proof}
Seien \(a, b,c \in \mathbb{N}\). Im Beweis nutzen wir die Eigenschaften:
\begin{enumerate}
\item \(\ImpLpNaturalwPluswZeroRpInMonoid{}\).
\item \(\LeqIsHalfOrderOnNaturalNumbers{}\).
\item \(\FaSLpEqualsInEquivalencerelationSRp{}\).
\end{enumerate}

\begin{longtable}{llcll p{5cm}}
1 & (1) & \multicolumn{3}{l}{\(a\leq b\)}  & \rA \\
2 & (2) & \multicolumn{3}{l}{\(b\leq c\)}  & \rA \\
1,2 & (3) & \multicolumn{3}{l}{\(a\leq c\)}  & \rTransitivityOrdRI{1,2} \\
1,2 & (4) & \multicolumn{3}{l}{\(a\leq b+c\)}  & \aInNaturalwbInNaturalwcInNaturalwaLeqbImpaLeqbPlusc{1} \\
1,2 & (5) & \((b-a)+(c-b)\)&=&\(((b-a)+c)-b\)  & \awbwcInNaturalwcLeqbImpLpaPlusbRpMinuscEqualsaPlusLpbMinuscRp{2} \\
    & (6) & &=&\((c+(b-a))-b\)  & \rCommutativeMonoid{} \\
1,2 & (7) &  &=&\(((c+b)-a)-b\)  & \awbwcInNaturalwcLeqbImpLpaPlusbRpMinuscEqualsaPlusLpbMinuscRp{1} \\
1,2 & (8) &  &=&\(((b+c)-a)-b\)  & \rCommutativeMonoid{} \\
1,2 & (9) &  &=&\(((b+(c-a))-b\))  & \awbwcInNaturalwcLeqbImpLpaPlusbRpMinuscEqualsaPlusLpbMinuscRp{3} \\
1,2 & (10) &  &=&\(((c-a)+b)-b\)  & \rCommutativeMonoid{} \\
1,2 & (11) &  &=&\(c-a\)  & \awbInNaturalImpLpaPlusbRpMinusbEqualsaPlusLpbMinusbRpEqualsa{} \\
1,2 & (12) &  \((b-a)+(c-b)\)&=&\(c-a\)  & \rTransitivityEqRI{5,11} \\
\end{longtable}

\end{proof}


\label{awbwcInNaturalwaLeqbLpbLeqcEqvbMinusaLeqcMinusaRp}
\begin{theorem}[\(b\leq c \dashv\vdash b-a\leq c-a\)]
Seien \(a,b,c\in\mathbb{N}\) und \(a\leq b\), dann gilt:
\[b\leq c \dashv\vdash b-a\leq c-a\]
\end{theorem}
\begin{proof}
Seien \(a, b,c \in \mathbb{N}\). Im Beweis nutzen wir die Eigenschaften:
\begin{enumerate}
\item \(\ImpLpNaturalwPluswZeroRpInMonoid{}\).
\item \(\LeqIsHalfOrderOnNaturalNumbers{}\).
\item \(\FaSLpEqualsInEquivalencerelationSRp{}\).
\end{enumerate}
\(\vdash:\)
\begin{longtable}{llcll p{5cm}}
1 & (1) & \multicolumn{3}{l}{\(a\leq b\)}  & \rA \\
2 & (2) & \multicolumn{3}{l}{\(b\leq c\)}  & \rA \\
1,2 & (3) & \multicolumn{3}{l}{\((b-a)+(c-b)=(c-a)\)}  & \awbwcInNaturalwaLeqbwbLeqcImpLpbMinusaRpPlusLpcMinusbRpEqualsLpcMinusaRp{1,2} \\
1,2 & (4) & \multicolumn{3}{l}{\((b-a)\leq (c-a)\)}  & \rLeqNI{3} \\
\end{longtable}
\(\dashv:\)
\begin{longtable}{llcll p{5cm}}
1 & (1) & \multicolumn{3}{l}{\(b-a\leq c-a\)}  & \rA \\
2 & (2) & \multicolumn{3}{l}{\(a\leq b\)}  & \rA \\
2 & (3) & \(b\)&=&\((b-a)+a\)  & \awbInNaturalwbLeqaImpLpaMinusbRpPlusbEqualsa{2} \\
1,2 & (4) &        &\(\leq\) &\((c-a)+a\)  & \awbwcInNaturalLpaLeqbEqvaPluscLeqbPluscRp{2} \\
1,2 & (5) &        &=&\(c\)  & \awbInNaturalwbLeqaImpLpaMinusbRpPlusbEqualsa{4} \\
1,2 & (6) & \multicolumn{3}{l}{\(b\leq c\)}  & \rTransitivityOrdRI{3,5} \\
\end{longtable}

\end{proof}

\label{awbwcInNaturalwaPlusbLeqcImpaLeqcMinusb}
\begin{theorem}[\(a,b,c\in\mathbb{N},a+b\leq c \vdash a\leq c-b\)]
\end{theorem}
\begin{proof}
Seien \(a, b, c \in \mathbb{N}\). Im Beweis nutzen wir die Eigenschaft \(\ImpLpNaturalwPluswZeroRpInMonoid{}\).
\[
\begin{array}{llcll p{5cm}}
             1 & (1) & \multicolumn{3}{l}{a+b\leq c}  & \rA \\
             1 & (2) & \multicolumn{3}{l}{b\leq c}  & \rA \\
               & (3) & a&=&(a+b)-b  & \awbInNaturalImpLpaPlusbRpMinusbEqualsaPlusLpbMinusbRpEqualsa{} \\
               & (4) & &\leq& c-b  & \awbwcInNaturalwaLeqbLpbLeqcEqvbMinusaLeqcMinusaRp{1,2} \\
\end{array}
\]
\end{proof}

\label{awbwcInNaturalwaPlusbLeqcImpbLeqcMinusa}
\begin{theorem}[\(a,b,c\in\mathbb{N},a+b\leq c \vdash b\leq c-a\)]
\end{theorem}
\begin{proof}
Seien \(a, b, c \in \mathbb{N}\). Im Beweis nutzen wir die Eigenschaft \(\ImpLpNaturalwPluswZeroRpInMonoid{}\).
\[
\begin{array}{llcll p{5cm}}
               & (1) & b+a&=&a+b  & \rCommutativeMonoid{} \\
             2 & (2) & &\leq& c  & \rA \\       
             2 & (3) & b+a&\leq& c  & \rTransitivityOrdRI{1,2} \\
             2 & (4) & \multicolumn{3}{l}{b\leq c-a}  & \awbwcInNaturalwaPlusbLeqcImpaLeqcMinusb{2} \\
\end{array}
\]
\end{proof}

\label{awbwcInNaturalwbPluscLeqaImpaMinusLpbPluscRpEqualsLpaMinusbRpMinusc}
\begin{theorem}[\(a,b,c\in\mathbb{N},b+c\leq a \vdash a-(b+c)=(a-b)-c\)]
\end{theorem}
\begin{proof}
Seien \(a, b, c \in \mathbb{N}\). Im Beweis nutzen wir die Eigenschaft \(\ImpLpNaturalwPluswZeroRpInMonoid{}\).
\[
\begin{array}{llcll p{5cm}}
          \multicolumn{6}{l}{\textbf{Lemma 1:} \quad b+c\leq a, c\leq a-b \vdash (b+c)+((a-b)-c) = (b+c)+(a-(b+c))} \\
          1  & (1) & \multicolumn{3}{l}{b+c\leq a}  & \rA \\
          1  & (2) & \multicolumn{3}{l}{b\leq a}  & \awbwcInNaturalLpaPlusbLeqcImpaLeqcRp{1} \\
          1  & (3) & \multicolumn{3}{l}{(b+c)+(a-(b+c))=a}  & \minusI{1} \\
          4  & (4) & \multicolumn{3}{l}{c\leq a-b}  & \rA \\
          4  & (5) & \multicolumn{3}{l}{c+((a-b)-c)=a-b}  & \minusI{4} \\
             & (6) & (b+c)+((a-b)-c)&=&b+(c+((a-b)-c))  & \rAssociativityMonoid{} \\
          5  & (7) &                &=&b+(a-b)  & \awbInNaturalwbLeqaImpLpaMinusbRpPlusbEqualsa{4,6} \\
          5  & (8) &                &=&a  & \awbInNaturalwbLeqaImpLpaMinusbRpPlusbEqualsa{2,7} \\
          5  & (9) &                &=&(b+c)+(a-(b+c)) & \rIE{3,9} \\
          \multicolumn{6}{l}{\textbf{Beweis des Theorems:} \quad b+c\leq a \vdash a-(b+c)=(a-b)-c} \\
          
          1  & (1) & \multicolumn{3}{l}{b+c\leq a}  & \rA \\
          1  & (2) & \multicolumn{3}{l}{c\leq a-b}  & \awbwcInNaturalwaPlusbLeqcImpbLeqcMinusa{1} \\
          1  & (3) & \multicolumn{3}{l}{(b+c)+(a-(b+c))=(b+c)+((a-b)-c)}  & \text{Lemma 1(1,2)} \\
          1  & (4) & \multicolumn{3}{l}{a-(b+c)=(a-b)-c}  & \aEqualsbEqvcPlusaEqualscPlusb{3} \\
\end{array}
\]
\end{proof}

\label{awbwcInNaturalwbPluscLeqaImpaMinusLpbPluscRpEqualsLpaMinuscRpMinusb}
\begin{theorem}[\(a,b,c\in\mathbb{N},b+c\leq a \vdash a-(b+c)=(a-c)-b\)]
\end{theorem}
\begin{proof}
Seien \(a, b, c \in \mathbb{N}\). Im Beweis nutzen wir die Eigenschaft \(\ImpLpNaturalwPluswZeroRpInMonoid{}\).
\[
\begin{array}{llcll p{5cm}}
          1  & (1) & \multicolumn{3}{l}{b+c\leq a}  & \rA \\
             & (2) & a-(b+c)&=&a-(c+b)  & \rCommutativeMonoid{} \\
          1  & (3) & &=&(a-c)-b  & \awbwcInNaturalwbPluscLeqaImpaMinusLpbPluscRpEqualsLpaMinusbRpMinusc{1} \\
          1  & (4) & \multicolumn{3}{l}{a-(b+c)=(a-c)-b}  & \rTransitivityEqRI{2,3} \\
\end{array}
\]
\end{proof}

\label{awbwcInNaturalwbPluscLeqaImpLpaMinusbRpMinuscEqualsLpaMinuscRpMinusb}
\begin{theorem}[\(a,b,c\in\mathbb{N},b+c\leq a \vdash (a-b)-c=(a-c)-b\)]
\end{theorem}
\begin{proof}
Seien \(a, b, c \in \mathbb{N}\). Im Beweis nutzen wir die Eigenschaft \(\ImpLpNaturalwPluswZeroRpInMonoid{}\).
\[
\begin{array}{llcll p{5cm}}
          1  & (1) & \multicolumn{3}{l}{b+c\leq a}  & \rA \\
             & (2) & (a-b)-c&=&a-(b+c)  & \awbwcInNaturalwbPluscLeqaImpaMinusLpbPluscRpEqualsLpaMinusbRpMinusc{1} \\
          1  & (3) & &=&(a-c)-b  & \awbwcInNaturalwbPluscLeqaImpaMinusLpbPluscRpEqualsLpaMinuscRpMinusb{1} \\
          1  & (4) & \multicolumn{3}{l}{(a-b)-c=(a-c)-b}  & \rTransitivityEqRI{2,3} \\
\end{array}
\]
\end{proof}



\label{awbwcwdInNaturalwaLeqcwbLeqdImpLpcMinusaRpPlusLpdMinusbRpEqualsLpcPlusdRpMinusLpaPlusbRp}
\begin{theorem}[\(a,b,c,d\in\mathbb{N},a\leq c, b\leq d \vdash (c-a)+(d-b)=(c+d)-(a+b)\)]
\end{theorem}
\begin{proof}
Seien \(a, b, c, d \in \mathbb{N}\). Im Beweis nutzen wir die Eigenschaft \(\ImpLpNaturalwPluswZeroRpInMonoid{}\).
\[
\begin{array}{llcll p{4.3cm}}
          1   & (1) & \multicolumn{3}{l}{a\leq c}  & \rA \\
          2   & (2) & \multicolumn{3}{l}{b\leq d}  & \rA \\
          1,2 & (3) & \multicolumn{3}{l}{a+b\leq c+d}  & \awbwcwdInNaturalwaLeqbwcLeqdImpaPluscLeqbPlusd{1,2} \\
          2 & (3) &  (c-a)+(d-b)&=&((c-a)+d)-b  & \awbwcInNaturalwcLeqbImpLpaPlusbRpMinuscEqualsaPlusLpbMinuscRp{2} \\
          1   & (4) &             &=&((c+d)-a)-b  & \awbwcInNaturalwcLeqaImpLpaPlusbRpMinuscEqualsLpaMinuscRpPlusb{1}  \\
          1,2 & (5) &             &=&(c+d)-(a+b)  & \awbwcInNaturalwbPluscLeqaImpaMinusLpbPluscRpEqualsLpaMinusbRpMinusc{3} \\
          1,2 & (6) &  (c-a)+(d-b)&=&(c+d)-(a+b)  & \rTransitivityOrdRI{3,5} \\
\end{array}
\]
\end{proof}


\label{awbwcwdInNaturalwcLeqawbLeqdLpaPlusbEqualscPlusdEqvaMinuscEqualsdMinusbRp}
\begin{theorem}[\(a+b=c+d \dashv\vdash a-c=d-b\)]
Seien \(a,b,c,d\in\mathbb{N}\), \(c\leq a\) und \(b\leq d\). Dann gilt:
\[a+b=c+d \dashv\vdash a-c=d-b.\]
\end{theorem}
\begin{proof}
Seien \(a,b,c,d\in\mathbb{N}\). Im Beweis nutzen wir die Eigenschaft \(\ImpLpNaturalwPluswZeroRpInMonoid{}\).

\(\vdash\):
    \[
	\begin{array}{llcll p{5cm}}
        1       &  (1)  & \multicolumn{3}{l}{c\leq a} & \rA \\
        2       &  (2)  & \multicolumn{3}{l}{b\leq d} & \rA \\
        3       &  (3)  & \multicolumn{3}{l}{a+b=c+d} & \rA \\
                &  (4)  & \multicolumn{3}{l}{b\leq a+b} & \awbInNaturalImpaLeqbPlusa{} \\
        2       &  (5)  & \multicolumn{3}{l}{c+b\leq c+d} & \awbwcInNaturalLpaLeqbEqvcPlusaLeqcPlusbRp{2} \\                
        3       &  (6)  & \multicolumn{3}{l}{(a+b)-b=(c+d)-b} & \awbwcInNaturalwcLeqaLpaEqualsbEqvaMinuscEqualsbMinuscRp{4,3} \\
                &  (7)  & \multicolumn{3}{l}{(a+b)-b=a} & \awbInNaturalImpLpaPlusbRpMinusbEqualsaPlusLpbMinusbRpEqualsa{} \\
        3       &  (8)  & \multicolumn{3}{l}{a=(c+d)-b} & \rIE{7,6} \\
        1,3     &  (9)  & \multicolumn{3}{l}{a-c=((c+d)-b)-c} & \awbwcInNaturalwcLeqaLpaEqualsbEqvaMinuscEqualsbMinuscRp{1,8} \\
        1,2,3   &  (10)  & ((c+d)-b)-c&=&((c+d)-c)-b & \awbwcInNaturalwbPluscLeqaImpLpaMinusbRpMinuscEqualsLpaMinuscRpMinusb{5} \\
                &  (11)  &     &=&d-b & \awbInNaturalImpLpaPlusbRpMinusbEqualsaPlusLpbMinusbRpEqualsa{} \\       
        1,2,3   &  (12)  &  (c+d)-b)-c &=&d-b & \rTransitivityEqRI{10,11} \\      
        1,2,3   &  (13)  &  \multicolumn{3}{l}{a-c=d-b} & \rIE{12,9} \\  
        \end{array}
    \]
\(\dashv\):
    \[
	\begin{array}{llcll p{5cm}}
        1       &  (1)  & \multicolumn{3}{l}{c\leq a} & \rA \\
        2       &  (2)  & \multicolumn{3}{l}{b\leq d} & \rA \\
        3       &  (3)  & \multicolumn{3}{l}{a-c=d-b} & \rA \\
        3       &  (4)  & \multicolumn{3}{l}{(a-c)+c=(d-b)+c} & \aEqualsbEqvaPluscEqualsbPlusc{3} \\
        1,3     &  (5)  & \multicolumn{3}{l}{(a-c)+c=a} & \awbInNaturalwbLeqaImpLpaMinusbRpPlusbEqualsa{1,4} \\   
        1,3     &  (6)  & \multicolumn{3}{l}{a=(d-b)+c} & \rIE{5,3} \\     
        1,3     &  (7)  & \multicolumn{3}{l}{a+b=((d-b)+c)+b} & \aEqualsbEqvaPluscEqualsbPlusc{6} \\ 
                &  (8)  & ((d-b)+c)+b&=&((d-b)+b)+c & \aInMwbInMwcInMImpLpaPlusbRpPluscEqualsLpaPluscRpPlusb{7} \\ 
        2       &  (9)  & &=&d+c & \awbInNaturalwbLeqaImpLpaMinusbRpPlusbEqualsa{2} \\ 
        2       &  (10)  & ((d-b)+c)+b&=&d+c & \rTransitivityEqRI{8,9} \\ 
        1,2,3   &  (11)  & a+b&=&d+c & \rIE{10,7} \\ 
        \end{array}
    \]
\end{proof}

\section{Induzierte strikte Ordnung der natürlichen Zahlen}

Basierend auf der bereits definierten Halbordnung \(\leq\) auf \(\mathbb{N}\), können wir nun eine strikte Ordnung auf den natürlichen Zahlen einführen. Diese strikte Ordnung, die wir mit dem Symbol \(<\) bezeichnen, wird durch die Beziehung zwischen zwei Elementen \(a\) und \(b\) der Menge \(\mathbb{N}\) definiert, wobei \(a\) echt kleiner als \(b\) ist.

\begin{definition}[Strikte Ordnung \(<\) auf \(\mathbb{N}\)]
Für alle \(a, b \in \mathbb{N}\) definieren wir:
\[
a < b := a \leq b \land a \neq b.
\]
\end{definition}

Diese Definition entspricht der allgemeinen Konstruktion einer induzierten strikten Ordnung aus einer Halbordnung, bei der die Gleichheit ausgeschlossen wird, um eine strikte Vergleichsrelation zu erhalten.

\label{aLneqbEqvExnInNaturalLpnNotEqualsZeroAndaPlusnEqualsbRp}
\begin{theorem}[\(a<b\dashv\vdash a-b\neq 0\land a+(b-a)=b \) ] Seien \(a,b\in\mathbb{N}\), dann gilt:
\[a<b\dashv\vdash a-b\neq 0\land a+(b-a)=b\]
\end{theorem}
\begin{proof}
Seien \(a,b\in\mathbb{N}\). \(\ImpLpNaturalwPluswZeroRpInAbelMonoid{}\), woraus folgt:

\(\vdash:\)
	\[
	\begin{array}{llll}
		1 & (1) & a<b & \rA \\
		1 & (2) & a\leq b & \InducedStrictOrderE{1} \\
		1 & (3) & a\neq b & \InducedStrictOrderE{1} \\
            1 & (4) & a+(b-a)=b & \minusI{2} \\
            5 & (5) & b-a = 0 & \rA \\
              & (6) & a + 0 = a & \rNeutralElementMonoid{} \\
          1,5 & (7) & a + 0 = b& \rIE{5,4} \\
          1,5 & (8) & a = b& \rIE{6,7} \\
        1,3,5 & (9) & \bot & \rBI{3,8} \\
        1,3 & (12) & a-b\neq 0 & \rCI{5,9} \\
        1,5 & (13) & a-b\neq 0\land a + (a-b) = b & \rCI{12,6} \\
	\end{array}
	\]
	\(\dashv:\)
	\[
	\begin{array}{llll}
		1 & (1) & a-b\neq 0\land a + (a-b) = b & \rA \\
            1 & (2) & a-b\neq 0 & \rAEa{1} \\
            1 & (3) & a + (a-b) = b & \rAEb{1} \\
            1 & (4) & a\leq b & \rLeqNI{3} \\
              & (5) & a-b=0\leftrightarrow a=b & \awbInNaturalLpaEqualsbEqvaMinusbEqualsZeroRp{} \\
            1 & (6) & a\neq b & \PLrQwnPImpnQ{5,2} \\
            1 & (7) & a<b & \InducedStrictOrderI{6,4} \\
	\end{array}
	\]
\end{proof}

\label{aInNaturalImpaEqualsZeroOrZeroLneqa}
\begin{theorem}[\(a\in\mathbb{N}\vdash a=0\lor 0<a\)]
\end{theorem}
\begin{proof}
\[
\begin{array}{llll}
            1 & (1) & a\in\mathbb{N}  & \rA{} \\
              & (2) & a=0\lor a\neq 0 & \ImpPOrnP{} \\
            3 & (3) & a=0 & \rA \\
            3 & (4) & a=0\lor 0<a & \rOIa{3} \\
            5 & (5) & a\neq 0 & \rA \\
            1 & (6) & 0\leq a & \ImpZeroLeqa{1} \\
            1,5 & (7) & 0<a & \InducedStrictOrderI{6,5} \\
            1,5 & (8) & a=0\lor 0<a & \rOIb{7} \\
            1  & (9) & a=0\lor 0<a & \rOE{2,3,4,5,8} \\
\end{array}
\]
\end{proof}

\label{aInNaturalwZeroLneqaImpExbInNaturalLpOnePlusbEqualsaRp}
\begin{theorem}[\(a\in\mathbb{N},0<a\vdash\exists b\in\mathbb{N}(1+b=a)\)]
\end{theorem}
\begin{proof}
        Im Beweis benutzen wir die Eigenschaft \(\ImpLpNaturalwPluswZeroRpInMonoid{}\). Hiermit gilt:
\[
\begin{array}{llll}
            1 & (1) & a\in\mathbb{N}  & \rA{} \\
              & (2) & 0\in\mathbb{N}  & \zeroIsNaturalNumber{} \\
            3 & (3) & 0<a & \rA \\
            3 & (4) & 0\leq a & \InducedStrictOrderE{3} \\
            3 & (5) & 0\neq a & \InducedStrictOrderE{3} \\
            3 & (6) & 0+1\leq a & \aInNaturalwbInNaturalwaLeqbwaNotEqualsbImpaPlusOneLeqb{1,2,4,5} \\
            3 & (7) & 0+1=1 & \rNeutralElementMonoid{} \\
            3 & (8) & 1\leq a & \rIE{7,6} \\
            3 & (9) & \exists b\in\mathbb{N}(1+b=a) & \rLeqNE{8} \\
\end{array}
\]
\end{proof}


\label{aLneqbEqvaLeqbMinusOne}
\begin{theorem}[\(a<b\dashv\vdash a\leq b-1\)]
Seien \(a,b\in\mathbb{N}\), dann gilt:
\[a<b\dashv\vdash a\leq b-1\]
\end{theorem}
\begin{proof}
        Seien \(a,b\in\mathbb{N}\). Im Beweis benutzen wir die Eigenschaft \(\ImpLpNaturalwPluswZeroRpInMonoid{}\). Hiermit gilt:        
\(\vdash:\)
\[
\begin{array}{llll}
            1 & (1) & a<b  & \rA{} \\
            1 & (2) & b-a\neq 0\land a+(b-a)=b  & \aLneqbEqvExnInNaturalLpnNotEqualsZeroAndaPlusnEqualsbRp{1} \\
            1 & (3) & b-a\neq 0  & \rAEa{2} \\
            1 & (4) & a+(b-a)=b  & \rAEb{2} \\
            1 & (5) & a+(b-a)\neq 0  & \bNotEqualsZeroImpaPlusbNotEqualsZero{3} \\
            1 & (6) & (a+(b-a))-1=b-1  & \rPredecessorUniqueness{4,5} \\
              & (7) & (a+(b-a))-1=a+((b-a)-1)  & \rAssociativityMonoid{} \\
            1 & (8) & a+((b-a)-1)=b-1  & \rIE{7,6} \\
            1 & (9) & a\leq b-1  & \rLeqNI{8} \\
\end{array}
\]
\(\dashv:\)
\[
\begin{array}{llll}
            1 & (1) & a\leq b-1  & \rA{} \\
            1 & (2) & a+((b-1)-a)=b-1  & \minusI{1} \\
            1 & (3) & (a+((b-1)-a))+1=(b-1)+1  & \aEqualsbEqvaPluscEqualsbPlusc{2} \\
              & (4) & (a+((b-1)-a))+1=a+(((b-1)-a)+1)  & \rAssociativityMonoid{} \\
              & (5) & ((b-1)-a)+1\neq 0  & \nInNaturalImpnPlusOneNotEqualsZero{} \\
              & (6) & (b-1)+1=b  & \rPredecessorEa{} \\
            1 & (7) & (a+((b-1)-a))+1=b  & \rIE{6,3} \\
            1 & (8) & a+(((b-1)-a)+1)=b  & \rIE{4,7} \\
            1 & (9) & a\leq b  & \rLeqNI{8} \\
            1 & (10) & a\leq b  & \InducedStrictOrderI{5,9} \\
\end{array}
\]
\end{proof}


\label{aLneqbPlusOneEqvaLeqb}
\begin{theorem}[\(a<b+1\dashv\vdash a\leq b\)]
Seien \(a,b\in\mathbb{N}\), dann gilt:
\[a<b+1\dashv\vdash a\leq b\]
\end{theorem}
\begin{proof}
        Seien \(a,b\in\mathbb{N}\). Im Beweis benutzen wir die Eigenschaft \(\ImpLpNaturalwPluswZeroRpInMonoid{}\). Hiermit gilt:

        
\(\vdash:\)
\[
\begin{array}{llll}
            1 & (1) & a<b+1  & \rA{} \\
            1 & (2) & a\leq (b+1)-1  & \aLneqbEqvaLeqbMinusOne{1} \\
              & (3) & b=(b+1)-1  & \rPredecessorEc{} \\
            1 & (4) & a\leq b  & \rIE{3,2} \\
\end{array}
\]
\(\dashv:\)
\[
\begin{array}{llll}
            1 & (1) & a\leq b  & \rA{} \\
              & (2) & b=(b+1)-1  & \rPredecessorEc{} \\
            1 & (3) & a\leq (b+1)-1  & \rIE{2,1} \\
            1 & (4) & a<b+1  & \aLneqbEqvaLeqbMinusOne{3} \\
\end{array}
\]
\end{proof}

\label{aInNaturalLpaLneqOneEqvaEqualsZeroRp}
\begin{theorem}[\(a<1\dashv\vdash a=0\)]
Sei \(a\in\mathbb{N}\), dann gilt:
\[a<1\dashv\vdash a=0\]
\end{theorem}
\begin{proof}
Sei \(a\in\mathbb{N}\). Im Beweis verwenden wir \(\LeqIsHalfOrderOnNaturalNumbers{}\):
\(\vdash:\)
	\[
	\begin{array}{llll}
		1 & (1) & a<1 & \rA \\
		1 & (2) & a\leq 1-1 & \aLneqbEqvaLeqbMinusOne{1} \\
		   & (3) & 1-1=0 & \aInNaturalImpaMinusaEqualsZero{} \\
        1 & (4) & a\leq 0 & \rIE{3,2} \\
          & (5) & 0\leq a & \ImpZeroLeqa{} \\
        1 & (6) & a=0 & \rAntisymmetryOrdRI{4,5} \\
	\end{array}
	\]
	\(\dashv:\)
	\[
	\begin{array}{llll}
		1 & (1) & a=0 & \rA \\
          & (2) & 0\leq 1 & \ImpZeroLeqa{} \\
          & (3) & 0\neq 1 & \ImpOneNotEqualsZero{} \\
        1 & (4) & a\neq 1 & \rIE{1,3} \\
        1 & (5) & a\leq 1 & \rIE{1,2} \\
        1 & (6) & a<1 & \InducedStrictOrderI{4,5} \\
	\end{array}
	\]
\end{proof}


\label{aInNaturalwaGeqOneEqvaNotEqualsZero}
\begin{theorem}[\(a\geq 1\dashv\vdash a\neq 0\)]
Sei \(a\in\mathbb{N}\), dann gilt:
\[a\geq 1\dashv\vdash a\neq 0\]
\end{theorem}
\begin{proof}
Sei \(a\in\mathbb{N}\).
\(\vdash:\)
	\[
	\begin{array}{llll}
		1 & (1) & a\geq 1 & \rA \\
		1 & (2) & 1\leq a & \rgeqE{1} \\
            1 & (3) & \neg(a<1) & \nLpaLneqbRpEqvbLeqa{2} \\
              & (4) & a<1\leftrightarrow a=0 & \aInNaturalLpaLneqOneEqvaEqualsZeroRp{} \\
            1 & (5) & a\neq 0 & \PLrQwnPImpnQ{4,3} \\
	\end{array}
	\]
	\(\dashv:\)
	\[
	\begin{array}{llll}
		1 & (1) &  a\neq 0 & \rA \\
            & (2) & a<1\leftrightarrow a=0 & \aInNaturalLpaLneqOneEqvaEqualsZeroRp{} \\
            1 & (3) &  \neg(a<1) & \PLrQwnQImpnP{2,1} \\
            1 & (4) &  1\leq a & \nLpaLneqbRpEqvbLeqa{3} \\
            1 & (5) &  a\geq 1 & \rgeqI{4} \\
	\end{array}
	\]
\end{proof}

\label{aInNaturalwaGneqOneImpaNotEqualsZero}
\begin{theorem}[\(a\in\mathbb{N},a> 1\vdash a\neq 0\)]
\end{theorem}
\begin{proof}
Sei \(a\in\mathbb{N}\).
	\[
	\begin{array}{llll}
		1 & (1) & a>1 & \rA \\
		1 & (2) & a\geq 1 & \aGneqbImpaGeqb{1} \\
            1 & (3) & s\neq 0 & \aInNaturalwaGeqOneEqvaNotEqualsZero{2} \\
	\end{array}
	\]
\end{proof}

\label{nInNaturalwnGneqOneImpZeroLeqnMinusOne}
\begin{theorem}[\(n\in\mathbb{N}, n>1\vdash 0\leq n-1\)]
\end{theorem}
\begin{proof}
Sei \(n\in\mathbb{N}\), dann gilt:
\(\vdash\):
    \[
	\begin{array}{llclll}
            1 &  (1)  & \multicolumn{3}{l}{n>1} & \rA \\
            1 &  (2)  & \multicolumn{3}{l}{n\neq 0} & \aInNaturalwaGneqOneImpaNotEqualsZero{1} \\
            1 &  (3)  & \multicolumn{3}{l}{n-1\in\mathbb{N}} & \rPredecessorI{2} \\
            1 &  (4)  & \multicolumn{3}{l}{0\leq n-1} & \ImpZeroLeqa{3} \\
        \end{array}
    \]
\end{proof}


\subsection{Invarianz der induzierten strikten Ordnungsrelation gegenüber Addition}

\label{awbwcInNaturalLpaLneqbEqvaPluscLneqbPluscRp}
\begin{theorem}[\(a<b\dashv\vdash a+c<b+c\)]
Seien \(a,b,c\in\mathbb{N}\), dann gilt:
\[a<b\dashv\vdash a+c<b+c\]
\end{theorem}
\begin{proof}
        Seien \(a,b,c\in\mathbb{N}\). 
\(\vdash:\)
\[
\begin{array}{llll}
            1 & (1) & a<b  & \rA{} \\
            1 & (2) & a\leq b  & \InducedStrictOrderE{1} \\
            1 & (3) & a\neq b  & \InducedStrictOrderE{1} \\
            1 & (4) & a+c\leq b+c  &  \awbwcInNaturalLpaLeqbEqvaPluscLeqbPluscRp{2} \\
            1 & (5) & a+c\neq b+c  & \aNotEqualsbEqvaPluscNotEqualsbPlusc{3} \\
            1 & (6) & a+c<b+c & \InducedStrictOrderI{4,5} \\
\end{array}
\]
\(\dashv:\)
\[
\begin{array}{llll}
            1 & (1) & a+c<b+c  & \rA{} \\
            1 & (2) & a+c\leq b+c  & \InducedStrictOrderE{1} \\
            1 & (3) & a+c\neq b+c  & \InducedStrictOrderE{1} \\
            1 & (4) & a\leq b  & \awbwcInNaturalLpaLeqbEqvaPluscLeqbPluscRp{2}
            \\
            1 & (5) & a\neq b  & \aNotEqualsbEqvaPluscNotEqualsbPlusc{3}
            \\
            1 & (5) & a< b  & \InducedStrictOrderI{4,5}
            \\
            
\end{array}
\]
\end{proof}

\label{aLneqbEqvcPlusaLneqcPlusb}
\begin{theorem}[\(a<b\dashv\vdash c+a<c+b\)]
Seien \(a,b,c\in\mathbb{N}\), dann gilt:
\[a<b\dashv\vdash c+a<c+b\]
\end{theorem}
\begin{proof}
        Seien \(a,b,c\in\mathbb{N}\). 
\(\vdash:\)
\[
\begin{array}{llll}
            1 & (1) & a<b  & \rA{} \\
            1 & (2) & a\leq b  & \InducedStrictOrderE{1} \\
            1 & (3) & a\neq b  & \InducedStrictOrderE{1} \\
            1 & (4) & c+a\leq c+b  &  \awbwcInNaturalLpaLeqbEqvcPlusaLeqcPlusbRp{2} \\
            1 & (5) & c+a\neq c+b  & \aNotEqualsbEqvcPlusaNotEqualscPlusb{3} \\
            1 & (6) & c+a<c+b & \InducedStrictOrderI{4,5} \\
\end{array}
\]
\(\dashv:\)
\[
\begin{array}{llll}
            1 & (1) & a+c<b+c  & \rA{} \\
            1 & (2) & a+c\leq b+c  & \InducedStrictOrderE{1} \\
            1 & (3) & a+c\neq b+c  & \InducedStrictOrderE{1} \\
            1 & (4) & a\leq b  & \awbwcInNaturalLpaLeqbEqvcPlusaLeqcPlusbRp{2}
            \\
            1 & (5) & a\neq b  & \aNotEqualsbEqvcPlusaNotEqualscPlusb{3}
            \\
            1 & (5) & a<b  & \InducedStrictOrderI{4,5}
            \\
            
\end{array}
\]
\end{proof}

\label{aInNaturalwbInNaturalwcInNaturalwaLeqbwcNotEqualsZeroImpaLneqbPlusc}
\begin{theorem}[\(a\in\mathbb{N},b\in\mathbb{N},c\in\mathbb{N}, a\leq b,c\neq 0\vdash a<b+c\)]
Seien \(a,b,c\in\mathbb{N}\), dann gilt:
\[a\leq b,c\neq 0\vdash a<b+c\]
\end{theorem}
\begin{proof}
        Seien \(a,b,c\in\mathbb{N}\). Im Beweis benutzen wir die Eigenschaften:
\begin{enumerate}
    \item \(\ImpLpNaturalwPluswZeroRpInMonoid{}\).
    \item \(\LeqIsHalfOrderOnNaturalNumbers{}\).
    \item \(\FaSLpEqualsInEquivalencerelationSRp{}\).
\end{enumerate}
Hiermit gilt:
\[
\begin{array}{llclll}
            1 & (1) & \multicolumn{3}{l}{c\neq 0}  & \rA \\
            1 & (2) & \multicolumn{3}{l}{c-1\in\mathbb{N}}  & \rPredecessorI{1} \\
            3 & (3) & a&\leq &b  & \rA  \\
            1,3 & (4) & &\leq &b+(c-1)  &  \aInNaturalwbInNaturalwcInNaturalwaLeqbImpaLeqbPlusc{2,3} \\
            1,3 & (5) & &< &((b+(c-1))+1  &  \aLneqbPlusOneEqvaLeqb{4} \\
            1,3 & (6) & &=& b+((c-1)+1) & \rAssociativityMonoid{} \\
            1,3 & (7) & &=& b+c & \rPredecessorI{2} \\
            1,3 & (8) & \multicolumn{3}{l}{a<b+c} & \rTransitivityEqRI{3,7} \\
\end{array}
\]
\end{proof}

\label{aInNaturalwbInNaturalwcInNaturalwaLeqbwcNotEqualsZeroImpaLneqcPlusb}
\begin{theorem}[\(a\in\mathbb{N},b\in\mathbb{N},c\in\mathbb{N}, a\leq b,c\neq 0 \vdash a<c+b\)]
\end{theorem}
\begin{proof}
        Seien \(a,b,c\in\mathbb{N}\). Im Beweis benutzen wir die Eigenschaften:
\begin{enumerate}
    \item \(\ImpLpNaturalwPluswZeroRpInMonoid{}\).
    \item \(\LeqIsHalfOrderOnNaturalNumbers{}\).
    \item \(\FaSLpEqualsInEquivalencerelationSRp{}\).
\end{enumerate}
Hiermit gilt:
\[
\begin{array}{llclll}
            1 & (1) & \multicolumn{3}{l}{c\neq 0}  & \rA \\
            2 & (2) & a&\leq &b  & \rA  \\
            1,2 & (3) & &<&b+c  &  \aInNaturalwbInNaturalwcInNaturalwaLeqbwcNotEqualsZeroImpaLneqbPlusc{1,2} \\
            1,2 & (4) & &= &c+b  &  \rCommutativeMonoid{} \\
            1,2 & (5) & \multicolumn{3}{l}{a<c+b} & \rTransitivityEqRI{2,4} \\
\end{array}
\]
\end{proof}

\label{awbInNaturalwbNotEqualsZeroImpaLneqaPlusb}
\begin{theorem}[\(a,b\in\mathbb{N},b\neq 0\vdash a<a+b\)]
\end{theorem}
\begin{proof}
        Seien \(a,b\in\mathbb{N}\). Im Beweis benutzen wir die Eigenschaften:
\begin{enumerate}
    \item \(\LeqIsHalfOrderOnNaturalNumbers{}\).
\end{enumerate}
Hiermit gilt:
\[
\begin{array}{llll}
              & (1) & a\leq a  & \rReflexivityOrdRI{} \\
            2 & (2) & b\neq 0  & \rA \\
            2 & (3) & a<a+b & \aInNaturalwbInNaturalwcInNaturalwaLeqbwcNotEqualsZeroImpaLneqbPlusc{1,2} \\
\end{array}
\]
\end{proof}

\label{awbInNaturalbNotEqualsZeroImpaLneqbPlusa}
\begin{theorem}[\(a,b\in\mathbb{N}b\neq 0\vdash a<b+a\)]
\end{theorem}
\begin{proof}
        Seien \(a,b\in\mathbb{N}\). Im Beweis benutzen wir die Eigenschaften:
\begin{enumerate}
    \item \(\LeqIsHalfOrderOnNaturalNumbers{}\).
\end{enumerate}
Hiermit gilt:
\[
\begin{array}{llll}
              & (1) & a\leq a  & \rReflexivityOrdRI{} \\
            2 & (2) & b\neq 0  & \rA \\
            2 & (3) & a<b+a & \aInNaturalwbInNaturalwcInNaturalwaLeqbwcNotEqualsZeroImpaLneqcPlusb{1,2} \\
\end{array}
\]
\end{proof}

\label{awbwcInNaturalLpaPlusbLeqcwbNotEqualsZeroImpaLneqc}
\begin{theorem}[\(a,b,c\in\mathbb{N}(a+b\leq c,b\neq 0 \vdash a<c)\)]
\end{theorem}
\begin{proof}
        Seien \(a,b,c\in\mathbb{N}\). Im Beweis benutzen wir die Eigenschaften:
\begin{enumerate}
    \item \(\LeqIsHalfOrderOnNaturalNumbers{}\).
\end{enumerate}
Hiermit gilt:
\[
\begin{array}{llclll}
            1 & (1) & \multicolumn{3}{l}{b\neq 0}  & \rA \\
            1 & (2) & a&<& a+b  & \awbInNaturalwbNotEqualsZeroImpaLneqaPlusb{1} \\
            1,3 & (3) &  &\leq &c & \rA \\
            1,3 & (4) & \multicolumn{3}{l}{a<c} & \rTransitivityOrdRI{2,3} \\
\end{array}
\]
\end{proof}


\label{awbwcInNaturalLpaPlusbLeqcImpbLneqcRp}
\begin{theorem}[\(a,b,c\in\mathbb{N}(a+b\leq c\vdash b<c)\)]
\end{theorem}
\begin{proof}
        Seien \(a,b,c\in\mathbb{N}\). Im Beweis benutzen wir die Eigenschaften:
\begin{enumerate}
    \item \(\LeqIsHalfOrderOnNaturalNumbers{}\).
\end{enumerate}
Hiermit gilt:
\[
\begin{array}{llclll}
            1 & (1) & \multicolumn{3}{l}{b\neq 0}  & \rA \\
            1 & (2) & b&<& a+b  & \awbInNaturalbNotEqualsZeroImpaLneqbPlusa{1} \\
            1,3 & (3) &  &\leq &c & \rA \\
            1,3 & (4) & \multicolumn{3}{l}{b<c} & \rTransitivityOrdRI{2,3} \\
\end{array}
\]
\end{proof}

\subsection{Eigenschaften der Differenz}

\label{awbInNaturalLpaLneqbEqvZeroLneqbMinusaRp}
\begin{theorem}[\(a,b\in\mathbb{N},a<b\dashv\vdash 0<b-a\)]
Seien \(a,b\in\mathbb{N}\), dann gilt:
\[a<b\dashv\vdash 0<b-a\]
\end{theorem}
\begin{proof}
    \(\vdash\):
    \[
    \begin{array}{llll}
        1 & (1) & a<b  & \rA \\
        1 & (2) & a\leq b  & \rLeqNE{1} \\
        1 & (3) & 0\leq b-a  & \awbInNaturalLpaLeqbEqvZeroLeqbMinusaRp{2} \\
        1 & (4) & a\neq b  & \rLeqNE{1} \\
        1 & (5) & 0\neq a-b  & \aNotEqualsbEqvaMinusbNotEqualsZero{4} \\
        1 & (6) & 0\leq a-b  & \rLeqNI{3,5} \\
    \end{array}
    \]
    \(\dashv\):
    \[
    \begin{array}{llll}
        1 & (1) & 0<b-a  & \rA \\
        1 & (2) & 0\leq b-a  & \rLeqNE{1} \\
        1 & (3) & a\leq b  & \awbInNaturalLpaLeqbEqvZeroLeqbMinusaRp{2} \\
        1 & (4) & 0\neq b-a  & \rLeqNE{1} \\
        1 & (5) & a\neq b  & \aNotEqualsbEqvaMinusbNotEqualsZero{4} \\
        1 & (6) & a<b  & \rLeqNI{3,5} \\
    \end{array}
    \]
\end{proof}


\label{awbwcInNaturalwaLeqbLpbLneqcEqvbMinusaLneqcMinusaRp}
\begin{theorem}[\(b<c \dashv\vdash b-a<c-a\)]
Seien \(a,b,c\in\mathbb{N}\) und \(a\leq b\), dann gilt:
\[b<c \dashv\vdash b-a<c-a\]
\end{theorem}
\begin{proof}
Seien \(a,b,c\in\mathbb{N}\).:
    \(\vdash\):
    \[
    \begin{array}{llll}
        1 & (1) & b<c  & \rA \\
        2 & (2) & a\leq b  & \rA \\
        1 & (3) & b\leq c  & \rLeqNE{1} \\
        1 & (4) & b-a\leq c-a  & \awbwcInNaturalwaLeqbLpbLeqcEqvbMinusaLeqcMinusaRp{1,2} \\
        1 & (5) & b\neq c  & \rLeqNE{1} \\
      1,2 & (6) & b-a\neq c-a  & \awbwcInNaturalwcLeqaLpaNotEqualsbEqvaMinuscNotEqualsbMinuscRp{2,5} \\
    \end{array}
    \]
    \(\dashv\):
    \[
    \begin{array}{llll}
        1 & (1) & b-a<c-a  & \rA \\
        2 & (2) & a\leq b  & \rA \\
        1 & (3) & b-a\leq c-a  & \rLeqNE{1} \\
        1 & (4) & b\leq c  & \awbwcInNaturalwaLeqbLpbLeqcEqvbMinusaLeqcMinusaRp{1,2} \\
        1 & (5) & b-a\neq c-a  & \rLeqNE{1} \\
      1,2 & (6) & b\neq c  & \awbwcInNaturalwcLeqaLpaNotEqualsbEqvaMinuscNotEqualsbMinuscRp{2,5} \\
    \end{array}
    \]
\end{proof}

\label{bLneqcwaLeqbImpbMinusaLneqc}
\begin{theorem}[\(b<c, a\leq b\vdash b-a<c\)]
\end{theorem}
\begin{proof}
Seien \(a, b, c \in \mathbb{N}\). 
\[
\begin{array}{llclll}
          1  & (1) & \multicolumn{3}{l}{a\leq b}  & \rA \\
          1  & (2) & \multicolumn{3}{l}{b-a\in\mathbb{N}}  & \minusI{1} \\
          1  & (3) & b-a&\leq& a+(b-a)  & \awbInNaturalImpaLeqaPlusb{2} \\
          1  & (4) & &=& b  & \minusI{1} \\
          5  & (5) & &< & c  & \rA \\
          1,5& (6) & b-a&<& c  & \rTransitivityOrdRI{3,5} \\
\end{array}
\]
\end{proof}

\section{Endliche Teilmengen der natürlichen Zahlen}

\begin{definition}[Endliches Teilmengen der natürlichen Zahlen]
    Sei \(i, n \in \mathbb{N}\) mit \(i \leq n\). Dann definieren wir das endliche Teilmengen der natürlichen Zahlen von \(i\) bis \(n\) als die Menge
    \[
    \{i, i+1, \dots, n\} := \{ x \in \mathbb{N} \mid i \leq x \land x \leq n \}.
    \]
    Für den Spezialfall \(i = 0\) und \(n\in\mathbb{N}\) mit \(n \neq 0\) schreiben wir \(\{0, 1, \dots, n-1\} := \{ x \in \mathbb{N} \mid x < n \}\).
\end{definition}

\subsubsection{Einführungsregeln für endliche Teilmengen der natürlichen Zahlen}

Die Einführungsregeln ermöglichen es, die Zugehörigkeit eines Elements zu einem endlichen Segment der natürlichen Zahlen zu zeigen, wenn die jeweiligen Bedingungen erfüllt sind.

\paragraph{Einführungsregel für die Menge}
\label{rule:rSegmentI}
Die Einführungsregel für die Menge \(\{i, i+1, \dots, n\}\) besagt, dass ein Element \(x \in \mathbb{N}\) in \(\{i, i+1, \dots, n\}\) liegt, wenn \(i \leq x \leq n\) gilt.
\[
\begin{array}{llll}
    i   & (1) & i \leq x & \dots \\
    j   & (2) & x \leq n & \dots \\
    i,j & (3) & x \in \{i, i+1, \dots, n\} & \rSegmentI{1,2}
\end{array}
\]

\paragraph{Einführungsregel für die Menge}
\label{rule:rSegmentZeroI}
Die Einführungsregel für die Menge \(\{0, 1, \dots, n-1\}\) besagt, dass ein Element \(x \in \mathbb{N}\) in \(\{0, 1, \dots, n-1\}\) liegt, wenn \(x < n\) gilt.
\[
\begin{array}{llll}
    i   & (1) & x < n & \dots \\
    i   & (2) & x \in \{0, 1, \dots, n-1\} & \rSegmentZeroI{1}
\end{array}
\]

\[
\begin{array}{llll}
    i   & (1) & x \leq n-1 & \dots \\
    i   & (2) & x \in \{0, 1, \dots, n-1\} & \rSegmentZeroI{1}
\end{array}
\]

\(i\) und \(j\) sind dabei Listen von Annahmen.

\subsubsection{Eliminationsregeln für endliche Teilmengen der natürlichen Zahlen}
\label{rule:rSegmentE}
Die Eliminationsregeln ermöglichen es, aus der Zugehörigkeit eines Elements zu einer endlichen Menge die entsprechenden Bedingungen abzuleiten.

\paragraph{Eliminationsregel für die Menge}
Die Eliminationsregel besagt, dass wenn \(x \in \{i, i+1, \dots, n\}\) gilt, dann sowohl \(i \leq x\) als auch \(x \leq n\) folgt.
\[
\begin{array}{llll}
    i & (1) & x \in \{i, i+1, \dots, n\} & \dots \\
    i & (2) & i \leq x & \rSegmentE{1} \\
    i & (3) & x \leq n & \rSegmentE{1} \\
    i & (4) & i \leq n & \rSegmentE{1} \\
    i & (5) & i\in\mathbb{N} & \rSegmentE{1} \\
    i & (6) & n\in\mathbb{N} & \rSegmentE{1} \\
\end{array}
\]

\paragraph{Eliminationsregel für die Menge \(\{0, 1, \dots, n-1\}\)}
\label{rule:rSegmentZeroE}
Die Eliminationsregel besagt, dass wenn \(x \in \{0, 1, \dots, n-1\}\) gilt, dann \(x < n\) folgt.
\[
\begin{array}{llll}
    i & (1) & x \in \{0, 1, \dots, n-1\} & \dots \\
    i & (2) & x < n & \rSegmentZeroE{1} \\
    i & (3) & 0 \leq n-1 & \rSegmentZeroE{1} \\
    i & (4) & n-1\in\mathbb{N} & \rSegmentZeroE{1} \\
\end{array}
\]

\label{nInNaturalwnNotEqualsZeroImpZeroInLbZerowOnewDotswnMinusOneRb}
\begin{theorem}[\(n\in\mathbb{N},n\neq 0\vdash 0\in\{0, 1, \dots, n-1\}\)]
\end{theorem}
\begin{proof}
Sei \(n\in\mathbb{N}\), dann gilt:
        \[
	\begin{array}{llll}
            1       &  (1) & n\neq 0 & \rA \\
            1       &  (2) & n-1\in\mathbb{N} & \rPredecessorEa{1} \\
            1       &  (3) & 0\leq n-1\in\mathbb{N} & \ImpZeroLeqa{2} \\
            1       &  (4) & 0\leq n-1\in\mathbb{N} & \rSegmentZeroI{3} \\
        \end{array}
	\]
\end{proof}

\label{nInNaturalwnGneqOneImpZeroInLbZerowOnewDotswnMinusOneRb}
\begin{theorem}[\(n\in\mathbb{N},n>1\vdash 0\in\{0, 1, \dots, n-1\}\)]
\end{theorem}
\begin{proof}
Sei \(n\in\mathbb{N}\), dann gilt:
        \[
	\begin{array}{llll}
            1       &  (1) & n>1 & \rA \\
            1       &  (2) & n\neq 0 & \aInNaturalwaGneqOneImpaNotEqualsZero{1} \\
            1       &  (3) & 0\leq n-1\in\mathbb{N} & \nInNaturalwnNotEqualsZeroImpZeroInLbZerowOnewDotswnMinusOneRb{2} \\
        \end{array}
	\]
\end{proof}


\section{Prinzip der starken Induktion}

Um das Prinzip der starken Induktion zu zeigen, benötigen wir zunächst ein paar Hilfssätze.

\label{aInNaturalwbInNaturalwaLeqbPlusOneImpaLeqbOraEqualsbPlusOne}
\begin{theorem}[\(a\in\mathbb{N},b\in\mathbb{N},a\leq b+1\vdash a\leq b\lor a=b+1\)]
\end{theorem}
\begin{proof}
        Seien \(a,b\in\mathbb{N}\). Im Beweis benutzen wir die Eigenschaft \(\ImpLpNaturalwPluswZeroRpInAbelMonoid{}\). Hiermit gilt:
        \[
	\begin{array}{llll}
            1       &  (1) & a\leq b+1 & \rA \\
            1       &  (2) & (b+1)-a\in\mathbb{N} & \minusI{1} \\
            1       &  (3) & (b+1)-a=0\lor 0<(b+1)-a & \aInNaturalImpaEqualsZeroOrZeroLneqa{2} \\
            4       &  (4) & (b+1)-a=0 & \rA \\  
            4       &  (5) & a=(b+1) & \awbInNaturalLpbEqualsaEqvaMinusbEqualsZeroRp{4} \\ 
            4       &  (6) & a\leq b\lor a = b+1 & \rOIb{5}\\
            7       &  (7) & 0<(b+1)-a & \rA\\
            7       &  (8) & a<b+1 & \awbInNaturalLpaLneqbEqvZeroLneqbMinusaRp{7}\\
            7       &  (9) & a\leq b+1 & \aLneqbPlusOneEqvaLeqb{8}\\
            7       &  (10) & a\leq b\lor a=b+1 & \rOIa{9} \\
            1       &  (11) & a\leq b\lor a=b+1 & \rOE{3,4,6,7,10} \\
        \end{array}
	\]
\end{proof}



\label{FanInNaturalLpFakInNaturalLpkLeqnToPLpkRpImpFanInNaturalLpPLpnRpRpRpRp}
\begin{theorem}[\(\forall n\in\mathbb{N}(\forall k\in\mathbb{N}(k\leq n\rightarrow P(k))\vdash \forall n\in\mathbb{N}(P(n)))\)]
\end{theorem}
\begin{proof}
\[
\begin{array}{llll}
1 & (1) & \forall n\in\mathbb{N}\left(\forall k\in\mathbb{N}(k\leq n\rightarrow P(k)\right) & \rA \\
2 & (2) & m\in\mathbb{N} & \rA \\
1,2 & (3) & \forall k\in\mathbb{N}\left(k\leq m\rightarrow P(k)\right) & \rSetUEb{1,2} \\
1,2 & (4) & m\leq m\rightarrow P(m) & \rSetUEb{3,2} \\
    & (5) & m\leq m & \rReflexivityOrdRI{} \\
1,2 & (6) & P(m) & \rRE{5,4} \\
1  & (7) & \forall n\in\mathbb{N}(P(n)) & \rSetUIa{2,6} \\
\end{array}
\]
\end{proof}

\begin{lemma}[Induktionsanfang (\(IA\))]
\[P(0)\vdash \forall k\in\mathbb{N}(k\leq 0\rightarrow P(k))\]
\end{lemma}
\begin{proof}
\[
\begin{array}{llll}
1 & (1) & P(0) & \rA \\
2 & (2) & k\in\mathbb{N} & \rA \\
3 & (3) & k\leq 0 & \rA \\
2,3 & (4) & k=0 & \aInNaturalwaLeqZeroImpaEqualsZero{2,3} \\
1,2,3 & (5) & P(k) & \rIE{1,4} \\
1,2 & (6) & k\leq 0\rightarrow P(k) & \rRI{3,5} \\
1 & (7) & \forall k\in\mathbb{N}(k\leq 0\rightarrow P(k)) & \rSetUIa{2,6} \\
\\
\end{array}
\]
\end{proof}

\begin{lemma}[Induktionsschritt unter Bedingung \(P(m+1)\) (\(IS_{P(m+1)}\))]
\[m\in\mathbb{N}, P(m+1), \forall k\in\mathbb{N}(k\leq m\rightarrow P(k)) \vdash \forall k\in\mathbb{N}(k\leq m+1\rightarrow P(k))\]
\end{lemma}
\begin{proof}
   \[
\begin{array}{lll p{3.56cm}}
1 & (1) & m\in\mathbb{N} & \rA \\
2 & (2) & P(m+1) & \rA \\
3 & (3) & \forall k\in\mathbb{N}(k\leq m\rightarrow P(k)) & \rA \\
4 & (4) & k\in\mathbb{N} & \rA \\
5 & (5) & k\leq m+1 & \rA \\
1,4,5 & (6) & k\leq m\lor k=m+1 & \aInNaturalwbInNaturalwaLeqbPlusOneImpaLeqbOraEqualsbPlusOne{1,4,5} \\
7 & (7) & k\leq m & \rA \\
7 & (8) & k\leq m & \rA \\
4 & (9) & k\leq m\rightarrow P(k) & \rSetUEc{4,3} \\
4,7 & (10) & P(k) & \rRE{8,9} \\
11 & (11) & k=m+1 & \rA \\
2,3,11 & (12) & P(k) & \rIE{11,2} \\
1,2,3,4,5 & (13) & P(k) & \rOE{6,7,10,11,12} \\
1,2,3,4 & (14) & k\leq m+1\rightarrow P(k) & \rRI{5,13} \\
1,2,3 & (15) & \forall k\in\mathbb{N}(k\leq m+1\rightarrow P(k)) & \rSetUIa{4,14} \end{array}
\] 
\end{proof}


\label{PLpZeroRpwFanInNaturalLpLpFakInNaturalLpkLeqnToPLpkRpRpToPLpnPlusOneRpRpImpFanInNaturalLpPLpnRpRp}
\begin{theorem}[\(P(0), \forall n \in \mathbb{N} ((\forall k\in\mathbb{N}(k\leq n\rightarrow P(k)) \rightarrow P(n+1)) \vdash \forall n \in \mathbb{N}(P(n))\) (Starkes Induktionsprinzip)]
\end{theorem}
\begin{proof}

   \[
\begin{array}{lll p{3.56cm}}
1 & (1) & P(0) & \rA \\
2 & (2) & \forall n \in \mathbb{N}(\forall k\in\mathbb{N}(k\leq n\rightarrow P(k)) \rightarrow P(n+1)) & \rA \\
1 & (3) & \forall k\in\mathbb{N}(k\leq 0\rightarrow P(k)) & \ensuremath{IA(1)} \\
4 & (4) & m\in\mathbb{N} & \rA \\
5 & (5) & \forall k\in\mathbb{N}(k\leq m\rightarrow P(k)) & \rA \\
2,4 & (6) & \forall k\in\mathbb{N}(k\leq m\rightarrow P(k)) \rightarrow P(m+1) & \rSetUEc{2,4} \\
2,4,5 & (7) & P(m+1) & \rRE{6,5} \\
2,4,5 & (8) & \forall k\in\mathbb{N}(k\leq m+1\rightarrow P(k)) & \ensuremath{IS_{P(m+1)}(4,7,5)} \\
1,2 & (9) & \forall n\in\mathbb{N}(\forall k\in\mathbb{N}(k\leq n\rightarrow P(k)) & \rInductionN{3,4,5,8} \\
1,2 & (10) & \forall n\in\mathbb{N}(P(n))) & \FanInNaturalLpFakInNaturalLpkLeqnToPLpkRpImpFanInNaturalLpPLpnRpRpRpRp{9} \\
\end{array}
\] 
\end{proof}


\subsubsection{Regel des starken Induktionsprinzips über den natürlichen Zahlen}
\label{rule:rStrongInductionN}

Die Regel des starken Induktionsprinzips über den natürlichen Zahlen (\(\rStrongInductionN{}\)) erlaubt es, nach der Herleitung von \(P(0)\) und der Annahme, dass für ein beliebiges \(n \in \mathbb{N}\) die Aussage \(P(n+1)\) aus der Voraussetzung \(P(k)\) für alle \(k \leq n\) folgt, direkt auf die Aussage \(\forall n \in \mathbb{N} P(n)\) zu schließen. Es ist somit nicht mehr notwendig, die Zwischenschritte \(\forall n \in \mathbb{N} ((\forall k\in\mathbb{N}(k\leq n\rightarrow P(k)) \rightarrow P(n+1))\) explizit aufzuschreiben.

\[
\begin{array}{llll}
    i & (1) & P(0) & ... \\
    2 & (2) & \forall k \in \mathbb{N} (k \leq n \rightarrow P(k)) & \rA \\
    3 & (3) & n \in \mathbb{N} & \rA \\
    2,3,j & (4) & P(n+1) & ... \\
    i,j & (5) & \forall n \in \mathbb{N} P(n) & \rStrongInductionN{1,2,3,4}
\end{array}
\]

Hierbei beziehen sich \(n\) und \(P(n)\) auf ein beliebiges Element und eine beliebige Aussage über \(\mathbb{N}\). Der Ausdruck \(\rStrongInductionN{1,4}\) zeigt an, dass die Regel des starken Induktionsprinzips angewendet wurde, um die allgemeine Aussage \(\forall n \in \mathbb{N} P(n)\) abzuleiten.

\(i\) und \(j\) sind dabei Listen von Annahmen, und \(n\) kommt in keiner der Annahmen \(i\) und \(j\) vor, aus denen \(P(0)\) und \(P(n+1)\) abgeleitet werden.

\section{Extremale Elemente und Schranken der natürlichen Zahlen}
\begin{theorem}[\(\vdash \min(\mathbb{N}) = 0\)]
\end{theorem}
\begin{proof}
        Sei \(a\in\mathbb{N}\), dann gilt:
	\[
	\begin{array}{llll}
          & (1) & 0\leq \mathbb{N} & \zeroIsNaturalNumber{} \\
		 & (2) & 0\leq a & \ImpZeroLeqa{} \\
          & (3) & \forall a\in\mathbb{N}(0\leq a) & \rSetUIa{2} \\
          & (4) & \min(\mathbb{N}) = 0 & \rMinI{1,3} \\
	\end{array}
	\]
\end{proof}

\subsection{Das Wohlordnungsprinzip}
\begin{lemma}[1]
\[A\subseteq\mathbb{N},\forall n\in A\exists m\in A(m<n)\vdash 0\notin A\]
\end{lemma}
\begin{proof}
	\[
	\begin{array}{llll}
        1  & (1) & A\subseteq\mathbb{N} & \rA \\
	2  & (2) & \forall n\in A\exists m\in A(m<n) & \rA \\
        3  & (3) & 0\in A & \rA \\
        2,3& (4) & \exists m\in A(m<0) & \rSetUEc{3,2} \\
        5  & (5) & m\in A\land m<0 & \rA \\
        5  & (6) & m\in A          & \rAEa{5} \\
        5  & (7) & m<0             & \rAEb{5} \\
        1,5& (8) & m\in\mathbb{N}  & \subseteqE{6,1} \\
        1,5& (9) & 0\leq m & \ImpZeroLeqa{8} \\
        1,5& (10) & \neg(m<0) & \nLpaLneqbRpEqvbLeqa{9} \\
        1,5& (11) & \bot & \rBI{7,10} \\
        1,2,3& (12) & \bot & \rOE{4,5,12} \\
        1,2& (13) & 0\notin A & \rCI{3,12} \\
	\end{array}
	\]
\end{proof}

\begin{lemma}[2]
\[\neg(\forall n\in A\exists m\in A(m<n))\vdash \exists n\in A(n=\min(A))\]
\end{lemma}
\begin{proof}
	\[
	\begin{array}{llll}
        1  & (1) & \neg(\forall n\in A\exists m\in A(m<n)) & \rA \\
        1  & (2) & \exists n\in A \forall m\in A(\neg(m<n)) & \nLpFaxInAExyInBLpPLpxwyRpRpRpEqvExxInAFayInBLpnPLpxwyRpRp{1} \\
        3  & (3) & n\in A\land \forall m\in A(\neg(m<n) & \rA\\
        3  & (4) & n\in A & \rAEa{3}\\
        3  & (5) & \forall m\in A(\neg(m<n) & \rAEb{3}\\
        3  & (6) & m\in A\rightarrow \neg(m<n) & \rSetUEb{5}\\
        7  & (7) & m\in A & \rA\\
        3,7& (8) & \neg(m<n) & \rRE{7,6}\\
        3,7& (9) & n\leq m & \nLpaLneqbRpEqvbLeqa{8}\\
        3  & (10) & \forall m\in A(n\leq m) & \rSetUIa{7,9}\\
        3  & (11) & n=\min(A) & \rMinI{4,10}\\
        3  & (12) & \exists n\in A(n=\min(A)) & \rSetEIa{4,11}\\
        1  & (13) & \exists n\in A(n=\min(A)) & \rSetEEb{2,3,12}\\
	\end{array}
	\]
\end{proof}

\label{ASubseteqNaturalwANotEqualsEmptysetImpExnInALpnEqualsMinLpARpRp}
\begin{theorem}[\(A\subseteq\mathbb{N},A\neq\emptyset\vdash\exists n\in A(n=\min(A))\) (Wohlordnungsprinzip)]
\end{theorem}
Wir beweisen die Aussage mit Hilfe des Prinzips der starken Induktion über \(n\in\mathbb{N}\):
\begin{proof}
	\[
	\begin{array}{llll}
        1  & (1) & A\subseteq\mathbb{N}  & \rA \\
		2  & (2) & A\neq\emptyset        & \rA \\
        3  & (3) & \forall n\in A\exists m\in A(m<n) & \rA \\
        1,3& (4) & 0\notin A & \text{Lemma (1)}(1,3) \\
        5  & (5) & \forall k\in\mathbb{N}(k\leq n\rightarrow k\notin A) & \rA \\
        6  & (6) & n+1\in A & \rA \\
        3,6& (7) & \exists m\in A(m<n+1)    & \rSetUEc{3,6} \\
        8  & (8) & m\in A\land m<n+1        & \rA \\
        8  & (9) & m\in A                   & \rAEa{8} \\
        8  & (10) & m<n+1                   & \rAEb{8} \\
        8  & (11) & m\leq n                 & \aLneqbPlusOneEqvaLeqb{10} \\
        1,8& (12) & m\in\mathbb{N}          & \subseteqE{9,1} \\
        1,5,8& (13) & m\leq n\rightarrow m\notin A          & \rSetUEc{12,5} \\
        1,5,8& (14) & m\notin A          & \rRE{11,13} \\
        1,5,8& (15) & \bot          & \rBI{9,14} \\
        1,3,5,6& (16) & \bot          & \rEE{7,8,15} \\
        1,3,5& (17) & n+1\notin A      & \rCI{6,16} \\
        1,3& (18) & \forall n\in\mathbb{N}(n\notin A)      & \rStrongInductionN{4,5,17} \\
        1,3& (19) & A=\emptyset      & \ImpFaALpEmptysetSubseteqARp{1,18} \\
        1,2,3& (20) & \bot      & \rBI{2,19} \\
        1,2 & (21) & \neg(\forall n\in A\exists m\in A(m<n))      & \rCI{3,20} \\
        1,2 & (22) & \exists n\in A(n=\min(A))      & \text{Lemma (2)}(21) \\
	\end{array}
	\]
\end{proof}

\chapter{Multiplikation von natürlichen Zahlen}

\begin{definition}[Multiplikation]
    Die Multiplikation von zwei natürlichen Zahlen \( a \) und \( b \) ist eine binäre Operation \( \cdot: \mathbb{N} \times \mathbb{N} \to \mathbb{N} \), die rekursiv wie folgt definiert wird:
    
    \begin{itemize}
        \item \textbf{Basisfall}: Für jede natürliche Zahl \( a \) gilt:
        \[
        a \cdot 0 := 0.
        \]
        
        \item \textbf{Rekursionsschritt}: Für jede natürliche Zahl \( b \) gilt:
        \[
        a \cdot (b+1) := (a \cdot b) + a.
        \]
    \end{itemize}
\end{definition}
\begin{remark}
In der Mathematik ist es üblich, das Multiplikationszeichen \(\cdot\) zwischen zwei Operanden wegzulassen, wenn keine Verwechslung mit anderen Operationen möglich ist. Daher kann \( a \cdot b \) auch als \( ab \) geschrieben werden.

Zudem gilt in der Arithmetik allgemein die Regel, dass Multiplikationen vor Additionen ausgeführt werden (Punktrechnung vor Strichrechnung). Diese Konvention ermöglicht es uns, bestimmte Klammern wegzulassen, ohne die Bedeutung eines Ausdrucks zu verändern.

Zum Beispiel bedeutet der Ausdruck \( ab + a \) nach den Regeln der Arithmetik, dass zunächst das Produkt \( ab \) berechnet und anschließend \( a \) addiert wird. Es ist daher nicht erforderlich, den Ausdruck als \( (a \cdot b) + a \) zu schreiben, da die Reihenfolge der Operationen durch die Regel der Punktrechnung vor Strichrechnung bereits eindeutig festgelegt ist.

In der Definition der Multiplikation von natürlichen Zahlen nutzen wir diese Regel: Der Ausdruck \( a \cdot (b+1) := a \cdot b + a \) bedeutet, dass zuerst \( a \cdot b \) berechnet wird und das Ergebnis dann um \( a \) erhöht wird. Die Klammer um \( b+1 \) stellt sicher, dass \( b \) zuerst um 1 erhöht wird, bevor das Ergebnis mit \( a \) multipliziert wird.
\end{remark}

\paragraph{Beweisregeln für die Multiplikation}
\label{rule:rMultI} 
Basierend auf der rekursiven Definition der Multiplikation können wir die folgenden Regeln für die Multiplikation formulieren:

\[
\begin{array}{llll}
	i & (1) & a \in \mathbb{N} & ... \\
	j & (2) & b \in \mathbb{N} & ... \\                                    
        i & (3) & 0=a \cdot 0  & \rMultI{1} \\
	i,j & (4) & a \cdot (b+1) = a \cdot b + a & \rMultI{1,2} \\
         & (5) & \cdot:\mathbb{N}\times\mathbb{N}\rightarrow\mathbb{N} & \rMultI{} \\
\end{array}
\]
\(i\) und \(j\) sind dabei Listen von Annahmen.

\label{aInNaturalImpZeroEqualsZeroMulta}
\begin{theorem}[\(a\in\mathbb{N}\vdash 0=0\cdot a\)]
\end{theorem}
\begin{proof}
        \[
	\begin{array}{llll}
            1       &  (1) & a\in\mathbb{N} & \rA \\
                    &  (2) & 0\in\mathbb{N} & \zeroIsNaturalNumber{} \\
                    &  (3) & 0=0\cdot 0 & \rMultI{2} \\
            4       &  (4) & n\in\mathbb{N} & \rA \\
            5       &  (5) & 0=0\cdot n & \rA \\
            5       &  (6) & 0\cdot (n+1)=0\cdot n+0 & \rMultI{5} \\
            5       &  (7) & 0\cdot (n+1)=0+0 & \rIE{5,6} \\
            5       &  (8) & 0=0+0 & \rAddI{2} \\
            5       &  (9) & 0\cdot (n+1)=0 & \rIE{8,7} \\
            5       &  (10) & 0=0\cdot (n+1) & \aIdbImpbIda{9} \\
                    &  (11) & \forall n\in\mathbb{N}(0=0\cdot n) & \rInductionN{3,4,5,10} \\
                    &  (12) & a\in\mathbb{N}\rightarrow (0=0\cdot a) & \rSetUEb{11} \\
            1       &  (13) & 0=0\cdot a & \rRE{1,12} \\
	\end{array}
	\]
\end{proof}

\label{aInNaturalImpaEqualsaMultOne}
\begin{theorem}[\(a\in\mathbb{N}\vdash a=a\cdot 1\) (Neutrales Element)]
\end{theorem}
\begin{proof}
        \[
	\begin{array}{llll}
            1       &  (1) & a\in\mathbb{N} & \rA \\
                    &  (2) & 0\in\mathbb{N} & \zeroIsNaturalNumber{} \\
                    &  (3) & 1\in\mathbb{N} & \oneIsNaturalNumber{} \\
                    &  (4) & 1=0+1 & \aInNaturalImpaEqualsZeroPlusa{3} \\
            1       &  (5) & a\cdot (0+1) = a\cdot 0 + a & \rMultI{1,2} \\
            1       &  (6) & a\cdot 1 = a\cdot 0 + a & \rIE{5,6} \\
            1       &  (7) & 0=a\cdot 0 & \rMultI{1} \\
            1       &  (8) & a\cdot 1=0+a & \rIE{7,6} \\
            1       &  (9) & a=0+a & \aInNaturalImpaEqualsZeroPlusa{1} \\
            1       &  (10) & a\cdot 1=a & \rIE{9,8} \\
            1       &  (11) & a=a\cdot 1 & \aIdbImpbIda{10} \\
	\end{array}
	\]
\end{proof}

\label{aInNaturalImpaEqualsOneMulta}
\begin{theorem}[\(a\in\mathbb{N}\vdash a=1\cdot a\) (Neutrales Element)]
\end{theorem}
\begin{proof}
        \[
	\begin{array}{llll}
            1       &  (1) & a\in\mathbb{N} & \rA \\
                    &  (2) & 1\in\mathbb{N} & \oneIsNaturalNumber{} \\            
                    &  (3) & 0=1\cdot 0 & \rMultI{2} \\
            4       &  (4) & n\in\mathbb{N} & \rA \\
            5       &  (5) & n = 1\cdot n & \rA \\
            5       &  (6) & 1\cdot (n+1)=1\cdot n+1 & \rMultI{2,4} \\
            5       &  (7) & 1\cdot (n+1)=n+1 & \rIE{5,6} \\
            5       &  (8) & n+1=1\cdot (n+1) & \aIdbImpbIda{7} \\
                    &  (9) & \forall n\in\mathbb{N}(n=1\cdot n) & \rInductionN{3,4,5,8} \\
                    &  (10) & a\in\mathbb{N}\rightarrow a=1\cdot a & \rSetUEb{9} \\
            1       &  (11) & a=1\cdot a & \rRE{10} \\                    
	\end{array}
	\]
\end{proof}

\label{aInNaturalImpaMultOneEqualsOneMultaEqualsa}
\begin{theorem}[\(a\in\mathbb{N}\vdash a\cdot 1=1\cdot a = a\) (Neutrales Element)]
\end{theorem}
\begin{proof}
        \[
	\begin{array}{llll}
            1   &  (1) & a\in\mathbb{N} & \rA \\
            1   &  (2) & a=a\cdot 1 & \aInNaturalImpaEqualsaMultOne{1} \\
            1   &  (3) & a=1\cdot a & \aInNaturalImpaEqualsOneMulta{1} \\
            1   &  (4) & a\cdot 1=1\cdot a & \rIE{2,3} \\
            1   &  (5) & 1\cdot a=a & \aIdbImpbIda{3} \\
            1   &  (6) & a\cdot 1=1\cdot a\land 1\cdot a=a & \rAI{4,5} \\
            1   &  (7) & a\cdot 1=1\cdot a = a & \rIIb{6} \\
	\end{array}
	\]
\end{proof}

\label{aInNaturalwbInNaturalImpaMultbInNatural}
\begin{theorem}[\(a\in\mathbb{N},b\in\mathbb{N}\vdash a\cdot b\in\mathbb{N}\)]
\end{theorem}
\begin{proof}
        \[
	\begin{array}{llll}
            1         &  (1) & a\in\mathbb{N} & \rA \\
            2         &  (2) & b\in\mathbb{N} & \rA \\
                      &  (3) & 0\in\mathbb{N} & \zeroIsNaturalNumber{} \\            
            1         &  (4) & 0=a\cdot 0 & \rMultI{1} \\
            1         &  (5) & a\cdot 0\in\mathbb{N} & \rIE{4,3} \\
            6         &  (6) & n\in\mathbb{N} & \rA \\
            7         &  (7) & a\cdot n\in\mathbb{N} & \rA \\
            1,6       &  (8) & a\cdot (n+1) = a\cdot n+a & \rMultI{1,6} \\
            1,6,7     &  (9) & a\cdot n+a\in\mathbb{N} & \aInNaturalwbInNaturalImpaPlusbInNatural{7,1} \\
            1,6,7     &  (10) & a\cdot (n+1)\in\mathbb{N} & \rIE{8,9} \\
            1     &  (11) & \forall n\in\mathbb{N}(a\cdot n\in\mathbb{N} & \rInductionN{5,6,7,10} \\
            1     &  (12) & b\in\mathbb{N}\rightarrow a\cdot b\in\mathbb{N} & \rSetUEb{11} \\
            1,2     &  (13) & a\cdot b\in\mathbb{N} & \rRE{2,12} \\            
	\end{array}
	\]
\end{proof}


\label{aInNaturalwbInNaturalwcInNaturalImpaLpbPluscRpEqualsabPlusac}
\begin{theorem}[\(a\in\mathbb{N},b\in\mathbb{N},c\in\mathbb{N}\vdash a(b+c)=ab+ac\) (Linksdistributivität)]
\end{theorem}
\begin{proof}
        \[
	\begin{array}{llll}
            1           &  (1) & a\in\mathbb{N} & \rA \\
            2           &  (2) & b\in\mathbb{N} & \rA \\
            3           &  (3) & c\in\mathbb{N} & \rA \\
            2           &  (4) & b=b+0 & \rAddI{2} \\
                        &  (5) & ab=ab & \rII{} \\
            2           &  (6) & a(b+0)=ab & \rIE{4,5} \\
            1           &  (7) & 0=a\cdot 0 & \rMultI{1} \\
            1,2         &  (8) & ab\in\mathbb{N} & \aInNaturalwbInNaturalImpaMultbInNatural{1,2} \\
            1,2         &  (9) & ab=ab+0 & \rAddI{8} \\
            1,2         &  (10) & ab=ab+a\cdot 0 & \rIE{7,9} \\
            1,2         &  (11) & a(b+0)=ab+a\cdot 0 & \rIE{4,10} \\
            12          &  (12) & n\in\mathbb{N} & \rA \\
            13          &  (13) & a(b+n)=ab+an & \rA \\
                        &  (14) & 1\in\mathbb{N} & \oneIsNaturalNumber{} \\
            2,12        &  (15) & b+(n+1)=(b+n)+1 & \aInNaturalwbInNaturalwcInNaturalImpaPlusLpbPluscRpEqualsLpaPlusbRpPlusc{1,2,12} \\
                        &  (16) & a(b+(n+1))=a(b+(n+1)) & \rII{} \\
            2,12        &  (17) & a(b+(n+1))=a((b+n)+1) & \rIE{15,16} \\
            2,12        &  (18) & b+n\in\mathbb{N} & \aInNaturalwbInNaturalImpaPlusbInNatural{2,12} \\
            1,2,12      &  (19) & a((b+n)+1)=a(b+n)+a & \rMultI{1,18} \\
            1,2,12,13   &  (20) & a((b+n)+1)=ab+an+a & \rIE{12,19} \\
            1,12        &  (21) & a(n+1)=an+a & \rMultI{1,12} \\
            1,2,12,13   &  (22) & a((b+n)+1)=ab+a(n+1) & \rIE{21,20} \\
            1,2,12,13   &  (24) & a(b+(n+1))=ab+a(n+1) & \rIE{17,22} \\
            1,2         &  (25) & \forall n\in\mathbb{N}(a(b+n)=ab+an) & \rInductionN{11,12,13,24} \\
            1,2         &  (26) & c\in\mathbb{N}\rightarrow a(b+c)=ab+ac & \rSetUEb{25} \\
            1,2,3       &  (27) & a(b+c)=ab+ac & \rRE{3,26} \\
        \end{array}
	\]
\end{proof}

\label{aInNaturalwbInNaturalwcInNaturalImpLpaPlusbRpcEqualsacPlusbc}
\begin{theorem}[\(a\in\mathbb{N},b\in\mathbb{N},c\in\mathbb{N}\vdash (a+b)c=ac+bc\) (Rechtsdistributivität)]
\end{theorem}
\begin{proof}
        \[
	\begin{array}{lll p{5cm}}
            1           &  (1) & a\in\mathbb{N} & \rA \\
            2           &  (2) & b\in\mathbb{N} & \rA \\
            3           &  (3) & c\in\mathbb{N} & \rA \\
            1,2         &  (4) & a+b\in\mathbb{N} & \aInNaturalwbInNaturalImpaPlusbInNatural{1,2} \\
            1,2         &  (5) & 0=(a+b)\cdot 0 & \rMultI{4} \\
            1           &  (6) & 0=a\cdot 0 & \rMultI{1} \\
            2           &  (7) & 0=b\cdot 0 & \rMultI{2} \\
                        &  (8) & 0\in\mathbb{N} & \zeroIsNaturalNumber{} \\
                        &  (9) & 0=0+0 & \rAddI{8} \\
            1           &  (10) & 0=a\cdot 0+0 & \rIE{6,9} \\
            1,2         &  (11) & 0=a\cdot 0+b\cdot 0 & \rIE{7,9} \\
            1,2         &  (12) & (a+b)\cdot 0=a\cdot 0+b\cdot 0 & \rIE{5,11} \\
            13          &  (13) & n\in\mathbb{N} & \rA \\
            14          &  (14) & (a+b)n=an+bn & \rA \\
            1,2,13      &  (15) & (a+b)(n+1)=(a+b)n+(a+b) & \rMultI{4,13} \\
            1,2,13,14   &  (16) & (a+b)(n+1)=(an+bn)+(a+b) & \rIE{14,15} \\            
            1,13        &  (17) & an\in\mathbb{N} & \aInNaturalwbInNaturalImpaMultbInNatural{1,13} \\
            2,13        &  (18) & bn\in\mathbb{N} & \aInNaturalwbInNaturalImpaMultbInNatural{2,13} \\ 1,2,13      &  (19) & (an+bn)+(a+b)=(an+a)+(bn+b) & \aInNaturalwbInNaturalwcInNaturalwdInNaturalImpLpaPlusbRpPlusLpcPlusdRpEqualsLpaPluscRpPlusLpbPlusdRp{17,18,1,2} \\  
            1,13        &  (20) & a(n+1)=(an+a) & \rMultI{1,13}  \\  
            2,13        &  (21) & b(n+1)=(bn+b) & \rMultI{2,13}  \\  
            1,2,13      &  (22) & (an+bn)+(a+b) = a(n+1)+(bn+b) & \rIE{20,19}  \\  
            1,2,13      &  (23) & (an+bn)+(a+b) = a(n+1)+b(n+1) & \rIE{21,22}  \\  
            1,2,13,14   &  (24) & (a+b)(n+1) = a(n+1)+b(n+1) & \rIE{23,16}  \\
            1,2         &  (25) & \forall n\in\mathbb{N}((a+b)n = an+bn) & \rInductionN{12,13,14,24}\\
            1,2         &  (26) & c\in\mathbb{N}\rightarrow (a+b)c = ac+bc & \rSetUEb{25}\\
            1,2,3       &  (27) & (a+b)c = ac+bc & \rRE{3,26}\\
        \end{array}
	\]
\end{proof}

\label{aInNaturalwbInNaturalImpLpaPlusOneRpbEqualsabPlusb}
\begin{theorem}[\(a\in\mathbb{N},b\in\mathbb{N}\vdash (a+1)b=ab+b\)]
\end{theorem}
\begin{proof}
        \[
	\begin{array}{lll p{5cm}}
            1   &  (1) & a\in\mathbb{N} & \rA \\
            2   &  (2) & b\in\mathbb{N} & \rA \\
            2   &  (3) & 1\in\mathbb{N} & \oneIsNaturalNumber{} \\
            2   &  (4) & b=1\cdot b & \aInNaturalImpaEqualsOneMulta{2}  \\
            1,2 &  (5) & (a+1)b=ab+1\cdot b & \aInNaturalwbInNaturalwcInNaturalImpLpaPlusbRpcEqualsacPlusbc{1,2,4}  \\
            1,2 &  (6) & (a+1)b=ab+b & \rIE{4,5}  \\
        \end{array}
	\]
\end{proof}

\label{aInNaturalwbInNaturalwcInNaturalImpaLpbcRpEqualsLpabRpc}
\begin{theorem}[\(a\in\mathbb{N},b\in\mathbb{N},c\in\mathbb{N}\vdash a(bc)=(ab)c\) (Assoziativität)]
\end{theorem}
\begin{proof}
        \[
	\begin{array}{llll}
            1           &  (1) & a\in\mathbb{N} & \rA \\
            2           &  (2) & b\in\mathbb{N} & \rA \\
            3           &  (3) & c\in\mathbb{N} & \rA \\
            2           &  (4) & 0=b\cdot 0 & \rMultI{2} \\
            1           &  (5) & 0=a\cdot 0 & \rMultI{1} \\
                        &  (6) & a(b\cdot 0)=a(b\cdot 0) & \rII{} \\
            2           &  (7) & a(b\cdot 0)=a\cdot 0 & \rIE{4,6} \\
            1,2         &  (8) & a(b\cdot 0)= 0 & \rIE{5,7} \\
            1,2         &  (9) & ab\in\mathbb{N} & \aInNaturalwbInNaturalImpaMultbInNatural{1,2} \\
            1,2         &  (10) & 0=(ab)\cdot 0 & \rMultI{9} \\
            1,2         &  (11) & a(b\cdot 0)=(ab)\cdot 0 & \rIE{8,10} \\
            12          &  (12) & n\in\mathbb{N} & \rA \\
            13          &  (13) & a(bn)=(ab)n & \rA \\
            1,2         &  (14) & (ab)(n+1)=(ab)n+ab & \rMultI{9,12} \\
            1,2,13      &  (15) & (ab)(n+1)=a(bn)+ab & \rIE{13,14} \\
            2,12        &  (16) & bn\in\mathbb{N} & \aInNaturalwbInNaturalImpaMultbInNatural{2,12}\\
            1,2,12,13   &  (17) & a(bn)+ab=a(bn+b) & \aInNaturalwbInNaturalwcInNaturalImpaLpbPluscRpEqualsabPlusac{1,16,2} \\
            2,12        &  (18) & b(n+1)=bn+b & \rMultI{2,12} \\
            1,2,12,13   &  (19) & a(bn)+ab=a(b(n+1)) & \rIE{18,17} \\
            1,2,12,13   &  (20) & (ab)(n+1)=a(b(n+1)) & \rIE{19,15} \\
            1,2         &  (21) & \forall n\in\mathbb{N}((ab)n=a(bn) & \rInductionN{11,12,13,20} \\
            1,2         &  (22) & c\in\mathbb{N}\rightarrow(ab)c=a(bc) & \rSetUEb{21} \\
            1,2,3       &  (23) & (ab)c=a(bc) & \rRE{3,22} \\
    \end{array}
	\]
\end{proof}

\label{aInNaturalwbInNaturalImpabEqualsba}
\begin{theorem}[\(a\in\mathbb{N},b\in\mathbb{N}\vdash ab=ba\) (Kommutativität)]
\end{theorem}
\begin{proof}
        \[
	\begin{array}{llll}
            1       &  (1) & a\in\mathbb{N} & \rA \\
            2       &  (2) & b\in\mathbb{N} & \rA \\
            1       &  (3) & 0=a\cdot 0 & \rMultI{1} \\
            1       &  (4) & 0=0\cdot a & \aInNaturalImpZeroEqualsZeroMulta{1} \\
            1       &  (5) & a\cdot 0=0\cdot a & \rIE{3,4} \\
            6       &  (6) & n\in\mathbb{N} & \rA \\
            7       &  (7) & an=na & \rA \\
            1,6     &  (8) & a(n+1)=an+a & \rMultI{1,6} \\
            1,6,7   &  (9) & a(n+1)=na+a & \rIE{7,8} \\
            1,6     &  (10) & (n+1)a=na+a &  \aInNaturalwbInNaturalImpLpaPlusOneRpbEqualsabPlusb{6,1} \\
            1,6,7   &  (11) & a(n+1)=(n+1)a & \rIE{10,9} \\
            1       &  (12) & \forall n\in\mathbb{N}(an=na) & \rInductionN{5,6,7,11} \\
            1       &  (13) & b\in\mathbb{N}\rightarrow an=na & \rSetUEb{12} \\
            1,2     &  (14) & ab=ba & \rRE{2,13} \\
    \end{array}
	\]
\end{proof}

\label{ImpLpNaturalwMultwOneRpInMonoid}
\begin{theorem}[\(\vdash (\mathbb{N},\cdot,1) \text{ ist ein Monoid}\)]
\end{theorem}
\begin{proof}
        \[
	\begin{array}{llll}
                &  (1) & \cdot:\mathbb{N}\times\mathbb{N}\rightarrow\mathbb{N} & \rMultI{} \\
                &  (2) & \forall a,b,c\in\mathbb{N}(a(bc)=(ab)c) & \aInNaturalwbInNaturalwcInNaturalImpaLpbcRpEqualsLpabRpc{} \\
                &  (3) & \forall a\in\mathbb{N}(a\cdot 1=1\cdot a=a) & \aInNaturalImpaMultOneEqualsOneMultaEqualsa{} \\
                &  (4) & (\mathbb{N},+,0) \text{ ist ein Monoid.} & \rMonoidI{1,2,3} \\
	\end{array}
	\]
\end{proof}

\label{ImpLpNaturalwMultwOneRpInAbelMonoid}
\begin{theorem}[\(\vdash (\mathbb{N},\cdot,1) \text{ ist ein abelscher Monoid}\)]
\end{theorem}
\begin{proof}
        \[
	\begin{array}{llll}
                &  (1) & (\mathbb{N},\cdot,1) \text{ ist ein Monoid.} & \ImpLpNaturalwMultwOneRpInMonoid{} \\
                &  (2) & \forall a\in\mathbb{N}\forall b\in\mathbb{N}(ab=ba) & \aInNaturalwbInNaturalImpabEqualsba{} \\
                &  (3) & (\mathbb{N},\cdot,1) \text{ ist ein abelscher Monoid.} & \rAbelianMonoidI{1,2} \\
	\end{array}
	\]
\end{proof}

\label{ImpLpNaturalwMultwOneRpInAbelSemiRing}
\begin{theorem}[\(\vdash (\mathbb{N},+,\cdot) \text{ ist ein abelscher Halbring}\)]
\end{theorem}
\begin{proof}
        \[
	\begin{array}{llll}
                &  (1) & (\mathbb{N},\cdot,1) \text{ ist ein abelscher Monoid.} & \ImpLpNaturalwMultwOneRpInAbelMonoid{} \\
                &  (2) & (\mathbb{N},+,0) \text{ ist ein abelscher Monoid.} & \ImpLpNaturalwPluswZeroRpInAbelMonoid{} \\
                &  (3) & \forall a,b,c\in\mathbb{N}((a+b)c=ac+bc) & \aInNaturalwbInNaturalwcInNaturalImpLpaPlusbRpcEqualsacPlusbc{} \\
                &  (4) & \forall a,b,c\in\mathbb{N}(a(b+c)=ab+ac) & \aInNaturalwbInNaturalwcInNaturalImpaLpbPluscRpEqualsabPlusac{} \\
                &  (5) & (\mathbb{N},+,\cdot) \text{ ist ein abelscher Halbring.} & \rAbelianSemiringI{1,2,4,3} \\
	\end{array}
	\]
\end{proof}

\label{aInNaturalImpZeroEqualsaMultZero}
\begin{theorem}[\(a\in\mathbb{N}\vdash 0=a\cdot 0\)]
\end{theorem}
\begin{proof}
        \[
	\begin{array}{llclll}
                &  (1) & 0&=&0\cdot a & \aInNaturalImpZeroEqualsZeroMulta{} \\
                &  (2) &  &=&a\cdot 0 & \rCommutativeMonoid{} \\
	\end{array}
	\]
\end{proof}

\label{aInNaturalwaEqualsZeroImpabEqualsZero}
\begin{theorem}[\(a\in\mathbb{N},a=0\vdash ab=0\)]
\end{theorem}
\begin{proof}
    Sei \(a\in\mathbb{N}\), dann gilt:
        \[
	\begin{array}{llll}
              1  &  (1) & a=0 & \rA \\
                 &  (2) & 0\cdot b=0 & \aInNaturalImpZeroEqualsZeroMulta{} \\
                 &  (3) & ab=0 & \rIE{1,2} \\
	\end{array}
	\]
\end{proof}

\label{aInNaturalwabNotEqualsZeroImpaNotEqualsZero}
\begin{theorem}[\(a\in\mathbb{N},ab\neq 0\vdash a\neq 0\)]
\end{theorem}
\begin{proof}
    Sei \(a\in\mathbb{N}\), dann gilt:
        \[
	\begin{array}{llll}
              1  &  (1) & ab\neq 0 & \rA \\
                 &  (2) & a=0\rightarrow ab=0 & \aInNaturalwaEqualsZeroImpabEqualsZero{} \\
              1  &  (3) & a\neq 0 & \PToQwnQImpnP{2,1} \\
	\end{array}
	\]
\end{proof}

\label{aInNaturalwbEqualsZeroImpabEqualsZero}
\begin{theorem}[\(a\in\mathbb{N},b=0\vdash ab=0\)]
\end{theorem}
\begin{proof}
    Sei \(a\in\mathbb{N}\), dann gilt:
        \[
	\begin{array}{llll}
              1  &  (1) & b=0 & \rA \\
                 &  (2) & a\cdot 0=0 & \aInNaturalImpZeroEqualsaMultZero{} \\
              1  &  (3) & ab=0 & \rIE{1,2} \\
	\end{array}
	\]
\end{proof}

\label{aInNaturalwabNotEqualsZeroImpbNotEqualsZero}
\begin{theorem}[\(a\in\mathbb{N},ab\neq 0\vdash b\neq 0\)]
\end{theorem}
\begin{proof}
    Sei \(a\in\mathbb{N}\), dann gilt:
        \[
	\begin{array}{llll}
              1  &  (1) & ab\neq 0 & \rA \\
                 &  (2) & b=0\rightarrow ab=0 & \aInNaturalwbEqualsZeroImpabEqualsZero{} \\
              1  &  (3) & b\neq 0 & \PToQwnQImpnP{2,1} \\
	\end{array}
	\]
\end{proof}


\label{aInNaturalwbInNaturalwcInNaturalLpaEqualsbImpacEqualsbcRp}
\begin{theorem}[\(a,b,c\in\mathbb{N},a=b\vdash ac=bc\)]
\end{theorem}
\begin{proof}
Seien \(a,b,c\in\mathbb{N}\), dann gilt:
        \[
	\begin{array}{llll}
            1       &  (1)  & a=b & \rA \\
                    &  (2)  & ac=ac & \rII{} \\
            1       &  (3)  & ac=bc & \rIE{1,2} \\       
	\end{array}
        \]
\end{proof}

\label{awbwcInNaturalwacNotEqualsbcImpaNotEqualsb}
\begin{theorem}[\(a,b,c\in\mathbb{N},ac\neq bc\vdash a\neq b\)]
\end{theorem}
\begin{proof}
Seien \(a,b,c\in\mathbb{N}\), dann gilt:
        \[
	\begin{array}{llll}
            1       &  (1)  & ac\neq bc & \rA \\
                    &  (2)  & a=b\rightarrow ac=bc & \aInNaturalwbInNaturalwcInNaturalLpaEqualsbImpacEqualsbcRp{} \\
            1       &  (3)  & a\neq b & \PToQwnQImpnP{2,1} \\       
	\end{array}
        \]
\end{proof}

\label{awbwcInNaturalwaEqualsbImpcaEqualscb}
\begin{theorem}[\(a,b,c\in\mathbb{N},a=b\vdash ca=cb\)]
\end{theorem}
\begin{proof}
Seien \(a,b,c\in\mathbb{N}\), dann gilt:
        \[
	\begin{array}{llll}
            1       &  (1)  & a=b & \rA \\
                    &  (2)  & ca=ca & \rII{} \\
            1       &  (3)  & ca=cb & \rIE{1,2} \\       
	\end{array}
        \]
\end{proof}

\label{awbwcInNaturalwcaNotEqualscbImpaNotEqualsb}
\begin{theorem}[\(a,b,c\in\mathbb{N},ca\neq cb\vdash a\neq b\)]
\end{theorem}
\begin{proof}
Seien \(a,b,c\in\mathbb{N}\), dann gilt:
        \[
	\begin{array}{llll}
            1       &  (1)  & ca\neq cb & \rA \\
                    &  (2)  & a=b\rightarrow ca=cb & \awbwcInNaturalwaEqualsbImpcaEqualscb{} \\
            1       &  (3)  & a\neq b & \PToQwnQImpnP{2,1} \\       
	\end{array}
        \]
\end{proof}


\section{Eigenschaften der Multiplikation in Bezug auf Relationen}

\label{awbwcInNaturalImpaLeqbImpacLeqbc}
\begin{theorem}[\(a,b,c\in\mathbb{N},a\leq b\vdash ac\leq bc\)]
\end{theorem}
\begin{proof}
Seien \(a,b,c\in\mathbb{N}\). \(\ImpLpNaturalwMultwOneRpInAbelSemiRing{}\) und daher gilt:
       \[
	\begin{array}{lllcll}
            1       &  (1)  & \multicolumn{3}{l}{a\leq b} & \rA \\
            1       &  (2)  & \multicolumn{3}{l}{a+(b-a)=b} & \minusI{1} \\
            1       &  (3)  & ac+(b-a)c&=&(a+(b-a))c & \rRightDistributiveAbelianSemigroup{} \\       
            1       &  (4)  &  &=&bc & \aInNaturalwbInNaturalwcInNaturalLpaEqualsbImpacEqualsbcRp{2} \\ 
            1       &  (5)  & \multicolumn{3}{l}{ac+(b-a)c=bc} & \rTransitivityOrdRI{3,4} \\ 
            1       &  (6)  & \multicolumn{3}{l}{ac\leq bc} & \rLeqNI{5} \\ 
	\end{array}
        \]

\end{proof}

\label{awbwcInNaturalImpaLeqbImpcaLeqcb}
\begin{theorem}[\(a,b,c\in\mathbb{N}\vdash a\leq b\vdash ca\leq cb\)]
\end{theorem}
\begin{proof}
Seien \(a,b,c\in\mathbb{N}\). \(\ImpLpNaturalwMultwOneRpInAbelSemiRing{}\) und daher gilt:
       \[
	\begin{array}{lllcll}
            1       &  (1)  & \multicolumn{3}{l}{a\leq b} & \rA \\
                    &  (2)  & ca&=&ac & \rCommutativeMonoid{} \\       
            1       &  (3)  &  &\leq &bc & \awbwcInNaturalImpaLeqbImpacLeqbc{1} \\ 
                    &  (4)  &  &\leq &cb & \rCommutativeMonoid{} \\ 
                    &  (5)  & \multicolumn{3}{l}{ca\leq cb} & \rTransitivityEqRI{2,5} \\ 
	\end{array}
        \]
\end{proof}

\label{awbInNaturalwbNotEqualsZeroImpaLeqab}
\begin{theorem}[\(a,b\in\mathbb{N}, b\neq 0\vdash a\leq ab\)]
\end{theorem}
\begin{proof}
        Seien \(a,b\in\mathbb{N}\). \(\LeqIsTotalOrderOnNaturalNumbers{}\) und daher gilt
        \[
	\begin{array}{llclll}
        1   &  (1)  & \multicolumn{3}{l}{b\neq 0} & \rA \\
        1   &  (2)  & \multicolumn{3}{l}{1\leq b} & \aInNaturalwaNotEqualsZerowOneLeqa{1} \\
            &  (3)  & a&=&1\cdot a & \rNeutralElementMonoid{} \\
        1   &  (4)  & 1\cdot a&\leq & ba & \awbwcInNaturalImpaLeqbImpacLeqbc{2} \\
        1   &  (5)  & a&\leq & ba & \rTransitivityOrdRI{2,4} \\

	\end{array}
	\]
\end{proof}

\label{awbInNaturalwbNotEqualsZeroImpaLeqba}
\begin{theorem}[\(a,b\in\mathbb{N}, b\neq 0\vdash a\leq ba\)]
\end{theorem}
\begin{proof}
        Seien \(a,b\in\mathbb{N}\). \(\LeqIsTotalOrderOnNaturalNumbers{}\) und daher gilt
        \[
	\begin{array}{llclll}
        1   &  (1)  & \multicolumn{3}{l}{b\neq 0} & \rA \\
        1   &  (2)  & a&\leq &ab & \awbInNaturalwbNotEqualsZeroImpaLeqab{1} \\
            &  (3)  & &=& ba & \rCommutativeMonoid{} \\
        1   &  (4)  & a&\leq & ba & \rTransitivityOrdRI{2,3} \\

	\end{array}
	\]
\end{proof}


\section{Eigenschaften der Multiplikation und Differenz}

\label{awbwcInNaturalwcLeqbImpLpbMinuscRpaEqualsbaMinusca}
\begin{theorem}[\(a,b,c\in\mathbb{N},c\leq b\vdash (b-c)a=ba-ca\)]
\end{theorem}
\begin{proof}
Seien \(a,b,c\in\mathbb{N}\). \(\ImpLpNaturalwMultwOneRpInAbelSemiRing{}\) und daher gilt:
       \[
	\begin{array}{lllcll}
            1       &  (1)  & \multicolumn{3}{l}{c\leq b} & \rA \\
            1       &  (2)  & \multicolumn{3}{l}{ca\leq ba} & \awbwcInNaturalImpaLeqbImpacLeqbc{1} \\
            1       &  (3)  & \multicolumn{3}{l}{c+(b-c)=b} & \minusI{1} \\
            1       &  (4)  & \multicolumn{3}{l}{ca+(ba-ca)=ba} & \minusI{2} \\
                    &  (5)  & ca+(b-c)a&=&(c+(b-c))a & \rRightDistributiveAbelianSemigroup{} \\
            1       &  (6)  &          &=& ba & \rIE{3,5} \\
            1       &  (7)  &          &=& ca+(ba-ca) & \rIE{4,6} \\
            1       &  (8)  & \multicolumn{3}{l}{ca+(b-c)a=ca+(ba-ca)} & \rTransitivityEqRI{5,7} \\
            1       &  (9)  & \multicolumn{3}{l}{(b-c)a=ba-ca} & \aEqualsbEqvcPlusaEqualscPlusb{8} \\
	\end{array}
        \]
\end{proof}


\label{awbwcInNaturalwcLeqbImpaLpbMinuscRpEqualsabMinusac}
\begin{theorem}[\(a,b,c\in\mathbb{N},c\leq b\vdash a(b-c)=ab-ac\)]
\end{theorem}
\begin{proof}
Seien \(a,b,c\in\mathbb{N}\). \(\ImpLpNaturalwMultwOneRpInAbelSemiRing{}\) und daher gilt:
       \[
	\begin{array}{lllcll}
            1       &  (1)  & \multicolumn{3}{l}{c\leq b} & \rA \\
            1       &  (2)  & \multicolumn{3}{l}{ac\leq ab} & \awbwcInNaturalImpaLeqbImpacLeqbc{1} \\
            1       &  (3)  & \multicolumn{3}{l}{c+(b-c)=b} & \minusI{1} \\
            1       &  (4)  & \multicolumn{3}{l}{ac+(ab-ac)=ab} & \minusI{2} \\
                    &  (5)  & ac+a(b-c)&=&a(c+(b-c)) & \rRightDistributiveAbelianSemigroup{} \\
            1       &  (6)  &          &=& ab & \rIE{3,5} \\
            1       &  (7)  &          &=& ac+(ab-ac) & \rIE{4,6} \\
            1       &  (8)  & \multicolumn{3}{l}{ac+a(b-c)=ac+(ab-ac)} & \rTransitivityEqRI{5,7} \\
            1       &  (9)  & \multicolumn{3}{l}{a(b-c)=ab-ac} & \aEqualsbEqvcPlusaEqualscPlusb{8} \\
	\end{array}
        \]
\end{proof}

\section{Weitere Eigenschaften der Multiplikation}

\label{awbInNaturalwaNotEqualsZerowabEqualsZeroImpbEqualsZero}
\begin{theorem}[\(a,b\in\mathbb{N},a\neq 0,ab=0\vdash b=0\)]
\end{theorem}
\begin{proof}
        Seien \(a,b\in\mathbb{N}\). \(\ImpLpNaturalwMultwOneRpInAbelSemiRing{}\)
        \[
	\begin{array}{llclll}
            1       &  (1)  & \multicolumn{3}{l}{a\neq 0} & \rA \\
            1       &  (2)  & \multicolumn{3}{l}{a=(a-1)+1} & \rPredecessorI{1} \\
            3       &  (3)  & 0&=&ab & \rA \\
            1,3     &  (4)  &  &=&((a-1)+1)b & \rIE{2} \\
            1,3     &  (5)  &  &=&(a-1)b+1\cdot b & \rRightDistributiveAbelianSemigroup{} \\
            1,3     &  (6)  &  &=&(a-1)b+b & \rNeutralElementMonoid{} \\
            1,3     &  (7)  & \multicolumn{3}{l}{(a-1)b+b=0} & \rTransitivityOrdRI{3,6} \\
            1,3     &  (8)  & \multicolumn{3}{l}{(a-1)b=0\land b=0} & \aInNaturalwbInNaturalwaPlusbEqualsZeroImpaEqualsZeroAndbEqualsZero{7}\\  
            1,3     &  (9)  & \multicolumn{3}{l}{b=0} & \rAEb{8}\\  
	\end{array}
	\]
\end{proof}

\label{awbInNaturalwbNotEqualsZerowabEqualsZeroImpaEqualsZero}
\begin{theorem}[\(a,b\in\mathbb{N},b\neq 0,ab=0\vdash a=0\)]
\end{theorem}
\begin{proof}
        Seien \(a,b\in\mathbb{N}\). \(\ImpLpNaturalwMultwOneRpInAbelSemiRing{}\)
        \[
	\begin{array}{llclll}
            1       &  (1)  & \multicolumn{3}{l}{b\neq 0} & \rA \\
                    &  (2)  & ba&=&ab & \rCommutativeMonoid{} \\
            3       &  (3)  &   &=&0 & \rA \\
            3       &  (4)  & \multicolumn{3}{l}{ba=0} & \rTransitivityEqRI{2,3} \\
            1,3     &  (5)  & \multicolumn{3}{l}{a\neq 0} & \awbInNaturalwaNotEqualsZerowabEqualsZeroImpbEqualsZero{1,4} \\            
	\end{array}
	\]
\end{proof}

\label{awbInNaturalwaNotEqualsZerowbNotEqualsZeroImpabNotEqualsZero}
\begin{theorem}[\(a,b\in\mathbb{N},a\neq 0,b\neq 0\vdash ab\neq 0\)]
\end{theorem}
\begin{proof}
        Seien \(a,b\in\mathbb{N}\). 
        \[
	\begin{array}{llll}
            1       &  (1)  & a\neq 0 & \rA \\
            2       &  (2)  & b\neq 0 & \rA \\
            3       &  (3)  & ab=0 & \rA \\
            2,3     &  (4)  & a=0 & \awbInNaturalwbNotEqualsZerowabEqualsZeroImpaEqualsZero{2,3} \\
            1,2,3   &  (5)  & \bot & \rBI{1,4} \\
            1,2   &  (6)  & ab\neq 0 & \rCI{3,5} \\
	\end{array}
	\]
\end{proof}

\label{awbwcInNaturalwcNotEqualsZerowacEqualsbcImpaEqualsb}
\begin{theorem}[\(a,b,c\in\mathbb{N},c\neq 0,ac=bc\vdash a=b\)]
\end{theorem}
\begin{proof}
        Seien \(a,b,c\in\mathbb{N}\). \(\LeqIsTotalOrderOnNaturalNumbers{}\) und daher gilt
        \[
	\begin{array}{llclll}
            1       &  (1)  & \multicolumn{3}{l}{ac=bc} & \rA \\
            2       &  (2)  & \multicolumn{3}{l}{c\neq 0} & \rA \\
                    &  (3)  & \multicolumn{3}{l}{a\leq b\lor b\leq a} & \rTotalityOrdRI{} \\
            4       &  (4)  & \multicolumn{3}{l}{a\leq b} & \rA \\
            1       &  (5)  & 0&=&bc-ac & \awbInNaturalLpbEqualsaEqvaMinusbEqualsZeroRp{1} \\
            1,4     &  (6)  &  &=&(b-a)c & \awbwcInNaturalwcLeqbImpLpbMinuscRpaEqualsbaMinusca{4} \\
            1,4     &  (7)  & \multicolumn{3}{l}{(b-a)c=0} & \rTransitivityEqRI{5,6} \\        
            1,2,4   &  (8)  & \multicolumn{3}{l}{b-a=0} & \awbInNaturalwbNotEqualsZerowabEqualsZeroImpaEqualsZero{2,7} \\
            1,2,4   &  (9)  & \multicolumn{3}{l}{a=b} & \awbInNaturalLpbEqualsaEqvaMinusbEqualsZeroRp{8} \\
            10      &  (10)  & \multicolumn{3}{l}{b\leq a} & \rA \\
            1       &  (11)  & 0&=&ac-bc & \awbInNaturalLpaEqualsbEqvaMinusbEqualsZeroRp{1} \\
            1,10    &  (12)  &  &=&(a-b)c & \awbwcInNaturalwcLeqbImpLpbMinuscRpaEqualsbaMinusca{10} \\
            1,10     &  (13)  & \multicolumn{3}{l}{(a-b)c=0} & \rTransitivityEqRI{11,12} \\  
            1,2,10   &  (14)  & \multicolumn{3}{l}{a-b=0} & \awbInNaturalwbNotEqualsZerowabEqualsZeroImpaEqualsZero{2,13} \\  
            1,2,10     &  (15)  & \multicolumn{3}{l}{a=b} & \awbInNaturalLpaEqualsbEqvaMinusbEqualsZeroRp{14} \\  
            1,2        &  (16)  & \multicolumn{3}{l}{a=b} & \rOE{3,4,9,10,15} \\  
	\end{array}
	\]
\end{proof}

\label{awbwcInNaturalwcNotEqualsZerowaNotEqualsbImpacNotEqualsbc}
\begin{theorem}[\(a,b,c\in\mathbb{N},c\neq 0,a\neq b\vdash ac\neq bc\)]
\end{theorem}
\begin{proof}
        Seien \(a,b,c\in\mathbb{N}\).
        \[
	\begin{array}{llclll}
            1       &  (1)  & \multicolumn{3}{l}{a\neq b} & \rA \\
            2       &  (2)  & \multicolumn{3}{l}{c\neq 0} & \rA \\
            2       &  (3)  & \multicolumn{3}{l}{ac=bc\rightarrow a=b} & \awbwcInNaturalwcNotEqualsZerowacEqualsbcImpaEqualsb{2} \\
            1,2       &  (3)  & \multicolumn{3}{l}{ac\neq bc} & \PToQwnQImpnP{3,1} \\  
	\end{array}
	\]
\end{proof}

\label{awbwcInNaturalwcNotEqualsZerowcaEqualscbImpaEqualsb}
\begin{theorem}[\(a,b,c\in\mathbb{N},c\neq 0,ca=cb\vdash a=b\)]
\end{theorem}
\begin{proof}
        Seien \(a,b,c\in\mathbb{N}\). \(\LeqIsTotalOrderOnNaturalNumbers{}\) und daher gilt
        \[
	\begin{array}{llclll}
                    &  (1)  & ac&=&ca & \rCommutativeMonoid{} \\
                2   &  (2)  & &=&cb & \rA \\
                2   &  (3)  & &=&bc & \rCommutativeMonoid{} \\
                2   &  (4)  &\multicolumn{3}{l}{ac=bc} & \rTransitivityEqRI{1,3} \\
                5   &  (5)  &\multicolumn{3}{l}{c\neq 0} & \rA \\
                2,5 &  (6)  &\multicolumn{3}{l}{a=b} & \awbwcInNaturalwcNotEqualsZerowacEqualsbcImpaEqualsb{4,5} \\
	\end{array}
	\]
\end{proof}

\label{awbwcInNaturalwcNotEqualsZerowaNotEqualsbImpcaNotEqualscb}
\begin{theorem}[\(a,b,c\in\mathbb{N},c\neq 0,a\neq b\vdash ca\neq cb\)]
\end{theorem}
\begin{proof}
        Seien \(a,b,c\in\mathbb{N}\).
        \[
	\begin{array}{llclll}
            1       &  (1)  & \multicolumn{3}{l}{a\neq b} & \rA \\
            2       &  (2)  & \multicolumn{3}{l}{c\neq 0} & \rA \\
            2       &  (3)  & \multicolumn{3}{l}{ca=cb\rightarrow a=b} & \awbwcInNaturalwcNotEqualsZerowcaEqualscbImpaEqualsb{2} \\
            1,2     &  (4)  & \multicolumn{3}{l}{ca\neq cb} & \PToQwnQImpnP{3,1} \\  
	\end{array}
	\]
\end{proof}

\subsection{Ordnungsrelationen und Multiplikation in den natürlichen Zahlen}

\label{awbwcInNaturalwcNotEqualsZerowacLeqbcImpaLeqb}
\begin{theorem}[\(a,b,c\in\mathbb{N},c\neq 0, ac\leq bc\vdash a\leq b\)]
\end{theorem}
\begin{proof}
Seien \(a,b,c\in\mathbb{N}\).
       \[
	\begin{array}{lllcll}
            1       &  (1)  & \multicolumn{3}{l}{ac\leq bc} & \rA \\
            2       &  (2)  & \multicolumn{3}{l}{c\neq 0} & \rA \\
            3       &  (3)  & \multicolumn{3}{l}{b<a} & \rA \\       
            3       &  (4)  & \multicolumn{3}{l}{b\leq a} & \InducedStrictOrderE{3} \\   
            3       &  (5)  & \multicolumn{3}{l}{b\neq a} & \InducedStrictOrderE{3} \\   
            3       &  (6)  & \multicolumn{3}{l}{bc\leq ac} & \awbwcInNaturalImpaLeqbImpacLeqbc{4} \\ 
            1,3       &  (7)  & \multicolumn{3}{l}{bc=ac} & \rAntisymmetryOrdRI{6,1} \\  
            1,2,3     &  (8)  & \multicolumn{3}{l}{b=a} & \awbwcInNaturalwcNotEqualsZerowacEqualsbcImpaEqualsb{2,7} \\  
            1,2,3     &  (9)  & \multicolumn{3}{l}{\bot} & \rBI{8,5} \\  
            1,2       &  (10)  & \multicolumn{3}{l}{\neg(b<a)} & \rBI{8,5} \\  
            1,2       &  (11)  & \multicolumn{3}{l}{a\leq b)} & \nLpaLneqbRpEqvbLeqa{10} \\   
	\end{array}
        \]
\end{proof}


\label{awbwcInNaturalwcNotEqualsZerowcaLeqcbImpaLeqb}
\begin{theorem}[\(a,b,c\in\mathbb{N},c\neq 0, ca\leq cb\vdash a\leq b\)]
\end{theorem}
\begin{proof}
Seien \(a,b,c\in\mathbb{N}\). \(\ImpLpNaturalwMultwOneRpInAbelSemiRing{}\) und daher gilt:
       \[
	\begin{array}{lllcll}
            1       &  (1)  & \multicolumn{3}{l}{c\neq 0} & \rA \\
                    &  (2)  & ac&=&ca & \rCommutativeMonoid{} \\
            3       &  (3)  & &\leq &cb & \rA \\
                    &  (4)  & &= &bc & \rCommutativeMonoid{} \\
            3       &  (5)  & ac&=&bc & \rTransitivityOrdRI{2,4} \\     
            1,3       &  (6)  & a&\leq &b & \awbwcInNaturalwcNotEqualsZerowacLeqbcImpaLeqb{1,5} \\ 
	\end{array}
        \]
\end{proof}

\label{awbwcInNaturalwcNotEqualsZerowacLneqbcEqvaLneqb}
\begin{theorem}[\(a,b,c\in\mathbb{N},c\neq 0, ac<bc\dashv\vdash a<b\)]
Seien \(a,b,c\in\mathbb{N}\) und \(c\neq 0\), dann gilt:
\[ac<bc\dashv\vdash a<b\]
\end{theorem}
\begin{proof}
Seien \(a,b,c\in\mathbb{N}\). 
       \[
	\begin{array}{lllcll}
            1       &  (1)  & \multicolumn{3}{l}{c\neq 0} & \rA \\
            2       &  (2)  & \multicolumn{3}{l}{ac<bc} & \rA \\
            2       &  (3)  & \multicolumn{3}{l}{ac\leq bc} & \InducedStrictOrderE{2} \\
            2       &  (4)  & \multicolumn{3}{l}{ac\neq bc} & \InducedStrictOrderE{2} \\
            1,2     &  (5)  & \multicolumn{3}{l}{a\leq b} & \awbwcInNaturalwcNotEqualsZerowacLeqbcImpaLeqb{1,3} \\
            2     &  (6)  & \multicolumn{3}{l}{a\neq b} & \awbwcInNaturalwacNotEqualsbcImpaNotEqualsb{4} \\
            1,2     &  (7)  & \multicolumn{3}{l}{a<b} & \InducedStrictOrderI{5,6} \\

	\end{array}
        \]
\(\dashv\):
       \[
	\begin{array}{lllcll}
            1       &  (1)  & \multicolumn{3}{l}{c\neq 0} & \rA \\
            2       &  (2)  & \multicolumn{3}{l}{a<b} & \rA \\
            2       &  (3)  & \multicolumn{3}{l}{a\leq b} & \InducedStrictOrderE{2} \\
            2       &  (4)  & \multicolumn{3}{l}{ac\leq bc} & \awbwcInNaturalImpaLeqbImpacLeqbc{3} \\
            2       &  (5)  & \multicolumn{3}{l}{a\neq b} & \InducedStrictOrderE{2} \\
            1,2       &  (6)  & \multicolumn{3}{l}{ac\neq bc} & \awbwcInNaturalwcNotEqualsZerowaNotEqualsbImpacNotEqualsbc{1,5} \\
            1,2       &  (7)  & \multicolumn{3}{l}{ac<bc} & \InducedStrictOrderI{4,6} \\
	\end{array}
        \]
\end{proof}


\label{awbwcInNaturalwcNotEqualsZerowcaLneqcbEqvaLneqb}
\begin{theorem}[\(ca<cb\dashv\vdash a<b\)]
Seien \(a,b,c\in\mathbb{N}\) und \(c\neq 0\), dann gilt:
\[ca<cb\dashv\vdash a<b\]
\end{theorem}
\begin{proof}
Seien \(a,b,c\in\mathbb{N}\).
\(\vdash\):
       \[
	\begin{array}{lllcll}
            1       &  (1)  & \multicolumn{3}{l}{c\neq 0} & \rA \\
            2       &  (2)  & \multicolumn{3}{l}{ca<cb} & \rA \\
            2       &  (3)  & \multicolumn{3}{l}{ca\leq cb} & \InducedStrictOrderE{2} \\
            2       &  (4)  & \multicolumn{3}{l}{ca\neq cb} & \InducedStrictOrderE{2} \\
            1,2     &  (5)  & \multicolumn{3}{l}{a\leq b} & \awbwcInNaturalwcNotEqualsZerowcaLeqcbImpaLeqb{1,3} \\
            2     &  (6)  & \multicolumn{3}{l}{a\neq b} & \awbwcInNaturalwcaNotEqualscbImpaNotEqualsb{4} \\
            1,2     &  (7)  & \multicolumn{3}{l}{a<b} & \InducedStrictOrderI{5,6} \\
	\end{array}
        \]
\(\dashv\):
       \[
	\begin{array}{lllcll}
            1       &  (1)  & \multicolumn{3}{l}{c\neq 0} & \rA \\
            2       &  (2)  & \multicolumn{3}{l}{a<b} & \rA \\
            2       &  (3)  & \multicolumn{3}{l}{a\leq b} & \InducedStrictOrderE{2} \\
            2       &  (4)  & \multicolumn{3}{l}{ca\leq cb} & \awbwcInNaturalImpaLeqbImpcaLeqcb{3} \\
            2       &  (5)  & \multicolumn{3}{l}{a\neq b} & \InducedStrictOrderE{2} \\
            1,2       &  (6)  & \multicolumn{3}{l}{ca\neq cb} & \awbwcInNaturalwcNotEqualsZerowaNotEqualsbImpcaNotEqualscb{1,5} \\
            1,2       &  (7)  & \multicolumn{3}{l}{ca<cb} & \InducedStrictOrderI{4,6} \\
	\end{array}
        \]
\end{proof}

\label{awbInNaturalwbGneqOneImpaLneqba}
\begin{theorem}[\(a,b\in\mathbb{N},b>1\vdash a<ba\)]
\end{theorem}
\begin{proof}
Seien \(a,b,c\in\mathbb{N}\). \(\ImpLpNaturalwMultwOneRpInAbelSemiRing{}\) und daher gilt:
       \[
	\begin{array}{lllcll}
            1       &  (1)  & \multicolumn{3}{l}{b>1} & \rA \\
            1       &  (2)  & \multicolumn{3}{l}{1<b} & \rgtE{1} \\
                    &  (3)  & a&=&1\cdot a & \rNeutralElementMonoid{} \\
            1       &  (4)  &  &<&ba & \awbwcInNaturalwcNotEqualsZerowacLneqbcEqvaLneqb{2} \\
            1       &  (5)  &  a&<&ba & \rTransitivityOrdRI{3,4} \\
	\end{array}
        \]
\end{proof}


\chapter{Division von natürlichen Zahlen}

\begin{definition}[Teilbarkeit]
Seien \( a, b \in \mathbb{N} \) mit \( b \neq 0 \). Wir sagen, \( b \) teilt \( a \) (geschrieben \( b \mid a \)), wenn:
\[
b \mid a \quad := \quad \exists k \in \mathbb{N} \ (a = b \cdot k).
\]
\end{definition}

\begin{definition}[\(\nmid\)]
Seien \( a, b \in \mathbb{N} \) mit \( b \neq 0 \). Wir sagen, \( b \) teilt \( a \) nicht (geschrieben \( b \nmid a \)), wenn:
\[
b \nmid a \quad := \quad \neg(b \mid a).
\]
\end{definition}

\paragraph{Beweisregeln für die Teilbarkeit}
\label{rule:rDivisibilityI} \label{rule:rDivisibilityE}
Basierend auf der Definition der Teilbarkeit können wir folgende Regeln formulieren:

\[
\begin{array}{llll}
    i & (1) & a \in \mathbb{N} & ... \\
    j & (2) & b \in \mathbb{N} & ... \\
    k & (3) & b \mid a & ... \\
    i,j,k & (4) & \exists k \in \mathbb{N} \ (a = b \cdot k) & \rDivisibilityE{1,2,3} \\
    i,j,k & (5) & b\neq 0 & \rDivisibilityE{1,2,3} \\
\end{array}
\]

\[
\begin{array}{llll}
    i & (1) & a \in \mathbb{N} & ... \\
    j & (2) & b \in \mathbb{N} & ... \\
    k & (3) & b \neq 0 & ... \\
    l & (4) & \exists k \in \mathbb{N} \ (a = b \cdot k) & ... \\
    i,j,k,l & (5) & b \mid a & \rDivisibilityI{1,2,3,4} \\
\end{array}
\]

\[
\begin{array}{llll}
    i & (1) & a \in \mathbb{N} & ... \\
    j & (2) & b \in \mathbb{N} & ... \\
    k & (3) & k \in \mathbb{N} & ... \\
    l & (4) & b \neq 0 & ... \\
    m & (5) & a = b \cdot k & ... \\
    i,j,k,l,m & (6) & b \mid a & \rDivisibilityI{1,2,3,4,5} \\
\end{array}
\]

\[
\begin{array}{llll}
    i & (1) & a \in \mathbb{N} & ... \\
    j & (2) & b \in \mathbb{N} & ... \\
    k & (3) & k \in \mathbb{N} & ... \\
    l & (4) & b \neq 0 & ... \\
    m & (5) & a = k\cdot b  & ... \\
    i,j,k,l,m & (6) & b \mid a & \rDivisibilityI{1,2,3,4,5} \\
\end{array}
\]

Dabei sind \( i \), \( j \), \(k\), \(l\) und \(m\) Listen von Annahmen.

\section{Eindeutigkeit der Division}

\label{ExkSubOnewkSubTwoInNaturalLpaEqualsbMultkSubOneAndaEqualsbMultkSubTwoRpImpkSubOneEqualskSubTwo}
\begin{theorem}[\(\exists k_1, k_2 \in \mathbb{N} (a = b \cdot k_1 \land a = b \cdot k_2) \vdash k_1 = k_2\) (Eindeutigkeit der Teilbarkeit)]
Seien \( a, b \in \mathbb{N} \) mit \( b \neq 0 \), und es gelte \( a = b \cdot k_1 \) sowie \( a = b \cdot k_2 \) für \( k_1, k_2 \in \mathbb{N} \). Dann folgt, dass \( k_1 = k_2 \).
\end{theorem}

\begin{proof}
Seien \( a, b \in \mathbb{N} \), dann gilt:
\[
\begin{array}{llll}
    1 & (1) & \exists k_1, k_2 \in \mathbb{N} (a = b \cdot k_1 \land a = b \cdot k_2) & \rA \\
    1  & (2) & k_1 \in \mathbb{N} \land \exists k_2 \in \mathbb{N} (a = b \cdot k_1 \land a = b \cdot k_2) & \rSetEEm{1} \\
    1 & (3) & k_1 \in \mathbb{N} & \rAEa{2} \\
    1 & (4) & \exists k_2 \in \mathbb{N} (a = b \cdot k_1 \land a = b \cdot k_2) & \rAEb{2} \\
    1 & (5) & k_2 \in \mathbb{N} \land (a = b \cdot k_1 \land a = b \cdot k_2) & \rSetEEm{4} \\
    1 & (6) & k_2 \in \mathbb{N} & \rAEa{5} \\
    1 & (7) & a = b \cdot k_1 \land a = b \cdot k_2 & \rAEb{5} \\
    1 & (8) & a = b \cdot k_1 & \rAEa{7} \\
    1 & (9) & a = b \cdot k_2 & \rAEb{7} \\
    1 & (10) & b \cdot k_1 = b \cdot k_2 & \rIE{8,9} \\
    11 & (11) & b\neq 0 & \rA \\
    1,11 & (12) & k_1 = k_2 & \awbwcInNaturalwcNotEqualsZerowacEqualsbcImpaEqualsb{11,10} \\
\end{array}
\]
\end{proof}

\section{Definition der Division}

\begin{definition}[Division (\( \frac{a}{b} \))]
Seien \( a, b \in \mathbb{N} \) mit \( b \neq 0 \). Die Division \( a \div b \) ist definiert durch:
\[
\forall a, b \in \mathbb{N} \ [b \neq 0 \land b \mid a \rightarrow (\frac{a}{b} \coloneqq \iota k \, (k \in \mathbb{N} \land a = b \cdot k))].
\]
\end{definition}

\begin{remark}
Die Division \( \frac{a}{b} \) ist nur definiert, wenn \( b \neq 0 \) und \( b \) ein Teiler von \( a \) ist. Dies stellt sicher, dass \( k \) eindeutig existiert und die Operation wohldefiniert ist.
\end{remark}

\begin{definition}[Division]
Die Division zweier natürlicher Zahlen \( a \) und \( b \) (mit \( b \neq 0 \)) ist eine partielle binäre Operation
\[
\div : \{(a, b) \in \mathbb{N} \times \mathbb{N} \mid b \neq 0 \text{ und } b \mid a\} \to \mathbb{N},
\]
definiert durch
\[
\frac{a}{b} := k, \quad \text{wobei } k \in \mathbb{N} \text{ und } a = b \cdot k.
\]
\end{definition}

\begin{remark}
In dieser Definition bedeutet \(\frac{a}{b}\), dass wir die natürliche Zahl \( k \) finden, für die \( a = b \cdot k \) gilt. Da in den natürlichen Zahlen keine Brüche oder Dezimalzahlen existieren, ist die Division nur dann definiert, wenn \( b \) ein Teiler von \( a \) ist. Die Voraussetzung \( b \neq 0 \) stellt sicher, dass die Division durch Null ausgeschlossen ist.
\end{remark}

\paragraph{Beweisregeln für die Division}
\label{rule:rDivisionI}
Basierend auf dieser Definition können wir folgende Regeln für die Division formulieren:

\[
\begin{array}{llll}
    i & (1) & a \in \mathbb{N} & ... \\
    j & (2) & b \in \mathbb{N} & ... \\
    k & (3) & b \mid a & ... \\
    i,j,k & (4) & a = b \cdot \frac{a}{b} & \rDivisionI{1,2,3} \\
    i,j,k & (5) & \frac{a}{b} \in \mathbb{N} & \rDivisionI{1,2,3} \\
    i,j,k & (6) & b\cdot \frac{a}{b} \in \mathbb{N} & \rDivisionI{1,2,3} \\
\end{array}
\]

\[
\begin{array}{llll}
    i & (1) & a \in \mathbb{N} & ... \\
    j & (2) & b \in \mathbb{N} & ... \\
    k & (3) & c \in \mathbb{N} & ... \\
    l & (4) & b\neq 0 & ... \\
    m & (5) & a=b\cdot c & ... \\
    i,j,k,l,m & (6) & c = \frac{a}{b} & \rDivisionI{1,2,3,4,5} \\
\end{array}
\]

\[
\begin{array}{llll}
    i & (1) & a \in \mathbb{N} & ... \\
    j & (2) & b \in \mathbb{N} & ... \\
    k & (3) & c \in \mathbb{N} & ... \\
    l & (4) & b\neq 0 & ... \\
    m & (5) & a=c\cdot b & ... \\
    i,j,k,l,m & (6) & c = \frac{a}{b} & \rDivisionI{1,2,3,4,5} \\
\end{array}
\]

Dabei sind \( i \), \( j \), \(k\), \(l\) und \( m \) Listen von Annahmen.

\label{aInNaturalImpOneMida}
\begin{theorem}[\(a\in\mathbb{N}\vdash 1\mid a\)]
\end{theorem}
\begin{proof}
Sei \(a\in\mathbb{N}\), dann gilt:
        \[
	\begin{array}{llll}
                &  (1)  & a=1\cdot a & \rNeutralElementMonoid{} \\
                &  (2)  & 1\neq 0 & \nInNaturalImpnPlusOneNotEqualsZero{} \\
                &  (2)  & 1\mid a & \rDivisibilityI{1} \\
    \end{array}
	\]
\end{proof}

\label{aInNaturalImpLpaRpDurchLpOneRpEqualsa}
\begin{theorem}[\(a\in\mathbb{N}\vdash \frac{a}{1}=a\)]
\end{theorem}
\begin{proof}
Sei \(a\in\mathbb{N}\), dann gilt:
        \[
	\begin{array}{llll}
                &  (1)  & a=1\cdot a & \rNeutralElementMonoid{} \\
                &  (2)  & 1\mid a& \aInNaturalImpOneMida{} \\
                &  (2)  & \frac{a}{1}=a & \rDivisionI{2,1} \\
    \end{array}
	\]
\end{proof}


\label{aInNaturalwaNotEqualsZeroImpLpZeroRpDurchLpaRpEqualsZero}
\begin{theorem}[\(a\in\mathbb{N},a\neq 0\vdash 0=\frac{0}{a}\)]
\end{theorem}
\begin{proof}
Im Beweis nutzen wir das Theorem \(\ImpLpNaturalwMultwOneRpInAbelSemiRing{}\). 
Sei \(a\in\mathbb{N}\), dann gilt:
        \[
	\begin{array}{llll}
        1       &  (1)  & a\neq 0 & \rA \\
                &  (2)  & 0=a\cdot 0 & \rNeutralElementMonoid{} \\
        1       &  (3)  & 0=\frac{0}{a} & \rDivisionI{1} \\
    \end{array}
	\]
\end{proof}


\label{awbInNaturalwbNotEqualsZerowaMidbImpLpbRpDurchLpaRpNotEqualsZero}
\begin{theorem}[\(a,b \in\mathbb{N}, b\neq 0, a\mid b\vdash \frac{b}{a}\neq 0\) ]
\end{theorem}
\begin{proof}
Seien \(a,b\in\mathbb{N}\).
Im Beweis nutzen wir das Theorem \(\ImpLpNaturalwMultwOneRpInAbelSemiRing{}\). 
    \[
    \begin{array}{llclll}
    1       &  (1)  & \multicolumn{3}{l}{a\mid b} & \rA \\
    2       &  (2)  & \multicolumn{3}{l}{b\neq 0} & \rA \\
    1       &  (3)  & \multicolumn{3}{l}{b=a\cdot \frac{b}{a}} & \rDivisionI{1} \\
    4       &  (4)  & \multicolumn{3}{l}{\frac{b}{a}=0} & \rA \\
    4       &  (5)  & \multicolumn{3}{l}{\frac{b}{a}=0} & \rA \\
    1,4     &  (6)  & b&=&a\cdot 0 & \rIE{4,2} \\
    1,4     &  (7)  &  &=&0 & \aInNaturalImpZeroEqualsZeroMulta{} \\
    1,4     &  (9)  & \multicolumn{3}{l}{b=0} & \rTransitivityEqRI{6,7} \\
    1,2,4   &  (10)  & \multicolumn{3}{l}{\bot} & \rBI{2,9} \\
    1,2   &  (11)  & \multicolumn{3}{l}{\frac{b}{a}\neq 0} & \rCI{4,10} \\
    \end{array}
    \]
\end{proof}

\label{awbInNaturalwbNotEqualsZerowaMidbImpaLeqb}
\begin{theorem}[\(a,b \in\mathbb{N}, b\neq 0, a\mid b\vdash a\leq b\)]
\end{theorem}
\begin{proof}
Seien \(a,b\in\mathbb{N}\).
Im Beweis nutzen wir das Theorem \(\ImpLpNaturalwMultwOneRpInAbelSemiRing{}\). 
    \[
	\begin{array}{llclll}
    1       &  (1)  & \multicolumn{3}{l}{a\mid b} & \rA \\
    2       &  (2)  & \multicolumn{3}{l}{b\neq 0} & \rA \\
    1,2     &  (3)  & \multicolumn{3}{l}{\frac{b}{a}\neq 0} & \awbInNaturalwbNotEqualsZerowaMidbImpLpbRpDurchLpaRpNotEqualsZero{1,2} \\
    1       &  (4)  & \multicolumn{3}{l}{b=a\cdot \frac{b}{a}} & \rDivisionI{1} \\
    1,2     &  (5)  & \multicolumn{3}{l}{\frac{b}{a}=(\frac{b}{a}-1)+1} & \rPredecessorI{3} \\
    1       &  (6)  & b&=&a\cdot ((\frac{b}{a}-1)+1) & \rIE{5,4} \\
    1       &  (7)  & &=& a\cdot (\frac{b}{a}-1)+a & \rLeftDistributiveAbelianSemigroup{} \\
    1       &  (8)  & &=& a+a\cdot (\frac{b}{a}-1) & \rCommutativeMonoid{} \\
    1       &  (9)  & b &=& a+a\cdot (\frac{b}{a}-1) & \rTransitivityEqRI{3,6} \\
    1       &  (10)  &  \multicolumn{3}{l}{a\leq b} & \rLeqNI{6} \\
    \end{array}
	\]
\end{proof}

\label{awbwcInNaturalwcMidawcMidbImpcMidLbaPlusbRb}
\begin{theorem}[\(a,b,c\in\mathbb{N}, c\mid a, c\mid b\vdash c\mid{a+b}\)]
\end{theorem}
\begin{proof}
Im Beweis nutzen wir das Theorem \(\ImpLpNaturalwMultwOneRpInAbelSemiRing{}\). 
Seien \(a,b,c\in\mathbb{N}\), dann gilt:
        \[
	\begin{array}{lllcll}
            1       &  (1)  & \multicolumn{3}{l}{c\mid a} & \rA \\
            2       &  (2)  & \multicolumn{3}{l}{c\mid b} & \rA \\
            1       &  (3)  & \multicolumn{3}{l}{a=c\cdot \frac{a}{c}} & \rDivisionI{1}  \\
            2       &  (4)  & \multicolumn{3}{l}{b=c\cdot \frac{b}{c}} & \rDivisionI{2}  \\
                    &  (5)  & a+b&=&a+b & \rII{}  \\
            1       &  (6)  & &=&c\cdot \frac{a}{c}+b & \rIE{3,5}  \\
            1,2     &  (7)  & &=&c\cdot \frac{a}{c}+c\cdot \frac{b}{c} & \rIE{4,6}  \\
            1,2     &  (8)  & &=&c(\frac{a}{c}+\frac{b}{c}) & \rLeftDistributiveAbelianSemigroup{}  \\
            1,2     &  (9) & \multicolumn{3}{l}{a+b=c(\frac{a}{c}+\frac{b}{c})} & \rTransitivityEqRI{5,8}  \\
            1     &  (10) & \multicolumn{3}{l}{c\neq 0} & \rDivisibilityE{1}  \\
            1,2   &  (11) & \multicolumn{3}{l}{c\mid{a+b}} & \rDivisibilityI{10,9}  \\
        \end{array}
	\]
\end{proof}



\label{aInNaturalwbInNaturalwcInNaturalwcMidawcMidbImpLpaPlusbRpDurchLpcRpEqualsLpaRpDurchLpcRpPlusLpbRpDurchLpcRp}
\begin{theorem}[\(a,b,c\in\mathbb{N}, c\mid a, c\mid b\vdash \frac{a+b}{c}=\frac{a}{c}+\frac{b}{c}\)]
\end{theorem}
\begin{proof}
Im Beweis nutzen wir das Theorem \(\ImpLpNaturalwMultwOneRpInAbelSemiRing{}\). 
Seien \(a,b,c\in\mathbb{N}\), dann gilt:
    \[
	\begin{array}{llclll}
    1       &  (1)  & \multicolumn{3}{l}{c\mid a} & \rA \\
    2       &  (2)  & \multicolumn{3}{l}{c\mid b} & \rA \\
    1,2     &  (3)  & \multicolumn{3}{l}{c\mid a+b} & \awbwcInNaturalwcMidawcMidbImpcMidLbaPlusbRb{1,2} \\
    1       &  (4)  & \multicolumn{3}{l}{a=c\cdot \frac{a}{c}} & \rDivisionI{1}  \\
    2       &  (5)  & \multicolumn{3}{l}{b=c\cdot \frac{b}{c}} & \rDivisionI{2}  \\
    1,2     &  (6)  & \multicolumn{3}{l}{a+b=c\cdot \frac{a+b}{c}} & \rDivisionI{3}  \\
    1,2     &  (7)  & c\cdot \frac{a+b}{c}&=&a+b & \rSymmetryEqRI{6}  \\
    1,2     &  (8)  & &=&c\cdot \frac{a}{c}+b & \rIE{4,7}  \\
    1,2     &  (9)  & &=&c\cdot \frac{a}{c}+c\cdot \frac{b}{c} & \rIE{5,8} \\
    1,2     &  (10)  & &=&c\cdot (\frac{a}{c}+\frac{b}{c}) & \rLeftDistributiveAbelianSemigroup{} \\
    1,2     &  (11)  & \multicolumn{3}{l}{c\cdot \frac{a+b}{c}=c\cdot (\frac{a}{c}+\frac{b}{c})} & \rTransitivityEqRI{7,10} \\
    1       &  (12)  & \multicolumn{3}{l}{c\neq 0} & \rDivisibilityE{1} \\
    1,2     &  (13)  & \multicolumn{3}{l}{\frac{a+b}{c}=\frac{a}{c}+\frac{b}{c}} & \awbwcInNaturalwcNotEqualsZerowcaEqualscbImpaEqualsb{12,11} \\

    \end{array}
	\]
\end{proof}

\label{awbwcwdInNaturalwcMidawdMidbImpcdMidab}
\begin{theorem}[\(a,b,c,d\in\mathbb{N},c\mid a, d\mid b\vdash cd\mid ab\)]
\end{theorem}
\begin{proof}
Seien \(a,b,c,d\in\mathbb{N}\).
Im Beweis nutzen wir das Theorem \(\ImpLpNaturalwMultwOneRpInAbelSemiRing{}\). 
    \[
	\begin{array}{llclll}
    1       &  (1)  & \multicolumn{3}{l}{c\mid a} & \rA \\
    2       &  (2)  & \multicolumn{3}{l}{d\mid b} & \rA \\
    1       &  (3)  & \multicolumn{3}{l}{a=c\cdot \frac{a}{c}} & \rDivisionI{1} \\
    2       &  (4)  & \multicolumn{3}{l}{b=d\cdot \frac{b}{d}} & \rDivisionI{2} \\
            &  (5)  & ab&=&ab & \rII{} \\
    1       &  (6)  & &=&(c\cdot \frac{a}{c})\cdot b & \rIE{3,5} \\
    1,2     &  (7)  & &=&(c\cdot \frac{a}{c})\cdot (d\cdot \frac{b}{d}) & \rIE{3,5} \\
    1,2       &  (8)  & &=&(c\cdot d)\cdot (\frac{a}{c}\cdot \frac{b}{d}) & \aInMwbInMwcInMwdInMImpLpaPlusbRpPlusLpcPlusdRpEqualsLpaPluscRpPlusLpbPlusdRp{} \\
    1,2       &  (9) &\multicolumn{3}{l}{ab=(c\cdot d)\cdot (\frac{a}{c}\cdot \frac{b}{d})} & \rTransitivityEqRI{5,8} \\
    1       &  (10)  & \multicolumn{3}{l}{c\neq 0} & \rDivisibilityE{1} \\
    2       &  (11)  & \multicolumn{3}{l}{d\neq 0} & \rDivisibilityE{2} \\
    1,2     &  (12)  & \multicolumn{3}{l}{cd\neq 0} & \awbInNaturalwaNotEqualsZerowbNotEqualsZeroImpabNotEqualsZero{10,11} \\
    1,2     &  (13)  & \multicolumn{3}{l}{cd\neq 0} & \awbInNaturalwaNotEqualsZerowbNotEqualsZeroImpabNotEqualsZero{10,11} \\
    1,2     &  (14)  & \multicolumn{3}{l}{cd\mid ab} & \rDivisibilityI{13,9} \\
    \end{array}
	\]
\end{proof}

\label{awbwcwdInNaturalwcMidawdMidbImpLpabRpDurchLpcdRpEqualsLpaRpDurchLpcRpMultLpbRpDurchLpdRp}
\begin{theorem}[\(a,b,c,d\in\mathbb{N},c\mid a, d\mid b\vdash \frac{ab}{cd}=\frac{a}{c}\cdot\frac{b}{d}\)]
\end{theorem}
\begin{proof}
Seien \(a,b,c,d\in\mathbb{N}\).
Im Beweis nutzen wir das Theorem \(\ImpLpNaturalwMultwOneRpInAbelSemiRing{}\). 
    \[
	\begin{array}{llclll}
    1       &  (1)  & \multicolumn{3}{l}{c\mid a} & \rA \\
    2       &  (2)  & \multicolumn{3}{l}{d\mid b} & \rA \\
    1       &  (3)  & \multicolumn{3}{l}{a=c\cdot \frac{a}{c}} & \rDivisionI{1} \\
    2       &  (4)  & \multicolumn{3}{l}{b=d\cdot \frac{b}{d}} & \rDivisionI{2} \\
            &  (5)  & ab&=&ab & \rII{} \\
    1       &  (6)  & &=&(c\cdot \frac{a}{c})\cdot b & \rIE{3,5} \\
    1,2     &  (7)  & &=&(c\cdot \frac{a}{c})\cdot (d\cdot \frac{b}{d}) & \rIE{3,5} \\
    1,2       &  (8)  & &=&(cd)\cdot (\frac{a}{c}\cdot \frac{b}{d}) & \aInMwbInMwcInMwdInMImpLpaPlusbRpPlusLpcPlusdRpEqualsLpaPluscRpPlusLpbPlusdRp{} \\
    1,2       &  (9) &\multicolumn{3}{l}{ab=(cd)\cdot (\frac{a}{c}\cdot \frac{b}{d})} & \rTransitivityEqRI{5,8} \\
    1,2       &  (10)  & \multicolumn{3}{l}{cd\mid ab} & \awbwcwdInNaturalwcMidawdMidbImpcdMidab{1,2} \\
    1,2       &  (11)  & \multicolumn{3}{l}{\frac{ab}{cd}=\frac{a}{c}\cdot\frac{b}{d}} & \rDivisionI{10,9} \\
    \end{array}
	\]
\end{proof}

\label{aInNaturalwbInNaturalwcInNaturalwdInNaturalwcMidawdMidbImpdcMidadPlusbc}
\begin{theorem}[\(a\in\mathbb{N},b\in\mathbb{N}, c\in\mathbb{N}, d\in\mathbb{N},c\mid a, d\mid b\vdash dc\mid ad+bc\)]
\end{theorem}
\begin{proof}
Seien \(a,b,c,d\in\mathbb{N}\).
Im Beweis nutzen wir das Theorem \(\ImpLpNaturalwMultwOneRpInAbelSemiRing{}\). 
    \[
	\begin{array}{llclll}
    1       &  (1)  & \multicolumn{3}{l}{c\mid a} & \rA \\
    2       &  (2)  & \multicolumn{3}{l}{d\mid b} & \rA \\
    1       &  (3)  & \multicolumn{3}{l}{a=c\cdot \frac{a}{c}} & \rDivisionI{1} \\
    2       &  (4)  & \multicolumn{3}{l}{b=d\cdot \frac{b}{d}} & \rDivisionI{2} \\
            &  (5)  & ad+bc&=&ad+bc & \rII{} \\
    1       &  (6)  & &=&((c\cdot \frac{a}{c})\cdot d)+ bc & \rIE{3,5} \\
    1,2     &  (7)  & &=&((c\cdot \frac{a}{c})\cdot d)+((d\cdot \frac{b}{d})\cdot c) & \rIE{4,6} \\
    1,2     &  (8)  & &=&(dc\cdot \frac{a}{c})+((d\cdot \frac{b}{d})\cdot c) & \MInAbelMonoidwawbwcInMImpLpabRpcEqualsLpcaRpb{} \\
    1,2     &  (9)  & &=&(dc\cdot \frac{a}{c})+(dc\cdot \frac{b}{d}) & \awbwcwdInNaturalwcMidawdMidbImpcdMidab{} \\
    1,2     &  (10)  & &=&dc(\frac{a}{c}+\frac{b}{d}) & \rLeftDistributiveAbelianSemigroup{} \\
    1,2     &  (11) &\multicolumn{3}{l}{ad+bc=dc(\frac{a}{c}+\frac{b}{d})} & \rTransitivityEqRI{5,10} \\
    1,2     &  (12)  & \multicolumn{3}{l}{c\neq 0} & \rDivisibilityE{1} \\
    2       &  (13)  & \multicolumn{3}{l}{d\neq 0} & \rDivisibilityE{2} \\
    1,2     &  (14)  & \multicolumn{3}{l}{dc\neq 0} & \awbInNaturalwaNotEqualsZerowbNotEqualsZeroImpabNotEqualsZero{13,12} \\
    1,2     &  (15)  & \multicolumn{3}{l}{dc\mid ad+bc} & \rDivisibilityI{14,11} \\
    \end{array}
	\]
\end{proof}

\label{aInNaturalwbInNaturalwcInNaturalwdInNaturalwcMidawdMidbImpLpadPlusbcRpDurchLpdcRpEqualsLpaRpDurchLpcRpPlusLpbRpDurchLpdRp}
\begin{theorem}[\(a\in\mathbb{N},b\in\mathbb{N}, c\in\mathbb{N}, d\in\mathbb{N},c\mid a, d\mid b\vdash \frac{ad+bc}{dc}=\frac{a}{c}+\frac{b}{d}\)]
\end{theorem}
\begin{proof}
Seien \(a,b,c,d\in\mathbb{N}\).
Im Beweis nutzen wir das Theorem \(\ImpLpNaturalwMultwOneRpInAbelSemiRing{}\). 
    \[
	\begin{array}{llclll}
    1       &  (1)  & \multicolumn{3}{l}{c\mid a} & \rA \\
    2       &  (2)  & \multicolumn{3}{l}{d\mid b} & \rA \\
    1       &  (3)  & \multicolumn{3}{l}{a=c\cdot \frac{a}{c}} & \rDivisionI{1} \\
    2       &  (4)  & \multicolumn{3}{l}{b=d\cdot \frac{b}{d}} & \rDivisionI{2} \\
            &  (5)  & ad+bc&=&ad+bc & \rII{} \\
    1       &  (6)  & &=&((c\cdot \frac{a}{c})\cdot d)+ bc & \rIE{3,5} \\
    1,2     &  (7)  & &=&((c\cdot \frac{a}{c})\cdot d)+((d\cdot \frac{b}{d})\cdot c) & \rIE{4,6} \\
    1,2     &  (8)  & &=&(dc\cdot \frac{a}{c})+((d\cdot \frac{b}{d})\cdot c) & \MInAbelMonoidwawbwcInMImpLpabRpcEqualsLpcaRpb{} \\
    1,2     &  (9)  & &=&(dc\cdot \frac{a}{c})+(dc\cdot \frac{b}{d}) & \awbwcwdInNaturalwcMidawdMidbImpcdMidab{} \\
    1,2     &  (10)  & &=&dc(\frac{a}{c}+\frac{b}{d}) & \rLeftDistributiveAbelianSemigroup{} \\
    1,2     &  (11) &\multicolumn{3}{l}{ad+bc=dc(\frac{a}{c}+\frac{b}{d})} & \rTransitivityEqRI{5,10} \\
    1,2     &  (12)  & \multicolumn{3}{l}{dc\mid ad+bc} & \aInNaturalwbInNaturalwcInNaturalwdInNaturalwcMidawdMidbImpdcMidadPlusbc{1,2} \\
    1,2     &  (13)  & \multicolumn{3}{l}{\frac{ad+bc}{dc}=\frac{a}{c}+\frac{b}{d}} & \rDivisionI{12,11} \\
    \end{array}
	\]
\end{proof}

\section{Teilbarkeit und Relationen}

\label{awbInNaturalwaMidbwaMidcwbLeqcImpLpbRpDurchLpaRpLeqLpcRpDurchLpaRp}
\begin{theorem}[\(a,b\in\mathbb{N}, a\mid b, a\mid c, b\leq c\vdash \frac{b}{a}\leq \frac{c}{a}\)]    
\end{theorem}
\begin{proof}
Seien \(a,b,c \in\mathbb{N}\).
    \[
	\begin{array}{llclll}
    1       &  (1)  & \multicolumn{3}{l}{a\mid b} & \rA \\
    2       &  (2)  & \multicolumn{3}{l}{a\mid c} & \rA \\
    1       &  (3)  & \multicolumn{3}{l}{b=a\cdot\frac{b}{a}} & \rDivisionI{1} \\
    2       &  (4)  & \multicolumn{3}{l}{c=a\cdot\frac{c}{a}} & \rDivisionI{2} \\
    1       &  (5)  & a\cdot \frac{b}{a}&=&b & \rSymmetryEqRI{3} \\
    7       &  (6)  & &\leq &c & \rA \\
    2,7     &  (7)  & &= &a\cdot\frac{c}{a} & \rIE{4,6} \\
    1,2,7   &  (8)  & a\cdot \frac{b}{a}&\leq  &a\cdot\frac{c}{a} & \rIE{4,6} \\
    1       &  (9)  & \multicolumn{3}{l}{a\neq 0} & \rDivisibilityE{1} \\
    1,2,7   &  (10)  & \frac{b}{a}&\leq  &\frac{c}{a} & \awbwcInNaturalwcNotEqualsZerowcaLeqcbImpaLeqb{9,8} \\
    \end{array}
	\]
\end{proof}

\section{Eigenschaften der Halbordnung}

\label{aInNaturalwaNotEqualsZeroImpaMidLbaRb}
\begin{theorem}[\(a\in\mathbb{N}, a\neq 0 \vdash a\mid{a}\)]
\end{theorem}
\begin{proof}
Sei \(a\in\mathbb{N}\).
Im Beweis nutzen wir das Theorem \(\ImpLpNaturalwMultwOneRpInAbelSemiRing{}\). 
    \[
	\begin{array}{llclll}
    1       &  (1)  & \multicolumn{3}{l}{a\neq 0} & \rA \\
            &  (2)  & \multicolumn{3}{l}{a=a\cdot 1} & \rNeutralElementMonoid{} \\
    1       &  (3)  & \multicolumn{3}{l}{a\mid a} & \rDivisibilityI{1,2} \\
    \end{array}
	\]
\end{proof}

\label{aInNaturalwaNotEqualsZeroImpLpaRpDurchLpaRpEqualsOne}
\begin{theorem}[\(a\in\mathbb{N}, a\neq 0 \vdash \frac{a}{a}=1\)]
\end{theorem}
\begin{proof}
Sei \(a\in\mathbb{N}\).
Im Beweis nutzen wir das Theorem \(\ImpLpNaturalwMultwOneRpInAbelSemiRing{}\). 
    \[
	\begin{array}{llclll}
    1       &  (1)  & \multicolumn{3}{l}{a\neq 0} & \rA \\
            &  (2)  & \multicolumn{3}{l}{a=a\cdot 1} & \rNeutralElementMonoid{} \\
    1       &  (3)  & \multicolumn{3}{l}{a\mid a} & \rDivisibilityI{1,2} \\
    1       &  (4)  & \multicolumn{3}{l}{\frac{a}{a}=1} & \rDivisionI{2,3} \\
    \end{array}
	\]
\end{proof}


\label{awbInNaturalwaMidbwbMidaImpaEqualsb}
\begin{theorem}[\(a,b\in\mathbb{N}, a\mid b, b\mid a \vdash a=b\)]
\end{theorem}
\begin{proof}
Seien \(a,b\in\mathbb{N}\).
Im Beweis nutzen wir das Theorem \(\LeqIsHalfOrderOnNaturalNumbers{}\). 
    \[
	\begin{array}{llclll}
    1       &  (1)  & \multicolumn{3}{l}{a\mid b} & \rA \\
    2       &  (2)  & \multicolumn{3}{l}{b\mid a} & \rA \\
    1       &  (3)  & \multicolumn{3}{l}{a\neq 0} & \rDivisibilityE{1} \\
    2       &  (4)  & \multicolumn{3}{l}{b\neq 0} & \rDivisibilityE{2} \\
    1,2       &  (5)  & \multicolumn{3}{l}{a\leq b} & \awbInNaturalwbNotEqualsZerowaMidbImpaLeqb{1,4} \\
    1,2       &  (6)  & \multicolumn{3}{l}{b\leq a} & \awbInNaturalwbNotEqualsZerowaMidbImpaLeqb{2,3} \\
    1,2       &  (7)  & \multicolumn{3}{l}{a=b} & \rAntisymmetryOrdRI{5,6} \\
    \end{array}
	\]
\end{proof}

\label{awbwcInNaturalwaMidbwbMidcImpaMidc}
\begin{theorem}[\(a,b,c\in\mathbb{N}, a\mid b, b\mid c \vdash a\mid c\)]
\end{theorem}
\begin{proof}
Sei \(a\in\mathbb{N}\).
Im Beweis nutzen wir das Theorem \(\LeqIsHalfOrderOnNaturalNumbers{}\). 
    \[
	\begin{array}{llclll}
        1       &  (1)  & \multicolumn{3}{l}{a\mid b} & \rA \\
        2       &  (2)  & \multicolumn{3}{l}{b\mid c} & \rA \\
        1       &  (3)  & \multicolumn{3}{l}{b=a\cdot \frac{b}{a}} & \rDivisionI{1} \\
        2       &  (4)  & \multicolumn{3}{l}{c=b\cdot \frac{c}{b}} & \rDivisionI{2} \\
        1       &  (5)  & \multicolumn{3}{l}{a\neq 0} & \rDivisibilityE{1} \\
        1,2     &  (6)  & c&=&(a\cdot \frac{b}{a})\cdot \frac{c}{b} & \rIE{3,4} \\
        1,2     &  (7)  & &=&a\cdot (\frac{b}{a}\cdot \frac{c}{b}) & \rAssociativityMonoid{} \\
        1,2     &  (8)  & \multicolumn{3}{l}{c=a\cdot (\frac{b}{a}\cdot \frac{c}{b})} & \rTransitivityEqRI{6,7} \\
        1,2     &  (9)  & \multicolumn{3}{l}{a\mid c} & \rDivisibilityI{5,8} \\
        \end{array}
    \]
\end{proof}


\label{awbwcInNaturalwaMidbwbMidcImpLpaRpDurchLpcRpEqualsLpbRpDurchLpaRpMultLpcRpDurchLpbRp}
\begin{theorem}[\(a,b,c\in\mathbb{N}, a\mid b, b\mid c \vdash \frac{c}{a}=\frac{b}{a}\cdot\frac{c}{b}\)]
\end{theorem}
\begin{proof}
Seien \(a,b\in\mathbb{N}\).
Im Beweis nutzen wir das Theorem \(\LeqIsHalfOrderOnNaturalNumbers{}\). 
    \[
	\begin{array}{llclll}
        1       &  (1)  & \multicolumn{3}{l}{a\mid b} & \rA \\
        2       &  (2)  & \multicolumn{3}{l}{b\mid c} & \rA \\
        1       &  (3)  & \multicolumn{3}{l}{b=a\cdot \frac{b}{a}} & \rDivisionI{1} \\
        2       &  (4)  & \multicolumn{3}{l}{c=b\cdot \frac{c}{b}} & \rDivisionI{2} \\
        1       &  (5)  & \multicolumn{3}{l}{a\neq 0} & \rDivisibilityE{1} \\
        1,2     &  (6)  & c&=&(a\cdot \frac{b}{a})\cdot \frac{c}{b} & \rIE{3,4} \\
        1,2     &  (7)  & &=&a\cdot (\frac{b}{a}\cdot \frac{c}{b}) & \rAssociativityMonoid{} \\
        1,2     &  (8)  & \multicolumn{3}{l}{c=a\cdot (\frac{b}{a}\cdot \frac{c}{b})} & \rTransitivityEqRI{6,7} \\
        1,2     &  (9)  & \multicolumn{3}{l}{\frac{c}{a}=\frac{b}{a}\cdot \frac{c}{b}} & \rDivisionI{5,8} \\
        \end{array}
    \]
\end{proof}

\label{MidIsHalfOrderOnNaturalNumbers}
\begin{theorem}[\(\mid\) ist eine Halbordnung auf \(\mathbb{N}\) ]
\end{theorem}
\begin{proof}
        \[
	\begin{array}{llll}
                    & (1) & \forall a \in \mathbb{N}  (a \mid a) & \aInNaturalwaNotEqualsZeroImpaMidLbaRb{} \\
                    & (2) & \forall a, b \in \mathbb{N}  ((a \mid b \land b \mid a) \rightarrow a = b) & \awbInNaturalwaMidbwbMidaImpaEqualsb{} \\
                    & (3) & \forall a, b, c \in \mathbb{N}  ((a \mid b \land b \mid c) \rightarrow a \mid c) & \awbwcInNaturalwaMidbwbMidcImpaMidc{} \\
                    & (4) & \mid \text{ ist eine Halbordnung auf } \mathbb{N} &  \rPartialOrderRelationI{1,2,3}
    \end{array}
	\]
\end{proof}



\label{awbInNaturalwabEqualsOneImpaEqualsOneAndbEqualsOne}
\begin{theorem}[\(a,b\in\mathbb{N},ab=1\vdash a=1\land b=1\)]
\end{theorem}
\begin{proof}
Sei \(a\in\mathbb{N}\).
Im Beweis nutzen wir das Theorem \(\MidIsHalfOrderOnNaturalNumbers{}\). 
    \[
	\begin{array}{llclll}
        1       &  (1)  & \multicolumn{3}{l}{ab=1} & \rA \\
                &  (2)  & \multicolumn{3}{l}{1\neq 0} & \ImpOneNotEqualsZero{} \\
        1       &  (3)  & \multicolumn{3}{l}{ab\neq 0} & \rIE{1,2} \\
        1       &  (4)  & \multicolumn{3}{l}{a\neq 0} & \aInNaturalwabNotEqualsZeroImpaNotEqualsZero{3} \\
        1       &  (5)  & \multicolumn{3}{l}{b\neq 0} & \aInNaturalwabNotEqualsZeroImpbNotEqualsZero{3} \\
        1       &  (6)  & \multicolumn{3}{l}{a\mid 1} & \rDivisibilityI{1,4} \\
        1       &  (7)  & \multicolumn{3}{l}{b\mid 1} & \rDivisibilityI{1,5} \\
                &  (8)  & \multicolumn{3}{l}{1\mid a} & \aInNaturalImpOneMida{} \\
                &  (9)  & \multicolumn{3}{l}{1\mid b} & \aInNaturalImpOneMida{} \\
        1       &  (10)  & \multicolumn{3}{l}{a=1} & \rAntisymmetryOrdRI{6,8} \\
        1       &  (11)  & \multicolumn{3}{l}{b=1} & \rAntisymmetryOrdRI{7,9} \\
        1       &  (12)  & \multicolumn{3}{l}{a=1\land b=1} & \rAI{10,11} \\
        \end{array}
    \]
\end{proof}


\label{awbInNaturalwaNotEqualsOnewaMidbImpaNMidbPlusOne}
\begin{theorem}[\(a,b\in\mathbb{N}, a\neq 1, a\mid b \vdash a\nmid b+1\)]
\end{theorem}
\begin{proof}
Seien \(a,b\in\mathbb{N}\).
    \[
	\begin{array}{llclll}
        1       &  (1)  & \multicolumn{3}{l}{a\mid b} & \rA \\
        1       &  (2)  & \multicolumn{3}{l}{b=a\cdot \frac{b}{a}} & \rDivisionI{1} \\
        3       &  (3)  & \multicolumn{3}{l}{a\mid b+1} & \rA \\
        3       &  (4)  & \multicolumn{3}{l}{b+1=a\cdot \frac{b+1}{a}} & \rDivisionI{3} \\
                &  (5)  & \multicolumn{3}{l}{b\leq b+1} & \aInNaturalImpaLeqaPlusOne{} \\
        1,3     &  (6)  & \multicolumn{3}{l}{\frac{b}{a}\leq \frac{b+1}{a}} & \awbInNaturalwaMidbwaMidcwbLeqcImpLpbRpDurchLpaRpLeqLpcRpDurchLpaRp{1,3,5} \\
                &  (7)  & 1&=&1+0 & \rNeutralElementMonoid{} \\
                &  (8)  &  &=&1+(a\cdot\frac{b}{a}-a\cdot\frac{b}{a}) & \aInNaturalImpaMinusaEqualsZero{} \\
                &  (9)  &  &=&(1+a\cdot\frac{b}{a})-a\cdot\frac{b}{a} & \awbInNaturalImpLpaPlusbRpMinusbEqualsaPlusLpbMinusbRpEqualsa{} \\
        1       &  (10)  &  &=&(1+b)-a\cdot\frac{b}{a} & \rIE{2,9} \\
        1       &  (11)  &  &=&(b+1)-a\cdot\frac{b}{a} & \rCommutativeMonoid{} \\
        1,3     &  (12)  &  &=&a\cdot \frac{b+1}{a}-a\cdot\frac{b}{a} & \rIE{4,11} \\
        1,3     &  (13)  &  &=&a\cdot(\frac{b+1}{a}-\frac{b}{a}) & \awbwcInNaturalwcLeqbImpaLpbMinuscRpEqualsabMinusac{} \\
        1,3     &  (14)  & 1&=&a\cdot(\frac{b+1}{a}-\frac{b}{a}) & \rTransitivityEqRI{7,13} \\
        1,3     &  (15)  & \multicolumn{3}{l}{a=1\land (\frac{b+1}{a}-\frac{b}{a})=1 } & \awbInNaturalwabEqualsOneImpaEqualsOneAndbEqualsOne{14} \\
        1,3     &  (16)  & \multicolumn{3}{l}{a=1 } & \rAEa{15} \\
        17     &  (17)  & \multicolumn{3}{l}{a\neq 1 } & \rA \\
        1,3,17     &  (18)  & \multicolumn{3}{l}{\bot} & \rBI{16,17} \\
        1,17     &  (19)  & \multicolumn{3}{l}{a\nmid b+1} & \rCI{3,18} \\
        \end{array}
    \]
\end{proof}

\chapter{Darstellung natürlicher Zahlen durch Division mit Rest}

\begin{definition}[Division mit Rest]
Seien \( a, b \in \mathbb{N} \) mit \( b \neq 0 \). Wir nennen \( a \) den \emph{Dividend} und \( b \) den \emph{Divisor}. Wir sagen, \( q \) sei der Quotient und \( r \) der Rest der Division von \( a \) durch \( b \), wenn:
\[
\exists q, r \in \mathbb{N} \ \big(a = b \cdot q + r \land 0 \leq r < b\big).
\]
\end{definition}


\section{Existenz der Division mit Rest}
In diesem Abschnitt seien \(a,b\in\mathbb{N}\).

\begin{tempdefinition}
    \[M:=\{r\in\mathbb{N}\mid \exists q\in\mathbb{N}(r=a-bq)\}\]
\end{tempdefinition}


\label{tIMDefineEqualsLbrInNaturalMidExqInNaturalLprEqualsaMinusbqRpRb}
\paragraph{Einführungsregel für die Elementzugehörigkeit von \(r\) in \(M\)}
\[
\begin{array}{llll}
    i & (1) & r=a-bq & ... \\
    i & (2) & r\in M & \tIMDefineEqualsLbrInNaturalMidExqInNaturalLprEqualsaMinusbqRpRb{1}  \\
\end{array}
\]

\label{tEMDefineEqualsLbrInNaturalMidExqInNaturalLprEqualsaMinusbqRpRb}
\paragraph{Eliminierungsregel für die Elementzugehörigkeit von \(r\) in \(M\)}
\[
\begin{array}{llll}
    i & (1) & r\in M & ... \\
    i & (2) & \exists q\in\mathbb{N}(r=a-bq) & \tEMDefineEqualsLbrInNaturalMidExqInNaturalLprEqualsaMinusbqRpRb{1}  \\
\end{array}
\]

Sei \(q \in \mathbb{N}\) eine neue, frische Variable, die zuvor nicht im Beweis verwendet wurde. Dann kann auch folgende Regel verwendet werden:
\[
\begin{array}{llll}
    i & (1) & r\in M & ... \\
    2 & (2) & r=a-bq & \rA  \\
    2,j & (3) & Q & ...  \\
    i,j & (4) & Q & \tEMDefineEqualsLbrInNaturalMidExqInNaturalLprEqualsaMinusbqRpRb{1,2,3}  \\
\end{array}
\]

\label{tMDefineEqualsLbrInNaturalMidExqInNaturalLprEqualsaMinusbqRpRbSubseteqNatural}
\begin{lemma}[\(M\subseteq\mathbb{N}\)]
\end{lemma}
\begin{proof}
    \[
	\begin{array}{llcll p{4.3cm}}
            &  (1)  & \multicolumn{3}{l}{M\subseteq \mathbb{N}} & \ImpLbxInAMidPLpxRpRbSubseteqA{} \\
        \end{array}
    \]
\end{proof}

\label{tMDefineEqualsLbrInNaturalMidExqInNaturalLprEqualsaMinusbqRpNotEqualsEmptyset}
\begin{lemma}[\(M\neq\emptyset\)]
\end{lemma}
\begin{proof}
    \[
	\begin{array}{llcll p{4.3cm}}
            &  (1)  &  a&=&a-0 & \aInNaturalImpaMinusZeroEqualsa{} \\
            &  (2)  &   &=&a-b\cdot 0 & \rIE{2,5} \\
            &  (3)  &  a&=&a-b\cdot 0 & \rTransitivityEqRI{5,6} \\
            &  (4)  &  \multicolumn{3}{l}{a\in M} & \tIMDefineEqualsLbrInNaturalMidExqInNaturalLprEqualsaMinusbqRpRb{3} \\
            &  (5)  &  \multicolumn{3}{l}{\exists x(x\in M)} & \rEI{4} \\
            &  (6)  &  \multicolumn{3}{l}{M\neq \emptyset} & \ExxInSImpSNotEqualsEmptyset{5} \\
        \end{array}
    \]
\end{proof}

\label{tnInMDefineEqualsLbrInNaturalMidExqInNaturalLprEqualsaMinusbqRpRbwbLeqnImpnMinusbInM}
\begin{lemma}[\(n\in M,b\leq n\vdash n-b\in M\)]
\end{lemma}
\begin{proof}
Im Beweis nutzen wir das Theorem \(\ImpLpNaturalwMultwOneRpInAbelSemiRing{}\). Sei \(n\in\mathbb{N}\) mit:
    \[
	\begin{array}{llcll p{4.3cm}}
           1 &  (1)  & \multicolumn{3}{l}{n\in M} & \rA \\
           2 &  (2)  & \multicolumn{3}{l}{b\leq n} & \rA \\
           3 &  (3)  & \multicolumn{3}{l}{n=a-bq} & \rA \\
           3 &  (4)  & \multicolumn{3}{l}{bq\leq a} & \minusE{3} \\
           2,3 &  (5)  & \multicolumn{3}{l}{b\leq a-bq} & \rIE{3,2} \\
           2,3 &  (6)  & b+bq&\leq& (a-bq)+bq & \awbwcInNaturalLpaLeqbEqvaPluscLeqbPluscRp{5} \\
           3 &  (7)  & &=& a & \awbInNaturalwbLeqaImpLpaMinusbRpPlusbEqualsa{4} \\
           2,3 &  (8)  & b+bq&\leq & a & \rTransitivityEqRI{6,7} \\
           2 &  (9)  & n-b&=& n-b & \minusI{2} \\
           2,3 &  (10)  &  &=& (a-bq)-b & \rIE{3,9} \\
           2,3 &  (11)  &  &=& a-(bq+b) & \awbwcInNaturalwbPluscLeqaImpaMinusLpbPluscRpEqualsLpaMinusbRpMinusc{8} \\
             &  (12)  &  &=& a-(bq+b\cdot 1) & \rNeutralElementMonoid{} \\
             &  (13)  &  &=& a-(b(q+1)) & \rLeftDistributiveAbelianSemigroup{} \\
            2,3 &  (14)  &  n-b&=& a-(b(q+1)) & \rTransitivityEqRI{9,13} \\
            2,3 &  (15)  &  \multicolumn{3}{l}{n-b\in M} & \tIMDefineEqualsLbrInNaturalMidExqInNaturalLprEqualsaMinusbqRpRb{14} \\
            1,2 &  (16)  &  \multicolumn{3}{l}{n-b\in M} & \tEMDefineEqualsLbrInNaturalMidExqInNaturalLprEqualsaMinusbqRpRb{1,3,15} \\
        \end{array}
    \]
\end{proof}

\label{tExnInMDefineEqualsLbrInNaturalMidExqInNaturalLprEqualsaMinusbqRpRbLpnEqualsMinLpMRp}
\begin{lemma}[\(\exists n\in M(n=\min(M)\)]
\end{lemma}
\begin{proof}
    \[
	\begin{array}{llcll p{4.3cm}}
          &  (1)  & \multicolumn{3}{l}{M\subseteq\mathbb{N}} & \tMDefineEqualsLbrInNaturalMidExqInNaturalLprEqualsaMinusbqRpRbSubseteqNatural{} \\
          &  (2)  & \multicolumn{3}{l}{M\neq\emptyset} & \tMDefineEqualsLbrInNaturalMidExqInNaturalLprEqualsaMinusbqRpNotEqualsEmptyset{} \\
          &  (3)  & \multicolumn{3}{l}{\exists n\in\mathbb{N}(n=\min(M))} & \ASubseteqNaturalwANotEqualsEmptysetImpExnInALpnEqualsMinLpARpRp{1,2} \\
        \end{array}
    \]
\end{proof}

\label{MinLpMDefineEqualsLbrInNaturalMidExqInNaturalLprEqualsaMinusbqRpRbRpInNatural}
\begin{lemma}[\(\min(M)\in\mathbb{N}\)]
\end{lemma}
\begin{proof}
    \[
	\begin{array}{llcll p{4.3cm}}
          &  (1)  & \multicolumn{3}{l}{\exists n\in M(n=\min(M)} & \tExnInMDefineEqualsLbrInNaturalMidExqInNaturalLprEqualsaMinusbqRpRbLpnEqualsMinLpMRp{} \\
          &  (2)  & \multicolumn{3}{l}{\min(M)\in M} & \rMinE{1} \\
          &  (3)  & \multicolumn{3}{l}{\min(M)\in \mathbb{N}} & \inE{2} \\
        \end{array}
    \]
\end{proof}


\label{tbNotEqualsZeroImpMinLpMDefineEqualsLbrInNaturalMidExqInNaturalLprEqualsaMinusbqRpRbRpLneqb}
\begin{lemma}[\(b\neq 0\vdash \min(M)<b\)]
\end{lemma}
\begin{proof}
    \[
	\begin{array}{llcll p{4.3cm}}
        1 &  (1)  & \multicolumn{3}{l}{b\neq 0} & \rA \\
          &  (2)  & \multicolumn{3}{l}{\exists n\in\mathbb{N}(n=\min(M))} & \tExnInMDefineEqualsLbrInNaturalMidExqInNaturalLprEqualsaMinusbqRpRbLpnEqualsMinLpMRp{} \\
         3 &  (3)  & \multicolumn{3}{l}{b\leq min(M)} & \rA \\
           &  (4)  & \multicolumn{3}{l}{\min(M)\in M} & \rMinE{2} \\
         3 &  (5)  & \multicolumn{3}{l}{\min(M)-b\in M} & \tnInMDefineEqualsLbrInNaturalMidExqInNaturalLprEqualsaMinusbqRpRbwbLeqnImpnMinusbInM{3,4} \\
         3 &  (6)  & \multicolumn{3}{l}{\min(M)-b\leq \min(M)} & \awbInNaturalaLeqbImpbMinusaLeqb{5,3} \\
           &  (7)  & \multicolumn{3}{l}{\min(M)\leq \min(M)-b} & \rMinE{2} \\
         3 &  (8)  & \multicolumn{3}{l}{\min(M)=\min(M)-b} & \rAntisymmetryOrdRI{6,7} \\
         3 &  (9)  & \multicolumn{3}{l}{b=0} & \awbInNaturalwaMinusbEqualsaImpbEqualsZero{8} \\
         1,3 &  (10)  & \multicolumn{3}{l}{\bot} & \rBI{1,9} \\
         1 &  (11)  & \multicolumn{3}{l}{\neg(b\leq\min(M))} & \rBI{1,9} \\
         1 &  (12)  & \multicolumn{3}{l}{\min(M)<b} & \aLneqbEqvnLpbLeqaRp{11} \\

        \end{array}
    \]
\end{proof}


\label{tExqInNaturalLpaEqualsbqPlusMinLpMDefineEqualsLbrInNaturalMidExqInNaturalLprEqualsaMinusbqRpRbRpRp}
\begin{lemma}[\(\exists q\in\mathbb{N}(a=bq+\min(M))\)]
\end{lemma}
\begin{proof}
\(\ImpLpNaturalwPluswZeroRpInAbelMonoid{}\) und daher gilt:
    \[
	\begin{array}{llcll p{4.3cm}}
        1 &  (1)  & \multicolumn{3}{l}{\exists n\in M(n=\min(M))} & \tExnInMDefineEqualsLbrInNaturalMidExqInNaturalLprEqualsaMinusbqRpRbLpnEqualsMinLpMRp{} \\
        1 &  (2)  & \multicolumn{3}{l}{\min(M)\in \mathbb{N}} & \MinLpMDefineEqualsLbrInNaturalMidExqInNaturalLprEqualsaMinusbqRpRbRpInNatural{1} \\
        3 &  (3)  & \multicolumn{3}{l}{\min(M)=a-bq} & \rA \\
        1,3 &  (4)  & \multicolumn{3}{l}{a-bq\in\mathbb{N}} & \rIE{3,2} \\
        1,3 &  (5)  & \multicolumn{3}{l}{bq\leq a} & \minusE{4} \\
          &  (6)  & bq+\min(M)&=&\min(M)+bq & \rCommutativeMonoid{} \\
        3 &  (7)  &  &=&(a-bq)+bq & \aEqualsbEqvaPluscEqualsbPlusc{3} \\
        1,3 &  (8)  & &=&a & \awbInNaturalwbLeqaImpLpaMinusbRpPlusbEqualsa{5} \\
        1,3 &  (9)  & bq+\min(M)&=&a & \rTransitivityEqRI{7,8} \\
        1,3 &  (10)  & a&=&bq+\min(M) & \rSymmetryEqRI{9} \\
        1,3 &  (11)  & \multicolumn{3}{l}{\exists q\in\mathbb{N}(a=bq+\min(M))} & \rEI{10} \\
        1 &  (12)  & \multicolumn{3}{l}{\exists q\in\mathbb{N}(a=bq+\min(M))} & \tEMDefineEqualsLbrInNaturalMidExqInNaturalLprEqualsaMinusbqRpRb{11} \\

        \end{array}
    \]
\end{proof}

\label{bNotEqualsZeroImpExrInNaturalLprLneqbAndExqInNaturalLpaEqualsbqPlusrRpRp}
\begin{theorem}[\(b\neq 0\vdash \exists r\in\mathbb{N}(r<b\land \exists q\in\mathbb{N}(a=bq+r))\)]
\end{theorem}
\begin{proof}
    \[
	\begin{array}{llcll p{4.3cm}}
        1 &  (1)  & \multicolumn{3}{l}{b\neq 0} & \rA \\
          &  (2)  & \multicolumn{3}{l}{\exists q\in\mathbb{N}(a=bq+\min(M))} & \tExqInNaturalLpaEqualsbqPlusMinLpMDefineEqualsLbrInNaturalMidExqInNaturalLprEqualsaMinusbqRpRbRpRp{} \\
        1 &  (3)  & \multicolumn{3}{l}{\min(M)<b} & \tbNotEqualsZeroImpMinLpMDefineEqualsLbrInNaturalMidExqInNaturalLprEqualsaMinusbqRpRbRpLneqb{1} \\
        1 &  (4)  & \multicolumn{3}{l}{\min(M)<b\land \exists q\in\mathbb{N}(a=bq+\min(M))} & \rAI{3,2} \\
        1 &  (5)  & \multicolumn{3}{l}{\min(M)<b\land \exists q\in\mathbb{N}(a=bq+\min(M))} & \rAI{3,2} \\
          &  (6)  & \multicolumn{3}{l}{\min(M)\in\mathbb{N}} & \MinLpMDefineEqualsLbrInNaturalMidExqInNaturalLprEqualsaMinusbqRpRbRpInNatural{} \\
        1 &  (7)  & \multicolumn{3}{l}{\exists r\in\mathbb{N}(r<b\land \exists q\in\mathbb{N}(a=bq+r))} & \rSetEIa{6,5} \\
        \end{array}
    \]
\end{proof}


\section{Eindeutigkeit der Division mit Rest}


\label{awbwcwdwfInNaturalwaLeqcwbLeqdImpfLpcMinusaRpPlusLpdMinusbRpEqualsLpfcPlusdRpMinusLpfaPlusbRp}
\begin{theorem}[\(a,b,c,d,f\in\mathbb{N},a\leq c, b\leq d\vdash f(c-a)+(d-b)=(fc+d)-(fa+b)\)]
\end{theorem}
\begin{proof}
Seien \(a,b,c,d,f\in\mathbb{N}\).
    \[
	\begin{array}{llcll p{4.3cm}}
        1       &  (1)  & \multicolumn{3}{l}{a\leq c} & \rA \\
        2       &  (2)  & \multicolumn{3}{l}{b\leq d} & \rA \\
        1       &  (3)  & \multicolumn{3}{l}{fa\leq fd} & \awbwcInNaturalImpaLeqbImpcaLeqcb{1} \\
        1       &  (4)  & f(c-a)+(d-b)&=&(fc-fa)+(d-b) & \awbwcInNaturalwcLeqbImpaLpbMinuscRpEqualsabMinusac{1} \\
        1,2       &  (5)  &             &=&(fc+d)-(fa+b) & \awbwcwdInNaturalwaLeqcwbLeqdImpLpcMinusaRpPlusLpdMinusbRpEqualsLpcPlusdRpMinusLpaPlusbRp{3,2} \\
        1,2       &  (6)  & f(c-a)+(d-b)&=&(fc+d)-(fa+b) & \rTransitivityEqRI{4,5} \\
        \end{array}
    \]
\end{proof}

\label{awbwcwdwfInNaturalwfNotEqualsZerowfLpcMinusaRpEqualsZeroAndLpdMinusbRpEqualsZeroImpcEqualsaAnddEqualsb}
\begin{theorem}[\(a,b,c,d,f\in\mathbb{N},f\neq 0, f(c-a)=0\land (d-b)=0 \vdash c=a\land d=b\)]
\end{theorem}
\begin{proof}
Seien \(a,b,c,d,f\in\mathbb{N}\).
    \[
	\begin{array}{llclll}
        1       &  (1)  & \multicolumn{3}{l}{f(c-a)=0\land (d-b)=0} & \rA \\
        2       &  (2)  & \multicolumn{3}{l}{f\neq 0} & \rA \\
        1       &  (3)  & \multicolumn{3}{l}{f(c-a)=0} & \rAEa{1} \\
        1       &  (4)  & \multicolumn{3}{l}{d-b=0} & \rAEb{1} \\
        1       &  (5)  & \multicolumn{3}{l}{d=b} & \awbInNaturalLpaEqualsbEqvaMinusbEqualsZeroRp{3} \\
        1,2     &  (6)  & \multicolumn{3}{l}{c-a=0} & \awbInNaturalwaNotEqualsZerowabEqualsZeroImpbEqualsZero{2,3} \\
        1,2     &  (7)  & \multicolumn{3}{l}{c=a} & \awbInNaturalLpaEqualsbEqvaMinusbEqualsZeroRp{6} \\
        1,2     &  (8)  & \multicolumn{3}{l}{c=a\land d=b} & \rAI{5,7} \\
        \end{array}
    \]
\end{proof}

\label{awbwcwdwfInNaturalwcLeqawbLeqdwLpfaPlusbEqualsfcPlusdEqvfLpaMinuscRpEqualsdMinusbRp}
\begin{theorem}[\(fa+b=fc+d \dashv\vdash f(a-c)=d-b\)]
Seien \(a,b,c,d,f\in\mathbb{N}\), \(c\leq a\) und \(b\leq d\). Dann gilt:
\[fa+b=fc+d \dashv\vdash f(a-c)=d-b.\]
\end{theorem}
\begin{proof}
Seien \(a,b,c,d,f\in\mathbb{N}\).
\(\vdash\):
    \[
	\begin{array}{llcll p{5cm}}
        1       &  (1)  & \multicolumn{3}{l}{c\leq a} & \rA \\
        2       &  (2)  & \multicolumn{3}{l}{b\leq d} & \rA \\
        3       &  (3)  & \multicolumn{3}{l}{fa+b=fc+d} & \rA \\
                &  (4)  & f(a-c)&=&fa-fc & \rLeftDistributiveSemigroup{} \\
        1,2,3   &  (5)  &       &=&d-b & \awbwcwdInNaturalwcLeqawbLeqdLpaPlusbEqualscPlusdEqvaMinuscEqualsdMinusbRp{1,2,3} \\
        1,2,3   &  (6)  &  f(a-c)&=&d-b & \rTransitivityEqRI{4,5} \\
        \end{array}
    \]
\(\dashv\):
    \[
	\begin{array}{llcll p{5cm}}
        1       &  (1)  & \multicolumn{3}{l}{c\leq a} & \rA \\
        2       &  (2)  & \multicolumn{3}{l}{b\leq d} & \rA \\
                &  (3)  & fa-fc&=&f(a-c) & \rLeftDistributiveSemigroup{} \\
        4       &  (4)  &       &=&d-b & \rA \\
        4       &  (5)  & fa-fc &=&d-b & \rTransitivityEqRI{3,4} \\
        1,2,4   &  (6)  & \multicolumn{3}{l}{fa+b=fc+d} & \awbwcwdInNaturalwcLeqawbLeqdLpaPlusbEqualscPlusdEqvaMinuscEqualsdMinusbRp{1,2,5} \\
        \end{array}
    \]
\end{proof}

\label{awbwcwdwfInNaturalwfLpaMinusbRpEqualsLpcMinusdRpwLpcMinusdRpLneqfImpaEqualsbAndcEqualsd}
\begin{theorem}[\(a,b,c,d,f\in\mathbb{N}, f(a-b)=(c-d), (c-d)<f\vdash a=b\land c=d\)]
\end{theorem}
\begin{proof}
Seien \(a,b,c,d,f\in\mathbb{N}\).
    \[
	\begin{array}{llcll p{5cm}}
        1       &  (1)  & \multicolumn{3}{l}{f(a-b)=(c-d)} & \rA \\
        2       &  (2)  & \multicolumn{3}{l}{(c-d)<f} & \rA \\
        1,2     &  (3)  & \multicolumn{3}{l}{f(a-b)<f} & \rIE{2,1} \\
        1,2     &  (4)  & \multicolumn{3}{l}{(a-b)<1} & \awbwcInNaturalwcNotEqualsZerowcaLeqcbImpaLeqb{3} \\
        1,2     &  (5)  & \multicolumn{3}{l}{a-b=0} & \aInNaturalLpaLneqOneEqvaEqualsZeroRp{4} \\
        1,2     &  (6)  & \multicolumn{3}{l}{a=b} & \awbInNaturalLpaEqualsbEqvaMinusbEqualsZeroRp{5} \\
        1     &  (7)  & c-d&=&f(a-b) & \rSymmetryEqRI{1} \\
        1,2   &  (8)  &    &=&f\cdot 0 & \rIE{5,7} \\
              &  (9)  &    &=&0 & \aInNaturalImpZeroEqualsZeroMulta{} \\
        1,2   &  (9)  & c-d&=&0 & \rTransitivityEqRI{7,9} \\
        1,2   &  (10) &\multicolumn{3}{l}{c=d} & \awbInNaturalLpaEqualsbEqvaMinusbEqualsZeroRp{9} \\
        1,2   &  (11) &\multicolumn{3}{l}{a=b\land c=d} & \rAI{6,10} \\
        \end{array}
    \]
\end{proof}


\label{awbInNaturalwbNotEqualsZerowExqSubOnewrSubOnewqSubTwowrSubTwoInNaturalLpaEqualsbMultqSubOnePlusrSubOneAndrSubOneLneqbAndaEqualsbMultqSubTwoPlusrSubTwoAndrSubTwoLneqbRpImpqSubOneEqualsqSubTwoAndrSubOneEqualsrSubTwo}
\begin{theorem}[
\(
a,b\in\mathbb{N},b\neq 0,\exists q_1, r_1, q_2, r_2 \in \mathbb{N} \,
\big(
a = b \cdot q_1 + r_1 \land r_1 < b \land 
\)
\newline
\(
a = b \cdot q_2 + r_2 \land r_2 < b
\big)
\vdash q_1 = q_2 \land r_1 = r_2
\)
(Eindeutigkeit der Division mit Rest)]
Seien \( a, b \in \mathbb{N} \) mit \( b \neq 0 \), und es gelte:
\[
a = b \cdot q_1 + r_1, \quad r_1 < b,
\]
sowie
\[
a = b \cdot q_2 + r_2, \quad r_2 < b,
\]
für \( q_1, r_1, q_2, r_2 \in \mathbb{N} \). Dann folgt, dass \( q_1 = q_2 \) und \( r_1 = r_2 \).
\end{theorem}
\begin{proof}
Seien \(a,b\in\mathbb{N}\). Wir wählen \( q_1, r_1, q_2, r_2 \in \mathbb{N} \) so, dass folgende Annahmen gelten. Im Beweis nutzen wir das Theorem \(\LeqIsTotalOrderOnNaturalNumbers{}\)
    \[
	\begin{array}{llcll p{5cm}}
        1       &  (1)  & \multicolumn{3}{l}{a=bq_1+r_1} & \rA \\
        2       &  (2)  & \multicolumn{3}{l}{r_1<b} & \rA \\
        3       &  (3)  & \multicolumn{3}{l}{a=bq_2+r_2} & \rA \\
        4       &  (4)  & \multicolumn{3}{l}{r_2<b} & \rA \\
        5       &  (5)  & \multicolumn{3}{l}{b\neq 0} & \rA \\
        1,3     &  (6)  & \multicolumn{3}{l}{bq_1+r_1=bq_2+r_2} & \rIE{1,3} \\
                &  (7)  & \multicolumn{3}{l}{r_1\leq r_2\lor r_2\leq r_1} & \rTotalityOrdRI{} \\
                &  (8)  & \multicolumn{3}{l}{q_1\leq q_2\lor q_2\leq q_1} & \rTotalityOrdRI{} \\
        9       &  (9)  & \multicolumn{3}{l}{r_1\leq r_2} & \rA \\
        10      &  (10)  & \multicolumn{3}{l}{q_1\leq q_2} & \rA \\
        1,3     &  (11)  & 0&=&b\cdot q_2+r_2-(b\cdot q_1+r_1) & \awbInNaturalLpaEqualsbEqvaMinusbEqualsZeroRp{6}  \\
        9,10    &  (12)  & &=& b(q_2-q_1)+(r_2-r_1) & \awbwcwdwfInNaturalwaLeqcwbLeqdImpfLpcMinusaRpPlusLpdMinusbRpEqualsLpfcPlusdRpMinusLpfaPlusbRp{10,9}  \\
        1,3,9,10    &  (13)  & 0 &=& b(q_2-q_1)+(r_2-r_1) & \rTransitivityEqRI{11,12}  \\
        1,3,5,9,10  &  (14)  & \multicolumn{3}{l}{q_1=q_2\land r_1=r_2} & \awbwcwdwfInNaturalwfNotEqualsZerowfLpcMinusaRpEqualsZeroAndLpdMinusbRpEqualsZeroImpcEqualsaAnddEqualsb{5,13}  \\
        15    &  (15)  & \multicolumn{3}{l}{q_2\leq q_1} & \rA  \\
        1,3    &  (16)  & \multicolumn{3}{l}{bq_2+r_2=bq_1+r_1} & \rSymmetryEqRI{6}  \\
        1,3,9,15  &  (17)  & \multicolumn{3}{l}{b(q_1-q_2)=r_2-r_1} & \awbwcwdwfInNaturalwcLeqawbLeqdwLpfaPlusbEqualsfcPlusdEqvfLpaMinuscRpEqualsdMinusbRp{9,15,16}  \\
        4,9  &  (18)  & \multicolumn{3}{l}{r_2-r_1<b} & \bLneqcwaLeqbImpbMinusaLneqc{4,9}  \\
        1,3,4,9,15  &  (19)  & \multicolumn{3}{l}{q_1=q_2\land r_2=r_1} & \awbwcwdwfInNaturalwfLpaMinusbRpEqualsLpcMinusdRpwLpcMinusdRpLneqfImpaEqualsbAndcEqualsd{17,19}  \\
        1,3,4,5,9  &  (20)  & \multicolumn{3}{l}{q_1=q_2\land r_2=r_1} & \rOE{8,10,14,15,20}  \\ 
        21  &  (21)  & \multicolumn{3}{l}{r_2\leq r_1} & \rA  \\ 
        22  &  (22)  & \multicolumn{3}{l}{q_1\leq q_2} & \rA  \\
        1,3,21,22  &  (23)  & \multicolumn{3}{l}{b(q_2-q_1)=r_1-r_2} & \awbwcwdwfInNaturalwcLeqawbLeqdwLpfaPlusbEqualsfcPlusdEqvfLpaMinuscRpEqualsdMinusbRp{22,21,16}  \\ 
        2,21  &  (24)  & \multicolumn{3}{l}{r_1-r_2<b} & \bLneqcwaLeqbImpbMinusaLneqc{2,21}  \\ 
        1,2,3,21,22  &  (25)  & \multicolumn{3}{l}{q_1=q_2\land r_2=r_1} & \awbwcwdwfInNaturalwfLpaMinusbRpEqualsLpcMinusdRpwLpcMinusdRpLneqfImpaEqualsbAndcEqualsd{23,24}  \\
        26  &  (26)  & \multicolumn{3}{l}{q_2\leq q_1} & \rA  \\
        1,3     &  (27)  & 0&=&b\cdot q_1+r_1-(b\cdot q_2+r_2) & \awbInNaturalLpaEqualsbEqvaMinusbEqualsZeroRp{6}  \\
        21,26    &  (28)  & &=& b(q_1-q_2)+(r_1-r_2) & \awbwcwdwfInNaturalwaLeqcwbLeqdImpfLpcMinusaRpPlusLpdMinusbRpEqualsLpfcPlusdRpMinusLpfaPlusbRp{21,26}  \\
        1,3,21,26    &  (29)  & 0 &=& b(q_2-q_1)+(r_2-r_1) & \rTransitivityEqRI{11,12}  \\
        1,3,5,21,26  &  (30)  & \multicolumn{3}{l}{q_1=q_2\land r_1=r_2} & \awbwcwdwfInNaturalwfNotEqualsZerowfLpcMinusaRpEqualsZeroAndLpdMinusbRpEqualsZeroImpcEqualsaAnddEqualsb{5,29}  \\
        1,2,3,5,21  &  (31)  & \multicolumn{3}{l}{q_1=q_2\land r_1=r_2} & \rOE{8,22,25,26,30}  \\
        1,2,3,4,5  &  (32)  & \multicolumn{3}{l}{q_1=q_2\land r_1=r_2} & \rOE{7,9,20,21,31}  \\
        \end{array}
    \]
\end{proof}

\section{Eindeutig bestimmte Zahlen der Division mit Rest}

In diesem Abschnitt seien \( a, b \in \mathbb{N} \).

\begin{definition}[Division mit Rest]
Seien \( b \neq 0 \). Die Division mit Rest liefert eindeutig bestimmte Zahlen \( a \div b \) und \( a \bmod b \), wobei gilt:
\[
a = b \cdot (a \div b) + (a \bmod b) \quad \text{und} \quad (a \bmod b) < b.
\]
Diese Zahlen sind definiert durch:
\[
a \div b \coloneqq \iota k \, (k \in \mathbb{N} \land a - b \cdot k \in \mathbb{N} \land a - b \cdot k < b)
\]
und
\[
a \bmod b \coloneqq a - b \cdot (a \div b).
\]
\end{definition}

\paragraph{Beweisregeln für die Division mit Rest}
\label{rule:rDivisionWithRemainderI} 

Basierend auf der Definition der Division mit Rest können folgende Regeln formuliert werden:

\[
\begin{array}{llll}
    i & (1) & b \neq 0 & ... \\
    i & (2) & a = b \cdot (a \div b) + (a \bmod b)  & \rDivisionWithRemainderI{1} \\
    i & (3) & (a \bmod b) < b & \rDivisionWithRemainderI{1} \\
    i & (4) & (a \bmod b)=a-b \cdot (a \div b) & \rDivisionWithRemainderI{1} \\
\end{array}
\]

\[
\begin{array}{llll}
    i & (1) & b \neq 0 & ... \\
    j & (2) & a = bq + r  & ... \\
    i & (3) & a \bmod b = r & \rDivisionWithRemainderI{1,2} \\
    i & (4) & a \div b=q & \rDivisionWithRemainderI{1,2} \\
\end{array}
\]

Dabei sind \( i \) und \( j \) Listen von Annahmen.


\subsection{Grundlegende Eigenschaften}



\subsubsection{Division mit Rest bei kleinerem Dividend}
\label{awbInNaturalwbNotEqualsZerowaLneqbImpaDivbEqualsZero}
\begin{theorem}[\(a,b\in\mathbb{N},b\neq 0, a<b\vdash a\div b=0\)]
\end{theorem}
\begin{proof}
    \[
	\begin{array}{llcll p{5cm}}
            1 &  (1)  & \multicolumn{3}{l}{a<b} & \rA \\
            2 &  (2)  & \multicolumn{3}{l}{b\neq 0} & \rA \\
            3 &  (3)  & \multicolumn{3}{l}{a\div b\neq 0} & \rA \\
            2 &  (4)  & \multicolumn{3}{l}{ a = b \cdot (a \div b) + (a\bmod b)} & \rDivisionWithRemainderI{2} \\
            3 &  (5)  & a&\leq &a\cdot (a \div b)   & \awbInNaturalwbNotEqualsZeroImpaLeqab{3} \\
            3 &  (6)  &  &< &b\cdot (a \div b)   & \awbwcInNaturalwcNotEqualsZerowacLneqbcEqvaLneqb{1} \\
            3 &  (7)  &  &< &b\cdot (a \div b)   & \awbwcInNaturalwcNotEqualsZerowacLneqbcEqvaLneqb{1} \\
            1 &  (8)  &  &\leq &b\cdot (a \div b)+(a\bmod b)   & \awbInNaturalImpaLeqaPlusb{} \\
            1,3 &  (9)  & a &< &b\cdot (a \div b)+(a\bmod b)   & \rTransitivityOrdRI{5,8} \\
            1,3 &  (10)  & \multicolumn{3}{l}{a\neq b\cdot (a \div b)+(a\bmod b)}   & \rStrictOrderRelationI{9} \\
            1,2,3 &  (11)  & \multicolumn{3}{l}{\bot}   & \rBI{4,10} \\
            1,2 &  (12)  & \multicolumn{3}{l}{a\div b=0}   & \rCE{3,11} \\
        \end{array}
    \]
\end{proof}

\label{awbInNaturalwbNotEqualsZerowaLneqbImpaEqualsaModb}
\begin{theorem}[\(a,b\in\mathbb{N},b\neq 0, a<b\vdash a\bmod b=a\)]
\end{theorem}
\begin{proof}
\(\ImpLpNaturalwPluswZeroRpInMonoid{}\) und daher gilt:
    \[
	\begin{array}{llcll p{5cm}}
            1 &  (1)  & \multicolumn{3}{l}{a<b} & \rA \\
            2 &  (2)  & \multicolumn{3}{l}{b\neq 0} & \rA \\
            2 &  (3)  & \multicolumn{3}{l}{ a = b \cdot (a \div b) + (a\bmod b)} & \rDivisionWithRemainderI{2} \\
            1,2 &  (4)  & \multicolumn{3}{l}{(a \div b)=0} & \awbInNaturalwbNotEqualsZerowaLneqbImpaDivbEqualsZero{2,1} \\
            2 &  (5)  & a &=& b \cdot (a \div b) + (a\bmod b) & \rDivisionWithRemainderI{2} \\
            1,2 &  (6)  &   &=& b \cdot 0 + (a\bmod b) & \rIE{4,5} \\
              &  (7)  &   &=& 0 + (a\bmod b) & \aInNaturalImpZeroEqualsZeroMulta{} \\
              &  (8)  &   &=& a\bmod b & \rNeutralElementMonoid{} \\
            1,2 &  (9)  &  a &=& a\bmod b & \rTransitivityEqRI{5,8} \\
            1,2 &  (10)  &  a\bmod b &=& a & \rSymmetryEqRI{9} \\
        \end{array}
    \]
\end{proof}


\label{awbInNaturalwbNotEqualsZeroLpaLneqbEqvaDivbEqualsZeroAndaEqualsaModbRp}
\begin{theorem}[\(a<b\dashv\vdash a\div b=0\land  a\bmod b=a\)]
Seien \(a,b\in\mathbb{N}\) und \(b\neq 0\), dann gilt
\[a<b\dashv\vdash a\div b=0\land  a=a\bmod b\]
\end{theorem}
\begin{proof}
\(\vdash\):
    \[
	\begin{array}{llcll p{5cm}}
            1 &  (1)  & \multicolumn{3}{l}{a<b} & \rA \\
            2 &  (2)  & \multicolumn{3}{l}{b\neq 0} & \rA \\
            1,2 &  (3)  & \multicolumn{3}{l}{ a\bmod b=a} & \awbInNaturalwbNotEqualsZerowaLneqbImpaEqualsaModb{2,1} \\
            1,2 &  (4)  & \multicolumn{3}{l}{a\div b=0} & \awbInNaturalwbNotEqualsZerowaLneqbImpaDivbEqualsZero{2,1} \\
            1,2 &  (5)  &  \multicolumn{3}{l}{a\div b=0\land a\bmod b=a} & \rAI{4,3} \\
        \end{array}
    \]
\(\dashv\):
    \[
	\begin{array}{llcll p{5cm}}
            1 &  (1)  & \multicolumn{3}{l}{b\neq 0} & \rA \\
            2 &  (2)  & \multicolumn{3}{l}{ a\div b=0\land a\bmod b=a} & \rA \\
            2 &  (4)  & \multicolumn{3}{l}{ a\bmod b=a} & \rAEb{2} \\
            1 &  (5)  & \multicolumn{3}{l}{a\bmod b<b} & \rDivisionWithRemainderI{1} \\
            1,2 &  (6)  & \multicolumn{3}{l}{a<b} & \rIE{4,5} \\
        \end{array}
    \]
\end{proof}

\subsubsection{Ordnungsrelationen bei der Division}

\label{awbInNaturalwbNotEqualsZeroImpaDivbLeqa}
\begin{theorem}[\(a,b\in\mathbb{N}, b\neq 0\vdash a\div b\leq a\)]
\end{theorem}
\begin{proof}
Seien \(a,b\in\mathbb{N}\), dann gilt:
    \[
	\begin{array}{llcll p{4.5cm}}
             1 &  (1)  & \multicolumn{3}{l}{b\neq 0} & \rA \\
             1 &  (2)  & a\div b&\leq& b\cdot (a\div b) & \awbInNaturalwbNotEqualsZeroImpaLeqba{1} \\
               &  (3)  &     &\leq& b\cdot (a\div b)+(a\bmod b) & \awbInNaturalImpaLeqaPlusb{} \\
             1 &  (4)  &     &=& a & \rDivisionWithRemainderI{1} \\
             1 &  (5)  & a\div b &\leq & a & \rTransitivityOrdRI{2,4} \\
        \end{array}
    \]
\end{proof}

\label{awbInNaturalwbGneqOneImpaDivbLneqa}
\begin{theorem}[\(a,b\in\mathbb{N}, b > 1\vdash a\div b<a\)]
\end{theorem}
\begin{proof}
Seien \(a,b\in\mathbb{N}\), dann gilt:
    \[
	\begin{array}{llcll p{4.5cm}}
             1 &  (1)  & \multicolumn{3}{l}{b > 1} & \rA \\
             1 &  (2)  & \multicolumn{3}{l}{b \neq 0} & \aInNaturalwaGneqOneImpaNotEqualsZero{1} \\
             1 &  (3)  & a\div b&<& b\cdot (a\div b) & \awbInNaturalwbGneqOneImpaLneqba{1} \\
               &  (4)  &     &\leq& b\cdot (a\div b)+(a\bmod b) & \awbInNaturalImpaLeqaPlusb{} \\
             1 &  (5)  &     &=& a & \rDivisionWithRemainderI{2} \\
             1 &  (6)  & a\div b &< & a & \rTransitivityOrdRI{2,5} \\
        \end{array}
    \]
\end{proof}

\subsubsection{Teilbarkeit und Rest}

\label{awbInNaturalwaNotEqualsZerowaMidbEqvbDivaEqualsLpbRpDurchLpaRpAndbModaEqualsZero}
\begin{theorem}[\(a\mid b\dashv\vdash b\div a=\frac{b}{a}\land b\bmod a=0\)]
Seien \(a,b\in\mathbb{N}\) und \(a\neq 0\), dann gilt
\[a\mid b\dashv\vdash b\div a=\frac{b}{a}\land b\bmod a=0\]
\end{theorem}
\begin{proof}
\(\ImpLpNaturalwPluswZeroRpInAbelMonoid{}\) und daher gilt:
\(\vdash\):
    \[
	\begin{array}{llcll p{5cm}}
            1 &  (1)  & \multicolumn{3}{l}{a\mid b} & \rA \\
            1 &  (2)  & b&=&a\cdot\frac{b}{a} & \rDivisionI{1} \\
            1 &  (3)  &  &=&a\cdot\frac{b}{a}+0 & \rNeutralElementMonoid{} \\
            1 &  (4)  & b &=&a\cdot\frac{b}{a}+0 & \rTransitivityEqRI{2,3} \\
            1 &  (5)  &  \multicolumn{3}{l}{a\neq 0} & \rDivisibilityE{1} \\
            1,2 &  (6)  & \multicolumn{3}{l}{b\div a=\frac{b}{a}} & \rDivisionWithRemainderI{5,4} \\
            1,2 &  (7)  & \multicolumn{3}{l}{b\bmod a=0} & \rDivisionWithRemainderI{5,4} \\
            1,2 &  (8)  &  \multicolumn{3}{l}{b\div a=\frac{b}{a}\land a\bmod b=0} & \rAI{6,7} \\
        \end{array}
    \]
\(\dashv\):
    \[
	\begin{array}{llcll p{5cm}}
            1 &  (1)  & \multicolumn{3}{l}{a\neq 0} & \rA \\
            2 &  (2)  & \multicolumn{3}{l}{ b\div a=\frac{b}{a}\land b\bmod a=0} & \rA \\
            2 &  (3)  & \multicolumn{3}{l}{ b\div a=\frac{b}{a}} & \rAEa{2} \\
            2 &  (4)  & \multicolumn{3}{l}{ b\bmod a=0} & \rAEb{2} \\
            1 &  (5)  & b&=&a\cdot (b\div a)+ (b\bmod a) & \rDivisionWithRemainderI{1} \\
            1,2 &  (6)  &  &=&a\cdot (b\div a)+ 0 & \rIE{4,5} \\
            1 &  (7)  &  &=&a\cdot (b\div a) & \rNeutralElementMonoid{} \\
            1,2 &  (8)  &  &=&a\cdot \frac{b}{a} & \rIE{3,7} \\
            1,2 &  (9)  &  b&=&a\cdot \frac{b}{a} & \rTransitivityEqRI{5,8} \\
            1,2 &  (10)  &  b&=&a\cdot \frac{b}{a} & \rDivisibilityI{1,9} \\
        \end{array}
    \]
\end{proof}

\label{awbInNaturalwaMidbImpbDivaEqualsLpbRpDurchLpaRp}
\begin{theorem}[\(a,b\in\mathbb{N},a\mid b\vdash b\div a=\frac{b}{a}\)]
\end{theorem}
\begin{proof}
    \[
	\begin{array}{llcll p{5cm}}
            1 &  (1)  & \multicolumn{3}{l}{a\mid b} & \rA \\
            1 &  (2)  & \multicolumn{3}{l}{b\neq 0} & \rDivisibilityE{1} \\
            1 &  (3)  & \multicolumn{3}{l}{b\div a=\frac{b}{a}\land b\bmod a=0} & \awbInNaturalwaNotEqualsZerowaMidbEqvbDivaEqualsLpbRpDurchLpaRpAndbModaEqualsZero{1,2} \\
            1 &  (4)  & \multicolumn{3}{l}{b\div a=\frac{b}{a}} & \rAEa{3} \\
        \end{array}
    \]
\end{proof}

\label{awbInNaturalwaMidbImpbModaEqualsZero}
\begin{theorem}[\(a,b\in\mathbb{N},a\mid b\vdash b\bmod a=0\)]
\end{theorem}
\begin{proof}
    \[
	\begin{array}{llcll p{5cm}}
            1 &  (1)  & \multicolumn{3}{l}{a\mid b} & \rA \\
            1 &  (2)  & \multicolumn{3}{l}{a\neq 0} & \rA \\
            1 &  (3)  & \multicolumn{3}{l}{b\div a=\frac{b}{a}\land b\bmod a=0} & \awbInNaturalwaNotEqualsZerowaMidbEqvbDivaEqualsLpbRpDurchLpaRpAndbModaEqualsZero{1,2} \\
            1 &  (4)  & \multicolumn{3}{l}{b\bmod a=0} & \rAEb{3} \\
        \end{array}
    \]
\end{proof}

\subsubsection{Invarianz des Restes bei Vielfachen des Divisors}

\label{awbwkInNaturalwbNotEqualsZeroImpLpaPlusbkRpModbEqualsaModb}
\begin{theorem}[\(a,b,k\in\mathbb{N},b\neq 0\vdash (a+bk)\bmod b=a\bmod b\)]
\end{theorem}
\begin{proof}
Seien \(a,b,k\in\mathbb{N}\). \(\ImpLpNaturalwMultwOneRpInAbelSemiRing{}\) und daher gilt:
    \[
	\begin{array}{llcll p{5cm}}
            1 &  (1)  & \multicolumn{3}{l}{b\neq 0} & \rA \\
            1 &  (2)  & \multicolumn{3}{l}{a=b\cdot (a\div b)+(a\bmod b)} & \rDivisionWithRemainderI{1} \\
            1 &  (3)  & a+bk&=&(b\cdot (a\div b)+(a\bmod b))+bk & \rIE{2,3} \\
              &  (4)  &     &=&(b\cdot (a\div b)+bk)+(a\bmod b) & \aInMwbInMwcInMImpLpaPlusbRpPluscEqualsLpaPluscRpPlusb{} \\
              &  (5)  &     &=&(b\cdot ((a\div b)+k)+(a\bmod b) & \rLeftDistributiveAbelianSemigroup{} \\
            1 &  (6)  & a+bk &=&(b\cdot ((a\div b)+k)+(a\bmod b) & \rTransitivityEqRI{3,5} \\
            1 &  (7)  & (a+bk)\bmod b &=&a\bmod b & \rDivisionWithRemainderI{1,6} \\
        \end{array}
    \]
\end{proof}

\label{awbwkInNaturalwbNotEqualsZeroImpLpaPlusbkRpDivbEqualsLpaDivbRpPlusk}
\begin{theorem}[\(a,b,k\in\mathbb{N},b\neq 0\vdash (a+bk)\div b=(a\div b)+k\)]
\end{theorem}
\begin{proof}
Seien \(a,b,k\in\mathbb{N}\). \(\ImpLpNaturalwMultwOneRpInAbelSemiRing{}\) und daher gilt:
    \[
	\begin{array}{llcll p{5cm}}
            1 &  (1)  & \multicolumn{3}{l}{b\neq 0} & \rA \\
            1 &  (2)  & \multicolumn{3}{l}{a=b\cdot (a\div b)+(a\bmod b)} & \rDivisionWithRemainderI{1} \\
            1 &  (3)  & a+bk&=&(b\cdot (a\div b)+(a\bmod b))+bk & \rIE{2,3} \\
              &  (4)  &     &=&(b\cdot (a\div b)+bk)+(a\bmod b) & \aInMwbInMwcInMImpLpaPlusbRpPluscEqualsLpaPluscRpPlusb{} \\
              &  (5)  &     &=&(b\cdot ((a\div b)+k)+(a\bmod b) & \rLeftDistributiveAbelianSemigroup{} \\
            1 &  (6)  & a+bk &=&(b\cdot ((a\div b)+k)+(a\bmod b) & \rTransitivityEqRI{3,5} \\
            1 &  (7)  & (a+bk)\div b &=&(a\div b)+k & \rDivisionWithRemainderI{1,6} \\
        \end{array}
    \]
\end{proof}

\chapter{Summen- und Produktzeichen}

\begin{definition}[Summe über eine endliche Indexmenge]
Sei \(\bigl(S,+,0\bigr)\) ein abelsches Monoid. Für eine endliche Indexmenge \(I\) mit \(\lvert I \rvert = n\) und eine Familie \(\{s_i\}_{i \in I}\subseteq S\) definieren wir die \emph{Summe} 
\[
  \sum_{i \in I} s_i
\]
durch folgende Schritte:
\begin{itemize}
    \item Falls \(I = \varnothing\), setzen wir
    \[
       \sum_{i \in \varnothing} s_i := 0.
    \]
    \item Falls \(I = \{i_1,i_2,\dots,i_n\}\) mit \(n \ge 1\), so wählen wir eine beliebige Anordnung 
    \[
      i_1, i_2, \dots, i_n
    \]
    der Indizes und definieren rekursiv:
    \[
      \sum_{k=1}^1 s_{i_k} \;:=\; s_{i_1}, 
      \quad\text{und}\quad
      \sum_{k=1}^{m+1} s_{i_k} \;:=\; \Bigl(\sum_{k=1}^m s_{i_k}\Bigr) \;+\; s_{i_{m+1}}
      \quad \text{für } m \in \{1,\dots,n-1\}.
    \]
    Wegen der Kommutativität und Assoziativität der Addition in \(S\) ist diese Definition unabhängig von der gewählten Reihenfolge.
\end{itemize}
\end{definition}

???

\paragraph{Wohldefiniertheit von \(\text{Sum}_I\)}
\begin{theorem}[Wohldefiniertheit der Summenabbildung]
Sei \((S, +, 0)\) ein abelscher Monoid, \(I\) eine endliche Indexmenge und \(a \colon I \to S\) eine Familie. Dann gilt:
\[
\forall I \, (\text{endlich}(I) \land a \colon I \to S \implies \exists! b \in S \, (\Sigma_I(a) = b)).
\]
\end{theorem}
\begin{proof}
Wir zeigen die Aussage durch vollständige Induktion über die Anzahl der Elemente der Indexmenge \(I\), also \(|I|\).

\paragraph{Basisfall: \(|I| = 0\)}
Falls \(I = \varnothing\), ist die Summenabbildung \(\Sigma_I(a)\) durch Definition gegeben als:
\[
\Sigma_I(a) := 0.
\]
Da \(0\) das neutrale Element des abelschen Monoids \((S, +, 0)\) ist, existiert genau ein \(b \in S\) mit \(\Sigma_I(a) = b\). Damit ist die Aussage für den Basisfall gezeigt.

\paragraph{Induktionsschritt: \(|I| = n+1\)}
Angenommen, die Aussage gilt für alle endlichen Indexmengen \(I'\) mit \(|I'| = n\). Sei nun \(I\) eine Indexmenge mit \(|I| = n+1\), und sei \(a \colon I \to S\) eine Familie. Nach Definition gilt:
\[
\Sigma_I(a) = a(j) + \Sigma_{I \setminus \{j\}}(a\restriction_{I \setminus \{j\}}),
\]
wobei \(j \in I\) ein beliebiges Element der Indexmenge ist und \(a\restriction_{I \setminus \{j\}}\) die Einschränkung der Familie \(a\) auf die Menge \(I \setminus \{j\}\) bezeichnet.

Es bleibt zu zeigen, dass \(\Sigma_I(a)\) unabhängig von der Wahl von \(j \in I\) ist. Sei hierzu \(j, k \in I\) mit \(j \neq k\). Dann gilt:
\[
\Sigma_I(a) = a(j) + \Sigma_{I \setminus \{j\}}(a\restriction_{I \setminus \{j\}})
\quad \text{und} \quad
\Sigma_I(a) = a(k) + \Sigma_{I \setminus \{k\}}(a\restriction_{I \setminus \{k\}}).
\]

Nach Induktionsannahme ist die Summenabbildung \(\Sigma_{I \setminus \{j\}}\) für die Menge \(I \setminus \{j\}\) wohldefiniert. Insbesondere gilt, dass für jede beliebige Permutation der Elemente von \(I \setminus \{j\}\) das Ergebnis der Summation gleich bleibt. Dasselbe gilt für \(\Sigma_{I \setminus \{k\}}\).

Die rekursive Definition und die Eigenschaften der abelschen Monoidstruktur (\(+\) ist kommutativ und assoziativ) gewährleisten, dass die Reihenfolge der Addition keinen Einfluss hat. Somit gilt:
\[
\Sigma_{I \setminus \{j\}}(a\restriction_{I \setminus \{j\}}) = \Sigma_{I \setminus \{k\}}(a\restriction_{I \setminus \{k\}}).
\]

Einsetzen in die Ausdrücke für \(\Sigma_I(a)\) ergibt:
\[
a(j) + \Sigma_{I \setminus \{j\}}(a\restriction_{I \setminus \{j\}})
= a(k) + \Sigma_{I \setminus \{k\}}(a\restriction_{I \setminus \{k\}}).
\]

Damit ist gezeigt, dass \(\Sigma_I(a)\) unabhängig von der Wahl von \(j \in I\) ist.

\paragraph{Eindeutigkeit}
Die Eindeutigkeit von \(\Sigma_I(a)\) folgt unmittelbar aus der rekursiven Definition. In jedem Schritt wird ein Element \(a(j)\) addiert, und die verbleibende Summe wird eindeutig durch die Induktionsannahme bestimmt. Somit existiert genau ein \(b \in S\), sodass \(\Sigma_I(a) = b\).

\paragraph{Abschließende Folgerung}
Da sowohl die Existenz als auch die Eindeutigkeit von \(\Sigma_I(a)\) gezeigt wurden, ist die Summenabbildung \(\Sigma_I(a)\) wohldefiniert.
\end{proof}


?????
\section{Induktive Definition von Summen- und Produktzeichen}
\begin{definition}[Summenzeichen und Produktzeichen, induktiv]
    Sei \((S, \ast)\) eine Halbgruppe, d.h., \(S\) ist eine Menge und \(\ast\) eine assoziative binäre Operation \(\ast: S \times S \to S\). Zur Vereinfachung der Notation wird die Operation \(\ast\) konventionell als \enquote{Addition} oder \enquote{Multiplikation} bezeichnet und erhält entsprechend das Symbol \(+\) oder \(\cdot\).

    \begin{itemize}
        \item \textbf{Addition} (Symbol \(+\)): Falls die Operation \(\ast\) als Addition bezeichnet wird und das Symbol \(+\) erhält, definieren wir das Summenzeichen für die Elemente \(a_{i_0}, \dots, a_{i_0+n} \in S\) induktiv durch:
        \[
        \sum_{i=i_0}^{i_0} a_i := a_{i_0},
        \]
        und für einen Übergang von \(n\) auf \(n+1\):
        \[
        \sum_{i=i_0}^{n+1} a_i := \sum_{i=i_0}^n a_i + a_{n+1}.
        \]

        \item \textbf{Multiplikation} (Symbol \(\cdot\)): Falls die Operation \(\ast\) als Multiplikation bezeichnet wird und das Symbol \(\cdot\) erhält, definieren wir das Produktzeichen für die Elemente \(a_{i_0}, \dots, a_{i_0+n} \in S\) induktiv durch:
        \[
        \prod_{i=i_0}^{i_0} a_i := a_{i_0},
        \]
        und für einen Übergang von \(n\) auf \(n+1\):
        \[
        \prod_{i=i_0}^{n+1} a_i := \prod_{i=i_0}^n a_i \cdot a_{n+1}.
        \]
    \end{itemize}

    Diese Definitionen gelten für beliebige assoziative Strukturen, in denen eine binäre Operation \(\ast\) als Addition oder Multiplikation festgelegt wird. Die spezifische Zuweisung der Symbole \(+\) und \(\cdot\) erleichtert die Notation und Interpretation in der jeweiligen Struktur.
\end{definition}

\subsubsection{Einführungsregeln für Summen- und Produktzeichen}
\label{rule:sumIntro} \label{rule:prodIntro}

\paragraph{Einführungsregel für das Summenzeichen \(\sum_{i=i_0}^n a_i\)}
Falls die Operation \(\ast\) als Addition bezeichnet wird, gilt für \(a_{i_0}, \dots, a_n \in S\):
\[
\begin{array}{llll}
    i   & (1) & a_1,...,a_n \in S & \dots \\
    j   & (2) & i_0 < n & \dots \\
    i,j   & (3) & \sum_{i=i_0}^n a_i = \sum_{i=i_0}^{n-1} a_i + a_n & \rSumI{1,2} \\
\end{array}
\]

\[
\begin{array}{llll}
    i   & (1) & a_1,...,a_n \in S & \dots \\
    j   & (2) & i_0 < n & \dots \\
    i,j   & (3) & \sum_{i=i_0}^n a_i = a_{i_0}+\sum_{i=i_0+1}^{n} a_i  & \rSumI{1,2} 
\end{array}
\]

Falls die Operation \(\ast\) als Addition bezeichnet wird, gilt für \(a_{i_0} \in S\):
\[
\begin{array}{llll}
    i & (1) & a_{i_0} \in S & \dots \\
    i & (2) & \sum_{i=i_0}^{i_0} a_i = a_{i_0} & \rSumI{1}
\end{array}
\]

\(i\) und \(j\) sind dabei Listen von Annahmen.

\paragraph{Einführungsregel für das Produktzeichen \(\prod_{i=i_0}^n a_i\)}
Falls die Operation \(\ast\) als Multiplikation bezeichnet wird, gilt für \(a_{i_0}, \dots, a_n \in S\):
\[
\begin{array}{llll}
    i   & (1) & a_n \in S & \dots \\
    j   & (2) & i_0 < n & \dots \\
    i,j & (3) & \prod_{i=i_0}^n a_i = \prod_{i=i_0}^{n-1} a_i \cdot a_n & \rProdI{1,2}
\end{array}
\]

\[
\begin{array}{llll}
    i   & (1) & a_n \in S & \dots \\
    j   & (2) & i_0 < n & \dots \\
    i,j & (3) & \prod_{i=i_0}^n a_i = a_{i_0}\cdot \prod_{i=i_0+1}^{n-1} a_i & \rProdI{1,2}
\end{array}
\]

Falls die Operation \(\ast\) als Multiplikation bezeichnet wird, gilt für \(a_{i_0} \in S\):
\[
\begin{array}{llll}
    i & (1) & a_{i_0} \in S & \dots \\
    i & (2) & \prod_{i=i_0}^{i_0} a_i = a_{i_0} & \rProdI{1}
\end{array}
\]

\(i\) und \(j\) sind dabei Listen von Annahmen.

\chapter{Endliche Summen natürlicher Zahlen}

\section{Linksdistributivität endlicher Summen}

\label{awsSubLbiSubZeroRbInNaturalImpaSumSubLbiEqualsiSubZeroRbPowerLbiSubZeroRbsSubiEqualsSumSubLbiEqualsiSubZeroRbPowerLbiSubZeroRbasSubi}
\begin{lemma}[\(a,s_{i_0}\in\mathbb{N}\vdash a\sum_{i=i_0}^{i_0} s_i=\sum_{i=i_0}^{i_0} as_i\) (Induktionsanfang)]
\end{lemma}
\begin{proof}
    Seien \(a,s_{i_0}\in\mathbb{N}\), dann gilt:
    \[
	\begin{array}{llcll p{5cm}}
              &  (1)  & a\sum_{i=i_0}^{i_0} s_i &=& as_{i_0} & \rSumI{} \\
              &  (2)  &  &=& \sum_{i=i_0}^{i_0} as_i & \rSumI{} \\
              &  (3)  &  a\sum_{i=i_0}^{i_0} s_i&=& \sum_{i=i_0}^{i_0} as_i & \rTransitivityEqRI{1,2} \\
        \end{array}
    \]
\end{proof}

\label{awnwsSubLbiSubZeroRbwDotswsSubLbnPlusOneRbInNaturalwaSumSubLbiEqualsiSubZeroRbPowerLbnRbsSubiEqualsSumSubLbiEqualsiSubZeroRbPowerLbnRbasSubiImpaSumSubLbiEqualsiSubZeroRbPowerLbnPlusOneRbsSubiEqualsSumSubLbiEqualsiSubZeroRbPowerLbnPlusOneRbasSubi}
\begin{lemma}[\(a,n,s_{i_0},\dots,s_{n+1}\in\mathbb{N},a\sum_{i=i_0}^{n} s_i=\sum_{i=i_0}^{n} as_i\vdash a\sum_{i=i_0}^{n+1} s_i=\sum_{i=i_0}^{n+1} as_i\) (Induktionsschritt)]
\end{lemma}
\begin{proof}
    Seien \(a,n,s_{i_0},\dots,s_{n+1}\in\mathbb{N}\). \(\ImpLpNaturalwMultwOneRpInAbelSemiRing{}\) und daher gilt:
    \[
	\begin{array}{llcll p{5cm}}
            1 &  (1)  & \multicolumn{3}{l}{a\sum_{i=i_0}^{n} s_i=\sum_{i=i_0}^{n} as_i} & \rA \\
              &  (2)  & a\sum_{i=i_0}^{n+1} s_i &=& a(\sum_{i=i_0}^{n} s_i+s_{n+1}) & \rSumI{} \\
              &  (3)  &  &=& a\sum_{i=i_0}^{n} s_i+as_{n+1} & \rLeftDistributiveAbelianSemigroup{} \\
            1 &  (4)  &  &=& \sum_{i=i_0}^{n} as_i+as_{n+1} & \rIE{1,3} \\
              &  (5)  &  &=& \sum_{i=i_0}^{n+1} as_i & \rSumI{} \\
            1 &  (6)  & a\sum_{i=i_0}^{n+1} s_i &=& \sum_{i=i_0}^{n+1} as_i & \rTransitivityEqRI{2,5} \\
        \end{array}
    \]
\end{proof}

\label{awnInNaturalwsSubLbiSubZeroRbwDotswsSubnInNaturalImpaSumSubLbiEqualsiSubZeroRbPowernsSubiEqualsSumSubLbiEqualsiSubZeroRbPowernasSubi}
\begin{theorem}[\(a,n\in\mathbb{N},s_{i_0},\dots, s_n\in\mathbb{N}\vdash a\sum_{i=i_0}^n s_i=\sum_{i=i_0}^n as_i\)]
\end{theorem}
\begin{proof}
    Seien \(a,s_{i_0},\dots,s_{n}\in\mathbb{N}\), dann gilt:
    \[
	\begin{array}{llcll p{5cm}}
               &  (1)  & \multicolumn{3}{l}{a\sum_{i=i_0}^{i_0} s_i=\sum_{i=i_0}^{i_0} as_i} & \awsSubLbiSubZeroRbInNaturalImpaSumSubLbiEqualsiSubZeroRbPowerLbiSubZeroRbsSubiEqualsSumSubLbiEqualsiSubZeroRbPowerLbiSubZeroRbasSubi{} \\
             2 &  (2)  & \multicolumn{3}{l}{n\in\mathbb{N}} & \rA \\
             3 &  (3)  & \multicolumn{3}{l}{a\sum_{i=i_0}^{n} s_i=\sum_{i=i_0}^{n} as_i} & \rA \\ 
             2,3 &  (4)  & \multicolumn{3}{l}{a\sum_{i=i_0}^{n+1} s_i=\sum_{i=i_0}^{n+1} as_i} & \awnwsSubLbiSubZeroRbwDotswsSubLbnPlusOneRbInNaturalwaSumSubLbiEqualsiSubZeroRbPowerLbnRbsSubiEqualsSumSubLbiEqualsiSubZeroRbPowerLbnRbasSubiImpaSumSubLbiEqualsiSubZeroRbPowerLbnPlusOneRbsSubiEqualsSumSubLbiEqualsiSubZeroRbPowerLbnPlusOneRbasSubi{2,3} \\ 
                &  (5)  & \multicolumn{3}{l}{\forall n\in\mathbb{N}(a\sum_{i=i_0}^{n} s_i=\sum_{i=i_0}^{n} as_i)} & \rInductionN{1,2,3,4} \\ 
                &  (6)  & \multicolumn{3}{l}{n\in\mathbb{N}\rightarrow a\sum_{i=i_0}^{n} s_i=\sum_{i=i_0}^{n} as_i} & \rSetEEb{5} \\ 
            2&  (7)  & \multicolumn{3}{l}{a\sum_{i=i_0}^{n} s_i=\sum_{i=i_0}^{n} as_i} & \rRE{2,6} \\ 
        \end{array}
    \]
\end{proof}

\section{Rechtssdistributivität endlicher Summen}

\label{awsSubLbiSubZeroRbInNaturalImpLpSumSubLbiEqualsiSubZeroRbPowerLbiSubZeroRbsSubiRpaEqualsSumSubLbiEqualsiSubZeroRbPowerLbiSubZeroRbsSubia}
\begin{lemma}[\(a,s_{i_0}\in\mathbb{N}\vdash (\sum_{i=i_0}^{i_0} s_i)a=\sum_{i=i_0}^{i_0} s_ia\) (Induktionsanfang)]
\end{lemma}
\begin{proof}
    Seien \(a,s_{i_0}\in\mathbb{N}\), dann gilt:
    \[
	\begin{array}{llcll p{5cm}}
              &  (1)  & (\sum_{i=i_0}^{i_0} s_i)a &=& s_{i_0}a & \rSumI{} \\
              &  (2)  &  &=& \sum_{i=i_0}^{i_0} s_ia & \rSumI{} \\
              &  (3)  &  (\sum_{i=i_0}^{i_0} s_i)a&=& \sum_{i=i_0}^{i_0} s_ia & \rTransitivityEqRI{1,2} \\
        \end{array}
    \]
\end{proof}

\label{awnwsSubLbiSubZeroRbwDotswsSubLbnPlusOneRbInNaturalwLpSumSubLbiEqualsiSubZeroRbPowerLbnRbsSubiRpaEqualsSumSubLbiEqualsiSubZeroRbPowerLbnRbsSubiaImpLpSumSubLbiEqualsiSubZeroRbPowerLbnPlusOneRbsSubiRpaEqualsSumSubLbiEqualsiSubZeroRbPowerLbnPlusOneRbsSubia}
\begin{lemma}[\(a,n,s_{i_0},\dots,s_{n+1}\in\mathbb{N},(\sum_{i=i_0}^{n} s_i)a=\sum_{i=i_0}^{n} s_ia\vdash (\sum_{i=i_0}^{n+1} s_i)a=\sum_{i=i_0}^{n+1} s_ia\) (Induktionsschritt)]
\end{lemma}
\begin{proof}
    Seien \(a,n,s_{i_0},\dots,s_{n+1}\in\mathbb{N}\). \(\ImpLpNaturalwMultwOneRpInAbelSemiRing{}\) und daher gilt:
    \[
	\begin{array}{llcll p{5cm}}
            1 &  (1)  & \multicolumn{3}{l}{(\sum_{i=i_0}^{n} s_i)a=\sum_{i=i_0}^{n} s_ia} & \rA \\
              &  (2)  & (\sum_{i=i_0}^{n+1} s_i)a &=& (\sum_{i=i_0}^{n} s_i+s_{n+1})a & \rSumI{} \\
              &  (3)  &  &=& (\sum_{i=i_0}^{n} s_i)a+s_{n+1}a & \rRightDistributiveAbelianSemigroup{} \\
            1 &  (4)  &  &=& \sum_{i=i_0}^{n} s_ia+s_{n+1}a & \rIE{1,3} \\
              &  (5)  &  &=& \sum_{i=i_0}^{n+1} s_ia & \rSumI{} \\
            1 &  (6)  & (\sum_{i=i_0}^{n+1} s_i)a &=& \sum_{i=i_0}^{n+1} s_ia & \rTransitivityEqRI{2,5} \\
        \end{array}
    \]
\end{proof}

\label{awnInNaturalwsSubLbiSubZeroRbwDotswsSubnInNaturalImpLpSumSubLbiEqualsiSubZeroRbPowernsSubiRpaEqualsSumSubLbiEqualsiSubZeroRbPowernsSubia}
\begin{theorem}[\(a,n\in\mathbb{N},s_{i_0},\dots, s_n\in\mathbb{N}\vdash (\sum_{i=i_0}^n s_i)a=\sum_{i=i_0}^n s_ia\)]
\end{theorem}
\begin{proof}
    Seien \(a,s_{i_0},\dots,s_{n}\in\mathbb{N}\), dann gilt:
    \[
	\begin{array}{llcll p{5cm}}
               &  (1)  & \multicolumn{3}{l}{(\sum_{i=i_0}^{i_0} s_i)a=\sum_{i=i_0}^{i_0} s_ia} & \awsSubLbiSubZeroRbInNaturalImpLpSumSubLbiEqualsiSubZeroRbPowerLbiSubZeroRbsSubiRpaEqualsSumSubLbiEqualsiSubZeroRbPowerLbiSubZeroRbsSubia{} \\
             2 &  (2)  & \multicolumn{3}{l}{n\in\mathbb{N}} & \rA \\
             3 &  (3)  & \multicolumn{3}{l}{(\sum_{i=i_0}^{n} s_i)a=\sum_{i=i_0}^{n} s_ia} & \rA \\ 
             2,3 &  (4)  & \multicolumn{3}{l}{(\sum_{i=i_0}^{n+1} s_i)a=\sum_{i=i_0}^{n+1} s_ia} & \awnwsSubLbiSubZeroRbwDotswsSubLbnPlusOneRbInNaturalwLpSumSubLbiEqualsiSubZeroRbPowerLbnRbsSubiRpaEqualsSumSubLbiEqualsiSubZeroRbPowerLbnRbsSubiaImpLpSumSubLbiEqualsiSubZeroRbPowerLbnPlusOneRbsSubiRpaEqualsSumSubLbiEqualsiSubZeroRbPowerLbnPlusOneRbsSubia{2,3} \\ 
                &  (5)  & \multicolumn{3}{l}{\forall n\in\mathbb{N}((\sum_{i=i_0}^{n} s_i)a=\sum_{i=i_0}^{n} s_ia)} & \rInductionN{1,2,3,4} \\ 
                &  (6)  & \multicolumn{3}{l}{n\in\mathbb{N}\rightarrow (\sum_{i=i_0}^{n} s_i)a=\sum_{i=i_0}^{n} s_ia} & \rSetEEb{5} \\ 
            2&  (7)  & \multicolumn{3}{l}{(\sum_{i=i_0}^{n} s_i)a=\sum_{i=i_0}^{n} s_ia} & \rRE{2,6} \\ 
        \end{array}
    \]
\end{proof}

\section{Rechtssdistributivität endlicher Summen mit Potenzen}

\label{awsSubLbiSubZeroRbInNaturalImpLpSumSubLbiEqualsiSubZeroRbPowerLbiSubZeroRbsSubiaPoweriRpaEqualsSumSubLbiEqualsiSubZeroRbPowerLbiSubZeroRbsSubiaPowerLbiPlusOneRb}
\begin{lemma}[\(a,s_{i_0}\in\mathbb{N}\vdash (\sum_{i=i_0}^{i_0} s_ia^i)a=\sum_{i=i_0}^{i_0} s_ia^{i+1}\) (Induktionsanfang)]
\end{lemma}
\begin{proof}
    Seien \(a,s_{i_0}\in\mathbb{N}\), dann gilt:
    \[
	\begin{array}{llcll p{5cm}}
              &  (1)  & (\sum_{i=i_0}^{i_0} s_ia^i)a &=& s_{i_0}a^ia & \rSumI{} \\
              &  (2)  &  &=& s_{i_0}a^{i+1} & \rPowerI{} \\
              &  (3)  &  &=& \sum_{i=i_0}^{i_0} s_ia^{i+1} & \rSumI{} \\
              &  (4)  &  (\sum_{i=i_0}^{i_0} s_ia^i)a&=& \sum_{i=i_0}^{i_0} s_ia^{i+1} & \rTransitivityEqRI{1,3} \\
        \end{array}
    \]
\end{proof}

\label{awnwsSubLbiSubZeroRbwDotswsSubLbnPlusOneRbInNaturalwLpSumSubLbiEqualsiSubZeroRbPowerLbnRbsSubiaPoweriRpaEqualsSumSubLbiEqualsiSubZeroRbPowerLbnRbsSubiaPowerLbiPlusOneRbImpLpSumSubLbiEqualsiSubZeroRbPowerLbnPlusOneRbsSubiaPoweriRpaEqualsSumSubLbiEqualsiSubZeroRbPowerLbnPlusOneRbsSubiaPowerLbiPlusOneRb}
\begin{lemma}[\(a,n,s_{i_0},\dots,s_{n+1}\in\mathbb{N},(\sum_{i=i_0}^{n} s_ia^i)a=\sum_{i=i_0}^{n} s_ia^{i+1}\vdash (\sum_{i=i_0}^{n+1} s_ia^i)a=\sum_{i=i_0}^{n+1} s_ia^{i+1}\) (Induktionsschritt)]
\end{lemma}
\begin{proof}
    Seien \(a,n,s_{i_0},\dots,s_{n+1}\in\mathbb{N}\). \(\ImpLpNaturalwMultwOneRpInAbelSemiRing{}\) und daher gilt:
    \[
	\begin{array}{llcll p{5cm}}
            1 &  (1)  & \multicolumn{3}{l}{(\sum_{i=i_0}^{n} s_ia^i)a=\sum_{i=i_0}^{n} s_ia^{i+1}} & \rA \\
              &  (2)  & (\sum_{i=i_0}^{n+1} s_ia^i)a &=& (\sum_{i=i_0}^{n} s_ia^i+s_{n+1}a^{n+1})a & \rSumI{} \\
              &  (3)  &  &=& (\sum_{i=i_0}^{n} s_ia^i)a+s_{n+1}a^{n+1}a & \rRightDistributiveAbelianSemigroup{} \\
            1 &  (4)  &  &=& \sum_{i=i_0}^{n} s_ia^{i+1}+s_{n+1}a^{n+1}a & \rIE{1,3} \\
              &  (5)  &  &=& \sum_{i=i_0}^{n} s_ia^{i+1}+s_{n+1}a^{n+2} & \rPowerI{} \\
              &  (6)  &  &=& \sum_{i=i_0}^{n+1} s_ia^{i+1} & \rSumI{} \\
            1 &  (7)  & (\sum_{i=i_0}^{n+1} s_ia^i)a &=& \sum_{i=i_0}^{n+1} s_ia^{i+1} & \rTransitivityEqRI{2,6} \\
        \end{array}
    \]
\end{proof}

\label{awnInNaturalwsSubLbiSubZeroRbwDotswsSubnInNaturalImpLpSumSubLbiEqualsiSubZeroRbPowernsSubiaPoweriRpaEqualsSumSubLbiEqualsiSubZeroRbPowernsSubiaPowerLbiPlusOneRb}
\begin{theorem}[\(a,n\in\mathbb{N},s_{i_0},\dots, s_n\in\mathbb{N}\vdash (\sum_{i=i_0}^n s_ia^i)a=\sum_{i=i_0}^n s_ia^{i+1}\)]
\end{theorem}
\begin{proof}
    Seien \(a,s_{i_0},\dots,s_{n}\in\mathbb{N}\), dann gilt:
    \[
	\begin{array}{llcll p{5cm}}
               &  (1)  & \multicolumn{3}{l}{(\sum_{i=i_0}^{i_0} s_ia^i)a=\sum_{i=i_0}^{i_0} s_ia^{i+1}} & \awsSubLbiSubZeroRbInNaturalImpLpSumSubLbiEqualsiSubZeroRbPowerLbiSubZeroRbsSubiaPoweriRpaEqualsSumSubLbiEqualsiSubZeroRbPowerLbiSubZeroRbsSubiaPowerLbiPlusOneRb{} \\
             2 &  (2)  & \multicolumn{3}{l}{n\in\mathbb{N}} & \rA \\
             3 &  (3)  & \multicolumn{3}{l}{(\sum_{i=i_0}^{n} s_ia^i)a=\sum_{i=i_0}^{n} s_ia^{i+1}} & \rA \\ 
             2,3 &  (4)  & \multicolumn{3}{l}{(\sum_{i=i_0}^{n+1} s_ia^i)a=\sum_{i=i_0}^{n+1} s_ia^{i+1}} & \awnwsSubLbiSubZeroRbwDotswsSubLbnPlusOneRbInNaturalwLpSumSubLbiEqualsiSubZeroRbPowerLbnRbsSubiaPoweriRpaEqualsSumSubLbiEqualsiSubZeroRbPowerLbnRbsSubiaPowerLbiPlusOneRbImpLpSumSubLbiEqualsiSubZeroRbPowerLbnPlusOneRbsSubiaPoweriRpaEqualsSumSubLbiEqualsiSubZeroRbPowerLbnPlusOneRbsSubiaPowerLbiPlusOneRb{2,3} \\ 
                &  (5)  & \multicolumn{3}{l}{\forall n\in\mathbb{N}(\sum_{i=i_0}^{n} s_ia^i)a=\sum_{i=i_0}^{n} s_ia^{i+1}} & \rInductionN{1,2,3,4} \\ 
                &  (6)  & \multicolumn{3}{l}{n\in\mathbb{N}\rightarrow (\sum_{i=i_0}^{n} s_ia^i)a=\sum_{i=i_0}^{n} s_ia^{i+1}} & \rSetEEb{5} \\ 
            2&  (7)  & \multicolumn{3}{l}{(\sum_{i=i_0}^{n} s_ia^i)a=\sum_{i=i_0}^{n} s_ia^{i+1}} & \rRE{2,6} \\ 
        \end{array}
    \]
\end{proof}

\section{Linksdistributivität endlicher Summen mit Potenzen}


\label{awnInNaturalwsSubLbiSubZeroRbwDotswsSubnInNaturalImpaSumSubLbiEqualsiSubZeroRbPowernsSubiaPoweriEqualsSumSubLbiEqualsiSubZeroRbPowernsSubiaPowerLbiPlusOneRb}
\begin{theorem}[\(a,n\in\mathbb{N},s_{i_0},\dots, s_n\in\mathbb{N}\vdash a\sum_{i=i_0}^n s_ia^i=\sum_{i=i_0}^n s_ia^{i+1}\)]
\end{theorem}
\begin{proof}
    Seien \(a,n,s_{i_0},\dots,s_{n}\in\mathbb{N}\). \(\ImpLpNaturalwMultwOneRpInAbelSemiRing{}\) und es gilt:
    \[
	\begin{array}{llcll p{5cm}}
               &  (1)  & a\sum_{i=i_0}^{i_0} s_ia^i&=&(\sum_{i=i_0}^{i_0} s_ia^i)a & \rCommutativeMonoid{} \\
               &  (2)  & &=&\sum_{i=i_0}^{i_0} s_ia^{i+1} & \awnInNaturalwsSubLbiSubZeroRbwDotswsSubnInNaturalImpLpSumSubLbiEqualsiSubZeroRbPowernsSubiaPoweriRpaEqualsSumSubLbiEqualsiSubZeroRbPowernsSubiaPowerLbiPlusOneRb{} \\
               &  (3)  & a\sum_{i=i_0}^{i_0} s_ia^i&=&\sum_{i=i_0}^{i_0} s_ia^{i+1} & \rTransitivityEqRI{1,2} \\
        \end{array}
    \]
\end{proof}




\chapter{n-adische Zahlendarstellung}

\section{Existenz der n-adischen Darstellung}

\subsection{Existenz der n-adischen Darstellung}

\paragraph{Induktionsanfang}

\label{FanInNaturalLpnGneqOneRpToExkInNaturalExaSubZerowaSubOnewDotswaSubkInLbZerowOnewDotswnMinusOneRbLpZeroEqualsSumSubLbiEqualsZeroRbPowerkaSubiMultnPoweriRp}
\begin{lemma}[\(n\in\mathbb{N}, n > 1 \vdash \exists k\in\mathbb{N}\exists a_0, a_1, \dots, a_k \in \{0,1,\dots,n-1\} ( 0 = \sum_{i=0}^k a_i \cdot n^i)\)]
\end{lemma}
\begin{proof}
Sei \(n\in\mathbb{N}\), dann gilt:
    \[
	\begin{array}{llcll p{4.5cm}}
            1 &  (1)  & \multicolumn{3}{l}{n > 1} & \rA \\
            1 &  (2)  &  \multicolumn{3}{l}{0\in\{0,1,...,n-1\}} & \nInNaturalwnGneqOneImpZeroInLbZerowOnewDotswnMinusOneRb{1} \\ 
              &  (3)  & 0&=&0\cdot 1 & \aInNaturalImpZeroEqualsZeroMulta{} \\
              &  (4)  & &=&0\cdot n^{0} & \rPowerI{} \\
              &  (5)  &  &=&\sum_{i=0}^0 0 \cdot n^{i} & \rSumI{} \\
              &  (6)  &  0&=&\sum_{i=0}^0 0 \cdot n^{i} & \rTransitivityEqRI{4,6} \\
             1&  (7)  & \multicolumn{3}{p{6cm}}{\(\exists a_0\in \{0,1,\dots,n-1\} ( 0 = \sum_{i=0}^0 a_i \cdot n^i)\)} & \rSetEIa{3,6} \\
              &  (8)  & \multicolumn{3}{l}{0\in\mathbb{N}} & \zeroIsNaturalNumber{} \\
             1 &  (9)  & \multicolumn{3}{p{6cm}}{\(\exists k\in\mathbb{N}\exists a_0, a_1, \dots, a_k \in \{0,1,\dots,n-1\} ( 0 = \sum_{i=0}^k a_i \cdot n^i)\)} & \rSetEIa{9,8} \\
        \end{array}
    \]
\end{proof}

\begin{lemma}[Induktionsschritt]
Seien \(n,m,a\in\mathbb{N}\) und
\[
n > 1, \exists k \in \mathbb{N}\exists a_0, a_1, \dots, a_k \in \{0, 1, \dots, n-1\} 
    (m+1)\div n = \sum_{i=0}^k a_i \cdot n^i\]
\[
     \vdash\exists k' \in \mathbb{N}, \, \exists a_0', a_1', \dots, a_{k'}' \in \{0, 1, \dots, n-1\}m + 1 = \sum_{i=0}^{k'} a_i' \cdot n^i.
\]
\end{lemma}
\begin{proof}
Seien \(n,m,a\in\mathbb{N}\), dann gilt:
    \[
	\begin{array}{llcll p{4.5cm}}
             1 &  (1)  & \multicolumn{3}{l}{n > 1} & \rA \\
             1 &  (2)  & \multicolumn{3}{l}{n \neq 1} & \aInNaturalwaGeqOneEqvaNotEqualsZero{1} \\
             4 &  (3)  & \multicolumn{3}{p{7cm}}{\(\exists k \in \mathbb{N}\exists a_0, a_1, \dots, a_k \in \{0, 1, \dots, n-1\} 
    (m+1)\div n = \sum_{i=0}^k a_i \cdot n^i\)}& \rA \\
    \multicolumn{6}{l}{\text{Wähle \(k\in\mathbb{N}\), sodass:}}\\
             4 &  (4)  & \multicolumn{3}{p{7cm}}{\(\exists a_0, a_1, \dots, a_k \in \{0, 1, \dots, n-1\} 
    ((m+1)\div n = \sum_{i=0}^k (a_i \cdot n^i))\)}& \rA \\
             5 &  (5)  & \multicolumn{3}{l}{(m+1)\div n = \sum_{i=0}^k a_i \cdot n^{i}}& \rA \\
             1 &  (6)  & \multicolumn{3}{l}{m+1 = n\cdot ((m+1)\div n)+((m+1)\bmod n)}& \rDivisionWithRemainderI{2} \\
             1,5 &  (7)  & m+1 &=& n\cdot (\sum_{i=0}^k a_i \cdot n^{i})+((m+1)\bmod n)& \rIE{5,6} \\
             1 &  (8)  &  &=& \sum_{i=0}^k a_{i} \cdot n^{i+1}+((m+1)\bmod n)& \awnInNaturalwsSubLbiSubZeroRbwDotswsSubnInNaturalImpaSumSubLbiEqualsiSubZeroRbPowernsSubiaPoweriEqualsSumSubLbiEqualsiSubZeroRbPowernsSubiaPowerLbiPlusOneRb{} \\
               &  (9)  &  &=& \sum_{i=0}^k a_{i} \cdot n^{i+1}+((m+1)\bmod n)\cdot 1& \rNeutralElementMonoid{} \\
               &  (10)  &  &=& \sum_{i=0}^k a_{i} \cdot n^{i+1}+((m+1)\bmod n)\cdot n^0& \rPowerI{} \\
               &  (11)  &  &=& ((m+1)\bmod n)\cdot n^0+\sum_{i=0}^k a_{i} \cdot n^{i+1}& \rCommutativeMonoid{} \\
             1,5 &  (12)  & m+1 &=& ((m+1)\bmod n)\cdot n^0+\sum_{i=0}^k a_{i} \cdot n^{i+1}& \rTransitivityEqRI{6,11} \\
             1 &  (13)  &\multicolumn{3}{l}{((m+1)\bmod n)<n}& \rDivisionWithRemainderI{2} \\
             1 &  (14)  &\multicolumn{3}{l}{((m+1)\bmod n)\in \{0,1,\dots,n-1\}}& \rSegmentZeroI{13} \\
             1,5 &  (15)  &\multicolumn{3}{p{7cm}}{\(\exists a_0'\in \{0,1,\dots,n-1\} (m+1 = a_0'\cdot n^0+\sum_{i=0}^k a_{i} \cdot n^{i+1}) \)}& \rSetEIa{13,12} \\
             1,5 &  (16)  &\multicolumn{3}{p{7cm}}{\(\exists a_0',\dots a_k'\in \{0,1,\dots,n-1\} (m+1 = a_0'\cdot n^0+\sum_{i=0}^k a_{i+1}' \cdot n^{i+1}) \)}& \rSetEm{15} \\
        \end{array}
    \]
\end{proof}
????
aInNaturalwaGeqOneEqvaNotEqualsZero

\begin{theorem}[\(\forall a \in \mathbb{N}\forall n \in \mathbb{N} \left( n > 1 \right) \rightarrow \exists k\in\mathbb{N}\exists a_0, a_1, \dots, a_k \in \{0,1,\dots,n-1\} \left( a = \sum_{i=0}^k a_i \cdot n^i \right)\)]
\end{theorem}

\begin{proof}
    \[
	\begin{array}{llcll p{4.5cm}}
              &  (1)  & \multicolumn{3}{p{6cm}}{\(\forall n \in \mathbb{N}( n > 1) \rightarrow \exists k\in\mathbb{N}\exists a_0, a_1, \dots, a_k \in \{0,1,\dots,n-1\} ( 0 = \sum_{i=0}^k a_i \cdot n^i)\)} & \FanInNaturalLpnGneqOneRpToExkInNaturalExaSubZerowaSubOnewDotswaSubkInLbZerowOnewDotswnMinusOneRbLpZeroEqualsSumSubLbiEqualsZeroRbPowerkaSubiMultnPoweriRp{} \\
        \end{array}
    \]
\end{proof}




\subsection{Eindeutigkeit der n-adischen Darstellung}

\begin{theorem}[Eindeutigkeit der n-adischen Darstellung]
    \(
    \forall a, n \in \mathbb{N}, n > 1, 
    \forall a_0, \dots, a_k, b_0, \dots, b_m \in \mathbb{N}, 
    \big(
        0 \leq a_i < n \land 
        0 \leq b_j < n \land 
        a = \sum_{i=0}^{k} a_i \cdot n^i \land 
        a = \sum_{j=0}^{m} b_j \cdot n^j
    \big) 
    \vdash 
    (k = m) \land (\forall i \leq k: a_i = b_i)
    \)
\end{theorem}



\subsection{Definition der n-adischen Darstellung}

\begin{definition}[n-adische Darstellung]
    Seien \( a \in \mathbb{N} \) und \( n \in \mathbb{N} \) mit \( n > 1 \). Die \textbf{n-adische Darstellung} von \( a \) ist die eindeutige Folge von Ziffern \( (a_k, a_{k-1}, \dots, a_0) \), sodass
    \[
    a = \sum_{i=0}^{k} a_i \cdot n^i,
    \]
    wobei \( 0 \leq a_i < n \) für alle \( i \) gilt.
\end{definition}

\subsection{Einführung des Dezimalsystems}

\begin{theorem}[Dezimalsystem als n-adische Darstellung]
    \(
    \forall a \in \mathbb{N} \vdash \exists a_0, a_1, \dots, a_k \in \mathbb{N} \,
    \big(
        0 \leq a_i < 10 \land 
        a = a_0 + a_1 \cdot 10 + a_2 \cdot 10^2 + \dots + a_k \cdot 10^k
    \big)
    \)
\end{theorem}

\begin{proof}
    \[
    \begin{array}{llcll p{5cm}}
        1 &  (1)  & a \in \mathbb{N} & \rA & \\
        2 &  (2)  & n = 10 & \rA & \\
        3 &  (3)  & \text{Existenz folgt aus Theorem 1} & & \\
        \multicolumn{5}{l}{\text{Damit ist die Existenz im Dezimalsystem gezeigt.}} \\
    \end{array}
    \]
\end{proof}

\begin{example}
    \(
    345 = 5 \cdot 10^0 + 4 \cdot 10^1 + 3 \cdot 10^2
    \)
\end{example}



\chapter{Folgen}

\section{Einführung in Folgen}
Eine \textbf{Folge} ist eine systematische Anordnung von Elementen in einer bestimmten Reihenfolge. Formal betrachtet ist eine Folge eine Abbildung, die jedem natürlichen Index \(n\) ein Element aus einer bestimmten Menge zuordnet.

\begin{definition}[Endliche und unendliche Folge]
    Sei \(A\) eine Menge.

    \begin{itemize}
        \item Eine \textbf{endliche Folge} in \(A\) der Länge \(n \in \mathbb{N}\) ist eine Abbildung
        \[
        a: \{0, 1, \dots, n\} \to A.
        \]
        Wir schreiben eine solche Folge häufig als \((a_k)_{k=0}^n\) oder in der Form \((a_0, a_1, \dots, a_n)\), wobei das \(k\)-te Element der Folge \(a\) durch \(a_k := a(k)\) gegeben ist.

        \item Eine \textbf{unendliche Folge} in \(A\) ist eine Abbildung
        \[
        a: \mathbb{N} \to A,
        \]
        die jedem \(n \in \mathbb{N}\) ein Element \(a(n) \in A\) zuordnet. Wir schreiben eine solche Folge häufig als \((a_n)_{n \in \mathbb{N}}\) oder in der Form \((a_0, a_1, a_2, \dots)\), wobei das \(n\)-te Element der Folge \(a\) durch \(a_n := a(n)\) gegeben ist.
    \end{itemize}
\end{definition}

\section{Menge aller Folgen}

\label{ExCFafLpfInCLrfDefineLbZerowOnewDotswnMinusOneRbToARp}
\begin{theorem}[Existenz von \(A^n\)]
 Sei \(A\) eine Menge. Dann existiert eine Menge \(C\) aller Funktionen von \(\{0,1,...n-1\}\) nach \(A\) so, dass für alle \(f\) gilt:
 \[f\in C\leftrightarrow f: \{0, 1, \dots, n-1\} \to A\]
\end{theorem}
\begin{proof}
	\[
	\begin{array}{ll  p{4.7cm} p{4cm}}
		1 & (1) & \ensuremath{f: \{0, 1, \dots, n-1\} \to A} & \rA \\
		1 & (2) & \ensuremath{f\subseteq \{0, 1, \dots, n-1\}\times A} & \toE{1} \\
		1 & (3) & \ensuremath{f \in \mathcal{P}(\{0, 1, \dots, n-1\} \times A)} &  \powersetI{1} \\
          & (4) & \ensuremath{f: \{0, 1, \dots, n-1\} \to A\rightarrow f \in \mathcal{P}(\{0, 1, \dots, n-1\} \times A)} &  \rRI{1,3} \\
          & (5) & \ensuremath{\forall f(f: \{0, 1, \dots, n-1\} \to A\rightarrow f \in \mathcal{P}(\{0, 1, \dots, n-1\} \times A))} &  \rUI{5} \\
          & (6) & \ensuremath{\exists C\forall f(f\in C\leftrightarrow f: \{0, 1, \dots, n-1\} \to A)} &  \ExALpFaxLpPLpxRpToxInARpRpImpExBFaxLpxInBLrPLpxRpRp{5} \\
	\end{array}
	\]
\end{proof}

\label{FaxLpxInELrxDefineLbZerowOnewDotswnMinusOneRbToARpAndFaxLpxInFLrxDefineLbZerowOnewDotswnMinusOneRbToARpImpEEqualsF}
\begin{theorem}[Eindeutigkeit der Menge \(A^n\)]
    Sei \(A\) eine Menge und seien \(E\) und \(F\) zwei Mengen, die beide die Menge aller Funktionen von \(\{0, 1, \dots, n-1\}\) nach \(A\) darstellen. Dann gilt unter Verwendung des Extensionalitätsaxioms:
    \[
    \forall x  (x \in E \leftrightarrow x: \{0, 1, \dots, n-1\} \to A) \land \forall x  (x \in F \leftrightarrow x: \{0, 1, \dots, n-1\} \to A) \vdash E = F.
    \]
\end{theorem}
\begin{tempdefinition}
    \[\forall x(Q(x):= x: \{0, 1, \dots, n-1\} \to A)\]
\end{tempdefinition}
\begin{proof}
    \[
    \begin{array}{ll p{6cm} p{4cm}}
        1 & (1) & \ensuremath{\forall x \, (x \in E \leftrightarrow x: \{0, 1, \dots, n-1\} \to A) \land \forall x \, (x \in F \leftrightarrow x: \{0, 1, \dots, n-1\} \to A)} &  \rA \\
        1 & (2) & \ensuremath{\forall x (x \in E \leftrightarrow Q(x)) \land \forall x (x \in C \leftrightarrow Q(x))} & \rIE{df(Q), 1} \\
        1 & (3) & \ensuremath{E=F} &  \FaxLpxInBLrPLpxRpRpAndFaxLpxInCLrPLpxRpRpImpBEqualsC{2} \\
    \end{array}
    \]
\end{proof}

\label{ExCFafLpfInCLrfDefineNaturalToARp}
\begin{theorem}[Existenz von \(A^\mathbb{N}\)]
 Sei \(A\) eine Menge. Dann existiert eine Menge \(C\) aller Funktionen von \(\mathbb{N}\) nach \(A\) so, dass für alle \(f\) gilt:
 \[f\in C\leftrightarrow f: \mathbb{N} \to A\]
\end{theorem}
\begin{proof}
	\[
	\begin{array}{ll  p{4.7cm} p{4cm}}
		1 & (1) & \ensuremath{f: \mathbb{N} \to A} & \rA \\
		1 & (2) & \ensuremath{f\subseteq \mathbb{N}\times A} & \toE{1} \\
		1 & (3) & \ensuremath{f \in \mathcal{P}(\mathbb{N} \times A)} &  \powersetI{1} \\
          & (4) & \ensuremath{f:\mathbb{N} \to A\rightarrow f \in \mathcal{P}(\mathbb{N} \times A)} &  \rRI{1,3} \\
          & (5) & \ensuremath{\forall f(f: \mathbb{N} \to A\rightarrow f \in \mathcal{P}(\mathbb{N} \times A))} &  \rUI{5} \\
          & (6) & \ensuremath{\exists C\forall f(f\in C\leftrightarrow f: \mathbb{N} \to A)} &  \ExALpFaxLpPLpxRpToxInARpRpImpExBFaxLpxInBLrPLpxRpRp{5} \\
	\end{array}
	\]
\end{proof}

\label{FaxLpxInELrxDefineNaturalToARpAndFaxLpxInFLrxDefineNaturalToARpImpEEqualsF}
\begin{theorem}[Eindeutigkeit der Menge \(A^\mathbb{N}\)]
    Sei \(A\) eine Menge und seien \(E\) und \(F\) zwei Mengen, die beide die Menge aller Funktionen von \(\mathbb{N}\) nach \(A\) darstellen. Dann gilt unter Verwendung des Extensionalitätsaxioms:
    \[
    \forall x  (x \in E \leftrightarrow x: \mathbb{N} \to A) \land \forall x  (x \in F \leftrightarrow x:\mathbb{N} \to A) \vdash E = F.
    \]
\end{theorem}
\begin{tempdefinition}
    \[\forall x(Q(x):= x: \mathbb{N} \to A)\]
\end{tempdefinition}
\begin{proof}
    \[
    \begin{array}{ll p{6cm} p{4cm}}
        1 & (1) & \ensuremath{\forall x \, (x \in E \leftrightarrow x: \mathbb{N} \to A) \land \forall x \, (x \in F \leftrightarrow x: \mathbb{N} \to A)} &  \rA \\
        1 & (2) & \ensuremath{\forall x (x \in E \leftrightarrow Q(x)) \land \forall x (x \in C \leftrightarrow Q(x))} & \rIE{df(Q), 1} \\
        1 & (3) & \ensuremath{E=F} &  \FaxLpxInBLrPLpxRpRpAndFaxLpxInCLrPLpxRpRpImpBEqualsC{2} \\
    \end{array}
    \]
\end{proof}

\begin{definition}[Menge aller Funktionen von \(\{0, 1, \dots, n-1\}\) nach \(A\)]
    Sei \(A\) eine Menge. Dann bezeichnet \(A^n\) die Menge aller Funktionen von \(\{0, 1, \dots, n-1\}\) nach \(A\), definiert als:
    \[
    A^n := \{ f \mid f: \{0, 1, \dots, n-1\} \to A \}.
    \]
\end{definition}

\begin{definition}[Menge aller endlichen Folgen]
    Sei \(A\) eine Menge. Die Menge aller endlichen Folgen mit Werten in \(A\) wird durch
    \[
    A^{*} := \bigcup_{n \in \mathbb{N}} A^n
    \]
    definiert, wobei \( A^n \) die Menge aller Abbildungen \( f: \{0, 1, \dots, n-1\} \to A \) darstellt. Jedes Element von \(A^{*}\) ist also eine endliche Folge in \(A\).
\end{definition}

Hierbei bezeichnet \(A^{n}\) die Menge aller endlichen Folgen der Länge \(n\) mit Elementen aus \(A\). Durch die Vereinigung über alle \(n\) erhalten wir die Menge aller möglichen endlichen Folgen in \(A\).

\begin{definition}[Menge aller unendlichen Folgen]
    Sei \(A\) eine Menge. Die Menge aller unendlichen Folgen mit Werten in \(A\) wird durch
    \[
    A^{\mathbb{N}} := \{f \mid f: \mathbb{N} \to A\}
    \]
    definiert. Jedes Element von \(A^{\mathbb{N}}\) ist also eine unendliche Folge in \(A\).
\end{definition}




\chapter{Zahlensysteme}

\section{Einführung in Zahlensysteme}
Ein Zahlensystem dient zur Darstellung von Zahlen und legt fest, wie Ziffern in verschiedenen Stellenwerten eine eindeutige Zahl repräsentieren. Im Allgemeinen wird eine natürliche Zahl \( n \in \mathbb{N} \) durch eine Abfolge von Symbolen dargestellt, die je nach System unterschiedliche Basen verwenden. Die gängigsten Zahlensysteme basieren auf Potenzen einer festgelegten Basis \( b \), wobei \( b \geq 2 \) ist.

In diesem Kapitel führen wir zunächst das Binärsystem (Basis 2) ein, das eine zentrale Rolle in der Informatik spielt und auf Potenzen von \(2\) basiert. Danach betrachten wir andere Zahlensysteme wie das Dezimalsystem (Basis 10) und das Hexadezimalsystem (Basis 16), die ebenfalls auf der Darstellung natürlicher Zahlen über Stellenwerten basieren.

\subsection{Die Zahl 2}
Für die Definition des Binärsystems benötigen wir die Zahl \(2\), welche als Nachfolger der Zahl \(1\) definiert ist.

\begin{definition}[\(2 := 1 \cup \{1\}\)]
\end{definition}

Diese Definition ermöglicht es, das Binärsystem mathematisch einzuführen, da es auf der Verwendung von Potenzen der Zahl \(2\) beruht.

\section{Das Binärsystem}
Das Binärsystem ist ein Stellenwertsystem zur Darstellung natürlicher Zahlen, bei dem jede Stelle ein Vielfaches einer Potenz von \(2\) ist. Jede Zahl kann als Summe von Potenzen der Basis \(2\) dargestellt werden, wobei jede Potenz mit \(0\) oder \(1\) multipliziert wird. Diese Darstellung wird als Bitfolge dargestellt, in der die Koeffizienten \(b_i\) aneinandergereiht sind.

\subsection{Menge der binären Darstellungen}

Die Menge der binären Darstellungen natürlicher Zahlen ohne führende Nullen wird durch folgende Eigenschaften definiert:

\begin{definition}[Menge der binären Darstellungen]
    Sei 
    \[
    \mathbb{B}_{\text{nat}} := \{(b_k)_{k=0}^n \in \{0,1\}^* \mid b_n = 1 \lor n = 0\}.
    \]
    Diese Bedingung stellt sicher, dass die höchste Potenz in der Darstellung \( n = \sum_{k=0}^n b_k \cdot 2^k \) entweder durch den Wert \( b_n = 1 \) besetzt ist oder \( n = 0 \) gilt. Damit sind führende Nullen ausgeschlossen, außer im speziellen Fall \( n = 0 \), bei dem die Darstellung aus nur einem Bit \( b_0 = 0 \) besteht. Die Menge \( \mathbb{B}_{\text{nat}} \) umfasst somit alle binären Darstellungen ohne führende Nullen, wobei \( n = 0 \) ebenfalls eingeschlossen ist.

\end{definition}

\paragraph{Repräsentation binärer Darstellungen}

Die Elemente der Menge \( \mathbb{B}_{\text{nat}} \) können auch als Bitfolgen geschrieben werden, indem die Bits \( b_k \) aneinandergereiht werden, beginnend mit dem höchstwertigen Bit. Für eine Binärdarstellung \( (b_k)_{k=0}^n \in \mathbb{B}_{\text{nat}} \) bedeutet dies, dass sie durch die Folge \( b_n b_{n-1} \ldots b_0 \) repräsentiert wird, wobei \( b_n = 1 \) gilt, falls \( n > 0 \). Beispielsweise:

\begin{itemize}
    \item Die natürliche Zahl \( 2 \) hat die Binärdarstellung \( (1, 0) \), was als Bitfolge \( 10 \) geschrieben wird.
    \item Die natürliche Zahl \( 0 \) hat die Binärdarstellung \( (0) \), was als Bitfolge \( 0 \) geschrieben wird.
\end{itemize}

Durch diese Schreibweise können die Binärdarstellungen anschaulich als klassische Bitfolgen dargestellt werden, die in praktischen Anwendungen wie der Computerarithmetik oder Speicherrepräsentation gebräuchlich sind.

\begin{remark}
    Diese Darstellung als Bitfolge entspricht der Konvention, Zahlen im Binärsystem darzustellen, und wird im Folgenden verwendet, um die Elemente aus \( \mathbb{B}_{\text{nat}} \) kompakt zu schreiben.
\end{remark}

\subsection{Binär-zu-Integer-Konvertierungsfunktion}

Um jeder binären Darstellung in \( \mathbb{B}_{\text{nat}} \) eindeutig eine natürliche Zahl zuzuordnen, definieren wir eine Konvertierungsfunktion. Diese Funktion wandelt eine Folge von Bits in die entsprechende natürliche Zahl um.

\begin{definition}[Binär-zu-Integer-Konvertierungsfunktion]
    Die \textbf{Binär-zu-Integer-Konvertierungsfunktion} \( \beta : \mathbb{B}_{\text{nat}} \to \mathbb{N} \) ist eine Abbildung, die jeder Folge \( (b_k)_{k=0}^n \in \mathbb{B}_{\text{nat}} \) die natürliche Zahl \( n \in \mathbb{N} \) zuordnet durch
    \[
    \beta((b_k)_{k=0}^n) \coloneqq \sum_{k=0}^n b_k \cdot 2^k.
    \]
    Dabei ist \((b_k)_{k=0}^n\) eine Folge von Bits, und die Summe \( \sum_{k=0}^n b_k \cdot 2^k \) beschreibt die durch diese Binärdarstellung repräsentierte Zahl.
\end{definition}

\begin{theorem}[Darstellbarkeit jeder natürlichen Zahl im Binärsystem]
    \(\beta : \mathbb{B}_{\text{nat}} \to \mathbb{N} \) ist surjektiv.
\end{theorem}

\subsection{Addition binärer Darstellungen}

Die Addition von natürlichen Zahlen kann im Binärsystem direkt auf den Bitfolgen durchgeführt werden. Für zwei Binärdarstellungen \( (b_k)_{k=0}^n \) und \( (c_k)_{k=0}^m \) definieren wir die Addition ihrer repräsentierten natürlichen Zahlen durch ein schrittweises Verfahren, das Überträge berücksichtigt. Die Addition erfolgt bitweise von der niederwertigsten Stelle (dem rechten Ende) zur höchstwertigen Stelle (dem linken Ende). 

\begin{definition}[Bitweise Addition ohne Zahlen größer als \(2\)]
    Sei \( (b_k)_{k=0}^n \in \mathbb{B}_{\text{nat}} \) und \( (c_k)_{k=0}^m \in \mathbb{B}_{\text{nat}} \). Die Summe \( (s_k)_{k=0}^l \) und der Übertrag \( u_k \) werden rekursiv wie folgt definiert:
    \begin{itemize}
        \item Initialisierung: \( u_0 = 0 \).
        \item Für jede Stelle \( k = 0, \ldots, l \) gilt:
        \begin{itemize}
            \item \( s_k = 1 \leftrightarrow(b_k = 1 \lxor c_k = 1 \lxor u_k = 1) \lor (b_k = 1 \wedge c_k = 1 \wedge u_k = 1)\).
            \item \( s_k = 0 \leftrightarrow \neg(s_k=1)\).
            \item \( u_{k+1} = 1 \leftrightarrow (b_k = 1 \wedge c_k = 1) \lor (b_k = 1 \wedge u_k = 1) \lor (c_k = 1 \wedge u_k = 1)\).
            \item \( u_{k+1} = 0 \leftrightarrow \neg(u_{k+1} = 1)\).
        \end{itemize}
    \end{itemize}
\end{definition}

\begin{definition}[Addition mit \(1\) im Binärsystem]
    Sei \( (b_k)_{k=0}^n \in \mathbb{B}_{\text{nat}} \) die Binärdarstellung einer natürlichen Zahl. Die Summe \( (s_k)_{k=0}^l \) und der Übertrag \( u_k \) bei der Addition von \(1\) werden rekursiv wie folgt definiert:
    \begin{itemize}
        \item Initialisierung: \( u_0 = 1 \) (da \(1\) zum niederwertigsten Bit addiert wird).
        \item Für jede Stelle \( k = 0, \ldots, l \) gilt:
        \begin{itemize}
            \item \( s_k = 1 \leftrightarrow (b_k = 1 \lxor u_k = 1) \).
            \item \( s_k = 0 \leftrightarrow \neg(s_k = 1)\).
            \item \( u_{k+1} = 1 \leftrightarrow (b_k = 1 \wedge u_k = 1)\).
            \item \( u_{k+1} = 0 \leftrightarrow \neg(u_{k+1} = 1)\).
        \end{itemize}
    \end{itemize}
    Hierbei ist \( l = n + 1 \), falls der höchste Übertrag \( u_{n+1} = 1 \) entsteht, und \( l = n \), falls kein Übertrag über das höchste Bit hinaus propagiert.
\end{definition}

\begin{example}
    Addieren wir die natürlichen Zahlen \( 5 \) (\(101\)) und \( 3 \) (\(11\)) im Binärsystem. Zunächst schreiben wir die Zahlen stellenweise untereinander, wobei wir die Bitfolgen an die gleiche Länge angleichen, indem führende Nullen hinzugefügt werden:
    \[
    \begin{aligned}
        & \quad 1\,0\,1 & (b_2, b_1, b_0) \\
        + & \quad 0\,1\,1 & (c_2, c_1, c_0).
    \end{aligned}
    \]

    Wir berechnen nun die Summe bitweise von rechts nach links, wobei \( u_k \) der Übertrag ist:

    \[
    \begin{aligned}
        k = 0: & \quad s_0 = 1 \leftrightarrow (b_0 = 1 \lxor c_0 = 1 \lxor u_0 = 1) \lor (b_0 = 1 \wedge c_0 = 1 \wedge u_0 = 1), \\
               & \quad b_0 = 1, c_0 = 1, u_0 = 0 \implies s_0 = 0, \\
               & \quad u_1 = 1 \leftrightarrow (b_0 = 1 \wedge c_0 = 1) \lor (b_0 = 1 \wedge u_0 = 1) \lor (c_0 = 1 \wedge u_0 = 1), \\
               & \quad b_0 = 1, c_0 = 1, u_0 = 0 \implies u_1 = 1. \\    
        k = 1: & \quad s_1 = 1 \leftrightarrow (b_1 = 1 \oplus c_1 = 1 \oplus u_1 = 1) \lor (b_1 = 1 \wedge c_1 = 1 \wedge u_1 = 1), \\
               & \quad b_1 = 0, c_1 = 1, u_1 = 1 \implies s_1 = 0, \\
               & \quad u_2 = 1 \leftrightarrow (b_1 = 1 \wedge c_1 = 1) \lor (b_1 = 1 \wedge u_1 = 1) \lor (c_1 = 1 \wedge u_1 = 1), \\
               & \quad b_1 = 0, c_1 = 1, u_1 = 1 \implies u_2 = 1. \\    
        k = 2: & \quad s_2 = 1 \leftrightarrow (b_2 = 1 \oplus c_2 = 1 \oplus u_2 = 1) \lor (b_2 = 1 \wedge c_2 = 1 \wedge u_2 = 1), \\
               & \quad b_2 = 1, c_2 = 0, u_2 = 1 \implies s_2 = 0, \\
               & \quad u_3 = 1 \leftrightarrow (b_2 = 1 \wedge c_2 = 1) \lor (b_2 = 1 \wedge u_2 = 1) \lor (c_2 = 1 \wedge u_2 = 1), \\
               & \quad b_2 = 1, c_2 = 0, u_2 = 1 \implies u_3 = 1. \\
        k = 3: & \quad s_3 = u_3 = 1.
    \end{aligned}
    \]

    Das Ergebnis der Addition ist die Bitfolge \( 1\,0\,0\,0 \), die die natürliche Zahl \( 8 \) repräsentiert.
\end{example}

\begin{remark}
    Die Addition von Binärdarstellungen entspricht der üblichen Addition im Dezimalsystem, wobei die Basis \( 2 \) die Übertragungsregeln definiert.
\end{remark}




\end{document}