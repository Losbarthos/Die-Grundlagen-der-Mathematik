% expl3-Spickzettel.tex
% Kurzübersicht über wichtige Datentypen, Variablen- und Funktionsmuster in expl3

\documentclass[a4paper,11pt]{article}

% --- Pakete für Darstellung ---
\usepackage[T1]{fontenc}
\usepackage[utf8]{inputenc}
\usepackage[ngerman]{babel}
\usepackage{geometry}
\geometry{margin=2.4cm}
\usepackage{booktabs}
\usepackage{longtable}
\usepackage{tabularx}
\usepackage{array}
\usepackage{xcolor}
\usepackage{hyperref}
\usepackage{microtype}
\usepackage{amsmath} % für \text in \meta
\usepackage{csquotes} % für \enquote
\usepackage{xltabular}
% --- expl3 laden, aber NICHT dauerhaft \ExplSyntaxOn setzen ---
\usepackage{expl3}

% flexible Spaltentypen für xltabular
\newcolumntype{Y}{>{\raggedright\arraybackslash}X} % normaler Text, umbruchfähig
\newcolumntype{K}{>{\ttfamily\footnotesize\raggedright\arraybackslash}X} % Code-Spalte

% Fallback für \meta (nutzt amsmath's \text)
\providecommand\meta[1]{\ensuremath{\langle\text{#1}\rangle}}

% Robustes Inline-Code-Makro: untrennbar, sodass vor "_" keine Lücke entsteht
\DeclareRobustCommand{\code}[1]{\texttt{\mbox{\detokenize{#1}}}}

\title{expl3\,–\,Spickzettel: Datentypen, Objekte und typische Befehle}
\author{(Kurzreferenz für den eigenen Gebrauch)}
\date{\today}

% Spaltentypen
\newcolumntype{L}[1]{>{\raggedright\arraybackslash}p{#1}}
\newcolumntype{C}[1]{>{\centering\arraybackslash}p{#1}}
\newcolumntype{R}[1]{>{\raggedleft\arraybackslash}p{#1}}

% Tabellen-Header
\newcommand\Header{\toprule
  \textbf{Kategorie} & \textbf{Objekt / Befehl} & \textbf{Kurzbeschreibung} & \textbf{Minimalbeispiel}\\
  \midrule\endfirsthead
  \toprule
  \textbf{Kategorie} & \textbf{Objekt / Befehl} & \textbf{Kurzbeschreibung} & \textbf{Minimalbeispiel}\\
  \midrule\endhead
}

\begin{document}
\maketitle
\tableofcontents
\bigskip

\section{Konventionen in \texttt{expl3}}
\begin{itemize}
  \item \textbf{Benennung:} \texttt{\meta{modul}\_\meta{name}:\meta{arg\_sig} } (z.\,B. \code{\seq_new:N}, \code{\tl_set:Nn}).
  \item \textbf{Variablen-Endungen:} \code{\_tl}, \code{\_seq}, \code{\_clist}, \code{\_prop}, \code{\_int}, \code{\_dim}, \code{\_skip}, \code{\_muskip}, \code{\_bool}, \code{\_fp}, \code{\_str}, \code{\_box}, \dots
  \item \textbf{Global vs. lokal:} Globale Variante mit \code{\_g} (z.\,B. \code{\seq_gput_right:Nn}).
\end{itemize}

\section{Datentypen und Kernoperationen}
\small
\begin{xltabular}{\linewidth}{L{2.6cm} L{3.6cm} Y K}
\Header
\textbf{Tokenliste (tl)} & \code{\tl_new:N} / \code{\tl_set:Nn} & Neue tl-Variable; setzen auf Inhalt. & \code{\tl_new:N \l_demo_tl}\\
 &  &  & \code{\tl_set:Nn \l_demo_tl {Hallo}} \\
 & \code{\tl_put_right:Nn} / \code{\tl_put_left:Nn} & Anfügen (rechts/links). & \code{\tl_put_right:Nn \l_demo_tl { Welt}} \\
\midrule
\textbf{Sequenz (seq)} & \code{\seq_new:N} & Neue Sequenz (stapel-/listenartig). & \code{\seq_new:N \l_demo_seq} \\
 & \code{\seq_set_split:Nnn} & Teilt Tokenliste an Trennzeichen in Elemente. & \code{\seq_set_split:Nnn \l_demo_seq { , } {a,b,c}} \\
 & \code{\seq_pop_left:NN} / \code{\seq_pop_right:NN} & Entnehmen eines Elements (liefert in Var). & \code{\seq_pop_left:NN \l_demo_seq \l_tmpa_tl} \\
 & \code{\seq_map_inline:Nn} & Über alle Elemente iterieren. & \code{\seq_map_inline:Nn \l_demo_seq {\tl_show:n{#1}}} \\
\midrule
\textbf{Kommaliste (clist)} & \code{\clist_new:N} / \code{\clist_set:Nn} & Neue Kommaliste; setzen. & \code{\clist_set:Nn \l_demo_clist {a,b,c}} \\
 & \code{\clist_map_inline:Nn} & Iteration über Elemente. & \code{\clist_map_inline:Nn \l_demo_clist { (#1) }} \\
\midrule
\textbf{Property-Liste (prop)} & \code{\prop_new:N} / \code{\prop_put:Nnn} & Schlüssel\,→\,Wert ablegen. & \code{\prop_put:Nnn \l_demo_prop {key} {val}} \\
 & \code{\prop_item:Nn} & Wert zu Schlüssel abrufen. & \code{\prop_item:Nn \l_demo_prop {key}} \\
 & \code{\prop_map_inline:Nn} & Über Paare iterieren. & \code{\prop_map_inline:Nn \l_demo_prop {#1 = #2 }} \\
\midrule
\textbf{Ganzzahl (int)} & \code{\int_new:N} / \code{\int_set:Nn} & Ganze Zahl setzen (\TeX{}-Integer). & \code{\int_set:Nn \l_demo_int { 1 + 2*3 }} \\
 & \code{\int_eval:n} & Ausdruck auswerten (expandierbar). & \code{\int_eval:n { \l_demo_int + 5 }} \\
\midrule
\textbf{Länge (dim)} & \code{\dim_new:N} / \code{\dim_set:Nn} & Länge (z.\,B. \code{pt}, \code{mm}). & \code{\dim_set:Nn \l_demo_dim {2em}} \\
 & \code{\dim_eval:n} & expandierbare Auswertung. & \code{\dim_eval:n { \l_demo_dim + 3pt }} \\
\midrule
\textbf{Kleber (skip)} & \code{\skip_new:N} / \code{\skip_set:Nn} & Kleber (elastische Länge). & \code{\skip_set:Nn \l_demo_skip {1ex plus 1fil}} \\
\midrule
\textbf{Mathe-Kleber (muskip)} & \code{\muskip_new:N} / \code{\muskip_set:Nn} & Mathe-Kleber. & \code{\muskip_set:Nn \l_demo_muskip {3mu plus 2mu}} \\
\midrule
\textbf{Boolean (bool)} & \code{\bool_new:N} / \code{\bool_set_true:N} & Wahrheitswert speichern. & \code{\bool_set_true:N \l_demo_bool} \\
 & \code{\bool_if:NTF} & Verzweigung nach Bool. & \code{\bool_if:NTF \l_demo_bool {A}{B}} \\
\midrule
\textbf{Gleitkomma (fp)} & \code{\fp_new:N} / \code{\fp_set:Nn} & Dezimalrechnung (l3fp). & \code{\fp_set:Nn \l_demo_fp { 3.1415 }} \\
 & \code{\fp_eval:n} & expandierbare Auswertung. & \code{\fp_eval:n { sqrt(2) }} \\
\midrule
\textbf{String (str)} & \code{\str_new:N} / \code{\str_set:Nn} & String \& Operationen (l3str). & \code{\str_set:Nn \l_demo_str {Hallo}} \\
 & \code{\str_if_eq:nnT} & Stringvergleich. & \code{\str_if_eq:nnT {a}{a}{gleich!}} \\
\midrule
\textbf{Box (box)} & \code{\box_new:N} / \code{\box_set_eq:NN} & Boxen anlegen/zuweisen. & \code{\box_new:N \l_demo_box} \\
 &  &  & \code{\box_set_eq:NN \l_demo_box \c_empty_box} \\
\midrule
\textbf{Regex} & \code{\regex_replace_all:nnN {|a|}{b}{\l_demo_tl}} & Reguläre Ausdrücke auf tl anwenden. & \code{\regex_replace_all:nnN {|a|}{b}{\l_demo_tl}} \\
\midrule
\textbf{IO-Streams} & \code{\iow_new:N} / \code{\iow_open:Nn} & Schreiben in Datei/Stream. & \code{\iow_open:Nn \l_demo_iow {demo.txt}} \\
 & \code{\iow_now:} & Sofort schreiben. & \code{\iow_now: {Hallo Welt!}} \\
\bottomrule
\end{xltabular}
\normalsize

\section{Typische Muster}
\subsection{Lokale vs. globale Änderungen}
\begin{tabularx}{\linewidth}{L{4cm} X}
\toprule
\textbf{Lokal} & \code{\seq_put_right:Nn \l_demo_seq {x}} (gilt in aktueller Gruppe) \\
\textbf{Global} & \code{\seq_gput_right:Nn \l_demo_seq {x}} (wirkt über Gruppen hinaus) \\
\bottomrule
\end{tabularx}

\subsection{Mapping/Iteration}
\begin{verbatim}
\seq_map_inline:Nn \l_demo_seq { \tl_show:n {Element:~#1} }
\prop_map_inline:Nn \l_demo_prop { \iow_term:x {#1 = #2} }
\end{verbatim}

\section{Argument-Signaturen (Kurzüberblick)}
Die Suffixe hinter Doppelpunkten geben an, wie Argumente übernommen/expandiert werden. Eine Minimalübersicht (ohne Vollständigkeitsanspruch):
\begin{description}
  \item[\texttt{N}] erwartet eine einzelne Kontrollsequenz (Variablen-/Funktionsname).
  \item[\texttt{n}] übernimmt die übergebene Tokenliste unverändert (keine Expansion).
  \item[\texttt{V}/\texttt{v}] Wert einer Variablen (\texttt{V:} Name direkt; \texttt{v:} Name wird zuerst aus Tokenliste gebildet).
  \item[\texttt{c}] bildet eine Kontrollsequenz aus Zeichen (\enquote{csname}-Art).
  \item[\texttt{o}/\texttt{x}/\texttt{f}] unterschiedliche Stufen der Expansion (einmal / voll / funktionsorientiert).\footnote{Details siehe \texttt{interface3.pdf} bzw. \texttt{source3} – hier nur als Merkhilfe aufgeführt.}
\end{description}

\section{Kleine Beispiele}
\subsection{Kommaliste in Sequenz umwandeln}
\begin{verbatim}
\clist_set:Nn \l_demo_clist {a,b,c}
\seq_set_from_clist:NN \l_demo_seq \l_demo_clist
\seq_use:Nn \l_demo_seq {,~}  % -> "a, b, c"
\end{verbatim}

\subsection{Property-Liste als Tabelle ausgeben}
\begin{verbatim}
\prop_clear:N \l_demo_prop
\prop_put:Nnn \l_demo_prop {name}{Ada}
\prop_put:Nnn \l_demo_prop {lang}{TeX}
\prop_map_inline:Nn \l_demo_prop { (#1 = #2) }
\end{verbatim}

\section{Eigene Ergänzungen}
Diese Datei ist als lebender Spickzettel gedacht. Weitere Module (\texttt{l3keys}, \texttt{l3tl-build}, \texttt{l3str}, \texttt{l3fp}, usw.) können in denselben Tabellenstil ergänzt werden.

\bigskip
\noindent\textit{Hinweis:} Diese Übersicht erhebt keinen Anspruch auf Vollständigkeit, sondern dient als kompakter Merkzettel der im Alltag häufig genutzten Konstrukte.

\end{document}