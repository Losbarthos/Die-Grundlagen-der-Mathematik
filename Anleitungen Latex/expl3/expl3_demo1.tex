%
% expl3_demo1.tex
%

\documentclass[a4paper,11pt]{article}
\usepackage[T1]{fontenc}
\usepackage[utf8]{inputenc}
\usepackage[ngerman]{babel}
\usepackage{microtype}
\usepackage{amsmath}
\usepackage{csquotes}
\usepackage{xparse}
\usepackage{expl3}
\usepackage{geometry}
\geometry{margin=2.2cm}
\usepackage{listings}
\lstset{language=[LaTeX]TeX,basicstyle=\ttfamily\small,columns=fullflexible,
        numbers=left,numberstyle=\scriptsize,numbersep=8pt,breaklines=true}

% --- expl3-Teil: Daten sammeln und ausgeben ---
\ExplSyntaxOn
\seq_new:N \g_demo_todo_seq % globale Sequenz: überlebt Umgebungen

\cs_new_protected:Nn \demo_todo_add:n
  { \seq_gput_right:Nn \g_demo_todo_seq { #1 } } % Eintrag rechts anhängen

\NewDocumentCommand \todo { m }
  {
    \textbf{TODO:}~#1 % sichtbare Ausgabe
    \demo_todo_add:n { #1 } % und parallel intern sammeln
  }

\NewDocumentCommand \printtodos { }
  {
    \par\smallskip
    \textbf{Gesammelte TODOs: }
    \seq_if_empty:NTF \g_demo_todo_seq
      { \emph{keine} }
      { \seq_use:Nn \g_demo_todo_seq { ;\space } }
  }

\NewDocumentCommand \cleartodos { }
  { \seq_gclear:N \g_demo_todo_seq } % Liste leeren

\ExplSyntaxOff
% --- Ende expl3-Teil ---

\title{Kleines \texttt{expl3}-Beispiel mit Zeile-für-Zeile-Erklärung}
\author{}
\date{}

\begin{document}
\maketitle

\section*{1.\;Demo-Ausgabe}
Hier ein kurzer Text mit zwei Aufgaben. Die \verb|\todo|-Einträge werden im Text angezeigt und intern gesammelt:
\begin{itemize}
  \item \todo{Notation in \S{}2 angleichen}
  \item \todo{Beweis von Lemma 1.3 ergänzen}
\end{itemize}

\printtodos % beide TODOs erscheinen gesammelt

\cleartodos % Liste leeren

\printtodos % jetzt ist die Liste leer

\bigskip
\noindent\rule{\linewidth}{0.4pt}

\section*{2.\;Quelltext des Beispiels}
Nachfolgend der komplette Quelltext des oben laufenden Beispiels, mit Zeilennummern. So kannst du \enquote{sehen}, was \emph{genau} im Dokument definiert ist.
\medskip

\begin{lstlisting}
\documentclass[a4paper,11pt]{article}
\usepackage[T1]{fontenc}
\usepackage[utf8]{inputenc}
\usepackage[ngerman]{babel}
\usepackage{microtype}
\usepackage{amsmath}
\usepackage{csquotes}
\usepackage{xparse}
\usepackage{expl3}

% --- expl3-Teil: Daten sammeln und ausgeben ---
\ExplSyntaxOn
\seq_new:N \g_demo_todo_seq 

\cs_new_protected:Nn \demo_todo_add:n
  { \seq_gput_right:Nn \g_demo_todo_seq { #1 } } 

\NewDocumentCommand \todo { m }
  {
    \textbf{TODO:}~#1 % sichtbare Ausgabe
    \demo_todo_add:n { #1 } % und parallel intern sammeln
  }

\NewDocumentCommand \printtodos { }
  {
    \par\smallskip
    \textbf{Gesammelte TODOs: }
    \seq_if_empty:NTF \g_demo_todo_seq
      { \emph{keine} }
      { \seq_use:Nn \g_demo_todo_seq { ;\space } }
  }

\NewDocumentCommand \cleartodos { }
  { \seq_gclear:N \g_demo_todo_seq } % Liste leeren

\ExplSyntaxOff
% --- Ende expl3-Teil ---

\begin{document}
\section*{Kleines \texttt{expl3}-Beispiel}
Hier ein kurzer Text mit zwei Aufgaben:
\begin{itemize}
  \item \todo{Notation in \S{}2 angleichen}
  \item \todo{Beweis von Lemma  1.3 ergaenzen}
\end{itemize}

\printtodos % beide TODOs erscheinen gesammelt

\cleartodos % Liste leeren

\printtodos % jetzt ist die Liste leer
\end{document}
\end{lstlisting}

\bigskip
\noindent\rule{\linewidth}{0.4pt}

\section*{3.\;Erklärung Zeile für Zeile (für \enquote{expl3}-Neulinge)}
Im Folgenden wird \emph{jede} Zeile des Quelltexts kurz, aber verständlich kommentiert. Begriffe wie \enquote{Sequenz} meinen in \texttt{expl3} eine Art \enquote{Liste}. \texttt{expl3} stellt saubere, moderne Programmierbausteine für \LaTeX{} bereit.

\begin{description}
 \item[\textbf{1} \texttt{\string\documentclass[a4paper,11pt]\{article\}}]
  Wählt die Dokumentklasse \enquote{article} mit A4-Seite und 11\,pt Schriftgröße.

 \item[\textbf{2} \texttt{\string\usepackage[T1]\{fontenc\}}]
  T1-Schriftkodierung – wichtig für Umlaute und korrekte Silbentrennung in europäischen Sprachen.

\item[\textbf{3} \texttt{\string\usepackage[utf8]\{inputenc\}}]
  Sagt \LaTeX{}, dass deine Quelldatei in UTF-8 gespeichert ist (bei modernen Engines teils entbehrlich, schadet aber nicht).

\item[\textbf{4} \texttt{\string\usepackage[ngerman]\{babel\}}]
  Deutsche Lokalisierung (Trennung, Datum, Kapitelüberschriften \dots).

\item[\textbf{5} \texttt{\string\usepackage\{microtype\}}]
  Mikrotypografie für schöneren Satz (z.\,B. Zeichenprotrusion, Laufweitenfeinheiten).

\item[\textbf{6} \texttt{\string\usepackage\{amsmath\}}]
  Mathematische Umgebungen und Symbole (u.\,a. wird hier \verb|\S| genutzt).

\item[\textbf{7} \texttt{\string\usepackage\{csquotes\}}]
  Komfortable, sprachsensitive Anführungszeichen via \verb|\enquote{...}|.

\item[\textbf{8} \texttt{\string\usepackage\{xparse\}}]
  Stellt \verb|\NewDocumentCommand| bereit – moderne Befehlsdefinitionen mit klarer Argument-Spezifikation.

\item[\textbf{9} \texttt{\string\usepackage\{expl3\}}]
  Lädt die \enquote{L3-Programmierung}: moderne Makro-Syntax und Datentypen (z.\,B. Sequenzen).

\item[\textbf{11} \texttt{\string\ExplSyntaxOn}]
  Aktiviert die \texttt{expl3}-Syntax mit Doppelpunkten/Unterstrichen in Befehlsnamen (z.\,B. \verb|\seq_use:Nn|).

\item[\textbf{12} \texttt{\textbackslash seq\_new:N \textbackslash g\_demo\_todo\_seq}]
  Legt eine \emph{globale} Sequenz \texttt{\textbackslash g\_demo\_todo\_seq} an. 
  \enquote{Global} bedeutet, dass die Sequenz auch dann erhalten bleibt, wenn eine Gruppe oder Umgebung endet. 
  In \LaTeX{} werden Gruppen zum Beispiel durch geschweifte Klammern \texttt{\{...\}} oder durch Umgebungen wie 
  \texttt{itemize}, \texttt{figure} oder \texttt{table} gebildet. Alles, was darin lokal definiert wird, 
  verschwindet beim Verlassen der Gruppe wieder – außer es wurde explizit als \enquote{global} angelegt.


\item[\textbf{14--15} \texttt{\textbackslash cs\_new\_protected:Nn \textbackslash demo\_todo\_add:n \{ ... \}}]
  Definiert eine \emph{geschützte} (nicht erweiterbare) Hilfsfunktion mit \emph{einem} Argument. Innen: \verb|\seq_gput_right:Nn| hängt ein Element rechts an die globale Sequenz an.

\item[\textbf{17--22} \texttt{\string\NewDocumentCommand \string\todo \{ m \} \{ ... \}}]
  Definiert den Nutzerbefehl \verb|\todo{<Text>}|. Er macht zwei Dinge:
  (a) gibt \enquote{\textbf{TODO:} <Text>} im Dokument aus, (b) speichert denselben Text zusätzlich intern in der Sequenz.
  So trennst du sichtbare Ausgabe von Datenhaltung.

\item[\textbf{24--32} \texttt{\string\NewDocumentCommand \string\printtodos \{ \} \{ ... \}}]
  Gibt die intern gesammelten Todos aus. \verb|\seq_if_empty:NTF| prüft, ob die Sequenz leer ist.
  Ist sie leer, wird \enquote{keine} gedruckt; sonst \verb|\seq_use:Nn ... { ;\space }|, das alle Elemente mit \enquote{;} verbindet.

\item[\textbf{34--35} \texttt{\textbackslash NewDocumentCommand \textbackslash cleartodos \{ \} \{ \textbackslash seq\_gclear:N ... \}}]
  Leert die Sequenz vollständig. Danach ist die Liste der Todos wieder leer.

\item[\textbf{37} \texttt{\string\ExplSyntaxOff}]
  Schaltet die \texttt{expl3}-Sondersyntax wieder ab – normale \LaTeX-Syntax gilt.

\item[\textbf{41} \texttt{\string\section* \{Kleines \string\texttt\{expl3\}-Beispiel\}}]
  Überschrift für den Beispielteil (ohne Nummerierung).

\item[\textbf{42--47} Umgebung \texttt{itemize} + zwei \texttt{\string\todo}-Aufrufe]
  Zwei Stichpunkte demonstrieren die Wirkung: Jeder \verb|\todo| druckt Text und sammelt den Eintrag intern.

\item[\textbf{49} \texttt{\string\printtodos}]
  Gibt alle bisher gesammelten Todos in einer Zeile aus (durch \enquote{;} getrennt).

\item[\textbf{51} \texttt{\string\cleartodos}]
  Leert die Sequenz: Alle Einträge werden verworfen.

\item[\textbf{53} \texttt{\string\printtodos} (erneut)]
  Nach dem Leeren ist die Liste leer – es erscheint \enquote{keine}.

  \item[Allgemeiner Hinweis]
  \textbf{Warum \texttt{expl3}?} Du bekommst strukturierte Datentypen (Sequenzen, Tokenlisten \dots), klar definierte, wiederverwendbare Funktionen und robuste Prüf-/Verzweigungsbefehle. Das macht Dokumentlogik übersichtlich und wartbar.
\end{description}

\end{document}