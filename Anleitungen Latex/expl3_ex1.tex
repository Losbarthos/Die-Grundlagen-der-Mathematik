\documentclass{article}

\usepackage[T1]{fontenc}
\usepackage[utf8]{inputenc}
\usepackage{lmodern}
\usepackage{expl3}
\usepackage{array}     % für p{..}-Spalten
\usepackage{csquotes}  % für \enquote

% Robuster Inline-Code für Tabellen/Fließtext
\newcommand{\Code}[1]{\texttt{\detokenize{#1}}}

\begin{document}

\section*{Beispiel: Unterschied zwischen \Code{\cs_new_protected:Npn} und \Code{\cs_new_protected:Npx}}

% --- Hier steht der gesamte Beispielcode ---
\begin{verbatim}
\ExplSyntaxOn
\tl_new:N \l_my_demo_tl
\tl_set:Nn \l_my_demo_tl {ABC}
\cs_new_protected:Npn \foo:n { \l_my_demo_tl } % Body NICHT expandiert
\cs_new_protected:Npx \bar:n { \l_my_demo_tl } % Body BEI Definition expandiert
\tl_set:Nn \l_my_demo_tl {XYZ}
\ExplSyntaxOff
\end{verbatim}

% --- Definitionen tatsächlich ausführen, damit wir unten Ergebnisse haben ---
\ExplSyntaxOn
\tl_new:N \l_my_demo_tl
\tl_set:Nn \l_my_demo_tl {ABC}
\cs_new_protected:Npn \foo:n { \l_my_demo_tl }
\cs_new_protected:Npx \bar:n { \l_my_demo_tl }
\tl_set:Nn \l_my_demo_tl {XYZ}
\ExplSyntaxOff

\bigskip
\noindent\textbf{Zeile-für-Zeile-Erklärung:}

\renewcommand{\arraystretch}{1.25}
\begin{tabular}{@{}p{0.48\linewidth}p{0.48\linewidth}@{}}
\Code{\ExplSyntaxOn} &
Schaltet die \Code{expl3}-Syntax ein. \\

\Code{\tl_new:N \l_my_demo_tl} &
Legt eine Tokenliste \Code{\l_my_demo_tl} an (\Code{l_} = lokal). \\

\Code{\tl_set:Nn \l_my_demo_tl {ABC}} &
Weist den Anfangswert \enquote{ABC} zu. \\

\Code{\cs_new_protected:Npn \foo:n { \l_my_demo_tl }} &
Definiert \Code{\foo:n}. Body wird \emph{nicht} bei der Definition expandiert –
beim Aufruf wird der \emph{aktuelle} Wert der Tokenliste verwendet. \\

\Code{\cs_new_protected:Npx \bar:n { \l_my_demo_tl }} &
Definiert \Code{\bar:n}. Body wird \emph{bei der Definition} einmal expandiert –
der zu diesem Zeitpunkt gültige Wert (\enquote{ABC}) wird \enquote{eingebacken}. \\

\Code{\tl_set:Nn \l_my_demo_tl {XYZ}} &
Ändert den Inhalt der Tokenliste auf \enquote{XYZ}. \\

\Code{\foo:n \{\}} &
Verwendet den aktuellen Wert (\enquote{XYZ}). \\

\Code{\bar:n \{\}} &
Gibt den bei der Definition gespeicherten Wert (\enquote{ABC}) aus. \\

\Code{\ExplSyntaxOff} &
Beendet die \Code{expl3}-Syntax. \\
\end{tabular}

\bigskip
\noindent\textbf{Ausgabe im Dokument:}\\
\ExplSyntaxOn
\noindent \Code{\foo:n \{\}} $\to$ \foo:n {} \\
\noindent \Code{\bar:n \{\}} $\to$ \bar:n {} \\
\ExplSyntaxOff

\end{document}
