% tex/impl/thmlookup.tex  (stabiler Stand)

\RequirePackage{xparse}
\RequirePackage{luacode}

% Lua-Modul laden
\directlua{
  if not thmlookup or not thmlookup.ref_by_structure then
    thmlookup = dofile("tex/impl/thmlookup.lua")
  end
}

% ---- Ausgabeformat für Referenzen ----
\makeatletter
\providecommand{\ThmLookupPrefix@theorem}{Th.}
\providecommand{\ThmLookupPrefix@definition}{Def.}
\providecommand{\ThmLookupPrefix@axiom}{Ax.}
\providecommand{\ThmLookupPrefix@fallback}{Ref.}
\makeatother

% \ThmLookupEmitRef{<env>}{<label>} -> "Th. 1.1" (verlinkt via \ref)
\NewDocumentCommand{\ThmLookupEmitRef}{m m}{%
  \ifcsname ThmLookupPrefix@#1\endcsname
    \csname ThmLookupPrefix@#1\endcsname
  \else
    \ThmLookupPrefix@fallback
  \fi
  ~\ref{#2}%
}

% \ThmLookupRegister{<label>}{<env>}{<number>}{<statement>}{<title>}
\NewDocumentCommand{\ThmLookupRegister}{m m m m g}{%
  \begingroup
    \def\mxTitle{}%
    \IfNoValueF{#5}{\def\mxTitle{#5}}%

    % sanft expandieren (z.B. \mxKey), aber keine TeX-Logik in Lua
    \edef\mxStmt{\unexpanded\expandafter{#4}}%
    \csname protected@edef\endcsname\mxLabel{#1}%
    \csname protected@edef\endcsname\mxEnv{#2}%
    \csname protected@edef\endcsname\mxNum{#3}%

    % INHALT detokenize'n (nicht den Makronamen)
    \edef\mxLabelStr{\expandafter\detokenize\expandafter{\mxLabel}}%
    \edef\mxEnvStr  {\expandafter\detokenize\expandafter{\mxEnv}}%
    \edef\mxNumStr  {\expandafter\detokenize\expandafter{\mxNum}}%
    \edef\mxStmtStr {\expandafter\detokenize\expandafter{\mxStmt}}%
    \edef\mxTitleStr{\expandafter\detokenize\expandafter{\mxTitle}}%

    \edef\mxLuaCode{%
      thmlookup.register(
        "\luaescapestring{\mxLabelStr}",
        "\luaescapestring{\mxEnvStr}",
        "\luaescapestring{\mxNumStr}",
        "\luaescapestring{\mxStmtStr}",
        "\luaescapestring{\mxTitleStr}"
      )%
    }%
    \directlua{\mxLuaCode}%
  \endgroup
}

% \ThmLookupRegisterId{<id>}{<env>}{<number>}
\NewDocumentCommand{\ThmLookupRegisterId}{m m m}{%
  \begingroup
    \csname protected@edef\endcsname\mxId{#1}%
    \csname protected@edef\endcsname\mxEnv{#2}%
    \csname protected@edef\endcsname\mxNum{#3}%

    \edef\mxIdStr {\expandafter\detokenize\expandafter{\mxId}}%
    \edef\mxEnvStr{\expandafter\detokenize\expandafter{\mxEnv}}%
    \edef\mxNumStr{\expandafter\detokenize\expandafter{\mxNum}}%

    \edef\mxLuaCode{%
      thmlookup.register_id(
        "\luaescapestring{\mxIdStr}",
        "\luaescapestring{\mxEnvStr}",
        "\luaescapestring{\mxNumStr}"
      )%
    }%
    \directlua{\mxLuaCode}%
  \endgroup
}

% \ThmRefByStructure{<expr>}  bzw. \ThmRefByStructure*{<expr>}
\NewDocumentCommand{\ThmRefByStructure}{s O{} m}{%
  \begingroup
    \def\mxStar{false}%
    \IfBooleanT{#1}{\def\mxStar{true}}%

    % KEINE Expansion: sonst wird z.B. \bigcup -> \DOTSB\bigcup@\slimits@ ...
    \edef\mxExprStr{\detokenize{#3}}%
    \edef\mxOptsStr{\detokenize{#2}}%

    \edef\mxLuaCode{%
      thmlookup.ref_by_structure(
        "\luaescapestring{\mxOptsStr}",
        "\luaescapestring{\mxExprStr}",
        \mxStar
      )%
    }%
    \directlua{\mxLuaCode}%
  \endgroup
}


% -------------------------
% Forward-Ref-Variante (auflösbar ab dem 2. Lauf via .aux)
% -------------------------
\makeatletter

% Wird aus der .aux gelesen und definiert die gecachte Referenz
\NewDocumentCommand{\mxFwdNeed}{m m m m}{%
  \directlua{
    thmlookup.define_ref_macro(
      "\luaescapestring{\detokenize{#1}}",
      "\luaescapestring{\detokenize{#2}}",
      "\luaescapestring{\detokenize{#3}}",
      #4
    )
  }%
}

% \ThmRefByStructureFwd{<expr>} bzw. \ThmRefByStructureFwd*{<expr>}
\NewDocumentCommand{\ThmRefByStructureFwd}{s O{} m}{%
  \begingroup
    \def\mxStarNum{0}%
    \IfBooleanT{#1}{\def\mxStarNum{1}}%

    \edef\mxExpr{\unexpanded\expandafter{#3}}%
    \edef\mxExprStr{\expandafter\detokenize\expandafter{\mxExpr}}%
    \edef\mxOptsStr{\detokenize{#2}}%

    % Query-Key in \mxFwdQKey schreiben lassen
    \directlua{
      local q = thmlookup.query_key("\luaescapestring{\mxOptsStr}", "\luaescapestring{\mxExprStr}")
      tex.sprint("\\gdef\\mxFwdQKey{" .. q .. "}")
    }%

    % immer in .aux schreiben
    \protected@write\@auxout{}{%
      \string\mxFwdNeed{\mxFwdQKey}{\mxOptsStr}{\mxExprStr}{\mxStarNum}%
    }%
    
    % nur falls noch nicht definiert: sofort versuchen
    \expandafter\ifx\csname mxFwdRef@\mxFwdQKey\endcsname\relax
      \mxFwdNeed{\mxFwdQKey}{\mxOptsStr}{\mxExprStr}{\mxStarNum}%
    \fi
    
    \csname mxFwdRef@\mxFwdQKey\endcsname


  \endgroup
}

\makeatother


\NewDocumentCommand{\ThmRefById}{m}{%
  \directlua{
    thmlookup.ref_by_id("\luaescapestring{\detokenize{#1}}")
  }%
}

% IMPORTANT: muss VOR dem Einlesen der .aux passieren, sonst bleiben Fwd-Refs "nicht gefunden"
\directlua{
  if not thmlookup.__prepared then
    thmlookup.prepare_run()
    thmlookup.__prepared = true
  end
}

