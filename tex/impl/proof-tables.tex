% tex/impl/proof-tables.tex
\RequirePackage{xparse}
\RequirePackage{array}
\RequirePackage{longtable}
\RequirePackage{amsthm} % für proof-Umgebung / qed

\providecommand{\axstrut}{\rule{0pt}{1.25\baselineskip}}
\providecommand{\axsep}{1pt}

% Spaltentyp für eingerückte Abhängigkeitslisten
\newcolumntype{G}{>{\raggedright\arraybackslash}l}


\newcounter{proofstepnr}

% ------------------------------------------------------------
% Induktions-Teile mit Referenz (thmlookup)
% Nummerierung: <aktuelles Theorem>(i), (ii), ...
% ------------------------------------------------------------

\newcounter{proofpartnr}

% Standard: an aktuelles Theorem koppeln
\providecommand{\mxProofPartParent}{\theformulaThm}
\NewDocumentCommand{\ProofPartSetParent}{m}{%
  \def\mxProofPartParent{#1}%
}

\renewcommand{\theproofpartnr}{\mxProofPartParent(\roman{proofpartnr})}

% interner Helfer: Teil registrieren + Header drucken
% Signatur wie \FormulaThmAuto: [Titel]{Anzeigeformel}[Keyformel][ID]
\NewDocumentCommand{\mxProofPartOpen}{m o m o o}{%
  % #1 = Tabellenpräambel, #2 = optional Titel, #3 = Formel (Anzeige), #4 = optional Key, #5 = optional ID

  \par\medskip
  \refstepcounter{proofpartnr}%

  % Label (stabil/ASCII; ohne Klammern/roman im Labelnamen)
  \edef\mxPPLbl{thm:pp:\mxProofPartParent:\arabic{proofpartnr}}%
  \label{\mxPPLbl}%

  % Key bestimmen (Default: Anzeigeformel)
  \def\mxKey{#3}%
  \IfNoValueF{#4}{%
    \IfBlankTF{#4}{}{%
      \def\mxKey{#4}%
    }%
  }%

  % Registrierung:
  % - Wenn ID angegeben: nur ID->Label registrieren (keine Strukturregistrierung => keine Duplikat-Key-Probleme)
  % - Sonst: Struktur registrieren (wie bei FormulaThmAuto), Referenz dann über \FormulaRefAuto{<Keyformel/Formel>}
  \IfNoValueTF{#5}{%
    \IfNoValueTF{#2}
      {\ThmLookupRegister{\mxPPLbl}{theorem}{\theproofpartnr}{\mxKey}}
      {\ThmLookupRegister{\mxPPLbl}{theorem}{\theproofpartnr}{\mxKey}{#2}}%
  }{%
    \IfBlankTF{#5}{}{%
      \ThmLookupRegisterId{#5}{theorem}{\mxPPLbl}%
    }%
  }%

  % Header: optional Titel links, eigene Referenz rechts
  \noindent
  \IfNoValueTF{#2}{\textbf{Teil}}{\textbf{#2}}%
  \hfill \ThmLookupEmitRef{theorem}{\mxPPLbl}%
  \par
  \[
    #3
  \]
  \par\small

  % Tabellen-Setup
  \setcounter{proofstepnr}{0}%
  \setlength{\tabcolsep}{2pt}%
  \renewcommand{\arraystretch}{1.15}%
  \begin{longtable}{#1}%
}

\newcommand{\mxProofPartClose}{%
  \end{longtable}\par
}

% ------------------------------------------------------------
% ProofParts mit Delta-Kontext (analog zu \FormulaThmDeltaR)
% ------------------------------------------------------------
% Signatur:
% \mxProofPartOpenDelta{<Tabellenpräambel>}[<Titel>]{<Anzeigeformel>}{<DeltaRows>}[<Keyformel>][<ID>]
\NewDocumentCommand{\mxProofPartOpenDelta}{m o m +G{} o o}{%
  % #1 = Tabellenpräambel
  % #2 = optional Titel
  % #3 = Anzeigeformel
  % #4 = DeltaRows (optional, default leer)
  % #5 = optional Keyformel
  % #6 = optional ID

  \par\medskip
  \refstepcounter{proofpartnr}%

  \edef\mxPPLbl{thm:pp:\mxProofPartParent:\arabic{proofpartnr}}%
  \label{\mxPPLbl}%

  % Key bestimmen (Default: Anzeigeformel)
  \def\mxKey{#3}%
  \IfNoValueF{#5}{%
    \IfBlankTF{#5}{}{%
      \def\mxKey{#5}%
    }%
  }%

  % Registrierung (wie bei \mxProofPartOpen)
  \IfNoValueTF{#6}{%
    \IfNoValueTF{#2}
      {\ThmLookupRegister{\mxPPLbl}{theorem}{\theproofpartnr}{\mxKey}}
      {\ThmLookupRegister{\mxPPLbl}{theorem}{\theproofpartnr}{\mxKey}{#2}}%
  }{%
    \IfBlankTF{#6}{}{%
      \ThmLookupRegisterId{#6}{theorem}{\mxPPLbl}%
    }%
  }%

  % Header
  \noindent
  \IfNoValueTF{#2}{\textbf{Teil}}{\textbf{#2}}%
  \hfill \ThmLookupEmitRef{theorem}{\mxPPLbl}%
  \par

  % Delta-Block (analog \FormulaThmDeltaR: leerer Titel)
  \IfBlankTF{#4}{}{%
    \begin{DeltaContext}{ }%
      #4%
    \end{DeltaContext}%
  }%

  % Anzeigeformel
  \[
    #3
  \]
  \par\small

  % Tabellen-Setup
  \setcounter{proofstepnr}{0}%
  \setlength{\tabcolsep}{2pt}%
  \renewcommand{\arraystretch}{1.15}%
  \begin{longtable}{#1}%
}

% ----- Public API: Induktions-Parts (normal / wide) -----

% \proofpartindDelta[<Titel>]{<Formel>}{<DeltaRows>}[<Keyformel>][<ID>]
\NewDocumentCommand{\proofpartindDelta}{o m +G{} o o}{%
  \mxProofPartOpenDelta{@{}G r >{$}l<{$} >{$}l<{$}@{}}[#1]{#2}{#3}[#4][#5]%
}
\newcommand{\closeproofpartindDelta}{\mxProofPartClose}

% \proofpartwideindDelta[<Titel>]{<Formel>}{<DeltaRows>}[<Keyformel>][<ID>]
\NewDocumentCommand{\proofpartwideindDelta}{o m +G{} o o}{%
  \mxProofPartOpenDelta{@{}G r >{$}r<{$} >{$}c<{$} >{$}l<{$} @{\hfill} l@{}}[#1]{#2}{#3}[#4][#5]%
}
\newcommand{\closeproofpartwideindDelta}{\mxProofPartClose}

% Optional: "DeltaR"-ähnlicher Komfort-Wrapper (Key fest als 3. Argument)
% \proofpartwideindDeltaR[<Titel>]{<Formel>}{<Key>}{<DeltaRows>}
\NewDocumentCommand{\proofpartwideindDeltaR}{o m m +m}{%
  \proofpartwideindDelta[#1]{#2}{#4}[#3]%
}


% ----- Public API: normal / wide -----

% \proofpartind[<Titel>]{<Formel>}[<Keyformel>][<ID>]
\NewDocumentCommand{\proofpartind}{o m o o}{%
  \mxProofPartOpen{@{}G r >{$}l<{$} >{$}l<{$}@{}}[#1]{#2}[#3][#4]%
}
\newcommand{\closeproofpartind}{\mxProofPartClose}

% \proofpartwideind[<Titel>]{<Formel>}[<Keyformel>][<ID>]
\NewDocumentCommand{\proofpartwideind}{o m o o}{%
  \mxProofPartOpen{@{}G r >{$}r<{$} >{$}c<{$} >{$}l<{$} @{\hfill} l@{}}[#1]{#2}[#3][#4]%
}
\newcommand{\closeproofpartwideind}{\mxProofPartClose}


\newenvironment{tabproof}
  {\par\small
   \setcounter{proofstepnr}{0}%
   \setlength{\tabcolsep}{2pt}%
   \renewcommand{\arraystretch}{1.15}%
   \begin{longtable}{@{}G r >{$}l<{$} >{$}l<{$}@{}}}
  {\end{longtable}\par}


% ------------------------------------------------------------
% Split-Variante wie in der alten main.tex:
% tabproofsplit = nur proof-Wrapper
% \proofpart öffnet je Teil eine eigene longtable
% \closeproofpart schließt sie wieder
% ------------------------------------------------------------

\newenvironment{tabproofsplit}
  {\begin{proof}\leavevmode\par\nobreak
   \setcounter{proofpartnr}{0}%
   \ProofPartSetParent{\theformulaThm}%
  }
  {\end{proof}}


\newcommand{\proofpart}[1]{%
  \par\medskip\noindent #1:\par%
  \par\small
  \setcounter{proofstepnr}{0}%
  \setlength{\tabcolsep}{2pt}%
  \renewcommand{\arraystretch}{1.15}%
  \begin{longtable}{@{}G r >{$}l<{$} >{$}l<{$}@{}}
}

\newcommand{\closeproofpart}{%
  \end{longtable}\par
}

% -------------------- WIDE (analog) --------------------

\NewDocumentEnvironment{tabproofsplitwide}{}{% BEGIN
  \begin{proof}\leavevmode\par\nobreak
  \setcounter{proofpartnr}{0}%
  \ProofPartSetParent{\theformulaThm}%
}{% END
  \end{proof}
}


\newcommand{\proofpartwide}[1]{%
  \par\medskip\noindent #1:\par%
  \par\small
  \setcounter{proofstepnr}{0}%
  \setlength{\tabcolsep}{2pt}%
  \renewcommand{\arraystretch}{1.15}%
  \begin{longtable}{@{}G r >{$}r<{$} >{$}c<{$} >{$}l<{$} @{\hfill} l@{}}


}

\newcommand{\closeproofpartwide}{%
  \end{longtable}\par
}

% Alias: falls du irgendwo noch tabproofwidesplit verwendest
\NewDocumentEnvironment{tabproofwidesplit}{}{% BEGIN
  \begin{tabproofsplitwide}
}{% END
  \end{tabproofsplitwide}
}



% tex/impl/proof-tables.tex

% Wide: (Deps) | Nr | Links | Rel | Rechts | Begründung
\NewDocumentEnvironment{tabproofwide}{}{%
  \par\small
  \setcounter{proofstepnr}{0}%
  \setlength{\tabcolsep}{2pt}%
  \renewcommand{\arraystretch}{1.15}%
  \begin{longtable}{@{}G r >{$}r<{$} >{$}c<{$} >{$}l<{$} @{\hfill} l@{}}

}{%
  \end{longtable}\par
}

% \proofstepwide[<deps>]{<links>}{<rel>}{<rechts>}{<begruendung>}
% Sternform: keine neue Schritt-Nr. (Fortsetzungszeile)
\NewDocumentCommand{\proofstepwide}{s O{} m m m m}{%
  \IfBooleanTF{#1}
    {#2 &  & #3 & #4 & #5 & #6\\}%
    {\stepcounter{proofstepnr}%
     #2 & \theproofstepnr & #3 & #4 & #5 & #6\\}%
}

% Zeile nur "Formel + Begründung" (wie bei dir: \proofstepwidestar[1]{A\subseteq C}{\rA})
\ProvideDocumentCommand{\proofstepwidestar}{O{} m m}{%
  \stepcounter{proofstepnr}%
  #1 & \theproofstepnr & #2 &  &  & #3\\%
}

\RenewDocumentCommand{\proofstepwidestar}{O{} m m}{%
  \stepcounter{proofstepnr}%
  #1 & \theproofstepnr
     & \multicolumn{3}{@{}>{$}l<{$}@{}}{#2}
     & #3\\%
}




% Proofzeilen
\NewDocumentCommand{\proofstep}{m m m}{%
  \stepcounter{proofstepnr}%
  #1 & \theproofstepnr & #2 & \ensuremath{#3}\\%
}



% wie \proofstep, aber ohne Schritt-Nummer (für Folgezeilen, Umbrüche, etc.)
\ProvideDocumentCommand{\proofstepstar}{m m m}{%
  #1 &  & #2 & #3\\%
}
\ProvideDocumentCommand{\proofstepstarwide}{m m m}{%
  #1 &  & #2 & #3\\%
}


% Axiom-/Theorem-Referenzzeilen (nutzt FormulaRefAuto)
\NewDocumentCommand{\proofaxline}{m m o}{%
  \stepcounter{proofstepnr}%
  #1 & \theproofstepnr & #2 &
  \axstrut \FormulaRefAuto{#2}[ax]%
  \IfNoValueF{#3}{\axstrut \textnormal{\scriptsize(#3)}}\\[\axsep]
}

\newcommand{\proofcase}[2][]{%
  & & \text{\bfseries Fall\if\relax#1\relax\else~#1\fi: } #2 & \\[-.35ex]%
}
\newcommand{\proofcasesummary}[2][]{%
  & & \text{\bfseries Schluss\if\relax#1\relax\else~ Fall~#1\fi: } #2 & \\[-.35ex]%
}
\newcommand{\proofcaserule}{\cline{3-4}\noalign{\vspace{-0.3ex}}}

