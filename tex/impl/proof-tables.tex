% tex/impl/proof-tables.tex
\RequirePackage{xparse}
\RequirePackage{array}
\RequirePackage{longtable}
\RequirePackage{amsthm} % für proof-Umgebung / qed

\providecommand{\axstrut}{\rule{0pt}{1.25\baselineskip}}
\providecommand{\axsep}{1pt}

% Spaltentyp für eingerückte Abhängigkeitslisten
\newcolumntype{G}{>{\raggedright\arraybackslash}l}


\newcounter{proofstepnr}

\newenvironment{tabproof}
  {\par\small
   \setcounter{proofstepnr}{0}%
   \setlength{\tabcolsep}{2pt}%
   \renewcommand{\arraystretch}{1.15}%
   \begin{longtable}{@{}G r >{$}l<{$} >{$}l<{$}@{}}}
  {\end{longtable}\par}


% ------------------------------------------------------------
% Split-Variante wie in der alten main.tex:
% tabproofsplit = nur proof-Wrapper
% \proofpart öffnet je Teil eine eigene longtable
% \closeproofpart schließt sie wieder
% ------------------------------------------------------------

\newenvironment{tabproofsplit}
  {\begin{proof}\leavevmode\par\nobreak}
  {\end{proof}}


\newcommand{\proofpart}[1]{%
  \par\medskip\noindent #1:\par%
  \par\small
  \setcounter{proofstepnr}{0}%
  \setlength{\tabcolsep}{2pt}%
  \renewcommand{\arraystretch}{1.15}%
  \begin{longtable}{@{}G r >{$}l<{$} >{$}l<{$}@{}}
}

\newcommand{\closeproofpart}{%
  \end{longtable}\par
}

% -------------------- WIDE (analog) --------------------

\NewDocumentEnvironment{tabproofsplitwide}{}{% BEGIN
  \begin{proof}\leavevmode\par\nobreak
}{% END
  \end{proof}
}

\newcommand{\proofpartwide}[1]{%
  \par\medskip\noindent #1:\par%
  \par\small
  \setcounter{proofstepnr}{0}%
  \setlength{\tabcolsep}{2pt}%
  \renewcommand{\arraystretch}{1.15}%
  \begin{longtable}{@{}G r >{$}p{0.34\textwidth}<{$} >{$}c<{$} >{$}p{0.34\textwidth}<{$} l@{}}
}

\newcommand{\closeproofpartwide}{%
  \end{longtable}\par
}

% Alias: falls du irgendwo noch tabproofwidesplit verwendest
\NewDocumentEnvironment{tabproofwidesplit}{}{% BEGIN
  \begin{tabproofsplitwide}
}{% END
  \end{tabproofsplitwide}
}



% tex/impl/proof-tables.tex

% Wide: (Deps) | Nr | Links | Rel | Rechts | Begründung
\NewDocumentEnvironment{tabproofwide}{}{%
  \par\small
  \setcounter{proofstepnr}{0}%
  \setlength{\tabcolsep}{2pt}%
  \renewcommand{\arraystretch}{1.15}%
  \begin{longtable}{@{}G r >{$}p{0.34\textwidth}<{$} >{$}c<{$} >{$}p{0.34\textwidth}<{$} l@{}}
}{%
  \end{longtable}\par
}

% \proofstepwide[<deps>]{<links>}{<rel>}{<rechts>}{<begruendung>}
% Sternform: keine neue Schritt-Nr. (Fortsetzungszeile)
\NewDocumentCommand{\proofstepwide}{s O{} m m m m}{%
  \IfBooleanTF{#1}
    {#2 &  & #3 & #4 & #5 & #6\\}%
    {\stepcounter{proofstepnr}%
     #2 & \theproofstepnr & #3 & #4 & #5 & #6\\}%
}

% Zeile nur "Formel + Begründung" (wie bei dir: \proofstepwidestar[1]{A\subseteq C}{\rA})
\ProvideDocumentCommand{\proofstepwidestar}{O{} m m}{%
  \stepcounter{proofstepnr}%
  #1 & \theproofstepnr & #2 &  &  & #3\\%
}



% Proofzeilen
\NewDocumentCommand{\proofstep}{m m m}{%
  \stepcounter{proofstepnr}%
  #1 & \theproofstepnr & #2 & \ensuremath{#3}\\%
}



% wie \proofstep, aber ohne Schritt-Nummer (für Folgezeilen, Umbrüche, etc.)
\ProvideDocumentCommand{\proofstepstar}{m m m}{%
  #1 &  & #2 & #3\\%
}
\ProvideDocumentCommand{\proofstepstarwide}{m m m}{%
  #1 &  & #2 & #3\\%
}


% Axiom-/Theorem-Referenzzeilen (nutzt FormulaRefAuto)
\NewDocumentCommand{\proofaxline}{m m o}{%
  \stepcounter{proofstepnr}%
  #1 & \theproofstepnr & #2 &
  \axstrut \FormulaRefAuto{#2}[ax]%
  \IfNoValueF{#3}{\axstrut \textnormal{\scriptsize(#3)}}\\[\axsep]
}

\newcommand{\proofcase}[2][]{%
  & & \text{\bfseries Fall\if\relax#1\relax\else~#1\fi: } #2 & \\[-.35ex]%
}
\newcommand{\proofcasesummary}[2][]{%
  & & \text{\bfseries Schluss\if\relax#1\relax\else~ Fall~#1\fi: } #2 & \\[-.35ex]%
}
\newcommand{\proofcaserule}{\cline{3-4}\noalign{\vspace{-0.3ex}}}

