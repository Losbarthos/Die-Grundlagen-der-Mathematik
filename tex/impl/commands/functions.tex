% tex/impl/commands/functions.tex
\ProvidesFile{functions.tex}[2026/01/04 Function-theoretic notation]

% ------------------------------------------------------------
% Pfeile für Abbildungen
% ------------------------------------------------------------
\providecommand{\inj}{\rightarrowtail}           % injektiv
\providecommand{\sur}{\twoheadrightarrow}        % surjektiv
\providecommand{\bij}{\xrightarrow{\sim}}        % bijektiv

% ------------------------------------------------------------
% Standard-Namen / Operatoren
% ------------------------------------------------------------
\providecommand{\Id}{\mathrm{id}}                % Identitäts-Operator (z.B. \Id_A)
\providecommand{\Graph}{\ensuremath{\mathrm{Graph}}}

% ------------------------------------------------------------
% Projekt-spezifische Operatoren (als MathOperator wie bei dir)
% ------------------------------------------------------------
\DeclareMathOperator{\Ausw}{Ausw}                % Auswahlmenge/Operator
\DeclareMathOperator{\Fib}{Fib}                  % Faser (Operator)
\DeclareMathOperator{\FibFam}{FibFam}            % Faserfamilie (falls genutzt)
\DeclareMathOperator{\Sec}{Sec}                  % Sektion / Rechtsinverse (z.B. \Sec_{G,L})

% ------------------------------------------------------------
% Graph einer Surjektion aus einer Injektion (mit Stützelement a_0)
% ------------------------------------------------------------
\DeclareRobustCommand{\Gsurjfrominj}[2]{G_{#1,#2}}

% commands/functions.tex
\DeclareMathOperator{\Succ}{succ} 
\DeclareMathOperator{\Pred}{pred}
