% tex/impl/commands/logic.tex
\ProvidesFile{logic.tex}[2026/01/04 Logical separators and symbols]

% ------------------------------------------------------------------
% Trennzeichen für Annahmen (dein \dsep)
% ------------------------------------------------------------------
\providecommand{\dsep}{,\allowbreak\ }

% ------------------------------------------------------------------
% Äquivalenz im Kalkül (dein \eqvdash)
% ------------------------------------------------------------------
\providecommand{\eqvdash}{\dashv\vdash}

% ------------------------------------------------------------------
% XOR (falls du es als Operator nutzt)
% ------------------------------------------------------------------
\providecommand*\lxor{\mathbin{\veebar}}

% ------------------------------------------------------------------
% Optional: kleine Hilfsmakros, die du häufig in Formeln brauchst
% (nur aufnehmen, wenn du sie wirklich verwendest)
% ------------------------------------------------------------------
\providecommand{\st}{\,\middle|\,} % für Mengenschreibweise {x \in A \st P(x)}

% ------------------------------------------------------------------
% Optional: konsistente Klammer-Makros (falls du das magst)
% ------------------------------------------------------------------
\providecommand{\paren}[1]{\ensuremath{\left(#1\right)}}
\providecommand{\brak}[1]{\ensuremath{\left[#1\right]}}
\providecommand{\set}[1]{\ensuremath{\left\{#1\right\}}}
\providecommand{\st}{\ensuremath{\,\middle|\,}}

