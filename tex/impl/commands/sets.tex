% tex/impl/commands/sets.tex
\ProvidesFile{sets.tex}[2026/01/04 Set-theoretic notation]

% Klammerhilfen: funktionieren auch in Textmodus
\providecommand{\paren}[1]{\ensuremath{\left(#1\right)}}
\providecommand{\brak}[1]{\ensuremath{\left[#1\right]}}
\providecommand{\set}[1]{\ensuremath{\left\{#1\right\}}}

% Potenzmenge
\providecommand{\powerset}{\ensuremath{\mathcal{P}}}

\makeatletter
% Mitgliedschaftsrelation: \MemRel  oder \MemRel(M)
\def\MemRel{%
  \@ifnextchar({\MemRel@args}{\MemRel@name}%
}
\def\MemRel@name{%
  \ensuremath{R}%
}
\def\MemRel@args(#1){%
  \ensuremath{R_{#1}}%
}

% (Falls du sie brauchst) CoverFam / PartFam: \CoverFam oder \CoverFam(M,A)
\def\CoverFam{%
  \@ifnextchar({\CoverFam@args}{\CoverFam@name}%
}
\def\CoverFam@name{%
  \ensuremath{\mathsf{CoverFam}}%
}
\def\CoverFam@args(#1,#2){%
  \ensuremath{\mathsf{CoverFam}\!\left(#1,#2\right)}%
}

\def\PartFam{%
  \@ifnextchar({\PartFam@args}{\PartFam@name}%
}
\def\PartFam@name{%
  \ensuremath{\mathsf{PartFam}}%
}
\def\PartFam@args(#1,#2){%
  \ensuremath{\mathsf{PartFam}\!\left(#1,#2\right)}%
}

% Vereinigungsabgeschlossene Familie: \UnionClosedFam oder \UnionClosedFam(M)
\def\UnionClosedFam{%
  \@ifnextchar({\UnionClosedFam@args}{\UnionClosedFam@name}%
}
\def\UnionClosedFam@name{%
  \ensuremath{\mathsf{UnionClosedFam}}%
}
\def\UnionClosedFam@args(#1){%
  \ensuremath{\mathsf{UnionClosedFam}\!\left(#1\right)}%
}
\makeatother

% ------------------------------------------------------------
% Kardinalität / Gleichmächtigkeit
% ------------------------------------------------------------
\providecommand{\EqCard}{\mathrel{\approx}}


% commands/sets.tex
\DeclareMathOperator{\Induktiv}{Induktiv}
