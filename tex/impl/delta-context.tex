% tex/impl/delta-context.tex
\RequirePackage{xparse}
\RequirePackage{array}

% Optional: Styling zentral anpassbar
\newcommand{\DeltaLabelStyle}[1]{\textbf{#1}}
\newcommand{\DeltaNoteStyle}[1]{#1}

% ------------------------------------------------------------
% \DeltaRow{Label}{Math-Inhalt}[Optional: Notiz/Referenz]
% - 2. Spalte ist (durch Tabellenspaltenformat) Mathematik
% - 3. Spalte ist Text/Notiz (z.B. \FormulaRefAuto{...})
% ------------------------------------------------------------
\NewDocumentCommand{\DeltaRow}{m m O{}}{%
  \DeltaLabelStyle{#1:} & #2 & \DeltaNoteStyle{#3} \\[-2pt]%
}

% tex/impl/delta-context.tex  (direkt nach \NewDocumentCommand{\DeltaRow}...)

\NewDocumentCommand{\DeltaDecl}{m m O{}}{%
  \DeltaRow{#1}{#2}[#3]%
}

\NewDocumentCommand{\DeltaPrem}{m m O{}}{%
  \DeltaRow{#1}{\triangleright\,#2}[#3]%
}



% ------------------------------------------------------------
% DeltaContext: mit Überschrift
% ------------------------------------------------------------
% ------------------------------------------------------------
% DeltaContext: mit Überschrift
% ------------------------------------------------------------
\NewDocumentEnvironment{DeltaContext}{m}{%
  \par\medskip
  \noindent\DeltaLabelStyle{#1}\par\smallskip
  \noindent\begin{tabular}{@{}l@{\quad}>{$}l<{$}@{\quad}l@{}}%
}{%
  \end{tabular}%
  \par\medskip
}


% ------------------------------------------------------------
% DeltaBlockFromRows: ohne Überschrift
% ------------------------------------------------------------
\NewDocumentCommand{\DeltaBlockFromRows}{m}{%
  \par\smallskip
  % Nach amsthm-Kopf sicher in neue Zeile (nur wenn wir schon in hmode sind)
  \ifhmode\newline\fi
  \noindent
  \begin{tabular}{@{}l@{\quad}>{$}l<{$}@{\quad}l@{}}%
    #1%
  \end{tabular}%
  \par\smallskip
}

