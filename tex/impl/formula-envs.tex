% tex/impl/formula-envs.tex
\RequirePackage{amsthm}
\RequirePackage{expl3}
\RequirePackage{xparse}

% ------------------------------------------------------------
% Formelregister / Formel-Stack (wie früher aus der alten main)
% ------------------------------------------------------------
\RequirePackage{longtable}

\providecommand{\ShowFormulaSource}[1]{%
  \begingroup\ttfamily\footnotesize\detokenize{#1}\endgroup
}

\ExplSyntaxOn
\newif\ifnostack
\nostackfalse

\seq_new:N \g_mx_formula_stack_seq
\bool_new:N \g_mx_formula_stack_enabled_bool
\bool_gset_true:N \g_mx_formula_stack_enabled_bool

\cs_new_protected:Npn \mx_push_formula:nnn #1#2#3
  {
    \bool_if:NT \g_mx_formula_stack_enabled_bool
      {
        \seq_gput_right:Nx \g_mx_formula_stack_seq
          { {F}{\exp_not:n{#1}}{#2}{\exp_not:n{#3}} }
      }
  }

\NewDocumentCommand{\MxPushFormula}{m m m}{ \mx_push_formula:nnn {#1}{#2}{#3} }




\cs_new_protected:Npn \mx_print_formula_item_aux:nnnn #1#2#3#4
  {
    \str_case:nnF {#1}
      {
        {F}{ $ #4 $ & \ThmLookupEmitRef{#2}{#3} \\ }
      }
      { } % Default: nichts
  }
\cs_new_protected:Npn \mx_print_formula_item:n #1
  { \mx_print_formula_item_aux:nnnn #1 }

\cs_new_protected:Npn \mx_print_formula_item_src_aux:nnnn #1#2#3#4
  {
    \str_case:nnF {#1}
      {
        {F}{ $ #4 $ & \ShowFormulaSource{#4} \\ }
      }
      { }
  }
\cs_new_protected:Npn \mx_print_formula_item_src:n #1
  { \mx_print_formula_item_src_aux:nnnn #1 }

\cs_new_protected:Npn \mx_print_formula_plain_aux:nnnn #1#2#3#4
  {
    \str_case:nnF {#1}
      {
        {F}{ \ShowFormulaSource{#4}\\ }
      }
      { }
  }
\cs_new_protected:Npn \mx_print_formula_plain:n #1
  { \mx_print_formula_plain_aux:nnnn #1 }

\cs_new_protected:Npn \mx_print_formula_item_src_ref_aux:nnnn #1#2#3#4
  {
    \str_case:nnF {#1}
      {
        {F}{ \ShowFormulaSource{#4} & \ThmLookupEmitRef{#2}{#3} \\ }
      }
      { }
  }
\cs_new_protected:Npn \mx_print_formula_item_src_ref:n #1
  { \mx_print_formula_item_src_ref_aux:nnnn #1 }

\NewDocumentCommand{\ClearFormulaStack}{}{
  \seq_gclear:N \g_mx_formula_stack_seq
}

\NewDocumentCommand{\PrintFormulaStack}{}{
  \seq_if_empty:NF \g_mx_formula_stack_seq
    {
      \chapter*{Formelregister}%
      \begin{longtable}{@{}p{0.75\linewidth}@{\qquad}l@{}}
        \seq_map_inline:Nn \g_mx_formula_stack_seq
          { \mx_print_formula_item:n {##1} }
      \end{longtable}
    }
}

\NewDocumentCommand{\PrintFormulaStackSources}{}{
  \seq_if_empty:NF \g_mx_formula_stack_seq
    {
      \chapter*{Formelregister (mit Quelltext)}%
      \begin{longtable}{@{}p{0.45\linewidth}@{\qquad}p{0.48\linewidth}@{}}
        \seq_map_inline:Nn \g_mx_formula_stack_seq
          { \mx_print_formula_item_src:n {##1} }
      \end{longtable}
    }
}

\NewDocumentCommand{\PrintFormulaStackSourcesPlain}{}{
  \seq_if_empty:NF \g_mx_formula_stack_seq
    {
      \chapter*{Formelregister (Quelltext)}%
      \begin{longtable}{@{}p{\linewidth}@{}}
        \seq_map_inline:Nn \g_mx_formula_stack_seq
          { \mx_print_formula_plain:n {##1} }
      \end{longtable}
    }
}

\NewDocumentCommand{\PrintFormulaStackSourcesPlainWithType}{}{
  \seq_if_empty:NF \g_mx_formula_stack_seq
    {
      \chapter*{Formelregister (Quelltext mit Referenz)}%
      \begin{longtable}{@{}p{0.75\linewidth}@{\qquad}l@{}}
        \seq_map_inline:Nn \g_mx_formula_stack_seq
          { \mx_print_formula_item_src_ref:n {##1} }
      \end{longtable}
    }
}
\ExplSyntaxOff


% ------------------------------------------------------------
% Remark/Hint/Example (werden im Skript via \begin{remark} ... genutzt)
% ------------------------------------------------------------
\theoremstyle{remark}
\newtheorem*{remark}{Bemerkung}
\newtheorem*{hint}{Hinweis}
\newtheorem*{example}{Beispiel}
\newtheorem*{notation*}{Notation}

% (optional) danach wieder zurückschalten, falls du später andere Styles willst
\theoremstyle{plain}


% ------------------------------------------------------------
% 1) Drei unabhängige Umgebungen / Zähler
%    (Kopfzeile endet mit Zeilenumbruch)
% ------------------------------------------------------------
\makeatletter
\newtheoremstyle{mxformula}%
  {3pt}% Space above
  {3pt}% Space below
  {\itshape}% Body font
  {0pt}% Indent amount
  {\bfseries}% Head font
  {.}% Head punctuation
  {\newline}% Head space  <-- DAS ist der entscheidende Punkt
  {\thmname{#1}\thmnumber{ #2}\thmnote{ (#3)}}% Head spec
\makeatother

\theoremstyle{mxformula}
\newtheorem{formulaThm}{Theorem}
\newtheorem{formulaDef}{Definition}
\newtheorem{formulaAx}{Axiom}



% ------------------------------------------------------------
% 1b) Alias-Umgebungen für "klassische" \begin{definition} ...
%     (nutzen die gleichen Zähler wie formulaDef/formulaThm/formulaAx)
% ------------------------------------------------------------
\theoremstyle{definition}
\newtheorem{definition}[formulaDef]{Definition}

\theoremstyle{plain}
\newtheorem{theorem}[formulaThm]{Theorem}

\theoremstyle{definition}
\newtheorem{axiom}[formulaAx]{Axiom}

% ------------------------------------------------------------
% 2) Reset-Logik:
%    - bei Kapitelwechsel -> wieder .1
%    - bei Sektionswechsel -> wieder ... .1
% ------------------------------------------------------------
\makeatletter
\@addtoreset{formulaThm}{chapter}
\@addtoreset{formulaDef}{chapter}
\@addtoreset{formulaAx}{chapter}

\@addtoreset{formulaThm}{section}
\@addtoreset{formulaDef}{section}
\@addtoreset{formulaAx}{section}
\makeatother

% ------------------------------------------------------------
% 3) Nummerierungsdarstellung:
%    - wenn section>0:  \thesection.<n>  -> 1.1.1
%    - sonst:           \thechapter.<n>  -> 1.1
%    (robust: falls keine chapters existieren -> fallback)
% ------------------------------------------------------------
\makeatletter
\newcommand{\Formula@chap}{%
  \ifcsname thechapter\endcsname
    \thechapter
  \else
    \arabic{section}%
  \fi
}
\makeatother

% Band-Prefix (Standard 0, wird pro Band überschrieben)
\providecommand{\FormulaBandID}{0}

\renewcommand{\theformulaThm}{%
  \FormulaBandID.\ifnum\value{section}>0 \thesection.\arabic{formulaThm}\else \csname Formula@chap\endcsname.\arabic{formulaThm}\fi
}
\renewcommand{\theformulaDef}{%
  \FormulaBandID.\ifnum\value{section}>0 \thesection.\arabic{formulaDef}\else \csname Formula@chap\endcsname.\arabic{formulaDef}\fi
}
\renewcommand{\theformulaAx}{%
  \FormulaBandID.\ifnum\value{section}>0 \thesection.\arabic{formulaAx}\else \csname Formula@chap\endcsname.\arabic{formulaAx}\fi
}


% ------------------------------------------------------------
% Auto-Makros + Delta-Makros (mit Push via \MxPushFormula)
% ------------------------------------------------------------

% \FormulaThmAuto[<Titel>]{<Druckformel>}[<Keyformel>][<ID>]
\NewDocumentCommand{\FormulaThmAuto}{o m o o}{%
  % Keyformel bestimmen (Default: Druckformel)
  \def\mxKey{#2}%
  \IfNoValueF{#3}{%
    \IfBlankTF{#3}{}{%
      \def\mxKey{#3}%
    }%
  }%

  % Umgebung starten (Titel optional) + Label erst JETZT bilden
  \IfNoValueTF{#1}
    {%
      \begin{formulaThm}%
      \edef\mxLbl{thm:auto:\theformulaThm}%
      \label{\mxLbl}%
      \ThmLookupRegister{\mxLbl}{theorem}{\theformulaThm}{\mxKey}%
    }
    {%
      \begin{formulaThm}[#1]%
      \edef\mxLbl{thm:auto:\theformulaThm}%
      \label{\mxLbl}%
      \ThmLookupRegister{\mxLbl}{theorem}{\theformulaThm}{\mxKey}{#1}%
    }%

  % optional: ID registrieren
  \IfNoValueF{#4}{%
    \IfBlankTF{#4}{}{%
      \ThmLookupRegisterId{#4}{theorem}{\mxLbl}%
    }%
  }%

  % Stack
  \MxPushFormula{theorem}{\mxLbl}{#2}%

  \[
    #2
  \]
  \end{formulaThm}%
}

\NewDocumentCommand{\FormulaDefAuto}{o m}{%
  \IfNoValueTF{#1}
    {\begin{formulaDef}\label{def:auto:\theformulaDef}%
      \ThmLookupRegister{def:auto:\theformulaDef}{definition}{\theformulaDef}{#2}%
    }
    {\begin{formulaDef}[#1]\label{def:auto:\theformulaDef}%
      \ThmLookupRegister{def:auto:\theformulaDef}{definition}{\theformulaDef}{#2}{#1}%
    }%
  \MxPushFormula{definition}{def:auto:\theformulaDef}{#2}%
  \[
    #2
  \]
  \end{formulaDef}%
}

\NewDocumentCommand{\FormulaAxiomAuto}{o m}{%
  \IfNoValueTF{#1}
    {\begin{formulaAx}\label{ax:auto:\theformulaAx}%
      \ThmLookupRegister{ax:auto:\theformulaAx}{axiom}{\theformulaAx}{#2}%
    }
    {\begin{formulaAx}[#1]\label{ax:auto:\theformulaAx}%
      \ThmLookupRegister{ax:auto:\theformulaAx}{axiom}{\theformulaAx}{#2}{#1}%
    }%
  \MxPushFormula{axiom}{ax:auto:\theformulaAx}{#2}%
  \[
    #2
  \]
  \end{formulaAx}%
}

% -------------------------
% Definition: Delta
% -------------------------
\NewDocumentCommand{\FormulaDefDelta}{o m +G{}}{%
  \IfNoValueTF{#1}
    {%
      \begin{formulaDef}%
      \edef\mxLbl{def:auto:\theformulaDef}%
      \ThmLookupRegister{\mxLbl}{definition}{\theformulaDef}{#2}%
    }
    {%
      \begin{formulaDef}[#1]%
      \edef\mxLbl{def:auto:\theformulaDef}%
      \ThmLookupRegister{\mxLbl}{definition}{\theformulaDef}{#2}{#1}%
    }%
  \MxPushFormula{definition}{\mxLbl}{#2}%
  \IfBlankTF{#3}{}{\DeltaBlockFromRows{#3}}%
  \phantomsection\label{\mxLbl}%
  \[
    #2
  \]%
  \end{formulaDef}%
}

% -------------------------
% Theorem: Delta
% -------------------------
\NewDocumentCommand{\FormulaThmDelta}{o m +G{}}{%
  \IfNoValueTF{#1}
    {%
      \begin{formulaThm}%
      \edef\mxLbl{thm:auto:\theformulaThm}%
      \ThmLookupRegister{\mxLbl}{theorem}{\theformulaThm}{#2}%
    }
    {%
      \begin{formulaThm}[#1]%
      \edef\mxLbl{thm:auto:\theformulaThm}%
      \ThmLookupRegister{\mxLbl}{theorem}{\theformulaThm}{#2}{#1}%
    }%
  \MxPushFormula{theorem}{\mxLbl}{#2}%
  \IfBlankTF{#3}{}{\DeltaBlockFromRows{#3}}%
  \phantomsection\label{\mxLbl}%
  \[
    #2
  \]%
  \end{formulaThm}%
}

% -------------------------
% Axiom: Delta
% -------------------------
\NewDocumentCommand{\FormulaAxiomDelta}{o m +m}{%
  \IfNoValueTF{#1}
    {%
      \begin{formulaAx}%
      \edef\mxLbl{ax:auto:\theformulaAx}%
      \ThmLookupRegister{\mxLbl}{axiom}{\theformulaAx}{#2}%
    }
    {%
      \begin{formulaAx}[#1]%
      \edef\mxLbl{ax:auto:\theformulaAx}%
      \ThmLookupRegister{\mxLbl}{axiom}{\theformulaAx}{#2}{#1}%
    }%
  \MxPushFormula{axiom}{\mxLbl}{#2}%
  \DeltaBlockFromRows{#3}%
  \phantomsection\label{\mxLbl}%
  \[
    #2
  \]%
  \end{formulaAx}%
}

% ------------------------------------------------------------
% DeltaK: Definition
% ------------------------------------------------------------
\NewDocumentCommand{\FormulaDefDeltaK}{o m m +m}{%
  \IfNoValueTF{#1}
    {%
      \begin{formulaDef}%
      \edef\mxLbl{def:auto:\theformulaDef}%
      \ThmLookupRegister{\mxLbl}{definition}{\theformulaDef}{#2}%
      \ThmLookupRegister{\mxLbl}{definition}{\theformulaDef}{#3}%
    }{%
      \begin{formulaDef}[#1]%
      \edef\mxLbl{def:auto:\theformulaDef}%
      \ThmLookupRegister{\mxLbl}{definition}{\theformulaDef}{#2}{#1}%
      \ThmLookupRegister{\mxLbl}{definition}{\theformulaDef}{#3}{#1}%
    }%
  \ThmLookupRegisterId{#3}{definition}{\mxLbl}%
  \MxPushFormula{definition}{\mxLbl}{#2}%
  \begin{DeltaContext}{ }%
    #4%
  \end{DeltaContext}%
  \phantomsection\label{\mxLbl}%
  \[
    #2
  \]%
  \end{formulaDef}%
}

% ------------------------------------------------------------
% DeltaK: Theorem
% ------------------------------------------------------------
\NewDocumentCommand{\FormulaThmDeltaK}{o m m +m}{%
  \IfNoValueTF{#1}
    {%
      \begin{formulaThm}%
      \edef\mxLbl{thm:auto:\theformulaThm}%
      \ThmLookupRegister{\mxLbl}{theorem}{\theformulaThm}{#2}%
    }{%
      \begin{formulaThm}[#1]%
      \edef\mxLbl{thm:auto:\theformulaThm}%
      \ThmLookupRegister{\mxLbl}{theorem}{\theformulaThm}{#2}{#1}%
    }%
  \ThmLookupRegisterId{#3}{theorem}{\mxLbl}%
  \MxPushFormula{theorem}{\mxLbl}{#2}%
  \begin{DeltaContext}{ }%
    #4%
  \end{DeltaContext}%
  \phantomsection\label{\mxLbl}%
  \[
    #2
  \]%
  \end{formulaThm}%
}

% ------------------------------------------------------------
% DeltaK: Axiom
% ------------------------------------------------------------
\NewDocumentCommand{\FormulaAxiomDeltaK}{o m m +m}{%
  \IfNoValueTF{#1}
    {%
      \begin{formulaAx}%
      \edef\mxLbl{ax:auto:\theformulaAx}%
      \ThmLookupRegister{\mxLbl}{axiom}{\theformulaAx}{#2}%
    }{%
      \begin{formulaAx}[#1]%
      \edef\mxLbl{ax:auto:\theformulaAx}%
      \ThmLookupRegister{\mxLbl}{axiom}{\theformulaAx}{#2}{#1}%
    }%
  \ThmLookupRegisterId{#3}{axiom}{\mxLbl}%
  \MxPushFormula{axiom}{\mxLbl}{#2}%
  \begin{DeltaContext}{ }%
    #4%
  \end{DeltaContext}%
  \phantomsection\label{\mxLbl}%
  \[
    #2
  \]%
  \end{formulaAx}%
}

\ExplSyntaxOn
\ProvideDocumentCommand{\FormulaRefAuto}{s O{} m o g}{%
  % #1 = Stern
  % #2 = opts (vor der Formel)
  % #3 = query (Formel oder ID)
  % #4 = shorthand nach der Formel in [...] (ax|def|thm)
  % #5 = (optional) altes Dummy-Argument in {...} -> wird ignoriert

  \tl_set:Nn \l_tmpa_tl {#2}

  \IfNoValueF{#4}{
    \str_case:nn {#4}{
      {ax}  { \tl_put_right:Nn \l_tmpa_tl { env=axiom } }
      {def} { \tl_put_right:Nn \l_tmpa_tl { env=definition } }
      {thm} { \tl_put_right:Nn \l_tmpa_tl { env=theorem } }
    }{}
  }

  % ID-Shortcut: reine Tokens ohne Backslash/Leerzeichen
  \regex_match:nnTF { \A [A-Za-z0-9:_\-]+ \Z } {#3}
    {
      \int_compare:nTF { \tl_count:n {#3} > 1 }
        { \ThmRefById{#3} }
        {
          \IfBooleanTF{#1}
            { \ThmRefByStructure*[\tl_use:N \l_tmpa_tl]{#3} }
            { \ThmRefByStructure [\tl_use:N \l_tmpa_tl]{#3} }
        }
    }
    {
      \IfBooleanTF{#1}
        { \ThmRefByStructure*[\tl_use:N \l_tmpa_tl]{#3} }
        { \ThmRefByStructure [\tl_use:N \l_tmpa_tl]{#3} }
    }
}
\ExplSyntaxOff

\ExplSyntaxOn
\ProvideDocumentCommand{\FormulaRefAutoFwd}{s O{} m o g}{%
  \tl_set:Nn \l_tmpa_tl {#2}
  \IfNoValueF{#4}{
    \str_case:nn {#4}{
      {ax}  { \tl_put_right:Nn \l_tmpa_tl { env=axiom } }
      {def} { \tl_put_right:Nn \l_tmpa_tl { env=definition } }
      {thm} { \tl_put_right:Nn \l_tmpa_tl { env=theorem } }
    }{}
  }
  \IfBooleanTF{#1}
    { \ThmRefByStructureFwd*[\tl_use:N \l_tmpa_tl]{#3} }
    { \ThmRefByStructureFwd [\tl_use:N \l_tmpa_tl]{#3} }
}
\ExplSyntaxOff




% tex/impl/delta-context.tex
\RequirePackage{xparse}
\RequirePackage{array}

% Optional: Styling zentral anpassbar
\newcommand{\DeltaLabelStyle}[1]{\textbf{#1}}
\newcommand{\DeltaNoteStyle}[1]{#1}

% ------------------------------------------------------------
% \DeltaRow{Label}{Math-Inhalt}[Optional: Notiz/Referenz]
% - 2. Spalte ist (durch Tabellenspaltenformat) Mathematik
% - 3. Spalte ist Text/Notiz (z.B. \FormulaRefAuto{...})
% ------------------------------------------------------------
\NewDocumentCommand{\DeltaRow}{m m O{}}{%
  \DeltaLabelStyle{#1:} & #2 & \DeltaNoteStyle{#3} \\[-2pt]%
}

% ------------------------------------------------------------
% DeltaContext: mit Überschrift
% ------------------------------------------------------------
% ------------------------------------------------------------
% DeltaContext: mit Überschrift
% ------------------------------------------------------------
\NewDocumentEnvironment{DeltaContext}{m}{%
  \par\medskip
  \noindent\DeltaLabelStyle{#1}\par\smallskip
  \noindent\begin{tabular}{@{}l@{\quad}>{$}l<{$}@{\quad}l@{}}%
}{%
  \end{tabular}%
  \par\medskip
}


% ------------------------------------------------------------
% DeltaBlockFromRows: ohne Überschrift
% ------------------------------------------------------------
\NewDocumentCommand{\DeltaBlockFromRows}{m}{%
  \par\smallskip
  % Nach amsthm-Kopf sicher in neue Zeile (nur wenn wir schon in hmode sind)
  \ifhmode\newline\fi
  \noindent
  \begin{tabular}{@{}l@{\quad}>{$}l<{$}@{\quad}l@{}}%
    #1%
  \end{tabular}%
  \par\smallskip
}




% --- PATCH: erlaubt auch \FormulaThmAuto{Formel}[Titel] ---
\makeatletter

\let\FormulaThmAuto@orig\FormulaThmAuto

% Wenn direkt ein [ kommt: Originalverhalten (Titel zuerst)
\renewcommand{\FormulaThmAuto}{%
  \@ifnextchar[{\FormulaThmAuto@orig}{\FormulaThmAuto@formulafirst}%
}

% Fall: zuerst {Formel}, danach optional [Titel]
\def\FormulaThmAuto@formulafirst#1{%
  \@ifnextchar[{\FormulaThmAuto@withtitle{#1}}{\FormulaThmAuto@orig{#1}}%
}

% WICHTIG: Titel wird als delimited Argument [..] gelesen
\def\FormulaThmAuto@withtitle#1[#2]{%
  \FormulaThmAuto@orig[#2]{#1}%
}

\makeatother
% --- END PATCH ---
