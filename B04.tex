%============================================================
%  Bd. 04 – Die Menge der natürlichen Zahlen
%============================================================

\documentclass[main.tex]{subfiles}

\ifSubfilesClassLoaded{
    % Bei direkter Kompilierung als eigenständiges Dokument werden die
    % externen Dokumente aus den vorherigen Bänden eingebunden, damit
    % Referenzen auf bereits bewiesene Aussagen aufgelöst werden können.
    \usepackage{xr}
    \externaldocument{_B01}
    \externaldocument{_B02}
    \externaldocument{_B03}
}{
   % Wenn dieses Kapitel per \subfile in main.tex eingebunden wird,
   % ist dieser Block leer.
}

\title{Bd. 04 – Die Menge der natürlichen Zahlen}
\author{Martin Kunze}
\date{}
\setcounter{file}{4}

\begin{document}

% Titel und Inhaltsverzeichnis ausgeben
\maketitle
\tableofcontents

% ------------------------------------------------------------
% Peano‑Axiome
% ------------------------------------------------------------
\chapter{Peano‑Axiome}

Die folgenden fünf Axiome charakterisieren die natürlichen Zahlen durch die
Konstante \(0\) und die Nachfolgerabbildung \(\mathrm{succ}\).  In Band 03
wurden diese Aussagen aus den Axiomen der Mengenlehre hergeleitet; hier
werden sie als grundlegende Axiome festgehalten.

% Null ist eine natürliche Zahl
\FormulaAxiomAuto[Null‑Axiom]{0 \in \mathbb{N}}[\FormulaRefAuto{\emptyset\in\mathbb{N}}]

% Nachfolgerabbildung bildet natürliche Zahlen auf natürliche Zahlen ab
\FormulaAxiomAuto[Nachfolger‑Axiom]{\forall x\in\mathbb{N}\,\bigl(\mathrm{succ}(x)\in \mathbb{N}\bigr)}[\FormulaRefAuto{\forall x \in \mathbb{N}\,(\mathrm{succ}(x) \in \mathbb{N})}]

% Null besitzt keinen Vorgänger
\FormulaAxiomAuto[Kein Vorgänger von Null]{\forall x\,\bigl(\mathrm{succ}(x) \neq 0\bigr)}[\FormulaRefAuto{\mathrm{succ}(x) \neq \emptyset}]

% Die Nachfolgerabbildung ist injektiv
\FormulaAxiomAuto[Injektivität]{\forall x\,\forall y\,\bigl(\mathrm{succ}(x)=\mathrm{succ}(y)\rightarrow x=y\bigr)}[\FormulaRefAuto{\mathrm{succ}(x)=\mathrm{succ}(y) \vdash x=y}]

% Induktionsprinzip für beliebige einstelliges Prädikat
\FormulaAxiomAuto[Induktionsprinzip]{\forall P\Bigl(P(0)\,\land\,\forall x\,\bigl(P(x)\rightarrow P\bigl(\mathrm{succ}(x)\bigr)\bigr)\,\rightarrow\,\forall x\in\mathbb{N}\,P(x)\Bigr)}[\FormulaRefAuto{P(\emptyset),\, \forall x\in\mathbb{N}\,(P(x) \rightarrow P(\mathrm{succ}(x))) \vdash \forall x\in\mathbb{N}\,(P(x))}]


% ------------------------------------------------------------
% Eigenschaften und Sätze der natürlichen Zahlen
% ------------------------------------------------------------
\chapter{Eigenschaften und Sätze der natürlichen Zahlen}

In den folgenden Abschnitten sollen künftig zentrale Eigenschaften und
Sätze der natürlichen Zahlen ausgearbeitet werden, wie etwa die
wohlbekannten Regeln für Addition und Multiplikation, Ordnungseigenschaften
oder Anwendungen des Induktionsprinzips.  Dieser Abschnitt dient
gegenwärtig als Platzhalter für zukünftige Beweise und bleibt daher
noch unausgefüllt.

\end{document}