%============================================================
%  Bd. 04 – Die Menge der natürlichen Zahlen
%============================================================

\documentclass[main.tex]{subfiles}

\ifSubfilesClassLoaded{
    % Bei direkter Kompilierung als eigenständiges Dokument werden die
    % externen Dokumente aus den vorherigen Bänden eingebunden, damit
    % Referenzen auf bereits bewiesene Aussagen aufgelöst werden können.
    \usepackage{xr}
    \externaldocument{_B01}
    \externaldocument{_B02}
    \externaldocument{_B03}
}{
   % Wenn dieses Kapitel per \subfile in main.tex eingebunden wird,
   % ist dieser Block leer.
}

\title{Bd. 04 – Die Menge der natürlichen Zahlen}
\author{Martin Kunze}
\date{}
\setcounter{file}{4}

\begin{document}

% Titel und Inhaltsverzeichnis ausgeben
\maketitle
\tableofcontents

% ------------------------------------------------------------
% Peano-Axiome
% ------------------------------------------------------------
\chapter{Peano-Axiome}

Nachdem wir in Band~03 die Menge der natürlichen Zahlen \(\mathbb{N}\) 
aus den Axiomen der Mengenlehre konstruiert haben, wollen wir sie hier 
in klassischer Weise mittels der Peano-Axiome charakterisieren. 
Die Peano-Axiome dienen dabei nicht als Herleitung, sondern als 
\textbf{implizite Definition} der Symbole \(0\) und \(S\) innerhalb der 
bereits gegebenen Menge \(\mathbb{N}\).

\begin{definition}[Peano-Axiome]
Die Symbole
\[
0 \quad \text{(Konstante)}, 
\qquad 
S \quad \text{(einstellige Nachfolgerfunktion)}
\]
werden durch die folgenden Axiome \(\Phi(0,S)\) \textbf{implizit definiert}, 
wobei \(\mathbb{N}\) die in Band~03 konstruierte Menge der natürlichen Zahlen 
bezeichnet. 
Die Axiome \(\Phi(0,S)\) legen die wesentlichen Eigenschaften der Struktur 
\((\mathbb{N},0,S)\) fest.
\end{definition}

\FormulaAxiomAuto[Totalität von \(S\)]{\exists n\in\mathbb{N}(S(m)=n)}[Sei \(m\in\mathbb{N}\). Dann gelte:]
\begin{remark} 
Das Axiom wird durch den Satz \FormulaRefAuto
{\mathrm{Induktiv}(A) \vdash \forall x\in A\,\exists y\in A\;((x,y)\in \mathrm{succ}_A)} motiviert.
\end{remark}

\FormulaAxiomAuto[Funktionale Eindeutigkeit von \(S\)]{m=n\vdash S(m)=S(n)}[Seien \(n,m\in\mathbb{N}\). Dann gelte:]
\begin{remark} 
Das Axiom wird durch den Satz \FormulaRefAuto
{(x,y)\in \mathrm{succ}_A,\,(x,z)\in \mathrm{succ}_A \vdash y=z} motiviert.
\end{remark}


\FormulaAxiomAuto[Null-Axiom]{0 \in \mathbb{N}}
\begin{remark} 
Das Axiom wird durch den Satz \FormulaRefAuto{\emptyset\in\mathbb{N}} motiviert.
\end{remark}

\FormulaAxiomAuto[Nachfolger-Axiom]{S(n) \in \mathbb{N}}[Sei \(n \in \mathbb{N}\). Dann gelte:]
\begin{remark}
Das Axiom wird durch den Satz 
\FormulaRefAuto{\forall x \in \mathbb{N}\,(\mathrm{succ}(x) \in \mathbb{N})} 
motiviert.
\end{remark}

\FormulaAxiomAuto[Kein Vorgänger von 0]{S(n) \neq 0}[Sei \(n \in \mathbb{N}\). Dann gelte:]
\begin{remark}
Das Axiom wird durch den Satz 
\FormulaRefAuto{\mathrm{succ}(x) \neq \emptyset} 
motiviert.
\end{remark}

\FormulaAxiomAuto[Injektivität]{S(n)=S(m) \;\vdash\; n=m}[Seien \(n,m \in \mathbb{N}\). Dann gelte:]
\begin{remark}
Das Axiom wird durch den Satz 
\FormulaRefAuto{\mathrm{succ}(x)=\mathrm{succ}(y) \vdash x=y} 
motiviert.
\end{remark}

\FormulaAxiomAuto[Induktionsprinzip]{P(0),\; \forall n\in\mathbb{N}(P(n)\rightarrow P(S(n)))\;\vdash\; \forall n\in\mathbb{N}(P(n))}[Sei \(P(n)\) eine Aussage über natürliche Zahlen. Dann gelte:]
\begin{remark}
Das Axiom wird durch den Satz 
\FormulaRefAuto{P(\emptyset),\, \forall x\in\mathbb{N}\,(P(x) \rightarrow P(\mathrm{succ}(x))) \vdash \forall x\in\mathbb{N}\,(P(x))} 
motiviert.
\end{remark}
\label{rule:Induktion}

\begin{remark}
Die Axiome \(\Phi(0,S)\) fixieren die Bedeutung der Symbole \(0\) und \(S\) 
implizit. 
Damit ist \(0\) das ausgezeichnete erste Element der Menge \(\mathbb{N}\), 
und \(S\) beschreibt die Nachfolgerabbildung. 
Bis auf Isomorphie ist die durch die Axiome beschriebene Struktur 
\((\mathbb{N},0,S)\) eindeutig bestimmt (Kategorität der Peano-Arithmetik).
\end{remark}


% ------------------------------------------------------------
% Grundlegende Eigenschaften (direkt nach "Peano-Axiome")
% ------------------------------------------------------------
\chapter{Grundlegende Eigenschaften}

\section{Existenz eines Vorgängers}
\FormulaThmAuto{%
n=0 \;\lor\; \exists k\in\mathbb{N}\,(n=S(k))
}[Sei \(n\in\mathbb{N}\), dann gilt:]
\begin{tabproof}
  \prooftext{Sei \(n\in\mathbb{N}\), wir beweisen die Behauptung mit Hilfe des Induktionsprinzips:}

  % Basis
  \proofcase[Basis]{n=0}
    \proofstep{}{0=0}{\rII}
    \proofstep{}{0=0 \;\lor\; \exists k\in\mathbb{N}\,(0=S(k))}{\rOIa{1}}

  % Schritt
  \proofcase[Induktionsschritt]{n\mapsto S(n)}
    \proofstep{3}{n=0\lor \exists k\in\mathbb{N}(n=S(k))}{\rA}
    \proofstep{}{n=n}{\rII}
    \proofstep{}{S(n)=S(n)}{\FormulaRefAuto{m=n\vdash S(m)=S(n)}{4}}
    \proofstep{}{\exists k\in\mathbb{N}(S(n)=S(k))}{\rEI{5}}
    \proofstep{}{n=0\lor \exists k\in\mathbb{N}(S(n)=S(k))}{\rOIa{6}}
    \proofstep{}{n=0\lor \exists k\in\mathbb{N}(n=S(k))}{\rInduktion{2,3,7}}
\end{tabproof}


\FormulaThmAuto{%
n\neq 0 \vdash \exists k\in\mathbb{N}\,(n=S(k))}[Sei \(n\in\mathbb{N}\), dann gilt:]
\begin{tabproof}
  \prooftext{Sei \(n\in\mathbb{N}\) beliebig. Dann gilt:}

  % Fixiere n und n≠0
  \proofstep{1}{n=0\lor \exists k\in\mathbb{N}(n=S(k))}{\FormulaRefAuto{%
n=0 \;\lor\; \exists k\in\mathbb{N}\,(n=S(k))
}}
    \proofstep{1}{n\neq 0\rightarrow \exists k\in\mathbb{N}(n=S(k))}{\FormulaRefAuto{\neg P \rightarrow Q \dashv \vdash P \lor Q}}
\end{tabproof}




\begin{definition}[Rekursions-Schema über \(\mathbb{N}\)]
Seien \(B\) und \(S\) gegebene Terme (oder Ausdrücke) mit
\[
B:\mathbb{N}\to\mathbb{N},\qquad 
S:\mathbb{N}\times\mathbb{N}\times\mathbb{N}\to\mathbb{N},\ (m,n,k)\mapsto S(m,n,k).
\]
Für eine Menge \(R\) schreiben wir \(\mathrm{Rec}_{B,S}(R)\) für die Konjunktion:
\begin{description}
  \item[Sortierung:] \(R \subseteq (\mathbb{N}\times\mathbb{N})\times\mathbb{N}\).
  \item[Basiseigenschaft:] \(\forall n\in\mathbb{N}:\ \bigl((0,n),\,B(n)\bigr)\in R\).
  \item[Schritteigenschaft:] \(\forall m,n,k\in\mathbb{N}:\ \bigl((m,n),\,k\bigr)\in R\ \vdash\ \bigl((S(m),n),\,S(m,n,k)\bigr)\in R\).
\end{description}
\end{definition}

% ============================================================
% Kapitel: Rekursion und Induktion über \mathbb{N}
% ============================================================

\chapter{Rekursion und Induktion über \(\mathbb{N}\)}
\label{chap:Rekursion}

In diesem Kapitel beweisen wir den (Dedekindschen) Rekursionssatz aus den
Peano-Axiomen, speziell unter Verwendung des Induktionsprinzips
\FormulaRefAuto{Induktionsprinzip}. Technisch arbeiten wir mit
\emph{rekursiv erzeugten Relationen} und deren kleinster, rekursiv
geschlossener Hülle. Daraus gewinnen wir den Funktionscharakter
(Existenz und Eindeutigkeit) der rekursiv definierten Abbildungen.
Die parametrisierte Version subsumiert deine Additions- und
Multiplikations-Rekursionen als Spezialfälle.

% ------------------------------------------------------------
% Rekursiv erzeugte Relationen
% ------------------------------------------------------------
\section{Rekursiv erzeugte Relationen}

\begin{definition}[Rekursions-Schema auf \(\mathbb{N}\)]
Sei \(A\) eine Menge, \(a_0\in A\) und \(s:A\to A\) eine Abbildung.
Für eine Relation \(R\subseteq \mathbb{N}\times A\) schreiben wir
\(\mathrm{Rec}_{a_0,s}(R)\) für die Konjunktion der Aussagen:
\begin{description}
  \item[Sortierung:] \(R \subseteq \mathbb{N}\times A\).
  \item[Basis:] \((0,a_0)\in R\).
  \item[Schritt:] \(\forall n\in\mathbb{N}\ \forall a\in A\,\bigl((n,a)\in R \ \vdash\ (S(n),\,s(a))\in R\bigr)\).
\end{description}
\end{definition}

\begin{remark}
Die Idee ist: \(R\) enthält für jede Stufe \(n\) genau den „Wert“ in \(A\),
der durch die Startvorgabe \(a_0\) und die Schrittvorschrift \(s\) induktiv
festgelegt wird. Ohne Minimalität können jedoch auch Übermengen von
Lösungen \(\mathrm{Rec}_{a_0,s}\) erfüllen. Daher schließen wir als Nächstes die
\emph{kleinste} Lösung ab.
\end{remark}

\begin{definition}[Kleinste rekursiv geschlossene Relation]
\[
R^\ast_{a_0,s}\ \coloneqq\ \bigcap\{\,R\subseteq \mathbb{N}\times A \mid \mathrm{Rec}_{a_0,s}(R)\,\}.
\]
\end{definition}

\FormulaThmAuto[\(R^\ast_{a_0,s}\) ist rekursiv geschlossen]{%
\mathrm{Rec}_{a_0,s}\bigl(R^\ast_{a_0,s}\bigr)}
\begin{tabproof}
  \prooftext{Aus der Definition von \(R^\ast_{a_0,s}\) als Schnitt aller \(R\) mit \(\mathrm{Rec}_{a_0,s}(R)\) folgt sofort:}
  \proofstep{}{R^\ast_{a_0,s}\subseteq \mathbb{N}\times A}{\rE}
  \proofstep{}{(0,a_0)\in R^\ast_{a_0,s}}{\rE}
  \proofstep{}{\forall n,a\,\bigl((n,a)\in R^\ast_{a_0,s}\ \vdash\ (S(n),s(a))\in R^\ast_{a_0,s}\bigr)}{\rE}
\end{tabproof}

\begin{remark}[Minimalität]
Per Konstruktion ist \(R^\ast_{a_0,s}\) die kleinste Relation mit \(\mathrm{Rec}_{a_0,s}\).
Für jedes \(R\) mit \(\mathrm{Rec}_{a_0,s}(R)\) gilt \(R^\ast_{a_0,s}\subseteq R\).
\end{remark}

% ------------------------------------------------------------
% Funktionscharakter und Totalität
% ------------------------------------------------------------
\section{Funktionscharakter und Totalität}

Wir zeigen als Nächstes, dass \(R^\ast_{a_0,s}\) \emph{pro Stufe} \(n\) genau
ein \(a\in A\) enthält. Das liefert den Graphen einer eindeutig bestimmten
Funktion \(f:\mathbb{N}\to A\).

\FormulaThmAuto[Funktionscharakter pro Stufe]{%
\forall n\in\mathbb{N}\;\forall a,b\in A\ \Bigl((n,a)\in R^\ast_{a_0,s}\ \wedge\ (n,b)\in R^\ast_{a_0,s}\ \vdash\ a=b\Bigr)}
\begin{tabproof}
  \prooftext{Induktion über \(n\) mit \FormulaRefAuto{Induktionsprinzip}.}
  \proofcase[Basis]{n=0}
    \proofstep{}{(0,a_0)\in R^\ast_{a_0,s}}{\text{Basis von }\mathrm{Rec}_{a_0,s}}
    \proofstep{}{(0,a)\in R^\ast_{a_0,s}\ \wedge\ (0,b)\in R^\ast_{a_0,s}\ \vdash\ a=b}{\text{Eindeutigkeit aus der festen Basis }a_0}
  \proofcase[Schritt]{n\mapsto S(n)}
    \proofstep{IH}{\forall a',b'\,(((n,a')\wedge(n,b'))\Rightarrow a'=b')}{\rA}
    \proofstep{}{(S(n),a)\wedge(S(n),b)}{\rA}
    \proofstep{}{ \exists a',b' \bigl((n,a')\wedge a=s(a')\ \wedge\ (n,b')\wedge b=s(b')\bigr)}{\text{aus Schrittregel}}
    \proofstep{}{a'=b'}{\text{aus IH}}
    \proofstep{}{s(a')=s(b')}{\text{Funktion }s}
    \proofstep{}{a=b}{\rE}
\end{tabproof}

\FormulaThmAuto[Totalität]{%
\forall n\in\mathbb{N}\ \exists!\, a\in A\ \bigl((n,a)\in R^\ast_{a_0,s}\bigr)}
\begin{tabproof}
  \prooftext{Induktion über \(n\) mit \FormulaRefAuto{Induktionsprinzip}.}
  \proofcase[Basis]{n=0}
    \proofstep{}{(0,a_0)\in R^\ast_{a_0,s}}{\text{Basis}}
    \proofstep{}{\exists!\,a\,((0,a)\in R^\ast_{a_0,s})}{\text{aus Basis + Funktionscharakter}}
  \proofcase[Schritt]{n\mapsto S(n)}
    \proofstep{IH}{\exists!\,a\,((n,a)\in R^\ast_{a_0,s})}{\rA}
    \proofstep{}{ \text{Sei }a\text{ der eindeutige Wert zu }n}{\text{aus IH}}
    \proofstep{}{(S(n),s(a))\in R^\ast_{a_0,s}}{\text{Schrittregel}}
    \proofstep{}{\exists!\,b\,((S(n),b)\in R^\ast_{a_0,s})}{\text{Funktionscharakter}}
\end{tabproof}

\begin{remark}[Graph einer Funktion]
Aus der Totalität und dem Funktionscharakter folgt, dass \(R^\ast_{a_0,s}\)
der Graph einer eindeutig bestimmten Abbildung \(f:\mathbb{N}\to A\) ist.
\end{remark}

% ------------------------------------------------------------
% Rekursionssatz (Dedekind)
% ------------------------------------------------------------
\section{Der Rekursionssatz}

\FormulaThmAuto[(Dedekindscher) Rekursionssatz]{%
\exists!\ f:\mathbb{N}\to A\ \Bigl( f(0)=a_0\ \wedge\ \forall n\in\mathbb{N}\ \bigl(f(S(n))=s(f(n))\bigr)\Bigr)}
\begin{tabproof}
  \prooftext{Definiere \(f\) durch den Graphen \(R^\ast_{a_0,s}\).}
  \proofstep{}{ \forall n\ \exists!\, a\ ((n,a)\in R^\ast_{a_0,s})}{\FormulaRefAuto{Totalität}}
  \proofstep{}{ f(n) := \text{das eindeutige } a \text{ mit } (n,a)\in R^\ast_{a_0,s}}{\rD}
  \proofstep{}{ f(0)=a_0 }{\text{aus der Basis}}
  \proofstep{}{ \forall n:\ f(S(n))=s(f(n)) }{\text{aus Schrittregel und Eindeutigkeit}}
  \prooftext{Eindeutigkeit: Sei \(g\) eine weitere Lösung. Induktion über \(n\) zeigt
  \(f(n)=g(n)\) für alle \(n\) (verwende \FormulaRefAuto{Induktionsprinzip}).}
\end{tabproof}

\begin{remark}
Dies ist die klassische Funktionsform des Rekursionssatzes. Die vorangehende
Graph-Formulierung über \(R^\ast_{a_0,s}\) liefert eine robuste Begründung
aus den Peano-Axiomen heraus.
\end{remark}

% ------------------------------------------------------------
% Parametrisierte Rekursion
% ------------------------------------------------------------
\section{Parametrisierte Rekursion}

Wir benötigen häufig Rekursionen mit einem „festen Parameter“ \(p\) in einer
weiteren Menge \(B\).

\begin{definition}[Rekursion mit Parameter]
Sei \(B\) eine Menge, \(B_0:B\to A\) eine Basisfunktion und
\(H:B\times A\to A\) eine Schrittabbildung mit der Eigenschaft, dass
für jedes \(p\in B\) die Abbildung \(H(p,\cdot):A\to A\) eine Funktion ist.
\end{definition}

\FormulaThmAuto[Rekursion mit festem Parameter]{%
\forall p\in B\ \exists!\ f_p:\mathbb{N}\to A\ \Bigl( f_p(0)=B_0(p)\ \wedge\ \forall n\ f_p(S(n))=H\bigl(p,f_p(n)\bigr)\Bigr)}
\begin{tabproof}
  \prooftext{Fixiere \(p\in B\). Wende den Rekursionssatz auf \(a_0:=B_0(p)\) und \(s:=H(p,\cdot)\) an.}
\end{tabproof}

\FormulaThmAuto[Zweiparametrige Rekursion]{%
\exists!\ F:\mathbb{N}\times B\to A\ \Bigl( F(0,p)=B_0(p),\ F(S(n),p)=H\bigl(p,F(n,p)\bigr)\Bigr)}
\begin{tabproof}
  \prooftext{Setze \(F(n,p):=f_p(n)\) aus dem vorigen Theorem. Eindeutigkeit folgt punktweise in \(p\) per Induktion über \(n\).}
\end{tabproof}

\begin{remark}[Anschluss an deine Relationen]
Deine Relationen \(\mathrm{Add}_{\mathrm{Rec}}\) und \(\mathrm{Mul}_{\mathrm{Rec}}\) sind genau die Graphen
der spezialisierten \(F\) mit
\[
\text{Addition:}\quad B_0(n)=n,\ \ H(n,k)=S(k),\qquad
\text{Multiplikation:}\quad B_0(n)=0,\ \ H(n,k)=k+n.
\]
\end{remark}

% ------------------------------------------------------------
% Anwendungen: Addition und Multiplikation
% ------------------------------------------------------------
\section{Anwendungen: Addition und Multiplikation}

\begin{definition}[Addition über Rekursion]
Die Abbildung \(+:\mathbb{N}\times\mathbb{N}\to\mathbb{N}\) ist die eindeutige Funktion mit
\[
m+0 = m,\qquad m+S(n) = S(m+n).
\]
\end{definition}

\begin{definition}[Multiplikation über Rekursion]
Die Abbildung \(\cdot:\mathbb{N}\times\mathbb{N}\to\mathbb{N}\) ist die eindeutige Funktion mit
\[
m\cdot 0 = 0,\qquad m\cdot S(n) = m\cdot n + m.
\]
\end{definition}

\begin{remark}
Diese Definitionen sind Instanzen der zweiparametrigen Rekursion. Deine
Relationen \(\mathrm{Add}_{\mathrm{Rec}}\) und \(\mathrm{Mul}_{\mathrm{Rec}}\) modellieren ihre Graphen
als kleinste rekursiv geschlossene Relationen.
\end{remark}

% ------------------------------------------------------------
% (Optional) Starke Rekursion als Ausblick
% ------------------------------------------------------------
\section*{Ausblick: Starke Rekursion}
\begin{remark}
Eine stärkere Form erlaubt, dass der Schritt von \emph{allen} Vorgängerwerten
abhängt:
\[
f(0)=a_0,\qquad
f(S(n))=H\bigl(n,\ f\!\upharpoonright\!\{0,\dots,n\}\bigr).
\]
Dies erhält man, indem man den Zustandsraum \(A\) durch endliche Initialfunktionen
erweitert. Der Beweis folgt demselben Schema (kleinste rekursiv geschlossene
Relation \(R^\ast\) und Induktion).
\end{remark}



\chapter{Addition als Funktion}
\section{Konstruktion der Menge der Addition}

Im Abschnitt nutzen wir die Metadefinition \(\mathbb{N}^3:=(\mathbb{N}\times\mathbb{N})\times\mathbb{N}\) und \(\mathbb{N}^2:=\mathbb{N}\times\mathbb{N}\).


\begin{definition}[Additions-Rekursion]
Sei \(R\) eine Menge. Wir schreiben\\ \(\mathrm{Add}_{\mathrm{Rec}}(R)\) für die
Konjunktion der folgenden drei Aussagen:
\begin{description}
  \item[Sortierung:] \( R \subseteq (\mathbb{N}\times\mathbb{N})\times\mathbb{N}\)
  \item[Basiseigenschaft:] Sei \(n\in\mathbb{N}\). Dann gelte: \[((0,n),\,n)\in R\]
  \item[Schritteigenschaft:] Seien \(m,n,k\in\mathbb{N}\). 
  Dann gelte: \[((m,n),\,k)\in R\vdash ((S(m),n),\,S(k))\in R\]
\end{description}
\end{definition}

\subsection{Existenz einer solchen Menge}

\FormulaThmAuto{((0,n),n)\in \mathbb{N}^3}[Sein \(n\in\mathbb{N}\), dann gilt:]
\begin{tabproof}
  \proofstep{}{0\in\mathbb{N}}{\FormulaRefAuto{0 \in \mathbb{N}}}
  \proofstep{}{((0,n),n)\in \mathbb{N}^3}{\FormulaRefAuto{a\in A,\, b\in B,\, c\in C\ \vdash\ ((a,b),c)\in (A\times B)\times C}{1}}
\end{tabproof}

\FormulaThmAuto{((m,n),k)\in\mathbb{N}^3 \vdash ((S(m),n),S(k))\in\mathbb{N}^3}[Seien \(m,n,k\in\mathbb{N}\), dann gilt:]
\begin{tabproof}
  \proofstep{}{S(m)\in\mathbb{N}}{\FormulaRefAuto{S(n) \in \mathbb{N}}}
  \proofstep{}{S(k)\in\mathbb{N}}{\FormulaRefAuto{S(n) \in \mathbb{N}}}
  \proofstep{}{((S(m),n),S(k))\in\mathbb{N}^3}{\FormulaRefAuto{a\in A,\, b\in B,\, c\in C\ \vdash\ ((a,b),c)\in (A\times B)\times C}{1,2}}
  \proofstep{}{((m,n),k)\in\mathbb{N}^3\rightarrow ((S(m),n),S(k))\in\mathbb{N}^3}{\FormulaRefAuto{Q \vdash P \rightarrow Q}{3}}
\end{tabproof}

\FormulaThmAuto{\mathrm{Add}_{\mathrm{Rec}}(\mathbb{N}^3)}
\begin{tabproof}
  \proofstep{}{\mathbb{N}^3\subseteq\mathbb{N}^3}{\FormulaRefAuto{A\subseteq A}}
  \proofstep{}{\mathrm{Add}_{\mathrm{Rec}}(\mathbb{N}^3)}{\rAI{\rAI{1,\FormulaRefAuto{((0,n),n)\in \mathbb{N}^3},\FormulaRefAuto{((m,n),k)\in\mathbb{N}^3 \vdash ((S(m),n),S(k))\in\mathbb{N}^3}}}}
\end{tabproof}

\FormulaThmAuto{\exists R\,\mathrm{Add}_{\mathrm{Rec}}(R)}
\begin{tabproof}
  \proofstep{}{\mathrm{Add}_{\mathrm{Rec}}(\mathbb{N}^3)}{\FormulaRefAuto{\mathrm{Add}_{\mathrm{Rec}}(\mathbb{N}^3)}}
  \proofstep{}{\exists R\,\mathrm{Add}_{\mathrm{Rec}}(R)}{\rEI{1}}
\end{tabproof}

\FormulaDefAuto[Addition als Schnitt]{%
\mathrm{Add} \;:=\; \bigcap_{\mathrm{Add}_{\mathrm{Rec}}(R)} R
}
[\FormulaRefAuto{\exists R\,\mathrm{Add}_{\mathrm{Rec}}(R)} ermöglicht uns die Definition der folgenden Menge:]

\FormulaThmAuto{\mathrm{Add}\subseteq\mathbb{N}^3}
\begin{tabproof}
  \proofstep{}{\mathrm{Add}_{\mathrm{Rec}}(\mathbb{N}^3)}{\FormulaRefAuto{\mathrm{Add}_{\mathrm{Rec}}(\mathbb{N}^3)}}
  \proofstep{}{\mathrm{Add}\subseteq\mathbb{N}^3}{\rIE{\FormulaRefAuto{%
\mathrm{Add} \;:=\; \bigcap_{\mathrm{Add}_{\mathrm{Rec}}(R)} R
},\FormulaRefAuto{P(C)\vdash \bigcap_{P(A)} A \subseteq C}{1}}}
\end{tabproof}

\subsection{Grundlegende Eigenschaften}

\FormulaThmAuto{x\in\mathrm{Add}\eqvdash \forall \mathrm{Add}_{\mathrm{Rec}}(R)(x\in R)}
\begin{tabproofwide}
  \proofstepwide[]{x\in \mathrm{Add}}{\leftrightarrow}{x\in \bigcap_{\mathrm{Add}_{\mathrm{Rec}}(R)} R}{\FormulaRefAuto{\mathrm{Add} \;:=\; \bigcap_{\mathrm{Add}_{\mathrm{Rec}}(R)} R}}
  \proofstepwide[]{}{\leftrightarrow}{\forall \mathrm{Add}_{\mathrm{Rec}}(R)(x\in R)}{\FormulaRefAuto{\exists A(P(A)) \vdash x \in \bigcap_{P(B)} B \leftrightarrow \forall C\, (P(C) \rightarrow x \in C)}{\FormulaRefAuto{\exists R\,\mathrm{Add}_{\mathrm{Rec}}(R)}}}
\end{tabproofwide}

% --- Add erfüllt die Basis-Eigenschaft ---
\FormulaThmAuto[Linksneutralität der Addition]{((0,n),n)\in \mathrm{Add}}[Sei \(n\in\mathbb{N}\), dann gilt:]
\begin{tabproof}
  % zeige: \forall Add_Rec(R) \, (((0,n),n)\in R)
  \prooftext{Sei \(n\in\mathbb{N}\), dann gilt:}
  \proofstep{1}{\mathrm{Add}_{\mathrm{Rec}}(R)}{\rA}
  \proofstep{1}{\forall n\in\mathbb{N}\,(((0,n),n)\in R)}%
    {\rAEn{1}}
  \proofstep{1}{((0,n),n)\in R}{\rUE{2}}
  \proofstep{}{\,\forall \mathrm{Add}_{\mathrm{Rec}}(R)\,(((0,n),n)\in R)}{\rUI{\rRI{1,3}}}

  % Charakterisierung von Add rückwärts anwenden
  \proofstep{}{((0,n),n)\in \mathrm{Add}}%
    {\FormulaRefAuto{x\in\mathrm{Add}\ \eqvdash\ \forall \mathrm{Add}_{\mathrm{Rec}}(R)(x\in R)}{4}}
\end{tabproof}


% --- Add erfüllt die Schritt-Eigenschaft (kurz & korrekt) ---
\FormulaThmAuto[Schrittregel der Addition]{((m,n),k)\in \mathrm{Add} \vdash ((S(m),n),S(k))\in \mathrm{Add}}[Seien \(m,n,k\in\mathbb{N}\), dann gilt:]
\begin{tabproof}
  \prooftext{Seien \(m,n,k\in\mathbb{N}\), dann gilt:}

  % Annahme: Ausgangspunkt in Add
  \proofstep{1}{((m,n),k)\in \mathrm{Add}}{\rA}

  % Charakterisierung (→-Richtung) nutzen: aus x∈Add folgt ∀ Add_Rec(S) (x∈S)
  \proofstep{1}{\forall \mathrm{Add}_{\mathrm{Rec}}(S)\,(((m,n),k)\in S)}%
    {\FormulaRefAuto{x\in\mathrm{Add}\ \eqvdash\ \forall \mathrm{Add}_{\mathrm{Rec}}(R)(x\in R)}{\FormulaRefAuto{\exists R\,\mathrm{Add}_{\mathrm{Rec}}(R)}}}

  % Wähle beliebiges R mit Add_Rec(R) und ziehe ((m,n),k)∈R
  \proofstep{3}{\mathrm{Add}_{\mathrm{Rec}}(R)}{\rA}
  \proofstep{1,3}{((m,n),k)\in R}{\rRE{\rUE{2},3}}

  % Schrittregel aus Add_Rec(R) anwenden
  \proofstep{3}{((m,n),k)\in R \rightarrow ((S(m),n),S(k))\in R}{\rUE{\rAEn{3}}}
  \proofstep{1,3}{((S(m),n),S(k))\in R}{\rRE{5,4}}

  % Verallgemeinere über alle R mit Add_Rec(R) und zurück nach Add (←-Richtung)
  \proofstep{1}{\,\forall \mathrm{Add}_{\mathrm{Rec}}(R)\,(((S(m),n),S(k))\in R)}{\rUI{\rRI{3,6}}}
  \proofstep{1}{((S(m),n),S(k))\in \mathrm{Add}}%
    {\FormulaRefAuto{x\in\mathrm{Add}\eqvdash \forall \mathrm{Add}_{\mathrm{Rec}}(R)(x\in R)}{7}}
\end{tabproof}

% --- S(0)+n = S(n) in Add ---
\FormulaThmAuto{((S(0),n),S(n))\in \mathrm{Add}}[Sei \(n\in\mathbb{N}\), dann gilt:]
\begin{tabproof}
  \prooftext{Sei \(n\in\mathbb{N}\), dann gilt:}

  % Basis
  \proofstep{}{((0,n),n)\in \mathrm{Add}}%
    {\FormulaRefAuto{((0,n),n)\in \mathrm{Add}}}

  % Schritt: Instanziere die Schrittregel mit m:=0 und k:=n
  \proofstep{}{((S(0),n),S(n))\in \mathrm{Add}}%
    {\FormulaRefAuto{((m,n),k)\in \mathrm{Add} \vdash ((S(m),n),S(k))\in \mathrm{Add}}{1}}
\end{tabproof}


\subsection{Funktionale Eindeutigkeit}
\subsubsection{Grundlegende Eigenschaften}

\paragraph{Linksneutralität der Addition}

% Aus ((0,n),y)∈Add und y≠n folgt ((0,m),m)∈Add\{((0,n),y)\}
\FormulaThmAuto{y\neq n \;\vdash\; ((0,m),m)\in \mathrm{Add}\setminus\{((0,n),y)\}}[Seien \(n,m,y\in\mathbb{N}\), dann gilt:]
\begin{tabproof}
  \proofstep{1}{y\neq n}{\rA}

  % Basiseigenschaft direkt als Theorem verwenden
  \proofstep{}{((0,m),m)\in \mathrm{Add}}{\FormulaRefAuto{((0,n),n)\in \mathrm{Add}}}

  % Fallunterscheidung
  \proofstep{}{m=n \lor m\neq n}{\FormulaRefAuto{P \lor \neg P}}

  % --- Fall 1: m = n ---
  \proofcase{m=n}
    \proofstep{4}{m=n}{\rA}
    \proofstep{1,4}{y\neq m}{\rIE{4,1}}
    % aus 6: ((0,n),m)≠((0,n),y); aus 5: ((0,m),m)=((0,n),m) ⇒ gesuchte Ungleichheit
    \proofstep{1,4}{((0,m),m)\neq((0,n),y)}%
      {\FormulaRefAuto{a\neq b\vdash (c,a)\neq (d,b)}{5}}
    \proofstep{1,4}{((0,m),m)\in \mathrm{Add}\setminus\{((0,n),y)\}}%
      {\FormulaRefAuto{a \in A,\, a \neq b \vdash a \in A \setminus \{b\}}{2,6}}

  % --- Fall 2: m ≠ n ---
  \proofcase{m\neq n}
    \proofstep{8}{m\neq n}{\rA}
    \proofstep{8}{(0,m)\neq(0,n)}{\FormulaRefAuto{a\neq b\vdash (c,a)\neq (d,b)}{8}}
    \proofstep{8}{((0,m),m)\neq((0,n),y)}{\FormulaRefAuto{a\neq b\vdash (a,c)\neq (b,d)}{9}}
    \proofstep{8}{((0,m),m)\in \mathrm{Add}\setminus\{((0,n),y)\}}%
      {\FormulaRefAuto{a \in A,\, a \neq b \vdash a \in A \setminus \{b\}}{2,10}}

  % Disjunktionseliminierung über (5)
  \proofstep{1}{((0,m),m)\in \mathrm{Add}\setminus\{((0,n),y)\}}{\rOE{3,4,7,8,11}}
\end{tabproof}

\FormulaThmAuto{((m,n),k)\in \mathrm{Add}\setminus\{((0,x),y)\}\vdash ((S(m),n),S(k))\in \mathrm{Add}\setminus\{((0,x),y)\}}[Seien \(m,n,k,x,y\in\mathbb{N}\), dann gilt:]
\begin{tabproof}
  \proofstep{1}{((m,n),k)\in \mathrm{Add}\setminus\{((0,x),y)\}}{\rA}
  \proofstep{1}{((m,n),k)\in \mathrm{Add}}{\FormulaRefAuto{x \in A \setminus B \vdash x \in A}{1}}
  \proofstep{1}{((S(m),n),S(k))\in \mathrm{Add}}{\FormulaRefAuto{((m,n),k)\in \mathrm{Add} \vdash ((S(m),n),S(k))\in \mathrm{Add}}{2}}
  \proofstep{1}{S(m)\neq 0}{\FormulaRefAuto{S(n) \neq 0}}
  \proofstep{1}{(S(m),n)\neq (0,x)}{\FormulaRefAuto{a\neq b\vdash (a,c)\neq (b,d)}{4}}
  \proofstep{1}{((S(m),n),S(k))\neq ((0,x),y)}{\FormulaRefAuto{a\neq b\vdash (a,c)\neq (b,d)}{5}}
  \proofstep{1}{((S(m),n),S(k))\in \mathrm{Add}\setminus\{((0,x),y)\}}{\FormulaRefAuto{a \in A,\, a \neq b \vdash a \in A \setminus \{b\}}{3,6}}
\end{tabproof}

\FormulaThmAuto{\mathrm{Add}\setminus\{((0,x),y)\}\subseteq\mathbb{N}^3}[Seien \(x,y\in\mathbb{N}\), dann gilt:]
\begin{tabproof}
  \proofstep{1}{\mathrm{Add}\subseteq\mathbb{N}^3}{\FormulaRefAuto{\mathrm{Add}\subseteq\mathbb{N}^3}}
  \proofstep{1}{\mathrm{Add}\setminus\{((0,x),y)\}\subseteq\mathbb{N}^3}{\FormulaRefAuto{A\subseteq C\vdash A\setminus B\subseteq C}{1}}
\end{tabproof}

\FormulaThmAuto{x\neq y\vdash \mathrm{Add}_{\mathrm{Rec}}(\mathrm{Add}\setminus\{((0,x),y)\})}[Seien \(x,y\in\mathbb{N}\), dann gilt:]
\begin{tabproof}
  \proofstep{1}{x\neq y}{\rA}
  \proofstep{}{\mathrm{Add}\setminus\{((0,x),y)\}\subseteq\mathbb{N}^3}%
    {\FormulaRefAuto{\mathrm{Add}\setminus\{((0,x),y)\}\subseteq\mathbb{N}^3}}
  \proofstep{1}{\forall n\in\mathbb{N}\,(((0,n),n)\in \mathrm{Add}\setminus\{((0,x),y)\})}%
    {\FormulaRefAuto{y\neq n \;\vdash\; ((0,m),m)\in \mathrm{Add}\setminus\{((0,n),y)\}}{1}}

  % --- die lange Zeile in zwei aufteilen (mit multirow im Begründungsfeld) ---
  \proofstep{1}{%
    \forall m,n,k\in\mathbb{N}\,\bigl(((m,n),k)\in \mathrm{Add}\setminus\{((0,x),y)\}\ \rightarrow\,}%
    {\multirow{2}{*}{\FormulaRefAuto{((m,n),k)\in \mathrm{Add}\setminus\{((0,x),y)\}\vdash ((S(m),n),S(k))\in \mathrm{Add}\setminus\{((0,x),y)\}}}}
  \proofstepstar{}{\rightarrow ((S(m),n),S(k))\in \mathrm{Add}\setminus\{((0,x),y)\}\bigr)}{}
    
    \proofstep{1}{ \mathrm{Add}_{\mathrm{Rec}}(\mathrm{Add}\setminus\{((0,x),y)\})}%
    {\rAI{\rAI{2,3},4}}
\end{tabproof}

\FormulaThmAuto{x\neq y\vdash \mathrm{Add} = \mathrm{Add}\setminus\{((0,x),y)\}}[Seien \(x,y\in\mathbb{N}\), dann gilt:]
\begin{tabproof}
  \proofstep{1}{x\neq y}{\rA}
  \proofstep{1}{\mathrm{Add}\setminus\{((0,x),y)\}\subseteq \mathrm{Add}}{\FormulaRefAuto{A\setminus B\subseteq A}}
  \proofstep{1}{\mathrm{Add}_{\mathrm{Rec}}(\mathrm{Add}\setminus\{((0,x),y)\})}{\FormulaRefAuto{x\neq y\vdash \mathrm{Add}_{\mathrm{Rec}}(\mathrm{Add}\setminus\{((0,x),y)\})}{1}}  
  \proofstep{1}{\mathrm{Add}\subseteq \mathrm{Add}\setminus\{((0,x),y)\}}{\FormulaRefAuto{P(C)\vdash \bigcap_{P(A)} A \subseteq C}{3}}  
  \proofstep{1}{\mathrm{Add}=\mathrm{Add}\setminus\{((0,x),y)\}}{\FormulaRefAuto{ A \subseteq B, B \subseteq A \vdash A = B }{2,4}}  
\end{tabproof}

\FormulaThmAuto{x\neq y\vdash ((0,x),y)\notin \mathrm{Add}}[Seien \(x,y\in\mathbb{N}\), dann gilt:]
\begin{tabproof}
  \proofstep{1}{x\neq y}{\rA}
  \proofstep{1}{\mathrm{Add} = \mathrm{Add}\setminus\{((0,x),y)\}}{\FormulaRefAuto{x\neq y\vdash \mathrm{Add} = \mathrm{Add}\setminus\{((0,x),y)\}}{1}}
  \proofstep{1}{((0,x),y)\notin \mathrm{Add}}{\FormulaRefAuto{A=A\setminus \{a\}\vdash a\notin A}{2}}
\end{tabproof}

\FormulaThmAuto{((0,x),y)\in \mathrm{Add}\vdash x=y}[Seien \(x,y\in\mathbb{N}\), dann gilt:]
\begin{tabproof}
  \proofstep{}{x\neq y\rightarrow ((0,x),y)\notin \mathrm{Add}}{\FormulaRefAuto{x\neq y\vdash ((0,x),y)\notin \mathrm{Add}}}
  \proofstep{}{((0,x),y)\in \mathrm{Add}\rightarrow x=y}{\FormulaRefAuto{P\rightarrow Q\eqvdash \neg Q\rightarrow \neg P}{1}}
\end{tabproof}

\FormulaThmAuto{((0,x),y)\in\mathrm{Add},((0,x),z)\in\mathrm{Add}\vdash y=z}[Seien \(x,y,z\in\mathbb{N}\), dann gilt:]
\begin{tabproof}
\proofstep{1}{((0,x),y)\in\mathrm{Add}}{\rA}
\proofstep{2}{((0,x),z)\in\mathrm{Add}}{\rA}
\proofstep{1}{x=y}{\FormulaRefAuto{((0,x),y)\in \mathrm{Add}\vdash x=y}{1}}
\proofstep{2}{x=z}{\FormulaRefAuto{((0,x),y)\in \mathrm{Add}\vdash x=y}{2}}
\proofstep{1,2}{y=z}{\FormulaRefAuto{a = b,\, a = c \vdash b = c}{3,4}}
\end{tabproof}

%============================================================
%  Unterkapitel: Ergebnis 0 bei Nachfolger in der ersten Komponente
%============================================================

\paragraph{Die Schrittregel der Addition schließt das Ergebnis \(0\) aus}

% --- Basis-Tupel liegen in Add \ {((S(m),n),0)} ---
\FormulaThmAuto{((0,n),n)\in \mathrm{Add}\setminus\{((S(m),k),0)\}}%
[Seien \(m,k,n\in\mathbb{N}\), dann gilt:]
\begin{tabproof}
  % Basis in Add
  \proofstep{}{((0,n),n)\in \mathrm{Add}}{\FormulaRefAuto{((0,n),n)\in \mathrm{Add}}}

  % Fallunterscheidung nach n
  \proofstep{}{n=0 \lor n\neq 0}{\FormulaRefAuto{P \lor \neg P}}

  % --- Fall 1: n = 0 ---
  \proofcase{n=0}
    \proofstep{3}{n=0}{\rA}
    \proofstep{}{S(m)\neq 0}{\FormulaRefAuto{S(n)\neq 0}}
    \proofstep{4}{(0,0)\neq (S(m),k)}{\FormulaRefAuto{a\neq b\vdash (c,a)\neq (d,b)}{4}}
    \proofstep{3,5}{((0,0),0)\neq ((S(m),k),0)}{\FormulaRefAuto{a\neq b\vdash (a,c)\neq (b,d)}{5}}
    \proofstep{1,6}{((0,n),n)\in \mathrm{Add}\setminus\{((S(m),k),0)\}}%
      {\FormulaRefAuto{a \in A,\, a \neq b \vdash a \in A \setminus \{b\}}{1,6}}

  % --- Fall 2: n ≠ 0 ---
  \proofcase{n\neq 0}
    \proofstep{8}{n\neq 0}{\rA}
    \proofstep{8}{((0,n),n)\neq ((S(m),k),0)}%
      {\FormulaRefAuto{a\neq b\vdash (c,a)\neq (d,b)}{8}}
    \proofstep{1,9}{((0,n),n)\in \mathrm{Add}\setminus\{((S(m),k),0)\}}%
      {\FormulaRefAuto{a \in A,\, a \neq b \vdash a \in A \setminus \{b\}}{1,9}}

  % Disjunktionseliminierung
  \proofstep{}{((0,n),n)\in \mathrm{Add}\setminus\{((S(m),k),0)\}}{\rOE{2,3,7,8,10}}
\end{tabproof}

% --- Schritt-Tupel bleiben nach Entfernen stabil ---
\FormulaThmAuto{((m,n),k)\in \mathrm{Add}\setminus\{((S(p),q),0)\}\;\vdash\; ((S(m),n),S(k))\in \mathrm{Add}\setminus\{((S(p),q),0)\}}%
[Seien \(m,n,k,p,q\in\mathbb{N}\), dann gilt:]
\begin{tabproof}
  \proofstep{1}{((m,n),k)\in \mathrm{Add}\setminus\{((S(p),q),0)\}}{\rA}
  \proofstep{1}{((m,n),k)\in \mathrm{Add}}{\FormulaRefAuto{x \in A \setminus B \vdash x \in A}{1}}
  \proofstep{1}{((S(m),n),S(k))\in \mathrm{Add}}%
    {\FormulaRefAuto{((m,n),k)\in \mathrm{Add} \vdash ((S(m),n),S(k))\in \mathrm{Add}}{2}}
  \proofstep{}{S(k)\neq 0}{\FormulaRefAuto{S(n)\neq 0}}
  \proofstep{4}{((S(m),n),S(k))\neq ((S(p),q),0)}%
    {\FormulaRefAuto{a\neq b\vdash (c,a)\neq (d,b)}{4}}
  \proofstep{3,5}{((S(m),n),S(k))\in \mathrm{Add}\setminus\{((S(p),q),0)\}}%
    {\FormulaRefAuto{a \in A,\, a \neq b \vdash a \in A \setminus \{b\}}{3,5}}
\end{tabproof}

% --- Triviale Obermenge ---
\FormulaThmAuto{\mathrm{Add}\setminus\{((S(m),n),0)\}\subseteq\mathbb{N}^3}%
[Seien \(m,n\in\mathbb{N}\), dann gilt:]
\begin{tabproof}
  \proofstep{}{\mathrm{Add}\subseteq\mathbb{N}^3}{\FormulaRefAuto{\mathrm{Add}\subseteq\mathbb{N}^3}}
  \proofstep{}{\mathrm{Add}\setminus\{((S(m),n),0)\}\subseteq\mathbb{N}^3}%
    {\FormulaRefAuto{A\subseteq C\vdash A\setminus B\subseteq C}{1}}
\end{tabproof}

% --- Rekursive Abgeschlossenheit der entfernten Menge ---
\FormulaThmAuto{\mathrm{Add}_{\mathrm{Rec}}\!\bigl(\mathrm{Add}\setminus\{((S(m),n),0)\}\bigr)}%
[Seien \(m,n\in\mathbb{N}\), dann gilt:]
\begin{tabproof}
  \proofstep{}{\mathrm{Add}\setminus\{((S(m),n),0)\}\subseteq\mathbb{N}^3}%
    {\FormulaRefAuto{\mathrm{Add}\setminus\{((S(m),n),0)\}\subseteq\mathbb{N}^3}}
  \proofstep{}{\forall n\in\mathbb{N}\;(((0,n),n)\in \mathrm{Add}\setminus\{((S(m),k),0)\})}%
    {\FormulaRefAuto{((0,n),n)\in \mathrm{Add}\setminus\{((S(m),k),0)\}}}
  \proofstep{}{\forall m,n,k\in\mathbb{N}\;\Bigl(((m,n),k)\in \mathrm{Add}\setminus\{((S(p),q),0)\}\rightarrow}%
    {\multirow{2}{*}{\FormulaRefAuto{((m,n),k)\in \mathrm{Add}\setminus\{((S(p),q),0)\}\;\vdash\; ((S(m),n),S(k))\in \mathrm{Add}\setminus\{((S(p),q),0)\}}}}  
  \proofstepstar{}{\ \ ((S(m),n),S(k))\in \mathrm{Add}\setminus\{((S(p),q),0)\}\Bigr)}{}
  \proofstep{}{ \mathrm{Add}_{\mathrm{Rec}}\!\bigl(\mathrm{Add}\setminus\{((S(m),n),0)\}\bigr)}{\rAI{\rAI{1,2},3}}
\end{tabproof}

% --- Gleichheit und Ausschluss ---
\FormulaThmAuto{\mathrm{Add}=\mathrm{Add}\setminus\{((S(m),n),0)\}}%
[Seien \(m,n\in\mathbb{N}\), dann gilt:]
\begin{tabproof}
  \proofstep{}{\mathrm{Add}\setminus\{((S(m),n),0)\}\subseteq \mathrm{Add}}{\FormulaRefAuto{A\setminus B\subseteq A}}
  \proofstep{}{\mathrm{Add}_{\mathrm{Rec}}\!\bigl(\mathrm{Add}\setminus\{((S(m),n),0)\}\bigr)}%
    {\FormulaRefAuto{\vdash \mathrm{Add}_{\mathrm{Rec}}(\mathrm{Add}\setminus\{((S(m),n),0)\})}}
  \proofstep{}{\mathrm{Add}\subseteq \mathrm{Add}\setminus\{((S(m),n),0)\}}%
    {\FormulaRefAuto{P(C)\vdash \bigcap_{P(A)} A \subseteq C}{2}}
  \proofstep{}{\mathrm{Add}=\mathrm{Add}\setminus\{((S(m),n),0)\}}%
    {\FormulaRefAuto{A \subseteq B,\, B \subseteq A \vdash A = B}{1,3}}
\end{tabproof}

\FormulaThmAuto{((S(m),n),0)\notin \mathrm{Add}}%
[Seien \(m,n\in\mathbb{N}\), dann gilt:]
\begin{tabproof}
  \proofstep{}{\mathrm{Add}=\mathrm{Add}\setminus\{((S(m),n),0)\}}%
    {\FormulaRefAuto{\mathrm{Add}=\mathrm{Add}\setminus\{((S(m),n),0)\}}}
  \proofstep{}{((S(m),n),0)\notin \mathrm{Add}}%
    {\FormulaRefAuto{A=A\setminus\{a\}\vdash a\notin A}{1}}
\end{tabproof}

% Aus ((S(m),n),0)\notin Add folgt: ((S(m),n),x)∈Add ⇒ x≠0

% 1) Hilfsaussage: x=0 ⇒ ((S(m),n),x)∉Add
\FormulaThmAuto{x=0 \;\vdash\; ((S(m),n),x)\notin \mathrm{Add}}%
[Seien \(m,n,x\in\mathbb{N}\), dann gilt:]
\begin{tabproof}
  \proofstep{1}{x=0}{\rA}
  \proofstep{}{((S(m),n),0)\notin \mathrm{Add}}%
    {\FormulaRefAuto{((S(m),n),0)\notin \mathrm{Add}}}
  \proofstep{1}{((S(m),n),x)\notin \mathrm{Add}}%
    {\rIE{1,2}} 
\end{tabproof}

% 2) Zielaussage per Kontraposition
\FormulaThmAuto{((S(m),n),x)\in \mathrm{Add}\;\vdash\; x\neq 0}%
[Seien \(m,n,x\in\mathbb{N}\), dann gilt:]
\begin{tabproof}
  \proofstep{}{x=0 \rightarrow ((S(m),n),x)\notin \mathrm{Add}}%
    {\FormulaRefAuto{x=0 \vdash ((S(m),n),x)\notin \mathrm{Add}}}
  \proofstep{}{((S(m),n),x)\in \mathrm{Add} \rightarrow x\neq 0}%
    {\FormulaRefAuto{P\rightarrow Q\;\eqvdash\; \neg Q\rightarrow \neg P}{1}}
\end{tabproof}

\paragraph{Erweiterte Schrittregel der Addition}

%============================================================
%  Nachfolger-Ergebnis entfernen, unter der Annahme: Vorgänger ∉ Add
%============================================================

% Kurz: E := ((S(m),n),S(k))

% --- Basis-Tupel liegen in Add \ {E} ---
\FormulaThmAuto{((0,t),t)\in \mathrm{Add}\setminus\{((S(m),n),S(k))\}}%
[Seien \(m,n,k,t\in\mathbb{N}\), dann gilt:]
\begin{tabproof}
  % Basis in Add
  \proofstep{}{((0,t),t)\in \mathrm{Add}}{\FormulaRefAuto{((0,n),n)\in \mathrm{Add}}}

  % Ungleichheit zum entfernten Element E via S(m)≠0
  \proofstep{}{S(m)\neq 0}{\FormulaRefAuto{S(n)\neq 0}}
  \proofstep{2}{(0,t)\neq (S(m),n)}{\FormulaRefAuto{a\neq b\vdash (c,a)\neq (d,b)}{2}}
  \proofstep{3}{((0,t),t)\neq ((S(m),n),S(k))}{\FormulaRefAuto{a\neq b\vdash (a,c)\neq (b,d)}{3}}

  % Ab in die Differenzmenge
  \proofstep{1,4}{((0,t),t)\in \mathrm{Add}\setminus\{((S(m),n),S(k))\}}%
    {\FormulaRefAuto{a \in A,\, a \neq b \vdash a \in A \setminus \{b\}}{1,4}}
\end{tabproof}

% --- Schritt-Tupel bleiben nach Entfernen stabil (unter ((m,n),k)∉Add) ---
\FormulaThmAuto{%
\begin{aligned}
 &((m,n),k)\notin \mathrm{Add},\; ((a,b),c)\in \mathrm{Add}\setminus\{((S(m),n),S(k))\}\\
 &\vdash\; ((S(a),b),S(c))\in \mathrm{Add}\setminus\{((S(m),n),S(k))\}
\end{aligned}}%
[Seien \(a,b,c,m,n,k\in\mathbb{N}\), dann gilt:]
\begin{tabproof}
  \proofstep{1}{((m,n),k)\notin \mathrm{Add}}{\rA}
  \proofstep{2}{((a,b),c)\in \mathrm{Add}\setminus\{((S(m),n),S(k))\}}{\rA}

  % Aus der Differenz folgt Add-Mitgliedschaft
  \proofstep{2}{((a,b),c)\in \mathrm{Add}}{\FormulaRefAuto{x \in A \setminus B \vdash x \in A}{2}}

  % Schritteigenschaft in Add
  \proofstep{2}{((S(a),b),S(c))\in \mathrm{Add}}%
    {\FormulaRefAuto{((m,n),k)\in \mathrm{Add} \vdash ((S(m),n),S(k))\in \mathrm{Add}}{3}}

  % Zeige Ungleichheit zum entfernten Element: indirekt
  \proofstep{5}{((S(a),b),S(c))=((S(m),n),S(k))}{\rA}
  \proofstep{5}{(S(a),b)=(S(m),n)\ \land\ S(c)=S(k)}{\FormulaRefAuto{(a,b) = (c,d)\eqvdash a=c\land b=d}{5}}

  % Aus S(c)=S(k) folgt c=k
  \proofstep{5}{c=k}{\FormulaRefAuto{S(n)=S(m) \;\vdash\; n=m}{\rAEb{6}}}

  % Erste Komponente weiter aufspalten
  \proofstep{5}{(S(a),b)=(S(m),n)}{\rAEa{6}}
  \proofstep{5}{S(a)=S(m)\land b=n}{\FormulaRefAuto{(a,b) = (c,d)\eqvdash a=c\land b=d}{8}}
  \proofstep{5}{S(a)=S(m)}{\rAEa{9}}
  \proofstep{5}{b=n}{\rAEb{9}}
  \proofstep{5}{a=m}{\FormulaRefAuto{S(n)=S(m) \;\vdash\; n=m}{10}}

  % Von ((a,b),c)∈Add zu ((m,n),k)∈Add per Substitution
  \proofstep{2,5}{((m,b),c)\in \mathrm{Add}}{\rIE{3,12}}
  \proofstep{2,5}{((m,n),c)\in \mathrm{Add}}{\rIE{13,11}}
  \proofstep{2,5}{((m,n),k)\in \mathrm{Add}}{\rIE{14,7}}

  % Widerspruch zur Annahme ((m,n),k)∉Add
  \proofstep{1,2,5}{\bot}{\rBI{1,15}}

  % Negationseinführung: also Ungleichheit
  \proofstep{1,2}{((S(a),b),S(c))\neq ((S(m),n),S(k))}{\rCI{5,16}}

  % Zurück in die Differenz
  \proofstep{1,2}{((S(a),b),S(c))\in \mathrm{Add}\setminus\{((S(m),n),S(k))\}}%
    {\FormulaRefAuto{a \in A,\, a \neq b \vdash a \in A \setminus \{b\}}{4,17}}
\end{tabproof}

% --- Triviale Obermenge ---
\FormulaThmAuto{\mathrm{Add}\setminus\{((S(m),n),S(k))\}\subseteq\mathbb{N}^3}%
[Seien \(m,n,k\in\mathbb{N}\), dann gilt:]
\begin{tabproof}
  \proofstep{}{\mathrm{Add}\subseteq\mathbb{N}^3}{\FormulaRefAuto{\mathrm{Add}\subseteq\mathbb{N}^3}}
  \proofstep{}{\mathrm{Add}\setminus\{((S(m),n),S(k))\}\subseteq\mathbb{N}^3}%
    {\FormulaRefAuto{A\subseteq C\vdash A\setminus B\subseteq C}{1}}
\end{tabproof}

% --- Rekursive Abgeschlossenheit der entfernten Menge (unter ((m,n),k)∉Add) ---
\FormulaThmAuto{((m,n),k)\notin \mathrm{Add}\;\vdash\; \mathrm{Add}_{\mathrm{Rec}}\!\bigl(\mathrm{Add}\setminus\{((S(m),n),S(k))\}\bigr)}%
[Seien \(m,n,k\in\mathbb{N}\), dann gilt:]
\begin{tabproof}
  \proofstep{1}{((m,n),k)\notin \mathrm{Add}}{\rA}
  \proofstep{}{\mathrm{Add}\setminus\{((S(m),n),S(k))\}\subseteq\mathbb{N}^3}%
    {\FormulaRefAuto{\mathrm{Add}\setminus\{((S(m),n),S(k))\}\subseteq\mathbb{N}^3}}

  % Basis in der entfernten Menge
  \proofstep{}{\forall t\in\mathbb{N}\;(((0,t),t)\in \mathrm{Add}\setminus\{((S(m),n),S(k))\})}%
    {\FormulaRefAuto{((0,t),t)\in \mathrm{Add}\setminus\{((S(m),n),S(k))\}}}

  % Schrittstabilität in der entfernten Menge unter der Annahme (m,n,k)∉Add
  \proofstep{1}{\forall a,b,c\in\mathbb{N}\;\Bigl(((a,b),c)\in \mathrm{Add}\setminus\{((S(m),n),S(k))\}\rightarrow}%
    {\multirow{2}{*}{\FormulaRefAuto{%
 &((m,n),k)\notin \mathrm{Add},\; ((a,b),c)\in \mathrm{Add}\setminus\{((S(m),n),S(k))\}\\
 &\vdash\; ((S(a),b),S(c))\in \mathrm{Add}\setminus\{((S(m),n),S(k))\}}{1}}}
  \proofstepstar{}{\ \ ((S(a),b),S(c))\in \mathrm{Add}\setminus\{((S(m),n),S(k))\}\Bigr)}{}

  \proofstep{1}{ \mathrm{Add}_{\mathrm{Rec}}\!\bigl(\mathrm{Add}\setminus\{((S(m),n),S(k))\}\bigr)}%
    {\rAI{\rAI{2,3},4}}
\end{tabproof}

% --- Gleichheit und Ausschluss ---
\FormulaThmAuto{((m,n),k)\notin \mathrm{Add}\;\vdash\; \mathrm{Add}=\mathrm{Add}\setminus\{((S(m),n),S(k))\}}%
[Seien \(m,n,k\in\mathbb{N}\), dann gilt:]
\begin{tabproof}
  \proofstep{1}{((m,n),k)\notin \mathrm{Add}}{\rA}
  \proofstep{}{\mathrm{Add}\setminus\{((S(m),n),S(k))\}\subseteq \mathrm{Add}}{\FormulaRefAuto{A\setminus B\subseteq A}}
  \proofstep{1}{\mathrm{Add}_{\mathrm{Rec}}\!\bigl(\mathrm{Add}\setminus\{((S(m),n),S(k))\}\bigr)}%
    {\FormulaRefAuto{((m,n),k)\notin \mathrm{Add}\;\vdash\; \mathrm{Add}_{\mathrm{Rec}}(\mathrm{Add}\setminus\{((S(m),n),S(k))\})}{1}}
  \proofstep{1}{\mathrm{Add}\subseteq \mathrm{Add}\setminus\{((S(m),n),S(k))\}}%
    {\FormulaRefAuto{P(C)\vdash \bigcap_{P(A)} A \subseteq C}{3}}
  \proofstep{1}{\mathrm{Add}=\mathrm{Add}\setminus\{((S(m),n),S(k))\}}%
    {\FormulaRefAuto{ A \subseteq B, B \subseteq A \vdash A = B }{2,4}}
\end{tabproof}

\FormulaThmAuto{((m,n),k)\notin \mathrm{Add}\;\vdash\; ((S(m),n),S(k))\notin \mathrm{Add}}%
[Seien \(m,n,k\in\mathbb{N}\), dann gilt:]
\begin{tabproof}
  \proofstep{1}{((m,n),k)\notin \mathrm{Add}}{\rA}
  \proofstep{1}{\mathrm{Add} = \mathrm{Add}\setminus\{((S(m),n),S(k))\}}%
    {\FormulaRefAuto{((m,n),k)\notin \mathrm{Add}\;\vdash\; \mathrm{Add}=\mathrm{Add}\setminus\{((S(m),n),S(k))\}}{1}}
  \proofstep{1}{((S(m),n),S(k))\notin \mathrm{Add}}{\FormulaRefAuto{A=A\setminus \{a\}\vdash a\notin A}{2}}
\end{tabproof}

% -----------------------------------------
% Rückrichtung per Kontraposition
% -----------------------------------------
\FormulaThmAuto{((S(m),n),S(k))\in \mathrm{Add}\;\vdash\; ((m,n),k)\in \mathrm{Add}}%
[Seien \(m,n,k\in\mathbb{N}\), dann gilt:]
\begin{tabproof}
  % Aus dem bereits gezeigten Ausschluss
  \proofstep{}{((m,n),k)\notin \mathrm{Add}\ \rightarrow\ ((S(m),n),S(k))\notin \mathrm{Add}}%
    {\FormulaRefAuto{((m,n),k)\notin \mathrm{Add}\;\vdash\; ((S(m),n),S(k))\notin \mathrm{Add}}}

  % Kontraposition: ¬Q→¬P äquivalent zu P→Q
  \proofstep{}{((S(m),n),S(k))\in \mathrm{Add}\ \rightarrow\ ((m,n),k)\in \mathrm{Add}}%
    {\FormulaRefAuto{P\rightarrow Q\;\eqvdash\; \neg Q\rightarrow \neg P}{1}}
\end{tabproof}

% -----------------------------------------
% Äquivalenz der Schritt-Paare in Add
% -----------------------------------------
\FormulaThmAuto[Erweiterte Schrittregel der Addition]{((m,n),k)\in \mathrm{Add}\ \eqvdash\ ((S(m),n),S(k))\in \mathrm{Add}}%
[Seien \(m,n,k\in\mathbb{N}\), dann gilt:]
\begin{tabproof}
  % (⇒) gegebene Schritt-Eigenschaft
  \proofstep{}{((m,n),k)\in \mathrm{Add}\ \rightarrow\ ((S(m),n),S(k))\in \mathrm{Add}}%
    {\FormulaRefAuto{((m,n),k)\in \mathrm{Add} \vdash ((S(m),n),S(k))\in \mathrm{Add}}}

  % (⇐) soeben bewiesene Rückrichtung
  \proofstep{}{((S(m),n),S(k))\in \mathrm{Add}\ \rightarrow\ ((m,n),k)\in \mathrm{Add}}%
    {\FormulaRefAuto{((S(m),n),S(k))\in \mathrm{Add}\;\vdash\; ((m,n),k)\in \mathrm{Add}}}

  % Äquivalenz-Einführung aus beidseitiger Implikation
  \proofstep{} {((m,n),k)\in \mathrm{Add}\ \leftrightarrow\ ((S(m),n),S(k))\in \mathrm{Add}}%
    {\rLRI{1,2}}
\end{tabproof}


\subsubsection{Das Theorem der funktionalen Eindeutigkeit}

% ===============================
% Eindeutigkeit für Add (Rechts-Eindeutigkeit)
% ===============================
\FormulaThmAuto{((m,n),x)\in\mathrm{Add},\,((m,n),y)\in\mathrm{Add}\vdash x=y}[Seien \(m,n,x,y\in\mathbb{N}\), dann gilt:]
\begin{tabproof}
  \prooftext{Seien \(m,n,x,y\in\mathbb{N}\).}
  \prooftext{Wir beweisen die Behauptung mit Hilfe des Induktionsprinzips:}
  % Umschreiben auf Tripel-Notation
  \proofstep{1}{((m,n),x)\in\mathrm{Add}}{\rA}
  \proofstep{2}{((m,n),y)\in\mathrm{Add}}{\rA}

  % Induktion über m: Aussage Q(m):= ∀n∀y∀z( ((m,n),y)∈Add ∧ ((m,n),z)∈Add → y=z )
  % Basis m=0
  \proofcase[Basis]{m=0}
    \proofstep{3}{((0,n),x)\in\mathrm{Add}}{\rA}
    \proofstep{4}{((0,n),y)\in\mathrm{Add}}{\rA}
    \proofstep{3,4}{x=y}{\FormulaRefAuto{((0,x),y)\in\mathrm{Add},((0,x),z)\in\mathrm{Add}\vdash y=z}{3,4}}

  % Induktionsschritt m→S(m)
  \proofcase[Schritt]{m\to S(m)}
    \proofstepstar{}{((m,n),u)\in\mathrm{Add},}{\multirow{2}{*}{\(\mathrm{IV}\)}}
    \proofstepstar{}{((m,n),v)\in\mathrm{Add}\vdash u=v}{}
    \proofstep{6}{((S(m),n),x)\in\mathrm{Add}}{\rA}
    \proofstep{7}{((S(m),n),y)\in\mathrm{Add}}{\rA}
    
    \proofstep{6}{x\neq 0}{\FormulaRefAuto{((S(m),n),x)\in \mathrm{Add}\vdash x\neq 0}{6}}
    \proofstep{7}{y\neq 0}{\FormulaRefAuto{((S(m),n),x)\in \mathrm{Add}\vdash x\neq 0}{7}}

    \proofstep{6}{\exists u\in\mathbb{N}\,(x=S(u))}{\FormulaRefAuto{n\neq 0 \vdash \exists k\in\mathbb{N}(n=S(k))}{8}}
    \proofstep{7}{\exists v\in\mathbb{N}\,(y=S(v))}{\FormulaRefAuto{n\neq 0 \vdash \exists k\in\mathbb{N}(n=S(k))}{9}}

    \proofstep{12}{x=S(u)}{\rA}
    \proofstep{13}{y=S(v)}{\rA}

    \proofstep{6,12}{((S(m),n),S(u))\in\mathrm{Add}}{\rIE{12,6}}
    \proofstep{7,13}{((S(m),n),S(v))\in\mathrm{Add}}{\rIE{13,7}}

    \proofstep{6,12}{((m,n),u)\in\mathrm{Add}}{\FormulaRefAuto{((m,n),k)\in \mathrm{Add}\ \eqvdash\ ((S(m),n),S(k))\in \mathrm{Add}}{14}}
    \proofstep{7,13}{((m,n),v)\in\mathrm{Add}}{\FormulaRefAuto{((m,n),k)\in \mathrm{Add}\ \eqvdash\ ((S(m),n),S(k))\in \mathrm{Add}}{15}}

    \proofstep{6,7,12,13}{u=v}{\mathrm{IV}(16,17)}
    \proofstep{6,7,12,13}{S(u)=S(v)}{\FormulaRefAuto{m=n\vdash S(m)=S(n)}{18}}
    \proofstep{6,7,12,13}{x=y}{\rIE{13,\rIE{12,19}}}
    \proofstep{6,7}{x=y}{\rEE{10,12,\rEE{11,13,20}}}
    \proofstep{1,2}{x=y}{\rInduktion{5,21}}
\end{tabproof}

\subsection{Totalität}

% -----------------------------------------
% Totalität der Addition (Existenz eines Ergebnisses)
% -----------------------------------------
\FormulaThmAuto{\exists x\in\mathbb{N}\,(((m,n),x)\in \mathrm{Add})}%
[Seien \(m,n\in\mathbb{N}\), dann gilt:]
\begin{tabproof}
  \prooftext{Wir beweisen die Aussage per Induktion über \(m\).}

  % -------- Basis m = 0 --------
  \proofcase[Basis]{m=0}
    \proofstep{}{((0,n),n)\in \mathrm{Add}}{\FormulaRefAuto{((0,n),n)\in \mathrm{Add}}}
    \proofstep{}{\exists x\in\mathbb{N}\,(((0,n),x)\in \mathrm{Add})}{\rEI{1}}

  % -------- Induktionsschritt m -> S(m) --------
  \proofcase[Schritt]{m\to S(m)}
    \proofstep{3}{\exists k\in\mathbb{N}\,(((m,n),k)\in \mathrm{Add})}{\rA}
    % ∃-Elim: fixiere ein solches k
    \proofstep{4}{((m,n),k)\in \mathrm{Add}}{\rA}
    % Schritt-Eigenschaft anwenden
    \proofstep{4}{((S(m),n),S(k))\in \mathrm{Add}}%
      {\FormulaRefAuto{((m,n),k)\in \mathrm{Add} \vdash ((S(m),n),S(k))\in \mathrm{Add}}{4}}
    % Existenz für S(m)
    \proofstep{4}{\exists x\in\mathbb{N}\,(((S(m),n),x)\in \mathrm{Add})}{\rEI{5}}

    \proofstep{3}{\exists x\in\mathbb{N}\,(((S(m),n),x)\in \mathrm{Add})}{\rEE{3,4,6}}

    \proofstep{}{\exists x\in\mathbb{N}\,(((S(m),n),x)\in \mathrm{Add})}{\rInduktion{2,3,7}}

\end{tabproof}

%============================================================
%  Kapitel: Addition
%============================================================

\section{Definition der Addition}

\begin{definition}[Schreibweise für die Addition]
Da die Menge \(\mathrm{Add}\) die Eigenschaften einer Funktion erfüllt, motiviert dies die Einführung einer neuen Funktion \(+:\mathbb{N}\times\mathbb{N}\mapsto \mathbb{N}\) in der Struktur
\((\mathbb{N},0,S)\). Die einzelnen Axiome sind durch bereits bewiesene Sätze
über \(\mathrm{Add}\) motiviert.
\end{definition}
\begin{remark}
Wir setzen \(1 \coloneqq S(0)\). Seien \(a,b,c\in\mathbb{N}\), dann  führen wir die Infix-Schreibweise \(a+b=c\) als Abkürzung für die Zugehörigkeit \(((a,b),c)\in \mathrm{Add}\) als Metadefinitionen ein. Ebenso motivert \FormulaRefAuto{((S(0),n),S(n))\in \mathrm{Add}} für alle \(n\in\mathbb{N}\) Schreibweise \(1+n=S(n)\), die wir im folgenden anstelle von \(S(n)\) als Metadefinition verwenden.
\end{remark}

% ------------------------------------------------
% Axiome der Addition (implizite Definition)
% ------------------------------------------------

\FormulaAxiomAuto[Totalität von \(+\)]%
{a+b\in\mathbb{N}}[Seien \(a,b\in\mathbb{N}\), dann gilt:]
\begin{remark}
Motiviert durch \(\FormulaRefAuto{F(x)\in B}\) und da \(+\) eine Funktion ist.
\end{remark}

\FormulaAxiomAuto[Linksneutralität (Basisgleichung)]%
{0+n=n}[Sei \(n\in\mathbb{N}\), dann gilt:]
\begin{remark}
Motiviert durch \FormulaRefAuto{((0,n),n)\in \mathrm{Add}}.
\end{remark}

\FormulaAxiomAuto[Erweiterte Schrittregel]%
{m+n=k\eqvdash (1+m)+n=1+k}[Seien \(m,n,k\in\mathbb{N}\), dann gilt:]
\begin{remark}
Motiviert durch die Schrittregel \FormulaRefAuto{((m,n),k)\in \mathrm{Add} \eqvdash ((S(m),n),S(k))\in \mathrm{Add}}.
\end{remark}

%============================================================
%  Kapitel: Multiplikation als Funktion
%============================================================

\chapter{Multiplikation als Funktion}
\section{Konstruktion der Menge der Multiplikation}

\begin{definition}[Multiplikations-Rekursion]
Sei \(R\) eine Menge. Wir schreiben \(\mathrm{Mul}_{\mathrm{Rec}}(R)\) für die
Konjunktion der folgenden drei Aussagen:
\begin{description}
  \item[Sortierung:] \( R \subseteq (\mathbb{N}\times\mathbb{N})\times\mathbb{N}\).
  \item[Basiseigenschaft:] Sei \(n\in\mathbb{N}\). Dann gelte: \[((0,n),\,0)\in R.\]
  \item[Schritteigenschaft:] Seien \(m,n,k\in\mathbb{N}\).
   Dann gelte: \[((m,n),\,k)\in R \ \vdash\ ((1+m,n),\,k+n)\in R.\]
\end{description}
\end{definition}

\subsection{Existenz einer solchen Menge}

% --- Basis-Tupel liegen in N^3 ---
\FormulaThmAuto{((0,n),0)\in \mathbb{N}^3}[Sei \(n\in\mathbb{N}\), dann gilt:]
\begin{tabproof}
  \proofstep{}{0\in\mathbb{N}}{\FormulaRefAuto{0 \in \mathbb{N}}}
  \proofstep{}{((0,n),0)\in \mathbb{N}^3}{\FormulaRefAuto{a\in A,\, b\in B,\, c\in C\ \vdash\ ((a,b),c)\in (A\times B)\times C}{1}}
\end{tabproof}

% --- Schritt bleibt in N^3 (nutzt Totalität von +) ---
\FormulaThmAuto{((m,n),k)\in\mathbb{N}^3 \vdash ((1+m,n),k+n)\in\mathbb{N}^3}[Seien \(m,n,k\in\mathbb{N}\), dann gilt:]
\begin{tabproof}
  \proofstep{}{k+n\in\mathbb{N}}{\FormulaRefAuto{a+b\in\mathbb{N}}}
  \proofstep{}{((1+m,n),k+n)\in\mathbb{N}^3}{\FormulaRefAuto{a\in A,\, b\in B,\, c\in C\ \vdash\ ((a,b),c)\in (A\times B)\times C}{1}}
  \proofstep{}{((m,n),k)\in\mathbb{N}^3\rightarrow ((1+m,n),k+n)\in\mathbb{N}^3}{\FormulaRefAuto{Q \vdash P \rightarrow Q}{2}}
\end{tabproof}

% --- N^3 erfüllt Mul_Rec ---
\FormulaThmAuto{\mathrm{Mul}_{\mathrm{Rec}}(\mathbb{N}^3)}
\begin{tabproof}
  \proofstep{}{\mathbb{N}^3\subseteq\mathbb{N}^3}{\FormulaRefAuto{A\subseteq A}}
  \proofstep{}{\mathrm{Mul}_{\mathrm{Rec}}(\mathbb{N}^3)}{\rAI{\rAI{1,\FormulaRefAuto{((0,n),0)\in \mathbb{N}^3},\FormulaRefAuto{((m,n),k)\in\mathbb{N}^3 \vdash ((1+m,n),k+n)\in\mathbb{N}^3}}}}
\end{tabproof}

% --- Existenz eines Mul_Rec-Zeugen ---
\FormulaThmAuto{\exists R\,\mathrm{Mul}_{\mathrm{Rec}}(R)}
\begin{tabproof}
  \proofstep{}{\mathrm{Mul}_{\mathrm{Rec}}(\mathbb{N}^3)}{\FormulaRefAuto{\mathrm{Mul}_{\mathrm{Rec}}(\mathbb{N}^3)}}
  \proofstep{}{\exists R\,\mathrm{Mul}_{\mathrm{Rec}}(R)}{\rEI{1}}
\end{tabproof}

% --- Definition als Schnitt aller Mul_Rec-Relationen ---
\FormulaDefAuto[Multiplikation als Schnitt]{%
\mathrm{Mul} \;:=\; \bigcap_{\mathrm{Mul}_{\mathrm{Rec}}(R)} R
}
[\FormulaRefAuto{\exists R\,\mathrm{Mul}_{\mathrm{Rec}}(R)} ermöglicht uns die Definition der folgenden Menge:]

% --- Mul ist sortiert ---
\FormulaThmAuto{\mathrm{Mul}\subseteq\mathbb{N}^3}
\begin{tabproof}
  \proofstep{}{\mathrm{Mul}_{\mathrm{Rec}}(\mathbb{N}^3)}{\FormulaRefAuto{\mathrm{Mul}_{\mathrm{Rec}}(\mathbb{N}^3)}}
  \proofstep{}{\mathrm{Mul}\subseteq\mathbb{N}^3}{\rIE{\FormulaRefAuto{\mathrm{Mul} \;:=\; \bigcap_{\mathrm{Mul}_{\mathrm{Rec}}(R)} R},\FormulaRefAuto{P(C)\vdash \bigcap_{P(A)} A \subseteq C}{1}}}
\end{tabproof}

\subsection{Grundlegende Eigenschaften}

% --- Charakterisierung als Schnitt ---
\FormulaThmAuto{x\in\mathrm{Mul}\ \eqvdash\ \forall \mathrm{Mul}_{\mathrm{Rec}}(R)(x\in R)}
\begin{tabproofwide}
  \proofstepwide[]{x\in \mathrm{Mul}}{\leftrightarrow}{x\in \bigcap_{\mathrm{Mul}_{\mathrm{Rec}}(R)} R}{\FormulaRefAuto{\mathrm{Mul} \;:=\; \bigcap_{\mathrm{Mul}_{\mathrm{Rec}}(R)} R}}
  \proofstepwide[]{}{\leftrightarrow}{\forall \mathrm{Mul}_{\mathrm{Rec}}(R)(x\in R)}{\FormulaRefAuto{\exists A(P(A)) \vdash x \in \bigcap_{P(B)} B \leftrightarrow \forall C\, (P(C) \rightarrow x \in C)}{\FormulaRefAuto{\exists R\,\mathrm{Mul}_{\mathrm{Rec}}(R)}}}
\end{tabproofwide}

% --- Mul erfüllt Basis ---
\FormulaThmAuto[Null ist linksneutral für \(\mathrm{Mul}\)]{((0,n),0)\in \mathrm{Mul}}[Sei \(n\in\mathbb{N}\), dann gilt:]
\begin{tabproof}
  \prooftext{Sei \(n\in\mathbb{N}\), dann gilt:}
  \proofstep{1}{\mathrm{Mul}_{\mathrm{Rec}}(R)}{\rA}
  \proofstep{1}{((0,n),0)\in R}{\rUE{\rAEn{1}}}
  \proofstep{}{\,\forall \mathrm{Mul}_{\mathrm{Rec}}(R)\,(((0,n),0)\in R)}{\rUI{\rRI{1,2}}}
  \proofstep{}{((0,n),0)\in \mathrm{Mul}}{\FormulaRefAuto{x\in\mathrm{Mul}\ \eqvdash\ \forall \mathrm{Mul}_{\mathrm{Rec}}(R)(x\in R)}{3}}
\end{tabproof}

% --- Mul erfüllt Schritt ---
\FormulaThmAuto[Schrittregel der Multiplikation]{((m,n),k)\in \mathrm{Mul} \vdash ((1+m,n),k+n)\in \mathrm{Mul}}[Seien \(m,n,k\in\mathbb{N}\), dann gilt:]
\begin{tabproof}
  \prooftext{Seien \(m,n,k\in\mathbb{N}\), dann gilt:}
  \proofstep{1}{((m,n),k)\in \mathrm{Mul}}{\rA}
  \proofstep{1}{\forall \mathrm{Mul}_{\mathrm{Rec}}(S)\,(((m,n),k)\in S)}{\FormulaRefAuto{x\in\mathrm{Mul}\ \eqvdash\ \forall \mathrm{Mul}_{\mathrm{Rec}}(R)(x\in R)}{\FormulaRefAuto{\exists R\,\mathrm{Mul}_{\mathrm{Rec}}(R)}}}
  \proofstep{3}{\mathrm{Mul}_{\mathrm{Rec}}(R)}{\rA}
  \proofstep{1,3}{((m,n),k)\in R}{\rRE{\rUE{2},3}}
  \proofstep{3}{((m,n),k)\in R \rightarrow ((1+m,n),k+n)\in R}{\rUE{\rAEn{3}}}
  \proofstep{1,3}{((1+m,n),k+n)\in R}{\rRE{5,4}}
  \proofstep{1}{\forall \mathrm{Mul}_{\mathrm{Rec}}(R)\,(((1+m,n),k+n)\in R)}{\rUI{\rRI{3,6}}}
  \proofstep{1}{((1+m,n),k+n)\in \mathrm{Mul}}{\FormulaRefAuto{x\in\mathrm{Mul}\ \eqvdash\ \forall \mathrm{Mul}_{\mathrm{Rec}}(R)(x\in R)}{7}}
\end{tabproof}

% --- 1*n = n ---
\FormulaThmAuto{((1,n),n)\in \mathrm{Mul}}[Sei \(n\in\mathbb{N}\), dann gilt:]
\begin{tabproof}
  \prooftext{Sei \(n\in\mathbb{N}\), dann gilt:}
  \proofstep{}{((0,n),0)\in \mathrm{Mul}}{\FormulaRefAuto{((0,n),0)\in \mathrm{Mul}}}
  \proofstep{}{((1,n),0+n)\in \mathrm{Mul}}{\FormulaRefAuto{((m,n),k)\in \mathrm{Mul} \vdash ((1+m,n),k+n)\in \mathrm{Mul}}{1}}
  \proofstep{}{((1,n),n)\in \mathrm{Mul}}{\rIE{2,\FormulaRefAuto{0+n=n}}}
\end{tabproof}

\subsection{Funktionale Eindeutigkeit}

%============================================================
% Abschnitt: Links-Nullabsorber der Multiplikation
%============================================================

\paragraph{Links-Nullabsorber der Multiplikation}

% Aus y≠0 folgt: ((0,m),0) bleibt in Mul \ {((0,n),y)}
\FormulaThmAuto{y\neq 0 \;\vdash\; ((0,m),0)\in \mathrm{Mul}\setminus\{((0,n),y)\}}%
[Seien \(m,n,y\in\mathbb{N}\), dann gilt:]
\begin{tabproof}
  \proofstep{1}{y\neq 0}{\rA}
  \proofstep{}{((0,m),0)\in \mathrm{Mul}}{\FormulaRefAuto{((0,n),0)\in \mathrm{Mul}}}
  \proofstep{1}{((0,m),0)\neq((0,n),y)}{\FormulaRefAuto{a\neq b\vdash (a,c)\neq (b,d)}{1}}
  \proofstep{1}{((0,m),0)\in \mathrm{Mul}\setminus\{((0,n),y)\}}%
    {\FormulaRefAuto{a \in A,\, a \neq b \vdash a \in A \setminus \{b\}}{2,3}}
\end{tabproof}

% Schritt bleibt in der entfernten Menge stabil
\FormulaThmAuto{((m,n),k)\in \mathrm{Mul}\setminus\{((0,x),y)\}\vdash ((1+m,n),k+n)\in \mathrm{Mul}\setminus\{((0,x),y)\}}%
[Seien \(m,n,k,x,y\in\mathbb{N}\), dann gilt:]
\begin{tabproof}
  \proofstep{1}{((m,n),k)\in \mathrm{Mul}\setminus\{((0,x),y)\}}{\rA}
  \proofstep{1}{((m,n),k)\in \mathrm{Mul}}{\FormulaRefAuto{x \in A \setminus B \vdash x \in A}{1}}
  \proofstep{1}{((1+m,n),k+n)\in \mathrm{Mul}}%
    {\FormulaRefAuto{((m,n),k)\in \mathrm{Mul} \vdash ((1+m,n),k+n)\in \mathrm{Mul}}{2}}
  \proofstep{}{1+m\neq 0}{\FormulaRefAuto{1+a\neq 0}}
  \proofstep{4}{(1+m,n)\neq (0,x)}{\FormulaRefAuto{a\neq b\vdash (a,c)\neq (b,d)}{4}}
  \proofstep{4}{((1+m,n),k+n)\neq ((0,x),y)}{\FormulaRefAuto{a\neq b\vdash (a,c)\neq (b,d)}{5}}
  \proofstep{3,6}{((1+m,n),k+n)\in \mathrm{Mul}\setminus\{((0,x),y)\}}%
    {\FormulaRefAuto{a \in A,\, a \neq b \vdash a \in A \setminus \{b\}}{3,6}}
\end{tabproof}

% Triviale Obermenge
\FormulaThmAuto{\mathrm{Mul}\setminus\{((0,x),y)\}\subseteq\mathbb{N}^3}[Seien \(x,y\in\mathbb{N}\), dann gilt:]
\begin{tabproof}
  \proofstep{}{\mathrm{Mul}\subseteq\mathbb{N}^3}{\FormulaRefAuto{\mathrm{Mul}\subseteq\mathbb{N}^3}}
  \proofstep{}{\mathrm{Mul}\setminus\{((0,x),y)\}\subseteq\mathbb{N}^3}%
    {\FormulaRefAuto{A\subseteq C\vdash A\setminus B\subseteq C}{1}}
\end{tabproof}

% Rekursive Abgeschlossenheit der entfernten Menge
\FormulaThmAuto{y\neq 0\vdash \mathrm{Mul}_{\mathrm{Rec}}(\mathrm{Mul}\setminus\{((0,x),y)\})}[Seien \(x,y\in\mathbb{N}\), dann gilt:]
\begin{tabproof}
  \proofstep{1}{y\neq 0}{\rA}
  \proofstep{}{\mathrm{Mul}\setminus\{((0,x),y)\}\subseteq\mathbb{N}^3}%
    {\FormulaRefAuto{\mathrm{Mul}\setminus\{((0,x),y)\}\subseteq\mathbb{N}^3}}
  \proofstep{1}{\forall n\in\mathbb{N}\,(((0,n),0)\in \mathrm{Mul}\setminus\{((0,x),y)\})}%
    {\FormulaRefAuto{y\neq 0 \;\vdash\; ((0,m),0)\in \mathrm{Mul}\setminus\{((0,n),y)\}}{1}}
  \proofstep{1}{\forall m,n,k\in\mathbb{N}\,\Bigl(((m,n),k)\in \mathrm{Mul}\setminus\{((0,x),y)\}\rightarrow}%
    {\multirow{2}{*}{\FormulaRefAuto{((m,n),k)\in \mathrm{Mul}\setminus\{((0,x),y)\}\vdash ((1+m,n),k+n)\in \mathrm{Mul}\setminus\{((0,x),y)\}}}}
  \proofstepstar{}{\ ((1+m,n),k+n)\in \mathrm{Mul}\setminus\{((0,x),y)\}\Bigr)}{}
  \proofstep{1}{\mathrm{Mul}_{\mathrm{Rec}}(\mathrm{Mul}\setminus\{((0,x),y)\})}{\rAI{\rAI{2,3},4}}
\end{tabproof}

% Gleichheit per Minimalität ⇒ Ausschluss
\FormulaThmAuto{y\neq 0\vdash \mathrm{Mul}=\mathrm{Mul}\setminus\{((0,x),y)\}}[Seien \(x,y\in\mathbb{N}\), dann gilt:]
\begin{tabproof}
  \proofstep{1}{y\neq 0}{\rA}
  \proofstep{1}{\mathrm{Mul}_{\mathrm{Rec}}(\mathrm{Mul}\setminus\{((0,x),y)\})}%
    {\FormulaRefAuto{y\neq 0\vdash \mathrm{Mul}_{\mathrm{Rec}}(\mathrm{Mul}\setminus\{((0,x),y)\})}{1}}
  \proofstep{1}{\mathrm{Mul}\subseteq \mathrm{Mul}\setminus\{((0,x),y)\}}%
    {\FormulaRefAuto{P(C)\vdash \bigcap_{P(A)} A \subseteq C}{2}}
  \proofstep{1}{\mathrm{Mul}\setminus\{((0,x),y)\}\subseteq \mathrm{Mul}}{\FormulaRefAuto{A\setminus B\subseteq A}}
  \proofstep{1}{\mathrm{Mul}=\mathrm{Mul}\setminus\{((0,x),y)\}}%
    {\FormulaRefAuto{A \subseteq B,\, B \subseteq A \vdash A = B}{3,4}}
\end{tabproof}

% Gesuchte Kurzform als Kontraposition: ((0,n),y)∈Mul ⇒ y=0
\FormulaThmAuto{((0,n),0)\in \mathrm{Mul}\ \vdash\ ((0,n),y)\in\mathrm{Mul}\ \rightarrow\ y=0}%
[Seien \(n,y\in\mathbb{N}\), dann gilt:]
\begin{tabproof}
  \proofstep{}{y\neq 0 \ \rightarrow\ ((0,n),y)\notin \mathrm{Mul}}%
    {\FormulaRefAuto{y\neq 0\vdash \mathrm{Mul}=\mathrm{Mul}\setminus\{((0,n),y)\}}}
  \proofstep{}{((0,n),y)\in\mathrm{Mul}\ \rightarrow\ y=0}%
    {\FormulaRefAuto{P\rightarrow Q\;\eqvdash\; \neg Q\rightarrow \neg P}{1}}
\end{tabproof}

\subsubsection{Eindeutigkeit des Ergebnisses}

% ===============================
% Eindeutigkeit für Mul (Rechts-Eindeutigkeit)
% ===============================
\FormulaThmAuto{((m,n),x)\in\mathrm{Mul},\,((m,n),y)\in\mathrm{Mul}\vdash x=y}[Seien \(m,n,x,y\in\mathbb{N}\), dann gilt:]
\begin{tabproof}
  \prooftext{Seien \(m,n,x,y\in\mathbb{N}\).}
  \prooftext{Wir beweisen die Behauptung per Induktion über \(m\).}

  % ------------------------------------------------
  % INDUKTIONSBASIS m = 0
  % ------------------------------------------------
  \proofcase[Basis]{m=0}
    \proofstep{1}{((0,n),x)\in\mathrm{Mul}}{\rA}
    \proofstep{2}{((0,n),y)\in\mathrm{Mul}}{\rA}
    % 0*n = 0
    \proofstep{1}{x=0}{\FormulaRefAuto{((0,n),0)\in \mathrm{Mul}\;\vdash\; ((0,n),x)\in\mathrm{Mul}\ \rightarrow\ x=0}{1}}
    \proofstep{2}{y=0}{\FormulaRefAuto{((0,n),0)\in \mathrm{Mul}\;\vdash\; ((0,n),y)\in\mathrm{Mul}\ \rightarrow\ y=0}{2}}
    \proofstep{1,2}{x=y}{\rIE{3,4}}

  % ------------------------------------------------
  % INDUKTIONSSCHRITT m -> S(m)
  % Q(m):= ∀n∀r∀s\bigl( ((m,n),r)∈Mul ∧ ((m,n),s)∈Mul → r=s \bigr)
  % ------------------------------------------------
  \proofcase[Schritt]{m\to S(m)}
    % Induktionsvoraussetzung als abgesetzter Stern-Block
    \proofstepstar{}{((m,n),u)\in\mathrm{Mul},}{\multirow{2}{*}{\(\mathrm{IV}\)}}
    \proofstepstar{}{((m,n),v)\in\mathrm{Mul}\;\vdash\; u=v}{}

    % Zielannahmen für S(m)
    \proofstep{5}{((1+m,n),x)\in\mathrm{Mul}}{\rA}
    \proofstep{6}{((1+m,n),y)\in\mathrm{Mul}}{\rA}

    % Schritt-Entfaltung der Multiplikation:
    % S(m)*n = (m*n) + n  (als Existenz eines u bzw. v mit x=u+n bzw. y=v+n)
    \proofstep{5}{\exists u\in\mathbb{N}\,\bigl(((m,n),u)\in\mathrm{Mul}\ \land\ x=u+n\bigr)}%
      {\FormulaRefAuto{((S(m),n),x)\in \mathrm{Mul}\;\vdash\; \exists u\ \bigl(((m,n),u)\in\mathrm{Mul}\ \land\ x=u+n\bigr)}{5}}
    \proofstep{6}{\exists v\in\mathbb{N}\,\bigl(((m,n),v)\in\mathrm{Mul}\ \land\ y=v+n\bigr)}%
      {\FormulaRefAuto{((S(m),n),y)\in \mathrm{Mul}\;\vdash\; \exists v\ \bigl(((m,n),v)\in\mathrm{Mul}\ \land\ y=v+n\bigr)}{6}}

    % ∃-Elimination für u
    \proofstep{7}{((m,n),u)\in\mathrm{Mul}\ \land\ x=u+n}{\rA}
    \proofstep{7}{((m,n),u)\in\mathrm{Mul}}{\rAEa{7}}
    \proofstep{7}{x=u+n}{\rAEb{7}}

    % ∃-Elimination für v
    \proofstep{8}{((m,n),v)\in\mathrm{Mul}\ \land\ y=v+n}{\rA}
    \proofstep{8}{((m,n),v)\in\mathrm{Mul}}{\rAEa{8}}
    \proofstep{8}{y=v+n}{\rAEb{8}}

    % Anwendung der Induktionsvoraussetzung auf ((m,n),u) und ((m,n),v)
    \proofstep{7,8}{u=v}{\mathrm{IV}(9,11)}

    % Aus u=v und x=u+n folgt per Substitution x=v+n
    \proofstep{13}{x=v+n}{\rIE{12,10}}

    % Aus y=v+n und x=v+n folgt x=y
    \proofstep{8,13}{x=y}{\rIE{11,13}}

    % ∃-Eliminationen wieder schließen
    \proofstep{5,6}{x=y}{\rEE{5,7,14}}
    % (Die erste rEE schließt die u-Existenz mit Zeile 14;
    %  analog wird die v-Existenz durch dieselbe Schlusszeile abgedeckt.)

  % Gesamtschluss über die Induktion
  \proofstep{}{x=y}{\rInduktion{3,15}}
\end{tabproof}

\subsection{Totalität}

% --- Existenz eines Ergebnisses für Mul ---
\FormulaThmAuto{\exists x\in\mathbb{N}\,(((m,n),x)\in \mathrm{Mul})}%
[Seien \(m,n\in\mathbb{N}\), dann gilt:]
\begin{tabproof}
  \prooftext{Induktion über \(m\).}
  \proofcase[Basis]{m=0}
    \proofstep{}{((0,n),0)\in \mathrm{Mul}}{\FormulaRefAuto{((0,n),0)\in \mathrm{Mul}}}
    \proofstep{}{\exists x\in\mathbb{N}\,(((0,n),x)\in \mathrm{Mul})}{\rEI{1}}

  \proofcase[Schritt]{m\to 1+mS(m)}
    \proofstep{3}{\exists k\in\mathbb{N}\,(((m,n),k)\in \mathrm{Mul})}{\rA}
    \proofstep{4}{((m,n),k)\in \mathrm{Mul}}{\rA}
    \proofstep{4}{((1+mS(m),n),k+n)\in \mathrm{Mul}}{\FormulaRefAuto{((m,n),k)\in \mathrm{Mul} \vdash ((1+mS(m),n),k+n)\in \mathrm{Mul}}{4}}
    \proofstep{4}{\exists x\in\mathbb{N}\,(((1+mS(m),n),x)\in \mathrm{Mul})}{\rEI{5}}
    \proofstep{3}{\exists x\in\mathbb{N}\,(((1+mS(m),n),x)\in \mathrm{Mul})}{\rEE{3,4,6}}
    \proofstep{}{\exists x\in\mathbb{N}\,(((1+mS(m),n),x)\in \mathrm{Mul})}{\rInduktion{2,3,7}}
\end{tabproof}

%============================================================
%  Kapitel: Multiplikation
%============================================================

\section{Definition der Multiplikation}

\begin{definition}[Schreibweise für die Multiplikation]
Da die Menge \(\mathrm{Mul}\) die Eigenschaften einer Funktion erfüllt, motiviert dies die Einführung
einer neuen Funktion \(\cdot:\mathbb{N}\times\mathbb{N}\mapsto \mathbb{N}\) in der Struktur
\((\mathbb{N},0,S,+)\). Für \(a,b,c\in\mathbb{N}\) führen wir die Infix-Schreibweise
\(a\cdot b=c\) als Abkürzung für \(((a,b),c)\in \mathrm{Mul}\) als Metadefinition ein.
\end{definition}

\begin{remark}
Wir verwenden \(1\coloneqq S(0)\) und die bereits eingeführte Infix-Addition.
\end{remark}

% ------------------------------------------------
% Axiome der Multiplikation (implizite Definition)
% ------------------------------------------------

\FormulaAxiomAuto[Totalität von \(\cdot\)]%
{a\cdot b\in\mathbb{N}}[Seien \(a,b\in\mathbb{N}\), dann gilt:]
\begin{remark}
Motiviert durch \(\FormulaRefAuto{F(x)\in B}\) für Funktionen und die Totalität von \(\mathrm{Mul}\).
\end{remark}

\FormulaAxiomAuto[Linksnull (Basisgleichung)]%
{0\cdot n=0}[Sei \(n\in\mathbb{N}\), dann gilt:]
\begin{remark}
Motiviert durch \FormulaRefAuto{((0,n),0)\in \mathrm{Mul}}.
\end{remark}

\FormulaAxiomAuto[Erweiterte Schrittregel]%
{m\cdot n=k\ \eqvdash\ (1+m)\cdot n=k+n}[Seien \(m,n,k\in\mathbb{N}\), dann gilt:]
\begin{remark}
Motiviert durch \FormulaRefAuto{((m,n),k)\in \mathrm{Mul} \vdash ((1+m,n),k+n)\in \mathrm{Mul}}.
\end{remark}



\section{Gruppenregeln der Addition}

% ------------------------------------------------
% Rechtsneutralität
% ------------------------------------------------
\FormulaThmAuto[Rechtsneutralität der Addition]{n+0=n}[Sei \(n\in\mathbb{N}\), dann gilt:]
\begin{tabproof}
  \prooftext{Induktion über \(n\).}

  % Basis n=0
  \proofcase[Basis]{n=0}
    \proofstep{}{0+0=0}{\FormulaRefAuto{0+n=n}}

  % Schritt n -> 1+n
  \proofcase[Schritt]{n\to 1+n}
    \proofstep{1}{n+0=n}{\rA}
    % Setze m:=n,\; b:=0,\; k:=n in der erweiterten Schrittregel
    \proofstep{1}{(1+n)+0=1+n}{\FormulaRefAuto{m+n=k\eqvdash (1+m)+n=1+k}{1}}
    \proofstep{}{n+0=n}{\rInduktion{1,2,3}}
\end{tabproof}

% ------------------------------------------------
% Ein-Schritt-Gesetz rechts: m+1 = 1+m
% ------------------------------------------------
\FormulaThmAuto[Ein-Schritt-Gesetz rechts]{m+1=1+m}[Sei \(m\in\mathbb{N}\), dann gilt:]
\begin{tabproof}
  \prooftext{Induktion über \(m\). Wir nutzen \(1+n=S(n)\) und Rechtsneutralität.}

  % Basis m=0
  \proofcase[Basis]{m=0}
    \proofstep{}{0+1=1}{\FormulaRefAuto{0+n=n}}
    \proofstep{}{1+0=1}{\FormulaRefAuto{1+n=S(n)}}
    \proofstep{}{0+1=1+0}{\rIE{1,2}}

  % Schritt m -> 1+m
  \proofcase[Schritt]{m\to 1+m}
    \proofstep{4}{m+1=1+m}{\rA}
    \proofstep{4}{(1+m)+1=1+(1+m)}{\FormulaRefAuto{m+n=k\eqvdash (1+m)+n=1+k}{4}}
    \proofstep{}{m+1=1+m}{\rInduktion{3,4,5}}
\end{tabproof}

% ------------------------------------------------
% Assoziativität: (m+n)+k = m+(n+k)
% ------------------------------------------------
\FormulaThmAuto[Assoziativität der Addition]{(m+n)+k = m+(n+k)}%
[Seien \(m,n,k\in\mathbb{N}\), dann gilt:]
\begin{tabproof}
  \prooftext{Induktion über \(m\).}

  % Basis m=0
  \proofcase[Basis]{m=0}
    \proofstep{}{(0+n)+k=n+k}{\FormulaRefAuto{0+n=n}}
    \proofstep{}{0+(n+k)=n+k}{\FormulaRefAuto{0+n=n}}
    \proofstep{}{(0+n)+k=0+(n+k)}{\rIE{1,2}}

  % Schritt m -> 1+m
  \proofcase[Schritt]{m\to 1+m}
    \proofstep{1}{(m+n)+k=m+(n+k)}{\rA}

    % 1) Überführe (1+m)+n via Schrittregel in 1+(m+n)
    \proofstep{}{m+n=(m+n)}{} % triviale Referenzgleichung
    \proofstep{}{(1+m)+n=1+(m+n)}%
      {\FormulaRefAuto{m+n=t\;\rightarrow\;(1+m)+n=1+t}{3}}

    % 2) Setze a:=m+n, b:=k in Schrittregel: (1+a)+b = 1+(a+b)
    \proofstep{}{(1+(m+n))+k = 1+((m+n)+k)}%
      {\FormulaRefAuto{a+b=t\;\rightarrow\;(1+a)+b=1+t}{\rRef}}

    % 3) Nutze IV auf (m+n)+k
    \proofstep{1}{1+((m+n)+k) = 1+(m+(n+k))}{\rIE{1,5}}

    % 4) Sammeln: linke Seite
    \proofstep{}{((1+m)+n)+k = 1+(m+(n+k))}{\rCE{4,6}}

    % 5) Rechte Seite in „1+(...)“-Form bringen
    \proofstep{}{1+m=S(m)}{\FormulaRefAuto{1+n=S(n)}}
    \proofstep{}{(1+m)+(n+k) = 1+(m+(n+k))}%
      {\FormulaRefAuto{a+b=t\;\rightarrow\;(1+a)+b=1+t}{\rRef}}

    % 6) Schluss
    \proofstep{}{((1+m)+n)+k = (1+m)+(n+k)}{}
\end{tabproof}


\end{document}