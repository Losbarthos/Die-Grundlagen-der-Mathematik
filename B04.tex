%============================================================
%  Bd. 04 – Die Menge der natürlichen Zahlen
%============================================================

\documentclass[main.tex]{subfiles}

\ifSubfilesClassLoaded{
    % Bei direkter Kompilierung als eigenständiges Dokument werden die
    % externen Dokumente aus den vorherigen Bänden eingebunden, damit
    % Referenzen auf bereits bewiesene Aussagen aufgelöst werden können.
    \usepackage{xr}
    \externaldocument{_B01}
    \externaldocument{_B02}
    \externaldocument{_B03}
}{
   % Wenn dieses Kapitel per \subfile in main.tex eingebunden wird,
   % ist dieser Block leer.
}

\title{Bd. 04 – Die Menge der natürlichen Zahlen}
\author{Martin Kunze}
\date{}
\setcounter{file}{4}

\begin{document}

% Titel und Inhaltsverzeichnis ausgeben
\maketitle
\tableofcontents

% ------------------------------------------------------------
% Peano-Axiome
% ------------------------------------------------------------
\chapter{Peano-Axiome}

Nachdem wir in Band~03 die Menge der natürlichen Zahlen \(\mathbb{N}\) 
aus den Axiomen der Mengenlehre konstruiert haben, wollen wir sie hier 
in klassischer Weise mittels der Peano-Axiome charakterisieren. 
Die Peano-Axiome dienen dabei nicht als Herleitung, sondern als 
\textbf{implizite Definition} der Symbole \(0\) und \(S\) innerhalb der 
bereits gegebenen Menge \(\mathbb{N}\).

\begin{definition}[Peano-Axiome]
Die Symbole
\[
0 \quad \text{(Konstante)}, 
\qquad 
S \quad \text{(einstellige Nachfolgerfunktion)}
\]
werden durch die folgenden Axiome \(\Phi(0,S)\) \textbf{implizit definiert}, 
wobei \(\mathbb{N}\) die in Band~03 konstruierte Menge der natürlichen Zahlen 
bezeichnet. 
Die Axiome \(\Phi(0,S)\) legen die wesentlichen Eigenschaften der Struktur 
\((\mathbb{N},0,S)\) fest.
\end{definition}

\FormulaAxiomAuto[Null-Axiom]{0 \in \mathbb{N}}
\begin{remark} 
Das Axiom wird durch den Satz \FormulaRefAuto{\emptyset\in\mathbb{N}} motiviert.
\end{remark}

\FormulaAxiomAuto[Nachfolger-Axiom]{S(n) \in \mathbb{N}}[Sei \(n \in \mathbb{N}\). Dann gelte:]
\begin{remark}
Das Axiom wird durch den Satz 
\FormulaRefAuto{\forall x \in \mathbb{N}\,(\mathrm{succ}(x) \in \mathbb{N})} 
motiviert.
\end{remark}

\FormulaAxiomAuto[Kein Vorgänger von 0]{S(n) \neq 0}[Sei \(n \in \mathbb{N}\). Dann gelte:]
\begin{remark}
Das Axiom wird durch den Satz 
\FormulaRefAuto{\mathrm{succ}(x) \neq \emptyset} 
motiviert.
\end{remark}

\FormulaAxiomAuto[Injektivität]{S(n)=S(m) \;\vdash\; n=m}[Seien \(n,m \in \mathbb{N}\). Dann gelte:]
\begin{remark}
Das Axiom wird durch den Satz 
\FormulaRefAuto{\mathrm{succ}(x)=\mathrm{succ}(y) \vdash x=y} 
motiviert.
\end{remark}

\FormulaAxiomAuto[Induktionsprinzip]{P(0),\; \forall n\in\mathbb{N}(P(n)\rightarrow P(S(n)))\;\vdash\; \forall n\in\mathbb{N}(P(n))}[Sei \(P(n)\) eine Aussage über natürliche Zahlen. Dann gelte:]
\begin{remark}
Das Axiom wird durch den Satz 
\FormulaRefAuto{P(\emptyset),\, \forall x\in\mathbb{N}\,(P(x) \rightarrow P(\mathrm{succ}(x))) \vdash \forall x\in\mathbb{N}\,(P(x))} 
motiviert.
\end{remark}

\begin{remark}
Die Axiome \(\Phi(0,S)\) fixieren die Bedeutung der Symbole \(0\) und \(S\) 
implizit. 
Damit ist \(0\) das ausgezeichnete erste Element der Menge \(\mathbb{N}\), 
und \(S\) beschreibt die Nachfolgerabbildung. 
Bis auf Isomorphie ist die durch die Axiome beschriebene Struktur 
\((\mathbb{N},0,S)\) eindeutig bestimmt (Kategorität der Peano-Arithmetik).
\end{remark}



% ------------------------------------------------------------
% Addition und Multiplikation
% ------------------------------------------------------------
\chapter{Addition und Multiplikation}

\section{Addition}

\begin{definition}[Addition natürlicher Zahlen]
Sei \(+\) ein \emph{neues} zweistelliges Funktionssymbol. 
Die Addition zweier natürlicher Zahlen \(n,m \in \mathbb{N}\) 
wird durch die folgenden Regeln \textbf{rekursiv definiert}:
\end{definition}

\FormulaDefAuto[Addition – Basisfall]{n + 0 \coloneqq n}[Sei \(n \in \mathbb{N}\).]

\FormulaDefAuto[Addition – Rekursionsvorschrift]{n + S(m) \coloneqq S(n+m)}[Seien \(n,m \in \mathbb{N}\).]


\section{Multiplikation}

\begin{definition}[Multiplikation natürlicher Zahlen]
Sei \(\cdot\) ein \emph{neues} zweistelliges Funktionssymbol. 
Die Multiplikation zweier natürlicher Zahlen \(n,m \in \mathbb{N}\) 
wird durch die folgenden Regeln \textbf{rekursiv definiert}:
\end{definition}

\FormulaDefAuto[Multiplikation – Basisfall]{n \cdot 0 \coloneqq 0}[Sei \(n \in \mathbb{N}\).]

\FormulaDefAuto[Multiplikation – Rekursionsvorschrift]{n \cdot S(m) \coloneqq (n \cdot m) + n}[Seien \(n,m \in \mathbb{N}\).]

% ------------------------------------------------------------
% Wohldefiniertheit der Addition und Multiplikation
% ------------------------------------------------------------
\chapter{Wohldefiniertheit der Addition und Multiplikation}

In diesem Kapitel zeigen wir, dass die rekursiv eingeführten Symbole \(+\) und \(\cdot\)
tatsächlich wohldefinierte Funktionen auf \(\mathbb{N}\) sind. Dazu arbeiten wir
mit den jeweiligen \emph{Funktionsgraphen} als Mengen geordneter Paare.

\section{Funktionsgraphen}

\FormulaDefAuto[Graph der Addition]{\mathrm{Add} := \Bigl\{\,((n,m),k)\;\Bigm|\; n,m,k \in \mathbb{N}\ \land\ n+m = k\,\Bigr\}}

\FormulaDefAuto[Graph der Multiplikation]{\mathrm{Mult} := \Bigl\{\,((n,m),k)\;\Bigm|\; n,m,k \in \mathbb{N}\ \land\ n\cdot m = k\,\Bigr\}}

\begin{remark}
Für die Funktionsbegriffe vergleichen wir mit \FormulaRefAuto{F \subseteq A \times B},
\FormulaRefAuto{\forall x \in A\,\forall y,z \in B\;\bigl((x,y)\in F \land (x,z)\in F \rightarrow y=z\bigr)} und
\FormulaRefAuto{\forall x \in A\,\exists y \in B\;(x,y)\in F}.
Als Produktzeichen verwenden wir \(A\times B\) im Sinn der Charakterisierung
\[
(a,b)\in A\times B \ \leftrightarrow\ a\in A \land b\in B,\qquad
((u,v),w)\in (A\times B)\times C \ \leftrightarrow\ u\in A \land v\in B \land w\in C.
\]
\end{remark}

\section{Addition ist eine Funktion \( \mathbb{N}\times\mathbb{N} \to \mathbb{N}\)}

\subsection*{(1) Graph liegt im Produkt}

\FormulaThmAuto[Addition: Graph im Produkt]{\mathrm{Add} \subseteq (\mathbb{N}\times\mathbb{N}) \times \mathbb{N}}
\begin{tabproofwide}
  \proofstepwide{z \in \mathrm{Add}}{\rightarrow}{\exists n,m,k\in\mathbb{N}\,\bigl(z=((n,m),k)\land n+m=k\bigr)}%
    {\FormulaRefAuto{\mathrm{Add} := \Bigl\{\,((n,m),k)\;\Bigm|\; n,m,k \in \mathbb{N}\ \land\ n+m = k\,\Bigr\}}}
  \proofstepwide{}{ \rightarrow}{z \in (\mathbb{N}\times\mathbb{N}) \times \mathbb{N}}%
    {\rUE{\text{Charakterisierung von }(A\times B)\times C}}
  \proofstepwidestar{\mathrm{Add} \subseteq (\mathbb{N}\times\mathbb{N}) \times \mathbb{N}}%
    {\FormulaRefAuto{ A \subseteq B := \forall x\,(x\in A \rightarrow x\in B)}{\rUI{2}}}
\end{tabproofwide}

\subsection*{(2) Funktionale Eindeutigkeit}

\FormulaThmAuto[Addition: Eindeutigkeit]{\forall n,m,k,l\in\mathbb{N}\ \Bigl( ((n,m),k)\in\mathrm{Add} \land ((n,m),l)\in\mathrm{Add} \ \rightarrow\ k=l \Bigr)}
\begin{tabproofwide}
  \proofstepwidestar[1]{((n,m),k)\in\mathrm{Add}}{\rA}
  \proofstepwidestar[2]{((n,m),l)\in\mathrm{Add}}{\rA}
  \proofstepwide[1]{((n,m),k)\in\mathrm{Add}}{\rightarrow}{n+m=k}%
    {\FormulaRefAuto{\mathrm{Add} := \Bigl\{\,((n,m),k)\;\Bigm|\; n,m,k \in \mathbb{N}\ \land\ n+m = k\,\Bigr\}}}
  \proofstepwide[2]{((n,m),l)\in\mathrm{Add}}{\rightarrow}{n+m=l}%
    {\FormulaRefAuto{\mathrm{Add} := \Bigl\{\,((n,m),k)\;\Bigm|\; n,m,k \in \mathbb{N}\ \land\ n+m = k\,\Bigr\}}}
  \proofstepwide[1,2]{k}{=}{l}%
    {\rChain{3,4}}
  \proofstepwidestar{\forall n,m,k,l\in\mathbb{N}\ \Bigl( ((n,m),k)\in\mathrm{Add} \land ((n,m),l)\in\mathrm{Add} \ \rightarrow\ k=l \Bigr)}%
    {\rUI{5}}
\end{tabproofwide}

\subsection*{(3) Totalität auf der Domäne}

\FormulaThmAuto[Addition: Totalität]{\forall n,m\in\mathbb{N}\ \exists k\in\mathbb{N}\ \bigl(((n,m),k)\in\mathrm{Add}\bigr)}
\begin{tabproofsplitwide}
  \proofpartwide{Basis \(m=0\)}
    \proofstepwidestar[1]{n\in\mathbb{N}}{\rA}
    \proofstepwidestar[]{0\in\mathbb{N}}{\FormulaRefAuto{0 \in \mathbb{N}}}
    \proofstepwide{}{=}{n+0}{\FormulaRefAuto{Addition – Basisfall}}
    \proofstepwide{}{=}{n}{\rChain{3}}
    \proofstepwide[1]{((n,0),n)}{\in}{\mathrm{Add}}%
      {\FormulaRefAuto{\mathrm{Add} := \Bigl\{\,((n,m),k)\;\Bigm|\; n,m,k \in \mathbb{N}\ \land\ n+m = k\,\Bigr\}}}
    \proofstepwidestar{\exists k\in\mathbb{N}\,(((n,0),k)\in\mathrm{Add})}%
      {\rEI{5}}
  \closeproofpartwide

  \proofpartwide{Induktionsschritt \(m\mapsto S(m)\)}
    \proofstepwidestar[1]{n\in\mathbb{N}}{\rA}
    \proofstepwidestar[2]{m\in\mathbb{N}}{\rA}
    \proofstepwidestar[3]{\exists k\in\mathbb{N}\,(((n,m),k)\in\mathrm{Add})}{\rA}
    \proofstepwide[3]{((n,m),k)}{\in}{\mathrm{Add}}{\rA}
    \proofstepwide[3]{n+m}{=}{k}{\FormulaRefAuto{\mathrm{Add} := \Bigl\{\,((n,m),k)\;\Bigm|\; n,m,k \in \mathbb{N}\ \land\ n+m = k\,\Bigr\}}}
    \proofstepwide{}{=}{n+S(m)}{\FormulaRefAuto{Addition – Rekursionsvorschrift}}
    \proofstepwide{}{=}{S(n+m)}{\rChain{5,6}}
    \proofstepwide{}{=}{S(k)}{\rChain{4,7}}
    \proofstepwidestar[]{S(k)\in\mathbb{N}}{\FormulaRefAuto{S(n) \in \mathbb{N}}}
    \proofstepwide{((n,S(m)),S(k))}{\in}{\mathrm{Add}}%
      {\FormulaRefAuto{\mathrm{Add} := \Bigl\{\,((n,m),k)\;\Bigm|\; n,m,k \in \mathbb{N}\ \land\ n+m = k\,\Bigr\}}}
    \proofstepwidestar{\exists k'\in\mathbb{N}\,(((n,S(m)),k')\in\mathrm{Add})}{\rEI{10}}
  \closeproofpartwide

  \proofpartwide{Schluss}
    \proofstepwidestar{}{ \forall m\in\mathbb{N}\ \exists k\in\mathbb{N}\,(((n,m),k)\in\mathrm{Add}) }%
      {\FormulaRefAuto{P(0),\, \forall x\in\mathbb{N}\,(P(x) \rightarrow P(\mathrm{succ}(x))) \vdash \forall x\in\mathbb{N}\,(P(x))}}
    \proofstepwidestar{}{ \forall n,m\in\mathbb{N}\ \exists k\in\mathbb{N}\,(((n,m),k)\in\mathrm{Add}) }{\rUI{1}}
  \closeproofpartwide
\end{tabproofsplitwide}

\section{Multiplikation ist eine Funktion \( \mathbb{N}\times\mathbb{N} \to \mathbb{N}\)}

\subsection*{(1) Graph liegt im Produkt}

\FormulaThmAuto[Multiplikation: Graph im Produkt]{\mathrm{Mult} \subseteq (\mathbb{N}\times\mathbb{N}) \times \mathbb{N}}
\begin{tabproofwide}
  \proofstepwide{z \in \mathrm{Mult}}{\rightarrow}{\exists n,m,k\in\mathbb{N}\,\bigl(z=((n,m),k)\land n\cdot m=k\bigr)}%
    {\FormulaRefAuto{\mathrm{Mult} := \Bigl\{\,((n,m),k)\;\Bigm|\; n,m,k \in \mathbb{N}\ \land\ n\cdot m = k\,\Bigr\}}}
  \proofstepwide{}{ \rightarrow}{z \in (\mathbb{N}\times\mathbb{N}) \times \mathbb{N}}%
    {\rUE{\text{Charakterisierung von }(A\times B)\times C}}
  \proofstepwidestar{\mathrm{Mult} \subseteq (\mathbb{N}\times\mathbb{N}) \times \mathbb{N}}%
    {\FormulaRefAuto{ A \subseteq B := \forall x\,(x\in A \rightarrow x\in B)}{\rUI{2}}}
\end{tabproofwide}

\subsection*{(2) Funktionale Eindeutigkeit}

\FormulaThmAuto[Multiplikation: Eindeutigkeit]{\forall n,m,k,l\in\mathbb{N}\ \Bigl( ((n,m),k)\in\mathrm{Mult} \land ((n,m),l)\in\mathrm{Mult} \ \rightarrow\ k=l \Bigr)}
\begin{tabproofwide}
  \proofstepwidestar[1]{((n,m),k)\in\mathrm{Mult}}{\rA}
  \proofstepwidestar[2]{((n,m),l)\in\mathrm{Mult}}{\rA}
  \proofstepwide[1]{((n,m),k)\in\mathrm{Mult}}{\rightarrow}{n\cdot m=k}%
    {\FormulaRefAuto{\mathrm{Mult} := \Bigl\{\,((n,m),k)\;\Bigm|\; n,m,k \in \mathbb{N}\ \land\ n\cdot m = k\,\Bigr\}}}
  \proofstepwide[2]{((n,m),l)\in\mathrm{Mult}}{\rightarrow}{n\cdot m=l}%
    {\FormulaRefAuto{\mathrm{Mult} := \Bigl\{\,((n,m),k)\;\Bigm|\; n,m,k \in \mathbb{N}\ \land\ n\cdot m = k\,\Bigr\}}}
  \proofstepwide[1,2]{k}{=}{l}%
    {\rChain{3,4}}
  \proofstepwidestar{\forall n,m,k,l\in\mathbb{N}\ \Bigl( ((n,m),k)\in\mathrm{Mult} \land ((n,m),l)\in\mathrm{Mult} \ \rightarrow\ k=l \Bigr)}%
    {\rUI{5}}
\end{tabproofwide}

\subsection*{(3) Totalität auf der Domäne}

\FormulaThmAuto[Multiplikation: Totalität]{\forall n,m\in\mathbb{N}\ \exists k\in\mathbb{N}\ \bigl(((n,m),k)\in\mathrm{Mult}\bigr)}
\begin{tabproofsplitwide}
  \proofpartwide{Basis \(m=0\)}
    \proofstepwidestar[1]{n\in\mathbb{N}}{\rA}
    \proofstepwidestar[]{0\in\mathbb{N}}{\FormulaRefAuto{0 \in \mathbb{N}}}
    \proofstepwide{}{=}{n\cdot 0}{\FormulaRefAuto{Multiplikation – Basisfall}}
    \proofstepwide{}{=}{0}{\rChain{3}}
    \proofstepwidestar[]{0\in\mathbb{N}}{\FormulaRefAuto{0 \in \mathbb{N}}}
    \proofstepwide{((n,0),0)}{\in}{\mathrm{Mult}}%
      {\FormulaRefAuto{\mathrm{Mult} := \Bigl\{\,((n,m),k)\;\Bigm|\; n,m,k \in \mathbb{N}\ \land\ n\cdot m = k\,\Bigr\}}}
    \proofstepwidestar{\exists k\in\mathbb{N}\,(((n,0),k)\in\mathrm{Mult})}{\rEI{6}}
  \closeproofpartwide

  \proofpartwide{Induktionsschritt \(m\mapsto S(m)\)}
    \proofstepwidestar[1]{n\in\mathbb{N}}{\rA}
    \proofstepwidestar[2]{m\in\mathbb{N}}{\rA}
    \proofstepwidestar[3]{\exists k\in\mathbb{N}\,(((n,m),k)\in\mathrm{Mult})}{\rA}
    \proofstepwide[3]{((n,m),k)}{\in}{\mathrm{Mult}}{\rA}
    \proofstepwide[3]{n\cdot m}{=}{k}{\FormulaRefAuto{\mathrm{Mult} := \Bigl\{\,((n,m),k)\;\Bigm|\; n,m,k \in \mathbb{N}\ \land\ n\cdot m = k\,\Bigr\}}}
    \proofstepwide{}{=}{n\cdot S(m)}{\FormulaRefAuto{Multiplikation – Rekursionsvorschrift}}
    \proofstepwide{}{=}{(n\cdot m)+n}{\rChain{5,6}}
    \proofstepwidestar[]{k\in\mathbb{N}}{\rA}
    \proofstepwidestar[]{(n\cdot m)+n \in \mathbb{N}}{\text{Induktion f.\ Addition / Nachfolger-Axiom}}
    \proofstepwide{((n,S(m)),(n\cdot m)+n)}{\in}{\mathrm{Mult}}%
      {\FormulaRefAuto{\mathrm{Mult} := \Bigl\{\,((n,m),k)\;\Bigm|\; n,m,k \in \mathbb{N}\ \land\ n\cdot m = k\,\Bigr\}}}
    \proofstepwidestar{\exists k'\in\mathbb{N}\,(((n,S(m)),k')\in\mathrm{Mult})}{\rEI{10}}
  \closeproofpartwide

  \proofpartwide{Schluss}
    \proofstepwidestar{}{ \forall m\in\mathbb{N}\ \exists k\in\mathbb{N}\,(((n,m),k)\in\mathrm{Mult}) }%
      {\FormulaRefAuto{P(0),\, \forall x\in\mathbb{N}\,(P(x) \rightarrow P(\mathrm{succ}(x))) \vdash \forall x\in\mathbb{N}\,(P(x))}}
    \proofstepwidestar{}{ \forall n,m\in\mathbb{N}\ \exists k\in\mathbb{N}\,(((n,m),k)\in\mathrm{Mult}) }{\rUI{1}}
  \closeproofpartwide
\end{tabproofsplitwide}

\begin{remark}
Im Induktionsschritt der Multiplikation wurde die Abgeschlossenheit von \(+\) über \(\mathbb{N}\) verwendet,
die ihrerseits unmittelbar per Induktion aus \FormulaRefAuto{Addition – Basisfall} und
\FormulaRefAuto{Addition – Rekursionsvorschrift} folgt (unter Nutzung des
\FormulaRefAuto{S(n) \in \mathbb{N}}).
\end{remark}

\end{document}