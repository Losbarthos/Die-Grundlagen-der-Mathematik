%============================================================
%  Bd. 03 - Mengenlehre %============================================================

\documentclass[main.tex]{subfiles}


\ifSubfilesClassLoaded{
    \usepackage{xr}
    \externaldocument{_B01}
  \externaldocument{_B02}
}{
   % Code für als Subfile eingebunden
}

\title{Bd. 03 - Mengenlehre}
\author{Martin Kunze}
\date{}
\setcounter{file}{3}




\begin{document}

\maketitle
\tableofcontents
%\listoftheorems



\chapter{Einführung}

Die Mengenlehre ist ein fundamentaler Teil der Mathematik, der die Grundlage für viele andere Bereiche bildet. In diesem Kapitel werden wir die Zermelo-Fraenkel (ZF) Axiome der Mengenlehre einführen und diskutieren. Dabei bezeichnen \( A \), \( B \), \( C \) und \( D \) stets Mengen, es sei denn, es wird ausdrücklich etwas anderes angegeben. Alle Variablen, die als Mengen bezeichnet werden, sind implizit durch Allquantoren gebunden, es sei denn, es wird ein anderer Quantor verwendet. Das bedeutet, dass Aussagen wie „\( A = B \)“ oder „\( A \neq B \)“ für alle Mengen \( A \) und \( B \) gelten, ohne dass dies explizit angegeben werden muss.

\chapter{Die Zermelo-Fraenkel-Axiome}
Nachdem wir die grundlegenden Begriffe und Notationen eingeführt haben, wenden wir uns nun den Zermelo-Fraenkel-Axiomen zu, die das Fundament der modernen Mengenlehre bilden. Diese Axiome definieren, wie Mengen gebildet werden können und welche Eigenschaften sie besitzen.
\begin{definition}[Begriff der Menge]
Der \textbf{Begriff der Menge} wird durch das \textbf{Element-Symbol} \(\in\) \textbf{implizit definiert}. 
Das Symbol \(\in\) ist ein binäres Prädikat, das die Mitgliedschaft zwischen einem Element und einer Menge 
ausdrückt, also \(x \in y\) bedeutet, dass \(x\) ein Element von \(y\) ist. Die Eigenschaften von \(\in\) 
werden durch die folgenden Axiome der Zermelo-Fraenkel-Mengenlehre festgelegt, welche zusammen die Menge von 
Aussagen \(\Phi(\in)\) bilden:
\end{definition}

\FormulaAxiomAuto[Extensionalität]{\forall x\, (x \in A \leftrightarrow x \in B) \eqvdash A = B}

\FormulaAxiomAuto[Leere Menge]{\exists O\;\bigl(\forall x\,(x \not\in O)\bigr)}

\FormulaDefAuto[Leere Menge]{\emptyset := \iota O\bigl(\forall x\,(x \not\in O)\bigr)}
\begin{remark}
    Hieraus gewinnen wir für alle \(x\):
    \[
    x \not\in \emptyset.
    \]
Das Symbol \(\emptyset\) wird oft verwendet, um die leere Menge zu repräsentieren. Es stammt aus der skandinavischen Schreibweise des Buchstabens "`O"' und wurde von den Mathematikern André Weil und Bourbaki eingeführt. Es ist wichtig zu beachten, dass \(\emptyset\) nur ein Symbol ist und nicht als eine Zahl oder ein anderes mathematisches Objekt betrachtet werden sollte. In der Mengenlehre repräsentiert es speziell eine Menge, die keine Elemente enthält.
\end{remark}

\FormulaThmAuto{ \exists! O\forall x (x \not\in O) }
\begin{tabproof}
  \proofstep{}{ \exists O\forall x (x \not\in O) }{ \FormulaRefAuto{\exists O\;\bigl(\forall x\,(x \not\in O)\bigr)} }
  \proofstep{2}{ \forall x (x \not\in O) }{ \rA }
  \proofstep{3}{ \forall x (x \not\in P) }{ \rA }
  \proofstep{2}{ \forall x (x \not\in O \lor x \in P) }{ \FormulaRefAuto{\forall x(F(x))\lor\forall x(G(x))\vdash\forall x(F(x)\lor G(x))} }
  \proofstep{3}{ \forall x (x \not\in P \lor x \in O) }{ \FormulaRefAuto{ \forall x(F(x))\lor\forall x(G(x))\vdash\forall x(F(x)\lor G(x))} }
  \proofstep{2,3}{ \forall x (x \in O \leftrightarrow x \in P) }{ \FormulaRefAuto{ \forall x (P(x) \leftrightarrow Q(x)) \dashv \vdash \forall x (\neg P(x) \lor Q(x)) \land \forall x (\neg Q(x) \lor P(x)) } }
  \proofstep{2,3}{ O = P }{ \FormulaRefAuto{ \forall x\, (x \in A \leftrightarrow x \in B) \eqvdash A = B } }
  \proofstep{}{ \exists! O (\forall x (x \not\in O)) }{ \UEI{1,2,3,6} }
\end{tabproof}

\FormulaAxiomAuto[Aussonderung]{\forall A\;\forall P\;\exists B\;\Bigl(\forall x\;\bigl(x \in B \;\leftrightarrow\; x \in A \,\land\, P(x)\bigr)\Bigr)}


\FormulaDefAuto[Aussonderung]{\{x\in A \,\mid\, P(x)\} := \iota B\bigl(\forall x\,(x\in B \leftrightarrow (x\in A \land P(x)))\bigr)}[Sei \(A\) eine Menge und \(P\) ein einstelliges Prädikat. Wir definieren:]

Hieraus gewinnen wir für alle \(x\) das Axiom:
\FormulaAxiomAuto[Aussonderung]{x \in \{x \in A \mid P(x)\} \eqvdash x \in A \land P(x)}
bzw.
\[x \in A, P(x)\vdash x \in \{x \in A \mid P(x)\}\]

\FormulaThmAuto[Zur Eindeutigkeit]{ \forall x (x \in B \leftrightarrow P(x)), \forall x (x \in C \leftrightarrow P(x)) \vdash B = C }
\begin{tabproof}
  \proofstep{1}{ \forall x (x \in B \leftrightarrow P(x)) }{ \rA }
  \proofstep{2}{ \forall x (x \in C \leftrightarrow P(x)) }{ \rA }
  \proofstep{1,2}{ \forall x (x \in B \leftrightarrow x \in C) }{ \FormulaRefAuto{ \forall x(P(x)\leftrightarrow Q(x)), \forall x(R(x)\leftrightarrow Q(x))\vdash \forall x(P(x)\leftrightarrow R(x)) }{1,2} }
  \proofstep{1,2}{ B = C }{ \FormulaRefAuto{ \forall x\, (x \in A \leftrightarrow x \in B) \eqvdash A = B }{3} }
\end{tabproof}

\FormulaThmAuto[Eindeutigkeit der Komprehension]{ \exists B(\forall x(x \in B\leftrightarrow P(x)))\vdash \exists! B(\forall x(x \in B\leftrightarrow P(x))) }
\begin{tabproof}
  \proofstep{1}{ \exists B(\forall x(x \in B\leftrightarrow P(x))) }{ \rA }
  \proofstep{2}{ \forall x(x \in A\leftrightarrow P(x)) }{ \rA }
  \proofstep{3}{ \forall x(x \in B\leftrightarrow P(x)) }{ \rA }
  \proofstep{2,3}{ A = B }{ \FormulaRefAuto{ \forall x (x \in B \leftrightarrow P(x)), \forall x (x \in C \leftrightarrow P(x)) \vdash B = C }{2,3} }
  \proofstep{1}{ \exists! B(\forall x(x \in B\leftrightarrow P(x))) }{ \UEI{1,2,3,4} }
\end{tabproof}

\FormulaThmAuto{ \forall x(P(x)\rightarrow x\in A)\vdash \exists! B(\forall x(x\in B\leftrightarrow P(x))) }
\begin{tabproof}
  \proofstep{1}{ \forall x (P(x) \rightarrow x \in A) }{ \rA }
  \proofstep{1}{ \forall x(x \in A \land P(x) \leftrightarrow P(x)) }{ \FormulaRefAuto{ \forall x(P(x)\rightarrow Q(x))\vdash \forall x((Q(x)\land P(x))\leftrightarrow P(x)) } }
  \proofstep{1}{ \exists B(\forall x(x \in B \leftrightarrow P(x))) }{ \rEI{ \FormulaRefAuto{ \{x\in A \,\mid\, P(x)\} := \iota B\bigl(\forall x\,(x\in B \leftrightarrow (x\in A \land P(x)))\bigr) } } }
  \proofstep{1}{ \exists! B(\forall x(x \in B \leftrightarrow P(x))) }{ \FormulaRefAuto{ \exists B(\forall x(x \in B\leftrightarrow P(x)))\vdash \exists! B(\forall x(x \in B\leftrightarrow P(x))) } }
\end{tabproof}

\FormulaDefAuto[Schnitt]{A \cap B := \{ x \in A \mid x \in B \}}

\FormulaAxiomAuto[Paarmenge]{ \forall A,B\;\exists C\;\bigl(\forall x\,(x \in C \leftrightarrow x=A \lor x=B)\bigr) }

\FormulaDefAuto[Paarmenge]{\{A,B\} := \iota C\Bigl(\forall x\;\bigl(x \in C \;\leftrightarrow\; x = A \lor x = B\bigr)\Bigr)}[Wir definieren für gegebene Mengen \(A\) und \(B\):]

Hieraus gewinnen wir für alle \(x\):
\FormulaAxiomAuto{x \in \{A,B\}\;\eqvdash\;(x = A \lor x = B)}



\FormulaDefAuto[Geordnetes Paar]{\forall a,b((a, b) := \{ \{ a \}, \{ a, b \}) \}}
\begin{remark}
	Ein geordnetes Paar ist eine Paarmenge und damit eindeutig bestimmt.
\end{remark}

\FormulaAxiomAuto[Vereinigung]{\forall A\;\exists C\;\Bigl(\forall x\;\bigl(x \in C \;\leftrightarrow\;\exists B\,(B \in A \,\land\, x \in B)\bigr)\Bigr)}

\FormulaDefAuto[Vereinigung]{\bigcup A := \iota C\Bigl(\forall x\;\bigl(x \in C \;\leftrightarrow\;\exists B\,(B \in A \land x \in B)\bigr)\Bigr)}[Für eine Menge \(A\) definieren wir:]
\begin{remark}
    Hieraus folgt für alle \(x\):
\[
x \in \bigcup A 
\;\eqvdash\;
\exists B\,(B \in A \,\land\, x \in B).
\]
\end{remark}

\FormulaDefAuto[Teilmenge]{ A \subseteq B := \forall x\,(x\in A \rightarrow x\in B) }[Für Mengen \(A\) und \(B\) definieren wir:]{%
In Worten: \(A\) ist Teilmenge von \(B\).
}

\FormulaAxiomAuto[Potenzmenge]{ \forall A\;\exists B\;\Bigl(\forall x\;\bigl(x \in B \leftrightarrow x \subseteq A\bigr)\Bigr) }

\FormulaDefAuto[Potenzmenge]{\mathcal{P}(A) := \iota B\Bigl(\forall x\;\bigl(x \in B \leftrightarrow x \subseteq A\bigr)\Bigr)}[Für eine Menge \(A\) definieren wir:]
\begin{remark}
Hieraus ergibt sich für alle \(x\):
\[
x \in \mathcal{P}(A) 
\;\eqvdash\; 
x \subseteq A.
\]
\end{remark}

\FormulaAxiomAuto[Unendlichkeit]{ \exists A\;\bigl(\emptyset \in A \land \forall x \in A\,(x \cup \{x\} \in A)\bigr) }

\FormulaAxiomAuto[Regularität (Fundamentalsatz)]{ A \neq \emptyset \vdash \exists x \in A \,(x \cap A = \emptyset) }

\begin{remark}
Dieses Axiom verhindert zyklische Mitgliedschaften, indem jede nicht-leere Menge ein 
\textbf{Minimalelement} enthält.    
\end{remark}

\FormulaAxiomAuto[Ersetzung (Relationenform)]{
\forall A\;\forall R\;\Bigl( \forall x\in A\;\exists! y\;(x,y)\in R\;\rightarrow\; \exists C\;\forall y\;\bigl( y\in C\;\leftrightarrow\; \exists x\in A\;(x,y)\in R \bigr) \Bigr)
}

\FormulaAxiomAuto[Auswahlaxiom (AC)]{
\forall x,y \in X \Bigl(x\neq \emptyset \land (x=y\lor x\cap y=\emptyset)\Bigr)
\;\vdash\;\exists Y\,\forall x \in X\,\exists z\bigl(Y\cap x = \{z\}\bigr)
}[Sei \(X\) eine Menge, dann gilt:]
\begin{remark}
Auf Methaebene definieren wir für beliebige Mengen \(x\) und \(A\): \[x\not\in A:=\neg(x\in A)\]
\end{remark}
\begin{remark}
Im folgenden Abschnitten bezeichnen wir \(A,B,C\) als Mengen. Für die Elemente von Mengen verwenden wir häufig Kleinbuchstaben als Symbole wie \(x,y,z\).
\end{remark}

\section{Ungleichheit von Mengen}

\FormulaThmAuto{
A \neq B \eqvdash \exists x (x \not\in A\land x\in B) \lor  \exists x (x \in A\land x\not\in B)
}
\begin{tabproofwide}
  \proofstepwide{A \neq B}{\leftrightarrow}{\neg(\forall x(x\in A\leftrightarrow x\in B))}{\FormulaRefAuto{P \leftrightarrow Q \dashv \vdash \neg P \leftrightarrow \neg Q}{\FormulaRefAuto{\forall x\, (x \in A \leftrightarrow x \in B) \eqvdash A = B}}}
  \proofstepwide{}{\leftrightarrow}{\exists x(x \not\in A\land x\in B)}{\multirow{2}{*}{\FormulaRefAuto{\neg\forall x(P(x)\leftrightarrow Q(x)) \eqvdash \exists x (P(x)\land \neg Q(x))\lor \exists x (Q(x)\land \neg P(x))}{1}}}
  \proofstepwide*{}{\lor}{\exists x(x \in A\land x\not\in B)}{}
  \proofstepwide{A \neq B}{\leftrightarrow}{\exists x(x \not\in A\land x\in B)}{\multirow{2}{*}{\rChain{1,2}}}
  \proofstepwide*{}{\lor}{\exists x(x \in A\land x\not\in B)}{}
\end{tabproofwide}

\FormulaThmAuto{x\in A, x\not\in B\vdash A\neq B}
\begin{tabproof}
  \proofstep{1}{x\in A}{\rA}
  \proofstep{2}{x\not\in B}{\rA}
  \proofstep{1}{\exists x(x\in A\land x\not\in B)}{\rEI{\rAI{1,2}}}
  \proofstep{1}{A\neq B}{\FormulaRefAuto{A \neq B \eqvdash \exists x (x \not\in A\land x\in B) \lor \exists x (x \in A\land x\not\in B)}{\rOIa{3}}}
\end{tabproof}


\chapter{Eigenschaften}

\section{Teilmengen}

\subsection{Grundlegende Eigenschaften}

\FormulaThmAuto{ A\subseteq B,\, x\in A \vdash x\in B }
\begin{tabproof}
  \proofstep{1}{ A\subseteq B }{ \rA }
  \proofstep{2}{ x\in A }{ \rA }
  \proofstep{1}{ x\in A \rightarrow x\in B }{ \rUE{ \FormulaRefAuto{ A \subseteq B := \forall x\,(x\in A \rightarrow x\in B) }{1} } }
  \proofstep{1,2}{ x\in B }{ \rRE{2,3} }
\end{tabproof}

\FormulaThmAuto{ A = B,\, x \in A \vdash x \in B }
\begin{tabproof}
  \proofstep{1}{ A = B }{ \rA }
  \proofstep{2}{ x \in A }{ \rA }
  \proofstep{1}{ x \in A \leftrightarrow x \in B }{ \rUE{ \FormulaRefAuto{ \forall x\, (x \in A \leftrightarrow x \in B) \eqvdash A = B } } }
  \proofstep{1,2}{ x \in B }{ \FormulaRefAuto{ P \leftrightarrow Q,\; P \vdash Q }{3,2} }
\end{tabproof}

\FormulaThmAuto{ a \in A,\; b \not\in A \vdash a \neq b }
\begin{tabproof}
  \proofstep{1}{ a \in A }{ \rA }
  \proofstep{2}{ b \not\in A }{ \rA }
  \proofstep{3}{ a = b }{ \rA }
  \proofstep{1,3}{ b \in A }{ \rIE{3,1} }
  \proofstep{1,2,3}{ \bot }{ \rAI{4,2} }
  \proofstep{1,2}{ a \neq b }{ \rCI{3,5} }
\end{tabproof}

\FormulaThmAuto{A\subseteq C,\, B\subseteq C,\, z\in A\lor z\in B \vdash z\in C}
\begin{tabproofwide}
  \proofstepwidestar[1]{A \subseteq C}{\rA}
  \proofstepwidestar[2]{B \subseteq C}{\rA}
  \proofstepwidestar[3]{z \in A \lor z \in B}{\rA}

  \proofstepwide[1]{z \in A}{\rightarrow}{z \in C}%
    {\rUE{\FormulaRefAuto{A \subseteq B := \forall x\,(x\in A \rightarrow x\in B)}{1}}}
  \proofstepwide[2]{z \in B}{\rightarrow}{z \in C}%
    {\rUE{\FormulaRefAuto{A \subseteq B := \forall x\,(x\in A \rightarrow x\in B)}{2}}}

  \proofstepwidestar[1,2,3]{z \in C}%
    {\FormulaRefAuto{P \rightarrow Q,\, R \rightarrow Q,\, P \lor R \vdash Q}{4,5,3}}
\end{tabproofwide}
\subsection{Ordnungsrelation}

\FormulaThmAuto[Reflexivität von Teilmengen]{ A \subseteq A }
\begin{tabproof}
  \proofstep{}{ x \in A \rightarrow x \in A }{ \FormulaRefAuto{ P \rightarrow P } }
  \proofstep{}{ \forall x(x \in A \rightarrow x \in A) }{ \rUI{1} }
  \proofstep{}{ A \subseteq A }{ \FormulaRefAuto{ A \subseteq B := \forall x\,(x \in A \rightarrow x \in B) }{2} }
\end{tabproof}


\FormulaThmAuto[Antisymmetrie von Teilmengen]{ A \subseteq B \land B \subseteq A \eqvdash A = B }
\begin{tabproofwide}
  \proofstepwide{A \subseteq B \land B \subseteq A}{\leftrightarrow}{\forall x(x \in A \rightarrow x \in B)}{\multirow{2}{*}{\FormulaRefAuto{ A \subseteq B := \forall x\,(x \in A \rightarrow x \in B) }}}
  \proofstepwide{}{ \land }{\forall x(x \in B \rightarrow x \in A)}{}
  \proofstepwide*{}{ \leftrightarrow }{\forall x(x \in A \leftrightarrow x \in B)}{\FormulaRefAuto{\forall x (P(x) \leftrightarrow Q(x)) \dashv \vdash \forall x (P(x) \rightarrow Q(x)) \land \forall x (Q(x) \rightarrow P(x))}{1}}
  \proofstepwide{}{ \leftrightarrow }{A = B}{\FormulaRefAuto{ \forall x\, (x \in A \leftrightarrow x \in B) \eqvdash A = B }{2}}
  \proofstepwide*{A \subseteq B \land B \subseteq A}{ \leftrightarrow }{A = B}{\rChain{1,3}}
\end{tabproofwide}

\FormulaThmAuto{ A \subseteq B, B \subseteq A \vdash A = B }
\begin{tabproof}
    \proofstep{1}{A\subseteq B}{\rA}
    \proofstep{2}{B\subseteq A}{\rA}
    \proofstep{1,2}{A\subseteq B\land B\subseteq A}{\rAI{1,2}}
    \proofstep{1,2}{A\subseteq B\land B\subseteq A}{\FormulaRefAuto{ A \subseteq B \land B \subseteq A \eqvdash A = B }{3}}
\end{tabproof}

\FormulaThmAuto[Transitivität von Teilmengen]{
A \subseteq B, B \subseteq C \vdash A \subseteq C
}
\begin{tabproof}
  \proofstep{1}{A \subseteq B}{\rA}
  \proofstep{2}{B \subseteq C}{\rA}
  \proofstep{1}{\forall x(x \in A \rightarrow x \in B)}{\FormulaRefAuto{A \subseteq B := \forall x(x \in A \rightarrow x \in B)}{1}}
  \proofstep{1}{\forall x(x \in B \rightarrow x \in C)}{\FormulaRefAuto{A \subseteq B := \forall x(x \in A \rightarrow x \in B)}{2}}
  \proofstep{1,2}{\forall x(x \in A \rightarrow x \in C)}{\FormulaRefAuto{\forall x(P(x) \rightarrow Q(x)), \forall x(Q(x) \rightarrow R(x)) \vdash \forall x(P(x) \rightarrow R(x))}{3,4}}
  \proofstep{1,2}{A \subseteq C}{\FormulaRefAuto{A \subseteq B := \forall x(x \in A \rightarrow x \in B)}{5}}
\end{tabproof}

\FormulaThmAuto[Rechtsverträglichkeit von \(\subseteq\) und \(=\)]{
A \subseteq B, B = C \vdash A \subseteq C
}
\begin{tabproof}
  \proofstep{1}{A \subseteq B}{\rA}
  \proofstep{2}{B = C}{\rA}
  \proofstep{3}{B \subseteq C}{\rAEa{\FormulaRefAuto{A \subseteq B \land B \subseteq A \eqvdash A = B}{2}}}
  \proofstep{4}{A \subseteq C}{\FormulaRefAuto{A \subseteq B, B \subseteq C \vdash A \subseteq C}{1,3}}
\end{tabproof}

\begin{remark}[Gemischte Kettenregel]
Auf Basis des vorangegangenen Theorems können \(\subseteq\) und \(=\) nun in einer \emph{Kette} \((\subseteq,=^{*})\) kombiniert werden,  da sie \emph{rechts-verträglich} sind. Ebenso ist \(=\) wegen \FormulaRefAuto{a = b \vdash b = a}{} außerdem Symmetrisch, was mit dem Stern in der Kette illustriert wird.
\end{remark}

\section{Leere Menge}


\FormulaThmAuto{\forall A\,(\emptyset\subseteq A)}
\begin{tabproof}
  \proofstep{}{x \not\in \emptyset}{\FormulaRefAuto{\emptyset := \iota O\bigl(\forall x\,(x \not\in O)\bigr)}{}}
  \proofstep{}{x \not\in A \rightarrow x \not\in \emptyset}{\FormulaRefAuto{Q \vdash P \rightarrow Q}{}}
  \proofstep{}{x \in \emptyset \rightarrow x \in A}{\FormulaRefAuto{P \rightarrow Q \eqvdash \neg Q \rightarrow \neg P}{2}}
  \proofstep{}{\forall x(x \in \emptyset \rightarrow x \in A)}{\rUI{3}}
  \proofstep{}{\emptyset \subseteq A}{\FormulaRefAuto{A \subseteq B := \forall x\,(x\in A \rightarrow x\in B)}{4}}
  \proofstep{}{\forall A(\emptyset \subseteq A)}{\rUI{5}}
\end{tabproof}

\FormulaThmAuto{
A \subseteq B,\, \forall x \in B\,(x \not\in A) \vdash A = \emptyset
}
\begin{tabproof}
  \proofstep{1}{A \subseteq B}{\rA}
  \proofstep{2}{\forall x \in B\,(x \not\in A)}{\rA}
  \proofstep{3}{x \in A}{\rA}
  \proofstep{1,3}{x \in B}{\FormulaRefAuto{A \subseteq B,\, x \in A \vdash x \in B}{1,3}}
  \proofstep{2}{x \in B \rightarrow x \not\in A}{\rUE{2}}
  \proofstep{2}{x \in A \rightarrow x \not\in B}{\FormulaRefAuto{P \rightarrow Q \eqvdash \neg Q \rightarrow \neg P}{5}}
  \proofstep{2,3}{x \not\in B}{\rRE{3,6}}
  \proofstep{1,2,3}{\bot}{\rBI{4,7}}
  \proofstep{1,2}{x \not\in A}{\rCI{3,8}}
  \proofstep{1,2}{\forall x\,(x \not\in A)}{\rUI{9}}
  \proofstep{1,2}{A = \emptyset}{\FormulaRefAuto{\emptyset := \iota O\,(\forall x\,(x \not\in O))}{10}}
\end{tabproof}

\FormulaThmAuto{a \in S \vdash S \neq \emptyset}
\begin{tabproof}
  \proofstep{1}{a \in S}{\rA}
  \proofstep{}{a \not\in \emptyset}{\FormulaRefAuto{\emptyset := \iota O\,(\forall x\,(x \not\in O))}}
  \proofstep{1}{S \neq \emptyset}{\FormulaRefAuto{x \in A,\, x \not\in B \vdash A \neq B}{1,2}}
\end{tabproof}

\FormulaThmAuto{\exists x\,(x \in S) \vdash S \neq \emptyset}
\begin{tabproof}
  \proofstep{1}{\exists x\,(x \in S)}{\rA}
  \proofstep{2}{a \in S}{\rA}
  \proofstep{2}{S \neq \emptyset}{\FormulaRefAuto{a \in S \vdash S \neq \emptyset}{2}}
  \proofstep{1}{S \neq \emptyset}{\rEE{1,2,3}}
\end{tabproof}


\section{Ausgesonderte Menge}

\FormulaThmAuto{x \not\in \{x \in A \mid P(x)\}\eqvdash (x \not\in A \lor \neg P(x))}
\begin{tabproofwide}
  \proofstepwide{x \not\in \{x \in A \mid P(x)\}}{\leftrightarrow}{\neg(x \in A \land P(x))}{\FormulaRefAuto{P \leftrightarrow Q \dashv \vdash \neg P \leftrightarrow \neg Q}{\FormulaRefAuto{\{x\in A \mid P(x)\} := \iota B(\forall x\,(x\in B \leftrightarrow (x\in A \land P(x)) ))}}}
  \proofstepwide{\neg(x \in A \land P(x))}{\leftrightarrow}{x \not\in A \lor \neg P(x)}{\FormulaRefAuto{\neg(P \land Q) \eqvdash \neg P \lor \neg Q}{1}}
  \proofstepwide{x \not\in \{x \in A \mid P(x)\}}{\leftrightarrow}{x \not\in A \lor \neg P(x)}{\rChain{1,2}}
\end{tabproofwide}

\FormulaThmAuto{\{ x \in A \mid P(x) \} \subseteq A}
\begin{tabproof}
  \proofstep{1}{x \in \{ x \in A \mid P(x) \}}{\rA}
  \proofstep{1}{x \in A \land P(x)}{\FormulaRefAuto{\{x\in A \,\mid\, P(x)\} := \iota B(\forall x\,(x\in B \leftrightarrow (x\in A \land P(x))))}{1}}
  \proofstep{1}{x \in A}{\rAEa{2}}
  \proofstep{}{ \{ x \in A \mid P(x) \} \subseteq A }{\FormulaRefAuto{A \subseteq B := \forall x\,(x \in A \rightarrow x \in B)}{\rUI{\rRI{1,3}}}}
\end{tabproof}

\FormulaThmAuto{\forall x \in A\,(P(x)),\, y \in A \vdash P(y)}
\begin{tabproof}
  \proofstep{1}{\forall x \in A\,(P(x))}{\rA}
  \proofstep{2}{y \in A}{\rA}
  \proofstep{1}{y \in A \rightarrow P(y)}{\rUE{1}}
  \proofstep{1,2}{P(y)}{\rRE{3,2}}
\end{tabproof}

\FormulaThmAuto{\forall x \in M(P(x)) \eqvdash M = \{x \in M \mid P(x)\}}
\begin{tabproofsplit}
\proofpart{\(\vdash\)}
  \proofstep{1}{\forall x \in M(P(x))}{\rA}
  \proofstep{1}{x \in M \rightarrow P(x)}{\rUE{1}}
  \proofstep{1}{x \in M \leftrightarrow (x \in M \land P(x))}{\FormulaRefAuto{P \rightarrow Q \vdash P \leftrightarrow (P \land Q)}{2}}
  \proofstep{1}{x \in M \leftrightarrow x \in \{x \in M \mid P(x)\}}{\FormulaRefAuto{\{x\in A \,\mid\, P(x)\} := \iota B\bigl(\forall x\,(x\in B \leftrightarrow (x\in A \land P(x)))\bigr)}{3}}
  \proofstep{1}{\forall x (x \in M \leftrightarrow x \in \{x \in M \mid P(x)\})}{\rUI{4}}
  \proofstep{1}{M = \{x \in M \mid P(x)\}}{\FormulaRefAuto{\forall x\, (x \in A \leftrightarrow x \in B) \eqvdash A = B}{5}}
\closeproofpart

\proofpart{\(\dashv\)}
  \proofstep{1}{M = \{x \in M \mid P(x)\}}{\rA}
  \proofstep{2}{y \in M}{\rA}
  \proofstep{1,2}{y \in \{x \in M \mid P(x)\}}{\FormulaRefAuto{A = B,\, x \in A \vdash x \in B}{1,2}}
  \proofstep{1,2}{P(y)}{\rAEb{\FormulaRefAuto{\{x\in A \,\mid\, P(x)\} := \iota B\bigl(\forall x\,(x\in B \leftrightarrow (x\in A \land P(x)))\bigr)}{3}}}
  \proofstep{1}{y \in M \rightarrow P(y)}{\rRI{2,4}}
  \proofstep{1}{\forall x \in M(P(x))}{\rUI{5}}
\closeproofpart
\end{tabproofsplit}

\FormulaThmAuto{A\subseteq \{x\in B\mid P(x)\}\vdash \forall x\in A(P(x))}
\begin{tabproof}
\proofstep{1}{A\subseteq \{x\in B\mid P(x)\}}{\rA}
\proofstep{2}{x\in A}{\rA}
\proofstep{1,2}{x\in \{x\in B\mid P(x)\}}{\FormulaRefAuto{ A\subseteq B,\, x\in A \vdash x\in B }{1,2}}
\proofstep{1,2}{P(x)}{\rAEb{\FormulaRefAuto{x \in \{x \in A \mid P(x)\} \eqvdash x \in A \land P(x)}{3}}}
\proofstep{1,2}{\forall x\in A(P(x))}{\rUI{\rRI{2,4}}}
\end{tabproof}

\section{Russel Paradoxon und die universelle Menge}

\FormulaThmAuto[Russells Paradoxon in der ZF-Mengenlehre]{\neg \exists U \forall A (A \in U \leftrightarrow A \not\in A)}
\begin{tabproof}
\proofstep{1}{\exists U \forall A (A \in U \leftrightarrow A \notin A)}{\rA}
\proofstep{2}{\forall A (A \in U \leftrightarrow A \notin A)}{\rA}
\proofstep{2}{U \in U \leftrightarrow U \notin U}{\rUE{2}}
\proofstep{}{\neg(U \in U \leftrightarrow U \notin U)}{\FormulaRefAuto{\neg (P\leftrightarrow \neg P)}}
\proofstep{2}{\bot}{\rBI{3,4}}
\proofstep{1}{\bot}{\rEE{1,2,5}}
\proofstep{}{\neg\exists U \forall A (A \in U \leftrightarrow A \notin A)}{\rCI{1,6}}
\end{tabproof}



Angenommen, es gibt eine universelle Menge \( U \) in der ZF-Mengenlehre, dann führt dies aufgrund des nachstehenden Satzes zu einem Widerspruch. 


\FormulaThmAuto{\exists U \forall A (A \in U)\vdash \forall A(A\not\in A\leftrightarrow A\in U\land A\not\in A)}
\begin{tabproof}
\proofstep{1}{\exists U \forall A (A \in U)}{\rA}
\proofstep{2}{\forall A (A \in U)}{\rA}
\proofstep{2}{\forall A (A \not\in A\leftrightarrow A\in U\land A\notin A)}{\FormulaRefAuto{\forall x(P(x)) \vdash \forall x(Q(x)\leftrightarrow P(x)\land Q(x))}{2}}
\proofstep{1}{\forall A (A \not\in A\leftrightarrow A\in U\land A\notin A)}{\rEE{1,2,3}}
\end{tabproof}

\FormulaThmAuto{\neg \exists U \forall A (A \in U)}
\begin{tabproof}
\proofstep{1}{\exists U \forall A (A \in U)}{\rA}
\proofstep{1}{\forall A(A\not\in A\leftrightarrow A\in U\land A\not\in A)}{\FormulaRefAuto{\exists U \forall A (A \in U)\vdash \forall A(A\not\in A\leftrightarrow A\in U\land A\not\in A)}{1}}
\proofstep{1}{\exists B\forall A(A\in B\leftrightarrow A\in U\land A\notin A)}{\FormulaRefAuto{\forall A\;\forall P\;\exists B\;\Bigl(\forall x\;\bigl(x \in B \;\leftrightarrow\; x \in A \,\land\, P(x)\bigr)\Bigr)}{2}}
\proofstep{4}{\forall A(A\in B\leftrightarrow A\in U\land A\notin A)}{\rA}
\proofstep{1,4}{\forall A(A\in B\leftrightarrow A\notin A)}{\FormulaRefAuto{\forall x(P(x)\leftrightarrow Q(x)), \forall x(R(x)\leftrightarrow Q(x))\vdash \forall x(P(x)\leftrightarrow R(x))}{4,2}}
\proofstep{1,4}{\exists B\forall A(A\in B\leftrightarrow A\notin A)}{\rEI{5}}
\proofstep{}{\neg\exists B\forall A(A\in B\leftrightarrow A\notin A)}{\FormulaRefAuto{\neg \exists U \forall A (A \in U \leftrightarrow A \not\in A)}}
\proofstep{1,4}{\bot}{\rBI{6,7}}
\proofstep{1}{\bot}{\rEE{3,4,8}}
\proofstep{}{\neg \exists U \forall A (A \in U)}{\rCI{1,9}}
\end{tabproof}

\section{Schnittmengen}

\FormulaThmAuto{x \in A \cap B \vdash x \in A}
\begin{tabproof}
\proofstep{1}{x \in A \cap B}{\rA}
\proofstep{1}{x \in A}{\rAEa{\FormulaRefAuto{\{x\in A \,\mid\, P(x)\} := \iota B\bigl(\forall x\,(x\in B \leftrightarrow (x\in A \land P(x)))\bigr)}{\FormulaRefAuto{A \cap B := \{ x \in A \mid x \in B \}}}}}
\end{tabproof}

\FormulaThmAuto{x \in A \cap B \vdash x \in B}
\begin{tabproof}
\proofstep{1}{x \in A \cap B}{\rA}
\proofstep{1}{x \in B}{\rAEb{\FormulaRefAuto{\{x\in A \,\mid\, P(x)\} := \iota B\bigl(\forall x\,(x\in B \leftrightarrow (x\in A \land P(x)))\bigr)}{\FormulaRefAuto{A \cap B := \{ x \in A \mid x \in B \}}}}}
\end{tabproof}

\FormulaThmAuto{x \in A \cap B \eqvdash x \in A \land x \in B}
\begin{tabproof}
\proofstepstar{}{x \in A \cap B \leftrightarrow x \in A \land x \in B}{\FormulaRefAuto{\{x\in A \,\mid\, P(x)\} := \iota B\bigl(\forall x\,(x\in B \leftrightarrow (x\in A \land P(x)))\bigr)}{\FormulaRefAuto{A \cap B := \{ x \in A \mid x \in B \}}}}
\end{tabproof}

\FormulaThmAuto{x \in A, x\in B \vdash x \in A\cap B}
\begin{tabproof}
\proofstep{1}{x \in A}{\rA}
\proofstep{2}{x \in B}{\rA}
\proofstep{1,2}{x \in A\land x\in B}{\rAI{1,2}}
\proofstep{1,2}{x \in A\cap B}{\FormulaRefAuto{x \in A \cap B \eqvdash x \in A \land x \in B}{3}}
\end{tabproof}

\FormulaThmAuto{x \in (A \cap B) \cap C \eqvdash (x \in A \land x \in B) \land x \in C}
\begin{tabproofwide}
  \proofstepwide{x \in (A \cap B)}{\leftrightarrow}{x \in A \land x \in B}{\FormulaRefAuto{x \in A \cap B \eqvdash x \in A \land x \in B}}
  \proofstepwide{x \in (A \cap B) \cap C}{\leftrightarrow}{x \in (A \cap B) \land x \in C}{\FormulaRefAuto{x \in A \cap B \eqvdash x \in A \land x \in B}}
  \proofstepwide{}{\leftrightarrow}{(x \in A \land x \in B) \land x \in C}{\rLRS{1}}
  \proofstepwide{x \in (A \cap B) \cap C}{\leftrightarrow}{(x \in A \land x \in B) \land x \in C}{\rChain{2,3}}
\end{tabproofwide}

\FormulaThmAuto{x \in A \cap (B \cap C) \eqvdash x \in A \land (x \in B \land x \in C)}
\begin{tabproofwide}
  \proofstepwide{x \in B \cap C}{\leftrightarrow}{x \in B \land x \in C}{\FormulaRefAuto{x \in A \cap B \eqvdash x \in A \land x \in B}}
  \proofstepwide{x \in A \cap (B \cap C)}{\leftrightarrow}{x \in A \land x \in (B \cap C)}{\FormulaRefAuto{x \in A \cap B \eqvdash x \in A \land x \in B}}
  \proofstepwide{}{\leftrightarrow}{x \in A \land (x \in B \land x \in C)}{\rLRS{1}}
  \proofstepwide{x \in A \cap (B \cap C)}{\leftrightarrow}{x \in A \land (x \in B \land x \in C)}{\rChain{2,3}}
\end{tabproofwide}


\FormulaThmAuto[Idempotenz des Schnitts]{A = A \cap A}
\begin{tabproofwide}
  \proofstepwide{x \in A \cap A}{\leftrightarrow}{x \in A}{\rAEa{\FormulaRefAuto{\{x\in A \mid P(x)\} := \iota B(\forall x\,(x\in B \leftrightarrow (x\in A \land P(x))))}{\FormulaRefAuto{A \cap B := \{ x \in A \mid x \in B \}}}}}
  \proofstepwide{A}{=}{A \cap A}{\FormulaRefAuto{\forall x\, (x \in A \leftrightarrow x \in B) \eqvdash A = B}{\rUI{1}}}
\end{tabproofwide}

\FormulaThmAuto[Kommutativität des Schnitts]{A \cap B = B \cap A}
\begin{tabproofwide}
  \proofstepwide{x \in A \cap B}{\leftrightarrow}{x \in A \land x \in B}{\FormulaRefAuto{x \in A \cap B \eqvdash x \in A \land x \in B}}
  \proofstepwide{}{ \leftrightarrow }{x \in B \land x \in A}{\FormulaRefAuto{P \land Q \vdash Q \land P}{1}}
  \proofstepwide{}{ \leftrightarrow }{x \in B \cap A}{\FormulaRefAuto{x \in A \cap B \eqvdash x \in A \land x \in B}{2}}
  \proofstepwide{x \in A \cap B}{\leftrightarrow}{x \in B \cap A}{\rChain{1,3}}
  \proofstepwide{A \cap B}{=}{B \cap A}{\FormulaRefAuto{\forall x\, (x \in A \leftrightarrow x \in B) \eqvdash A = B}{\rUI{4}}}
\end{tabproofwide}


\FormulaThmAuto[Assoziativität des Schnitts]{(A \cap B) \cap C = A \cap (B \cap C)}
\begin{tabproofwide}
  \proofstepwide{x \in (A \cap B) \cap C}{\leftrightarrow}{(x \in A \land x \in B) \land x \in C}{\FormulaRefAuto{x \in (A \cap B) \cap C \eqvdash (x \in A \land x \in B) \land x \in C}}
  \proofstepwide{}{\leftrightarrow}{x \in A \land (x \in B \land x \in C)}{\FormulaRefAuto{P \land (Q \land R) \eqvdash (P \land Q) \land R}{1}}
  \proofstepwide{}{\leftrightarrow}{x \in A \cap (B \cap C)}{\FormulaRefAuto{x \in A \cap (B \cap C) \eqvdash x \in A \land (x \in B \land x \in C)}{}}
  \proofstepwide{x \in (A \cap B) \cap C}{\leftrightarrow}{x \in A \cap (B \cap C)}{\rChain{1,3}}
  \proofstepwide{(A \cap B) \cap C}{=}{A \cap (B \cap C)}{\FormulaRefAuto{\forall x\, (x \in A \leftrightarrow x \in B) \eqvdash A = B}{\rUI{4}}}
\end{tabproofwide}


\FormulaThmAuto{A \cap B \subseteq A}
\begin{tabproof}
  \proofstep{1}{x \in A \cap B}{\rA}
  \proofstep{1}{x \in A}{\rAEa{\FormulaRefAuto{\{x\in A \,\mid\, P(x)\} := \iota B\bigl(\forall x\,(x\in B \leftrightarrow (x\in A \land P(x)))\bigr)}{\FormulaRefAuto{A \cap B := \{ x \in A \mid x \in B \}}{1}}}}
  \proofstep{}{A \cap B \subseteq A}{\FormulaRefAuto{ A \subseteq B := \forall x\,(x\in A \rightarrow x\in B) }{\rUI{\rRI{1,2}}}}
\end{tabproof}

\FormulaThmAuto{A \cap B \subseteq B}
\begin{tabproof}
  \proofstep{1}{x \in A \cap B}{\rA}
  \proofstep{1}{x \in B}{\rAEb{\FormulaRefAuto{\{x\in A \,\mid\, P(x)\} := \iota B\bigl(\forall x\,(x\in B \leftrightarrow (x\in A \land P(x)))\bigr)}{\FormulaRefAuto{A \cap B := \{ x \in A \mid x \in B \}}{1}}}}
  \proofstep{}{A \cap B \subseteq B}{\FormulaRefAuto{ A \subseteq B := \forall x\,(x\in A \rightarrow x\in B) }{\rUI{\rRI{1,2}}}}
\end{tabproof}

\FormulaThmAuto{A \subseteq B \vdash A \cap C \subseteq B \cap C}
\begin{tabproof}
  \proofstep{1}{A \subseteq B}{\rA}
  \proofstep{2}{x \in A \cap C}{\rA}
  \proofstep{2}{x \in A}{\FormulaRefAuto{x \in A \cap B \vdash x \in A}{2}}
  \proofstep{2}{x \in C}{\FormulaRefAuto{x \in A \cap B \vdash x \in B}{2}}
  \proofstep{1,2}{x \in B}{\FormulaRefAuto{A \subseteq B,\, x \in A \vdash x \in B}{1,3}}
  \proofstep{1,2}{x \in B \cap C}{\FormulaRefAuto{x \in A \cap B \eqvdash x \in A \land x \in B}{\rAI{5,4}}}
  \proofstep{1}{A \cap C \subseteq B \cap C}{\FormulaRefAuto{A \subseteq B := \forall x\,(x \in A \rightarrow x \in B)}{\rUI{\rRI{2,6}}}}
\end{tabproof}


\FormulaThmAuto{A \subseteq B \eqvdash A \cap B = A}
\begin{tabproofsplitwide}
  \proofpartwide{\(\vdash\)}
    \proofstepwidestar[1]{A \subseteq B}{\rA}
    \proofstepwidestar[]{A \cap B \subseteq A}{\FormulaRefAuto{A \cap B \subseteq A}}
    \proofstepwide{A}{=}{A \cap A}{\FormulaRefAuto{A = A \cap A}}
    \proofstepwide[1]{}{\subseteq}{A \cap B}{\FormulaRefAuto{A \subseteq B \vdash A \cap C \subseteq B \cap C}{1}}
    \proofstepwide[1]{A}{\subseteq}{A \cap B}{\rChain{3,4}}
    \proofstepwide[1]{A \cap B}{=}{A}{\FormulaRefAuto{A \subseteq B \land B \subseteq A \eqvdash A = B}{\rAI{2,5}}}
  \closeproofpartwide

  \proofpartwide{\(\dashv\)}
    \proofstepwidestar[1]{A \cap B = A}{\rA}
    \proofstepwide[1]{A}{\subseteq}{A \cap B}{\rAEb{\FormulaRefAuto{A \subseteq B \land B \subseteq A \eqvdash A = B}{1}}}
    \proofstepwide[1]{A \cap B}{\subseteq}{B}{\FormulaRefAuto{A \cap B \subseteq B}}
    \proofstepwide[1]{A}{\subseteq}{B}{\FormulaRefAuto{A \subseteq B, B \subseteq C \vdash A \subseteq C}{2,3}}
  \closeproofpartwide
\end{tabproofsplitwide}

\FormulaThmAuto{B \subseteq A \eqvdash A \cap B = B}
\begin{tabproof}
  \proofstep{}{B \subseteq A \leftrightarrow B \cap A = B}{\FormulaRefAuto{A \subseteq B \eqvdash A \cap B = A}}
  \proofstep{}{B \cap A = A \cap B}{\FormulaRefAuto{A \cap B = B \cap A}}
  \proofstep{}{B \subseteq A \leftrightarrow A \cap B = B}{\rLRS{2,1}}
\end{tabproof}

\FormulaThmAuto{\emptyset \cap A = \emptyset}
\begin{tabproof}
  \proofstep{}{ \emptyset \subseteq A }{ \rUE{\FormulaRefAuto{\forall A\,(\emptyset \subseteq A)} } }
  \proofstep{}{ \emptyset \cap A = \emptyset }{ \FormulaRefAuto{ A \subseteq B \eqvdash A \cap B = A }{1} }
\end{tabproof}


\FormulaThmAuto{A \cap \emptyset = \emptyset}
\begin{tabproof}
  \proofstep{}{ \emptyset \subseteq A }{ \rUE{\FormulaRefAuto{\forall A\,(\emptyset \subseteq A)} } }
  \proofstep{}{ A \cap \emptyset = \emptyset }{ \FormulaRefAuto{ B \subseteq A \eqvdash A \cap B = B }{1} }
\end{tabproof}


\FormulaThmAuto{A \cap B = \emptyset,\ x \in A \vdash x \notin B}
\begin{tabproof}
  \proofstep{1}{A \cap B = \emptyset}{\rA}
  \proofstep{2}{x \in A}{\rA}
  \proofstep{3}{x \in B}{\rA}
  \proofstep{2,3}{x \in A \cap B}{\FormulaRefAuto{x \in A \cap B \eqvdash x \in A \land x \in B}{\rAI{2,3}}}
  \proofstep{1,2,3}{x \in \emptyset}{\rIE{1,4}}
  \proofstep{}{x \notin \emptyset}{\rUE{\FormulaRefAuto{\emptyset := \iota O\bigl(\forall x\,(x \not\in O)\bigr)}{}}}
  \proofstep{}{ \bot }{\rBI{5,6}}
  \proofstep{1,2}{x \notin B}{\rCE{1,2}}
\end{tabproof}

\FormulaThmAuto{A \cap B = \emptyset,\ x \in B \vdash x \notin A}
\begin{tabproof}
  \proofstep{1}{A \cap B = \emptyset}{\rA}
  \proofstep{2}{x \in B}{\rA}
  \proofstep{}{B \cap A = A \cap B}{\FormulaRefAuto{A \cap B = B \cap A}}
  \proofstep{1}{B \cap A = \emptyset}{\rIE{1,3}}
  \proofstep{1,2}{x \notin A}{\FormulaRefAuto{A \cap B = \emptyset,\ x \in A \vdash x \notin B}{4,2}}
\end{tabproof}


\subsection{Der unendliche Schnitt}

In diesem Abschnitt sei \( P \) ein Prädikat, das einer Menge \( A \) eine Eigenschaft zuweist.

\FormulaThmAuto{P(C) \vdash \{ x \in B \mid \forall A (P(A) \rightarrow x \in A) \} \subseteq \{ x \in C \mid \forall A (P(A) \rightarrow x \in A) \}}
\begin{notation*}
Wir bezeichnen mit \( I_B := \{ x \in B \mid \forall A (P(A) \rightarrow x \in A) \} \) und entsprechend \( I_C := \{ x \in C \mid \forall A (P(A) \rightarrow x \in A) \} \).
\end{notation*}
\begin{tabproof}
  \proofstep{1}{P(C)}{\rA}
  \proofstep{2}{x \in I_B}{\rA}
  \proofstep{2}{\forall A (P(A) \rightarrow x \in A)}{\rAEb{\FormulaRefAuto{\{x\in A \,\mid\, P(x)\} := \iota B\bigl(\forall x\,(x\in B \leftrightarrow (x\in A \land P(x)))\bigr)}{2}}}
  \proofstep{2}{P(C) \rightarrow x \in C}{\rUE{4}}
  \proofstep{1,2}{x \in C}{\rRE{1,4}}
  \proofstep{1,2}{x \in I_C}{\FormulaRefAuto{\{x\in A \,\mid\, P(x)\} := \iota B\bigl(\forall x\,(x\in B \leftrightarrow (x\in A \land P(x)))\bigr)}{\rAI{5,3}}}
  \proofstep{1}{I_B \subseteq I_C}{\FormulaRefAuto{A \subseteq B := \forall x\,(x \in A \rightarrow x \in B)}{\rUI{\rRI{2,6}}}}
\end{tabproof}

\FormulaThmAuto{P(B),\, P(C) \vdash \{ x \in B \mid \forall A (P(A) \rightarrow x \in A) \} = \{ x \in C \mid \forall A (P(A) \rightarrow x \in A) \}}
\begin{notation*}
Wir bezeichnen mit \( I_B := \{ x \in B \mid \forall A (P(A) \rightarrow x \in A) \} \) und entsprechend \( I_C := \{ x \in C \mid \forall A (P(A) \rightarrow x \in A) \} \).
\end{notation*}
\begin{tabproof}
  \proofstep{1}{P(B)}{\rA}
  \proofstep{2}{P(C)}{\rA}
  \proofstep{2}{I_B \subseteq I_C}{\FormulaRefAuto{P(C) \vdash \{ x \in B \mid \forall A (P(A) \rightarrow x \in A) \} \subseteq \{ x \in C \mid \forall A (P(A) \rightarrow x \in A) \}}{2}}
  \proofstep{1}{I_C \subseteq I_B}{\FormulaRefAuto{P(C) \vdash \{ x \in B \mid \forall A (P(A) \rightarrow x \in A) \} \subseteq \{ x \in C \mid \forall A (P(A) \rightarrow x \in A) \}}{1}}
  \proofstep{1,2}{I_B = I_C}{\FormulaRefAuto{A \subseteq B \land B \subseteq A \eqvdash A = B}{3,4}}
\end{tabproof}

\FormulaThmAuto{\exists C\forall B\Bigl(P(B)\rightarrow C = \{ x \in B \mid \forall A (P(A) \rightarrow x \in A) \}\Bigr)}
\begin{notation*}
Wir bezeichnen mit \( I_B := \{ x \in B \mid \forall A (P(A) \rightarrow x \in A) \} \) und entsprechend \( I_D := \{ x \in D \mid \forall A (P(A) \rightarrow x \in A) \} \).
\end{notation*}
\begin{tabproofsplit}
  \proofpart{Fall 1: \( \exists D(P(D)) \vdash \exists C\forall B\bigl(P(B)\rightarrow C = I_B\bigr) \)}
    \proofstep{1}{\exists D(P(D))}{\rA}
    \proofstep{2}{P(D)}{\rA}
    \proofstep{3}{P(B)}{\rA}
    \proofstep{2,3}{I_D = I_B}{\FormulaRefAuto{P(B),\, P(C) \vdash \{ x \in B \mid \forall A (P(A) \rightarrow x \in A) \} = \{ x \in C \mid \forall A (P(A) \rightarrow x \in A) \}}}
    \proofstep{2}{\exists C\,\forall B\,\bigl(P(B) \rightarrow C = I_B\bigr)}{\rEI{\rUI{\rRI{3,4}}}}
  \closeproofpart

  \proofpart{Fall 2: \( \forall D(\neg P(D)) \vdash \exists C\forall B\bigl(P(B)\rightarrow C = I_B\bigr) \)}
    \proofstep{1}{\forall D(\neg P(D))}{\rA}
    \proofstep{1}{\forall B\,\bigl(\neg P(B) \lor C = I_B\bigr)}{\FormulaRefAuto{\forall x(F(x)) \lor \forall x(G(x)) \vdash \forall x(F(x) \lor G(x))}}
    \proofstep{1}{\forall B\,\bigl(P(B) \rightarrow C = I_B\bigr)}{\rLRS{\FormulaRefAuto{\neg(P \lor Q) \eqvdash \neg P \land \neg Q}{}, 2}}
    \proofstep{1}{\exists C\,\forall B\,\bigl(P(B) \rightarrow C = I_B\bigr)}{\rEI{3}}
  \closeproofpart

  \proofpart{Fallunterscheidung über das klassische Prinzip \( P \lor \neg P \)}
    \proofstep{}{ \exists D(P(D)) \lor \forall D(\neg P(D)) }{\rLRS{\FormulaRefAuto{\forall x(\neg P(x)) \eqvdash \neg \exists x(P(x))}{}, \FormulaRefAuto{P \lor \neg P}}}
    \proofstep{}{ \exists C\,\forall B\,\bigl(P(B) \rightarrow C = I_B\bigr) }{\rOE{1,1,5,1,4}}
  \closeproofpart
\end{tabproofsplit}


\FormulaThmAuto{
  \begin{aligned}
    &P(B_0),\;
    \forall B\Bigl(P(B)\rightarrow C= \{ x \in B \mid \forall A (P(A) \rightarrow x \in A) \}\Bigr),\\
    &\forall B\Bigl(P(B)\rightarrow D= \{ x \in B \mid \forall A (P(A) \rightarrow x \in A) \}\Bigr)
    \vdash C=D
  \end{aligned}
}
\begin{notation*}
Wir bezeichnen mit \( I_B := \{ x \in B \mid \forall A (P(A) \rightarrow x \in A) \} \).
\end{notation*}
\begin{tabproof}
    \proofstep{1}{\forall B\bigl(P(B)\rightarrow C=I_B\bigr)}{\rA}
    \proofstep{2}{\forall B\bigl(P(B)\rightarrow D=I_B\bigr)}{\rA}
    \proofstep{3}{P(B_0)}{\rA}
    \proofstep{1}{P(B_0) \rightarrow C=I_{B_0}}{\rUE{1}}
    \proofstep{1,3}{C = I_{B_0}}{\rRE{3,4}}
    \proofstep{2}{P(B_0) \rightarrow D=I_{B_0}}{\rUE{2}}
    \proofstep{2,3}{D = I_{B_0}}{\rRE{3,6}}
    \proofstep{1,2,3}{C = D}{\FormulaRefAuto{a = b,\, c = b \vdash a = c}{5,7}}
\end{tabproof}


\FormulaThmAuto{P(D)\vdash \exists! C\forall B(P(B)\rightarrow C= \{ x \in B \mid \forall A (P(A) \rightarrow x \in A) \})}
\begin{notation*}
Es sei \( I_B := \{ x \in B \mid \forall A (P(A) \rightarrow x \in A) \} \).
\end{notation*}
\begin{tabproof}
  \proofstep{}{ \exists C\forall B\bigl(P(B)\rightarrow C=I_B\bigr) }{ \FormulaRefAuto{\exists C\forall B\Bigl(P(B)\rightarrow C = \{ x \in B \mid \forall A (P(A) \rightarrow x \in A) \}\Bigr)} }
  \proofstep{2}{\forall B\bigl(P(B)\rightarrow C=I_B\bigr)}{\rA}
  \proofstep{3}{\forall B\bigl(P(B)\rightarrow D=I_B\bigr)}{\rA}
  \proofstep{4}{P(D)}{\rA}
  \proofstep{2,3,4}{C = D}{\FormulaRefAuto{P(B_0),\forall B\bigl(P(B)\rightarrow C= \{ x \in B \mid \forall A (P(A) \rightarrow x \in A) \}\bigr),\, \forall B\bigl(P(B)\rightarrow D= \{ x \in B \mid \forall A (P(A) \rightarrow x \in A) \}\bigr) \vdash C=D}{4,2,3}}
  \proofstep{4}{\exists! C\,\forall B\bigl(P(B)\rightarrow C=I_B\bigr)}{\UEI{1,2,3,5}}
\end{tabproof}

\FormulaDefAuto[Der unendliche Schnitt]{\exists A\,P(A)\rightarrow \bigcap_{P(B)} B := \iota C\,\forall D\Bigl(P(D)\rightarrow C = \{ x \in D \mid \forall A (P(A)\rightarrow x \in A) \}\Bigr)}

% Gleichbedeutende Notation mittels Mengenschreibweise
% (nur eingeführt, wenn der Schnitt definiert ist)
\FormulaDefAuto[Notationserweiterung]{\exists A\,P(A)\rightarrow\bigcap \{\,B \mid P(B)\,\} := \bigcap_{P(B)} B}


\FormulaThmAuto{P(A) \vdash \bigcap_{P(B)} B = \{ x \in A \mid \forall D (P(D) \rightarrow x \in D) \}}
\begin{notation*}
Wir bezeichnen mit \( I_A := \{ x \in A \mid \forall A (P(A) \rightarrow x \in A) \} \) und entsprechend \( I_D := \{ x \in D \mid \forall A (P(A) \rightarrow x \in A) \} \).
\end{notation*}
\begin{tabproof}
  \proofstep{1}{P(A)}{\rA}
  \proofstep{1}{\exists M(P(M))}{\rEI{1}}
  \proofstep{1}{\forall D\Bigl(P(D) \rightarrow \bigcap_{P(B)} B = I_D\Bigr)}{\FormulaRefAuto{\exists A\,P(A)\rightarrow \bigcap_{P(B)} B := \iota C\,\forall D\bigl(P(D)\rightarrow C = \{ x \in D \mid \forall A (P(A)\rightarrow x \in A) \}\bigr)}{2}}
  \proofstep{1}{P(A)\rightarrow \bigcap_{P(B)} B = I_A}{\rUE{3}}
  \proofstep{1}{\bigcap_{P(B)} B = I_A}{\rRE{4,1}}
\end{tabproof}

\FormulaThmAuto{P(A) \vdash x \in \bigcap_{P(B)} B\leftrightarrow x \in \{\,x \in A \mid \forall D\,(P(D) \rightarrow x \in D)\,\}}
\begin{notation*}
Wir bezeichnen mit \( I_A := \{ x \in A \mid \forall A (P(A) \rightarrow x \in A) \} \).
\end{notation*}
\begin{tabproof}
  \proofstep{1}{P(A)}{\rA}
  \proofstep{1}{x\in \bigcap_{P(B)} B\leftrightarrow x\in I_A}{\rUE{\FormulaRefAuto{\forall x\, (x \in A \leftrightarrow x \in B) \eqvdash A = B}{\FormulaRefAuto{P(A) \vdash \bigcap_{P(B)} B = \{ x \in A \mid \forall D\, (P(D) \rightarrow x \in D) \}}{1}}}}
\end{tabproof}

\FormulaThmAuto{P(A) \vdash x \in \bigcap_{P(B)} B \leftrightarrow \forall C\, (P(C) \rightarrow x \in C)}
\begin{notation*}
Wir bezeichnen mit \( I_A := \{ x \in A \mid \forall A (P(A) \rightarrow x \in A) \} \).    
\end{notation*}
\begin{tabproofwide}
  \proofstepwidestar[1]{P(A)}{\rA}
  \proofstepwidestar[2]{\forall C\, (P(C)\rightarrow x\in C)}{\rA}
  \proofstepwidestar[]{\forall C\, (P(C)\rightarrow x\in C) \rightarrow x\in A}{\rRI{2,\FormulaRefAuto{P(a), \forall x\, (P(x) \rightarrow Q(x)) \vdash Q(a)}{1,2}}}
  \proofstepwide[1]{x\in \bigcap_{P(B)} B}{\leftrightarrow}{x\in I_A}{\FormulaRefAuto{P(A) \vdash x \in \bigcap_{P(B)} B\leftrightarrow x \in \{\,x \in A \mid \forall D\,(P(D) \rightarrow x \in D)\,\}}{1}}

  % ---- hier die aufgesplittete Zeile (2 Zeilen) ----
  \proofstepwide[1]{}{\leftrightarrow}{x\in A}{\multirow{2}{*}{\FormulaRefAuto{x \in \{x \in A \mid P(x)\} \eqvdash x \in A \land P(x)}{4}}}
  \proofstepwide*{}{\land}{\forall C\, (P(C)\rightarrow x\in C)}{}
  % --------------------------------------------------

  \proofstepwide[1]{}{\leftrightarrow}{\forall C\, (P(C)\rightarrow x\in C)}{\FormulaRefAuto{P \rightarrow Q \vdash P \leftrightarrow (Q \land P)}{3}}
  \proofstepwide[1]{x\in \bigcap_{P(B)} B}{\leftrightarrow}{\forall C\, (P(C)\rightarrow x\in C)}{\rChain{4,6}}
\end{tabproofwide}

\FormulaThmAuto{\exists A(P(A)) \vdash x \in \bigcap_{P(B)} B \leftrightarrow \forall C\, (P(C) \rightarrow x \in C)}
\begin{tabproof}
    \proofstep{1}{\exists A(P(A))}{\rA}
    \proofstep{2}{P(A)}{\rA}
    \proofstep{2}{x \in \bigcap_{P(B)} B \leftrightarrow \forall C\, (P(C) \rightarrow x \in C)}{\FormulaRefAuto{P(A) \vdash x \in \bigcap_{P(B)} B \leftrightarrow \forall C\, (P(C) \rightarrow x \in C)}{2}}
    \proofstep{1}{x \in \bigcap_{P(B)} B \leftrightarrow \forall C\, (P(C) \rightarrow x \in C)}{\rEE{1,2,3}}
\end{tabproof}



\FormulaThmAuto{P(C)\vdash \bigcap_{P(A)} A \subseteq C}
\begin{tabproof}
  \proofstep{1}{P(C)}{\rA}
  \proofstep{2}{x \in \bigcap_{P(A)} A}{\rA}
  \proofstep{1,2}{\forall A\,\bigl(P(A)\rightarrow x\in A\bigr)}{\FormulaRefAuto{P \leftrightarrow Q, P \vdash Q}{\FormulaRefAuto{P(A) \vdash x \in \bigcap_{P(B)} B \leftrightarrow \forall C\, (P(C) \rightarrow x \in C)}{1},\,2}}
  \proofstep{1,2}{x \in C}{\FormulaRefAuto{P(a), \forall x\,\bigl(P(x) \rightarrow Q(x)\bigr) \vdash Q(a)}{1,3}}
  \proofstep{1}{\forall x\,\bigl(x\in \bigcap_{P(A)} A \rightarrow x\in C\bigr)}{\rUI{\rRI{2,4}}}
  \proofstep{1}{\bigcap_{P(A)} A \subseteq C}{\FormulaRefAuto{A \subseteq B := \forall x\,\bigl(x\in A \rightarrow x\in B\bigr)}{5}}
\end{tabproof}


\section{Eigenschaften der Paarmenge}

\FormulaThmAuto{x \notin \{a,b\} \eqvdash x \neq a \land x \neq b}
\begin{tabproofwide}
  \proofstepwide{x \notin \{a,b\}}{\leftrightarrow}{\neg(x = a \lor x = b)}%
    {\FormulaRefAuto{\{A,B\} := \iota C\Bigl(\forall x\;\bigl(x \in C \;\leftrightarrow\; x = A \lor x = B\bigr)\Bigr)}}
  \proofstepwide{}{\leftrightarrow}{x \neq a \land x \neq b}%
    {\FormulaRefAuto{\neg(P \lor Q) \eqvdash \neg P \land \neg Q}{1}}
  \proofstepwide{x \notin \{a,b\}}{\leftrightarrow}{x \neq a \land x \neq b}%
    {\rChain{1,2}}
\end{tabproofwide}

\FormulaThmAuto{a \in \{a,b\}}
\begin{tabproof}
  \proofstep{}{a = a}{\rIE{}}
  \proofstep{}{a = a \lor a = b}{\rOIa{1}}
  \proofstep{}{a \in \{a,b\}}{\FormulaRefAuto{\{A,B\} := \iota C\Bigl(\forall x\,\bigl(x \in C \leftrightarrow x = A \lor x = B\bigr)\Bigr)}{2}}
\end{tabproof}

\FormulaThmAuto{b \in \{a,b\}}
\begin{tabproof}
  \proofstep{}{b = b}{\rIE{}}
  \proofstep{}{b = a \lor b = b}{\rOIb{1}}
  \proofstep{}{b \in \{a,b\}}{\FormulaRefAuto{\{A,B\} := \iota C\Bigl(\forall x\,\bigl(x \in C \leftrightarrow x = A \lor x = B\bigr)\Bigr)}{2}}
\end{tabproof}

\FormulaThmAuto{\{a,b\} = \{b,a\}}
\begin{tabproofwide}
  \proofstepwide{x \in \{a,b\}}{\leftrightarrow}{x = a \lor x = b}%
    {\FormulaRefAuto{\{A,B\} := \iota C\Bigl(\forall x\,\bigl(x \in C \leftrightarrow x = A \lor x = B\bigr)\Bigr)}{}}
  \proofstepwide{}{\leftrightarrow}{x = b \lor x = a}%
    {\FormulaRefAuto{P \lor Q \vdash Q \lor P}{1}}
  \proofstepwide{}{\leftrightarrow}{x \in \{b,a\}}%
    {\FormulaRefAuto{\{A,B\} := \iota C\Bigl(\forall x\,\bigl(x \in C \leftrightarrow x = A \lor x = B\bigr)\Bigr)}{2}}
  \proofstepwide{x \in \{a,b\}}{\leftrightarrow}{x \in \{b,a\}}%
    {\rChain{1,3}}
  \proofstepwide{\{a,b\}}{=}{\{b,a\}}%
    {\FormulaRefAuto{\forall x\, (x \in A \leftrightarrow x \in B) \eqvdash A = B}{\rUI{4}}}
\end{tabproofwide}

\FormulaThmAuto{\{a,b\} \neq \emptyset}
\begin{tabproof}
  \proofstep{}{a \in \{a,b\}}{\FormulaRefAuto{a \in \{a,b\}}{}}
  \proofstep{}{\{a,b\} \neq \emptyset}{\FormulaRefAuto{\exists x\,(x \in S) \vdash S \neq \emptyset}{1}}
\end{tabproof}

\FormulaDefAuto[Einermenge]{\{a\} := \{a,a\}}

\FormulaThmAuto{a \in \{a\}}
\begin{tabproof}
  \proofstep{}{a \in \{a,a\}}{\FormulaRefAuto{a \in \{a,b\}}{}}
  \proofstep{}{a \in \{a\}}{\FormulaRefAuto{\{a\} := \{a,a\}}{1}}
\end{tabproof}

\FormulaThmAuto{x \in \{a\} \eqvdash x = a}
\begin{tabproofwide}
  \proofstepwide{x \in \{a\}}{\leftrightarrow}{x \in \{a,a\}}%
    {\rIE{\FormulaRefAuto{\{a\} := \{a,a\}}{}, 1}}
  \proofstepwide{}{\leftrightarrow}{x = a \lor x = a}%
    {\rIE{\FormulaRefAuto{\{A,B\} := \iota C\Bigl(\forall x\,\bigl(x \in C \leftrightarrow x = A \lor x = B\bigr)\Bigr)}{}, 1}}
  \proofstepwide{}{\leftrightarrow}{x = a}%
    {\FormulaRefAuto{P \lor P \eqvdash P}{2}}
  \proofstepwide{x \in \{a\}}{\leftrightarrow}{x = a}%
    {\rChain{1,3}}
\end{tabproofwide}

\FormulaThmAuto{x \notin \{a\} \eqvdash x \neq a}
\begin{tabproof}
  \proofstep{}{x \notin \{a\} \leftrightarrow x \neq a}%
    {\FormulaRefAuto{P \leftrightarrow Q \dashv \vdash \neg P \leftrightarrow \neg Q}{\FormulaRefAuto{x \in \{a\} \eqvdash x = a}{}}}
\end{tabproof}

\FormulaThmAuto{\{a\}=\{b,c\}\vdash a=b\land a=c}
\begin{tabproof}
  \proofstep{1}{\{a\}=\{b,c\}}{\rA}
  \proofstep{1}{b\in\{a\}}{\rIE{1,\FormulaRefAuto{a\in\{a,b\}}}}
  \proofstep{1}{a=b}{\FormulaRefAuto{a = b \vdash b = a}{\FormulaRefAuto{x \in \{a\} \eqvdash x = a}{2}}}
  \proofstep{1}{c\in\{a\}}{\rIE{1,\FormulaRefAuto{b\in\{a,b\}}}}
  \proofstep{1}{a=c}{\FormulaRefAuto{a = b \vdash b = a}{\FormulaRefAuto{x \in \{a\} \eqvdash x = a}{2}}}
  \proofstep{1}{a=b\land a=c}{\rAI{3,5}}
\end{tabproof}

\FormulaThmAuto{\exists x\in \{a,b\} P(x)\vdash P(a)\lor P(b)}
\begin{tabproof}
  \proofstep{1}{\exists x\in \{a,b\} P(x)}{\rA}
  \proofstep{2}{x\in \{a,b\}\land P(x)}{\rA}
  \proofstep{2}{x\in \{a,b\}}{\rAEa{2}}
  \proofstep{2}{P(x)}{\rAEb{2}}
  \proofstep{2}{x=a\lor x=b}{\FormulaRefAuto{x \in \{A,B\}\;\eqvdash\;(x = A \lor x = B)}{3}}
  \proofstep{6}{x=a}{\rA}
  \proofstep{2,6}{P(a)}{\rIE{6,4}}
  \proofstep{2,6}{P(a)\lor P(b)}{\rOIa{7}}
  \proofstep{9}{x=b}{\rA}
  \proofstep{2,9}{P(b)}{\rIE{9,4}}
  \proofstep{2,9}{P(a)\lor P(b)}{\rOIb{10}}
  \proofstep{2}{P(a)\lor P(b)}{\rOE{5,6,8,9,11}}
  \proofstep{1}{P(a)\lor P(b)}{\rEE{1,2,12}}
\end{tabproof}

\FormulaThmAuto{a \in A \vdash \{a\} \subseteq A}
\begin{tabproof}
  \proofstep{1}{x \in \{a\}}{\rA}
  \proofstep{2}{a \in A}{\rA}
  \proofstep{1}{x = a}{\FormulaRefAuto{x \in \{a\} \eqvdash x = a}{1}}
  \proofstep{1,2}{x \in A}{\rIE{3,2}}
  \proofstep{2}{\{a\} \subseteq A}{\FormulaRefAuto{A \subseteq B := \forall x\,(x \in A \rightarrow x \in B)}{\rUI{\rRI{1,4}}}}
\end{tabproof}

\FormulaThmAuto{a\in A\vdash A\cap \{A,a\}\neq\emptyset}
\begin{tabproof}
  \proofstep{1}{a \in A}{\rA}
  \proofstep{}{a \in \{A,a\}}{\FormulaRefAuto{b \in \{a,b\}}}
  \proofstep{1}{a \in A\cap \{A,a\}}{\FormulaRefAuto{x \in A, x\in B \vdash x \in A\cap B}{1,2}}
  \proofstep{1}{A\cap \{A,a\}\neq\emptyset}{\FormulaRefAuto{a \in S \vdash S \neq \emptyset}{3}}
\end{tabproof}

\FormulaThmAuto{a\in A\vdash A\cap \{a,A\}\neq\emptyset}
\begin{tabproof}
  \proofstep{1}{a \in A}{\rA}
  \proofstep{}{a \in \{a,A\}}{\FormulaRefAuto{a \in \{a,b\}}}
  \proofstep{1}{a \in A\cap \{a,A\}}{\FormulaRefAuto{x \in A, x\in B \vdash x \in A\cap B}{1,2}}
  \proofstep{1}{A\cap \{a,A\}\neq\emptyset}{\FormulaRefAuto{a \in S \vdash S \neq \emptyset}{3}}
\end{tabproof}

\FormulaThmAuto{\{a\}=\{b\}\eqvdash a=b}
\begin{tabproofsplit}
    \proofpart{\(\vdash\)}
    \proofstep{1}{\{a\}=\{b\}}{ \rA}
    \proofstep{1}{\forall x(x\in\{a\}\leftrightarrow x\in\{b\})}{\FormulaRefAuto{\forall x\, (x \in A \leftrightarrow x \in B) \eqvdash A = B}}
    \proofstep{}{a\in\{a\}}{\FormulaRefAuto{a \in \{a\}}}
    \proofstep{1}{a\in\{b\}}{\FormulaRefAuto{P\leftrightarrow Q, P\vdash Q}{\rUE{2},3}}
    \proofstep{1}{a=b}{\FormulaRefAuto{x \in \{a\} \eqvdash x = a}{4}}
    \closeproofpart
    \proofpart{\(\dashv\)}
    \proofstep{1}{a=b}{ \rA}
    \proofstep{}{\{a\}=\{a\}}{ \rII}
    \proofstep{1}{\{a\}=\{b\}}{ \rIE{1,2}}
    \closeproofpart
\end{tabproofsplit}

\FormulaThmAuto{a=c,b=d\vdash \{a,b\}\subseteq \{c,d\}}
\begin{tabproof}
  \proofstep{1}{a=c}{\rA}
  \proofstep{2}{b=d}{\rA}
  \proofstep{3}{x\in\{a,b\}}{\rA}
  \proofstep{3}{x=a\lor x=b}{\FormulaRefAuto{x \in \{A,B\}\;\eqvdash\;(x = A \lor x = B)}{3}}

  \proofcase[1]{x=a \vdash x\in\{c,d\}}
  \proofstep{5}{x=a}{\rA}
  \proofstep{1,5}{x=c}{\FormulaRefAuto{a = b,\, b = c \vdash a = c}{5,1}}
  \proofstep{1,5}{x=c\lor x=d}{\rOIa{6}}
  \proofstep{1,5}{x\in \{c,d\}}{\FormulaRefAuto{x \in \{A,B\}\;\eqvdash\;(x = A \lor x = B)}{7}}

  \proofcase[2]{x=b \vdash x\in\{c,d\}}
  \proofstep{9}{x=b}{\rA}
  \proofstep{2,9}{x=d}{\FormulaRefAuto{a = b,\, b = c \vdash a = c}{9,2}}
  \proofstep{2,9}{x=c\lor x=d}{\rOIb{10}}
  \proofstep{2,9}{x\in \{c,d\}}{\FormulaRefAuto{x \in \{A,B\}\;\eqvdash\;(x = A \lor x = B)}{11}}
  
  \proofcasesummary[1]{\{a,b\}\subseteq\{c,d\}}
  \proofstep{1,2,3}{x\in \{c,d\}}{\rOE{4,5,8,9,12}}
  \proofstep{1,2}{\{a,b\}\subseteq\{c,d\}}%
    {\FormulaRefAuto{ A \subseteq B := \forall x\,(x\in A \rightarrow x\in B)}{\rUI{\rRI{3,13}}}}
\end{tabproof}

\FormulaThmAuto{a=d,b=c\vdash \{a,b\}\subseteq \{c,d\}}
\begin{tabproof}
  \proofstep{1}{a=d}{\rA}
  \proofstep{2}{b=c}{\rA}
  \proofstep{3}{x\in\{a,b\}}{\rA}
  \proofstep{3}{x=a\lor x=b}{\FormulaRefAuto{x \in \{A,B\}\;\eqvdash\;(x = A \lor x = B)}{3}}

  \proofcase[1]{x=a \vdash x\in\{c,d\}}
  \proofstep{5}{x=a}{\rA}
  \proofstep{1,5}{x=d}{\FormulaRefAuto{a = b,\, b = c \vdash a = c}{5,1}}
  \proofstep{1,5}{x=c\lor x=d}{\rOIb{6}}
  \proofstep{1,5}{x\in \{c,d\}}{\FormulaRefAuto{x \in \{A,B\}\;\eqvdash\;(x = A \lor x = B)}{7}}

  \proofcase[2]{x=b \vdash x\in\{c,d\}}
  \proofstep{9}{x=b}{\rA}
  \proofstep{2,9}{x=c}{\FormulaRefAuto{a = b,\, b = c \vdash a = c}{9,2}}
  \proofstep{2,9}{x=c\lor x=d}{\rOIa{10}}
  \proofstep{2,9}{x\in \{c,d\}}{\FormulaRefAuto{x \in \{A,B\}\;\eqvdash\;(x = A \lor x = B)}{11}}
  
  \proofcasesummary[1]{\{a,b\}\subseteq\{c,d\}}
  \proofstep{1,2,3}{x\in \{c,d\}}{\rOE{4,5,8,9,12}}
  \proofstep{1,2}{\{a,b\}\subseteq\{c,d\}}%
    {\FormulaRefAuto{ A \subseteq B := \forall x\,(x\in A \rightarrow x\in B)}{\rUI{\rRI{3,13}}}}
\end{tabproof}

\FormulaThmAuto{a=c,b=d\vdash \{a,b\}=\{c,d\}}
% --- Beweis ---
\begin{tabproof}
  \proofstep{1}{a=c}{\rA}
  \proofstep{2}{b=d}{\rA}
  \proofstep{2}{c=a}{\FormulaRefAuto{a=b\vdash b=a}{1}}
  \proofstep{2}{b=d}{\FormulaRefAuto{a=b\vdash b=a}{2}}

  \proofstep{1,2}{\{a,b\}\subseteq\{c,d\}}{\FormulaRefAuto{a=c,b=d\vdash \{a,b\}\subseteq \{c,d\}}{1,2}}
  \proofstep{1,2}{\{c,d\}\subseteq\{a,b\}}{\FormulaRefAuto{a=c,b=d\vdash \{a,b\}\subseteq \{c,d\}}{3,4}}

  \proofstep{1,2}{\{a,b\}=\{c,d\}}{\FormulaRefAuto{ A \subseteq B, B \subseteq A \vdash A = B }{5,6}}
\end{tabproof}

\FormulaThmAuto{a=c\land b=d\vdash \{a,b\}=\{c,d\}}
\begin{tabproof}
    \proofstep{1}{a=c\land b=d}{\rA}
    \proofstep{1}{a=c}{\rAEa{1}}
    \proofstep{1}{b=d}{\rAEb{1}}
    \proofstep{1}{\{a,b\}=\{c,d\}}{\FormulaRefAuto{a=c,b=d\vdash \{a,b\}=\{c,d\}}{2,3}}
\end{tabproof}

\FormulaThmAuto{a=d, b=c\vdash \{a,b\}=\{c,d\}}
% --- Beweis ---
\begin{tabproof}
  \proofstep{1}{a=d}{\rA}
  \proofstep{2}{b=c}{\rA}
  \proofstep{3}{d=a}{\FormulaRefAuto{a=b\vdash b=a}{1}}   % Symmetrie aus 1
  \proofstep{4}{c=b}{\FormulaRefAuto{a=b\vdash b=a}{2}}   % Symmetrie aus 2

  % Erstes Inklusionsziel: {a,b} ⊆ {c,d}
  \proofstep{1,2}{\{a,b\}\subseteq\{c,d\}}{%
    \FormulaRefAuto{a=d,b=c\vdash \{a,b\}\subseteq \{c,d\}}{1,2}}

  % Zweites Inklusionsziel: {c,d} ⊆ {a,b}
  % Anwenden des selben Teilmengen-Theorems mit Umbenennung (a',b',c',d')=(c,d,a,b)
  % Voraussetzungen dafür sind c=b (Zeile 4) und d=a (Zeile 3)
  \proofstep{4,3}{\{c,d\}\subseteq\{a,b\}}{%
    \FormulaRefAuto{a=d,b=c\vdash \{a,b\}\subseteq \{c,d\}}{4,3}}

  % Gleichheit aus beidseitiger Inklusion
  \proofstep{1,2}{\{a,b\}=\{c,d\}}{%
    \FormulaRefAuto{ A \subseteq B, B \subseteq A \vdash A = B }{5,6}}
\end{tabproof}

\FormulaThmAuto{a=d\land b=c\vdash \{a,b\}=\{c,d\}}
\begin{tabproof}
    \proofstep{1}{a=d\land b=c}{\rA}
    \proofstep{1}{a=d}{\rAEa{1}}
    \proofstep{1}{b=c}{\rAEb{1}}
    \proofstep{1}{\{a,b\}=\{c,d\}}{\FormulaRefAuto{a=d, b=c\vdash \{a,b\}=\{c,d\}}{2,3}}
\end{tabproof}

\FormulaThmAuto{\{a,b\}=\{c,d\}\vdash (a=c\lor a=d)\land (b=c\lor b=d)}
\begin{tabproof}
    \proofstep{1}{\{a,b\}=\{c,d\}}{ \rA}
    \proofstep{}{a\in \{a,b\}}{\FormulaRefAuto{a\in \{a,b\}}}
    \proofstep{1}{a\in \{c,d\}}{\FormulaRefAuto{A=B, x\in A\vdash x\in B}{1,2}}
    \proofstep{1}{a=c\lor a=d}{\FormulaRefAuto{x \in \{A,B\}\;\eqvdash\;(x = A \lor x = B)}{3}}
    \proofstep{}{b\in \{a,b\}}{\FormulaRefAuto{b\in \{a,b\}}}
    \proofstep{1}{b\in \{c,d\}}{\FormulaRefAuto{A=B, x\in A\vdash x\in B}{1,5}}
    \proofstep{1}{b=c\lor b=d}{\FormulaRefAuto{x \in \{A,B\}\;\eqvdash\;(x = A \lor x = B)}{3}}
    \proofstep{1}{(a=c\lor a=d)\land (b=c\lor b=d)}{\rAI{4,7}}
\end{tabproof}

\FormulaThmAuto{\{a,b\}=\{c,d\},a=c,b=c\vdash d=c}
\begin{tabproof}
\proofstep{1}{\{a,b\}=\{c,d\}}{\rA}
\proofstep{2}{a=c}{\rA}
\proofstep{3}{b=c}{\rA}
\proofstep{}{d\in\{c,d\}}{\FormulaRefAuto{b\in\{a,b\}}}
\proofstep{1}{d\in\{a,b\}}{\rIE{1,4}}
\proofstep{1}{d=a\lor d=b}{\FormulaRefAuto{x \in \{A,B\}\;\eqvdash\;(x = A \lor x = B)}{5}}
\proofstep{1,2,3}{d=c}{\FormulaRefAuto{a = c,\, b = c, d=a\lor d=b \vdash d = c}{2,3,5}}
\end{tabproof}

\FormulaThmAuto{\{a,b\}=\{c,d\},a=d,b=d\vdash c=d}
\begin{tabproof}
\proofstep{1}{\{a,b\}=\{c,d\}}{\rA}
\proofstep{2}{a=d}{\rA}
\proofstep{3}{b=d}{\rA}
\proofstep{}{c\in\{c,d\}}{\FormulaRefAuto{a\in\{a,b\}}}
\proofstep{1}{c\in\{a,b\}}{\rIE{1,4}}
\proofstep{1}{c=a\lor c=b}{\FormulaRefAuto{x \in \{A,B\}\;\eqvdash\;(x = A \lor x = B)}{5}}
\proofstep{1,2,3}{c=d}{\FormulaRefAuto{a = c,\, b = c, d=a\lor d=b \vdash d = c}{2,3,5}}
\end{tabproof}

\FormulaThmAuto{\{a,b\}=\{c,d\} \;\eqvdash\; \bigl((a=c\land b=d)\;\lor\;(a=d\land b=c)\bigr)}
\begin{tabproofsplit}
  %%%%%%%%%%%%%%%%%%%%%%%%%%%%%
  \proofpart{\(\vdash\)}
  \proofstep{1}{\{a,b\}=\{c,d\}}{\rA}
  \proofstep{1}{(a=c\lor a=d)\land (b=c\lor b=d)}%
    {\FormulaRefAuto{\{a,b\}=\{c,d\}\vdash (a=c\lor a=d)\land (b=c\lor b=d)}{1}}
    \proofstep{1}{(a=c\land b=c)\lor (a=c\land b=d)\lor}%
  {\multirow{2}{*}{$\FormulaRefAuto{(P \lor Q) \land (R \lor S) \dashv \vdash
   (P \land R) \lor (P \land S) \lor (Q \land R) \lor (Q \land S)}{2}$}}
\proofstepstar{}{(a=d\land b=c)\lor (a=d\land b=d)}{}
  \proofstep{4}{a=c\land b=c}{\rA}
  \proofstep{4}{a=c}{\rAEa{4}}
  \proofstep{4}{b=c}{\rAEb{4}}
  \proofstep{1,4}{d=c}{\FormulaRefAuto{\{a,b\}=\{c,d\},a=c,b=c\vdash d=c}{1,5,6}}
  \proofstep{1,4}{b=d}{\FormulaRefAuto{a = b, c = b\vdash a = c}{5,6}}
  \proofstep{1,4}{(a=c\land b=d) \lor (a=d\land b=c)}{\rOIa{\rAI{5,8}}}
  \proofstep{10}{(a=c\land b=d) \lor (a=d\land b=c)}{\rA}
  \proofstep{11}{a=d\land b=d}{\rA}
  \proofstep{11}{a=d}{\rAEa{12}}
  \proofstep{11}{b=d}{\rAEb{12}}
  \proofstep{1,11}{c=d}{\FormulaRefAuto{\{a,b\}=\{c,d\},a=d,b=d\vdash c=d}{1,12,13}}
  \proofstep{1,11}{b=c}{\FormulaRefAuto{a = b, c = b\vdash a = c}{14,16}}
  \proofstep{1,11}{(a=c\land b=d) \lor (a=d\land b=c)}{\rOIb{\rAI{12,15}}}
  \proofstep{1}{(a=c\land b=d)\lor (a=d\land b=c)}{\rOEn{3,4,9,10,10,11,16}}
  %%%%%%%%%%%%%%%%%%%%%%%%%%%%%
  \closeproofpart
  \proofpart{\(\dashv\)}
  \proofstep{1}{(a=c\land b=d)\lor (a=d\land b=c)}{\rA}
  \proofstep{2}{a=c\land b=d}{\rA}
  \proofstep{2}{a=c}{\rAEa{2}}
  \proofstep{2}{b=d}{\rAEb{2}}
  \proofstep{2}{\{a,b\}=\{c,d\}}{\FormulaRefAuto{a=c,b=d\vdash \{a,b\}=\{c,d\}}{3,4}}
  \proofstep{6}{a=d\land b=d}{\rA}
  \proofstep{6}{a=d}{\rAEa{2}}
  \proofstep{6}{b=c}{\rAEb{2}}
  \proofstep{6}{\{a,b\}=\{d,c\}}{\FormulaRefAuto{a=c,b=d\vdash \{a,b\}=\{c,d\}}{7,8}}
  \proofstep{6}{\{a,b\}=\{c,d\}}{\rIE{\FormulaRefAuto{\{a,b\} = \{b,a\}},9}}
  \proofstep{1}{\{a,b\}=\{c,d\}}{\rOE{1,2,5,6,10}}
  \closeproofpart
\end{tabproofsplit}

\FormulaThmAuto{\{a,b\}=\{c,d\}, a=c \vdash b=d}
\begin{tabproof}
\proofstep{1}{\{a,b\}=\{c,d\}}{\rA}
\proofstep{2}{a=c}{\rA}
\proofstep{1}{(a=c\land b=d)\;\lor\;(a=d\land b=c)}{\FormulaRefAuto{\{a,b\}=\{c,d\} \;\eqvdash\; \bigl((a=c\land b=d)\;\lor\;(a=d\land b=c)\bigr)}{1}}
\proofstep{4}{a=c\land b=d}{\rA}
\proofstep{4}{b=d}{\rAEb{4}}
\proofstep{6}{a=d\land b=c}{\rA}
\proofstep{6}{a=d}{\rAEa{6}}
\proofstep{6}{b=c}{\rAEb{6}}
\proofstep{2}{c=a}{\FormulaRefAuto{a=b \vdash b=a}{2}}
\proofstep{9,7}{c=d}{\FormulaRefAuto{a=b,\, b=c \vdash a=c}{9,7}}
\proofstep{8,10}{b=d}{\FormulaRefAuto{a=b,\, b=c \vdash a=c}{8,10}}
% Disjunktionselimination
\proofstep{1,2}{b=d}{\rOE{3,4,5,6,11}}
\end{tabproof}


\section{Eigenschaften geordneter Paare}

\FormulaThmAuto{(a,b) = (c,d)\eqvdash a=c\land b=d}
\begin{tabproofsplit}
    \proofpart{\(\vdash\)}
    \proofstep{1}{(a,b) = (c,d)}{\rA}
    \proofstep{1}{\{\{a\},\{a,b\}\}=\{\{c\},\{c,d\}\}}{\rIE{\FormulaRefAuto{\forall a,b((a, b) := \{ \{ a \}, \{ a, b \}) \}},1}}
    \proofstep{1}{\{a\}=\{c\}\land \{a,b\}=\{c,d\}}{\multirow{2}{*}{$\FormulaRefAuto{\{a,b\}=\{c,d\} \;\eqvdash\; \bigl((a=c\land b=d)\;\lor\;(a=d\land b=c)\bigr)}{2}$}}
    \proofstepstar{}{\lor \{a\}=\{c,d\}\land \{a,b\}=\{c\}}{}
    \proofstep{4}{\{a\}=\{c\}\land \{a,b\}=\{c,d\}}{\rA}
    \proofstep{4}{\{a\}=\{c\}}{\rAEa{4}}
    \proofstep{4}{\{a,b\}=\{c,d\}}{\rAEb{4}}
    \proofstep{4}{a=c}{\FormulaRefAuto{\{a\}=\{b\}\eqvdash a=b}{5}}
    \proofstep{4}{b=d}{\FormulaRefAuto{\{a,b\}=\{c,d\}, a=c \vdash b=d}{6,7}}
    \proofstep{4}{a=c\land b=d}{\rAI{7,8}}
    \proofstep{10}{\{a\}=\{c,d\}\land \{a,b\}=\{c\}}{\rA}
    \proofstep{10}{\{a\}=\{c,d\}}{\rAEa{13}}
    \proofstep{10}{\{a,b\}=\{c\}}{\rAEb{13}}
    \proofstep{10}{a=c\land a=d}{\FormulaRefAuto{\{a\}=\{b,c\}\vdash a=b\land a=c}{11}}
    \proofstep{10}{c=a\land c=b}{\FormulaRefAuto{\{a\}=\{b,c\}\vdash a=b\land a=c}{\FormulaRefAuto{a = b \vdash b = a}{12}}}
    \proofstep{10}{a=c}{\rAEa{13}}
    \proofstep{10}{a=d}{\rAEb{13}}
    \proofstep{10}{c=b}{\rAEb{14}}
    \proofstep{10}{b=d}{\FormulaRefAuto{a = b,\, a = c \vdash b = c}{17,\FormulaRefAuto{a = b,\, a = c \vdash b = c}{15,16}}}
    \proofstep{10}{a=c\land b=d}{\rAI{15,18}}
    \proofstep{1}{a=c\land b=d}{\rOE{3,4,9,10,19}}
    \closeproofpart
    \proofpart{\(\dashv\)}
    \proofstep{1}{a=c\land b=d}{\rA}
    \proofstep{1}{\{a,b\}=\{c,d\}}{\FormulaRefAuto{a=c\land b=d\vdash \{a,b\}=\{c,d\}}{1}}
    \proofstep{1}{a=c}{\rAEa{1}}
    \proofstep{1}{{a}={c}}{\FormulaRefAuto{\{a\}=\{b\}\eqvdash a=b}{3}}
    \proofstep{1}{\{\{a\},\{a,b\}\}=\{\{c\},\{c,d\}\}}{\FormulaRefAuto{a=c, b=d\vdash \{a,b\}=\{c,d\}}{4,2}}
    \closeproofpart
\end{tabproofsplit}

\section{Definition und Eigenschaften der Differenz}

\FormulaDefAuto{A \setminus B := \{ x \in A \mid x \notin B \}}

\FormulaThmAuto{x \in A \setminus B \leftrightarrow x \in A \land x \notin B}
\begin{tabproofwide}
  \proofstepwide{x \in A \setminus B}{\leftrightarrow}{x \in \{x \in A \mid x \notin B\}}%
    {\rUE{\FormulaRefAuto{\forall x\, (x \in A \leftrightarrow x \in B) \eqvdash A = B}{\FormulaRefAuto{A \setminus B := \{ x \in A \mid x \notin B \}}{}}}}
  \proofstepwide{}{\leftrightarrow}{x \in A \land x \notin B}%
    {\FormulaRefAuto{\{x \in A \mid P(x)\} := \iota B\bigl(\forall x\,(x \in B \leftrightarrow (x \in A \land P(x)))\bigr)}{1}}
\end{tabproofwide}

\FormulaThmAuto{c \in A \setminus \{a\} \vdash c \neq a}
\begin{tabproof}
  \proofstep{1}{c \in A \setminus \{a\}}{\rA}
  \proofstep{1}{c \notin \{a\}}{\rAEb{\FormulaRefAuto{x \in A \setminus B \leftrightarrow x \in A \land x \notin B}{1}}}
  \proofstep{1}{c \neq a}{\FormulaRefAuto{x \notin \{a\} \eqvdash x \neq a}{2}}
\end{tabproof}

\FormulaThmAuto{c \in A \setminus \{a,b\} \vdash c \neq a}
\begin{tabproof}
  \proofstep{1}{c \in A \setminus \{a,b\}}{\rA}
  \proofstep{1}{c \notin \{a,b\}}{\rAEb{\FormulaRefAuto{x \in A \setminus B \leftrightarrow x \in A \land x \notin B}{1}}}
  \proofstep{1}{c \neq a}{\rAEa{\FormulaRefAuto{x \notin \{a,b\} \eqvdash x \neq a \land x \neq b}{2}}}
\end{tabproof}

\FormulaThmAuto{c \in A \setminus \{a,b\} \vdash c \neq b}
\begin{tabproof}
  \proofstep{1}{c \in A \setminus \{a,b\}}{\rA}
  \proofstep{1}{c \notin \{a,b\}}{\rAEb{\FormulaRefAuto{x \in A \setminus B \leftrightarrow x \in A \land x \notin B}{1}}}
  \proofstep{1}{c \neq b}{\rAEb{\FormulaRefAuto{x \notin \{a,b\} \eqvdash x \neq a \land x \neq b}{2}}}
\end{tabproof}

\FormulaThmAuto{c \in A \setminus B,\, b \in B \vdash c \neq b}
\begin{tabproof}
  \proofstep{1}{c \in A \setminus B}{\rA}
  \proofstep{2}{b \in B}{\rA}
  \proofstep{1}{c \notin B}{\rAEb{\FormulaRefAuto{x \in A \setminus B \leftrightarrow x \in A \land x \notin B}{1}}}
  \proofstep{1,2}{c \neq b}{\FormulaRefAuto{ a \in A,\; b \not\in A \vdash a \neq b }{2,3}}
\end{tabproof}

\FormulaThmAuto{a \notin A \setminus \{a\}}
\begin{tabproof}
  \proofstep{1}{a \in A \setminus \{a\}}{\rA}
  \proofstep{}{a = a}{\rII}
  \proofstep{1}{a \neq a}{\FormulaRefAuto{c \in A \setminus \{a\} \vdash c \neq a}{1}}
  \proofstep{1}{\bot}{\rBI{2,3}}
  \proofstep{}{a \notin A \setminus \{a\}}{\rCI{1,4}}
\end{tabproof}

\FormulaThmAuto{a \notin A \setminus \{a,b\}}
\begin{tabproof}
  \proofstep{1}{a \in A \setminus \{a,b\}}{\rA}
  \proofstep{}{a = a}{\rII}
  \proofstep{1}{a \neq a}{\FormulaRefAuto{c \in A \setminus \{a,b\} \vdash c \neq a}{1}}
  \proofstep{1}{\bot}{\rBI{2,3}}
  \proofstep{}{a \notin A \setminus \{a,b\}}{\rCI{1,4}}
\end{tabproof}

\FormulaThmAuto{b \notin A \setminus \{a,b\}}
\begin{tabproof}
  \proofstep{1}{b \in A \setminus \{a,b\}}{\rA}
  \proofstep{}{b = b}{\rII}
  \proofstep{1}{b \neq b}{\FormulaRefAuto{c \in A \setminus \{a,b\} \vdash c \neq b}{1}}
  \proofstep{1}{\bot}{\rBI{2,3}}
  \proofstep{}{b \notin A \setminus \{a,b\}}{\rCI{1,4}}
\end{tabproof}

\FormulaThmAuto{a \in A,\, a \neq b \vdash a \in A \setminus \{b\}}
\begin{tabproof}
  \proofstep{1}{a \in A}{\rA}
  \proofstep{2}{a \neq b}{\rA}
  \proofstep{2}{a \notin \{b\}}{\FormulaRefAuto{x \notin \{a\} \eqvdash x \neq a}{2}}
  \proofstep{1,2}{a \in A \setminus \{b\}}{\FormulaRefAuto{x \in A \setminus B \leftrightarrow x \in A \land x \notin B}{\rAI{1,3}}}
\end{tabproof}

\FormulaThmAuto{a \in A,\, b \neq a \vdash a \in A \setminus \{b\}}
\begin{tabproof}
  \proofstep{1}{a \in A}{\rA}
  \proofstep{2}{b \neq a}{\rA}
  \proofstep{2}{a \neq b}{\FormulaRefAuto{a \neq b \vdash b \neq a}{2}}
  \proofstep{1,2}{a \in A \setminus \{b\}}{\FormulaRefAuto{a \in A,\, a \neq b \vdash a \in A \setminus \{b\}}{1,3}}
\end{tabproof}


\section{Definition und Eigenschaften der Vereinigung}

\FormulaDefAuto[Vereinigung zweier Mengen]{A \cup B := \bigcup \{A, B\}}

\FormulaThmAuto{x\in A \cup B \eqvdash x\in \bigcup \{A, B\}}
\begin{tabproof}
\proofstep{}{z \in A \cup B\leftrightarrow z \in \bigcup \{A, B\}}{\rUE{\FormulaRefAuto{\forall x\, (x \in A \leftrightarrow x \in B) \eqvdash A = B}{\FormulaRefAuto{A \cup B := \bigcup \{A, B\}}}}}
\end{tabproof}

\FormulaThmAuto{A \subseteq B \vdash \bigcup A \subseteq \bigcup B}
\begin{tabproof}
  \proofstep{1}{A \subseteq B}{\rA}
  \proofstep{2}{x \in \bigcup A}{\rA}
  \proofstep{2}{\exists X\,(X \in A \land x \in X)}{\FormulaRefAuto{\bigcup A := \iota C\bigl(\forall x\,(x \in C \leftrightarrow \exists B\,(B \in A \land x \in B))\bigr)}{2}}
  \proofstep{4}{C \in A \land x \in C}{\rA}
  \proofstep{4}{C \in A}{\rAEa{4}}
  \proofstep{4}{x \in C}{\rAEb{4}}
  \proofstep{1,4}{C \in B}{\FormulaRefAuto{A \subseteq B,\, x \in A \vdash x \in B}{5,1}}
  \proofstep{1,4}{x \in \bigcup B}{\FormulaRefAuto{\bigcup A := \iota C\bigl(\forall x\,(x \in C \leftrightarrow \exists B\,(B \in A \land x \in B))\bigr)}{\rEI{\rAI{7,6}}}}
  \proofstep{1,2}{x \in \bigcup B}{\rEE{3,4,8}}
  \proofstep{1}{\bigcup A \subseteq \bigcup B}{\FormulaRefAuto{A \subseteq B := \forall x\,(x \in A \rightarrow x \in B)}{\rUI{\rRE{2,9}}}}
\end{tabproof}

\FormulaThmAuto{z \in A \cup B \eqvdash z \in A \lor z \in B}
\begin{tabproofwide}
  \proofstepwide{z \in A \cup B}{\leftrightarrow}{z \in \bigcup \{A, B\}}%
    {\FormulaRefAuto{x\in A \cup B \eqvdash x\in \bigcup \{A, B\}}}
  \proofstepwide{}{\leftrightarrow}{\exists C\,(C \in \{A, B\} \land z \in C)}%
    {\FormulaRefAuto{\bigcup A := \iota C\bigl(\forall x\,(x \in C \leftrightarrow \exists B\,(B \in A \land x \in B))\bigr)}{1}}
  \proofstepwide{}{\leftrightarrow}{\exists C\,((C = A \lor C = B) \land z \in C)}%
    {\rLRS{\FormulaRefAuto{\{A,B\} := \iota C\bigl(\forall x\,(x \in C \leftrightarrow x = A \lor x = B)\bigr)}{}}}
  \proofstepwide{}{\leftrightarrow}{z \in A \lor z \in B}%
    {\FormulaRefAuto{\exists c\,((c = a \lor c = b) \land P(c)) \dashv \vdash P(a) \lor P(b)}{3}}
\end{tabproofwide}

\FormulaThmAuto{z \in A \vdash z \in A \cup B}
\begin{tabproof}
  \proofstep{1}{z \in A}{\rA}
  \proofstep{1}{z \in A \lor z \in B}{\rOIa{1}}
  \proofstep{1}{z \in A \cup B}{\FormulaRefAuto{z \in A \cup B \eqvdash z \in A \lor z \in B}{2}}
\end{tabproof}

\FormulaThmAuto{z \in B \vdash z \in A \cup B}
\begin{tabproof}
  \proofstep{1}{z \in B}{\rA}
  \proofstep{1}{z \in A \lor z \in B}{\rOIb{1}}
  \proofstep{1}{z \in A \cup B}{\FormulaRefAuto{z \in A \cup B \eqvdash z \in A \lor z \in B}{2}}
\end{tabproof}

\FormulaThmAuto{x \in \{a,b\} \eqvdash x \in \{a\} \cup \{b\}}
\begin{tabproofwide}
  \proofstepwide{x \in \{a\}}{\leftrightarrow}{x = a}%
    {\FormulaRefAuto{x \in \{a\} \eqvdash x = a}}
  \proofstepwide{x \in \{b\}}{\leftrightarrow}{x = b}%
    {\FormulaRefAuto{x \in \{a\} \eqvdash x = a}}
  \proofstepwide{B \in \{\{a\},\{b\}\}}{\leftrightarrow}{B = \{a\} \lor B = \{b\}}%
    {\FormulaRefAuto{\{A,B\} := \iota C\bigl(\forall x\,(x \in C \leftrightarrow x = A \lor x = B)\bigr)}}
  \proofstepwide{x \in \{a,b\}}{\leftrightarrow}{x = a \lor x = b}%
    {\FormulaRefAuto{\{A,B\} := \iota C\bigl(\forall x\,(x \in C \leftrightarrow x = A \lor x = B)\bigr)}}
  \proofstepwide{}{\leftrightarrow}{x \in \{a\} \lor x = b}{\rLRS{1}}
  \proofstepwide{}{\leftrightarrow}{x \in \{a\} \lor x \in \{b\}}{\rLRS{2}}

  % --- Zeile 7: auf zwei Zeilen gesplittet ---
  \proofstepwide{}{\leftrightarrow}{\exists B\,\bigl(B = \{a\} \lor B = \{b\}\bigr)}%
    {\multirow{2}{*}{\FormulaRefAuto{\exists c\,((c = a \lor c = b) \land P(c)) \dashv \vdash P(a) \lor P(b)}{6}}}
  \proofstepwide*{}{\land}{x \in B}{}
  % -------------------------------------------

  \proofstepwide{}{\leftrightarrow}{\exists B\,(B \in \{\{a\},\{b\}\} \land x \in B)}{\rLRS{3}}
  \proofstepwide{}{\leftrightarrow}{x \in \bigcup \{\{a\},\{b\}\}}%
    {\FormulaRefAuto{\bigcup A := \iota C\bigl(\forall x\,(x \in C \leftrightarrow \exists B\,(B \in A \land x \in B))\bigr)}{8}}
  \proofstepwide{}{\leftrightarrow}{x \in \{a\} \cup \{b\}}%
    {\FormulaRefAuto{A \cup B := \bigcup \{A, B\}}{8}}
  \proofstepwide{x \in \{a,b\}}{\leftrightarrow}{x \in \{a\} \cup \{b\}}{\rChain{4,10}}
\end{tabproofwide}

\FormulaThmAuto{\{a,b\} = \{a\} \cup \{b\}}
\begin{tabproof}
  \proofstep{}{x \in \{a,b\} \leftrightarrow x \in \{a\} \cup \{b\}}{\FormulaRefAuto{x \in \{a,b\} \eqvdash x \in \{a\} \cup \{b\}}}
  \proofstep{}{\{a,b\} = \{a\} \cup \{b\}}{\FormulaRefAuto{\forall x\,(x \in A \leftrightarrow x \in B) \eqvdash A = B}{\rUI{1}}}
\end{tabproof}

\FormulaThmAuto{a \in A \cup \{a\}}
\begin{tabproof}
  \proofstep{}{a \in \{a\}}{\FormulaRefAuto{a \in \{a\}}}
  \proofstep{}{a \in A \cup \{a\}}{\FormulaRefAuto{z \in B \vdash z \in A \cup B}{1}}
\end{tabproof}

\FormulaThmAuto{a \in \{a\} \cup A}
\begin{tabproof}
  \proofstep{}{a \in \{a\}}{\FormulaRefAuto{a \in \{a\}}}
  \proofstep{}{a \in \{a\} \cup A}{\FormulaRefAuto{z \in A \vdash z \in A \cup B}{1}}
\end{tabproof}

\FormulaThmAuto{\{a\} \cup A\neq\emptyset}
\begin{tabproof}
  \proofstep{}{a \in \{a\} \cup A}{\FormulaRefAuto{a \in \{a\} \cup A}}
  \proofstep{}{\{a\} \cup A\neq\emptyset}{\FormulaRefAuto{a\in S\vdash S\neq\emptyset}}
\end{tabproof}

\FormulaThmAuto{A\cup \{a\}\neq\emptyset}
\begin{tabproof}
  \proofstep{}{a \in A\cup \{a\}}{\FormulaRefAuto{a \in A\cup \{a\}}}
  \proofstep{}{\{a\} \cup A\neq\emptyset}{\FormulaRefAuto{a\in S\vdash S\neq\emptyset}}
\end{tabproof}


\FormulaThmAuto{A\cup \{a\}=A\eqvdash a\in A}
\begin{tabproofsplit}
  \proofpart{$\vdash$}
    \proofstep{1}{A \cup \{a\} = A}{\rA}
    \proofstep{}{a \in A \cup \{a\}}{\FormulaRefAuto{a \in A \cup \{a\}}}
    \proofstep{}{a \in A}{\rIE{1,2}}
  \closeproofpart

  \proofpart{$\dashv$}
    % Annahme
    \proofstep{1}{a \in A}{\rA}

    % (i) A \cup {a} \subseteq A
    \proofstep{2}{x \in A \cup \{a\}}{\rA}
    \proofstep{2}{x \in A \lor x \in \{a\}}%
      {\FormulaRefAuto{z \in A \cup B \eqvdash z \in A \lor z \in B}{2}}

    % Fall 1: x \in A
    \proofstep{3}{x \in A}{\rA}

    % Fall 2: x \in {a}  =>  x = a  =>  x \in A
    \proofstep{5}{x \in \{a\}}{\rA}
    \proofstep{5}{x = a}{\FormulaRefAuto{x \in \{a\} \eqvdash x = a}{5}}
    \proofstep{1,5}{x \in A}{\rIE{6,1}}

    % Fälle zusammenführen
    \proofstep{1,2}{x \in A}{\rOE{3,4,4,5,7}}

    % Universalisierung: Teilmengenrichtung
    \proofstep{}{A \cup \{a\} \subseteq A}{\FormulaRefAuto{A \subseteq B := \forall x\,(x \in A \rightarrow x \in B)}{\rUI{\rRI{2,8}}}}

    % (ii) A \subseteq A \cup {a}
    \proofstep{}{A \subseteq A \cup \{a\}}%
      {\FormulaRefAuto{z \in A \vdash z \in A \cup B}}

    % (iii) Gleichheit aus beidseitiger Inklusion
    \proofstep{}{A \cup \{a\} = A}{\FormulaRefAuto{A \subseteq B \land B \subseteq A \eqvdash A = B}{9,10}}
  \closeproofpart
\end{tabproofsplit}


\FormulaThmAuto{a \notin B,\, A = B \cup \{a\} \vdash A \not\subseteq B}
\begin{tabproof}
  \proofstep{1}{a \notin B}{\rA}
  \proofstep{2}{A = B \cup \{a\}}{\rA}
  \proofstep{3}{A \subseteq B}{\rA}
  \proofstep{}{a \in B \cup \{a\}}{\FormulaRefAuto{a \in A \cup \{a\}}}
  \proofstep{2}{a \in A}{\rIE{2,4}}
  \proofstep{2,3}{a \in B}{\FormulaRefAuto{A \subseteq B,\, x \in A \vdash x \in B}{5,3}}
  \proofstep{1,2,3}{\bot}{\rBI{1,6}}
  \proofstep{1,2}{A \not\subseteq B}{\rCI{3,7}}
\end{tabproof}

\FormulaThmAuto{x \in A \eqvdash x \in A \cup A}
\begin{tabproofwide}
  \proofstepwide{x \in A}{\leftrightarrow}{x \in A \lor x \in A}%
    {\FormulaRefAuto{P \lor P \eqvdash P}}
  \proofstepwide{}{\leftrightarrow}{x \in A \cup A}%
    {\FormulaRefAuto{z \in A \cup B \eqvdash z \in A \lor z \in B}}
  \proofstepwide{x \in A}{\leftrightarrow}{x \in A \cup A}%
    {\rChain{1,2}}
\end{tabproofwide}

\FormulaThmAuto{A = A \cup A}
\begin{tabproofwide}
  \proofstepwide{x \in A}{\leftrightarrow}{x \in A \cup A}%
    {\FormulaRefAuto{x \in A \eqvdash x \in A \cup A}}
  \proofstepwide{A}{=}{A \cup A}%
    {\FormulaRefAuto{\forall x\, (x \in A \leftrightarrow x \in B) \eqvdash A = B}{\rUI{1}}}
\end{tabproofwide}

\FormulaThmAuto{x \in A \cup B \eqvdash x \in B \cup A}
\begin{tabproofwide}
  \proofstepwide{x \in A \cup B}{\leftrightarrow}{x \in A \lor x \in B}%
    {\FormulaRefAuto{z \in A \cup B \eqvdash z \in A \lor z \in B}}
  \proofstepwide{}{\leftrightarrow}{x \in B \lor x \in A}%
    {\FormulaRefAuto{P \lor Q \vdash Q \lor P}{1}}
  \proofstepwide{}{\leftrightarrow}{x \in B \cup A}%
    {\FormulaRefAuto{z \in A \cup B \eqvdash z \in A \lor z \in B}{2}}
  \proofstepwide{x \in A \cup B}{\leftrightarrow}{x \in B \cup A}%
    {\rChain{1,3}}
\end{tabproofwide}

\FormulaThmAuto{A \cup B = B \cup A}
\begin{tabproofwide}
  \proofstepwide{x \in A \cup B}{\leftrightarrow}{x \in B \cup A}%
    {\FormulaRefAuto{x \in A \cup B \eqvdash x \in B \cup A}}
  \proofstepwide{A \cup B}{=}{B \cup A}%
    {\FormulaRefAuto{\forall x\, (x \in A \leftrightarrow x \in B) \eqvdash A = B}{\rUI{1}}}
\end{tabproofwide}

\FormulaThmAuto{x \in A \eqvdash x \in A \cup \emptyset}
\begin{tabproofwide}
  \proofstepwide{x \in A}{\leftrightarrow}{x \in A \lor x \in \emptyset}%
    {\FormulaRefAuto{\forall x(\neg Q(x)) \vdash P \leftrightarrow P \lor Q(a)}{\FormulaRefAuto{\emptyset := \iota O\bigl(\forall x\,(x \not\in O)\bigr)}}}
  \proofstepwide{}{\leftrightarrow}{x \in A \cup \emptyset}%
    {\FormulaRefAuto{z \in A \cup B \eqvdash z \in A \lor z \in B}{1}}
  \proofstepwide{x \in A}{\leftrightarrow}{x \in A \cup \emptyset}%
    {\rChain{1,2}}
\end{tabproofwide}

\FormulaThmAuto{A = A \cup \emptyset}
\begin{tabproofwide}
  \proofstepwide{x \in A}{\leftrightarrow}{x \in A \cup \emptyset}%
    {\FormulaRefAuto{x \in A \eqvdash x \in A \cup \emptyset}}
  \proofstepwide{A}{=}{A \cup \emptyset}%
    {\FormulaRefAuto{\forall x\, (x \in A \leftrightarrow x \in B) \eqvdash A = B}{\rUI{1}}}
\end{tabproofwide}

\FormulaThmAuto{A = \emptyset \cup A}
\begin{tabproofwide}
  \proofstepwide{A}{=}{A \cup \emptyset}%
    {\FormulaRefAuto{A = A \cup \emptyset}}
  \proofstepwide{A \cup \emptyset}{=}{\emptyset \cup A}%
    {\FormulaRefAuto{A \cup B = B \cup A}}
  \proofstepwide{A}{=}{\emptyset \cup A}%
    {\rIE{2,1}}
\end{tabproofwide}

\FormulaThmAuto{A \cup \{A\} = \{\emptyset\} \eqvdash A = \emptyset}
\begin{tabproofsplit}
\proofpart{\(\vdash\)}
  \proofstep{1}{\{\emptyset\} = A \cup \{A\}}{\rA}
  \proofstep{}{A \in A \cup \{A\}}{\FormulaRefAuto{a \in A \cup \{a\}}}
  \proofstep{1}{A \in \{\emptyset\}}{\FormulaRefAuto{A = B,\, x \in A \vdash x \in B}{1,2}}
  \proofstep{1}{A = \emptyset}{\FormulaRefAuto{x \in \{a\} \eqvdash x = a}{3}}
\closeproofpart

\proofpart{\(\dashv\)}
  \proofstep{1}{A = \emptyset}{\rA}
  \proofstep{}{\{\emptyset\} = \emptyset \cup \{\emptyset\}}{\FormulaRefAuto{A = \emptyset \cup A}}
  \proofstep{1}{A \cup \{A\} = \{\emptyset\}}{\rIE{1,\FormulaRefAuto{a = b \vdash b = a}{2}}}
\closeproofpart
\end{tabproofsplit}



\FormulaThmAuto{z \in A \cup B \eqvdash z \not\in A \rightarrow z \in B}
\begin{tabproofwide}
  \proofstepwide{z \in A \cup B}{\leftrightarrow}{z \in A \lor z \in B}%
    {\FormulaRefAuto{z \in A \cup B \eqvdash z \in A \lor z \in B}}
  \proofstepwide{}{ \leftrightarrow }{z \not\in A \rightarrow z \in B}%
    {\FormulaRefAuto{P \rightarrow Q \dashv \vdash \neg P \lor Q}{1}}
\end{tabproofwide}

\FormulaThmAuto{z \in A \cup B \eqvdash z \not\in B \rightarrow z \in A}
\begin{tabproofwide}
  \proofstepwide{z \in A \cup B}{\leftrightarrow}{z \in A \lor z \in B}%
    {\FormulaRefAuto{z \in A \cup B \eqvdash z \in A \lor z \in B}}
  \proofstepwide{}{\leftrightarrow}{z \not\in B \rightarrow z \in A}%
    {\FormulaRefAuto{P \rightarrow Q \dashv \vdash \neg P \lor Q}{1}}
\end{tabproofwide}

\FormulaThmAuto{A=B\vdash A\cup C = B\cup C}
\begin{tabproof}
    \proofstep{1}{A=B}{\rA}
    \proofstep{1}{x\in A\cup C\leftrightarrow x\in B\cup C}{\rIE{1,\FormulaRefAuto{P\leftrightarrow P}}}
    \proofstep{1}{A\cup C=B\cup C}{\FormulaRefAuto{\forall x\, (x \in A \leftrightarrow x \in B) \eqvdash A = B}{\rUI{2}}}
\end{tabproof}


\FormulaThmAuto{z \in (A \cup B) \cup C \eqvdash (z \in A \lor z \in B) \lor z \in C}
\begin{tabproofwide}
  % Zeile 1 -> zwei Zeilen
  \proofstepwide{z \in (A \cup B) \cup C}{\leftrightarrow}{z \in (A \cup B)}%
    {\multirow{2}{*}{\FormulaRefAuto{z \in A \cup B \eqvdash z \in A \lor z \in B}}}
  \proofstepwide*{}{\lor}{z \in C}{}

  % Zeile 2 -> zwei Zeilen
  \proofstepwide{}{\leftrightarrow}{(z \in A \lor z \in B)}%
    {\multirow{2}{*}{\rLRS{\FormulaRefAuto{z \in A \cup B \eqvdash z \in A \lor z \in B}{},1}}}
  \proofstepwide*{}{\lor}{z \in C}{}
\end{tabproofwide}

\FormulaThmAuto{z \in A \cup (B \cup C) \eqvdash z \in A \lor (z \in B \lor z \in C)}
\begin{tabproofwide}
  % Zeile 1 -> zwei Zeilen
  \proofstepwide{z \in A \cup (B \cup C)}{\leftrightarrow}{z \in A}%
    {\multirow{2}{*}{\FormulaRefAuto{z \in A \cup B \eqvdash z \in A \lor z \in B}}}
  \proofstepwide*{}{\lor}{z \in (B \cup C)}{}

  % Zeile 2 -> zwei Zeilen
  \proofstepwide{}{\leftrightarrow}{z \in A}%
    {\multirow{2}{*}{\rLRS{\FormulaRefAuto{z \in A \cup B \eqvdash z \in A \lor z \in B}{},1}}}
  \proofstepwide*{}{\lor}{(z \in B \lor z \in C)}{}
\end{tabproofwide}

\FormulaThmAuto{z \in (A \cup B) \cup C \eqvdash z \in A \cup (B \cup C)}
\begin{tabproofwide}
  \proofstepwide{z \in (A \cup B) \cup C}{\leftrightarrow}{(z \in A \lor z \in B) \lor z \in C}%
    {\FormulaRefAuto{z \in (A \cup B) \cup C \eqvdash (z \in A \lor z \in B) \lor z \in C}}
  \proofstepwide{}{\leftrightarrow}{z \in A \lor (z \in B \lor z \in C)}%
    {\FormulaRefAuto{P \lor (Q \lor R) \eqvdash (P \lor Q) \lor R}{1}}
  \proofstepwide{}{\leftrightarrow}{z \in A \cup (B \cup C)}%
    {\FormulaRefAuto{z \in A \cup (B \cup C) \eqvdash z \in A \lor (z \in B \lor z \in C)}{2}}
\end{tabproofwide}

\FormulaThmAuto[Assoziativität der Vereinigung]{(A \cup B) \cup C = A \cup (B \cup C)}
\begin{tabproofwide}
  \proofstepwide{z \in (A \cup B) \cup C}{\leftrightarrow}{z \in A \cup (B \cup C)}%
    {\FormulaRefAuto{z \in (A \cup B) \cup C \eqvdash z \in A \cup (B \cup C)}}
  \proofstepwide{(A \cup B) \cup C}{=}{A \cup (B \cup C)}%
    {\FormulaRefAuto{\forall x\, (x \in A \leftrightarrow x \in B) \eqvdash A = B}{\rUI{1}}}
\end{tabproofwide}

\FormulaThmAuto{z \in A \cup (B \cap C) \eqvdash z \in A \lor (z \in B \land z \in C)}
\begin{tabproofwide}
  % Zeile 1 -> zwei Zeilen
  \proofstepwide{z \in A \cup (B \cap C)}{\leftrightarrow}{z \in A}%
    {\multirow{2}{*}{\FormulaRefAuto{z \in A \cup B \eqvdash z \in A \lor z \in B}}}
  \proofstepwide*{}{\lor}{z \in (B \cap C)}{}

  % Zeile 2 -> zwei Zeilen
  \proofstepwide{}{\leftrightarrow}{z \in A}%
    {\multirow{2}{*}{\rLRS{\FormulaRefAuto{x \in A \cap B \eqvdash x \in A \land x \in B}{},1}}}
  \proofstepwide*{}{\lor}{(z \in B \land z \in C)}{}
\end{tabproofwide}

\FormulaThmAuto{z \in (A \cap B) \cup C \eqvdash (z \in A \land z \in B) \lor z \in C}
\begin{tabproofwide}
  % Zeile 1 -> zwei Zeilen
  \proofstepwide{z \in (A \cap B) \cup C}{\leftrightarrow}{z \in (A \cap B)}%
    {\multirow{2}{*}{\FormulaRefAuto{z \in A \cup B \eqvdash z \in A \lor z \in B}}}
  \proofstepwide*{}{\lor}{z \in C}{}

  % Zeile 2 -> zwei Zeilen
  \proofstepwide{}{\leftrightarrow}{(z \in A \land z \in B)}%
    {\multirow{2}{*}{\rLRS{\FormulaRefAuto{x \in A \cap B \eqvdash x \in A \land x \in B}{},1}}}
  \proofstepwide*{}{\lor}{z \in C}{}
\end{tabproofwide}

\FormulaThmAuto{z \in A \cap (B \cup C) \eqvdash z \in A \land (z \in B \lor z \in C)}
\begin{tabproofwide}
  % Zeile 1 -> zwei Zeilen
  \proofstepwide{z \in A \cap (B \cup C)}{\leftrightarrow}{z \in A}%
    {\multirow{2}{*}{\FormulaRefAuto{x \in A \cap B \eqvdash x \in A \land x \in B}}}
  \proofstepwide*{}{\land}{z \in (B \cup C)}{}

  % Zeile 2 -> zwei Zeilen
  \proofstepwide{}{\leftrightarrow}{z \in A}%
    {\multirow{2}{*}{\rLRS{\FormulaRefAuto{z \in A \cup B \eqvdash z \in A \lor z \in B}{},1}}}
  \proofstepwide*{}{\land}{(z \in B \lor z \in C)}{}
\end{tabproofwide}

\FormulaThmAuto{z \in (A \cup B) \cap C \eqvdash (z \in A \lor z \in B) \land z \in C}
\begin{tabproofwide}
  % Zeile 1 -> zwei Zeilen
  \proofstepwide{z \in (A \cup B) \cap C}{\leftrightarrow}{z \in (A \cup B)}%
    {\multirow{2}{*}{\FormulaRefAuto{x \in A \cap B \eqvdash x \in A \land x \in B}}}
  \proofstepwide*{}{\land}{z \in C}{}

  % Zeile 2 -> zwei Zeilen
  \proofstepwide{}{\leftrightarrow}{(z \in A \lor z \in B)}%
    {\multirow{2}{*}{\rLRS{\FormulaRefAuto{z \in A \cup B \eqvdash z \in A \lor z \in B}{},1}}}
  \proofstepwide*{}{\land}{z \in C}{}
\end{tabproofwide}

\FormulaThmAuto{z \in (A \cup B) \cap (C \cup D) \eqvdash (z \in A \lor z \in B) \land (z \in C \lor z \in D)}
\begin{tabproofwide}
  % Zeile 1 -> zwei Zeilen
  \proofstepwide{z \in (A \cup B) \cap (C \cup D)}{\leftrightarrow}{z \in (A \cup B)}%
    {\multirow{2}{*}{\FormulaRefAuto{x \in A \cap B \eqvdash x \in A \land x \in B}}}
  \proofstepwide*{}{\land}{z \in (C \cup D)}{}

  % Zeile 2 -> zwei Zeilen
  \proofstepwide{}{\leftrightarrow}{(z \in A \lor z \in B)}%
    {\multirow{2}{*}{\rLRS{\FormulaRefAuto{z \in A \cup B \eqvdash z \in A \lor z \in B}{},1}}}
  \proofstepwide*{}{\land}{(z \in C \lor z \in D)}{}
\end{tabproofwide}

\FormulaThmAuto{z \in (A \cap B) \cup (C \cap D) \eqvdash (z \in A \land z \in B) \lor (z \in C \land z \in D)}
\begin{tabproofwide}
  % Zeile 1 -> zwei Zeilen
  \proofstepwide{z \in (A \cap B) \cup (C \cap D)}{\leftrightarrow}{z \in (A \cap B)}%
    {\multirow{2}{*}{\FormulaRefAuto{z \in A \cup B \eqvdash z \in A \lor z \in B}}}
  \proofstepwide*{}{\lor}{z \in (C \cap D)}{}

  % Zeile 2 -> zwei Zeilen
  \proofstepwide{}{\leftrightarrow}{(z \in A \land z \in B)}%
    {\multirow{2}{*}{\rLRS{\FormulaRefAuto{x \in A \cap B \eqvdash x \in A \land x \in B}{},1}}}
  \proofstepwide*{}{\lor}{(z \in C \land z \in D)}{}
\end{tabproofwide}

\FormulaThmAuto{z \in A \cap (B \cup C) \eqvdash z \in (A \cap B) \cup (A \cap C)}
\begin{tabproofwide}
  % Zeile 1 -> zwei Zeilen
  \proofstepwide{z \in A \cap (B \cup C)}{\leftrightarrow}{z \in A}%
    {\multirow{2}{*}{\FormulaRefAuto{z \in A \cap (B \cup C) \eqvdash z \in A \land (z \in B \lor z \in C)}}}
  \proofstepwide*{}{\land}{(z \in B \lor z \in C)}{}

  % Zeile 2 -> zwei Zeilen
  \proofstepwide{}{\leftrightarrow}{(z \in A \land z \in B)}%
    {\multirow{2}{*}{\FormulaRefAuto{P \land (Q \lor R) \dashv \vdash (P \land Q) \lor (P \land R)}{1}}}
  \proofstepwide*{}{\lor}{(z \in A \land z \in C)}{}

  % Zeile 3 -> zwei Zeilen
  \proofstepwide{}{\leftrightarrow}{z \in (A \cap B)}%
    {\multirow{2}{*}{\FormulaRefAuto{z \in (A \cap B) \cup (C \cap D) \eqvdash (z \in A \land z \in B) \lor (z \in C \land z \in D)}{2}}}
  \proofstepwide*{}{\cup}{z \in (A \cap C)}{}

  % Abschluss (unverändert)
  \proofstepwide{z \in A \cap (B \cup C)}{\leftrightarrow}{z \in (A \cap B) \cup (A \cap C)}%
    {\rChain{1,3}}
\end{tabproofwide}

\FormulaThmAuto{z \in (A \cup B) \cap C \eqvdash z \in (A \cap C) \cup (B \cap C)}
\begin{tabproofwide}
  % Zeile 1 -> zwei Zeilen
  \proofstepwide{z \in (A \cup B) \cap C}{\leftrightarrow}{(z \in A \lor z \in B)}%
    {\multirow{2}{*}{\FormulaRefAuto{z \in (A \cup B) \cap C \eqvdash (z \in A \lor z \in B) \land z \in C}}}
  \proofstepwide*{}{\land}{z \in C}{}

  % Zeile 2 -> zwei Zeilen
  \proofstepwide{}{\leftrightarrow}{(z \in A \land z \in C)}%
    {\multirow{2}{*}{\FormulaRefAuto{(P \lor Q) \land R \dashv \vdash (P \land R) \lor (Q \land R)}{1}}}
  \proofstepwide*{}{\lor}{(z \in B \land z \in C)}{}

  % Zeile 3 -> zwei Zeilen
  \proofstepwide{}{\leftrightarrow}{z \in (A \cap C)}%
    {\multirow{2}{*}{\FormulaRefAuto{z \in (A \cap B) \cup (C \cap D) \eqvdash (z \in A \land z \in B) \lor (z \in C \land z \in D)}{2}}}
  \proofstepwide*{}{\cup}{z \in (B \cap C)}{}

  % Abschluss
  \proofstepwide{z \in (A \cup B) \cap C}{\leftrightarrow}{z \in (A \cap C) \cup (B \cap C)}%
    {\rChain{1,3}}
\end{tabproofwide}

\FormulaThmAuto{z \in (A \cap B) \cup C \eqvdash z \in (A \cup C) \cap (B \cup C)}
\begin{tabproofwide}
  % Zeile 1 -> zwei Zeilen
  \proofstepwide{z \in (A \cap B) \cup C}{\leftrightarrow}{(z \in A \land z \in B)}%
    {\multirow{2}{*}{\FormulaRefAuto{z \in (A \cap B) \cup C \eqvdash (z \in A \land z \in B) \lor z \in C}}}
  \proofstepwide*{}{\lor}{z \in C}{}

  % Zeile 2 -> zwei Zeilen
  \proofstepwide{}{\leftrightarrow}{(z \in A \lor z \in C)}%
    {\multirow{2}{*}{\FormulaRefAuto{(P \land Q) \lor R \dashv \vdash (P \lor R) \land (Q \lor R)}{1}}}
  \proofstepwide*{}{\land}{(z \in B \lor z \in C)}{}

  % Zeile 3 -> zwei Zeilen
  \proofstepwide{}{\leftrightarrow}{z \in (A \cup C)}%
    {\multirow{2}{*}{\FormulaRefAuto{z \in (A \cup B) \cap (C \cup D) \eqvdash (z \in A \lor z \in B) \land (z \in C \lor z \in D)}{2}}}
  \proofstepwide*{}{\cap}{z \in (B \cup C)}{}

  % Abschluss
  \proofstepwide{z \in (A \cap B) \cup C}{\leftrightarrow}{z \in (A \cup C) \cap (B \cup C)}%
    {\rChain{1,3}}
\end{tabproofwide}

\FormulaThmAuto{z \in A \cup (B \cap C) \eqvdash z \in (A \cup B) \cap (A \cup C)}
\begin{tabproofwide}
  % Zeile 1 -> zwei Zeilen
  \proofstepwide{z \in A \cup (B \cap C)}{\leftrightarrow}{z \in A}%
    {\multirow{2}{*}{\FormulaRefAuto{z \in A \cup (B \cap C) \eqvdash z \in A \lor (z \in B \land z \in C)}}}
  \proofstepwide*{}{\lor}{(z \in B \land z \in C)}{}

  % Zeile 2 -> zwei Zeilen
  \proofstepwide{}{\leftrightarrow}{(z \in A \lor z \in B)}%
    {\multirow{2}{*}{\FormulaRefAuto{P \lor (Q \land R) \dashv \vdash (P \lor Q) \land (P \lor R)}{1}}}
  \proofstepwide*{}{\land}{(z \in A \lor z \in C)}{}

  % Zeile 3 -> zwei Zeilen
  \proofstepwide{}{\leftrightarrow}{z \in (A \cup B)}%
    {\multirow{2}{*}{\FormulaRefAuto{z \in (A \cup B) \cap (C \cup D) \eqvdash (z \in A \lor z \in B) \land (z \in C \lor z \in D)}{2}}}
  \proofstepwide*{}{\cap}{z \in (A \cup C)}{}

  % Abschluss
  \proofstepwide{z \in A \cup (B \cap C)}{\leftrightarrow}{z \in (A \cup B) \cap (A \cup C)}%
    {\rChain{1,3}}
\end{tabproofwide}

\FormulaThmAuto{A \cup (B \cap C) = (A \cup B) \cap (A \cup C)}
\begin{tabproofwide}
  \proofstepwide{z \in A \cup (B \cap C)}{\leftrightarrow}{z \in (A \cup B) \cap (A \cup C)}%
    {\FormulaRefAuto{z \in A \cup (B \cap C) \eqvdash z \in (A \cup B) \cap (A \cup C)}}
  \proofstepwide{A \cup (B \cap C)}{=}{(A \cup B) \cap (A \cup C)}%
    {\FormulaRefAuto{\forall x\, (x \in A \leftrightarrow x \in B) \eqvdash A = B}{\rUI{1}}}
\end{tabproofwide}

\FormulaThmAuto{(A \cap B) \cup C = (A \cup C) \cap (B \cup C)}
\begin{tabproofwide}
  \proofstepwide{z \in (A \cap B) \cup C}{\leftrightarrow}{z \in (A \cup C) \cap (B \cup C)}%
    {\FormulaRefAuto{z \in (A \cap B) \cup C \eqvdash z \in (A \cup C) \cap (B \cup C)}}
  \proofstepwide{(A \cap B) \cup C}{=}{(A \cup C) \cap (B \cup C)}%
    {\FormulaRefAuto{\forall x\, (x \in A \leftrightarrow x \in B) \eqvdash A = B}{\rUI{1}}}
\end{tabproofwide}

\FormulaThmAuto{A \cap (B \cup C) = (A \cap B) \cup (A \cap C)}
\begin{tabproofwide}
  \proofstepwide{z \in A \cap (B \cup C)}{\leftrightarrow}{z \in (A \cap B) \cup (A \cap C)}%
    {\FormulaRefAuto{z \in A \cap (B \cup C) \eqvdash z \in (A \cap B) \cup (A \cap C)}}
  \proofstepwide{A \cap (B \cup C)}{=}{(A \cap B) \cup (A \cap C)}%
    {\FormulaRefAuto{\forall x\, (x \in A \leftrightarrow x \in B) \eqvdash A = B}{\rUI{1}}}
\end{tabproofwide}

\FormulaThmAuto{(A \cup B) \cap C = (A \cap C) \cup (B \cap C)}
\begin{tabproofwide}
  \proofstepwide{z \in (A \cup B) \cap C}{\leftrightarrow}{z \in (A \cap C) \cup (B \cap C)}%
    {\FormulaRefAuto{z \in (A \cup B) \cap C \eqvdash z \in (A \cap C) \cup (B \cap C)}}
  \proofstepwide{(A \cup B) \cap C}{=}{(A \cap C) \cup (B \cap C)}%
    {\FormulaRefAuto{\forall x\, (x \in A \leftrightarrow x \in B) \eqvdash A = B}{\rUI{1}}}
\end{tabproofwide}

\section[Eigenschaften von Teilmengen]{Eigenschaften von Teilmengen in Bezug auf Vereinigung und Durchschnitt}


\FormulaThmAuto{A \subseteq A \cup B}
\begin{tabproofwide}
  \proofstepwide{x \in A}{\rightarrow}{x \in A \lor x \in B}%
    {\FormulaRefAuto{P \rightarrow P \lor Q}}
  \proofstepwide{}{ \rightarrow}{x \in A \cup B}%
    {\FormulaRefAuto{z \in A \cup B \eqvdash z \in A \lor z \in B}{1}}
  \proofstepwide{x \in A}{\rightarrow}{x \in A \cup B}%
    {\rChain{1,2}}
  \proofstepwidestar{A \subseteq A \cup B}%
    {\FormulaRefAuto{ A \subseteq B := \forall x\,(x\in A \rightarrow x\in B) }{\rUI{3}}}
\end{tabproofwide}

\FormulaThmAuto{A \subseteq B \cup A}
\begin{tabproofwide}
  \proofstepwide{x \in A}{\rightarrow}{x \in B \lor x \in A}%
    {\FormulaRefAuto{P \rightarrow Q \lor P}}
  \proofstepwide{}{\rightarrow}{x \in B \cup A}%
    {\FormulaRefAuto{z \in A \cup B \eqvdash z \in A \lor z \in B}{1}}
  \proofstepwide{x \in A}{\rightarrow}{x \in B \cup A}%
    {\rChain{1,2}}
  \proofstepwidestar{A \subseteq B \cup A}%
    {\FormulaRefAuto{A \subseteq B := \forall x\,(x\in A \rightarrow x\in B)}{\rUI{3}}}
\end{tabproofwide}

\FormulaThmAuto{A \subseteq C,\, B \subseteq C \vdash A \cup B \subseteq C}
\begin{tabproofwide}
  \proofstepwidestar[1]{A \subseteq C}{\rA}
  \proofstepwidestar[2]{B \subseteq C}{\rA}
  \proofstepwide{z \in A \cup B}{\rightarrow}{z \in A \lor z \in B}%
    {\FormulaRefAuto{z \in A \cup B \eqvdash z \in A \lor z \in B}}
  \proofstepwide[1,2]{}{ \rightarrow}{z \in C}%
    {\FormulaRefAuto{A\subseteq C,\, B\subseteq C,\, z\in A\lor z\in B \vdash z\in C}{1,2,3}}
  \proofstepwide[1,2]{z \in A \cup B}{\rightarrow}{z \in C}%
    {\rChain{3,4}}
  \proofstepwidestar[1,2]{A \cup B \subseteq C}%
    {\FormulaRefAuto{A \subseteq B := \forall x\,(x\in A \rightarrow x\in B)}{\rUI{5}}}
\end{tabproofwide}


\FormulaThmAuto{A \subseteq B \vdash A \cup C \subseteq B \cup C}
\begin{tabproofwide}
  \proofstepwidestar[1]{A \subseteq B}{\rA}
  \proofstepwide{z \in A \cup C}{\rightarrow}{z \in A \lor z \in C}%
    {\FormulaRefAuto{z \in A \cup B \eqvdash z \in A \lor z \in B}}
  \proofstepwide[1]{}{ \rightarrow}{z \in B \lor z \in C}%
    {\FormulaRefAuto{P \rightarrow Q,\, P \lor R \vdash Q \lor R}%
      {\rUE{\FormulaRefAuto{A \subseteq B := \forall x\,(x\in A \rightarrow x\in B)}{1}},2}}
  \proofstepwide[1]{}{ \rightarrow}{z \in B \cup C}%
    {\FormulaRefAuto{z \in A \cup B \eqvdash z \in A \lor z \in B}{3}}
  \proofstepwide[1]{z \in A \cup C}{\rightarrow}{z \in B \cup C}%
    {\rChain{2,4}}
  \proofstepwidestar[1]{A \cup C \subseteq B \cup C}%
    {\FormulaRefAuto{A \subseteq B := \forall x\,(x\in A \rightarrow x\in B)}{\rUI{5}}}
\end{tabproofwide}

\FormulaThmAuto{A \subseteq B \vdash C \cup A \subseteq C \cup B}
\begin{tabproofwide}
  \proofstepwidestar[1]{A \subseteq B}{\rA}
  \proofstepwide{C \cup A}{=}{A \cup C}%
    {\FormulaRefAuto{A \cup B = B \cup A}}
  \proofstepwide[1]{}{ \subseteq}{B \cup C}%
    {\FormulaRefAuto{A \subseteq B \vdash A \cup C \subseteq B \cup C}{1}}
  \proofstepwide[1]{}{=}{C \cup B}%
    {\FormulaRefAuto{A \cup B = B \cup A}{3}}
  \proofstepwide[1]{C \cup A}{\subseteq}{C \cup B}%
    {\rChain{2,4}}
\end{tabproofwide}

\FormulaThmAuto{A \subseteq B,\, C \subseteq D \vdash A \cup C \subseteq B \cup D}
\begin{tabproofwide}
  \proofstepwidestar[1]{A \subseteq B}{\rA}
  \proofstepwidestar[2]{C \subseteq D}{\rA}
  \proofstepwide[1]{A \cup C}{\subseteq}{B \cup C}%
    {\FormulaRefAuto{A \subseteq B \vdash A \cup C \subseteq B \cup C}{1}}
  \proofstepwide[2]{}{ \subseteq}{B \cup D}%
    {\FormulaRefAuto{A \subseteq B \vdash C \cup A \subseteq C \cup B}{2}}
  \proofstepwide[1,2]{A \cup C}{\subseteq}{B \cup D}%
    {\rChain{3,4}}
\end{tabproofwide}

\FormulaThmAuto{A \subseteq B,\, C \subseteq D \vdash A \cap C \subseteq B \cap D}
\begin{tabproofwide}
  \proofstepwidestar[1]{A \subseteq B}{\rA}
  \proofstepwidestar[2]{C \subseteq D}{\rA}
  \proofstepwide{x \in A \cap C}{\rightarrow}{x \in A \land x \in C}%
    {\FormulaRefAuto{x \in A \cap B \eqvdash x \in A \land x \in B}}
  \proofstepwide[1]{}{ \rightarrow}{x \in B \land x \in C}%
    {\FormulaRefAuto{P \rightarrow Q,\, P \land R \vdash Q \land R}{\rUE{\FormulaRefAuto{A \subseteq B := \forall x\,(x \in A \rightarrow x \in B)}{1}},3}}
  \proofstepwide[1,2]{}{ \rightarrow}{x \in B \land x \in D}%
    {\FormulaRefAuto{P \rightarrow Q,\, R \land P \vdash R \land Q}{\rUE{\FormulaRefAuto{A \subseteq B := \forall x\,(x \in A \rightarrow x \in B)}{2}},4}}
  \proofstepwide[1,2]{}{ \rightarrow}{x \in B \cap D}%
    {\FormulaRefAuto{x \in A \cap B \eqvdash x \in A \land x \in B}{5}}
  \proofstepwide[1,2]{x \in A \cap C}{\rightarrow}{x \in B \cap D}%
    {\rChain{3,6}}
  \proofstepwidestar[1,2]{A \cap C \subseteq B \cap D}%
    {\FormulaRefAuto{A \subseteq B := \forall x\,(x \in A \rightarrow x \in B)}{\rUI{7}}}
\end{tabproofwide}

\FormulaThmAuto{a \in A,\, b \in B \vdash \{a,b\} \subseteq A \cup B}
\begin{tabproofwide}
  \proofstepwidestar[1]{a \in A}{\rA}
  \proofstepwidestar[2]{b \in B}{\rA}
  \proofstepwide[1]{\{a\}}{\subseteq}{A}%
    {\FormulaRefAuto{a \in A \vdash \{a\} \subseteq A}{1}}
  \proofstepwide[2]{\{b\}}{\subseteq}{B}%
    {\FormulaRefAuto{a \in A \vdash \{a\} \subseteq A}{2}}
  \proofstepwide[1,2]{\{a\} \cup \{b\}}{\subseteq}{A \cup B}%
    {\FormulaRefAuto{A \subseteq B,\, C \subseteq D \vdash A \cup C \subseteq B \cup D}{3,4}}
  \proofstepwide{\{a,b\}}{=}{\{a\} \cup \{b\}}%
    {\FormulaRefAuto{\{a,b\} = \{a\} \cup \{b\}}}
  \proofstepwidestar{\{a,b\} \subseteq A \cup B}%
    {\rIE{6,5}}
\end{tabproofwide}

\FormulaThmAuto{a \in A,\, b \in A \vdash \{a,b\} \subseteq A}
\begin{tabproofwide}
  \proofstepwidestar[1]{a \in A}{\rA}
  \proofstepwidestar[2]{b \in A}{\rA}
  \proofstepwide[1,2]{\{a,b\}}{\subseteq}{A \cup A}%
    {\FormulaRefAuto{a \in A,\, b \in B \vdash \{a,b\} \subseteq A \cup B}{1,2}}
  \proofstepwide[1,2]{}{=}{A}%
    {\FormulaRefAuto{A = A \cup A}{3}}
  \proofstepwidestar[1,2]{\{a,b\} \subseteq A}%
    {\rChain{3,4}}
\end{tabproofwide}

\section{Ausschluss gegenseitiger Mitgliedschaft}

\FormulaThmAuto{a\in b\vdash b\not\in a}
\begin{tabproof}
  \proofstep{1}{a \in b}{\rA}
  \proofstep{2}{b \in a}{\rA}
  \proofstep{ }{\{a, b\} \neq \emptyset}{\FormulaRefAuto{\{a,b\} \neq \emptyset}}
  \proofstep{ }{\exists x \in \{a, b\}\,(x \cap \{a, b\} = \emptyset)}{\FormulaRefAuto{ A \neq \emptyset \vdash \exists x \in A \,(x \cap A = \emptyset) }{3}}
  \proofstep{ }{a \cap \{a, b\} = \emptyset \;\lor\; b \cap \{a, b\} = \emptyset}{\FormulaRefAuto{\exists x\in \{a,b\} P(x)\vdash P(a)\lor P(b)}{4}}
  \proofstep{6}{a \cap \{a, b\} = \emptyset}{\rA}
  \proofstep{2}{a \cap \{a, b\} \neq \emptyset}{\FormulaRefAuto{a\in A\vdash A\cap \{A,a\}\neq\emptyset}{2}}
  \proofstep{2,6}{\bot}{\rBI{6,7}}
  \proofstep{9}{b \cap \{a, b\} = \emptyset}{\rA}
  \proofstep{1}{b \cap \{a, b\} \neq \emptyset}{\FormulaRefAuto{a\in A\vdash A\cap \{A,a\}\neq\emptyset}{1}}
  \proofstep{1,9}{\bot}{\rBI{9,10}}
  \proofstep{1,2}{\bot}{\rOE{5,6,8,9,11}}
  \proofstep{1}{b\notin a}{\rCI{2,12}}
\end{tabproof}

\FormulaThmAuto{a\not\in a}
\begin{tabproof}
  \proofstep{1}{a \in a}{\rA}
  \proofstep{1}{a \notin a}{\FormulaRefAuto{a\in b\vdash b\not\in a}}
  \proofstep{1}{\bot}{\rBI{1,2}}
  \proofstep{}{a\notin a}{\rCE{1,3}}
\end{tabproof}

\FormulaThmAuto{a\cup\{a\}=b\cup\{b\}\eqvdash a=b}
\begin{tabproofsplit}
\proofpart{\(\vdash\)}
  \proofstep{1}{a \cup \{a\} = b \cup \{b\}}{\rA}

  % aus der Gleichheit folgen die beiden Disjunktionen
  \proofstep{}{a \in a \cup \{a\}}{\FormulaRefAuto{a \in A \cup \{a\}}}
  \proofstep{1}{a \in b \cup \{b\}}{\rIE{1,2}}
  \proofstep{1}{a \in b \lor a \in \{b\}}{\FormulaRefAuto{z \in A \cup B \eqvdash z \in A \lor z \in B}{3}}
  \proofstep{1}{a \in b \lor a = b}{\rLRS{\FormulaRefAuto{x \in \{a\} \eqvdash x = a}{},4}}

  \proofstep{}{b \in b \cup \{b\}}{\FormulaRefAuto{a \in A \cup \{a\}}}
  \proofstep{1}{b \in a \cup \{a\}}{\rIE{1,6}}
  \proofstep{1}{b \in a \lor b \in \{a\}}{\FormulaRefAuto{z \in A \cup B \eqvdash z \in A \lor z \in B}{7}}
  \proofstep{1}{b \in a \lor b = a}{\rLRS{\FormulaRefAuto{x \in \{a\} \eqvdash x = a}{},8}}

  % indirekt: a \neq b  ⇒  a \in b und b \in a; Widerspruch mit a∈b ⟹ b∉a
  \proofstep{10}{a \neq b}{\rA}

  % aus (a∈b ∨ a=b) und a≠b folgt a∈b
  \proofstep{1,10}{a \in b}{\FormulaRefAuto{P\lor Q, \neg Q\vdash P}{5,10}}

  % aus (b∈a ∨ b=a) und b≠a (Symmetrie der Ungleichheit) folgt b∈a
  \proofstep{1,10}{b \in a}{\FormulaRefAuto{P\lor Q, \neg Q\vdash P}{9,10}}

  % Widerspruch mit dem Lemma a∈b ⟹ b∉a
  \proofstep{1,10}{\bot}{\rBI{11,12}}
  \proofstep{1}{a = b}{\rCE{10,13}}
\closeproofpart

\proofpart{\(\dashv\)}
  \proofstep{1}{a = b}{\rA}
  \proofstep{1}{a \cup \{a\} = b \cup \{b\}}{\FormulaRefAuto{A=B\vdash A\cup C = B\cup C}{1}}
\closeproofpart
\end{tabproofsplit}


\chapter{Die Potenzmenge}

\section{Eigenschaften der Potenzmenge}

\FormulaThmAuto{A \subseteq B \vdash \mathcal{P}(A) \subseteq \mathcal{P}(B)}
\begin{tabproofwide}
  \proofstepwidestar[1]{A \subseteq B}{\rA}
  \proofstepwide{x \in \mathcal{P}(A)}{\rightarrow}{x \subseteq A}%
    {\FormulaRefAuto{\mathcal{P}(A) := \iota B\Bigl(\forall x\;\bigl(x \in B \leftrightarrow x \subseteq A\bigr)\Bigr)}}
  \proofstepwide[1]{}{ \rightarrow}{x \subseteq B}%
    {\FormulaRefAuto{A \subseteq B,\, B \subseteq C \vdash A \subseteq C}{2,1}}
  \proofstepwide[1]{}{ \rightarrow}{x \in \mathcal{P}(B)}%
    {\FormulaRefAuto{\mathcal{P}(A) := \iota B\Bigl(\forall x\;\bigl(x \in B \leftrightarrow x \subseteq A\bigr)\Bigr)}{3}}
  \proofstepwide[1]{x \in \mathcal{P}(A)}{\rightarrow}{x \in \mathcal{P}(B)}%
    {\rChain{2,4}}
  \proofstepwidestar[1]{\mathcal{P}(A) \subseteq \mathcal{P}(B)}%
    {\FormulaRefAuto{A \subseteq B := \forall x\,(x\in A \rightarrow x\in B)}{\rUI{5}}}
\end{tabproofwide}

\FormulaThmAuto{a \in A \vdash \{a\} \in \mathcal{P}(A)}
\begin{tabproof}
  \proofstep{1}{a \in A}{\rA}
  \proofstep{1}{\{a\} \subseteq A}{\FormulaRefAuto{a \in A \vdash \{a\} \subseteq A}{1}}
  \proofstep{1}{\{a\} \in \mathcal{P}(A)}{\FormulaRefAuto{\mathcal{P}(A) := \iota B\Bigl(\forall x\;\bigl(x \in B \leftrightarrow x \subseteq A\bigr)\Bigr)}{2}}
\end{tabproof}

\FormulaThmAuto{A \subseteq B,\, a \in \mathcal{P}(A) \vdash a \in \mathcal{P}(B)}
\begin{tabproof}
  \proofstep{1}{A \subseteq B}{\rA}
  \proofstep{2}{a \in \mathcal{P}(A)}{\rA}
  \proofstep{1}{\mathcal{P}(A) \subseteq \mathcal{P}(B)}{\FormulaRefAuto{A \subseteq B \vdash \mathcal{P}(A) \subseteq \mathcal{P}(B)}{1}}
  \proofstep{1,2}{a \in \mathcal{P}(B)}{\rRE{\rUE{\FormulaRefAuto{A \subseteq B := \forall x\,(x \in A \rightarrow x \in B)}{3}},2}}
\end{tabproof}


\FormulaThmAuto{a \in \mathcal{P}(A) \vdash \forall B\,(a \in \mathcal{P}(A \cup B))}
\begin{tabproof}
  \proofstep{1}{a \in \mathcal{P}(A)}{\rA}
  \proofstep{}{A \subseteq A \cup B}{\FormulaRefAuto{A \subseteq A \cup B}{}}
  \proofstep{1}{a \in \mathcal{P}(A \cup B)}%
    {\FormulaRefAuto{A \subseteq B,\, a \in \mathcal{P}(A) \vdash a \in \mathcal{P}(B)}{2,1}}
  \proofstep{1}{\forall B\,(a \in \mathcal{P}(A \cup B))}{\rUI{3}}
\end{tabproof}





\chapter{Das kartesische Produkt}


\section{Existenz des karthesischen Produktes}

\FormulaThmAuto{a \in A,\, b \in B \vdash (a,b) \in \mathcal{P}(\mathcal{P}(A \cup B))}
\begin{tabproof}
  \proofstep{1}{a \in A}{\rA}
  \proofstep{2}{b \in B}{\rA}
  \proofstep{1,2}{\{a,b\} \subseteq A \cup B}{\FormulaRefAuto{a \in A,\, b \in B \vdash \{a,b\} \subseteq A \cup B}{1,2}}
  \proofstep{1,2}{\{a,b\} \in \mathcal{P}(A \cup B)}{\FormulaRefAuto{\mathcal{P}(A) := \iota B\bigl(\forall x\,(x \in B \leftrightarrow x \subseteq A)\bigr)}{3}}
  \proofstep{1}{a \in A \cup B}{\FormulaRefAuto{z \in A \vdash z \in A \cup B}{1}}
  \proofstep{1}{\{a\} \in \mathcal{P}(A \cup B)}{\FormulaRefAuto{a \in A \vdash \{a\} \in \mathcal{P}(A)}{5}}
  \proofstep{1,2}{\{\{a\},\{a,b\}\} \subseteq \mathcal{P}(A \cup B)}{\FormulaRefAuto{a \in A,\, b \in B \vdash \{a,b\} \subseteq A \cup B}{6,4}}
  \proofstep{1,2}{\{\{a\},\{a,b\}\} \in \mathcal{P}(\mathcal{P}(A \cup B))}{\FormulaRefAuto{\mathcal{P}(A) := \iota B\bigl(\forall x\,(x \in B \leftrightarrow x \subseteq A)\bigr)}{7}}
  \proofstep{1,2}{(a,b) \in \mathcal{P}(\mathcal{P}(A \cup B))}{\rIE{\FormulaRefAuto{\forall a,b((a, b) := \{ \{ a \}, \{ a, b \}) \}},8}}
\end{tabproof}

\FormulaThmAuto{\exists a \in A\, \exists b \in B \big(x=(a,b)\big)\vdash x \in \mathcal{P}(\mathcal{P}(A \cup B))}
\begin{tabproof}
  \proofstep{1}{\exists a \in A\, \exists b \in B \big(x=(a,b)\big)}{\rA}
  \proofstep{2}{(a \in A\land b \in B)\land x=(a,b)}{\rA}
  \proofstep{2}{a \in A}{\FormulaRefAuto{(P\land Q)\land R\vdash P}{2}}
  \proofstep{2}{b \in B}{\FormulaRefAuto{(P\land Q)\land R\vdash Q}{2}}
  \proofstep{2}{x=(a,b)}{\rAEb{2}}
  \proofstep{2}{(a,b) \in \mathcal{P}(\mathcal{P}(A \cup B))}{\FormulaRefAuto{a \in A,\, b \in B \vdash (a,b) \in \mathcal{P}(\mathcal{P}(A \cup B))}{3,4}}
  \proofstep{2}{x \in \mathcal{P}(\mathcal{P}(A \cup B))}{\rIE{5,6}}  
  \proofstep{1}{x \in \mathcal{P}(\mathcal{P}(A \cup B))}{\rEE{1,2,7}}  
\end{tabproof}

\FormulaThmAuto[Eindeutige Existenz des kartesischen Produkts]{\exists! C\, \forall (a,b) \bigl(x \in C \leftrightarrow \exists a \in A\, \exists b \in B \big(x=(a,b)\big)\bigr)}
\begin{tabproof}
  \proofstep{}{ \exists! C\, \forall (a,b) \bigl((a,b) \in C \leftrightarrow \exists a \in A\, \exists b \in B}{\multirow{2}{*}{\rEI{\FormulaRefAuto{\forall x(P(x)\rightarrow x\in A)\vdash \exists! B(\forall x(x\in B\leftrightarrow P(x)))}{\rUI{\FormulaRefAuto{\exists a \in A,\, \exists b \in B \big(x=(a,b)\big)\vdash x \in \mathcal{P}(\mathcal{P}(A \cup B))}}}}}}
  \proofstepstar{}{ \big(x=(a,b)\big)\bigr)}{}
\end{tabproof}

\FormulaDefAuto[Kartesisches Produkt ($A \times B$)]{A \times B := \iota C\, \forall x\, \bigl(x \in C \leftrightarrow \exists a \in A\, \exists b \in B \big(x=(a,b)\big)\bigr)}
\begin{remark}
    Hieraus ergibt sich folgendes:
\[
    x \in A \times B \eqvdash \exists a \in A\, \exists b \in B \big(x=(a,b)\big)
\]
\end{remark}

\FormulaThmAuto[Kartesisches Produkt ($A \times B$)]{(a,b)\in A\times B\eqvdash a\in A\land b\in B}
\begin{tabproofsplit}
\proofpart{\(\vdash\)}
  \proofstep{1}{(a,b)\in A\times B}{\rA}
  \proofstep{1}{\exists c\in A\exists d\in B((a,b)=(c,d))}{\FormulaRefAuto{A \times B := \iota C\, \forall x\, \bigl(x \in C \leftrightarrow \exists a \in A\, \exists b \in B \big(x=(a,b)\big)\bigr)}{1}}
  \proofstep{3}{c\in A\land d\in B\land (c,d)=(a,b)}{\rA}
  \proofstep{3}{c\in A}{\rAEn{3}}
  \proofstep{3}{d\in B}{\rAEn{3}}
  \proofstep{3}{(c,d)=(a,b)}{\rAEn{3}}
  \proofstep{3}{c=a\land d=b}{\FormulaRefAuto{(a,b)=(c,d)\eqvdash a=c\land b=d}{6}}
  \proofstep{3}{c=a}{\rAEa{7}}
  \proofstep{3}{d=b}{\rAEb{7}}
  \proofstep{3}{a\in A}{\rIE{8,4}}
  \proofstep{3}{b\in B}{\rIE{9,5}}
  \proofstep{3}{a\in A\land b\in B}{\rAI{10,11}}
  \proofstep{1}{a\in A\land b\in B}{\rEE{2,3,12}}
\closeproofpart
\proofpart{\(\dashv\)}
  \proofstep{1}{a\in A\land b\in B}{\rA}
  \proofstep{}{(a,b)=(a,b)}{\rII}
  \proofstep{1}{a\in A\land b\in B\land (a,b)=(a,b)}{\rAI{1,2}}
  \proofstep{1}{\exists a\in A\exists b\in B((a,b)=(a,b))}{\rEI{3}}
  \proofstep{1}{(a,b)\in A\times B}{\FormulaRefAuto{A \times B := \iota C\, \forall x\, \bigl(x \in C \leftrightarrow \exists a \in A\, \exists b \in B \big(x=(a,b)\big)\bigr)}{4}}
\closeproofpart
\end{tabproofsplit}

\FormulaThmAuto{a\in A,b\in B\vdash (a,b)\in A\times B}
\begin{tabproof}
      \proofstep{1}{a\in A}{\rA}
      \proofstep{2}{b\in B}{\rA}
      \proofstep{1,2}{a\in A\land b\in B}{\rAI{1,2}}
      \proofstep{1,2}{(a,b)\in A\times B}{\FormulaRefAuto{(a,b)\in A\times B\eqvdash a\in A\land b\in B}{3}}
\end{tabproof}



\chapter{Funktionen}
Nachdem wir die grundlegenden Begriffe der Mengenlehre eingeführt haben, wenden wir uns nun den 
\textbf{Funktionen} zu. Diese sind zentrale Objekte der Mathematik und können als spezielle Mengen 
von geordneten Paaren aufgefasst werden.

\begin{definition}[Begriff der Funktion]
Der \textbf{Begriff der Funktion} wird durch das Symbol
\[
F\colon A \to B
\]
\textbf{implizit definiert}. 
Das Symbol \(F\colon A\to B\) bedeutet, dass \(F\) eine Teilmenge des kartesischen Produkts \(A\times B\) ist,
die die folgenden Eigenschaften erfüllt. Wir fassen diese Eigenschaften als Menge von Aussagen 
\(\Phi(F,A,B)\) zusammen:
\end{definition}

\FormulaAxiomAuto[Existenz als Menge geordneter Paare]{F \subseteq A \times B}

\FormulaAxiomAuto[Funktionale Eindeutigkeit]{\forall x \in A\,\forall y,z \in B\;\bigl((x,y)\in F \land (x,z)\in F \rightarrow y=z\bigr)}

\FormulaAxiomAuto[Totale Definition auf der Domäne]{\forall x \in A\,\exists y \in B\;(x,y)\in F}

\begin{remark}
Die Bedingung (1) sagt aus, dass eine Funktion nichts anderes als eine Teilmenge des kartesischen Produkts ist.  
Bedingung (2) fordert die Eindeutigkeit des Funktionswerts.  
Bedingung (3) garantiert, dass jedes \(x\in A\) tatsächlich ein Bild in \(B\) hat.  
\end{remark}

\begin{definition}[Domäne und Kodomäne]
Sei \(F\colon A\to B\) eine Funktion. Dann heißt
\begin{itemize}
  \item \(A\) der \textbf{Definitionsbereich (Domäne)} von \(F\), notiert als \(\mathrm{dom}(F)\),
  \item \(B\) der \textbf{Wertebereich (Kodomain)} von \(F\).
\end{itemize}
\end{definition}


\FormulaThmAuto[Eindeutigkeit des Funktionswertes]{\forall x \in A\,\exists! y \in B\;(x,y)\in F}[Sei \(F\colon A\to B\) eine Funktion, dann gilt:]
\begin{tabproof}
  \proofstep{}{(x,y)\in F \land (x,z)\in F \rightarrow y=z}{\rUE{\FormulaRefAuto{\forall x \in A\,\forall y,z \in B\;\bigl((x,y)\in F \land (x,z)\in F \rightarrow y=z\bigr)}}}
  \proofstep{}{\exists y\in B(x,y)\in F}{\rUE{\FormulaRefAuto{\forall x \in A\,\exists y \in B\;(x,y)\in F}}}
  \proofstep{3}{(x,y)\in F}{\rA}
  \proofstep{4}{(x,z)\in F}{\rA}
  \proofstep{3,4}{(x,y)\in F \land (x,z)\in F}{\rAI{3,4}}
  \proofstep{3,4}{y=z}{\rRE{1,5}}
  \proofstep{}{\forall x\in A\exists! y\in B(x,y)\in F}{\rUI{\UEI{2,3,4,6}}}
\end{tabproof}

\begin{remark}
Im Folgenden verwenden wir für \(x \in A\) die Notation
\[
F(x) := y \quad\text{genau dann, wenn}\quad (x,y)\in F.
\]
Dies ist wohldefiniert aufgrund des obigen Theorems.
\end{remark}

\FormulaDefAuto[Graph einer Funktion]{\mathrm{Graph}(F) := \{\, (x,y) \in A \times B \mid (x,y)\in F \,\}}%
[Sei \(F\colon A\to B\) eine Funktion. Wir definieren den \textbf{Graphen} von \(F\) als die Menge aller Paare aus \(A\times B\), die zu \(F\) gehören:]


\FormulaThmAuto[Ersetzung über Funktionen]{
  \exists! C\forall y(y\in C\leftrightarrow \exists x\in A (F(x)=y)
}[Sei \(F\colon A\to B\) eine Funktion, dann gilt:]

\begin{tabproof}
  \proofstep{}{\forall x \in A\,\exists! y \in B\;(x,y)\in F}{\FormulaRefAuto{\forall x \in A\,\exists! y \in B\;(x,y)\in F}}
  \proofstep{}{\exists C\;\forall y\;\bigl( y\in C\;\leftrightarrow\; \exists x\in A\;F(x)=y\bigr)}{\rRE{\rUE{\FormulaRefAuto{
\forall A\;\forall R\;\Bigl( \forall x\in A\;\exists! y\;R(x,y)\in R\;\rightarrow\; \exists C\;\forall y\;\bigl( y\in C\;\leftrightarrow\; \exists x\in A\;(x,y)\in R \bigr) \Bigr)}
},1}}
  \proofstep{}{\exists! C\;\forall y\;\bigl( y\in C\;\leftrightarrow\; \exists x\in A\;F(x)=y\bigr)}{\FormulaRefAuto{ \exists B(\forall x(x \in B\leftrightarrow P(x)))\vdash \exists! B(\forall x(x \in B\leftrightarrow P(x))) }{2}}
\end{tabproof}

\FormulaDefAuto[Funktionsbild]{F(A) := \{y \mid \exists x \in A\,(F(x)=y)\}}%
[Für eine Funktion \(F:A\to B\) definieren wir das Bild von \(A\) als die Menge aller Werte, 
die \(F\) für Elemente aus \(A\) annimmt:]

\chapter{Konstruktion der natürlichen Zahlen}

Wir konstruieren die Menge der natürlichen Zahlen allein aus den Axiomen der Mengenlehre, insbesondere dem Unendlichkeitsaxiom.  Dieser Ansatz kommt ohne eine explizite Nachfolgerfunktion als primitives Symbol aus; diese wird vielmehr aus der Mengenoperation \(x \cup \{x\}\) gewonnen.

% === Kapitel: Konstruktion der natürlichen Zahlen (umformuliert – Fokus: Funktion erster Ordnung auf A) ===

\section{Nachfolger und induktive Mengen}

\subsection{Induktive Mengen}

\FormulaDefAuto[induktive Menge]{\mathrm{Induktiv}(A) := \emptyset \in A \,\land\, \forall x\in A\,(x \cup \{x\} \in A)}[Für eine Menge \(A\) definieren wir:]

% Existenz einer induktiven Menge (aus dem Unendlichkeitsaxiom)
\FormulaThmAuto{\exists A(\mathrm{Induktiv}(A))}
\begin{tabproof}
  \proofstep{}{ \exists A\bigl(\emptyset \in A \land \forall x \in A\,(x \cup \{x\} \in A)\bigr) }{\FormulaRefAuto{\exists A\bigl(\emptyset \in A \land \forall x \in A\,(x \cup \{x\} \in A)\bigr)}}
  \proofstep{}{\exists A(\mathrm{Induktiv}(A)) }{\rIE{\FormulaRefAuto{\mathrm{Induktiv}(A) := \emptyset \in A \,\land\, \forall x\in A\,(x \cup \{x\} \in A)},1}}
\end{tabproof}

% Sofortige Konsequenzen aus der Definition
\FormulaThmAuto{\mathrm{Induktiv}(A)\vdash \emptyset\in A}
\begin{tabproof}
  \proofstep{1}{\mathrm{Induktiv}(A)}{\rA}
  \proofstep{1}{\emptyset \in A \land \forall x \in A\,(x \cup \{x\} \in A)}{\rIE{\FormulaRefAuto{\mathrm{Induktiv}(A) := \emptyset \in A \,\land\, \forall x\in A\,(x \cup \{x\} \in A)},1}}
  \proofstep{1}{\emptyset \in A}{\rAEa{2}}
\end{tabproof}

\FormulaThmAuto{\mathrm{Induktiv}(A)\vdash \forall x \in A\,(x \cup \{x\} \in A)}
\begin{tabproof}
  \proofstep{1}{\mathrm{Induktiv}(A)}{\rA}
  \proofstep{1}{\emptyset \in A \land \forall x \in A\,(x \cup \{x\} \in A)}{\rIE{\FormulaRefAuto{\mathrm{Induktiv}(A) := \emptyset \in A \,\land\, \forall x\in A\,(x \cup \{x\} \in A)},1}}
  \proofstep{1}{\forall x \in A\,(x \cup \{x\} \in A)}{\rAEb{2}}
\end{tabproof}

\FormulaThmAuto{\mathrm{Induktiv}(A),\,x\in A\vdash x \cup \{x\} \in A}
\begin{tabproof}
  \proofstep{1}{\mathrm{Induktiv}(A)}{\rA}
  \proofstep{2}{x\in A}{\rA}
  \proofstep{1}{\forall u \in A\,(u \cup \{u\} \in A)}{\FormulaRefAuto{\mathrm{Induktiv}(A)\vdash \forall x \in A\,(x \cup \{x\} \in A)}{1}}
  \proofstep{1,2}{x \cup \{x\} \in A}{\rRE{\rUE{3},2}}
\end{tabproof}

% Schnitt induktiver Mengen ist induktiv
\FormulaThmAuto{\mathrm{Induktiv}(A),\, \mathrm{Induktiv}(B)\vdash \mathrm{Induktiv}(A\cap B)}
\begin{tabproof}
  \proofstep{1}{\mathrm{Induktiv}(A)}{\rA}
  \proofstep{2}{\mathrm{Induktiv}(B)}{\rA}

  % Null in A∩B
  \proofstep{1}{\,\emptyset\in A}{\FormulaRefAuto{\mathrm{Induktiv}(A)\vdash \emptyset\in A}{1}}
  \proofstep{2}{\,\emptyset\in B}{\FormulaRefAuto{\mathrm{Induktiv}(A)\vdash \emptyset\in A}{2}}
  \proofstep{1,2}{\,\emptyset\in A\cap B}{\FormulaRefAuto{x\in A,\, x\in B \vdash x\in A\cap B}{3,4}}

  % Abschluss unter Nachfolger-Operation für A∩B
  \proofstep{1}{\forall x\in A\,(x\cup\{x\}\in A)}{\FormulaRefAuto{\mathrm{Induktiv}(A)\vdash \forall x \in A\,(x \cup \{x\} \in A)}{1}}
  \proofstep{2}{\forall x\in B\,(x\cup\{x\}\in B)}{\FormulaRefAuto{\mathrm{Induktiv}(A)\vdash \forall x \in A\,(x \cup \{x\} \in A)}{2}}
  \proofstep{8}{x\in A\cap B}{\rA}
  \proofstep{8}{x\in A}{\FormulaRefAuto{x\in A\cap B \vdash x\in A}{9}}
  \proofstep{8}{x\in B}{\FormulaRefAuto{x\in A\cap B \vdash x\in B}{9}}
  \proofstep{1,8}{x\cup\{x\}\in A}{\rRE{\rUE{6},9}}
  \proofstep{2,8}{x\cup\{x\}\in B}{\rRE{\rUE{7},10}}
  \proofstep{1,2,8}{x\cup\{x\}\in A\cap B}{\FormulaRefAuto{x\in A,\, x\in B \vdash x\in A\cap B}{11,12}}
  \proofstep{1,2}{\,\forall x\in A\cap B\,(x\cup\{x\}\in A\cap B)}{\rUI{\rRI{8,13}}}

  % Schluss: Induktivität von A∩B
  \proofstep{1,2}{\,\emptyset\in A\cap B \;\land\; \forall x\in A\cap B\,(x\cup\{x\}\in A\cap B)}{\rAI{5,14}}
  \proofstep{1,2}{\,\mathrm{Induktiv}(A\cap B)}%
    {\rIE{\FormulaRefAuto{\mathrm{Induktiv}(A) := \emptyset \in A \,\land\, \forall x\in A\,(x \cup \{x\} \in A)},15}}
\end{tabproof}

\subsection{Die Nachfolger-Funktion}

\FormulaDefAuto
{\mathrm{succ}_A := \{\, (x,y) \in A \times A \mid y = x \cup \{x\} \,\}}%
[Sei \(A\) eine Menge. Dann definieren wir:]

\FormulaThmAuto{\mathrm{succ}_A \subseteq A \times A}
\begin{tabproof}
  \proofstep{}{\mathrm{succ}_A\subseteq A\times A}{\FormulaRefAuto{\{ x \in A \mid P(x) \} \subseteq A}{\FormulaRefAuto
{\mathrm{succ}_A := \{\, (x,y) \in A \times A \mid y = x \cup \{x\} \,\}}}}
\end{tabproof}

\FormulaThmAuto[Funktionale Eindeutigkeit von \(\mathrm{succ}_A\)]%
{(x,y)\in \mathrm{succ}_A,\,(x,z)\in \mathrm{succ}_A \vdash y=z}[Seien \(x,y,z\in A\), dann gilt:]
\begin{tabproof}
  \proofstep{1}{(x,y)\in \mathrm{succ}_A}{\rA}
  \proofstep{2}{(x,z)\in \mathrm{succ}_A}{\rA}
  \proofstep{1}{y = x\cup\{x\}}{\FormulaRefAuto{x \in \{x \in A \mid P(x)\} \eqvdash x \in A \land P(x)}{\FormulaRefAuto
{\mathrm{succ}_A := \{\, (x,y) \in A \times A \mid y = x \cup \{x\} \,\}}{1}}}
  \proofstep{2}{z = x\cup\{x\}}{\FormulaRefAuto{x \in \{x \in A \mid P(x)\} \eqvdash x \in A \land P(x)}{\FormulaRefAuto
{\mathrm{succ}_A := \{\, (x,y) \in A \times A \mid y = x \cup \{x\} \,\}}{2}}}
  \proofstep{1,2}{y=z}{\rIE{4,3}}
\end{tabproof}

\FormulaThmAuto{\mathrm{Induktiv}(A), x\in A\vdash (x,x\cup \{x\})\in \mathrm{succ}_A}
\begin{tabproof}
  \proofstep{1}{ \mathrm{Induktiv}(A) }{\rA}
  \proofstep{2}{ x\in A }{\rA}
  \proofstep{1,2}{ x\cup\{x\} \in A }{\FormulaRefAuto{\mathrm{Induktiv}(A),\,x\in A\vdash x \cup \{x\} \in A}{1,2}}
  \proofstep{1,2}{ (x,x\cup\{x\}) \in A\times A}{\FormulaRefAuto{a\in A,b\in B\vdash (a,b)\in A\times B}{3}}
  \proofstep{}{ (x,x\cup\{x\})=(x,x\cup\{x\})}{\rII}
  \proofstep{1,2}{(x,x\cup\{x\}) \in \mathrm{succ}_A}{\FormulaRefAuto{x \in \{x \in A \mid P(x)\} \eqvdash x \in A \land P(x)}{4,5}}
\end{tabproof}



\FormulaThmAuto[Totalität von \(\mathrm{succ}_A\) auf induktiven Mengen]%
{\mathrm{Induktiv}(A) \vdash \forall x\in A\,\exists y\in A\;((x,y)\in \mathrm{succ}_A)}
\begin{tabproof}
  \proofstep{1}{ \mathrm{Induktiv}(A) }{\rA}
  \proofstep{2}{ x\in A }{\rA}
  \proofstep{1,2}{ x\cup\{x\} \in A }{\FormulaRefAuto{\mathrm{Induktiv}(A),\,x\in A\vdash x \cup \{x\} \in A}{1,2}}
  \proofstep{1,2}{ (x,x\cup \{x\})\in \mathrm{succ}_A }{\FormulaRefAuto{\mathrm{Induktiv}(A), x\in A\vdash (x,x\cup \{x\})\in \mathrm{succ}_A}{1,2]}}
  \proofstep{1,2}{ \exists y\in A((x,y)\in \mathrm{succ}_A) }{\rEI{\rAI{3,4}}}  
  \proofstep{1,2}{\forall x\in A\exists y\in A((x,y)\in \mathrm{succ}_A) }{\rUI{\rRI{2,5}}}  
\end{tabproof}

\begin{remark}
Diese Sätze zeigen genau die Funktionseigenschaften von \(\mathrm{succ}_A\)  und führen zu folgender Definition.
\end{remark}

\FormulaDefAuto[Nachfolger]{\mathrm{succ}(x) := x \cup \{x\}}[Sei A eine induktive Menge, dann definieren wir die Funktion \(\mathrm{succ}_A:A\mapsto A\)  für alle \(x\in A\) durch:]

\section{Konstruktion der Menge der natürlichen Zahlen}

\FormulaThmAuto{
  \exists! C\; \forall B \Bigl(\mathrm{Induktiv}(B) \rightarrow 
    C = \{ x \in B \mid \forall A (\mathrm{Induktiv}(A) \rightarrow x \in A) \}\Bigr)
}
\begin{tabproofwide}
  % Schritt 1: Existenz einer induktiven Menge
  \proofstepwidestar[]{\exists A(\mathrm{Induktiv}(A))}%
    {\FormulaRefAuto{\exists A(\mathrm{Induktiv}(A))}}

  % Schritt 2: Einzigartige Existenz des Schnitts aller induktiven Mengen
  \proofstepwide{}{}{\exists! C\; \forall B(\mathrm{Induktiv}(B)\rightarrow }%
    {\multirow{2}{*}{\FormulaRefAuto{P(D)\vdash \exists! C\forall B(P(B)\rightarrow C= \{ x \in B \mid \forall A (P(A) \rightarrow x \in A) \})}{1}}}
  \proofstepwide{}{}%
    {C=\{x\in B \mid \forall A (\mathrm{Induktiv}(A)\rightarrow x\in A)\})}%
    {}
\end{tabproofwide}


\FormulaDefAuto[Menge der natürlichen Zahlen]{\mathbb{N} := \bigcap \{ A \mid \mathrm{Induktiv}(A) \}}[Sei \(\mathrm{Induktiv}(A)\) wie oben definiert.  Wir setzen:]

\begin{remark}
  Die Definition sagt: Ein Element \(x\) gehört genau dann zu \(\mathbb{N}\), wenn es zu jeder induktiven Menge gehört.  Damit ist \(\mathbb{N}\) die \emph{kleinste} induktive Menge, denn für jede andere induktive Menge \(B\) gilt \(\mathbb{N} \subseteq B\).
\end{remark}

\FormulaThmAuto{x\in\mathbb{N}\eqvdash \forall \mathrm{Induktiv}(N)(x\in N)}
\begin{tabproofwide}
    \proofstepwide[]{x\in \mathbb{N}}{\leftrightarrow}{x\in \bigcap \{ A \mid \mathrm{Induktiv}(A) \}}{\FormulaRefAuto{\mathbb{N} := \bigcap \{ A \mid \mathrm{Induktiv}(A) \}}}
    \proofstepwide[]{}{\leftrightarrow}{\forall \mathrm{Induktiv}(N)(x\in N)}{\FormulaRefAuto{\exists A(P(A)) \vdash x \in \bigcap_{P(B)} B \leftrightarrow \forall C\, (P(C) \rightarrow x \in C)}{\FormulaRefAuto{\exists A(\mathrm{Induktiv}(A))}}}
\end{tabproofwide}

\section{Eigenschaften der Menge der natürlichen Zahlen}

Im Folgenden werden wir einige grundlegende Eigenschaften der so konstruierten Menge \(\mathbb{N}\) beweisen.

\subsection{Induktivität der Menge der natürlichen Zahlen}


\FormulaThmAuto[Null-Axiom]{\emptyset\in\mathbb{N}}
\begin{tabproof}
    \proofstep{1}{\mathrm{Induktiv}(N)}{\rA}
    \proofstep{1}{\emptyset\in N}{\FormulaRefAuto{\mathrm{Induktiv}(A)\vdash \emptyset\in A}{1}}
    \proofstep{}{\forall \mathrm{Induktiv}(N)(x\in N)}{\rUI{\rRI{1,2}}}
    \proofstep{}{\emptyset\in\mathbb{N}}{\FormulaRefAuto{x\in\mathbb{N}\eqvdash \forall \mathrm{Induktiv}(N)(x\in N)}{3}}
\end{tabproof}

\FormulaThmAuto[Nachfolger-Axiom]{\forall x \in \mathbb{N}\,(\mathrm{succ}(x) \in \mathbb{N})}
\begin{tabproof}
  % Nehme ein beliebiges x \in \mathbb{N} an und zeige succ(x) \in \mathbb{N}
  \proofstep{1}{x \in \mathbb{N}}{\rA}

  % Charakterisierung von \mathbb{N} benutzen
  \proofstep{1}{\forall \mathrm{Induktiv}(N)(x\in N)}%
    {\FormulaRefAuto{x\in\mathbb{N}\ \eqvdash\ \forall \mathrm{Induktiv}(N)(x\in N)}{1}}

  % Sei N induktiv; daraus folgt x \in N und dann succ(x) \in N
  \proofstep{3}{\mathrm{Induktiv}(N)}{\rA}
  \proofstep{1}{\mathrm{Induktiv}(N)\rightarrow x\in N}{\rUE{2}}
  \proofstep{1,3}{x\in N}{\rRE{4,3}}
  \proofstep{1,3}{\mathrm{succ}(x)\in N}{\FormulaRefAuto{\mathrm{Induktiv}(A),x\in A\vdash \mathrm{succ}(x) \in A}{3,5}}

  % Verallgemeinere auf alle induktiven N
  \proofstep{1}{\,\forall \mathrm{Induktiv}(N)\,(\mathrm{succ}(x)\in N)}{\rUI{\rRI{3,6}}}

  % Charakterisierung von \mathbb{N} rückwärts auf succ(x) anwenden
  \proofstep{1}{\,\mathrm{succ}(x)\in \mathbb{N}}%
    {\FormulaRefAuto{x\in\mathbb{N}\eqvdash \forall \mathrm{Induktiv}(N)(x\in N)}{7}}

  % Schluss: All-Intro über x in \mathbb{N}
  \proofstep{}{\,\forall x\in \mathbb{N}\,(\mathrm{succ}(x)\in \mathbb{N})}{\rUI{\rRI{1,8}}}
\end{tabproof}

\FormulaThmAuto{\mathrm{Induktiv}(\mathbb{N})}
\begin{tabproof}
  \proofstep{}{\,\emptyset\in\mathbb{N}}%
    {\FormulaRefAuto{\emptyset\in\mathbb{N}}{}}
  \proofstep{}{\,\forall x\in\mathbb{N}\,(\mathrm{succ}(x)\in\mathbb{N})}%
    {\FormulaRefAuto{\forall x \in \mathbb{N}\,(\mathrm{succ}(x) \in \mathbb{N})}{}}
  \proofstep{}{\,\emptyset\in\mathbb{N}\ \land\ \forall x\in\mathbb{N}\,(\mathrm{succ}(x)\in\mathbb{N})}%
    {\rAI{1,2}}
  \proofstep{}{\,\mathrm{Induktiv}(\mathbb{N})}%
    {\rIE{\FormulaRefAuto{\mathrm{Induktiv}(A) := \emptyset \in A \,\land\, \forall x\in A\,(\mathrm{succ}(x) \in A)},3}}
\end{tabproof}

\subsection{Minimalität der Menge der natürlichen Zahlen}

\FormulaThmAuto[Minimalität]{\mathrm{Induktiv}(A) \vdash \mathbb{N} \subseteq A}[Sei \(A\) eine Menge, dann gilt:]
\begin{tabproof}
  \proofstep{1}{ \mathrm{Induktiv}(A) }{\rA}
  \proofstep{2}{ \mathbb{N} \subseteq A }{\FormulaRefAuto{P(C)\vdash \bigcap_{P(A)} A \subseteq C}{1}}
\end{tabproof}

\section{Herleitung der Peano-Axiome}

Die folgenden Abschnitte zeigen, wie sich die Peano-Axiome aus den zuvor eingeführten Definitionen und Sätzen ergeben.

\subsection{Das Null-Axiom}

Aus \FormulaRefAuto{\emptyset\in\mathbb{N}} folgt, dass die Null in den natürlichen Zahlen enthalten ist.

\subsection{Nachfolger‑Axiom}

Aus \FormulaRefAuto{\forall x \in \mathbb{N}\,(\mathrm{succ}(x) \in \mathbb{N})} ergibt sich, dass die Menge der natürlichen Zahlen mit der Nachfolgerfunktion abgeschlossen ist.

\subsection{Kein Vorgänger von Null}

\FormulaThmAuto{\mathrm{succ}(x) \neq \emptyset}
\begin{tabproof}
  \proofstep{}{ x \cup \{x\}\neq\emptyset }{\FormulaRefAuto{A\cup \{a\}\neq\emptyset}}
  \proofstep{}{\mathrm{succ}(x)\neq\emptyset}{\rIE{\FormulaRefAuto{\mathrm{succ}(x) := x \cup \{x\}},2}}
\end{tabproof}

\subsection{Injektivität der Nachfolgerfunktion}

\FormulaThmAuto{\mathrm{succ}(x) = \mathrm{succ}(y) \vdash x = y}

\begin{tabproof}
  \proofstep{1}{ \mathrm{succ}(x) = \mathrm{succ}(y) }{\rA}
  \proofstep{1}{ x \cup \{x\} = y \cup \{y\} }{\rIE{\FormulaRefAuto{\mathrm{succ}(x) := x \cup \{x\}},1}}
  \proofstep{1}{ x = y }{\FormulaRefAuto{a\cup\{a\}=b\cup\{b\}\eqvdash a=b}{2}}
\end{tabproof}

\subsection{Induktionsprinzip}

\FormulaThmAuto{P(\emptyset),\, \forall x\in\mathbb{N}\,(P(x) \rightarrow P(\mathrm{succ}(x))) \vdash \mathrm{Induktiv}(\{x \in \mathbb{N} \mid P(x)\})}[Für jedes einstellige Prädikat \(P\) gilt:]
\begin{notation*}
    Im Beweis bezeichnen wir mit \(S := \{x \in \mathbb{N} \mid P(x)\}\).
\end{notation*}
\begin{tabproof}
  \proofstep{1}{ P(\emptyset) }{\rA}
  \proofstep{2}{ \forall x \in \mathbb{N}\,(P(x) \rightarrow P(\mathrm{succ}(x))) }{\rA}
  \proofstep{1}{\emptyset\in S}{\FormulaRefAuto{x \in \{x \in A \mid P(x)\} \eqvdash x \in A \land P(x)}{\FormulaRefAuto{\emptyset\in\mathbb{N}},1}}
  \proofstep{4}{n\in S}{\rA}
  \proofstep{4}{n\in \mathbb{N}}{\rAEa{\FormulaRefAuto{x \in \{x \in A \mid P(x)\} \eqvdash x \in A \land P(x)}{4}}}
  \proofstep{4}{P(n)}{\rAEb{\FormulaRefAuto{x \in \{x \in A \mid P(x)\} \eqvdash x \in A \land P(x)}{4}}}
  \proofstep{4}{\mathrm{succ}(n)\in\mathbb{N}}{\rRE{\rUE{\FormulaRefAuto{\forall x \in \mathbb{N}\,(\mathrm{succ}(x) \in \mathbb{N})}},5}}
  \proofstep{2,4}{P(\mathrm{succ}(n))}{\rRE{\rRE{\rUE{2},5},6}}
  \proofstep{2,4}{\mathrm{succ}(n)\in S}{\FormulaRefAuto{x \in \{x \in A \mid P(x)\} \eqvdash x \in A \land P(x)}{7,8}}
  \proofstep{2}{\forall n\in S(\mathrm{succ}(n)\in S)}{\rUI{\rRI{4,9}}}
  \proofstep{1,2}{ \mathrm{Induktiv}(S) }{\FormulaRefAuto{\mathrm{Induktiv}(A) := \emptyset \in A \,\land\, \forall x\in A\,(\mathrm{succ}(x) \in A)}{\rAI{3,10}}}
\end{tabproof}

\FormulaThmAuto[Induktion auf \(\mathbb{N}\)]{P(\emptyset),\, \forall x\in\mathbb{N}\,(P(x) \rightarrow P(\mathrm{succ}(x))) \vdash \forall x\in\mathbb{N}\,(P(x))}[Für jedes einstellige Prädikat \(P\) gilt:]
\begin{tabproof}
  \proofstep{1}{ P(\emptyset) }{\rA}
  \proofstep{2}{ \forall x \in \mathbb{N}\,(P(x) \rightarrow P(\mathrm{succ}(x))) }{\rA}
  \proofstep{1,2}{ \mathrm{Induktiv}(\{x \in \mathbb{N} \mid P(x)\}) }{\FormulaRefAuto{P(\emptyset),\, \forall x\in\mathbb{N}\,(P(x) \rightarrow P(\mathrm{succ}(x))) \vdash \mathrm{Induktiv}(\{x \in \mathbb{N} \mid P(x)\})}{1,2}}
  \proofstep{}{ \mathbb{N} \subseteq \{x \in \mathbb{N} \mid P(x)\} }{\FormulaRefAuto{\mathrm{Induktiv}(A) \vdash \mathbb{N} \subseteq A}{3}}
  \proofstep{1,2}{ \forall x\in\mathbb{N}\,(P(x)) }{\FormulaRefAuto{A\subseteq \{x\in B\mid P(x)\}\vdash \forall x\in A(P(x))}{4}}
\end{tabproof}


\end{document}

