%============================================================
%  Bd. 03 - Mengenlehre %============================================================

\documentclass[main.tex]{subfiles}


\ifSubfilesClassLoaded{
    \usepackage{xr}
    \externaldocument{_B01}
  \externaldocument{_B02}
}{
   % Code für als Subfile eingebunden
}

\title{Bd. 03 - Mengenlehre}
\author{Martin Kunze}
\date{}
\setcounter{file}{3}




\begin{document}

\maketitle
\tableofcontents
%\listoftheorems



\chapter{Einführung}

Die Mengenlehre ist ein fundamentaler Teil der Mathematik, der die Grundlage für viele andere Bereiche bildet. In diesem Kapitel werden wir die Zermelo-Fraenkel (ZF) Axiome der Mengenlehre einführen und diskutieren. Dabei bezeichnen \( A \), \( B \), \( C \) und \( D \) stets Mengen, es sei denn, es wird ausdrücklich etwas anderes angegeben. Alle Variablen, die als Mengen bezeichnet werden, sind implizit durch Allquantoren gebunden, es sei denn, es wird ein anderer Quantor verwendet. Das bedeutet, dass Aussagen wie „\( A = B \)“ oder „\( A \neq B \)“ für alle Mengen \( A \) und \( B \) gelten, ohne dass dies explizit angegeben werden muss.

\chapter{Die Zermelo-Fraenkel-Axiome}

Nachdem wir die grundlegenden Begriffe und Notationen eingeführt haben, wenden wir uns nun den Zermelo-Fraenkel-Axiomen zu, die das Fundament der modernen Mengenlehre bilden. Diese Axiome definieren, wie Mengen gebildet werden können und welche Eigenschaften sie besitzen.

\FormulaDefDeltaK[Begriff der Menge (Zermelo-Fraenkel-Axiome)]{\in \text{(zweistelliges Prädikat)}}{Menge}{
  \DeltaRow{\textbf{Axiome}}{}
  % — Notations-/Meta-Zeilen —
  % — Extensionalität —
  \DeltaRow{Extensionalität}
           {\forall x\, (x \in A \leftrightarrow x \in B) \vdash A = B}
           [\FormulaRefAuto{\forall x\, (x \in A \leftrightarrow x \in B) \vdash A = B}]
  %
  % — Leere Menge —
  \DeltaRow{Leere Menge}
           {\exists O\;\bigl(\forall x\,(x \not\in O)\bigr)}
           [\FormulaRefAuto{\exists O\;\bigl(\forall x\,(x \not\in O)\bigr)}]
  %
  % — Aussonderung (Schema) —
  \DeltaRow{Aussonderung}
           {\exists B\;\forall x\;\bigl(x \in B \;\leftrightarrow\; x \in A \,\land\, P(x)\bigr)}
           [\FormulaRefAuto{\exists B\;\forall x\;\bigl(x \in B \;\leftrightarrow\; x \in A \,\land\, P(x)\bigr)}]
  %
  % — Paarmenge —
  \DeltaRow{Paarmenge}
           {\exists C\;\bigl(\forall x\,(x \in C \leftrightarrow x=A \lor x=B)\bigr) }
           [\FormulaRefAuto{\exists C\;\bigl(\forall x\,(x \in C \leftrightarrow x=A \lor x=B)\bigr)}]
  %
  % — Vereinigung —
  \DeltaRow{Vereinigung}
           {\exists C\;\forall x\;\bigl(x \in C \;\leftrightarrow\;\exists B\,(B \in A \,\land\, x \in B)\bigr)}
           [\FormulaRefAuto{\exists C\;\forall x\;\bigl(x \in C \;\leftrightarrow\;\exists B\,(B \in A \,\land\, x \in B)\bigr)}]
  %
  % — Regularität (Fundierung) —
  \DeltaRow{Regularität}
           {A \neq \emptyset \vdash \exists x \in A \,(x \cap A = \emptyset)}
           [\FormulaRefAuto{A \neq \emptyset \vdash \exists x \in A \,(x \cap A = \emptyset)}]
  %
  % — Potenzmenge —
  \DeltaRow{Potenzmenge}
           {\exists B\forall x\bigl(x \in B \leftrightarrow x \subseteq A\bigr)}
           [\FormulaRefAuto{\exists B\forall x\bigl(x \in B \leftrightarrow x \subseteq A\bigr)}]
  %
  % — Ersetzung (Schema) —
  \DeltaRow{Ersetzung}
           {\exists B\;\forall y\;\bigl( y\in B\;\leftrightarrow\; \exists x\in A\;y=F(x) \bigr)}
           [\FormulaRefAuto{\exists B\;\forall y\;\bigl( y\in B\;\leftrightarrow\; \exists x\in A\;y=F(x) \bigr)}]
  %
  % — Auswahlaxiom (eine Form) —
  \DeltaRow{Auswahlaxiom}
           {\exists F\colon A \to B \, (G \circ F = \Id_A)}
           [\FormulaRefAuto{\exists F\colon A \to B \, (G \circ F = \Id_B)}]
  %
  % — Unendlichkeit —
  \DeltaRow{Unendlichkeit}
           {\exists A\,\bigl(\emptyset\in A \land \forall x\in A\,(\,x\cup\{x\}\in A\,)\bigr)}
           [\FormulaRefAuto{\exists A\,(\emptyset \in A \land \forall x \in A\,(x \cup \{x\} \in A))}]
%
    \DeltaRow{\textbf{Neue Symbole}}{}
    \DeltaRow{Mengen}{A\dsep B\dsep C\dsep O\dsep x}
    %%begin novalidate
    \DeltaRow{Leere Menge}{\emptyset}[\FormulaRefAuto{\emptyset := \iota O\bigl(\forall x\,(x \not\in O)\bigr)}]
    %%end novalidate
    \DeltaRow{Funktionen}{F\colon A\to B}[\FormulaRefAuto{Funktion}]
    \DeltaRow{Surjektive Funktionen}{G\colon B\sur A}[\FormulaRefAuto{Surjektive Funktion}]
    \DeltaRow{Identität}{\Id_A\colon A\to A}[\FormulaRefAuto{Identität}]
    \DeltaRow{Element}{ }
}

\chapter{Extensionalität}

\section{Axiom der Extensionalität}

\FormulaAxiomDelta[Extensionalität]{\forall x\, (x \in A \leftrightarrow x \in B) \vdash A = B}%
{%
\DeltaRow{Mengen}{A\dsep B}% 
}%

\FormulaThmDelta[Extensionalität]{\forall x\, (x \in A \leftrightarrow x \in B) \eqvdash A = B}%
{%
\DeltaRow{Mengen}{A\dsep B}% 
}%
\begin{tabproofsplit}
    \proofpart{\(\vdash\)}
        \proofstep{}{\forall x\, (x \in A \leftrightarrow x \in B) \rightarrow A = B}{\FormulaRefAuto{\forall x\, (x \in A \leftrightarrow x \in B) \vdash A = B}}
    \closeproofpart
    \proofpart{\(\dashv\)}
        \proofstep{1}{A=B}{\rA}
        \proofstep{2}{x\in A}{\rA}
        \proofstep{1,2}{x\in B}{\rIE{1,2}}
        \proofstep{1}{x\in A\rightarrow x\in B}{\rRI{2,3}}
        \proofstep{5}{x\in B}{\rA}
        \proofstep{1,5}{x\in A}{\rIE{1,5}}
        \proofstep{1}{x\in B\rightarrow x\in A}{\rRI{5,6}}
        \proofstep{1}{x\in A\leftrightarrow x\in A}{\rLRI{4,7}}
        \proofstep{1}{\forall x\, (x\in A\leftrightarrow x\in A)}{\rUI{8}}
    \closeproofpart
\end{tabproofsplit}

\section{Ungleichheit von Mengen}

\FormulaThmAuto{
A \neq B \eqvdash \exists x (x \not\in A\land x\in B) \lor  \exists x (x \in A\land x\not\in B)
}
\begin{tabproofwide}
  \proofstepwide{A \neq B}{\leftrightarrow}{\neg(\forall x(x\in A\leftrightarrow x\in B))}{\FormulaRefAuto{P \leftrightarrow Q \dashv \vdash \neg P \leftrightarrow \neg Q}{\FormulaRefAuto{\forall x\, (x \in A \leftrightarrow x \in B) \eqvdash A = B}}}
  \proofstepwide{}{\leftrightarrow}{\exists x(x \not\in A\land x\in B)}{\multirow{2}{*}{\FormulaRefAuto{\neg\forall x(P(x)\leftrightarrow Q(x)) \eqvdash \exists x (P(x)\land \neg Q(x))\lor \exists x (Q(x)\land \neg P(x))}{1}}}
  \proofstepwide*{}{\lor}{\exists x(x \in A\land x\not\in B)}{}
  \proofstepwide{A \neq B}{\leftrightarrow}{\exists x(x \not\in A\land x\in B)}{\multirow{2}{*}{\rChain{1,2}}}
  \proofstepwide*{}{\lor}{\exists x(x \in A\land x\not\in B)}{}
\end{tabproofwide}

\FormulaThmAuto{x\in A, x\not\in B\vdash A\neq B}
\begin{tabproof}
  \proofstep{1}{x\in A}{\rA}
  \proofstep{2}{x\not\in B}{\rA}
  \proofstep{1}{\exists x(x\in A\land x\not\in B)}{\rEI{\rAI{1,2}}}
  \proofstep{1}{A\neq B}{\FormulaRefAuto{A \neq B \eqvdash \exists x (x \not\in A\land x\in B) \lor \exists x (x \in A\land x\not\in B)}{\rOIa{3}}}
\end{tabproof}


\chapter{Teilmengen}

\section{Definition der Teilmenge}

\FormulaDefDelta[Teilmenge]{ A \subseteq B := \forall x\,(x\in A \rightarrow x\in B) }%
{%
\DeltaRow{Mengen}{A\dsep B}%
}
\begin{remark}
    In Worten: \(A\) ist Teilmenge von \(B\).
\end{remark}

\section{Grundlegende Eigenschaften}

\FormulaThmAuto{ A\subseteq B,\, x\in A \vdash x\in B }
\begin{tabproof}
  \proofstep{1}{ A\subseteq B }{ \rA }
  \proofstep{2}{ x\in A }{ \rA }
  \proofstep{1}{ x\in A \rightarrow x\in B }{ \rUE{ \FormulaRefAuto{ A \subseteq B := \forall x\,(x\in A \rightarrow x\in B) }{1} } }
  \proofstep{1,2}{ x\in B }{ \rRE{2,3} }
\end{tabproof}

\FormulaThmAuto{ A = B,\, x \in A \vdash x \in B }
\begin{tabproof}
  \proofstep{1}{ A = B }{ \rA }
  \proofstep{2}{ x \in A }{ \rA }
  \proofstep{1}{ x \in A \leftrightarrow x \in B }{ \rUE{ \FormulaRefAuto{ \forall x\, (x \in A \leftrightarrow x \in B) \eqvdash A = B } } }
  \proofstep{1,2}{ x \in B }{ \FormulaRefAuto{ P \leftrightarrow Q,\; P \vdash Q }{3,2} }
\end{tabproof}

\FormulaThmAuto{ a \in A,\; b \not\in A \vdash a \neq b }
\begin{tabproof}
  \proofstep{1}{ a \in A }{ \rA }
  \proofstep{2}{ b \not\in A }{ \rA }
  \proofstep{3}{ a = b }{ \rA }
  \proofstep{1,3}{ b \in A }{ \rIE{3,1} }
  \proofstep{1,2,3}{ \bot }{ \rAI{4,2} }
  \proofstep{1,2}{ a \neq b }{ \rCI{3,5} }
\end{tabproof}

\FormulaThmAuto{A\subseteq C,\, B\subseteq C,\, z\in A\lor z\in B \vdash z\in C}
\begin{tabproofwide}
  \proofstepwidestar[1]{A \subseteq C}{\rA}
  \proofstepwidestar[2]{B \subseteq C}{\rA}
  \proofstepwidestar[3]{z \in A \lor z \in B}{\rA}

  \proofstepwide[1]{z \in A}{\rightarrow}{z \in C}%
    {\rUE{\FormulaRefAuto{A \subseteq B := \forall x\,(x\in A \rightarrow x\in B)}{1}}}
  \proofstepwide[2]{z \in B}{\rightarrow}{z \in C}%
    {\rUE{\FormulaRefAuto{A \subseteq B := \forall x\,(x\in A \rightarrow x\in B)}{2}}}

  \proofstepwidestar[1,2,3]{z \in C}%
    {\FormulaRefAuto{P \rightarrow Q,\, R \rightarrow Q,\, P \lor R \vdash Q}{4,5,3}}
\end{tabproofwide}
\section{Ordnungsrelation}

\FormulaThmAuto[Reflexivität von Teilmengen]{ A \subseteq A }
\begin{tabproof}
  \proofstep{}{ x \in A \rightarrow x \in A }{ \FormulaRefAuto{ P \rightarrow P } }
  \proofstep{}{ \forall x(x \in A \rightarrow x \in A) }{ \rUI{1} }
  \proofstep{}{ A \subseteq A }{ \FormulaRefAuto{ A \subseteq B := \forall x\,(x \in A \rightarrow x \in B) }{2} }
\end{tabproof}


\FormulaThmAuto[Antisymmetrie von Teilmengen]{ A \subseteq B \land B \subseteq A \eqvdash A = B }
\begin{tabproofwide}
  \proofstepwide{A \subseteq B \land B \subseteq A}{\leftrightarrow}{\forall x(x \in A \rightarrow x \in B)}{\multirow{2}{*}{\FormulaRefAuto{ A \subseteq B := \forall x\,(x \in A \rightarrow x \in B) }}}
  \proofstepwide{}{ \land }{\forall x(x \in B \rightarrow x \in A)}{}
  \proofstepwide*{}{ \leftrightarrow }{\forall x(x \in A \leftrightarrow x \in B)}{\FormulaRefAuto{\forall x (P(x) \leftrightarrow Q(x)) \dashv \vdash \forall x (P(x) \rightarrow Q(x)) \land \forall x (Q(x) \rightarrow P(x))}{1}}
  \proofstepwide{}{ \leftrightarrow }{A = B}{\FormulaRefAuto{ \forall x\, (x \in A \leftrightarrow x \in B) \eqvdash A = B }{2}}
  \proofstepwide*{A \subseteq B \land B \subseteq A}{ \leftrightarrow }{A = B}{\rChain{1,3}}
\end{tabproofwide}

\FormulaThmAuto{ A \subseteq B, B \subseteq A \vdash A = B }
\begin{tabproof}
    \proofstep{1}{A\subseteq B}{\rA}
    \proofstep{2}{B\subseteq A}{\rA}
    \proofstep{1,2}{A\subseteq B\land B\subseteq A}{\rAI{1,2}}
    \proofstep{1,2}{A\subseteq B\land B\subseteq A}{\FormulaRefAuto{ A \subseteq B \land B \subseteq A \eqvdash A = B }{3}}
\end{tabproof}

\FormulaThmAuto[Transitivität von Teilmengen]{
A \subseteq B, B \subseteq C \vdash A \subseteq C
}
\begin{tabproof}
  \proofstep{1}{A \subseteq B}{\rA}
  \proofstep{2}{B \subseteq C}{\rA}
  \proofstep{1}{\forall x(x \in A \rightarrow x \in B)}{\FormulaRefAuto{A \subseteq B := \forall x(x \in A \rightarrow x \in B)}{1}}
  \proofstep{1}{\forall x(x \in B \rightarrow x \in C)}{\FormulaRefAuto{A \subseteq B := \forall x(x \in A \rightarrow x \in B)}{2}}
  \proofstep{1,2}{\forall x(x \in A \rightarrow x \in C)}{\FormulaRefAuto{\forall x(P(x) \rightarrow Q(x)), \forall x(Q(x) \rightarrow R(x)) \vdash \forall x(P(x) \rightarrow R(x))}{3,4}}
  \proofstep{1,2}{A \subseteq C}{\FormulaRefAuto{A \subseteq B := \forall x(x \in A \rightarrow x \in B)}{5}}
\end{tabproof}

\FormulaThmAuto[Rechtsverträglichkeit von \(\subseteq\) und \(=\)]{
A \subseteq B, B = C \vdash A \subseteq C
}
\begin{tabproof}
  \proofstep{1}{A \subseteq B}{\rA}
  \proofstep{2}{B = C}{\rA}
  \proofstep{3}{B \subseteq C}{\rAEa{\FormulaRefAuto{A \subseteq B \land B \subseteq A \eqvdash A = B}{2}}}
  \proofstep{4}{A \subseteq C}{\FormulaRefAuto{A \subseteq B, B \subseteq C \vdash A \subseteq C}{1,3}}
\end{tabproof}

\begin{remark}[Gemischte Kettenregel]
Auf Basis des vorangegangenen Theorems können \(\subseteq\) und \(=\) nun in einer \emph{Kette} \((\subseteq,=^{*})\) kombiniert werden,  da sie \emph{rechts-verträglich} sind. Ebenso ist \(=\) wegen \FormulaRefAuto{a = b \vdash b = a}{} außerdem Symmetrisch, was mit dem Stern in der Kette illustriert wird.
\end{remark}



\chapter{Leere Menge}
\FormulaAxiomAuto[Leere Menge]{\exists O\;\bigl(\forall x\,(x \not\in O)\bigr)}
%%begin novalidate
\section{Axiom der leeren Menge}
\FormulaDefAuto[Leere Menge]{\emptyset := \iota O\bigl(\forall x\,(x \not\in O)\bigr)}
%%end novalidate

\section{Definition der leeren Menge}

\FormulaThmAuto{ \exists! O\forall x (x \not\in O) }
\begin{tabproof}
  \proofstep{}{ \exists O\forall x (x \not\in O) }{ \FormulaRefAuto{\exists O\;\bigl(\forall x\,(x \not\in O)\bigr)} }
  \proofstep{2}{ \forall x (x \not\in O) }{ \rA }
  \proofstep{3}{ \forall x (x \not\in P) }{ \rA }
  \proofstep{2}{ \forall x (x \not\in O \lor x \in P) }{ \FormulaRefAuto{\forall x(F(x))\lor\forall x(G(x))\vdash\forall x(F(x)\lor G(x))} }
  \proofstep{3}{ \forall x (x \not\in P \lor x \in O) }{ \FormulaRefAuto{ \forall x(F(x))\lor\forall x(G(x))\vdash\forall x(F(x)\lor G(x))} }
  \proofstep{2,3}{ \forall x (x \in O \leftrightarrow x \in P) }{ \FormulaRefAuto{ \forall x (P(x) \leftrightarrow Q(x)) \dashv \vdash \forall x (\neg P(x) \lor Q(x)) \land \forall x (\neg Q(x) \lor P(x)) }{4,5} }
  \proofstep{2,3}{ O = P }{ \FormulaRefAuto{ \forall x\, (x \in A \leftrightarrow x \in B) \eqvdash A = B } }
  \proofstep{}{ \exists! O (\forall x (x \not\in O)) }{ \UEI{1,2,3,6} }
\end{tabproof}

\FormulaThmDelta[Leere Menge]{x \not\in \emptyset}%
{%
\DeltaRow{Mengen}{x}% 
}
\begin{tabproof}
  \proofstep{}{ \forall x\,\bigl(x \not\in O\bigr) }{\FormulaRefAuto{\emptyset := \iota O\bigl(\forall x\,(x \not\in O)\bigr)}}
  \proofstep{}{ x \not\in O }{\rUE{1}}
\end{tabproof}

\FormulaThmDelta{\forall x\, x\notin A\vdash A=\emptyset}{
\DeltaRow{Mengen}{x\dsep A}
}
\begin{tabproof}
    \proofstep{1}{\forall x\, x\notin A}{\rA}
    \proofstep{}{\forall x\, x\notin \emptyset}{\FormulaRefAuto{\exists! O\forall x (x \not\in O)}}
    \proofstep{}{\exists! O\forall x (x \not\in O)}{\FormulaRefAuto{\exists! O\forall x (x \not\in O)}}
    
\end{tabproof}

\section{Grundlegende Eigenschaften}


\FormulaThmAuto{\forall A\,(\emptyset\subseteq A)}
\begin{tabproof}
  \proofstep{}{x \not\in \emptyset}{\FormulaRefAuto{\emptyset := \iota O\bigl(\forall x\,(x \not\in O)\bigr)}{}}
  \proofstep{}{x \not\in A \rightarrow x \not\in \emptyset}{\FormulaRefAuto{Q \vdash P \rightarrow Q}{}}
  \proofstep{}{x \in \emptyset \rightarrow x \in A}{\FormulaRefAuto{P \rightarrow Q \eqvdash \neg Q \rightarrow \neg P}{2}}
  \proofstep{}{\forall x(x \in \emptyset \rightarrow x \in A)}{\rUI{3}}
  \proofstep{}{\emptyset \subseteq A}{\FormulaRefAuto{A \subseteq B := \forall x\,(x\in A \rightarrow x\in B)}{4}}
  \proofstep{}{\forall A(\emptyset \subseteq A)}{\rUI{5}}
\end{tabproof}

\FormulaThmAuto{
A \subseteq B,\, \forall x \in B\,(x \not\in A) \vdash A = \emptyset
}
\begin{tabproof}
  \proofstep{1}{A \subseteq B}{\rA}
  \proofstep{2}{\forall x \in B\,(x \not\in A)}{\rA}
  \proofstep{3}{x \in A}{\rA}
  \proofstep{1,3}{x \in B}{\FormulaRefAuto{A \subseteq B,\, x \in A \vdash x \in B}{1,3}}
  \proofstep{2}{x \in B \rightarrow x \not\in A}{\rUE{2}}
  \proofstep{2}{x \in A \rightarrow x \not\in B}{\FormulaRefAuto{P \rightarrow Q \eqvdash \neg Q \rightarrow \neg P}{5}}
  \proofstep{2,3}{x \not\in B}{\rRE{3,6}}
  \proofstep{1,2,3}{\bot}{\rBI{4,7}}
  \proofstep{1,2}{x \not\in A}{\rCI{3,8}}
  \proofstep{1,2}{\forall x\,(x \not\in A)}{\rUI{9}}
  \proofstep{1,2}{A = \emptyset}{\FormulaRefAuto{\emptyset := \iota O\,(\forall x\,(x \not\in O))}{10}}
\end{tabproof}

\FormulaThmAuto{a \in A \vdash A \neq \emptyset}
\begin{tabproof}
  \proofstep{1}{a \in A}{\rA}
  \proofstep{}{a \not\in \emptyset}{\FormulaRefAuto{\emptyset := \iota O\,(\forall x\,(x \not\in O))}}
  \proofstep{1}{A \neq \emptyset}{\FormulaRefAuto{x \in A,\, x \not\in B \vdash A \neq B}{1,2}}
\end{tabproof}

\FormulaThmAuto{\exists x\,(x \in S) \vdash S \neq \emptyset}
\begin{tabproof}
  \proofstep{1}{\exists x\,(x \in S)}{\rA}
  \proofstep{2}{a \in S}{\rA}
  \proofstep{2}{S \neq \emptyset}{\FormulaRefAuto{a \in A \vdash A \neq \emptyset}{2}}
  \proofstep{1}{S \neq \emptyset}{\rEE{1,2,3}}
\end{tabproof}

\FormulaThmAuto{A\neq\emptyset \eqvdash \exists x\,(x \in A)}
\begin{tabproofsplit}
    \proofpart{\(\vdash\)}
        \proofstep{1}{A\neq\emptyset}{\rA}
        \proofstep{2}{\forall x\, x\notin A}{\rA}
        \proofstep{2}{A=\emptyset}{???}
        \proofstep{1,2}{\bot}{\rBI{1,3}}
        \proofstep{1}{\neg\forall x\, x\notin A}{\rCI{2,4}}
        \proofstep{1}{\exists x\, x\in A}{\FormulaRefAuto{\forall x(\neg P(x)) \eqvdash \neg\exists x (P(x))}{4}}
    \closeproofpart
    \proofpart{\(\dashv\)}
        \proofstep{1}{\exists x\,(x \in A)}{\rA}
        \proofstep{2}{a \in A}{\rA}
        \proofstep{2}{A \neq \emptyset}{\FormulaRefAuto{a \in A \vdash A \neq \emptyset}{2}}
        \proofstep{1}{A \neq \emptyset}{\rEE{1,2,3}}
    \closeproofpart
\end{tabproofsplit}
    

\chapter{Aussonderung}

\section{Axiom der Aussonderung}

\FormulaAxiomDelta[Aussonderung]{\exists B\;\forall x\;\bigl(x \in B \;\leftrightarrow\; x \in A \,\land\, P(x)\bigr)}%
{%
\DeltaRow{Mengen}{A}% 
\DeltaRow{Einstellige Prädikate}{P}%
}%

\section{Definition der ausgesonderten Menge}

\FormulaThmDelta[Zur Eindeutigkeit]{ \forall x (x \in A \leftrightarrow P(x)), \forall x (x \in B \leftrightarrow P(x)) \vdash A = B }{%
\DeltaRow{Mengen}{A\dsep B}% 
\DeltaRow{Einstellige Prädikate}{P}%
}
\begin{tabproof}
  \proofstep{1}{ \forall x (x \in A \leftrightarrow P(x)) }{ \rA }
  \proofstep{2}{ \forall x (x \in B \leftrightarrow P(x)) }{ \rA }
  \proofstep{1,2}{ \forall x (x \in A \leftrightarrow x \in C) }{ \FormulaRefAuto{ \forall x(P(x)\leftrightarrow Q(x)), \forall x(R(x)\leftrightarrow Q(x))\vdash \forall x(P(x)\leftrightarrow R(x)) }{1,2} }
  \proofstep{1,2}{ A = B }{ \FormulaRefAuto{ \forall x\, (x \in A \leftrightarrow x \in B) \eqvdash A = B }{3} }
\end{tabproof}

\FormulaThmDelta[Eindeutigkeit der Komprehension]{ \exists A(\forall x(x \in A\leftrightarrow P(x)))\vdash \exists! A(\forall x(x \in A\leftrightarrow P(x))) }{%
\DeltaRow{Mengen}{A}% 
\DeltaRow{Einstellige Prädikate}{P}%
}
\begin{tabproof}
  \proofstep{1}{ \exists A(\forall x(x \in A\leftrightarrow P(x))) }{ \rA }
  \proofstep{2}{ \forall x(x \in A\leftrightarrow P(x)) }{ \rA }
  \proofstep{3}{ \forall x(x \in B\leftrightarrow P(x)) }{ \rA }
  \proofstep{2,3}{ A = B }{ \FormulaRefAuto{ \forall x (x \in A \leftrightarrow P(x)), \forall x (x \in B \leftrightarrow P(x)) \vdash A = B}{2,3} }
  \proofstep{1}{ \exists! A(\forall x(x \in A\leftrightarrow P(x))) }{ \UEI{1,2,3,4} }
\end{tabproof}

\FormulaThmDelta{ \forall x(P(x)\rightarrow x\in A)\vdash \exists! B(\forall x(x\in B\leftrightarrow P(x))) }{%
\DeltaRow{Mengen}{A\dsep B}% 
\DeltaRow{Einstellige Prädikate}{P}%
}
\begin{tabproof}
  \proofstep{1}{ \forall x (P(x) \rightarrow x \in A) }{ \rA }
  \proofstep{1}{ \forall x(x \in A \land P(x) \leftrightarrow P(x)) }{ \FormulaRefAuto{ \forall x(P(x)\rightarrow Q(x))\vdash \forall x((Q(x)\land P(x))\leftrightarrow P(x)) } }
  \proofstep{1}{ \exists B(\forall x(x \in B \leftrightarrow P(x))) }{ \rEI{ \FormulaRefAuto{ \{x\in A \,\mid\, P(x)\} := \iota B\bigl(\forall x\,(x\in B \leftrightarrow (x\in A \land P(x)))\bigr) } } }
  \proofstep{1}{ \exists! B(\forall x(x \in B \leftrightarrow P(x))) }{ \FormulaRefAuto{ \exists A(\forall x(x \in A\leftrightarrow P(x)))\vdash \exists! A(\forall x(x \in A\leftrightarrow P(x))) } }
\end{tabproof}

%%begin novalidate
\FormulaDefDelta[Aussonderung]{\{x\in A \,\mid\, P(x)\} := \iota B\bigl(\forall x\,(x\in B \leftrightarrow (x\in A \land P(x)))\bigr)}%
{%
\DeltaRow{Mengen}{A}% 
\DeltaRow{Einstellige Prädikate}{P}%
}%
%%end novalidate

\FormulaThmDelta[Aussonderung]{x \in \{x \in A \mid P(x)\} \eqvdash x \in A \land P(x)}%
{%
\DeltaRow{Mengen}{A}% 
\DeltaRow{Einstellige Prädikate}{P}%
}%

\FormulaThmDelta{x \in A, P(x)\vdash x \in \{x \in A \mid P(x)\}}%
{%
\DeltaRow{Mengen}{A\dsep x}% 
\DeltaRow{Einstellige Prädikate}{P}%
}
\begin{tabproof}
    \proofstep{1}{ x \in A }{ \rA }
    \proofstep{2}{ P(x) }{ \rA }
    \proofstep{1,2}{ x \in A\land P(x) }{ \rAI{1,2} }
    \proofstep{1,2}{ x \in \{x \in A \mid P(x)\} }{ \FormulaRefAuto{x \in \{x \in A \mid P(x)\} \eqvdash x \in A \land P(x)}{3} }
\end{tabproof}

\FormulaThmDelta{x \in A\dsep P(x)\dsep B=\{x \in A \mid P(x)\}\vdash x \in B}
{%
\DeltaRow{Mengen}{A\dsep B\dsep x}% 
\DeltaRow{Einstellige Prädikate}{P}%
}
\begin{tabproof}
    \proofstep{1}{ x \in A }{ \rA }
    \proofstep{2}{ P(x) }{ \rA }
    \proofstep{3}{ B=\{x \in A \mid P(x)\} }{ \rA }
    \proofstep{1,2}{ x \in \{x \in A \mid P(x)\} }{ \FormulaRefAuto{x \in A, P(x)\vdash x \in \{x \in A \mid P(x)\}}{1,2} }
    \proofstep{1,2,3}{ x \in B }{ \rIE{3,4}}
\end{tabproof}

\section{Grundlegende Eigenschaften}

\FormulaThmAuto{x \in \{x \in A \mid P(x)\}\vdash x\in A}
\begin{tabproof}
  \proofstep{1}{x \in \{x \in A \mid P(x)\}}{\rA}
  \proofstep{1}{x\in A\land P(x)}{\FormulaRefAuto{x \in \{x \in A \mid P(x)\} \eqvdash x \in A \land P(x)}{1}}
  \proofstep{1}{P(x)}{\rAEa{2}}
\end{tabproof}

\FormulaThmAuto{x \in A\dsep A=\{x \in B \mid P(x)\}\vdash x\in B}
\begin{tabproof}
  \proofstep{1}{x \in A}{\rA}
  \proofstep{2}{A=\{x \in B \mid P(x)\}}{\rA}
  \proofstep{1,2}{x\in \{x \in B \mid P(x)\}}{\rIE{2,1}}
  \proofstep{1,2}{x\in B}{\FormulaRefAuto{x \in \{x \in A \mid P(x)\}\vdash x\in A}{3}}
\end{tabproof}

\FormulaThmAuto{x \in \{x \in A \mid P(x)\}\vdash P(x)}
\begin{tabproof}
  \proofstep{1}{x \in \{x \in A \mid P(x)\}}{\rA}
  \proofstep{1}{x\in A\land P(x)}{\FormulaRefAuto{x \in \{x \in A \mid P(x)\} \eqvdash x \in A \land P(x)}{1}}
  \proofstep{1}{P(x)}{\rAEb{2}}
\end{tabproof}

\FormulaThmAuto{x \in A\dsep A=\{x \in B \mid P(x)\}\vdash P(x)}
\begin{tabproof}
  \proofstep{1}{x \in A}{\rA}
  \proofstep{2}{A=\{x \in B \mid P(x)\}}{\rA}
  \proofstep{1,2}{x\in \{x \in B \mid P(x)\}}{\rIE{2,1}}
  \proofstep{1,2}{P(x)}{\FormulaRefAuto{x \in \{x \in A \mid P(x)\}\vdash P(x)}{3}}
\end{tabproof}

\FormulaThmAuto{x \not\in \{x \in A \mid P(x)\}\eqvdash (x \not\in A \lor \neg P(x))}
\begin{tabproofwide}
  \proofstepwide{x \not\in \{x \in A \mid P(x)\}}{\leftrightarrow}{\neg(x \in A \land P(x))}{\FormulaRefAuto{P \leftrightarrow Q \dashv \vdash \neg P \leftrightarrow \neg Q}{\FormulaRefAuto{\{x\in A \mid P(x)\} := \iota B(\forall x\,(x\in B \leftrightarrow (x\in A \land P(x)) ))}}}
  \proofstepwide{\neg(x \in A \land P(x))}{\leftrightarrow}{x \not\in A \lor \neg P(x)}{\FormulaRefAuto{\neg(P \land Q) \eqvdash \neg P \lor \neg Q}{1}}
  \proofstepwide{x \not\in \{x \in A \mid P(x)\}}{\leftrightarrow}{x \not\in A \lor \neg P(x)}{\rChain{1,2}}
\end{tabproofwide}

\FormulaThmAuto{\{ x \in A \mid P(x) \} \subseteq A}
\begin{tabproof}
  \proofstep{1}{x \in \{ x \in A \mid P(x) \}}{\rA}
  \proofstep{1}{x \in A \land P(x)}{\FormulaRefAuto{\{x\in A \,\mid\, P(x)\} := \iota B(\forall x\,(x\in B \leftrightarrow (x\in A \land P(x))))}{1}}
  \proofstep{1}{x \in A}{\rAEa{2}}
  \proofstep{}{ \{ x \in A \mid P(x) \} \subseteq A }{\FormulaRefAuto{A \subseteq B := \forall x\,(x \in A \rightarrow x \in B)}{\rUI{\rRI{1,3}}}}
\end{tabproof}

\FormulaThmAuto{A=\{ x \in B \mid P(x) \}\vdash  A\subseteq B}
\begin{tabproof}
  \proofstep{1}{A=\{ x \in B \mid P(x) \}}{\rA}
  \proofstep{1}{\{ x \in B \mid P(x) \}\subseteq B}{\FormulaRefAuto{\{ x \in A \mid P(x) \} \subseteq A}{1}}
\proofstep{1}{A\subseteq B}{\rIE{1,2}}
\end{tabproof}

\FormulaThmAuto{\forall x \in A\,(P(x)),\, y \in A \vdash P(y)}
\begin{tabproof}
  \proofstep{1}{\forall x \in A\,(P(x))}{\rA}
  \proofstep{2}{y \in A}{\rA}
  \proofstep{1}{y \in A \rightarrow P(y)}{\rUE{1}}
  \proofstep{1,2}{P(y)}{\rRE{3,2}}
\end{tabproof}

\FormulaThmAuto{\forall x \in M(P(x)) \eqvdash M = \{x \in M \mid P(x)\}}
\begin{tabproofsplit}
\proofpart{\(\vdash\)}
  \proofstep{1}{\forall x \in M(P(x))}{\rA}
  \proofstep{1}{x \in M \rightarrow P(x)}{\rUE{1}}
  \proofstep{1}{x \in M \leftrightarrow (x \in M \land P(x))}{\FormulaRefAuto{P \rightarrow Q \vdash P \leftrightarrow (P \land Q)}{2}}
  \proofstep{1}{x \in M \leftrightarrow x \in \{x \in M \mid P(x)\}}{\FormulaRefAuto{\{x\in A \,\mid\, P(x)\} := \iota B\bigl(\forall x\,(x\in B \leftrightarrow (x\in A \land P(x)))\bigr)}{3}}
  \proofstep{1}{\forall x (x \in M \leftrightarrow x \in \{x \in M \mid P(x)\})}{\rUI{4}}
  \proofstep{1}{M = \{x \in M \mid P(x)\}}{\FormulaRefAuto{\forall x\, (x \in A \leftrightarrow x \in B) \eqvdash A = B}{5}}
\closeproofpart

\proofpart{\(\dashv\)}
  \proofstep{1}{M = \{x \in M \mid P(x)\}}{\rA}
  \proofstep{2}{y \in M}{\rA}
  \proofstep{1,2}{y \in \{x \in M \mid P(x)\}}{\FormulaRefAuto{A = B,\, x \in A \vdash x \in B}{1,2}}
  \proofstep{1,2}{P(y)}{\rAEb{\FormulaRefAuto{\{x\in A \,\mid\, P(x)\} := \iota B\bigl(\forall x\,(x\in B \leftrightarrow (x\in A \land P(x)))\bigr)}{3}}}
  \proofstep{1}{y \in M \rightarrow P(y)}{\rRI{2,4}}
  \proofstep{1}{\forall x \in M(P(x))}{\rUI{5}}
\closeproofpart
\end{tabproofsplit}

\FormulaThmAuto{A\subseteq \{x\in B\mid P(x)\}\vdash \forall x\in A(P(x))}
\begin{tabproof}
\proofstep{1}{A\subseteq \{x\in B\mid P(x)\}}{\rA}
\proofstep{2}{x\in A}{\rA}
\proofstep{1,2}{x\in \{x\in B\mid P(x)\}}{\FormulaRefAuto{ A\subseteq B,\, x\in A \vdash x\in B }{1,2}}
\proofstep{1,2}{P(x)}{\rAEb{\FormulaRefAuto{x \in \{x \in A \mid P(x)\} \eqvdash x \in A \land P(x)}{3}}}
\proofstep{1,2}{\forall x\in A(P(x))}{\rUI{\rRI{2,4}}}
\end{tabproof}

\chapter{Russel Paradoxon und die universelle Menge}

\FormulaThmAuto[Russells Paradoxon in der ZF-Mengenlehre]{\neg \exists U \forall A (A \in U \leftrightarrow A \not\in A)}
\begin{tabproof}
\proofstep{1}{\exists U \forall A (A \in U \leftrightarrow A \notin A)}{\rA}
\proofstep{2}{\forall A (A \in U \leftrightarrow A \notin A)}{\rA}
\proofstep{2}{U \in U \leftrightarrow U \notin U}{\rUE{2}}
\proofstep{}{\neg(U \in U \leftrightarrow U \notin U)}{\FormulaRefAuto{\neg (P\leftrightarrow \neg P)}}
\proofstep{2}{\bot}{\rBI{3,4}}
\proofstep{1}{\bot}{\rEE{1,2,5}}
\proofstep{}{\neg\exists U \forall A (A \in U \leftrightarrow A \notin A)}{\rCI{1,6}}
\end{tabproof}



Angenommen, es gibt eine universelle Menge \( U \) in der ZF-Mengenlehre, dann führt dies aufgrund des nachstehenden Satzes zu einem Widerspruch. 


\FormulaThmAuto{\exists U \forall A (A \in U)\vdash \forall A(A\not\in A\leftrightarrow A\in U\land A\not\in A)}
\begin{tabproof}
\proofstep{1}{\exists U \forall A (A \in U)}{\rA}
\proofstep{2}{\forall A (A \in U)}{\rA}
\proofstep{2}{\forall A (A \not\in A\leftrightarrow A\in U\land A\notin A)}{\FormulaRefAuto{\forall x(P(x)) \vdash \forall x(Q(x)\leftrightarrow P(x)\land Q(x))}{2}}
\proofstep{1}{\forall A (A \not\in A\leftrightarrow A\in U\land A\notin A)}{\rEE{1,2,3}}
\end{tabproof}

\FormulaThmAuto{\neg \exists U \forall A (A \in U)}
\begin{tabproof}
\proofstep{1}{\exists U \forall A (A \in U)}{\rA}
\proofstep{1}{\forall A(A\not\in A\leftrightarrow A\in U\land A\not\in A)}{\FormulaRefAuto{\exists U \forall A (A \in U)\vdash \forall A(A\not\in A\leftrightarrow A\in U\land A\not\in A)}{1}}
\proofstep{1}{\exists B\forall A(A\in B\leftrightarrow A\in U\land A\notin A)}{\FormulaRefAuto{\exists B\;\forall x\;\bigl(x \in B \;\leftrightarrow\; x \in A \,\land\, P(x)\bigr)}{2}}
\proofstep{4}{\forall A(A\in B\leftrightarrow A\in U\land A\notin A)}{\rA}
\proofstep{1,4}{\forall A(A\in B\leftrightarrow A\notin A)}{\FormulaRefAuto{\forall x(P(x)\leftrightarrow Q(x)), \forall x(R(x)\leftrightarrow Q(x))\vdash \forall x(P(x)\leftrightarrow R(x))}{4,2}}
\proofstep{1,4}{\exists B\forall A(A\in B\leftrightarrow A\notin A)}{\rEI{5}}
\proofstep{}{\neg\exists B\forall A(A\in B\leftrightarrow A\notin A)}{\FormulaRefAuto{\neg \exists U \forall A (A \in U \leftrightarrow A \not\in A)}}
\proofstep{1,4}{\bot}{\rBI{6,7}}
\proofstep{1}{\bot}{\rEE{3,4,8}}
\proofstep{}{\neg \exists U \forall A (A \in U)}{\rCI{1,9}}
\end{tabproof}

\chapter{Schnittmengen}

\section{Definition der Schnittmenge}

\FormulaDefDelta[Schnitt]{A \cap B := \{ x \in A \mid x \in B \}}{%
\DeltaRow{Mengen}{A\dsep B}%
}

\section{Grundlegende Eigenschaften}

\FormulaThmAuto{x \in A \cap B \vdash x \in A}
\begin{tabproof}
\proofstep{1}{x \in A \cap B}{\rA}
\proofstep{1}{x \in A}{\rAEa{\FormulaRefAuto{\{x\in A \,\mid\, P(x)\} := \iota B\bigl(\forall x\,(x\in B \leftrightarrow (x\in A \land P(x)))\bigr)}{\FormulaRefAuto{A \cap B := \{ x \in A \mid x \in B \}}}}}
\end{tabproof}

\FormulaThmAuto{x \in A \cap B \vdash x \in B}
\begin{tabproof}
\proofstep{1}{x \in A \cap B}{\rA}
\proofstep{1}{x \in B}{\rAEb{\FormulaRefAuto{\{x\in A \,\mid\, P(x)\} := \iota B\bigl(\forall x\,(x\in B \leftrightarrow (x\in A \land P(x)))\bigr)}{\FormulaRefAuto{A \cap B := \{ x \in A \mid x \in B \}}}}}
\end{tabproof}

\FormulaThmAuto{x \in A \cap B \eqvdash x \in A \land x \in B}
\begin{tabproof}
\proofstepstar{}{x \in A \cap B \leftrightarrow x \in A \land x \in B}{\FormulaRefAuto{\{x\in A \,\mid\, P(x)\} := \iota B\bigl(\forall x\,(x\in B \leftrightarrow (x\in A \land P(x)))\bigr)}{\FormulaRefAuto{A \cap B := \{ x \in A \mid x \in B \}}}}
\end{tabproof}

\FormulaThmAuto{x \in A, x\in B \vdash x \in A\cap B}
\begin{tabproof}
\proofstep{1}{x \in A}{\rA}
\proofstep{2}{x \in B}{\rA}
\proofstep{1,2}{x \in A\land x\in B}{\rAI{1,2}}
\proofstep{1,2}{x \in A\cap B}{\FormulaRefAuto{x \in A \cap B \eqvdash x \in A \land x \in B}{3}}
\end{tabproof}

\FormulaThmAuto{x \in (A \cap B) \cap C \eqvdash (x \in A \land x \in B) \land x \in C}
\begin{tabproofwide}
  \proofstepwide{x \in (A \cap B)}{\leftrightarrow}{x \in A \land x \in B}{\FormulaRefAuto{x \in A \cap B \eqvdash x \in A \land x \in B}}
  \proofstepwide{x \in (A \cap B) \cap C}{\leftrightarrow}{x \in (A \cap B) \land x \in C}{\FormulaRefAuto{x \in A \cap B \eqvdash x \in A \land x \in B}}
  \proofstepwide{}{\leftrightarrow}{(x \in A \land x \in B) \land x \in C}{\rLRS{1}}
  \proofstepwide{x \in (A \cap B) \cap C}{\leftrightarrow}{(x \in A \land x \in B) \land x \in C}{\rChain{2,3}}
\end{tabproofwide}

\FormulaThmAuto{x \in A \cap (B \cap C) \eqvdash x \in A \land (x \in B \land x \in C)}
\begin{tabproofwide}
  \proofstepwide{x \in B \cap C}{\leftrightarrow}{x \in B \land x \in C}{\FormulaRefAuto{x \in A \cap B \eqvdash x \in A \land x \in B}}
  \proofstepwide{x \in A \cap (B \cap C)}{\leftrightarrow}{x \in A \land x \in (B \cap C)}{\FormulaRefAuto{x \in A \cap B \eqvdash x \in A \land x \in B}}
  \proofstepwide{}{\leftrightarrow}{x \in A \land (x \in B \land x \in C)}{\rLRS{1}}
  \proofstepwide{x \in A \cap (B \cap C)}{\leftrightarrow}{x \in A \land (x \in B \land x \in C)}{\rChain{2,3}}
\end{tabproofwide}


\FormulaThmAuto[Idempotenz des Schnitts]{A = A \cap A}
\begin{tabproofwide}
  \proofstepwide{x \in A \cap A}{\leftrightarrow}{x \in A}{\rAEa{\FormulaRefAuto{\{x\in A \mid P(x)\} := \iota B(\forall x\,(x\in B \leftrightarrow (x\in A \land P(x))))}{\FormulaRefAuto{A \cap B := \{ x \in A \mid x \in B \}}}}}
  \proofstepwide{A}{=}{A \cap A}{\FormulaRefAuto{\forall x\, (x \in A \leftrightarrow x \in B) \eqvdash A = B}{\rUI{1}}}
\end{tabproofwide}

\FormulaThmAuto[Kommutativität des Schnitts]{A \cap B = B \cap A}
\begin{tabproofwide}
  \proofstepwide{x \in A \cap B}{\leftrightarrow}{x \in A \land x \in B}{\FormulaRefAuto{x \in A \cap B \eqvdash x \in A \land x \in B}}
  \proofstepwide{}{ \leftrightarrow }{x \in B \land x \in A}{\FormulaRefAuto{P \land Q \vdash Q \land P}{1}}
  \proofstepwide{}{ \leftrightarrow }{x \in B \cap A}{\FormulaRefAuto{x \in A \cap B \eqvdash x \in A \land x \in B}{2}}
  \proofstepwide{x \in A \cap B}{\leftrightarrow}{x \in B \cap A}{\rChain{1,3}}
  \proofstepwide{A \cap B}{=}{B \cap A}{\FormulaRefAuto{\forall x\, (x \in A \leftrightarrow x \in B) \eqvdash A = B}{\rUI{4}}}
\end{tabproofwide}


\FormulaThmAuto[Assoziativität des Schnitts]{(A \cap B) \cap C = A \cap (B \cap C)}
\begin{tabproofwide}
  \proofstepwide{x \in (A \cap B) \cap C}{\leftrightarrow}{(x \in A \land x \in B) \land x \in C}{\FormulaRefAuto{x \in (A \cap B) \cap C \eqvdash (x \in A \land x \in B) \land x \in C}}
  \proofstepwide{}{\leftrightarrow}{x \in A \land (x \in B \land x \in C)}{\FormulaRefAuto{P \land (Q \land R) \eqvdash (P \land Q) \land R}{1}}
  \proofstepwide{}{\leftrightarrow}{x \in A \cap (B \cap C)}{\FormulaRefAuto{x \in A \cap (B \cap C) \eqvdash x \in A \land (x \in B \land x \in C)}{}}
  \proofstepwide{x \in (A \cap B) \cap C}{\leftrightarrow}{x \in A \cap (B \cap C)}{\rChain{1,3}}
  \proofstepwide{(A \cap B) \cap C}{=}{A \cap (B \cap C)}{\FormulaRefAuto{\forall x\, (x \in A \leftrightarrow x \in B) \eqvdash A = B}{\rUI{4}}}
\end{tabproofwide}


\FormulaThmAuto{A \cap B \subseteq A}
\begin{tabproof}
  \proofstep{1}{x \in A \cap B}{\rA}
  \proofstep{1}{x \in A}{\rAEa{\FormulaRefAuto{\{x\in A \,\mid\, P(x)\} := \iota B\bigl(\forall x\,(x\in B \leftrightarrow (x\in A \land P(x)))\bigr)}{\FormulaRefAuto{A \cap B := \{ x \in A \mid x \in B \}}{1}}}}
  \proofstep{}{A \cap B \subseteq A}{\FormulaRefAuto{ A \subseteq B := \forall x\,(x\in A \rightarrow x\in B) }{\rUI{\rRI{1,2}}}}
\end{tabproof}

\FormulaThmAuto{A \cap B \subseteq B}
\begin{tabproof}
  \proofstep{1}{x \in A \cap B}{\rA}
  \proofstep{1}{x \in B}{\rAEb{\FormulaRefAuto{\{x\in A \,\mid\, P(x)\} := \iota B\bigl(\forall x\,(x\in B \leftrightarrow (x\in A \land P(x)))\bigr)}{\FormulaRefAuto{A \cap B := \{ x \in A \mid x \in B \}}{1}}}}
  \proofstep{}{A \cap B \subseteq B}{\FormulaRefAuto{ A \subseteq B := \forall x\,(x\in A \rightarrow x\in B) }{\rUI{\rRI{1,2}}}}
\end{tabproof}

\FormulaThmAuto{A \subseteq B \vdash A \cap C \subseteq B \cap C}
\begin{tabproof}
  \proofstep{1}{A \subseteq B}{\rA}
  \proofstep{2}{x \in A \cap C}{\rA}
  \proofstep{2}{x \in A}{\FormulaRefAuto{x \in A \cap B \vdash x \in A}{2}}
  \proofstep{2}{x \in C}{\FormulaRefAuto{x \in A \cap B \vdash x \in B}{2}}
  \proofstep{1,2}{x \in B}{\FormulaRefAuto{A \subseteq B,\, x \in A \vdash x \in B}{1,3}}
  \proofstep{1,2}{x \in B \cap C}{\FormulaRefAuto{x \in A \cap B \eqvdash x \in A \land x \in B}{\rAI{5,4}}}
  \proofstep{1}{A \cap C \subseteq B \cap C}{\FormulaRefAuto{A \subseteq B := \forall x\,(x \in A \rightarrow x \in B)}{\rUI{\rRI{2,6}}}}
\end{tabproof}


\FormulaThmAuto{A \subseteq B \eqvdash A \cap B = A}
\begin{tabproofsplitwide}
  \proofpartwide{\(\vdash\)}
    \proofstepwidestar[1]{A \subseteq B}{\rA}
    \proofstepwidestar[]{A \cap B \subseteq A}{\FormulaRefAuto{A \cap B \subseteq A}}
    \proofstepwide{A}{=}{A \cap A}{\FormulaRefAuto{A = A \cap A}}
    \proofstepwide[1]{}{\subseteq}{A \cap B}{\FormulaRefAuto{A \subseteq B \vdash A \cap C \subseteq B \cap C}{1}}
    \proofstepwide[1]{A}{\subseteq}{A \cap B}{\rChain{3,4}}
    \proofstepwide[1]{A \cap B}{=}{A}{\FormulaRefAuto{A \subseteq B \land B \subseteq A \eqvdash A = B}{\rAI{2,5}}}
  \closeproofpartwide

  \proofpartwide{\(\dashv\)}
    \proofstepwidestar[1]{A \cap B = A}{\rA}
    \proofstepwide[1]{A}{\subseteq}{A \cap B}{\rAEb{\FormulaRefAuto{A \subseteq B \land B \subseteq A \eqvdash A = B}{1}}}
    \proofstepwide[1]{A \cap B}{\subseteq}{B}{\FormulaRefAuto{A \cap B \subseteq B}}
    \proofstepwide[1]{A}{\subseteq}{B}{\FormulaRefAuto{A \subseteq B, B \subseteq C \vdash A \subseteq C}{2,3}}
  \closeproofpartwide
\end{tabproofsplitwide}

\FormulaThmAuto{B \subseteq A \eqvdash A \cap B = B}
\begin{tabproof}
  \proofstep{}{B \subseteq A \leftrightarrow B \cap A = B}{\FormulaRefAuto{A \subseteq B \eqvdash A \cap B = A}}
  \proofstep{}{B \cap A = A \cap B}{\FormulaRefAuto{A \cap B = B \cap A}}
  \proofstep{}{B \subseteq A \leftrightarrow A \cap B = B}{\rLRS{2,1}}
\end{tabproof}

\FormulaThmAuto{\emptyset \cap A = \emptyset}
\begin{tabproof}
  \proofstep{}{ \emptyset \subseteq A }{ \rUE{\FormulaRefAuto{\forall A\,(\emptyset \subseteq A)} } }
  \proofstep{}{ \emptyset \cap A = \emptyset }{ \FormulaRefAuto{ A \subseteq B \eqvdash A \cap B = A }{1} }
\end{tabproof}


\FormulaThmAuto{A \cap \emptyset = \emptyset}
\begin{tabproof}
  \proofstep{}{ \emptyset \subseteq A }{ \rUE{\FormulaRefAuto{\forall A\,(\emptyset \subseteq A)} } }
  \proofstep{}{ A \cap \emptyset = \emptyset }{ \FormulaRefAuto{ B \subseteq A \eqvdash A \cap B = B }{1} }
\end{tabproof}


\FormulaThmAuto{A \cap B = \emptyset,\ x \in A \vdash x \notin B}
\begin{tabproof}
  \proofstep{1}{A \cap B = \emptyset}{\rA}
  \proofstep{2}{x \in A}{\rA}
  \proofstep{3}{x \in B}{\rA}
  \proofstep{2,3}{x \in A \cap B}{\FormulaRefAuto{x \in A \cap B \eqvdash x \in A \land x \in B}{\rAI{2,3}}}
  \proofstep{1,2,3}{x \in \emptyset}{\rIE{1,4}}
  \proofstep{}{x \notin \emptyset}{\rUE{\FormulaRefAuto{\emptyset := \iota O\bigl(\forall x\,(x \not\in O)\bigr)}{}}}
  \proofstep{}{ \bot }{\rBI{5,6}}
  \proofstep{1,2}{x \notin B}{\rCE{1,2}}
\end{tabproof}

\FormulaThmAuto{A \cap B = \emptyset,\ x \in B \vdash x \notin A}
\begin{tabproof}
  \proofstep{1}{A \cap B = \emptyset}{\rA}
  \proofstep{2}{x \in B}{\rA}
  \proofstep{}{B \cap A = A \cap B}{\FormulaRefAuto{A \cap B = B \cap A}}
  \proofstep{1}{B \cap A = \emptyset}{\rIE{1,3}}
  \proofstep{1,2}{x \notin A}{\FormulaRefAuto{A \cap B = \emptyset,\ x \in A \vdash x \notin B}{4,2}}
\end{tabproof}


\section{Der unendliche Schnitt}

In diesem Abschnitt sei \( P \) ein Prädikat, das einer Menge \( A \) eine Eigenschaft zuweist.

\FormulaThmAuto{P(C) \vdash \{ x \in B \mid \forall A (P(A) \rightarrow x \in A) \} \subseteq \{ x \in C \mid \forall A (P(A) \rightarrow x \in A) \}}
\begin{notation*}
Wir bezeichnen mit \( I_B := \{ x \in B \mid \forall A (P(A) \rightarrow x \in A) \} \) und entsprechend \( I_C := \{ x \in C \mid \forall A (P(A) \rightarrow x \in A) \} \).
\end{notation*}
\begin{tabproof}
  \proofstep{1}{P(C)}{\rA}
  \proofstep{2}{x \in I_B}{\rA}
  \proofstep{2}{\forall A (P(A) \rightarrow x \in A)}{\rAEb{\FormulaRefAuto{\{x\in A \,\mid\, P(x)\} := \iota B\bigl(\forall x\,(x\in B \leftrightarrow (x\in A \land P(x)))\bigr)}{2}}}
  \proofstep{2}{P(C) \rightarrow x \in C}{\rUE{4}}
  \proofstep{1,2}{x \in C}{\rRE{1,4}}
  \proofstep{1,2}{x \in I_C}{\FormulaRefAuto{\{x\in A \,\mid\, P(x)\} := \iota B\bigl(\forall x\,(x\in B \leftrightarrow (x\in A \land P(x)))\bigr)}{\rAI{5,3}}}
  \proofstep{1}{I_B \subseteq I_C}{\FormulaRefAuto{A \subseteq B := \forall x\,(x \in A \rightarrow x \in B)}{\rUI{\rRI{2,6}}}}
\end{tabproof}

\FormulaThmAuto{P(B),\, P(C) \vdash \{ x \in B \mid \forall A (P(A) \rightarrow x \in A) \} = \{ x \in C \mid \forall A (P(A) \rightarrow x \in A) \}}
\begin{notation*}
Wir bezeichnen mit \( I_B := \{ x \in B \mid \forall A (P(A) \rightarrow x \in A) \} \) und entsprechend \( I_C := \{ x \in C \mid \forall A (P(A) \rightarrow x \in A) \} \).
\end{notation*}
\begin{tabproof}
  \proofstep{1}{P(B)}{\rA}
  \proofstep{2}{P(C)}{\rA}
  \proofstep{2}{I_B \subseteq I_C}{\FormulaRefAuto{P(C) \vdash \{ x \in B \mid \forall A (P(A) \rightarrow x \in A) \} \subseteq \{ x \in C \mid \forall A (P(A) \rightarrow x \in A) \}}{2}}
  \proofstep{1}{I_C \subseteq I_B}{\FormulaRefAuto{P(C) \vdash \{ x \in B \mid \forall A (P(A) \rightarrow x \in A) \} \subseteq \{ x \in C \mid \forall A (P(A) \rightarrow x \in A) \}}{1}}
  \proofstep{1,2}{I_B = I_C}{\FormulaRefAuto{A \subseteq B \land B \subseteq A \eqvdash A = B}{3,4}}
\end{tabproof}

\FormulaThmAuto{\exists C\forall B\Bigl(P(B)\rightarrow C = \{ x \in B \mid \forall A (P(A) \rightarrow x \in A) \}\Bigr)}
\begin{notation*}
Wir bezeichnen mit \( I_B := \{ x \in B \mid \forall A (P(A) \rightarrow x \in A) \} \) und entsprechend \( I_D := \{ x \in D \mid \forall A (P(A) \rightarrow x \in A) \} \).
\end{notation*}
\begin{tabproofsplit}
  \proofpart{Fall 1: \( \exists D(P(D)) \vdash \exists C\forall B\bigl(P(B)\rightarrow C = I_B\bigr) \)}
    \proofstep{1}{\exists D(P(D))}{\rA}
    \proofstep{2}{P(D)}{\rA}
    \proofstep{3}{P(B)}{\rA}
    \proofstep{2,3}{I_D = I_B}{\FormulaRefAuto{P(B),\, P(C) \vdash \{ x \in B \mid \forall A (P(A) \rightarrow x \in A) \} = \{ x \in C \mid \forall A (P(A) \rightarrow x \in A) \}}}
    \proofstep{2}{\exists C\,\forall B\,\bigl(P(B) \rightarrow C = I_B\bigr)}{\rEI{\rUI{\rRI{3,4}}}}
  \closeproofpart

  \proofpart{Fall 2: \( \forall D(\neg P(D)) \vdash \exists C\forall B\bigl(P(B)\rightarrow C = I_B\bigr) \)}
    \proofstep{1}{\forall D(\neg P(D))}{\rA}
    \proofstep{1}{\forall B\,\bigl(\neg P(B) \lor C = I_B\bigr)}{\FormulaRefAuto{\forall x(F(x)) \lor \forall x(G(x)) \vdash \forall x(F(x) \lor G(x))}}
    \proofstep{1}{\forall B\,\bigl(P(B) \rightarrow C = I_B\bigr)}{\rLRS{\FormulaRefAuto{\neg(P \lor Q) \eqvdash \neg P \land \neg Q}{}, 2}}
    \proofstep{1}{\exists C\,\forall B\,\bigl(P(B) \rightarrow C = I_B\bigr)}{\rEI{3}}
  \closeproofpart

  \proofpart{Fallunterscheidung über das klassische Prinzip \( P \lor \neg P \)}
    \proofstep{}{ \exists D(P(D)) \lor \forall D(\neg P(D)) }{\rLRS{\FormulaRefAuto{\forall x(\neg P(x)) \eqvdash \neg \exists x(P(x))}{}, \FormulaRefAuto{P \lor \neg P}}}
    \proofstep{}{ \exists C\,\forall B\,\bigl(P(B) \rightarrow C = I_B\bigr) }{\rOE{1,1,5,1,4}}
  \closeproofpart
\end{tabproofsplit}


\FormulaThmAuto{
  \begin{aligned}
    &P(B_0),\;
    \forall B\Bigl(P(B)\rightarrow C= \{ x \in B \mid \forall A (P(A) \rightarrow x \in A) \}\Bigr),\\
    &\forall B\Bigl(P(B)\rightarrow D= \{ x \in B \mid \forall A (P(A) \rightarrow x \in A) \}\Bigr)
    \vdash C=D
  \end{aligned}
}
\begin{notation*}
Wir bezeichnen mit \( I_B := \{ x \in B \mid \forall A (P(A) \rightarrow x \in A) \} \).
\end{notation*}
\begin{tabproof}
    \proofstep{1}{\forall B\bigl(P(B)\rightarrow C=I_B\bigr)}{\rA}
    \proofstep{2}{\forall B\bigl(P(B)\rightarrow D=I_B\bigr)}{\rA}
    \proofstep{3}{P(B_0)}{\rA}
    \proofstep{1}{P(B_0) \rightarrow C=I_{B_0}}{\rUE{1}}
    \proofstep{1,3}{C = I_{B_0}}{\rRE{3,4}}
    \proofstep{2}{P(B_0) \rightarrow D=I_{B_0}}{\rUE{2}}
    \proofstep{2,3}{D = I_{B_0}}{\rRE{3,6}}
    \proofstep{1,2,3}{C = D}{\FormulaRefAuto{a = b,\, c = b \vdash a = c}{5,7}}
\end{tabproof}


\FormulaThmAuto{P(D)\vdash \exists! C\forall B(P(B)\rightarrow C= \{ x \in B \mid \forall A (P(A) \rightarrow x \in A) \})}
\begin{notation*}
Es sei \( I_B := \{ x \in B \mid \forall A (P(A) \rightarrow x \in A) \} \).
\end{notation*}
\begin{tabproof}
  \proofstep{}{ \exists C\forall B\bigl(P(B)\rightarrow C=I_B\bigr) }{ \FormulaRefAuto{\exists C\forall B\Bigl(P(B)\rightarrow C = \{ x \in B \mid \forall A (P(A) \rightarrow x \in A) \}\Bigr)} }
  \proofstep{2}{\forall B\bigl(P(B)\rightarrow C=I_B\bigr)}{\rA}
  \proofstep{3}{\forall B\bigl(P(B)\rightarrow D=I_B\bigr)}{\rA}
  \proofstep{4}{P(D)}{\rA}
  \proofstep{2,3,4}{C = D}{\FormulaRefAuto{P(B_0),\forall B\bigl(P(B)\rightarrow C= \{ x \in B \mid \forall A (P(A) \rightarrow x \in A) \}\bigr),\, \forall B\bigl(P(B)\rightarrow D= \{ x \in B \mid \forall A (P(A) \rightarrow x \in A) \}\bigr) \vdash C=D}{4,2,3}}
  \proofstep{4}{\exists! C\,\forall B\bigl(P(B)\rightarrow C=I_B\bigr)}{\UEI{1,2,3,5}}
\end{tabproof}
%%begin novalidate
\FormulaDefAuto[Der unendliche Schnitt]{\exists A\,P(A)\rightarrow \bigcap_{P(B)} B := \iota C\,\forall D\Bigl(P(D)\rightarrow C = \{ x \in D \mid \forall A (P(A)\rightarrow x \in A) \}\Bigr)}
%%end novalidate
% Gleichbedeutende Notation mittels Mengenschreibweise
% (nur eingeführt, wenn der Schnitt definiert ist)
\FormulaDefAuto[Notationserweiterung]{\exists A\,P(A)\rightarrow\bigcap \{\,B \mid P(B)\,\} := \bigcap_{P(B)} B}


\FormulaThmAuto{P(A) \vdash \bigcap_{P(B)} B = \{ x \in A \mid \forall D (P(D) \rightarrow x \in D) \}}
\begin{notation*}
Wir bezeichnen mit \( I_A := \{ x \in A \mid \forall A (P(A) \rightarrow x \in A) \} \) und entsprechend \( I_D := \{ x \in D \mid \forall A (P(A) \rightarrow x \in A) \} \).
\end{notation*}
\begin{tabproof}
  \proofstep{1}{P(A)}{\rA}
  \proofstep{1}{\exists M(P(M))}{\rEI{1}}
  \proofstep{1}{\forall D\Bigl(P(D) \rightarrow \bigcap_{P(B)} B = I_D\Bigr)}{\FormulaRefAuto{\exists A\,P(A)\rightarrow \bigcap_{P(B)} B := \iota C\,\forall D\bigl(P(D)\rightarrow C = \{ x \in D \mid \forall A (P(A)\rightarrow x \in A) \}\bigr)}{2}}
  \proofstep{1}{P(A)\rightarrow \bigcap_{P(B)} B = I_A}{\rUE{3}}
  \proofstep{1}{\bigcap_{P(B)} B = I_A}{\rRE{4,1}}
\end{tabproof}

\FormulaThmAuto{P(A) \vdash x \in \bigcap_{P(B)} B\leftrightarrow x \in \{\,x \in A \mid \forall D\,(P(D) \rightarrow x \in D)\,\}}
\begin{notation*}
Wir bezeichnen mit \( I_A := \{ x \in A \mid \forall A (P(A) \rightarrow x \in A) \} \).
\end{notation*}
\begin{tabproof}
  \proofstep{1}{P(A)}{\rA}
  \proofstep{1}{x\in \bigcap_{P(B)} B\leftrightarrow x\in I_A}{\rUE{\FormulaRefAuto{\forall x\, (x \in A \leftrightarrow x \in B) \eqvdash A = B}{\FormulaRefAuto{P(A) \vdash \bigcap_{P(B)} B = \{ x \in A \mid \forall D\, (P(D) \rightarrow x \in D) \}}{1}}}}
\end{tabproof}

\FormulaThmAuto{P(A) \vdash x \in \bigcap_{P(B)} B \leftrightarrow \forall C\, (P(C) \rightarrow x \in C)}
\begin{notation*}
Wir bezeichnen mit \( I_A := \{ x \in A \mid \forall A (P(A) \rightarrow x \in A) \} \).    
\end{notation*}
\begin{tabproofwide}
  \proofstepwidestar[1]{P(A)}{\rA}
  \proofstepwidestar[2]{\forall C\, (P(C)\rightarrow x\in C)}{\rA}
  \proofstepwidestar[]{\forall C\, (P(C)\rightarrow x\in C) \rightarrow x\in A}{\rRI{2,\FormulaRefAuto{P(a), \forall x\, (P(x) \rightarrow Q(x)) \vdash Q(a)}{1,2}}}
  \proofstepwide[1]{x\in \bigcap_{P(B)} B}{\leftrightarrow}{x\in I_A}{\FormulaRefAuto{P(A) \vdash x \in \bigcap_{P(B)} B\leftrightarrow x \in \{\,x \in A \mid \forall D\,(P(D) \rightarrow x \in D)\,\}}{1}}

  % ---- hier die aufgesplittete Zeile (2 Zeilen) ----
  \proofstepwide[1]{}{\leftrightarrow}{x\in A}{\multirow{2}{*}{\FormulaRefAuto{x \in \{x \in A \mid P(x)\} \eqvdash x \in A \land P(x)}{4}}}
  \proofstepwide*{}{\land}{\forall C\, (P(C)\rightarrow x\in C)}{}
  % --------------------------------------------------

  \proofstepwide[1]{}{\leftrightarrow}{\forall C\, (P(C)\rightarrow x\in C)}{\FormulaRefAuto{P \rightarrow Q \vdash P \leftrightarrow (Q \land P)}{3}}
  \proofstepwide[1]{x\in \bigcap_{P(B)} B}{\leftrightarrow}{\forall C\, (P(C)\rightarrow x\in C)}{\rChain{4,6}}
\end{tabproofwide}

\FormulaThmAuto{\exists A(P(A)) \vdash x \in \bigcap_{P(B)} B \leftrightarrow \forall C\, (P(C) \rightarrow x \in C)}
\begin{tabproof}
    \proofstep{1}{\exists A(P(A))}{\rA}
    \proofstep{2}{P(A)}{\rA}
    \proofstep{2}{x \in \bigcap_{P(B)} B \leftrightarrow \forall C\, (P(C) \rightarrow x \in C)}{\FormulaRefAuto{P(A) \vdash x \in \bigcap_{P(B)} B \leftrightarrow \forall C\, (P(C) \rightarrow x \in C)}{2}}
    \proofstep{1}{x \in \bigcap_{P(B)} B \leftrightarrow \forall C\, (P(C) \rightarrow x \in C)}{\rEE{1,2,3}}
\end{tabproof}



\FormulaThmAuto{P(C)\vdash \bigcap_{P(A)} A \subseteq C}
\begin{tabproof}
  \proofstep{1}{P(C)}{\rA}
  \proofstep{2}{x \in \bigcap_{P(A)} A}{\rA}
  \proofstep{1,2}{\forall A\,\bigl(P(A)\rightarrow x\in A\bigr)}{\FormulaRefAuto{P \leftrightarrow Q, P \vdash Q}{\FormulaRefAuto{P(A) \vdash x \in \bigcap_{P(B)} B \leftrightarrow \forall C\, (P(C) \rightarrow x \in C)}{1},\,2}}
  \proofstep{1,2}{x \in C}{\FormulaRefAuto{P(a), \forall x\,\bigl(P(x) \rightarrow Q(x)\bigr) \vdash Q(a)}{1,3}}
  \proofstep{1}{\forall x\,\bigl(x\in \bigcap_{P(A)} A \rightarrow x\in C\bigr)}{\rUI{\rRI{2,4}}}
  \proofstep{1}{\bigcap_{P(A)} A \subseteq C}{\FormulaRefAuto{A \subseteq B := \forall x\,\bigl(x\in A \rightarrow x\in B\bigr)}{5}}
\end{tabproof}

\chapter{Paarmenge}

\section{Axiom der Paarmenge}

\FormulaAxiomDelta[Paarmenge]{\exists C\;\bigl(\forall x\,(x \in C \leftrightarrow x=A \lor x=B)\bigr) }{%
\DeltaRow{Mengen}{A\dsep B}%
}

\section{Definition der Paarmenge}
%%begin novalidate
\FormulaDefDelta[Paarmenge]{\{A,B\} := \iota C\Bigl(\forall x\;\bigl(x \in C \;\leftrightarrow\; x = A \lor x = B\bigr)\Bigr)}{%
\DeltaRow{Mengen}{A\dsep B}%
}
%%end novalidate
\FormulaThmDelta{x \in \{A,B\}\;\eqvdash\;(x = A \lor x = B)}{%
\DeltaRow{Mengen}{x\dsep A\dsep B}%
}
\begin{tabproof}
  \proofstep{}{ \forall x\;\bigl(x \in C \;\leftrightarrow\; x = A \lor x = B\bigr) }{\FormulaRefAuto{\{A,B\} := \iota C\Bigl(\forall x\;\bigl(x \in C \;\leftrightarrow\; x = A \lor x = B\bigr)\Bigr)}}
  \proofstep{}{ x \in C \;\leftrightarrow\; x = A \lor x = B }{\rUE{1}}
\end{tabproof}

\FormulaDefDelta[Geordnetes Paar]{(a, b) := \{ \{ a \}, \{ a, b \} \}}{
\DeltaRow{Mengen}{a\dsep b}%
}

\section{Grundlegende Eigenschaften}

\FormulaThmAuto{x \notin \{a,b\} \eqvdash x \neq a \land x \neq b}
\begin{tabproofwide}
  \proofstepwide{x \notin \{a,b\}}{\leftrightarrow}{\neg(x = a \lor x = b)}%
    {\FormulaRefAuto{\{A,B\} := \iota C\Bigl(\forall x\;\bigl(x \in C \;\leftrightarrow\; x = A \lor x = B\bigr)\Bigr)}}
  \proofstepwide{}{\leftrightarrow}{x \neq a \land x \neq b}%
    {\FormulaRefAuto{\neg(P \lor Q) \eqvdash \neg P \land \neg Q}{1}}
  \proofstepwide{x \notin \{a,b\}}{\leftrightarrow}{x \neq a \land x \neq b}%
    {\rChain{1,2}}
\end{tabproofwide}

\FormulaThmAuto{a \in \{a,b\}}
\begin{tabproof}
  \proofstep{}{a = a}{\rIE{}}
  \proofstep{}{a = a \lor a = b}{\rOIa{1}}
  \proofstep{}{a \in \{a,b\}}{\FormulaRefAuto{\{A,B\} := \iota C\Bigl(\forall x\,\bigl(x \in C \leftrightarrow x = A \lor x = B\bigr)\Bigr)}{2}}
\end{tabproof}

\FormulaThmAuto{b \in \{a,b\}}
\begin{tabproof}
  \proofstep{}{b = b}{\rIE{}}
  \proofstep{}{b = a \lor b = b}{\rOIb{1}}
  \proofstep{}{b \in \{a,b\}}{\FormulaRefAuto{\{A,B\} := \iota C\Bigl(\forall x\,\bigl(x \in C \leftrightarrow x = A \lor x = B\bigr)\Bigr)}{2}}
\end{tabproof}

\FormulaThmAuto{\{a,b\} = \{b,a\}}
\begin{tabproofwide}
  \proofstepwide{x \in \{a,b\}}{\leftrightarrow}{x = a \lor x = b}%
    {\FormulaRefAuto{\{A,B\} := \iota C\Bigl(\forall x\,\bigl(x \in C \leftrightarrow x = A \lor x = B\bigr)\Bigr)}{}}
  \proofstepwide{}{\leftrightarrow}{x = b \lor x = a}%
    {\FormulaRefAuto{P \lor Q \vdash Q \lor P}{1}}
  \proofstepwide{}{\leftrightarrow}{x \in \{b,a\}}%
    {\FormulaRefAuto{\{A,B\} := \iota C\Bigl(\forall x\,\bigl(x \in C \leftrightarrow x = A \lor x = B\bigr)\Bigr)}{2}}
  \proofstepwide{x \in \{a,b\}}{\leftrightarrow}{x \in \{b,a\}}%
    {\rChain{1,3}}
  \proofstepwide{\{a,b\}}{=}{\{b,a\}}%
    {\FormulaRefAuto{\forall x\, (x \in A \leftrightarrow x \in B) \eqvdash A = B}{\rUI{4}}}
\end{tabproofwide}

\FormulaThmAuto{\{a,b\} \neq \emptyset}
\begin{tabproof}
  \proofstep{}{a \in \{a,b\}}{\FormulaRefAuto{a \in \{a,b\}}{}}
  \proofstep{}{\{a,b\} \neq \emptyset}{\FormulaRefAuto{\exists x\,(x \in S) \vdash S \neq \emptyset}{1}}
\end{tabproof}

\FormulaDefAuto[Einermenge]{\{a\} := \{a,a\}}

\FormulaThmAuto{a \in \{a\}}
\begin{tabproof}
  \proofstep{}{a \in \{a,a\}}{\FormulaRefAuto{a \in \{a,b\}}{}}
  \proofstep{}{a \in \{a\}}{\FormulaRefAuto{\{a\} := \{a,a\}}{1}}
\end{tabproof}

\FormulaThmAuto{x \in \{a\} \eqvdash x = a}
\begin{tabproofwide}
  \proofstepwide{x \in \{a\}}{\leftrightarrow}{x \in \{a,a\}}%
    {\rIE{\FormulaRefAuto{\{a\} := \{a,a\}}{}, 1}}
  \proofstepwide{}{\leftrightarrow}{x = a \lor x = a}%
    {\rIE{\FormulaRefAuto{\{A,B\} := \iota C\Bigl(\forall x\,\bigl(x \in C \leftrightarrow x = A \lor x = B\bigr)\Bigr)}{}, 1}}
  \proofstepwide{}{\leftrightarrow}{x = a}%
    {\FormulaRefAuto{P \lor P \eqvdash P}{2}}
  \proofstepwide{x \in \{a\}}{\leftrightarrow}{x = a}%
    {\rChain{1,3}}
\end{tabproofwide}

\FormulaThmAuto{x \notin \{a\} \eqvdash x \neq a}
\begin{tabproof}
  \proofstep{}{x \notin \{a\} \leftrightarrow x \neq a}%
    {\FormulaRefAuto{P \leftrightarrow Q \dashv \vdash \neg P \leftrightarrow \neg Q}{\FormulaRefAuto{x \in \{a\} \eqvdash x = a}{}}}
\end{tabproof}

\FormulaThmAuto{\{a\}=\{b,c\}\vdash a=b\land a=c}
\begin{tabproof}
  \proofstep{1}{\{a\}=\{b,c\}}{\rA}
  \proofstep{1}{b\in\{a\}}{\rIE{1,\FormulaRefAuto{a\in\{a,b\}}}}
  \proofstep{1}{a=b}{\FormulaRefAuto{a = b \vdash b = a}{\FormulaRefAuto{x \in \{a\} \eqvdash x = a}{2}}}
  \proofstep{1}{c\in\{a\}}{\rIE{1,\FormulaRefAuto{b\in\{a,b\}}}}
  \proofstep{1}{a=c}{\FormulaRefAuto{a = b \vdash b = a}{\FormulaRefAuto{x \in \{a\} \eqvdash x = a}{2}}}
  \proofstep{1}{a=b\land a=c}{\rAI{3,5}}
\end{tabproof}

\FormulaThmAuto{\exists x\in \{a,b\} P(x)\vdash P(a)\lor P(b)}
\begin{tabproof}
  \proofstep{1}{\exists x\in \{a,b\} P(x)}{\rA}
  \proofstep{2}{x\in \{a,b\}\land P(x)}{\rA}
  \proofstep{2}{x\in \{a,b\}}{\rAEa{2}}
  \proofstep{2}{P(x)}{\rAEb{2}}
  \proofstep{2}{x=a\lor x=b}{\FormulaRefAuto{x \in \{A,B\}\;\eqvdash\;(x = A \lor x = B)}{3}}
  \proofstep{6}{x=a}{\rA}
  \proofstep{2,6}{P(a)}{\rIE{6,4}}
  \proofstep{2,6}{P(a)\lor P(b)}{\rOIa{7}}
  \proofstep{9}{x=b}{\rA}
  \proofstep{2,9}{P(b)}{\rIE{9,4}}
  \proofstep{2,9}{P(a)\lor P(b)}{\rOIb{10}}
  \proofstep{2}{P(a)\lor P(b)}{\rOE{5,6,8,9,11}}
  \proofstep{1}{P(a)\lor P(b)}{\rEE{1,2,12}}
\end{tabproof}

\FormulaThmAuto{a \in A \vdash \{a\} \subseteq A}
\begin{tabproof}
  \proofstep{1}{x \in \{a\}}{\rA}
  \proofstep{2}{a \in A}{\rA}
  \proofstep{1}{x = a}{\FormulaRefAuto{x \in \{a\} \eqvdash x = a}{1}}
  \proofstep{1,2}{x \in A}{\rIE{3,2}}
  \proofstep{2}{\{a\} \subseteq A}{\FormulaRefAuto{A \subseteq B := \forall x\,(x \in A \rightarrow x \in B)}{\rUI{\rRI{1,4}}}}
\end{tabproof}

\FormulaThmAuto{a\in A\vdash A\cap \{A,a\}\neq\emptyset}
\begin{tabproof}
  \proofstep{1}{a \in A}{\rA}
  \proofstep{}{a \in \{A,a\}}{\FormulaRefAuto{b \in \{a,b\}}}
  \proofstep{1}{a \in A\cap \{A,a\}}{\FormulaRefAuto{x \in A, x\in B \vdash x \in A\cap B}{1,2}}
  \proofstep{1}{A\cap \{A,a\}\neq\emptyset}{\FormulaRefAuto{a \in A \vdash A \neq \emptyset}{3}}
\end{tabproof}

\FormulaThmAuto{a\in A\vdash A\cap \{a,A\}\neq\emptyset}
\begin{tabproof}
  \proofstep{1}{a \in A}{\rA}
  \proofstep{}{a \in \{a,A\}}{\FormulaRefAuto{a \in \{a,b\}}}
  \proofstep{1}{a \in A\cap \{a,A\}}{\FormulaRefAuto{x \in A, x\in B \vdash x \in A\cap B}{1,2}}
  \proofstep{1}{A\cap \{a,A\}\neq\emptyset}{\FormulaRefAuto{a \in A \vdash A \neq \emptyset}{3}}
\end{tabproof}

\FormulaThmAuto{\{a\}=\{b\}\eqvdash a=b}
\begin{tabproofsplit}
    \proofpart{\(\vdash\)}
    \proofstep{1}{\{a\}=\{b\}}{ \rA}
    \proofstep{1}{\forall x(x\in\{a\}\leftrightarrow x\in\{b\})}{\FormulaRefAuto{\forall x\, (x \in A \leftrightarrow x \in B) \eqvdash A = B}}
    \proofstep{}{a\in\{a\}}{\FormulaRefAuto{a \in \{a\}}}
    \proofstep{1}{a\in\{b\}}{\FormulaRefAuto{P\leftrightarrow Q, P\vdash Q}{\rUE{2},3}}
    \proofstep{1}{a=b}{\FormulaRefAuto{x \in \{a\} \eqvdash x = a}{4}}
    \closeproofpart
    \proofpart{\(\dashv\)}
    \proofstep{1}{a=b}{ \rA}
    \proofstep{}{\{a\}=\{a\}}{ \rII}
    \proofstep{1}{\{a\}=\{b\}}{ \rIE{1,2}}
    \closeproofpart
\end{tabproofsplit}

\FormulaThmAuto{a=c,b=d\vdash \{a,b\}\subseteq \{c,d\}}
\begin{tabproof}
  \proofstep{1}{a=c}{\rA}
  \proofstep{2}{b=d}{\rA}
  \proofstep{3}{x\in\{a,b\}}{\rA}
  \proofstep{3}{x=a\lor x=b}{\FormulaRefAuto{x \in \{A,B\}\;\eqvdash\;(x = A \lor x = B)}{3}}

  \proofcase[1]{x=a \vdash x\in\{c,d\}}
  \proofstep{5}{x=a}{\rA}
  \proofstep{1,5}{x=c}{\FormulaRefAuto{a = b,\, b = c \vdash a = c}{5,1}}
  \proofstep{1,5}{x=c\lor x=d}{\rOIa{6}}
  \proofstep{1,5}{x\in \{c,d\}}{\FormulaRefAuto{x \in \{A,B\}\;\eqvdash\;(x = A \lor x = B)}{7}}

  \proofcase[2]{x=b \vdash x\in\{c,d\}}
  \proofstep{9}{x=b}{\rA}
  \proofstep{2,9}{x=d}{\FormulaRefAuto{a = b,\, b = c \vdash a = c}{9,2}}
  \proofstep{2,9}{x=c\lor x=d}{\rOIb{10}}
  \proofstep{2,9}{x\in \{c,d\}}{\FormulaRefAuto{x \in \{A,B\}\;\eqvdash\;(x = A \lor x = B)}{11}}
  
  \proofcasesummary[1]{\{a,b\}\subseteq\{c,d\}}
  \proofstep{1,2,3}{x\in \{c,d\}}{\rOE{4,5,8,9,12}}
  \proofstep{1,2}{\{a,b\}\subseteq\{c,d\}}%
    {\FormulaRefAuto{ A \subseteq B := \forall x\,(x\in A \rightarrow x\in B)}{\rUI{\rRI{3,13}}}}
\end{tabproof}

\FormulaThmAuto{a=d,b=c\vdash \{a,b\}\subseteq \{c,d\}}
\begin{tabproof}
  \proofstep{1}{a=d}{\rA}
  \proofstep{2}{b=c}{\rA}
  \proofstep{3}{x\in\{a,b\}}{\rA}
  \proofstep{3}{x=a\lor x=b}{\FormulaRefAuto{x \in \{A,B\}\;\eqvdash\;(x = A \lor x = B)}{3}}

  \proofcase[1]{x=a \vdash x\in\{c,d\}}
  \proofstep{5}{x=a}{\rA}
  \proofstep{1,5}{x=d}{\FormulaRefAuto{a = b,\, b = c \vdash a = c}{5,1}}
  \proofstep{1,5}{x=c\lor x=d}{\rOIb{6}}
  \proofstep{1,5}{x\in \{c,d\}}{\FormulaRefAuto{x \in \{A,B\}\;\eqvdash\;(x = A \lor x = B)}{7}}

  \proofcase[2]{x=b \vdash x\in\{c,d\}}
  \proofstep{9}{x=b}{\rA}
  \proofstep{2,9}{x=c}{\FormulaRefAuto{a = b,\, b = c \vdash a = c}{9,2}}
  \proofstep{2,9}{x=c\lor x=d}{\rOIa{10}}
  \proofstep{2,9}{x\in \{c,d\}}{\FormulaRefAuto{x \in \{A,B\}\;\eqvdash\;(x = A \lor x = B)}{11}}
  
  \proofcasesummary[1]{\{a,b\}\subseteq\{c,d\}}
  \proofstep{1,2,3}{x\in \{c,d\}}{\rOE{4,5,8,9,12}}
  \proofstep{1,2}{\{a,b\}\subseteq\{c,d\}}%
    {\FormulaRefAuto{ A \subseteq B := \forall x\,(x\in A \rightarrow x\in B)}{\rUI{\rRI{3,13}}}}
\end{tabproof}

\FormulaThmAuto{a=c,b=d\vdash \{a,b\}=\{c,d\}}
% --- Beweis ---
\begin{tabproof}
  \proofstep{1}{a=c}{\rA}
  \proofstep{2}{b=d}{\rA}
  \proofstep{2}{c=a}{\FormulaRefAuto{a=b\vdash b=a}{1}}
  \proofstep{2}{b=d}{\FormulaRefAuto{a=b\vdash b=a}{2}}

  \proofstep{1,2}{\{a,b\}\subseteq\{c,d\}}{\FormulaRefAuto{a=c,b=d\vdash \{a,b\}\subseteq \{c,d\}}{1,2}}
  \proofstep{1,2}{\{c,d\}\subseteq\{a,b\}}{\FormulaRefAuto{a=c,b=d\vdash \{a,b\}\subseteq \{c,d\}}{3,4}}

  \proofstep{1,2}{\{a,b\}=\{c,d\}}{\FormulaRefAuto{ A \subseteq B, B \subseteq A \vdash A = B }{5,6}}
\end{tabproof}

\FormulaThmAuto{a=c\land b=d\vdash \{a,b\}=\{c,d\}}[Sei \(A\) eine Menge und \(a,b,c,d\in A\), dann gilt:]
\begin{tabproof}
    \proofstep{1}{a=c\land b=d}{\rA}
    \proofstep{1}{a=c}{\rAEa{1}}
    \proofstep{1}{b=d}{\rAEb{1}}
    \proofstep{1}{\{a,b\}=\{c,d\}}{\FormulaRefAuto{a=c,b=d\vdash \{a,b\}=\{c,d\}}{2,3}}
\end{tabproof}

\FormulaThmAuto{a=d, b=c\vdash \{a,b\}=\{c,d\}}
% --- Beweis ---
\begin{tabproof}
  \proofstep{1}{a=d}{\rA}
  \proofstep{2}{b=c}{\rA}
  \proofstep{3}{d=a}{\FormulaRefAuto{a=b\vdash b=a}{1}}   % Symmetrie aus 1
  \proofstep{4}{c=b}{\FormulaRefAuto{a=b\vdash b=a}{2}}   % Symmetrie aus 2

  % Erstes Inklusionsziel: {a,b} ⊆ {c,d}
  \proofstep{1,2}{\{a,b\}\subseteq\{c,d\}}{%
    \FormulaRefAuto{a=d,b=c\vdash \{a,b\}\subseteq \{c,d\}}{1,2}}

  % Zweites Inklusionsziel: {c,d} ⊆ {a,b}
  % Anwenden des selben Teilmengen-Theorems mit Umbenennung (a',b',c',d')=(c,d,a,b)
  % Voraussetzungen dafür sind c=b (Zeile 4) und d=a (Zeile 3)
  \proofstep{4,3}{\{c,d\}\subseteq\{a,b\}}{%
    \FormulaRefAuto{a=d,b=c\vdash \{a,b\}\subseteq \{c,d\}}{4,3}}

  % Gleichheit aus beidseitiger Inklusion
  \proofstep{1,2}{\{a,b\}=\{c,d\}}{%
    \FormulaRefAuto{ A \subseteq B, B \subseteq A \vdash A = B }{5,6}}
\end{tabproof}

\FormulaThmAuto{a=d\land b=c\vdash \{a,b\}=\{c,d\}}
\begin{tabproof}
    \proofstep{1}{a=d\land b=c}{\rA}
    \proofstep{1}{a=d}{\rAEa{1}}
    \proofstep{1}{b=c}{\rAEb{1}}
    \proofstep{1}{\{a,b\}=\{c,d\}}{\FormulaRefAuto{a=d, b=c\vdash \{a,b\}=\{c,d\}}{2,3}}
\end{tabproof}

\FormulaThmAuto{\{a,b\}=\{c,d\}\vdash (a=c\lor a=d)\land (b=c\lor b=d)}
\begin{tabproof}
    \proofstep{1}{\{a,b\}=\{c,d\}}{ \rA}
    \proofstep{}{a\in \{a,b\}}{\FormulaRefAuto{a\in \{a,b\}}}
    \proofstep{1}{a\in \{c,d\}}{\FormulaRefAuto{A=B, x\in A\vdash x\in B}{1,2}}
    \proofstep{1}{a=c\lor a=d}{\FormulaRefAuto{x \in \{A,B\}\;\eqvdash\;(x = A \lor x = B)}{3}}
    \proofstep{}{b\in \{a,b\}}{\FormulaRefAuto{b\in \{a,b\}}}
    \proofstep{1}{b\in \{c,d\}}{\FormulaRefAuto{A=B, x\in A\vdash x\in B}{1,5}}
    \proofstep{1}{b=c\lor b=d}{\FormulaRefAuto{x \in \{A,B\}\;\eqvdash\;(x = A \lor x = B)}{3}}
    \proofstep{1}{(a=c\lor a=d)\land (b=c\lor b=d)}{\rAI{4,7}}
\end{tabproof}

\FormulaThmAuto{\{a,b\}=\{c,d\},a=c,b=c\vdash d=c}
\begin{tabproof}
\proofstep{1}{\{a,b\}=\{c,d\}}{\rA}
\proofstep{2}{a=c}{\rA}
\proofstep{3}{b=c}{\rA}
\proofstep{}{d\in\{c,d\}}{\FormulaRefAuto{b\in\{a,b\}}}
\proofstep{1}{d\in\{a,b\}}{\rIE{1,4}}
\proofstep{1}{d=a\lor d=b}{\FormulaRefAuto{x \in \{A,B\}\;\eqvdash\;(x = A \lor x = B)}{5}}
\proofstep{1,2,3}{d=c}{\FormulaRefAuto{a = c,\, b = c, d=a\lor d=b \vdash d = c}{2,3,5}}
\end{tabproof}

\FormulaThmAuto{\{a,b\}=\{c,d\},a=d,b=d\vdash c=d}
\begin{tabproof}
\proofstep{1}{\{a,b\}=\{c,d\}}{\rA}
\proofstep{2}{a=d}{\rA}
\proofstep{3}{b=d}{\rA}
\proofstep{}{c\in\{c,d\}}{\FormulaRefAuto{a\in\{a,b\}}}
\proofstep{1}{c\in\{a,b\}}{\rIE{1,4}}
\proofstep{1}{c=a\lor c=b}{\FormulaRefAuto{x \in \{A,B\}\;\eqvdash\;(x = A \lor x = B)}{5}}
\proofstep{1,2,3}{c=d}{\FormulaRefAuto{a = c,\, b = c, d=a\lor d=b \vdash d = c}{2,3,5}}
\end{tabproof}

\FormulaThmAuto{\{a,b\}=\{c,d\} \;\eqvdash\; \bigl((a=c\land b=d)\;\lor\;(a=d\land b=c)\bigr)}
\begin{tabproofsplit}
  %%%%%%%%%%%%%%%%%%%%%%%%%%%%%
  \proofpart{\(\vdash\)}
  \proofstep{1}{\{a,b\}=\{c,d\}}{\rA}
  \proofstep{1}{(a=c\lor a=d)\land (b=c\lor b=d)}%
    {\FormulaRefAuto{\{a,b\}=\{c,d\}\vdash (a=c\lor a=d)\land (b=c\lor b=d)}{1}}
    \proofstep{1}{(a=c\land b=c)\lor (a=c\land b=d)\lor}%
  {\multirow{2}{*}{$\FormulaRefAuto{(P \lor Q) \land (R \lor S) \dashv \vdash
   (P \land R) \lor (P \land S) \lor (Q \land R) \lor (Q \land S)}{2}$}}
\proofstepstar{}{(a=d\land b=c)\lor (a=d\land b=d)}{}
  \proofstep{4}{a=c\land b=c}{\rA}
  \proofstep{4}{a=c}{\rAEa{4}}
  \proofstep{4}{b=c}{\rAEb{4}}
  \proofstep{1,4}{d=c}{\FormulaRefAuto{\{a,b\}=\{c,d\},a=c,b=c\vdash d=c}{1,5,6}}
  \proofstep{1,4}{b=d}{\FormulaRefAuto{a = b, c = b\vdash a = c}{5,6}}
  \proofstep{1,4}{(a=c\land b=d) \lor (a=d\land b=c)}{\rOIa{\rAI{5,8}}}
  \proofstep{10}{(a=c\land b=d) \lor (a=d\land b=c)}{\rA}
  \proofstep{11}{a=d\land b=d}{\rA}
  \proofstep{11}{a=d}{\rAEa{12}}
  \proofstep{11}{b=d}{\rAEb{12}}
  \proofstep{1,11}{c=d}{\FormulaRefAuto{\{a,b\}=\{c,d\},a=d,b=d\vdash c=d}{1,12,13}}
  \proofstep{1,11}{b=c}{\FormulaRefAuto{a = b, c = b\vdash a = c}{14,16}}
  \proofstep{1,11}{(a=c\land b=d) \lor (a=d\land b=c)}{\rOIb{\rAI{12,15}}}
  \proofstep{1}{(a=c\land b=d)\lor (a=d\land b=c)}{\rOEn{3,4,9,10,10,11,16}}
  %%%%%%%%%%%%%%%%%%%%%%%%%%%%%
  \closeproofpart
  \proofpart{\(\dashv\)}
  \proofstep{1}{(a=c\land b=d)\lor (a=d\land b=c)}{\rA}
  \proofstep{2}{a=c\land b=d}{\rA}
  \proofstep{2}{a=c}{\rAEa{2}}
  \proofstep{2}{b=d}{\rAEb{2}}
  \proofstep{2}{\{a,b\}=\{c,d\}}{\FormulaRefAuto{a=c,b=d\vdash \{a,b\}=\{c,d\}}{3,4}}
  \proofstep{6}{a=d\land b=d}{\rA}
  \proofstep{6}{a=d}{\rAEa{2}}
  \proofstep{6}{b=c}{\rAEb{2}}
  \proofstep{6}{\{a,b\}=\{d,c\}}{\FormulaRefAuto{a=c,b=d\vdash \{a,b\}=\{c,d\}}{7,8}}
  \proofstep{6}{\{a,b\}=\{c,d\}}{\rIE{\FormulaRefAuto{\{a,b\} = \{b,a\}},9}}
  \proofstep{1}{\{a,b\}=\{c,d\}}{\rOE{1,2,5,6,10}}
  \closeproofpart
\end{tabproofsplit}

\FormulaThmAuto{\{a,b\}=\{c,d\}, a=c \vdash b=d}
\begin{tabproof}
\proofstep{1}{\{a,b\}=\{c,d\}}{\rA}
\proofstep{2}{a=c}{\rA}
\proofstep{1}{(a=c\land b=d)\;\lor\;(a=d\land b=c)}{\FormulaRefAuto{\{a,b\}=\{c,d\} \;\eqvdash\; \bigl((a=c\land b=d)\;\lor\;(a=d\land b=c)\bigr)}{1}}
\proofstep{4}{a=c\land b=d}{\rA}
\proofstep{4}{b=d}{\rAEb{4}}
\proofstep{6}{a=d\land b=c}{\rA}
\proofstep{6}{a=d}{\rAEa{6}}
\proofstep{6}{b=c}{\rAEb{6}}
\proofstep{2}{c=a}{\FormulaRefAuto{a=b \vdash b=a}{2}}
\proofstep{9,7}{c=d}{\FormulaRefAuto{a=b,\, b=c \vdash a=c}{9,7}}
\proofstep{8,10}{b=d}{\FormulaRefAuto{a=b,\, b=c \vdash a=c}{8,10}}
% Disjunktionselimination
\proofstep{1,2}{b=d}{\rOE{3,4,5,6,11}}
\end{tabproof}


\section{Geordnete Paare}

\FormulaThmAuto{(a,b) = (c,d)\eqvdash a=c\land b=d}[Sei \(A\) eine Menge und \(a,b,c,d\in A\), dann gilt:]
\begin{tabproofsplit}
    \proofpart{\(\vdash\)}
    \proofstep{1}{(a,b) = (c,d)}{\rA}
    \proofstep{1}{\{\{a\},\{a,b\}\}=\{\{c\},\{c,d\}\}}{\rIE{\FormulaRefAuto{(a, b) := \{ \{ a \}, \{ a, b \} \}},1}}
    \proofstep{1}{\{a\}=\{c\}\land \{a,b\}=\{c,d\}}{\multirow{2}{*}{$\FormulaRefAuto{\{a,b\}=\{c,d\} \;\eqvdash\; \bigl((a=c\land b=d)\;\lor\;(a=d\land b=c)\bigr)}{2}$}}
    \proofstepstar{}{\lor \{a\}=\{c,d\}\land \{a,b\}=\{c\}}{}
    \proofstep{4}{\{a\}=\{c\}\land \{a,b\}=\{c,d\}}{\rA}
    \proofstep{4}{\{a\}=\{c\}}{\rAEa{4}}
    \proofstep{4}{\{a,b\}=\{c,d\}}{\rAEb{4}}
    \proofstep{4}{a=c}{\FormulaRefAuto{\{a\}=\{b\}\eqvdash a=b}{5}}
    \proofstep{4}{b=d}{\FormulaRefAuto{\{a,b\}=\{c,d\}, a=c \vdash b=d}{6,7}}
    \proofstep{4}{a=c\land b=d}{\rAI{7,8}}
    \proofstep{10}{\{a\}=\{c,d\}\land \{a,b\}=\{c\}}{\rA}
    \proofstep{10}{\{a\}=\{c,d\}}{\rAEa{13}}
    \proofstep{10}{\{a,b\}=\{c\}}{\rAEb{13}}
    \proofstep{10}{a=c\land a=d}{\FormulaRefAuto{\{a\}=\{b,c\}\vdash a=b\land a=c}{11}}
    \proofstep{10}{c=a\land c=b}{\FormulaRefAuto{\{a\}=\{b,c\}\vdash a=b\land a=c}{\FormulaRefAuto{a = b \vdash b = a}{12}}}
    \proofstep{10}{a=c}{\rAEa{13}}
    \proofstep{10}{a=d}{\rAEb{13}}
    \proofstep{10}{c=b}{\rAEb{14}}
    \proofstep{10}{b=d}{\FormulaRefAuto{a = b,\, a = c \vdash b = c}{17,\FormulaRefAuto{a = b,\, a = c \vdash b = c}{15,16}}}
    \proofstep{10}{a=c\land b=d}{\rAI{15,18}}
    \proofstep{1}{a=c\land b=d}{\rOE{3,4,9,10,19}}
    \closeproofpart
    \proofpart{\(\dashv\)}
    \proofstep{1}{a=c\land b=d}{\rA}
    \proofstep{1}{\{a,b\}=\{c,d\}}{\FormulaRefAuto{a=c\land b=d\vdash \{a,b\}=\{c,d\}}{1}}
    \proofstep{1}{a=c}{\rAEa{1}}
    \proofstep{1}{{a}={c}}{\FormulaRefAuto{\{a\}=\{b\}\eqvdash a=b}{3}}
    \proofstep{1}{\{\{a\},\{a,b\}\}=\{\{c\},\{c,d\}\}}{\FormulaRefAuto{a=c, b=d\vdash \{a,b\}=\{c,d\}}{4,2}}
    \closeproofpart
\end{tabproofsplit}

\FormulaThmAuto{a=c,\, b=d\vdash (a,b) = (c,d)}[Sei \(A\) eine Menge und \(a,b,c,d\in A\), dann gilt:]
\begin{tabproof}
  \proofstep{1}{a=c}{\rA}
  \proofstep{2}{b=d}{\rA}
  \proofstep{1,2}{a=c\land b=d}{\rAI{1,2}}
  \proofstep{1,2}{(a,b) = (c,d)}{\FormulaRefAuto{(a,b) = (c,d)\eqvdash a=c\land b=d}{3}}
\end{tabproof}

\FormulaThmAuto{a\neq b\vdash (c,a)\neq (d,b)}[Sei \(A\) eine Menge und \(a,b,c,d\in A\), dann gilt:]
\begin{tabproof}
  \proofstep{1}{a\neq b}{\rA}
  \proofstep{2}{(c,a)=(d,b)}{\rA}
  \proofstep{2}{a=b}{\rAEb{\FormulaRefAuto{(a,b) = (c,d)\eqvdash a=c\land b=d}{2}}}
  \proofstep{1,2}{\bot}{\rBI{1,3}}
  \proofstep{1}{(c,a)\neq (d,b)}{\rCI{2,4}}
\end{tabproof}

\FormulaThmAuto{a\neq b\vdash (a,c)\neq (b,d)}[Sei \(A\) eine Menge und \(a,b,c,d\in A\), dann gilt:]
\begin{tabproof}
  \proofstep{1}{a\neq b}{\rA}
  \proofstep{2}{(a,c)=(b,d)}{\rA}
  \proofstep{2}{a=b}{\rAEa{\FormulaRefAuto{(a,b) = (c,d)\eqvdash a=c\land b=d}{2}}}
  \proofstep{1,2}{\bot}{\rBI{1,3}}
  \proofstep{1}{(a,c)\neq (b,d)}{\rCI{2,4}}
\end{tabproof}

\subsection{Verschachtelte Paare und Tripel}

\FormulaDefAuto[Tripel (rechtsassoziiert)]{(x,y,z) := (x,(y,z))}[Seien \(x,y,z\) Elemente einer Menge \(A\). Dann definieren wir:]


% --- Kernsatz: Komponenten-Eindeutigkeit des Tripels ---
\FormulaThmAuto{(a,b,c)=(a',b',c') \eqvdash a=a' \land b=b' \land c=c'}[Seien \(a,b,c,a',b',c'\) Elemente einer Menge \(A\). Dann gilt:]
\begin{tabproofsplit}
  \proofpart{\(\vdash\)}
    \proofstep{1}{(a,b,c)=(a',b',c')}{\rA}
    \proofstep{1}{(a,(b,c))=(a',(b',c'))}{\rIE{\FormulaRefAuto{(x,y,z) := (x,(y,z))},1}}
    \proofstep{1}{a=a'\land (b,c)=(b',c')}{\FormulaRefAuto{(a,b) = (c,d)\eqvdash a=c\land b=d}{2}}
    \proofstep{1}{a=a'}{\rAEa{3}}
    \proofstep{1}{(b,c)=(b',c')}{\rAEb{3}}
    \proofstep{1}{b=b'\land c=c'}{\FormulaRefAuto{(a,b) = (c,d)\eqvdash a=c\land b=d}{5}}
    \proofstep{1}{b=b'}{\rAEa{6}}
    \proofstep{1}{c=c'}{\rAEb{6}}
    \proofstep{1}{a=a'\land b=b'}{\rAI{4,7}}
    \proofstep{1}{a=a'\land b=b'\land c=c'}{\rAI{9,8}}
  \closeproofpart

  \proofpart{\(\dashv\)}
    \proofstep{1}{a=a'\land b=b'\land c=c'}{\rA}
    \proofstep{1}{a=a'}{\rAEa{1}}
    \proofstep{1}{b=b'}{\rAEn{1}}
    \proofstep{1}{c=c'}{\rAEn{1}}
    \proofstep{1}{(b,c)=(b',c')}{\FormulaRefAuto{a=c,\, b=d\vdash (a,b) = (c,d)}{3,4}}
    \proofstep{1}{(a,(b,c))=(a',(b',c'))}{\FormulaRefAuto{a=c,\, b=d\vdash (a,b) = (c,d)}{2,5}}
    \proofstep{1}{(a,b,c)=(a',b',c')}{\rIE{\FormulaRefAuto{(x,y,z) := (x,(y,z))},6}}
  \closeproofpart
\end{tabproofsplit}

% 1) Aus komponentenweiser Gleichheit folgt Tripelgleichheit (rechtsassoziiert)
\FormulaThmAuto{a=a',\, b=b',\, c=c' \vdash (a,b,c)=(a',b',c')}[Seien \(a,b,c,a',b',c'\) Elemente einer Menge \(A\). Dann gilt:]
\begin{tabproof}
  \proofstep{1}{a=a'}{\rA}
  \proofstep{2}{b=b'}{\rA}
  \proofstep{3}{c=c'}{\rA}
  \proofstep{1,2,3}{(a,b,c)=(a',b',c')}{\FormulaRefAuto{(a,b,c)=(a',b',c') \eqvdash a=a' \land b=b' \land c=c'}\rAI{\rAI{1,2},3}}
\end{tabproof}

% 2) Komponenten-Eindeutigkeit für linksassoziierte Kodierung (Äquivalenz)
\FormulaThmAuto{((a,b),c)=((a',b'),c') \eqvdash a=a' \land b=b' \land c=c'}[Seien \(a,b,c,a',b',c'\) Elemente einer Menge \(A\). Dann gilt:]
\begin{tabproofsplit}
  \proofpart{\(\vdash\)}
    \proofstep{1}{((a,b),c)=((a',b'),c')}{\rA}
    \proofstep{1}{(a,b)=(a',b')\land c=c'}{\FormulaRefAuto{(a,b) = (c,d)\eqvdash a=c\land b=d}{1}}
    \proofstep{1}{(a,b)=(a',b')}{\rAEa{2}}
    \proofstep{1}{c=c'}{\rAEb{2}}
    \proofstep{1}{a=a'\land b=b'}{\FormulaRefAuto{(a,b) = (c,d)\eqvdash a=c\land b=d}{3}}
    \proofstep{1}{a=a'}{\rAEa{5}}
    \proofstep{1}{b=b'}{\rAEb{5}}
    \proofstep{1}{a=a'\land b=b'}{\rAI{6,7}}
    \proofstep{1}{a=a'\land b=b'\land c=c'}{\rAI{8,4}}
  \closeproofpart

  \proofpart{\(\dashv\)}
    \proofstep{1}{a=a'\land b=b'\land c=c'}{\rA}
    \proofstep{1}{a=a'}{\rAEa{1}}
    \proofstep{1}{b=b'}{\rAEn{1}}
    \proofstep{1}{c=c'}{\rAEn{1}}
    \proofstep{2,3}{(a,b)=(a',b')}{\FormulaRefAuto{a=c,\, b=d\vdash (a,b) = (c,d)}{2,3}}
    \proofstep{5,4}{((a,b),c)=((a',b'),c')}{\FormulaRefAuto{a=c,\, b=d\vdash (a,b) = (c,d)}{5,4}}
  \closeproofpart
\end{tabproofsplit}

% 3) Aus komponentenweiser Gleichheit folgt linksassoziierte Paar-Gleichheit (kurz)
\FormulaThmAuto{a=a',\, b=b',\, c=c' \vdash ((a,b),c)=((a',b'),c')}[Seien \(a,b,c,a',b',c'\) Elemente einer Menge \(A\). Dann gilt:]
\begin{tabproof}
  \proofstep{1}{a=a'}{\rA}
  \proofstep{2}{b=b'}{\rA}
  \proofstep{3}{c=c'}{\rA}
  \proofstep{1,2,3}{((a,b),c)=((a',b'),c')}{\FormulaRefAuto{((a,b),c)=((a',b'),c') \eqvdash a=a' \land b=b' \land c=c'}\rAI{\rAI{1,2},3}}
\end{tabproof}

\chapter{Die Differenz}

\FormulaDefAuto{A \setminus B := \{ x \in A \mid x \notin B \}}

\FormulaThmAuto{x \in A \setminus B \eqvdash x \in A \land x \notin B}
\begin{tabproofwide}
  \proofstepwide{x \in A \setminus B}{\leftrightarrow}{x \in \{x \in A \mid x \notin B\}}%
    {\rUE{\FormulaRefAuto{\forall x\, (x \in A \leftrightarrow x \in B) \eqvdash A = B}{\FormulaRefAuto{A \setminus B := \{ x \in A \mid x \notin B \}}{}}}}
  \proofstepwide{}{\leftrightarrow}{x \in A \land x \notin B}%
    {\FormulaRefAuto{\{x \in A \mid P(x)\} := \iota B\bigl(\forall x\,(x \in B \leftrightarrow (x \in A \land P(x)))\bigr)}{1}}
\end{tabproofwide}

\FormulaThmAuto{x \in A \setminus B \vdash x \in A}
\begin{tabproof}
  \proofstep{1}{x \in A \setminus B}{\rA}
  \proofstep{1}{x \in A \land x \notin B}{\FormulaRefAuto{x \in A \setminus B \eqvdash x \in A \land x \notin B}{1}}
  \proofstep{1}{x \in A}{\rAEa{2}}
\end{tabproof}

\FormulaThmAuto{c \in A \setminus \{a\} \vdash c \neq a}
\begin{tabproof}
  \proofstep{1}{c \in A \setminus \{a\}}{\rA}
  \proofstep{1}{c \notin \{a\}}{\rAEb{\FormulaRefAuto{x \in A \setminus B \eqvdash x \in A \land x \notin B}{1}}}
  \proofstep{1}{c \neq a}{\FormulaRefAuto{x \notin \{a\} \eqvdash x \neq a}{2}}
\end{tabproof}

\FormulaThmAuto{c \in A \setminus \{a,b\} \vdash c \neq a}
\begin{tabproof}
  \proofstep{1}{c \in A \setminus \{a,b\}}{\rA}
  \proofstep{1}{c \notin \{a,b\}}{\rAEb{\FormulaRefAuto{x \in A \setminus B \eqvdash x \in A \land x \notin B}{1}}}
  \proofstep{1}{c \neq a}{\rAEa{\FormulaRefAuto{x \notin \{a,b\} \eqvdash x \neq a \land x \neq b}{2}}}
\end{tabproof}

\FormulaThmAuto{c \in A \setminus \{a,b\} \vdash c \neq b}
\begin{tabproof}
  \proofstep{1}{c \in A \setminus \{a,b\}}{\rA}
  \proofstep{1}{c \notin \{a,b\}}{\rAEb{\FormulaRefAuto{x \in A \setminus B \eqvdash x \in A \land x \notin B}{1}}}
  \proofstep{1}{c \neq b}{\rAEb{\FormulaRefAuto{x \notin \{a,b\} \eqvdash x \neq a \land x \neq b}{2}}}
\end{tabproof}

\FormulaThmAuto{c \in A \setminus B,\, b \in B \vdash c \neq b}
\begin{tabproof}
  \proofstep{1}{c \in A \setminus B}{\rA}
  \proofstep{2}{b \in B}{\rA}
  \proofstep{1}{c \notin B}{\rAEb{\FormulaRefAuto{x \in A \setminus B \eqvdash x \in A \land x \notin B}{1}}}
  \proofstep{1,2}{c \neq b}{\FormulaRefAuto{ a \in A,\; b \not\in A \vdash a \neq b }{2,3}}
\end{tabproof}

\FormulaThmAuto{a \notin A \setminus \{a\}}
\begin{tabproof}
  \proofstep{1}{a \in A \setminus \{a\}}{\rA}
  \proofstep{}{a = a}{\rII}
  \proofstep{1}{a \neq a}{\FormulaRefAuto{c \in A \setminus \{a\} \vdash c \neq a}{1}}
  \proofstep{1}{\bot}{\rBI{2,3}}
  \proofstep{}{a \notin A \setminus \{a\}}{\rCI{1,4}}
\end{tabproof}

\FormulaThmAuto{a \notin A \setminus \{a,b\}}
\begin{tabproof}
  \proofstep{1}{a \in A \setminus \{a,b\}}{\rA}
  \proofstep{}{a = a}{\rII}
  \proofstep{1}{a \neq a}{\FormulaRefAuto{c \in A \setminus \{a,b\} \vdash c \neq a}{1}}
  \proofstep{1}{\bot}{\rBI{2,3}}
  \proofstep{}{a \notin A \setminus \{a,b\}}{\rCI{1,4}}
\end{tabproof}

\FormulaThmAuto{b \notin A \setminus \{a,b\}}
\begin{tabproof}
  \proofstep{1}{b \in A \setminus \{a,b\}}{\rA}
  \proofstep{}{b = b}{\rII}
  \proofstep{1}{b \neq b}{\FormulaRefAuto{c \in A \setminus \{a,b\} \vdash c \neq b}{1}}
  \proofstep{1}{\bot}{\rBI{2,3}}
  \proofstep{}{b \notin A \setminus \{a,b\}}{\rCI{1,4}}
\end{tabproof}

\FormulaThmAuto{a \in A,\, a \neq b \vdash a \in A \setminus \{b\}}
\begin{tabproof}
  \proofstep{1}{a \in A}{\rA}
  \proofstep{2}{a \neq b}{\rA}
  \proofstep{2}{a \notin \{b\}}{\FormulaRefAuto{x \notin \{a\} \eqvdash x \neq a}{2}}
  \proofstep{1,2}{a \in A \setminus \{b\}}{\FormulaRefAuto{x \in A \setminus B \eqvdash x \in A \land x \notin B}{\rAI{1,3}}}
\end{tabproof}

\FormulaThmAuto{a \in A,\, b \neq a \vdash a \in A \setminus \{b\}}
\begin{tabproof}
  \proofstep{1}{a \in A}{\rA}
  \proofstep{2}{b \neq a}{\rA}
  \proofstep{2}{a \neq b}{\FormulaRefAuto{a \neq b \vdash b \neq a}{2}}
  \proofstep{1,2}{a \in A \setminus \{b\}}{\FormulaRefAuto{a \in A,\, a \neq b \vdash a \in A \setminus \{b\}}{1,3}}
\end{tabproof}

\FormulaThmAuto{A\setminus B\subseteq A}
\begin{tabproof}
  \proofstep{1}{x\in A\setminus B}{\rA}
  \proofstep{1}{x\in A}{\FormulaRefAuto{x \in A \setminus B \vdash x \in A}{2}}
  \proofstep{1}{A\setminus B\subseteq A}{\FormulaRefAuto{A \subseteq B := \forall x\,(x \in A \rightarrow x \in B)}{\rUI{\rRI{1,2}}}}
\end{tabproof}


\FormulaThmAuto{A\subseteq C\vdash A\setminus B\subseteq C}
\begin{tabproof}
  \proofstep{1}{A\subseteq C}{\rA}
  \proofstep{2}{x\in A\setminus B}{\rA}
  \proofstep{2}{x\in A}{\FormulaRefAuto{x \in A \setminus B \vdash x \in A}{2}}
  \proofstep{1,2}{x\in C}{\FormulaRefAuto{A\subseteq B, x\in A\vdash x\in B}{1,3}}
  \proofstep{1,2}{A\setminus B\subseteq C}{\FormulaRefAuto{A \subseteq B := \forall x\,(x \in A \rightarrow x \in B)}{\rUI{\rRI{2,4}}}}
\end{tabproof}

\FormulaThmAuto{A=A\setminus \{a\}\vdash a\notin A}
\begin{tabproof}
  \proofstep{1}{A=A\setminus \{a\}}{\rA}
  \proofstep{}{a\notin A\setminus \{a\}}{\FormulaRefAuto{a \notin A \setminus \{a\}}}
  \proofstep{1}{a\notin A}{\rIE{1,2}}
\end{tabproof}


\chapter{Vereinigung}

\section{Axiom der Vereinigung}

\FormulaAxiomDelta[Vereinigung]{\exists C\;\forall x\;\bigl(x \in C \;\leftrightarrow\;\exists B\,(B \in A \,\land\, x \in B)\bigr)}{
\DeltaRow{Mengen}{A}%
}

\section{Definition der Vereinigung}
%%begin novalidate
\FormulaDefDelta[Vereinigung]{\bigcup A := \iota C\Bigl(\forall x\;\bigl(x \in C \;\leftrightarrow\;\exists B\,(B \in A \land x \in B)\bigr)\Bigr)}{
\DeltaRow{Mengen}{A}%
}
%%end novalidate

\FormulaThmDelta{x \in \bigcup A \;\eqvdash\;\exists B\,(B \in A \,\land\, x \in B)}%
{%
\DeltaRow{Mengen}{x\dsep A\dsep B}%
}
\begin{tabproof}
  \proofstep{}{ \forall x\;\bigl(x \in \bigcup A \;\leftrightarrow\;\exists B\,(B \in A \land x \in B)\bigr) }{\FormulaRefAuto{\bigcup A := \iota C\Bigl(\forall x\;\bigl(x \in C \;\leftrightarrow\;\exists B\,(B \in A \land x \in B)\bigr)\Bigr)}}
  \proofstep{}{ x \in \bigcup A \;\leftrightarrow\;\exists B\,(B \in A \land x \in B) }{\rUE{1}}
\end{tabproof}

\section{Grundlegende Eigenschaften}

\FormulaDefAuto[Vereinigung zweier Mengen]{A \cup B := \bigcup \{A, B\}}

\FormulaThmAuto{x\in A \cup B \eqvdash x\in \bigcup \{A, B\}}
\begin{tabproof}
\proofstep{}{z \in A \cup B\leftrightarrow z \in \bigcup \{A, B\}}{\rUE{\FormulaRefAuto{\forall x\, (x \in A \leftrightarrow x \in B) \eqvdash A = B}{\FormulaRefAuto{A \cup B := \bigcup \{A, B\}}}}}
\end{tabproof}

\FormulaThmAuto{A \subseteq B \vdash \bigcup A \subseteq \bigcup B}
\begin{tabproof}
  \proofstep{1}{A \subseteq B}{\rA}
  \proofstep{2}{x \in \bigcup A}{\rA}
  \proofstep{2}{\exists X\,(X \in A \land x \in X)}{\FormulaRefAuto{\bigcup A := \iota C\bigl(\forall x\,(x \in C \leftrightarrow \exists B\,(B \in A \land x \in B))\bigr)}{2}}
  \proofstep{4}{C \in A \land x \in C}{\rA}
  \proofstep{4}{C \in A}{\rAEa{4}}
  \proofstep{4}{x \in C}{\rAEb{4}}
  \proofstep{1,4}{C \in B}{\FormulaRefAuto{A \subseteq B,\, x \in A \vdash x \in B}{5,1}}
  \proofstep{1,4}{x \in \bigcup B}{\FormulaRefAuto{\bigcup A := \iota C\bigl(\forall x\,(x \in C \leftrightarrow \exists B\,(B \in A \land x \in B))\bigr)}{\rEI{\rAI{7,6}}}}
  \proofstep{1,2}{x \in \bigcup B}{\rEE{3,4,8}}
  \proofstep{1}{\bigcup A \subseteq \bigcup B}{\FormulaRefAuto{A \subseteq B := \forall x\,(x \in A \rightarrow x \in B)}{\rUI{\rRE{2,9}}}}
\end{tabproof}

\FormulaThmAuto{z \in A \cup B \eqvdash z \in A \lor z \in B}
\begin{tabproofwide}
  \proofstepwide{z \in A \cup B}{\leftrightarrow}{z \in \bigcup \{A, B\}}%
    {\FormulaRefAuto{x\in A \cup B \eqvdash x\in \bigcup \{A, B\}}}
  \proofstepwide{}{\leftrightarrow}{\exists C\,(C \in \{A, B\} \land z \in C)}%
    {\FormulaRefAuto{\bigcup A := \iota C\bigl(\forall x\,(x \in C \leftrightarrow \exists B\,(B \in A \land x \in B))\bigr)}{1}}
  \proofstepwide{}{\leftrightarrow}{\exists C\,((C = A \lor C = B) \land z \in C)}%
    {\rLRS{\FormulaRefAuto{\{A,B\} := \iota C\bigl(\forall x\,(x \in C \leftrightarrow x = A \lor x = B)\bigr)}{}}}
  \proofstepwide{}{\leftrightarrow}{z \in A \lor z \in B}%
    {\FormulaRefAuto{\exists c\,((c = a \lor c = b) \land P(c)) \dashv \vdash P(a) \lor P(b)}{3}}
\end{tabproofwide}

\FormulaThmAuto{z \in A \vdash z \in A \cup B}
\begin{tabproof}
  \proofstep{1}{z \in A}{\rA}
  \proofstep{1}{z \in A \lor z \in B}{\rOIa{1}}
  \proofstep{1}{z \in A \cup B}{\FormulaRefAuto{z \in A \cup B \eqvdash z \in A \lor z \in B}{2}}
\end{tabproof}

\FormulaThmAuto{z \in B \vdash z \in A \cup B}
\begin{tabproof}
  \proofstep{1}{z \in B}{\rA}
  \proofstep{1}{z \in A \lor z \in B}{\rOIb{1}}
  \proofstep{1}{z \in A \cup B}{\FormulaRefAuto{z \in A \cup B \eqvdash z \in A \lor z \in B}{2}}
\end{tabproof}

\FormulaThmAuto{x \in \{a,b\} \eqvdash x \in \{a\} \cup \{b\}}
\begin{tabproofwide}
  \proofstepwide{x \in \{a\}}{\leftrightarrow}{x = a}%
    {\FormulaRefAuto{x \in \{a\} \eqvdash x = a}}
  \proofstepwide{x \in \{b\}}{\leftrightarrow}{x = b}%
    {\FormulaRefAuto{x \in \{a\} \eqvdash x = a}}
  \proofstepwide{B \in \{\{a\},\{b\}\}}{\leftrightarrow}{B = \{a\} \lor B = \{b\}}%
    {\FormulaRefAuto{\{A,B\} := \iota C\bigl(\forall x\,(x \in C \leftrightarrow x = A \lor x = B)\bigr)}}
  \proofstepwide{x \in \{a,b\}}{\leftrightarrow}{x = a \lor x = b}%
    {\FormulaRefAuto{\{A,B\} := \iota C\bigl(\forall x\,(x \in C \leftrightarrow x = A \lor x = B)\bigr)}}
  \proofstepwide{}{\leftrightarrow}{x \in \{a\} \lor x = b}{\rLRS{1}}
  \proofstepwide{}{\leftrightarrow}{x \in \{a\} \lor x \in \{b\}}{\rLRS{2}}

  % --- Zeile 7: auf zwei Zeilen gesplittet ---
  \proofstepwide{}{\leftrightarrow}{\exists B\,\bigl(B = \{a\} \lor B = \{b\}\bigr)}%
    {\multirow{2}{*}{\FormulaRefAuto{\exists c\,((c = a \lor c = b) \land P(c)) \dashv \vdash P(a) \lor P(b)}{6}}}
  \proofstepwide*{}{\land}{x \in B}{}
  % -------------------------------------------

  \proofstepwide{}{\leftrightarrow}{\exists B\,(B \in \{\{a\},\{b\}\} \land x \in B)}{\rLRS{3}}
  \proofstepwide{}{\leftrightarrow}{x \in \bigcup \{\{a\},\{b\}\}}%
    {\FormulaRefAuto{\bigcup A := \iota C\bigl(\forall x\,(x \in C \leftrightarrow \exists B\,(B \in A \land x \in B))\bigr)}{8}}
  \proofstepwide{}{\leftrightarrow}{x \in \{a\} \cup \{b\}}%
    {\FormulaRefAuto{A \cup B := \bigcup \{A, B\}}{8}}
  \proofstepwide{x \in \{a,b\}}{\leftrightarrow}{x \in \{a\} \cup \{b\}}{\rChain{4,10}}
\end{tabproofwide}

\FormulaThmAuto{\{a,b\} = \{a\} \cup \{b\}}
\begin{tabproof}
  \proofstep{}{x \in \{a,b\} \leftrightarrow x \in \{a\} \cup \{b\}}{\FormulaRefAuto{x \in \{a,b\} \eqvdash x \in \{a\} \cup \{b\}}}
  \proofstep{}{\{a,b\} = \{a\} \cup \{b\}}{\FormulaRefAuto{\forall x\,(x \in A \leftrightarrow x \in B) \eqvdash A = B}{\rUI{1}}}
\end{tabproof}

\FormulaThmAuto{a \in A \cup \{a\}}
\begin{tabproof}
  \proofstep{}{a \in \{a\}}{\FormulaRefAuto{a \in \{a\}}}
  \proofstep{}{a \in A \cup \{a\}}{\FormulaRefAuto{z \in B \vdash z \in A \cup B}{1}}
\end{tabproof}

\FormulaThmAuto{a \in \{a\} \cup A}
\begin{tabproof}
  \proofstep{}{a \in \{a\}}{\FormulaRefAuto{a \in \{a\}}}
  \proofstep{}{a \in \{a\} \cup A}{\FormulaRefAuto{z \in A \vdash z \in A \cup B}{1}}
\end{tabproof}

\FormulaThmAuto{\{a\} \cup A\neq\emptyset}
\begin{tabproof}
  \proofstep{}{a \in \{a\} \cup A}{\FormulaRefAuto{a \in \{a\} \cup A}}
  \proofstep{}{\{a\} \cup A\neq\emptyset}{\FormulaRefAuto{a\in S\vdash S\neq\emptyset}}
\end{tabproof}

\FormulaThmAuto{A\cup \{a\}\neq\emptyset}
\begin{tabproof}
  \proofstep{}{a \in A\cup \{a\}}{\FormulaRefAuto{a \in A\cup \{a\}}}
  \proofstep{}{\{a\} \cup A\neq\emptyset}{\FormulaRefAuto{a\in S\vdash S\neq\emptyset}}
\end{tabproof}


\FormulaThmAuto{A\cup \{a\}=A\eqvdash a\in A}
\begin{tabproofsplit}
  \proofpart{$\vdash$}
    \proofstep{1}{A \cup \{a\} = A}{\rA}
    \proofstep{}{a \in A \cup \{a\}}{\FormulaRefAuto{a \in A \cup \{a\}}}
    \proofstep{}{a \in A}{\rIE{1,2}}
  \closeproofpart

  \proofpart{$\dashv$}
    % Annahme
    \proofstep{1}{a \in A}{\rA}

    % (i) A \cup {a} \subseteq A
    \proofstep{2}{x \in A \cup \{a\}}{\rA}
    \proofstep{2}{x \in A \lor x \in \{a\}}%
      {\FormulaRefAuto{z \in A \cup B \eqvdash z \in A \lor z \in B}{2}}

    % Fall 1: x \in A
    \proofstep{3}{x \in A}{\rA}

    % Fall 2: x \in {a}  =>  x = a  =>  x \in A
    \proofstep{5}{x \in \{a\}}{\rA}
    \proofstep{5}{x = a}{\FormulaRefAuto{x \in \{a\} \eqvdash x = a}{5}}
    \proofstep{1,5}{x \in A}{\rIE{6,1}}

    % Fälle zusammenführen
    \proofstep{1,2}{x \in A}{\rOE{3,4,4,5,7}}

    % Universalisierung: Teilmengenrichtung
    \proofstep{}{A \cup \{a\} \subseteq A}{\FormulaRefAuto{A \subseteq B := \forall x\,(x \in A \rightarrow x \in B)}{\rUI{\rRI{2,8}}}}

    % (ii) A \subseteq A \cup {a}
    \proofstep{}{A \subseteq A \cup \{a\}}%
      {\FormulaRefAuto{z \in A \vdash z \in A \cup B}}

    % (iii) Gleichheit aus beidseitiger Inklusion
    \proofstep{}{A \cup \{a\} = A}{\FormulaRefAuto{A \subseteq B \land B \subseteq A \eqvdash A = B}{9,10}}
  \closeproofpart
\end{tabproofsplit}


\FormulaThmAuto{a \notin B,\, A = B \cup \{a\} \vdash A \not\subseteq B}
\begin{tabproof}
  \proofstep{1}{a \notin B}{\rA}
  \proofstep{2}{A = B \cup \{a\}}{\rA}
  \proofstep{3}{A \subseteq B}{\rA}
  \proofstep{}{a \in B \cup \{a\}}{\FormulaRefAuto{a \in A \cup \{a\}}}
  \proofstep{2}{a \in A}{\rIE{2,4}}
  \proofstep{2,3}{a \in B}{\FormulaRefAuto{A \subseteq B,\, x \in A \vdash x \in B}{5,3}}
  \proofstep{1,2,3}{\bot}{\rBI{1,6}}
  \proofstep{1,2}{A \not\subseteq B}{\rCI{3,7}}
\end{tabproof}

\FormulaThmAuto{x \in A \eqvdash x \in A \cup A}
\begin{tabproofwide}
  \proofstepwide{x \in A}{\leftrightarrow}{x \in A \lor x \in A}%
    {\FormulaRefAuto{P \lor P \eqvdash P}}
  \proofstepwide{}{\leftrightarrow}{x \in A \cup A}%
    {\FormulaRefAuto{z \in A \cup B \eqvdash z \in A \lor z \in B}}
  \proofstepwide{x \in A}{\leftrightarrow}{x \in A \cup A}%
    {\rChain{1,2}}
\end{tabproofwide}

\FormulaThmAuto{A = A \cup A}
\begin{tabproofwide}
  \proofstepwide{x \in A}{\leftrightarrow}{x \in A \cup A}%
    {\FormulaRefAuto{x \in A \eqvdash x \in A \cup A}}
  \proofstepwide{A}{=}{A \cup A}%
    {\FormulaRefAuto{\forall x\, (x \in A \leftrightarrow x \in B) \eqvdash A = B}{\rUI{1}}}
\end{tabproofwide}

\FormulaThmAuto{x \in A \cup B \eqvdash x \in B \cup A}
\begin{tabproofwide}
  \proofstepwide{x \in A \cup B}{\leftrightarrow}{x \in A \lor x \in B}%
    {\FormulaRefAuto{z \in A \cup B \eqvdash z \in A \lor z \in B}}
  \proofstepwide{}{\leftrightarrow}{x \in B \lor x \in A}%
    {\FormulaRefAuto{P \lor Q \vdash Q \lor P}{1}}
  \proofstepwide{}{\leftrightarrow}{x \in B \cup A}%
    {\FormulaRefAuto{z \in A \cup B \eqvdash z \in A \lor z \in B}{2}}
  \proofstepwide{x \in A \cup B}{\leftrightarrow}{x \in B \cup A}%
    {\rChain{1,3}}
\end{tabproofwide}

\FormulaThmAuto{A \cup B = B \cup A}
\begin{tabproofwide}
  \proofstepwide{x \in A \cup B}{\leftrightarrow}{x \in B \cup A}%
    {\FormulaRefAuto{x \in A \cup B \eqvdash x \in B \cup A}}
  \proofstepwide{A \cup B}{=}{B \cup A}%
    {\FormulaRefAuto{\forall x\, (x \in A \leftrightarrow x \in B) \eqvdash A = B}{\rUI{1}}}
\end{tabproofwide}

\FormulaThmAuto{x \in A \eqvdash x \in A \cup \emptyset}
\begin{tabproofwide}
  \proofstepwide{x \in A}{\leftrightarrow}{x \in A \lor x \in \emptyset}%
    {\FormulaRefAuto{\forall x(\neg Q(x)) \vdash P \leftrightarrow P \lor Q(a)}{\FormulaRefAuto{\emptyset := \iota O\bigl(\forall x\,(x \not\in O)\bigr)}}}
  \proofstepwide{}{\leftrightarrow}{x \in A \cup \emptyset}%
    {\FormulaRefAuto{z \in A \cup B \eqvdash z \in A \lor z \in B}{1}}
  \proofstepwide{x \in A}{\leftrightarrow}{x \in A \cup \emptyset}%
    {\rChain{1,2}}
\end{tabproofwide}

\FormulaThmAuto{A = A \cup \emptyset}
\begin{tabproofwide}
  \proofstepwide{x \in A}{\leftrightarrow}{x \in A \cup \emptyset}%
    {\FormulaRefAuto{x \in A \eqvdash x \in A \cup \emptyset}}
  \proofstepwide{A}{=}{A \cup \emptyset}%
    {\FormulaRefAuto{\forall x\, (x \in A \leftrightarrow x \in B) \eqvdash A = B}{\rUI{1}}}
\end{tabproofwide}

\FormulaThmAuto{A = \emptyset \cup A}
\begin{tabproofwide}
  \proofstepwide{A}{=}{A \cup \emptyset}%
    {\FormulaRefAuto{A = A \cup \emptyset}}
  \proofstepwide{A \cup \emptyset}{=}{\emptyset \cup A}%
    {\FormulaRefAuto{A \cup B = B \cup A}}
  \proofstepwide{A}{=}{\emptyset \cup A}%
    {\rIE{2,1}}
\end{tabproofwide}

\FormulaThmAuto{A \cup \{A\} = \{\emptyset\} \eqvdash A = \emptyset}
\begin{tabproofsplit}
\proofpart{\(\vdash\)}
  \proofstep{1}{\{\emptyset\} = A \cup \{A\}}{\rA}
  \proofstep{}{A \in A \cup \{A\}}{\FormulaRefAuto{a \in A \cup \{a\}}}
  \proofstep{1}{A \in \{\emptyset\}}{\FormulaRefAuto{A = B,\, x \in A \vdash x \in B}{1,2}}
  \proofstep{1}{A = \emptyset}{\FormulaRefAuto{x \in \{a\} \eqvdash x = a}{3}}
\closeproofpart

\proofpart{\(\dashv\)}
  \proofstep{1}{A = \emptyset}{\rA}
  \proofstep{}{\{\emptyset\} = \emptyset \cup \{\emptyset\}}{\FormulaRefAuto{A = \emptyset \cup A}}
  \proofstep{1}{A \cup \{A\} = \{\emptyset\}}{\rIE{1,\FormulaRefAuto{a = b \vdash b = a}{2}}}
\closeproofpart
\end{tabproofsplit}



\FormulaThmAuto{z \in A \cup B \eqvdash z \not\in A \rightarrow z \in B}
\begin{tabproofwide}
  \proofstepwide{z \in A \cup B}{\leftrightarrow}{z \in A \lor z \in B}%
    {\FormulaRefAuto{z \in A \cup B \eqvdash z \in A \lor z \in B}}
  \proofstepwide{}{ \leftrightarrow }{z \not\in A \rightarrow z \in B}%
    {\FormulaRefAuto{P \rightarrow Q \dashv \vdash \neg P \lor Q}{1}}
\end{tabproofwide}

\FormulaThmAuto{z \in A \cup B \eqvdash z \not\in B \rightarrow z \in A}
\begin{tabproofwide}
  \proofstepwide{z \in A \cup B}{\leftrightarrow}{z \in A \lor z \in B}%
    {\FormulaRefAuto{z \in A \cup B \eqvdash z \in A \lor z \in B}}
  \proofstepwide{}{\leftrightarrow}{z \not\in B \rightarrow z \in A}%
    {\FormulaRefAuto{P \rightarrow Q \dashv \vdash \neg P \lor Q}{1}}
\end{tabproofwide}

\FormulaThmAuto{A=B\vdash A\cup C = B\cup C}
\begin{tabproof}
    \proofstep{1}{A=B}{\rA}
    \proofstep{1}{x\in A\cup C\leftrightarrow x\in B\cup C}{\rIE{1,\FormulaRefAuto{P\leftrightarrow P}}}
    \proofstep{1}{A\cup C=B\cup C}{\FormulaRefAuto{\forall x\, (x \in A \leftrightarrow x \in B) \eqvdash A = B}{\rUI{2}}}
\end{tabproof}


\FormulaThmAuto{z \in (A \cup B) \cup C \eqvdash (z \in A \lor z \in B) \lor z \in C}
\begin{tabproofwide}
  % Zeile 1 -> zwei Zeilen
  \proofstepwide{z \in (A \cup B) \cup C}{\leftrightarrow}{z \in (A \cup B)}%
    {\multirow{2}{*}{\FormulaRefAuto{z \in A \cup B \eqvdash z \in A \lor z \in B}}}
  \proofstepwide*{}{\lor}{z \in C}{}

  % Zeile 2 -> zwei Zeilen
  \proofstepwide{}{\leftrightarrow}{(z \in A \lor z \in B)}%
    {\multirow{2}{*}{\rLRS{\FormulaRefAuto{z \in A \cup B \eqvdash z \in A \lor z \in B}{},1}}}
  \proofstepwide*{}{\lor}{z \in C}{}
\end{tabproofwide}

\FormulaThmAuto{z \in A \cup (B \cup C) \eqvdash z \in A \lor (z \in B \lor z \in C)}
\begin{tabproofwide}
  % Zeile 1 -> zwei Zeilen
  \proofstepwide{z \in A \cup (B \cup C)}{\leftrightarrow}{z \in A}%
    {\multirow{2}{*}{\FormulaRefAuto{z \in A \cup B \eqvdash z \in A \lor z \in B}}}
  \proofstepwide*{}{\lor}{z \in (B \cup C)}{}

  % Zeile 2 -> zwei Zeilen
  \proofstepwide{}{\leftrightarrow}{z \in A}%
    {\multirow{2}{*}{\rLRS{\FormulaRefAuto{z \in A \cup B \eqvdash z \in A \lor z \in B}{},1}}}
  \proofstepwide*{}{\lor}{(z \in B \lor z \in C)}{}
\end{tabproofwide}

\FormulaThmAuto{z \in (A \cup B) \cup C \eqvdash z \in A \cup (B \cup C)}
\begin{tabproofwide}
  \proofstepwide{z \in (A \cup B) \cup C}{\leftrightarrow}{(z \in A \lor z \in B) \lor z \in C}%
    {\FormulaRefAuto{z \in (A \cup B) \cup C \eqvdash (z \in A \lor z \in B) \lor z \in C}}
  \proofstepwide{}{\leftrightarrow}{z \in A \lor (z \in B \lor z \in C)}%
    {\FormulaRefAuto{P \lor (Q \lor R) \eqvdash (P \lor Q) \lor R}{1}}
  \proofstepwide{}{\leftrightarrow}{z \in A \cup (B \cup C)}%
    {\FormulaRefAuto{z \in A \cup (B \cup C) \eqvdash z \in A \lor (z \in B \lor z \in C)}{2}}
\end{tabproofwide}

\FormulaThmAuto[Assoziativität der Vereinigung]{(A \cup B) \cup C = A \cup (B \cup C)}
\begin{tabproofwide}
  \proofstepwide{z \in (A \cup B) \cup C}{\leftrightarrow}{z \in A \cup (B \cup C)}%
    {\FormulaRefAuto{z \in (A \cup B) \cup C \eqvdash z \in A \cup (B \cup C)}}
  \proofstepwide{(A \cup B) \cup C}{=}{A \cup (B \cup C)}%
    {\FormulaRefAuto{\forall x\, (x \in A \leftrightarrow x \in B) \eqvdash A = B}{\rUI{1}}}
\end{tabproofwide}

\FormulaThmAuto{z \in A \cup (B \cap C) \eqvdash z \in A \lor (z \in B \land z \in C)}
\begin{tabproofwide}
  % Zeile 1 -> zwei Zeilen
  \proofstepwide{z \in A \cup (B \cap C)}{\leftrightarrow}{z \in A}%
    {\multirow{2}{*}{\FormulaRefAuto{z \in A \cup B \eqvdash z \in A \lor z \in B}}}
  \proofstepwide*{}{\lor}{z \in (B \cap C)}{}

  % Zeile 2 -> zwei Zeilen
  \proofstepwide{}{\leftrightarrow}{z \in A}%
    {\multirow{2}{*}{\rLRS{\FormulaRefAuto{x \in A \cap B \eqvdash x \in A \land x \in B}{},1}}}
  \proofstepwide*{}{\lor}{(z \in B \land z \in C)}{}
\end{tabproofwide}

\FormulaThmAuto{z \in (A \cap B) \cup C \eqvdash (z \in A \land z \in B) \lor z \in C}
\begin{tabproofwide}
  % Zeile 1 -> zwei Zeilen
  \proofstepwide{z \in (A \cap B) \cup C}{\leftrightarrow}{z \in (A \cap B)}%
    {\multirow{2}{*}{\FormulaRefAuto{z \in A \cup B \eqvdash z \in A \lor z \in B}}}
  \proofstepwide*{}{\lor}{z \in C}{}

  % Zeile 2 -> zwei Zeilen
  \proofstepwide{}{\leftrightarrow}{(z \in A \land z \in B)}%
    {\multirow{2}{*}{\rLRS{\FormulaRefAuto{x \in A \cap B \eqvdash x \in A \land x \in B}{},1}}}
  \proofstepwide*{}{\lor}{z \in C}{}
\end{tabproofwide}

\FormulaThmAuto{z \in A \cap (B \cup C) \eqvdash z \in A \land (z \in B \lor z \in C)}
\begin{tabproofwide}
  % Zeile 1 -> zwei Zeilen
  \proofstepwide{z \in A \cap (B \cup C)}{\leftrightarrow}{z \in A}%
    {\multirow{2}{*}{\FormulaRefAuto{x \in A \cap B \eqvdash x \in A \land x \in B}}}
  \proofstepwide*{}{\land}{z \in (B \cup C)}{}

  % Zeile 2 -> zwei Zeilen
  \proofstepwide{}{\leftrightarrow}{z \in A}%
    {\multirow{2}{*}{\rLRS{\FormulaRefAuto{z \in A \cup B \eqvdash z \in A \lor z \in B}{},1}}}
  \proofstepwide*{}{\land}{(z \in B \lor z \in C)}{}
\end{tabproofwide}

\FormulaThmAuto{z \in (A \cup B) \cap C \eqvdash (z \in A \lor z \in B) \land z \in C}
\begin{tabproofwide}
  % Zeile 1 -> zwei Zeilen
  \proofstepwide{z \in (A \cup B) \cap C}{\leftrightarrow}{z \in (A \cup B)}%
    {\multirow{2}{*}{\FormulaRefAuto{x \in A \cap B \eqvdash x \in A \land x \in B}}}
  \proofstepwide*{}{\land}{z \in C}{}

  % Zeile 2 -> zwei Zeilen
  \proofstepwide{}{\leftrightarrow}{(z \in A \lor z \in B)}%
    {\multirow{2}{*}{\rLRS{\FormulaRefAuto{z \in A \cup B \eqvdash z \in A \lor z \in B}{},1}}}
  \proofstepwide*{}{\land}{z \in C}{}
\end{tabproofwide}

\FormulaThmAuto{z \in (A \cup B) \cap (C \cup D) \eqvdash (z \in A \lor z \in B) \land (z \in C \lor z \in D)}
\begin{tabproofwide}
  % Zeile 1 -> zwei Zeilen
  \proofstepwide{z \in (A \cup B) \cap (C \cup D)}{\leftrightarrow}{z \in (A \cup B)}%
    {\multirow{2}{*}{\FormulaRefAuto{x \in A \cap B \eqvdash x \in A \land x \in B}}}
  \proofstepwide*{}{\land}{z \in (C \cup D)}{}

  % Zeile 2 -> zwei Zeilen
  \proofstepwide{}{\leftrightarrow}{(z \in A \lor z \in B)}%
    {\multirow{2}{*}{\rLRS{\FormulaRefAuto{z \in A \cup B \eqvdash z \in A \lor z \in B}{},1}}}
  \proofstepwide*{}{\land}{(z \in C \lor z \in D)}{}
\end{tabproofwide}

\FormulaThmAuto{z \in (A \cap B) \cup (C \cap D) \eqvdash (z \in A \land z \in B) \lor (z \in C \land z \in D)}
\begin{tabproofwide}
  % Zeile 1 -> zwei Zeilen
  \proofstepwide{z \in (A \cap B) \cup (C \cap D)}{\leftrightarrow}{z \in (A \cap B)}%
    {\multirow{2}{*}{\FormulaRefAuto{z \in A \cup B \eqvdash z \in A \lor z \in B}}}
  \proofstepwide*{}{\lor}{z \in (C \cap D)}{}

  % Zeile 2 -> zwei Zeilen
  \proofstepwide{}{\leftrightarrow}{(z \in A \land z \in B)}%
    {\multirow{2}{*}{\rLRS{\FormulaRefAuto{x \in A \cap B \eqvdash x \in A \land x \in B}{},1}}}
  \proofstepwide*{}{\lor}{(z \in C \land z \in D)}{}
\end{tabproofwide}

\FormulaThmAuto{z \in A \cap (B \cup C) \eqvdash z \in (A \cap B) \cup (A \cap C)}
\begin{tabproofwide}
  % Zeile 1 -> zwei Zeilen
  \proofstepwide{z \in A \cap (B \cup C)}{\leftrightarrow}{z \in A}%
    {\multirow{2}{*}{\FormulaRefAuto{z \in A \cap (B \cup C) \eqvdash z \in A \land (z \in B \lor z \in C)}}}
  \proofstepwide*{}{\land}{(z \in B \lor z \in C)}{}

  % Zeile 2 -> zwei Zeilen
  \proofstepwide{}{\leftrightarrow}{(z \in A \land z \in B)}%
    {\multirow{2}{*}{\FormulaRefAuto{P \land (Q \lor R) \dashv \vdash (P \land Q) \lor (P \land R)}{1}}}
  \proofstepwide*{}{\lor}{(z \in A \land z \in C)}{}

  % Zeile 3 -> zwei Zeilen
  \proofstepwide{}{\leftrightarrow}{z \in (A \cap B)}%
    {\multirow{2}{*}{\FormulaRefAuto{z \in (A \cap B) \cup (C \cap D) \eqvdash (z \in A \land z \in B) \lor (z \in C \land z \in D)}{2}}}
  \proofstepwide*{}{\cup}{z \in (A \cap C)}{}

  % Abschluss (unverändert)
  \proofstepwide{z \in A \cap (B \cup C)}{\leftrightarrow}{z \in (A \cap B) \cup (A \cap C)}%
    {\rChain{1,3}}
\end{tabproofwide}

\FormulaThmAuto{z \in (A \cup B) \cap C \eqvdash z \in (A \cap C) \cup (B \cap C)}
\begin{tabproofwide}
  % Zeile 1 -> zwei Zeilen
  \proofstepwide{z \in (A \cup B) \cap C}{\leftrightarrow}{(z \in A \lor z \in B)}%
    {\multirow{2}{*}{\FormulaRefAuto{z \in (A \cup B) \cap C \eqvdash (z \in A \lor z \in B) \land z \in C}}}
  \proofstepwide*{}{\land}{z \in C}{}

  % Zeile 2 -> zwei Zeilen
  \proofstepwide{}{\leftrightarrow}{(z \in A \land z \in C)}%
    {\multirow{2}{*}{\FormulaRefAuto{(P \lor Q) \land R \dashv \vdash (P \land R) \lor (Q \land R)}{1}}}
  \proofstepwide*{}{\lor}{(z \in B \land z \in C)}{}

  % Zeile 3 -> zwei Zeilen
  \proofstepwide{}{\leftrightarrow}{z \in (A \cap C)}%
    {\multirow{2}{*}{\FormulaRefAuto{z \in (A \cap B) \cup (C \cap D) \eqvdash (z \in A \land z \in B) \lor (z \in C \land z \in D)}{2}}}
  \proofstepwide*{}{\cup}{z \in (B \cap C)}{}

  % Abschluss
  \proofstepwide{z \in (A \cup B) \cap C}{\leftrightarrow}{z \in (A \cap C) \cup (B \cap C)}%
    {\rChain{1,3}}
\end{tabproofwide}

\FormulaThmAuto{z \in (A \cap B) \cup C \eqvdash z \in (A \cup C) \cap (B \cup C)}
\begin{tabproofwide}
  % Zeile 1 -> zwei Zeilen
  \proofstepwide{z \in (A \cap B) \cup C}{\leftrightarrow}{(z \in A \land z \in B)}%
    {\multirow{2}{*}{\FormulaRefAuto{z \in (A \cap B) \cup C \eqvdash (z \in A \land z \in B) \lor z \in C}}}
  \proofstepwide*{}{\lor}{z \in C}{}

  % Zeile 2 -> zwei Zeilen
  \proofstepwide{}{\leftrightarrow}{(z \in A \lor z \in C)}%
    {\multirow{2}{*}{\FormulaRefAuto{(P \land Q) \lor R \dashv \vdash (P \lor R) \land (Q \lor R)}{1}}}
  \proofstepwide*{}{\land}{(z \in B \lor z \in C)}{}

  % Zeile 3 -> zwei Zeilen
  \proofstepwide{}{\leftrightarrow}{z \in (A \cup C)}%
    {\multirow{2}{*}{\FormulaRefAuto{z \in (A \cup B) \cap (C \cup D) \eqvdash (z \in A \lor z \in B) \land (z \in C \lor z \in D)}{2}}}
  \proofstepwide*{}{\cap}{z \in (B \cup C)}{}

  % Abschluss
  \proofstepwide{z \in (A \cap B) \cup C}{\leftrightarrow}{z \in (A \cup C) \cap (B \cup C)}%
    {\rChain{1,3}}
\end{tabproofwide}

\FormulaThmAuto{z \in A \cup (B \cap C) \eqvdash z \in (A \cup B) \cap (A \cup C)}
\begin{tabproofwide}
  % Zeile 1 -> zwei Zeilen
  \proofstepwide{z \in A \cup (B \cap C)}{\leftrightarrow}{z \in A}%
    {\multirow{2}{*}{\FormulaRefAuto{z \in A \cup (B \cap C) \eqvdash z \in A \lor (z \in B \land z \in C)}}}
  \proofstepwide*{}{\lor}{(z \in B \land z \in C)}{}

  % Zeile 2 -> zwei Zeilen
  \proofstepwide{}{\leftrightarrow}{(z \in A \lor z \in B)}%
    {\multirow{2}{*}{\FormulaRefAuto{P \lor (Q \land R) \dashv \vdash (P \lor Q) \land (P \lor R)}{1}}}
  \proofstepwide*{}{\land}{(z \in A \lor z \in C)}{}

  % Zeile 3 -> zwei Zeilen
  \proofstepwide{}{\leftrightarrow}{z \in (A \cup B)}%
    {\multirow{2}{*}{\FormulaRefAuto{z \in (A \cup B) \cap (C \cup D) \eqvdash (z \in A \lor z \in B) \land (z \in C \lor z \in D)}{2}}}
  \proofstepwide*{}{\cap}{z \in (A \cup C)}{}

  % Abschluss
  \proofstepwide{z \in A \cup (B \cap C)}{\leftrightarrow}{z \in (A \cup B) \cap (A \cup C)}%
    {\rChain{1,3}}
\end{tabproofwide}

\FormulaThmAuto{A \cup (B \cap C) = (A \cup B) \cap (A \cup C)}
\begin{tabproofwide}
  \proofstepwide{z \in A \cup (B \cap C)}{\leftrightarrow}{z \in (A \cup B) \cap (A \cup C)}%
    {\FormulaRefAuto{z \in A \cup (B \cap C) \eqvdash z \in (A \cup B) \cap (A \cup C)}}
  \proofstepwide{A \cup (B \cap C)}{=}{(A \cup B) \cap (A \cup C)}%
    {\FormulaRefAuto{\forall x\, (x \in A \leftrightarrow x \in B) \eqvdash A = B}{\rUI{1}}}
\end{tabproofwide}

\FormulaThmAuto{(A \cap B) \cup C = (A \cup C) \cap (B \cup C)}
\begin{tabproofwide}
  \proofstepwide{z \in (A \cap B) \cup C}{\leftrightarrow}{z \in (A \cup C) \cap (B \cup C)}%
    {\FormulaRefAuto{z \in (A \cap B) \cup C \eqvdash z \in (A \cup C) \cap (B \cup C)}}
  \proofstepwide{(A \cap B) \cup C}{=}{(A \cup C) \cap (B \cup C)}%
    {\FormulaRefAuto{\forall x\, (x \in A \leftrightarrow x \in B) \eqvdash A = B}{\rUI{1}}}
\end{tabproofwide}

\FormulaThmAuto{A \cap (B \cup C) = (A \cap B) \cup (A \cap C)}
\begin{tabproofwide}
  \proofstepwide{z \in A \cap (B \cup C)}{\leftrightarrow}{z \in (A \cap B) \cup (A \cap C)}%
    {\FormulaRefAuto{z \in A \cap (B \cup C) \eqvdash z \in (A \cap B) \cup (A \cap C)}}
  \proofstepwide{A \cap (B \cup C)}{=}{(A \cap B) \cup (A \cap C)}%
    {\FormulaRefAuto{\forall x\, (x \in A \leftrightarrow x \in B) \eqvdash A = B}{\rUI{1}}}
\end{tabproofwide}

\FormulaThmAuto{(A \cup B) \cap C = (A \cap C) \cup (B \cap C)}
\begin{tabproofwide}
  \proofstepwide{z \in (A \cup B) \cap C}{\leftrightarrow}{z \in (A \cap C) \cup (B \cap C)}%
    {\FormulaRefAuto{z \in (A \cup B) \cap C \eqvdash z \in (A \cap C) \cup (B \cap C)}}
  \proofstepwide{(A \cup B) \cap C}{=}{(A \cap C) \cup (B \cap C)}%
    {\FormulaRefAuto{\forall x\, (x \in A \leftrightarrow x \in B) \eqvdash A = B}{\rUI{1}}}
\end{tabproofwide}

\section[Eigenschaften von Teilmengen]{Eigenschaften in Bezug auf Teilmengen}


\FormulaThmAuto{A \subseteq A \cup B}
\begin{tabproofwide}
  \proofstepwide{x \in A}{\rightarrow}{x \in A \lor x \in B}%
    {\FormulaRefAuto{P \rightarrow P \lor Q}}
  \proofstepwide{}{ \rightarrow}{x \in A \cup B}%
    {\FormulaRefAuto{z \in A \cup B \eqvdash z \in A \lor z \in B}{1}}
  \proofstepwide{x \in A}{\rightarrow}{x \in A \cup B}%
    {\rChain{1,2}}
  \proofstepwidestar{A \subseteq A \cup B}%
    {\FormulaRefAuto{ A \subseteq B := \forall x\,(x\in A \rightarrow x\in B) }{\rUI{3}}}
\end{tabproofwide}

\FormulaThmAuto{A \subseteq B \cup A}
\begin{tabproofwide}
  \proofstepwide{x \in A}{\rightarrow}{x \in B \lor x \in A}%
    {\FormulaRefAuto{P \rightarrow Q \lor P}}
  \proofstepwide{}{\rightarrow}{x \in B \cup A}%
    {\FormulaRefAuto{z \in A \cup B \eqvdash z \in A \lor z \in B}{1}}
  \proofstepwide{x \in A}{\rightarrow}{x \in B \cup A}%
    {\rChain{1,2}}
  \proofstepwidestar{A \subseteq B \cup A}%
    {\FormulaRefAuto{A \subseteq B := \forall x\,(x\in A \rightarrow x\in B)}{\rUI{3}}}
\end{tabproofwide}

\FormulaThmAuto{A \subseteq C,\, B \subseteq C \vdash A \cup B \subseteq C}
\begin{tabproofwide}
  \proofstepwidestar[1]{A \subseteq C}{\rA}
  \proofstepwidestar[2]{B \subseteq C}{\rA}
  \proofstepwide{z \in A \cup B}{\rightarrow}{z \in A \lor z \in B}%
    {\FormulaRefAuto{z \in A \cup B \eqvdash z \in A \lor z \in B}}
  \proofstepwide[1,2]{}{ \rightarrow}{z \in C}%
    {\FormulaRefAuto{A\subseteq C,\, B\subseteq C,\, z\in A\lor z\in B \vdash z\in C}{1,2,3}}
  \proofstepwide[1,2]{z \in A \cup B}{\rightarrow}{z \in C}%
    {\rChain{3,4}}
  \proofstepwidestar[1,2]{A \cup B \subseteq C}%
    {\FormulaRefAuto{A \subseteq B := \forall x\,(x\in A \rightarrow x\in B)}{\rUI{5}}}
\end{tabproofwide}


\FormulaThmAuto{A \subseteq B \vdash A \cup C \subseteq B \cup C}
\begin{tabproofwide}
  \proofstepwidestar[1]{A \subseteq B}{\rA}
  \proofstepwide{z \in A \cup C}{\rightarrow}{z \in A \lor z \in C}%
    {\FormulaRefAuto{z \in A \cup B \eqvdash z \in A \lor z \in B}}
  \proofstepwide[1]{}{ \rightarrow}{z \in B \lor z \in C}%
    {\FormulaRefAuto{P \rightarrow Q,\, P \lor R \vdash Q \lor R}%
      {\rUE{\FormulaRefAuto{A \subseteq B := \forall x\,(x\in A \rightarrow x\in B)}{1}},2}}
  \proofstepwide[1]{}{ \rightarrow}{z \in B \cup C}%
    {\FormulaRefAuto{z \in A \cup B \eqvdash z \in A \lor z \in B}{3}}
  \proofstepwide[1]{z \in A \cup C}{\rightarrow}{z \in B \cup C}%
    {\rChain{2,4}}
  \proofstepwidestar[1]{A \cup C \subseteq B \cup C}%
    {\FormulaRefAuto{A \subseteq B := \forall x\,(x\in A \rightarrow x\in B)}{\rUI{5}}}
\end{tabproofwide}

\FormulaThmAuto{A \subseteq B \vdash C \cup A \subseteq C \cup B}
\begin{tabproofwide}
  \proofstepwidestar[1]{A \subseteq B}{\rA}
  \proofstepwide{C \cup A}{=}{A \cup C}%
    {\FormulaRefAuto{A \cup B = B \cup A}}
  \proofstepwide[1]{}{ \subseteq}{B \cup C}%
    {\FormulaRefAuto{A \subseteq B \vdash A \cup C \subseteq B \cup C}{1}}
  \proofstepwide[1]{}{=}{C \cup B}%
    {\FormulaRefAuto{A \cup B = B \cup A}{3}}
  \proofstepwide[1]{C \cup A}{\subseteq}{C \cup B}%
    {\rChain{2,4}}
\end{tabproofwide}

\FormulaThmAuto{A \subseteq B,\, C \subseteq D \vdash A \cup C \subseteq B \cup D}
\begin{tabproofwide}
  \proofstepwidestar[1]{A \subseteq B}{\rA}
  \proofstepwidestar[2]{C \subseteq D}{\rA}
  \proofstepwide[1]{A \cup C}{\subseteq}{B \cup C}%
    {\FormulaRefAuto{A \subseteq B \vdash A \cup C \subseteq B \cup C}{1}}
  \proofstepwide[2]{}{ \subseteq}{B \cup D}%
    {\FormulaRefAuto{A \subseteq B \vdash C \cup A \subseteq C \cup B}{2}}
  \proofstepwide[1,2]{A \cup C}{\subseteq}{B \cup D}%
    {\rChain{3,4}}
\end{tabproofwide}

\FormulaThmAuto{A \subseteq B,\, C \subseteq D \vdash A \cap C \subseteq B \cap D}
\begin{tabproofwide}
  \proofstepwidestar[1]{A \subseteq B}{\rA}
  \proofstepwidestar[2]{C \subseteq D}{\rA}
  \proofstepwide{x \in A \cap C}{\rightarrow}{x \in A \land x \in C}%
    {\FormulaRefAuto{x \in A \cap B \eqvdash x \in A \land x \in B}}
  \proofstepwide[1]{}{ \rightarrow}{x \in B \land x \in C}%
    {\FormulaRefAuto{P \rightarrow Q,\, P \land R \vdash Q \land R}{\rUE{\FormulaRefAuto{A \subseteq B := \forall x\,(x \in A \rightarrow x \in B)}{1}},3}}
  \proofstepwide[1,2]{}{ \rightarrow}{x \in B \land x \in D}%
    {\FormulaRefAuto{P \rightarrow Q,\, R \land P \vdash R \land Q}{\rUE{\FormulaRefAuto{A \subseteq B := \forall x\,(x \in A \rightarrow x \in B)}{2}},4}}
  \proofstepwide[1,2]{}{ \rightarrow}{x \in B \cap D}%
    {\FormulaRefAuto{x \in A \cap B \eqvdash x \in A \land x \in B}{5}}
  \proofstepwide[1,2]{x \in A \cap C}{\rightarrow}{x \in B \cap D}%
    {\rChain{3,6}}
  \proofstepwidestar[1,2]{A \cap C \subseteq B \cap D}%
    {\FormulaRefAuto{A \subseteq B := \forall x\,(x \in A \rightarrow x \in B)}{\rUI{7}}}
\end{tabproofwide}

\FormulaThmAuto{a \in A,\, b \in B \vdash \{a,b\} \subseteq A \cup B}
\begin{tabproofwide}
  \proofstepwidestar[1]{a \in A}{\rA}
  \proofstepwidestar[2]{b \in B}{\rA}
  \proofstepwide[1]{\{a\}}{\subseteq}{A}%
    {\FormulaRefAuto{a \in A \vdash \{a\} \subseteq A}{1}}
  \proofstepwide[2]{\{b\}}{\subseteq}{B}%
    {\FormulaRefAuto{a \in A \vdash \{a\} \subseteq A}{2}}
  \proofstepwide[1,2]{\{a\} \cup \{b\}}{\subseteq}{A \cup B}%
    {\FormulaRefAuto{A \subseteq B,\, C \subseteq D \vdash A \cup C \subseteq B \cup D}{3,4}}
  \proofstepwide{\{a,b\}}{=}{\{a\} \cup \{b\}}%
    {\FormulaRefAuto{\{a,b\} = \{a\} \cup \{b\}}}
  \proofstepwidestar{\{a,b\} \subseteq A \cup B}%
    {\rIE{6,5}}
\end{tabproofwide}

\FormulaThmAuto{a \in A,\, b \in A \vdash \{a,b\} \subseteq A}
\begin{tabproofwide}
  \proofstepwidestar[1]{a \in A}{\rA}
  \proofstepwidestar[2]{b \in A}{\rA}
  \proofstepwide[1,2]{\{a,b\}}{\subseteq}{A \cup A}%
    {\FormulaRefAuto{a \in A,\, b \in B \vdash \{a,b\} \subseteq A \cup B}{1,2}}
  \proofstepwide[1,2]{}{=}{A}%
    {\FormulaRefAuto{A = A \cup A}{3}}
  \proofstepwidestar[1,2]{\{a,b\} \subseteq A}%
    {\rChain{3,4}}
\end{tabproofwide}

\chapter{Regularität}

\FormulaAxiomDelta[Regularität (Fundamentalsatz)]{ A \neq \emptyset \vdash \exists x \in A \,(x \cap A = \emptyset) }{
\DeltaRow{Mengen}{A}%
}
\begin{remark}
Dieses Axiom verhindert zyklische Mitgliedschaften, indem jede nicht-leere Menge ein 
\textbf{Minimalelement} enthält.    
\end{remark}

\section{Ausschluss gegenseitiger Mitgliedschaft}

\FormulaThmAuto{a\in b\vdash b\not\in a}
\begin{tabproof}
  \proofstep{1}{a \in b}{\rA}
  \proofstep{2}{b \in a}{\rA}
  \proofstep{ }{\{a, b\} \neq \emptyset}{\FormulaRefAuto{\{a,b\} \neq \emptyset}}
  \proofstep{ }{\exists x \in \{a, b\}\,(x \cap \{a, b\} = \emptyset)}{\FormulaRefAuto{ A \neq \emptyset \vdash \exists x \in A \,(x \cap A = \emptyset) }{3}}
  \proofstep{ }{a \cap \{a, b\} = \emptyset \;\lor\; b \cap \{a, b\} = \emptyset}{\FormulaRefAuto{\exists x\in \{a,b\} P(x)\vdash P(a)\lor P(b)}{4}}
  \proofstep{6}{a \cap \{a, b\} = \emptyset}{\rA}
  \proofstep{2}{a \cap \{a, b\} \neq \emptyset}{\FormulaRefAuto{a\in A\vdash A\cap \{A,a\}\neq\emptyset}{2}}
  \proofstep{2,6}{\bot}{\rBI{6,7}}
  \proofstep{9}{b \cap \{a, b\} = \emptyset}{\rA}
  \proofstep{1}{b \cap \{a, b\} \neq \emptyset}{\FormulaRefAuto{a\in A\vdash A\cap \{A,a\}\neq\emptyset}{1}}
  \proofstep{1,9}{\bot}{\rBI{9,10}}
  \proofstep{1,2}{\bot}{\rOE{5,6,8,9,11}}
  \proofstep{1}{b\notin a}{\rCI{2,12}}
\end{tabproof}

\FormulaThmAuto{a\not\in a}
\begin{tabproof}
  \proofstep{1}{a \in a}{\rA}
  \proofstep{1}{a \notin a}{\FormulaRefAuto{a\in b\vdash b\not\in a}}
  \proofstep{1}{\bot}{\rBI{1,2}}
  \proofstep{}{a\notin a}{\rCE{1,3}}
\end{tabproof}

\FormulaThmAuto{a\cup\{a\}=b\cup\{b\}\eqvdash a=b}
\begin{tabproofsplit}
\proofpart{\(\vdash\)}
  \proofstep{1}{a \cup \{a\} = b \cup \{b\}}{\rA}

  % aus der Gleichheit folgen die beiden Disjunktionen
  \proofstep{}{a \in a \cup \{a\}}{\FormulaRefAuto{a \in A \cup \{a\}}}
  \proofstep{1}{a \in b \cup \{b\}}{\rIE{1,2}}
  \proofstep{1}{a \in b \lor a \in \{b\}}{\FormulaRefAuto{z \in A \cup B \eqvdash z \in A \lor z \in B}{3}}
  \proofstep{1}{a \in b \lor a = b}{\rLRS{\FormulaRefAuto{x \in \{a\} \eqvdash x = a}{},4}}

  \proofstep{}{b \in b \cup \{b\}}{\FormulaRefAuto{a \in A \cup \{a\}}}
  \proofstep{1}{b \in a \cup \{a\}}{\rIE{1,6}}
  \proofstep{1}{b \in a \lor b \in \{a\}}{\FormulaRefAuto{z \in A \cup B \eqvdash z \in A \lor z \in B}{7}}
  \proofstep{1}{b \in a \lor b = a}{\rLRS{\FormulaRefAuto{x \in \{a\} \eqvdash x = a}{},8}}

  % indirekt: a \neq b  ⇒  a \in b und b \in a; Widerspruch mit a∈b ⟹ b∉a
  \proofstep{10}{a \neq b}{\rA}

  % aus (a∈b ∨ a=b) und a≠b folgt a∈b
  \proofstep{1,10}{a \in b}{\FormulaRefAuto{P\lor Q, \neg Q\vdash P}{5,10}}

  % aus (b∈a ∨ b=a) und b≠a (Symmetrie der Ungleichheit) folgt b∈a
  \proofstep{1,10}{b \in a}{\FormulaRefAuto{P\lor Q, \neg Q\vdash P}{9,10}}

  % Widerspruch mit dem Lemma a∈b ⟹ b∉a
  \proofstep{1,10}{\bot}{\rBI{11,12}}
  \proofstep{1}{a = b}{\rCE{10,13}}
\closeproofpart

\proofpart{\(\dashv\)}
  \proofstep{1}{a = b}{\rA}
  \proofstep{1}{a \cup \{a\} = b \cup \{b\}}{\FormulaRefAuto{A=B\vdash A\cup C = B\cup C}{1}}
\closeproofpart
\end{tabproofsplit}

\chapter{Potenzmenge}

\section{Axiom der Potenzmenge}

\FormulaAxiomDelta[Potenzmenge]{\exists B\forall x\bigl(x \in B \leftrightarrow x \subseteq A\bigr)}%
{%
\DeltaRow{Mengen}{A}%
}

\section{Definition der Potenzmenge}
%%begin novalidate
\FormulaDefDelta[Potenzmenge]{\mathcal{P}(A) := \iota B\Bigl(\forall x\;\bigl(x \in B \leftrightarrow x \subseteq A\bigr)\Bigr)}%
{%
\DeltaRow{Mengen}{A}%
}
%%end novalidate

\FormulaThmDelta[Potenzmenge]{x \in \mathcal{P}(A)\;\eqvdash\; x \subseteq A}%
{%
\DeltaRow{Mengen}{x\dsep A}%
}
\begin{tabproof}
  \proofstep{}{ \forall x\;\bigl(x \in B \leftrightarrow x \subseteq A\bigr) }{\FormulaRefAuto{\mathcal{P}(A) := \iota B\Bigl(\forall x\;\bigl(x \in B \leftrightarrow x \subseteq A\bigr)\Bigr)}}
  \proofstep{}{ x \in \mathcal{P}(A)\;\leftrightarrow\; x \subseteq A }{\rUE{1}}
\end{tabproof}

\section{Grundlegende Eigenschaften}

\FormulaThmAuto{A \subseteq B \vdash \mathcal{P}(A) \subseteq \mathcal{P}(B)}
\begin{tabproofwide}
  \proofstepwidestar[1]{A \subseteq B}{\rA}
  \proofstepwide{x \in \mathcal{P}(A)}{\rightarrow}{x \subseteq A}%
    {\FormulaRefAuto{\mathcal{P}(A) := \iota B\Bigl(\forall x\;\bigl(x \in B \leftrightarrow x \subseteq A\bigr)\Bigr)}}
  \proofstepwide[1]{}{ \rightarrow}{x \subseteq B}%
    {\FormulaRefAuto{A \subseteq B,\, B \subseteq C \vdash A \subseteq C}{2,1}}
  \proofstepwide[1]{}{ \rightarrow}{x \in \mathcal{P}(B)}%
    {\FormulaRefAuto{\mathcal{P}(A) := \iota B\Bigl(\forall x\;\bigl(x \in B \leftrightarrow x \subseteq A\bigr)\Bigr)}{3}}
  \proofstepwide[1]{x \in \mathcal{P}(A)}{\rightarrow}{x \in \mathcal{P}(B)}%
    {\rChain{2,4}}
  \proofstepwidestar[1]{\mathcal{P}(A) \subseteq \mathcal{P}(B)}%
    {\FormulaRefAuto{A \subseteq B := \forall x\,(x\in A \rightarrow x\in B)}{\rUI{5}}}
\end{tabproofwide}

\FormulaThmAuto{a \in A \vdash \{a\} \in \mathcal{P}(A)}
\begin{tabproof}
  \proofstep{1}{a \in A}{\rA}
  \proofstep{1}{\{a\} \subseteq A}{\FormulaRefAuto{a \in A \vdash \{a\} \subseteq A}{1}}
  \proofstep{1}{\{a\} \in \mathcal{P}(A)}{\FormulaRefAuto{\mathcal{P}(A) := \iota B\Bigl(\forall x\;\bigl(x \in B \leftrightarrow x \subseteq A\bigr)\Bigr)}{2}}
\end{tabproof}

\FormulaThmAuto{A \subseteq B,\, a \in \mathcal{P}(A) \vdash a \in \mathcal{P}(B)}
\begin{tabproof}
  \proofstep{1}{A \subseteq B}{\rA}
  \proofstep{2}{a \in \mathcal{P}(A)}{\rA}
  \proofstep{1}{\mathcal{P}(A) \subseteq \mathcal{P}(B)}{\FormulaRefAuto{A \subseteq B \vdash \mathcal{P}(A) \subseteq \mathcal{P}(B)}{1}}
  \proofstep{1,2}{a \in \mathcal{P}(B)}{\rRE{\rUE{\FormulaRefAuto{A \subseteq B := \forall x\,(x \in A \rightarrow x \in B)}{3}},2}}
\end{tabproof}


\FormulaThmAuto{a \in \mathcal{P}(A) \vdash \forall B\,(a \in \mathcal{P}(A \cup B))}
\begin{tabproof}
  \proofstep{1}{a \in \mathcal{P}(A)}{\rA}
  \proofstep{}{A \subseteq A \cup B}{\FormulaRefAuto{A \subseteq A \cup B}{}}
  \proofstep{1}{a \in \mathcal{P}(A \cup B)}%
    {\FormulaRefAuto{A \subseteq B,\, a \in \mathcal{P}(A) \vdash a \in \mathcal{P}(B)}{2,1}}
  \proofstep{1}{\forall B\,(a \in \mathcal{P}(A \cup B))}{\rUI{3}}
\end{tabproof}

\section{Das kartesische Produkt}

\subsection{Existenz des karthesischen Produktes}

\FormulaThmAuto{a \in A,\, b \in B \vdash (a,b) \in \mathcal{P}(\mathcal{P}(A \cup B))}
\begin{tabproof}
  \proofstep{1}{a \in A}{\rA}
  \proofstep{2}{b \in B}{\rA}
  \proofstep{1,2}{\{a,b\} \subseteq A \cup B}{\FormulaRefAuto{a \in A,\, b \in B \vdash \{a,b\} \subseteq A \cup B}{1,2}}
  \proofstep{1,2}{\{a,b\} \in \mathcal{P}(A \cup B)}{\FormulaRefAuto{\mathcal{P}(A) := \iota B\bigl(\forall x\,(x \in B \leftrightarrow x \subseteq A)\bigr)}{3}}
  \proofstep{1}{a \in A \cup B}{\FormulaRefAuto{z \in A \vdash z \in A \cup B}{1}}
  \proofstep{1}{\{a\} \in \mathcal{P}(A \cup B)}{\FormulaRefAuto{a \in A \vdash \{a\} \in \mathcal{P}(A)}{5}}
  \proofstep{1,2}{\{\{a\},\{a,b\}\} \subseteq \mathcal{P}(A \cup B)}{\FormulaRefAuto{a \in A,\, b \in B \vdash \{a,b\} \subseteq A \cup B}{6,4}}
  \proofstep{1,2}{\{\{a\},\{a,b\}\} \in \mathcal{P}(\mathcal{P}(A \cup B))}{\FormulaRefAuto{\mathcal{P}(A) := \iota B\bigl(\forall x\,(x \in B \leftrightarrow x \subseteq A)\bigr)}{7}}
  \proofstep{1,2}{(a,b) \in \mathcal{P}(\mathcal{P}(A \cup B))}{\rIE{\FormulaRefAuto{(a, b) := \{ \{ a \}, \{ a, b \} \}},8}}
\end{tabproof}

\FormulaThmAuto{\exists a \in A\, \exists b \in B \big(x=(a,b)\big)\vdash x \in \mathcal{P}(\mathcal{P}(A \cup B))}
\begin{tabproof}
  \proofstep{1}{\exists a \in A\, \exists b \in B \big(x=(a,b)\big)}{\rA}
  \proofstep{2}{(a \in A\land b \in B)\land x=(a,b)}{\rA}
  \proofstep{2}{a \in A}{\FormulaRefAuto{(P\land Q)\land R\vdash P}{2}}
  \proofstep{2}{b \in B}{\FormulaRefAuto{(P\land Q)\land R\vdash Q}{2}}
  \proofstep{2}{x=(a,b)}{\rAEb{2}}
  \proofstep{2}{(a,b) \in \mathcal{P}(\mathcal{P}(A \cup B))}{\FormulaRefAuto{a \in A,\, b \in B \vdash (a,b) \in \mathcal{P}(\mathcal{P}(A \cup B))}{3,4}}
  \proofstep{2}{x \in \mathcal{P}(\mathcal{P}(A \cup B))}{\rIE{5,6}}  
  \proofstep{1}{x \in \mathcal{P}(\mathcal{P}(A \cup B))}{\rEE{1,2,7}}  
\end{tabproof}

\FormulaThmAuto[Eindeutige Existenz des kartesischen Produkts]{\exists! C\, \forall (a,b) \bigl(x \in C \leftrightarrow \exists a \in A\, \exists b \in B \big(x=(a,b)\big)\bigr)}
\begin{tabproof}
  \proofstep{}{ \exists! C\, \forall (a,b) \bigl((a,b) \in C \leftrightarrow \exists a \in A\, \exists b \in B}{\multirow{2}{*}{\rEI{\FormulaRefAuto{\forall x(P(x)\rightarrow x\in A)\vdash \exists! B(\forall x(x\in B\leftrightarrow P(x)))}{\rUI{\FormulaRefAuto{\exists a \in A,\, \exists b \in B \big(x=(a,b)\big)\vdash x \in \mathcal{P}(\mathcal{P}(A \cup B))}}}}}}
  \proofstepstar{}{ \big(x=(a,b)\big)\bigr)}{}
\end{tabproof}
%%begin novalidate
\FormulaDefAuto[Kartesisches Produkt ($A \times B$)]{A \times B := \iota C\, \forall x\, \bigl(x \in C \leftrightarrow \exists a \in A\, \exists b \in B \big(x=(a,b)\big)\bigr)}
%%end novalidate
\begin{remark}
    Hieraus ergibt sich folgendes:
\[
    x \in A \times B \eqvdash \exists a \in A\, \exists b \in B \big(x=(a,b)\big)
\]
\end{remark}


\subsection{Grundlegende Eigenschaften}

\FormulaThmAuto[Kartesisches Produkt ($A \times B$)]{(a,b)\in A\times B\eqvdash a\in A\land b\in B}
\begin{tabproofsplit}
\proofpart{\(\vdash\)}
  \proofstep{1}{(a,b)\in A\times B}{\rA}
  \proofstep{1}{\exists c\in A\exists d\in B((a,b)=(c,d))}{\FormulaRefAuto{A \times B := \iota C\, \forall x\, \bigl(x \in C \leftrightarrow \exists a \in A\, \exists b \in B \big(x=(a,b)\big)\bigr)}{1}}
  \proofstep{3}{c\in A\land d\in B\land (c,d)=(a,b)}{\rA}
  \proofstep{3}{c\in A}{\rAEn{3}}
  \proofstep{3}{d\in B}{\rAEn{3}}
  \proofstep{3}{(c,d)=(a,b)}{\rAEn{3}}
  \proofstep{3}{c=a\land d=b}{\FormulaRefAuto{(a,b)=(c,d)\eqvdash a=c\land b=d}{6}}
  \proofstep{3}{c=a}{\rAEa{7}}
  \proofstep{3}{d=b}{\rAEb{7}}
  \proofstep{3}{a\in A}{\rIE{8,4}}
  \proofstep{3}{b\in B}{\rIE{9,5}}
  \proofstep{3}{a\in A\land b\in B}{\rAI{10,11}}
  \proofstep{1}{a\in A\land b\in B}{\rEE{2,3,12}}
\closeproofpart
\proofpart{\(\dashv\)}
  \proofstep{1}{a\in A\land b\in B}{\rA}
  \proofstep{}{(a,b)=(a,b)}{\rII}
  \proofstep{1}{a\in A\land b\in B\land (a,b)=(a,b)}{\rAI{1,2}}
  \proofstep{1}{\exists a\in A\exists b\in B((a,b)=(a,b))}{\rEI{3}}
  \proofstep{1}{(a,b)\in A\times B}{\FormulaRefAuto{A \times B := \iota C\, \forall x\, \bigl(x \in C \leftrightarrow \exists a \in A\, \exists b \in B \big(x=(a,b)\big)\bigr)}{4}}
\closeproofpart
\end{tabproofsplit}

\FormulaThmAuto{(a,b)\in A\times B\eqvdash (b,a)\in B\times A}
\begin{tabproofwide}
  \proofstepwide{(a,b)\in A\times B}{\leftrightarrow}{a\in A\land b\in B}%
    {\FormulaRefAuto{(a,b)\in A\times B\eqvdash a\in A\land b\in B}}
  \proofstepwide{}{\leftrightarrow}{b\in B\land a\in A}%
    {\FormulaRefAuto{P\land Q\eqvdash Q\land P}{1}}
  \proofstepwide{}{\leftrightarrow}{(b,a)\in B\times A}%
    {\FormulaRefAuto{(a,b)\in A\times B\eqvdash a\in A\land b\in B}{2}}
  \proofstepwide{(a,b)\in A\times B}{\leftrightarrow}{(b,a)\in B\times A}%
    {\rChain{1,3}}
\end{tabproofwide}


\FormulaThmAuto{(a,b)\in A\times B\vdash a\in A}
\begin{tabproof}
      \proofstep{1}{(a,b)\in A\times B}{\rA}
      \proofstep{1}{a\in A\land b\in B}{\FormulaRefAuto{(a,b)\in A\times B\eqvdash a\in A\land b\in B}{1}}
      \proofstep{1}{a\in A}{\rAEa{2}}
\end{tabproof}

\FormulaThmAuto{(a,b)\in A\times B\vdash b\in B}
\begin{tabproof}
      \proofstep{1}{(a,b)\in A\times B}{\rA}
      \proofstep{1}{a\in A\land b\in B}{\FormulaRefAuto{(a,b)\in A\times B\eqvdash a\in A\land b\in B}{1}}
      \proofstep{1}{a\in B}{\rAEb{2}}
\end{tabproof}


\FormulaThmAuto{a\in A\dsep b\in B\vdash (a,b)\in A\times B}
\begin{tabproof}
      \proofstep{1}{a\in A}{\rA}
      \proofstep{2}{b\in B}{\rA}
      \proofstep{1,2}{a\in A\land b\in B}{\rAI{1,2}}
      \proofstep{1,2}{(a,b)\in A\times B}{\FormulaRefAuto{(a,b)\in A\times B\eqvdash a\in A\land b\in B}{3}}
\end{tabproof}

% ============================================================
%  Dreifache Produkte als verschachtelte Paare
% ============================================================

% ((a,b),c) ∈ (A×B)×C  ⇒  a∈A ∧ b∈B ∧ c∈C
\FormulaThmAuto{((a,b),c)\in (A\times B)\times C\ \eqvdash\ a\in A\land b\in B\land c\in C}
\begin{tabproofsplit}
\proofpart{\(\vdash\)}
  \proofstep{1}{((a,b),c)\in (A\times B)\times C}{\rA}
  % Erst Projektion aus (A×B)×C
  \proofstep{1}{(a,b)\in A\times B\ \land\ c\in C}{\FormulaRefAuto{(a,b)\in A\times B\eqvdash a\in A\land b\in B}{1}}
  \proofstep{1}{(a,b)\in A\times B}{\rAEa{2}}
  \proofstep{1}{c\in C}{\rAEb{2}}
  % Dann Projektion aus A×B
  \proofstep{1}{a\in A\land b\in B}{\FormulaRefAuto{(a,b)\in A\times B\eqvdash a\in A\land b\in B}{3}}
  \proofstep{1}{a\in A}{\rAEa{5}}
  \proofstep{1}{b\in B}{\rAEb{5}}
  \proofstep{1}{a\in A\land b\in B\land c\in C}{\rAI{\rAI{6,7},4}}
\closeproofpart
\proofpart{\(\dashv\)}
  \proofstep{1}{a\in A\land b\in B\land c\in C}{\rA}
  \proofstep{1}{a\in A\land b\in B}{\rAEa{1}}
  \proofstep{1}{c\in C}{\rAEb{1}}
  \proofstep{1}{(a,b)\in A\times B}{\FormulaRefAuto{a\in A\dsep b\in B\vdash (a,b)\in A\times B}{2}}
  \proofstep{1}{((a,b),c)\in (A\times B)\times C}{\FormulaRefAuto{a\in A, b\in B\vdash (a,b)\in A\times B}{4,3}}
\closeproofpart
\end{tabproofsplit}

% (a,(b,c)) ∈ A×(B×C)  ⇒  a∈A ∧ b∈B ∧ c∈C
\FormulaThmAuto{(a,(b,c))\in A\times (B\times C)\ \eqvdash\ a\in A\land b\in B\land c\in C}
\begin{tabproofsplit}
\proofpart{\(\vdash\)}
  \proofstep{1}{(a,(b,c))\in A\times (B\times C)}{\rA}
  % Erst Projektion aus A×(B×C)
  \proofstep{1}{a\in A\ \land\ (b,c)\in B\times C}{\FormulaRefAuto{(a,b)\in A\times B\eqvdash a\in A\land b\in B}{1}}
  \proofstep{1}{a\in A}{\rAEa{2}}
  \proofstep{1}{(b,c)\in B\times C}{\rAEb{2}}
  % Dann Projektion aus B×C
  \proofstep{1}{b\in B\land c\in C}{\FormulaRefAuto{(a,b)\in A\times B\eqvdash a\in A\land b\in B}{4}}
  \proofstep{1}{b\in B}{\rAEa{5}}
  \proofstep{1}{c\in C}{\rAEb{5}}
  \proofstep{1}{a\in A\land b\in B\land c\in C}{\rAI{3,\rAI{6,7}}}
\closeproofpart
\proofpart{\(\dashv\)}
  \proofstep{1}{a\in A\land b\in B\land c\in C}{\rA}
  \proofstep{1}{a\in A}{\rAEa{1}}
  \proofstep{1}{b\in B\land c\in C}{\rAEb{1}}
  \proofstep{1}{(b,c)\in B\times C}{\FormulaRefAuto{a\in A\dsep b\in B\vdash (a,b)\in A\times B}{3}}
  \proofstep{1}{(a,(b,c))\in A\times (B\times C)}{\FormulaRefAuto{(a,b)\in A\times B\eqvdash a\in A\land b\in B}{2,4}}
\closeproofpart
\end{tabproofsplit}

% Einführungsrichtungen getrennt (falls du sie einzeln haben willst):

% a∈A, b∈B, c∈C  ⊢  (a,(b,c))∈A×(B×C)
\FormulaThmAuto{a\in A,\, b\in B,\, c\in C\ \vdash\ (a,(b,c))\in A\times (B\times C)}
\begin{tabproof}
  \proofstep{1}{a\in A}{\rA}
  \proofstep{2}{b\in B}{\rA}
  \proofstep{3}{c\in C}{\rA}
  \proofstep{2,3}{(b,c)\in B\times C}{\FormulaRefAuto{a\in A\dsep b\in B\vdash (a,b)\in A\times B}{2,3}}
  \proofstep{1,4}{(a,(b,c))\in A\times (B\times C)}{\FormulaRefAuto{(a,b)\in A\times B\eqvdash a\in A\land b\in B}{1,4}}
\end{tabproof}

% a∈A, b∈B, c∈C  ⊢  ((a,b),c)∈(A×B)×C
\FormulaThmAuto{a\in A,\, b\in B,\, c\in C\ \vdash\ ((a,b),c)\in (A\times B)\times C}
\begin{tabproof}
  \proofstep{1}{a\in A}{\rA}
  \proofstep{2}{b\in B}{\rA}
  \proofstep{3}{c\in C}{\rA}
  \proofstep{1,2}{(a,b)\in A\times B}{\FormulaRefAuto{a\in A\dsep b\in B\vdash (a,b)\in A\times B}{1,2}}
  \proofstep{4,3}{((a,b),c)\in (A\times B)\times C}{\FormulaRefAuto{(a,b)\in A\times B\eqvdash a\in A\land b\in B}{4,3}}
\end{tabproof}

\subsection{Relationen}

\FormulaDefDeltaK[Begriff der Relation]{R \subseteq A \times B}{Relation}{
  \DeltaRow{Mengen}{A \dsep B \dsep R}
  \DeltaRow{\textbf{Axiome}}{}
  %
  % — Menge geordneter Paare —
  \DeltaRow{Menge geordneter Paare}
           {R \subseteq A \times B}
           [\FormulaRefAuto{R \subseteq A \times B}]
  %
  \DeltaRow{\textbf{Neue Symbole}}{}
  \DeltaRow{Relationen}{ }
}

\chapter{Funktionen}
Nachdem wir die grundlegenden Begriffe der Mengenlehre eingeführt haben, wenden wir uns nun den 
\textbf{Funktionen} zu. Diese sind zentrale Objekte der Mathematik und können als spezielle Mengen 
von geordneten Paaren aufgefasst werden.

\FormulaDefDeltaK[Begriff der Funktion]{F\colon A\to B}{Funktion}{
  \DeltaRow{Mengen}{A\dsep B\dsep F\dsep x\dsep y\dsep z}
  \DeltaRow{\textbf{Axiome}}{}
  %
  % — Existenz als Menge geordneter Paare —
  \DeltaRow{Menge geordneter Paare}
           {F \subseteq A \times B}
           [\FormulaRefAuto{F \subseteq A \times B}]
  %
  % — Funktionale Eindeutigkeit —
  \DeltaRow{Funktionale Eindeutigkeit}
           {(x,y)\in F\dsep (x,z)\in F \vdash y=z}
           [\FormulaRefAuto{(x,y)\in F\dsep (x,z)\in F \vdash y=z}]
  %
  % — Aussonderung (Schema) —
  \DeltaRow{Totalität}
           {x\in A \vdash \exists y\,(x,y)\in F}
           [\FormulaRefAuto{x\in A \vdash \exists y\,(x,y)\in F}]
%
    \DeltaRow{\textbf{Neue Symbole}}{}
    \DeltaRow{Funktionen}{ }
}

\section{Axiom der Menge geordneter Paare}

\FormulaAxiomDelta[Menge geordneter Paare]{F \subseteq A \times B}{
\DeltaRow{Mengen}{A\dsep B\dsep F}
}

\FormulaThmDelta{(x,y)\in F\vdash (x,y)\in A\times B}{
\DeltaRow{Mengen}{x\dsep y\dsep A\dsep B}
\DeltaRow{Funktionen}{F\colon A\to B}
}
\begin{tabproof}
  \proofstep{1}{(x,y)\in F}{\rA}
  \proofstep{}{F\subseteq A\times B}{\FormulaRefAuto{F \subseteq A \times B}}
  \proofstep{1}{(x,y)\in A\times B}{\FormulaRefAuto{A\subseteq B,\, x\in A \vdash x\in B}{2,1}}
\end{tabproof}

\FormulaThmDelta{(x,y)\in F\vdash x\in A}{
\DeltaRow{Mengen}{x\dsep y\dsep A\dsep B}
\DeltaRow{Funktionen}{F\colon A\to B}
}
\begin{tabproof}
  \proofstep{1}{(x,y)\in F}{\rA}
  \proofstep{1}{(x,y)\in A\times B}{\FormulaRefAuto{(x,y)\in F\vdash (x,y)\in A\times B}{1}}
  \proofstep{1}{x\in A}{\FormulaRefAuto{(a,b)\in A\times B\vdash a\in A}{2}}
\end{tabproof}

\FormulaThmDelta{(x,y)\in F\vdash y\in B}{
\DeltaRow{Mengen}{x\dsep y\dsep A\dsep B}
\DeltaRow{Funktionen}{F\colon A\to B}
}
\begin{tabproof}
  \proofstep{1}{(x,y)\in F}{\rA}
  \proofstep{1}{(x,y)\in A\times B}{\FormulaRefAuto{(x,y)\in F\vdash (x,y)\in A\times B}{1}}
  \proofstep{1}{y\in B}{\FormulaRefAuto{(a,b)\in A\times B\vdash b\in B}{2}}
\end{tabproof}

\section{Axiom der funktionalen Eindeutigkeit}

\FormulaAxiomDelta[Funktionale Eindeutigkeit]
{(x,y)\in F\dsep (x,z)\in F \vdash y=z}
{
\DeltaRow{Mengen}{x\dsep y\dsep z\dsep F}
}

\section{Axiom der Totalität}

\FormulaAxiomDelta[Totalität]
{x\in A \vdash \exists y\,(x,y)\in F}
{
\DeltaRow{Mengen}{x\dsep A\dsep F}
}

\section{Funktionswert}

\FormulaThmDelta[Eindeutigkeit des Funktionswertes]{x\in A\vdash \exists! y\;(x,y)\in F}{
\DeltaRow{Mengen}{x\dsep A}
\DeltaRow{Funktionen}{F\colon A\to B}
}
\begin{tabproof}
  \proofstep{1}{x\in A}{\rA}
  \proofstep{1}{\exists y(x,y)\in F}{\FormulaRefAuto{x\in A \vdash \exists y\,(x,y)\in F}{1}}
  \proofstep{2}{(x,y)\in F}{\rA}
  \proofstep{3}{(x,z)\in F}{\rA}
  \proofstep{2,3}{y=z}{\FormulaRefAuto{(x,y)\in F\dsep (x,z)\in F \vdash y=z}{2,3}}
  \proofstep{1}{\exists! y(x,y)\in F}{\UEI{1,2,3,4}}
\end{tabproof}
%%begin novalidate
\FormulaDefDelta[Funktionswert]{x\in A\vdash (F(x) \coloneqq \iota y\,\bigl((x,y)\in F\bigr)}{
\DeltaRow{Mengen}{x\dsep A}
\DeltaRow{Funktionen}{F\colon A\to B}
}
%%end novalidate

\FormulaThmDelta{x\in A\vdash (x,F(x))\in F}{\DeltaRow{Mengen}{x\dsep A}
\DeltaRow{Funktionen}{F\colon A\to B}
}
\begin{tabproof}
    \proofstep{1}{x\in A}{\rA}
    \proofstep{1}{(x,F(x))\in F}{\FormulaRefAuto{x\in A\vdash (F(x) \coloneqq \iota y\,\bigl((x,y)\in F\bigr)}{1}}
\end{tabproof}

\subsection{Eigenschaften des Funktionswertes}
\FormulaThmDelta{x\in A\dsep y\in A\dsep x=y\vdash F(x)=F(y)}
{
\DeltaRow{Mengen}{x\dsep y\dsep A}
\DeltaRow{Funktionen}{F\colon A\to B}
}
\begin{tabproof}
    \proofstep{1}{x\in A}{\rA}
    \proofstep{2}{y\in A}{\rA}
    \proofstep{3}{x=y}{\rA}
    \proofstep{1}{(x,F(x))\in F}{\FormulaRefAuto{x\in A\vdash (F(x) \coloneqq \iota y\,\bigl((x,y)\in F\bigr)}{1}}
    \proofstep{2}{(y,F(y))\in F}{\FormulaRefAuto{x\in A\vdash (F(x) \coloneqq \iota y\,\bigl((x,y)\in F\bigr)}{2}}
    \proofstep{1,3}{(y,F(x))\in F}{\rIE{3,4}}
    \proofstep{1,2,3}{F(x)=F(y)}{\FormulaRefAuto{(x,y)\in F\dsep (x,z)\in F \vdash y=z}{6,5}}
\end{tabproof}

\FormulaThmDelta{x\in A\dsep F(x)=y\vdash (x,y)\in F}
{
\DeltaRow{Mengen}{x\dsep y\dsep A}
\DeltaRow{Funktionen}{F\colon A\to B}
}
\begin{tabproof}
  \proofstep{1}{x\in A}{\rA}
  \proofstep{2}{R(x)=y}{\rA}
  \proofstep{1}{(x,F(x))\in F}{\FormulaRefAuto{x\in A\vdash (F(x) \coloneqq \iota y\,\bigl((x,y)\in F\bigr)}{1}}
  \proofstep{1,2}{(x,y)\in F}{\rIE{2,3}}
\end{tabproof}

\FormulaThmDelta{x\in A\dsep y=F(x)\vdash (x,y)\in F}
{
\DeltaRow{Mengen}{x\dsep y\dsep A}
\DeltaRow{Funktionen}{F\colon A\to B}
}
\begin{tabproof}
  \proofstep{1}{x\in A}{\rA}
  \proofstep{2}{y=F(x)}{\rA}
  \proofstep{2}{F(x)=y}{\FormulaRefAuto{a=b\vdash b=a}{2}}
  \proofstep{1,2}{(x,y)\in F}{\FormulaRefAuto{x\in A\dsep F(x)=y\vdash (x,y)\in F}{1,3}}
\end{tabproof}

\FormulaThmDelta{x\in A\dsep (x,y)\in F\vdash y=F(x)}{
\DeltaRow{Mengen}{x\dsep y\dsep A}
\DeltaRow{Funktionen}{F\colon A\to B}
}
\begin{tabproof}
  \proofstep{1}{(x,y)\in F}{\rA}
  \proofstep{2}{x\in A}{\rA}
  \proofstep{2}{(x,F(x))\in F}{\FormulaRefAuto{x\in A\vdash (F(x) \coloneqq \iota y\,\bigl((x,y)\in F\bigr)}{2}}
  \proofstep{1}{y=F(x)}{\FormulaRefAuto{(x,y)\in F\dsep (x,z)\in F \vdash y=z}{1,2}}
\end{tabproof}

\FormulaThmDelta{x\in A\dsep (x,y)\in F\vdash F(x)=y}{
\DeltaRow{Mengen}{x\dsep y\dsep A}
\DeltaRow{Funktionen}{F\colon A\to B}
}
\begin{tabproof}
  \proofstep{1}{(x,y)\in F}{\rA}
  \proofstep{2}{x\in A}{\rA}
  \proofstep{2}{(x,F(x))\in F}{\FormulaRefAuto{x\in A\vdash (F(x) \coloneqq \iota y\,\bigl((x,y)\in F\bigr)}{2}}
  \proofstep{1}{y=F(x)}{\FormulaRefAuto{(x,y)\in F\dsep (x,z)\in F \vdash y=z}{2,1}}
\end{tabproof}


\FormulaThmDelta{x\in A\vdash F(x)\in B}{
\DeltaRow{Mengen}{A\dsep B}
\DeltaRow{Funktionen}{F\colon A\to B}
}
\begin{tabproof}
  \proofstep{1}{x\in A}{\rA}
  \proofstep{}{F\subseteq A\times B}{\FormulaRefAuto{F \subseteq A \times B}}
  \proofstep{1}{(x,F(x))\in F}{\FormulaRefAuto{x\in A\vdash (F(x) \coloneqq \iota y\,\bigl((x,y)\in F\bigr)}{1}}
  \proofstep{1}{(x,F(x))\in A\times B}{\FormulaRefAuto{ A\subseteq B,\, x\in A \vdash x\in B }{2,3}}
  \proofstep{1}{F(x)\in B}{\FormulaRefAuto{(a,b)\in A\times B\vdash b\in B}{4}}
\end{tabproof}

\FormulaThmDelta{(x,y)\in F\vdash y=F(x)}{
\DeltaRow{Mengen}{x\dsep y\dsep A\dsep B}
\DeltaRow{Funktion}{F\colon A\to B}
}
\begin{tabproof}
  \proofstep{1}{(x,y)\in F}{\rA}
  \proofstep{1}{x\in A}{\FormulaRefAuto{(x,y)\in F\vdash x\in A}{1}}
  \proofstep{1}{(x,F(x))\in F}{\FormulaRefAuto{x\in A\vdash (F(x) \coloneqq \iota y\,\bigl((x,y)\in F\bigr)}{2}}
  \proofstep{1}{y=F(x)}{\FormulaRefAuto{(x,y)\in F\dsep (x,z)\in F \vdash y=z}{1,3}}
\end{tabproof}

\FormulaThmDelta{(x,y)\in F\vdash F(x)=y}{
\DeltaRow{Mengen}{x\dsep y\dsep A\dsep B}
\DeltaRow{Funktion}{F\colon A\to B}
}
\begin{tabproof}
  \proofstep{1}{(x,y)\in F}{\rA}
  \proofstep{1}{x\in A}{\FormulaRefAuto{(x,y)\in F\vdash x\in A}{1}}
  \proofstep{1}{(x,F(x))\in F}{\FormulaRefAuto{x\in A\vdash (F(x) \coloneqq \iota y\,\bigl((x,y)\in F\bigr)}{2}}
  \proofstep{1}{y=F(x)}{\FormulaRefAuto{(x,y)\in F\dsep (x,z)\in F \vdash y=z}{3,1}}
\end{tabproof}

\FormulaThmDelta{F(x)=G(x)\dsep (x,y)\in F\vdash (x,y)\in G}{
\DeltaRow{Mengen}{x\dsep A\dsep B}
\DeltaRow{Funktion}{F,G\colon A\to B}
}
\begin{tabproof}
  \proofstep{1}{F(x)=G(x)}{\rA}
  \proofstep{2}{(x,y)\in F}{\rA}
  \proofstep{2}{y=F(x)}{\FormulaRefAuto{(x,y)\in F\vdash F(x)=y}{2}}
  \proofstep{1,2}{y=G(x)}{\FormulaRefAuto{a=b, b=c \vdash a=c}{3,1}}
  \proofstep{2}{x\in A}{\FormulaRefAuto{(x,y)\in F\vdash x\in A}{2}}
  \proofstep{1,2}{(x,y)\in G}{\FormulaRefAuto{x\in A\dsep y=F(x)\vdash (x,y)\in F}{5,4}}
\end{tabproof}

\subsection{Zur Gleichheit von Funktionen}

\FormulaThmDelta{\forall x\in A (F(x)=G(x)) \vdash \forall (x,y)((x,y)\in F\rightarrow (x,y)\in G)}{
\DeltaRow{Mengen}{A\dsep B}
\DeltaRow{Funktion}{F,G\colon A\to B}
}
\begin{tabproof}
  \proofstep{1}{\forall x\in A (F(x)=G(x))}{\rA}
  \proofstep{2}{(x,y)\in F}{\rA}
  \proofstep{2}{x\in A}{\FormulaRefAuto{(x,y)\in F\vdash x\in A}{2}}
  \proofstep{1,2}{F(x)=G(x)}{\rRE{\rUE{1},4}}
  \proofstep{1,2}{(x,y)\in G}{\FormulaRefAuto{F(x)=G(x)\dsep (x,y)\in F\vdash (x,y)\in G}{5,3}}
  \proofstep{1}{(x,y)\in F\rightarrow (x,y)\in G}{\rRI{2,6}}
  \proofstep{1}{\forall (x,y)((x,y)\in F\rightarrow (x,y)\in G)}{\rUI{7}}
\end{tabproof}



\FormulaThmDelta{\forall x\in A (F(x)=G(x)) \vdash \forall (x,y)((x,y)\in G\rightarrow (x,y)\in F)}{
\DeltaRow{Mengen}{A\dsep B}
\DeltaRow{Funktion}{F,G\colon A\to B}
}
\begin{tabproof}
  \proofstep{1}{\forall x\in A (F(x)=G(x))}{\rA}
  \proofstep{2}{x\in A}{\rA}
  \proofstep{1,2}{F(x)=G(x)}{\rRE{\rUE{1},2}}
  \proofstep{1,2}{G(x)=F(x)}{\FormulaRefAuto{a=b\vdash b=a}{3}}
  \proofstep{1}{\forall x\in A(G(x)=F(x))}{\rUI{\rRI{2,4}}}
  \proofstep{1}{\forall (x,y)((x,y)\in G\rightarrow (x,y)\in F)}{\FormulaRefAuto{\forall x\in A (F(x)=G(x)) \vdash \forall (x,y)((x,y)\in F\rightarrow (x,y)\in G)}{5}}
\end{tabproof}

\FormulaThmDelta{\forall x\in A (F(x)=G(x)) \eqvdash \forall (x,y)((x,y)\in F\leftrightarrow (x,y)\in G)}{
\DeltaRow{Mengen}{A\dsep B}
\DeltaRow{Funktion}{F,G\colon A\to B}
}
\begin{tabproofsplit}
\proofpart{\(\vdash\)}
  \proofstep{1}{\forall x\in A (F(x)=G(x))}{\rA}
  \proofstep{1}{\forall (x,y)((x,y)\in F\rightarrow (x,y)\in G)}{\FormulaRefAuto{\forall x\in A (F(x)=G(x)) \vdash \forall (x,y)((x,y)\in F\rightarrow (x,y)\in G)}{1}}
  \proofstep{1}{\forall (x,y)((x,y)\in G\rightarrow (x,y)\in F)}{\FormulaRefAuto{\forall x\in A (F(x)=G(x)) \vdash \forall (x,y)((x,y)\in G\rightarrow (x,y)\in F)}{1}}
  \proofstep{1}{\forall (x,y)((x,y)\in F\leftrightarrow (x,y)\in G)}{\FormulaRefAuto{\forall x (P(x) \leftrightarrow Q(x)) \dashv \vdash \forall x (P(x) \rightarrow Q(x)) \land \forall x (Q(x) \rightarrow P(x))}{2,3}}
\closeproofpart
\proofpart{\(\dashv\)}
  \proofstep{1}{\forall (x,y)((x,y)\in F\leftrightarrow (x,y)\in G)}{\rA}
  \proofstep{2}{x\in A}{\rA}
  \proofstep{2}{(x,F(x))\in F}{\FormulaRefAuto{x\in A\vdash (F(x) \coloneqq \iota y\,\bigl((x,y)\in F\bigr)}{2}}
  \proofstep{1}{(x,F(x))\in F\leftrightarrow (x,F(x))\in G}{\rUI{1}}
  \proofstep{1,2}{(x,F(x))\in G}{\FormulaRefAuto{P \leftrightarrow Q, P \vdash Q}{4,3}}
  \proofstep{2}{(x,G(x))\in G}{\FormulaRefAuto{x\in A\vdash (F(x) \coloneqq \iota y\,\bigl((x,y)\in F\bigr)}{2}}
  \proofstep{1,2}{F(x)=G(x)}{\FormulaRefAuto{(x,y)\in F\dsep (x,z)\in F \vdash y=z}{5,6}}
  \proofstep{1,2}{\forall x\in A(F(x)=G(x))}{\rUI{\rRI{2,7}}}
\closeproofpart
\end{tabproofsplit}

\FormulaThmDelta{\forall x\in A (F(x)=G(x)) \eqvdash F=G}{
\DeltaRow{Mengen}{A\dsep B}
\DeltaRow{Funktion}{F,G\colon A\to B}
}
\begin{tabproofwide}
  \proofstepwide{\forall x\in A (F(x)=G(x))}{\leftrightarrow}{\forall (x,y)((x,y)\in F\leftrightarrow (x,y)\in G)}%
    {\FormulaRefAuto{\forall x\in A (F(x)=G(x)) \eqvdash \forall (x,y)((x,y)\in F\leftrightarrow (x,y)\in G)}}
  \proofstepwide{}{\leftrightarrow}{F=G}%
    {\FormulaRefAuto{\forall x\, (x \in A \leftrightarrow x \in B) \eqvdash A = B}{1}}
   \proofstepwide{\forall x\in A (F(x)=G(x))}{\leftrightarrow}{F=G}%
    {\rChain{1,2}}
\end{tabproofwide}

\FormulaThmDelta{x\in A\dsep F=G\vdash F(x)=G(x)}{
\DeltaRow{Mengen}{A\dsep B}
\DeltaRow{Funktion}{F,G\colon A\to B}
}
\begin{tabproof}
    \proofstep{1}{x\in A}{\rA}
    \proofstep{2}{F=G}{\rA}
    \proofstep{2}{\forall x\in A (F(x)=G(x))}{\FormulaRefAuto{\forall x\in A (F(x)=G(x)) \eqvdash F=G}{2}}
    \proofstep{1,2}{\forall x\in A (F(x)=G(x))}{\rRE{1,\rUI{3}}}
\end{tabproof}


\section{Axiom der Ersetzung}

\FormulaAxiomDelta[Ersetzung]{
\exists B\;\forall y\;\bigl( y\in B\;\leftrightarrow\; \exists x\in A\;y=F(x) \bigr)
}{
\DeltaRow{Mengen}{A}
\DeltaRow{Funktionen}{F\colon A\to B}
}

\section{Die Bildmenge}

%%begin novalidate
\FormulaDefDelta[Bildmenge]
{F[A] \coloneqq \iota C\Bigl(\forall y\;\bigl( y\in C \leftrightarrow \exists x\in A\,y=F(x) \bigr)\Bigr)}
{
  \DeltaRow{Mengen}{A}
  \DeltaRow{Funktionen}{F\colon A\to B}
}
%%end novalidate

\FormulaThmDelta{y\in F[A]\eqvdash \exists x\in A\,y=F(x)}
{
  \DeltaRow{Mengen}{x\dsep y\dsep A}
  \DeltaRow{Funktionen}{F\colon A\to B }
}
\begin{tabproof}
  \proofstep{}{ \forall y\;\bigl( y\in R[A] \leftrightarrow \exists x\in A\,y=R(x)\bigr) }{\FormulaRefAuto{F[A] \coloneqq \iota C\Bigl(\forall y\;\bigl( y\in C \leftrightarrow \exists x\in A\,y=F(x) \bigr)\Bigr)}}
  \proofstep{}{ y\in R[A] \leftrightarrow \exists x\in A\,y=R(x) }{\rUE{1}}
\end{tabproof}

\FormulaThmDelta{x\in A\vdash F(x)\in F[A]}
{
  \DeltaRow{Mengen}{A\dsep B}
  \DeltaRow{Funktionen}{F\colon A\to B }
}
\begin{tabproof}
  \proofstep{1}{x\in A}{\rA}
  \proofstep{}{F(x)=F(x)}{\rII}
  \proofstep{1}{x\in A\land F(x)=F(x)}{\rAI{1,2}}
  \proofstep{1}{\exists x\in A\, F(x)=F(x)}{\rEI{3}}
  \proofstep{1}{F(x)\in F[A]}{\FormulaRefAuto{y\in F[A]\eqvdash \exists x\in A\,y=F(x)}{4}}
\end{tabproof}

\FormulaThmDelta{F\subseteq A\times F[A]}
{
  \DeltaRow{Mengen}{A\dsep B}
  \DeltaRow{Funktionen}{F\colon A\to B }
}
\begin{tabproof}
    \proofstep{1}{(x,y)\in F}{\rA}
    \proofstep{1}{x\in A}{\FormulaRefAuto{(x,y)\in F\vdash x\in A}{1}}
    \proofstep{1}{F(x)\in F[A]}{\FormulaRefAuto{x\in A\vdash F(x)\in F[A]}{2}}
    \proofstep{1}{y=F(x)}{\FormulaRefAuto{x\in A\dsep (x,y)\in F\vdash F(x)=y}{2,1}}
    \proofstep{1}{y\in F[A]}{\rIE{4,3}}
    \proofstep{1}{(x,y)\in A\times F[A]}{\FormulaRefAuto{a\in A\dsep b\in B\vdash (a,b)\in A\times B}{2,5}}
    \proofstep{1}{F\subseteq A\times F[A]}{\FormulaRefAuto{A \subseteq B := \forall x\,(x\in A \rightarrow x\in B)}{\rUI{\rRI{1,6}}}}
\end{tabproof}


\FormulaDefDelta[Graph einer Funktion]{\mathrm{Graph}(F) := \{\, (x,y) \in A \times B \mid (x,y)\in F \,\}}%
{
\DeltaRow{Mengen}{x\dsep y\dsep A\dsep B}
\DeltaRow{Funktionen}{F\colon A\to B}
}


\section{Eigenschaften von Funktionen}
\subsection{Injektivität}

\FormulaDefDeltaK[Begriff der injektiven Funktion]{F\colon A\rightarrowtail B}{Injektive Funktion}{
  \DeltaRow{Mengen}{A\dsep B\dsep x\dsep y}
  \DeltaRow{Funktion}{F\colon A\to B}[\FormulaRefAuto{Funktion}]
  %
  \DeltaRow{\textbf{Axiome}}{}
  %
  % — Injektivität —
  \DeltaRow{Injektivität}
           {x\in A\dsep y\in A\dsep F(x)=F(y)\vdash x=y}
           [\FormulaRefAuto{x\in A\dsep y\in A\dsep F(x)=F(y)\vdash x=y}]
  \DeltaRow{\textbf{Neue Symbole}}{}
  \DeltaRow{Injektive Funktionen}{}
}

\FormulaAxiomDelta[Injektivität]%
{x\in A\dsep y\in A\dsep F(x)=F(y)\vdash x=y}%
{
\DeltaRow{Mengen}{x\dsep y\dsep A\dsep B}
\DeltaRow{Funktionen}{F\colon A\to B}
}[Injektivität]


\FormulaThmDelta%
{(x,z_1)\in F\dsep (y,z_2)\in F\dsep F(x)=F(y)\vdash x=y}%
{
\DeltaRow{Mengen}{x\dsep y\dsep z_1\dsep z_2\dsep A\dsep B}
\DeltaRow{Injektive Funktionen}{F\colon A\to B}
}
\begin{tabproof}
  \proofstep{1}{(x,z_1)\in F}{\rA}
  \proofstep{2}{(y,z_2)\in F}{\rA}
  \proofstep{3}{F(x)=F(y)}{\rA}
  \proofstep{1}{x\in A}{\FormulaRefAuto{(x,y)\in F\vdash x\in A}{1}}
  \proofstep{2}{y\in A}{\FormulaRefAuto{(x,y)\in F\vdash x\in A}{2}}
  \proofstep{1,2,3}{x=y}{\FormulaRefAuto{x\in A\dsep y\in A\dsep F(x)=F(y)\vdash x=y}{4,5,3}}
\end{tabproof}

\subsection{Surjektivität}

\FormulaDefDeltaK[Begriff der surjektiven Funktion]{F\colon A\twoheadrightarrow B}{Surjektive Funktion}{
  \DeltaRow{Mengen}{A\dsep B\dsep x\dsep y}
  \DeltaRow{Funktion}{F\colon A\to B}[\FormulaRefAuto{Funktion}]
%
  \DeltaRow{\textbf{Axiome}}{}
  %
  % — Surjekvität —
  \DeltaRow{Surjektivität}
           {y\in B\vdash \exists x\in A\; F(x)=y}
           [\FormulaRefAuto{y\in B\vdash \exists x\in A\; F(x)=y}]
  \DeltaRow{\textbf{Neue Symbole}}{}
  \DeltaRow{Surjektive Funktionen}{ }
}

\FormulaAxiomDelta[Surjektivität]%
{y\in B\vdash \exists x\in A\; F(x)=y}
{
\DeltaRow{Mengen}{y\dsep A\dsep B}
\DeltaRow{Funktionen}{F\colon A\to B}
}[Surjektivität]

\FormulaThmDelta{F\colon A\to B \vdash F\colon A\twoheadrightarrow F[A]}{
\DeltaRow{Mengen}{A\dsep B\dsep F}
}
\begin{tabproof}
    \proofstep{1}{F\colon A\to B}{\rA}
    \proofstep{1}{F\subseteq A\times F[A]}{\FormulaRefAuto{F\subseteq A\times F[A]}{1}}
    \proofstep{1}{\forall x,y\,((x,y)\in F\dsep (x,z)\in F\rightarrow y=z)}{\FormulaRefAuto{(x,y)\in F\dsep (x,z)\in F \vdash y=z}{1}}
    \proofstep{1}{\forall x\,(x\in A\rightarrow \exists y (x,y)\in F)}{\FormulaRefAuto{x\in A \vdash \exists y\,(x,y)\in F}{1}}
    \proofstep{1}{\forall y\,(y\in F[A]\rightarrow \exists x\in A (F(x)=y))}{\FormulaRefAuto{y\in F[A]\eqvdash \exists x\in A\,y=F(x)}{1}}
    \proofstep{1}{F\colon A\twoheadrightarrow F[A]}{\FormulaRefAuto{Surjektive Funktion}{2,3,4,5}}
\end{tabproof}

\FormulaThmDelta{F\colon A\to \mathcal{P}(A) \vdash \neg(F\colon A\sur \mathcal{P}(A))}{
\DeltaRow{Mengen}{A\dsep B\dsep F}
}
\begin{tabproofwide}
    \proofstepwidestar[1]{F\colon A\sur \mathcal{P}(A)}{\rA}
    \proofstepwidestar[1]{\forall y\in \mathcal{P}(A)\,\exists x\in A\,(F(x)=y)}{\FormulaRefAuto{y\in B\vdash \exists x\in A\; F(x)=y}{1}}
    \proofstepwidestar[1]{\exists x\in A\,\bigl(F(x)=\{x\in A\mid x\notin F(x)\}\bigr)}{\rUE{2}}
    \proofstepwidestar[4]{a\in A\land F(a)=\{x\in A\mid x\notin F(x)\}}{\rA}
    \proofstepwidestar[4]{a\in A}{\rAEa{4}}
    \proofstepwidestar[4]{F(a)=\{x\in A\mid x\notin F(x)\}}{\rAEb{4}}
    \proofstepwide[4]{a\in F(a)}{\leftrightarrow}{a\in \{x\in A\mid x\notin F(x)\}}{\rUE{\FormulaRefAuto{\forall x\, (x \in A \leftrightarrow x \in B) \eqvdash A = B}{6}}}
    \proofstepwide[4]{}{\leftrightarrow}{a\in A \land a\notin F(a)}{\FormulaRefAuto{x \in \{x \in A \mid P(x)\} \eqvdash x \in A \land P(x)}{7}}
    \proofstepwide[4]{}{\leftrightarrow}{a\notin F(a)}{\FormulaRefAuto{Q\vdash (P\leftrightarrow Q\land R)\leftrightarrow (P\leftrightarrow R)}{5}}
    \proofstepwide[4]{a\in F(a)}{\leftrightarrow}{a\notin F(a)}{\rChain{7,9}}
    \proofstepwidestar[]{\neg (a\in F(a)\leftrightarrow a\notin F(a))}{\FormulaRefAuto{\neg(P\leftrightarrow\neg P)}}
    \proofstepwidestar[4]{\bot}{\rBI{10,11}}
    \proofstepwidestar[1]{\bot}{\rEE{3,4,12}}
    \proofstepwidestar[]{\neg(F\colon A\sur \mathcal{P}(A))}{\rCI{1,13}}
\end{tabproofwide}


\subsection{Bijektivität}

\FormulaDefDeltaK[Begriff der bijektiven Funktion]{F\colon A\bij B}{Bijektive Funktion}{
  \DeltaRow{Mengen}{A\dsep B\dsep x\dsep y}
  \DeltaRow{Funktion}{F\colon A\to B}[\FormulaRefAuto{Funktion}]
%
  \DeltaRow{\textbf{Axiome}}{}
  %
  % — Injektivität —
  \DeltaRow{Injektivität}
           {x\in A\dsep y\in A\dsep F(x)=F(y)\vdash x=y}
           [\FormulaRefAuto{x\in A\dsep y\in A\dsep F(x)=F(y)\vdash x=y}]
  %
  % — Surjekvität —
  \DeltaRow{Surjektivität}
           {y\in B\vdash \exists x\in A\; F(x)=y}
           [\FormulaRefAuto{y\in B\vdash \exists x\in A\; F(x)=y}]
  \DeltaRow{\textbf{Neue Symbole}}{}
  \DeltaRow{Bijektive Funktionen}{ }
}

\FormulaThmDelta{F\colon A\inj B\dsep F\colon A\sur B \vdash F\colon A\bij B}{
\DeltaRow{Mengen}{A\dsep B}
\DeltaRow{Funktionen}{F\colon A\to B}
}
\begin{tabproof}
    \proofstep{1}{F\colon A\inj B}{\rA}
    \proofstep{2}{F\colon A\sur B}{\rA}
    \proofstep{1}{\forall x,y\in A\,(F(x)=F(y)\rightarrow x=y)}{\FormulaRefAuto{x\in A\dsep y\in A\dsep F(x)=F(y)\vdash x=y}{1}}
    \proofstep{1}{\forall y\in B\,\exists x\in A\,(F(x)=y)}{\FormulaRefAuto{y\in B\vdash \exists x\in A\; F(x)=y}{2}}
    \proofstep{1,2}{F\colon A\bij B}{\FormulaRefAuto{Bijektive Funktion}{3,4}}
\end{tabproof}

\FormulaThmDelta{F\colon A\bij B \vdash F\colon A\inj B}{
\DeltaRow{Mengen}{A\dsep B}
}
\begin{tabproof}
    \proofstep{1}{F\colon A\bij B}{\rA}
    \proofstep{1}{F\colon A\to B}{\FormulaRefAuto{Bijektive Funktion}{1}}
    \proofstep{1}{\forall x,y\in A\, (F(x)=F(y)\rightarrow x=y)}{\FormulaRefAuto{Bijektive Funktion}{1}}
    \proofstep{1}{F\colon A\inj B}{\FormulaRefAuto{Injektive Funktion}{2,3}}
\end{tabproof}

\FormulaThmDelta{F\colon A\bij B \vdash F\colon A\sur B}{
\DeltaRow{Mengen}{A\dsep B}
}
\begin{tabproof}
    \proofstep{1}{F\colon A\bij B}{\rA}
    \proofstep{1}{F\colon A\to B}{\FormulaRefAuto{Bijektive Funktion}{1}}
    \proofstep{1}{\forall y\in B\exists x\in A\; F(x)=y}{\FormulaRefAuto{Bijektive Funktion}{1}}
    \proofstep{1}{F\colon A\inj B}{\FormulaRefAuto{Injektive Funktion}{2,3}}
\end{tabproof}

\FormulaThmDelta{F\colon A\inj B\land F\colon A\sur B \eqvdash F\colon A\bij B }{
\DeltaRow{Mengen}{A\dsep B}
}
\begin{tabproofsplit}
    \proofpart{\(\vdash\)}
    \proofstep{1}{F\colon A\inj B\land F\colon A\sur B}{\rA}
    \proofstep{1}{F\colon A\inj B}{\rAEa{1}}
    \proofstep{1}{F\colon A\sur B}{\rAEb{2}}
    \proofstep{1}{F\colon A\bij B}{\FormulaRefAuto{F\colon A\inj B\dsep F\colon A\sur B \vdash F\colon A\bij B}{2,3}}
    \closeproofpart
    \proofpart{\(\dashv\)}
    \proofstep{1}{F\colon A\bij B}{\rA}
    \proofstep{1}{F\colon A\inj B}{\FormulaRefAuto{F\colon A\bij B \vdash F\colon A\inj B}{1}}
    \proofstep{1}{F\colon A\sur B}{\FormulaRefAuto{F\colon A\bij B \vdash F\colon A\sur B}{1}}
    \proofstep{1}{F\colon A\inj B\land F\colon A\sur B}{\rAI{2,3}}
    \closeproofpart
\end{tabproofsplit}


\section{Beispiele von Funktionen}

\subsection{Einschränkung einer Funktion}

\subsubsection{Definition der Einschränkung}

% — Definition —
\FormulaDefDeltaK[Einschränkung von \(F\) auf \(C\)]{%
C\subseteq A \vdash F\restriction_C := \{\, (x,y) \in F \mid x \in C \,\}
}{Einschränkung}{
  \DeltaRow{Mengen}{x \dsep y \dsep A \dsep B \dsep C}
  \DeltaRow{Funktionen}{F\colon A\to B}
}

\subsubsection{Die Funktionseigenschaften}

% — Existenz als Menge geordneter Paare / Teilmenge von F —
\FormulaThmDelta[Existenz als Teilmenge von \(F\)]{%
C\subseteq A \vdash F\restriction_C \subseteq F
}{
  \DeltaRow{Mengen}{x \dsep y \dsep A \dsep B \dsep C}
  \DeltaRow{Funktionen}{F\colon A\to B}
}
\begin{tabproof}
  \proofstep{1}{C\subseteq A}{\rA}
  \proofstep{}{%
    F\restriction_C = \{\, (x,y) \in F \mid x \in C \,\}
  }{\FormulaRefAuto{C\subseteq A \vdash F\restriction_C := \{\, (x,y) \in F \mid x \in C \,\}}{1}}
  \proofstep{}{%
    \{\, (x,y) \in F \mid x \in C \,\} \subseteq F
  }{\FormulaRefAuto{\{ x \in A \mid P(x) \} \subseteq A}}
  \proofstep{}{%
    F\restriction_C \subseteq F
  }{\rIE{2,3}}
\end{tabproof}


% — Projektion auf die erste Komponente: x liegt in C —
\FormulaThmDelta{%
C\subseteq A \dsep (x,y)\in F\restriction_C \vdash x\in C
}{
  \DeltaRow{Mengen}{x \dsep y \dsep A \dsep B \dsep C}
  \DeltaRow{Funktionen}{F\colon A\to B}
}
\begin{tabproof}
  \proofstep{1}{C\subseteq A}{\rA}
  \proofstep{2}{(x,y)\in F\restriction_C}{\rA}
  \proofstep{1,2}{(x,y)\in \{\, (x,y) \in F \mid x \in C \,\}}%
    {\rIE{\FormulaRefAuto{C\subseteq A \vdash F\restriction_C := \{\, (x,y) \in F \mid x \in C \,\}}{1},2}}
  \proofstep{1,2}{x\in C}{\FormulaRefAuto{x\in \{x\in A\mid P(x)\}\vdash P(x)}{3}}
\end{tabproof}

\FormulaThmDelta{%
C\subseteq A \dsep (x,y)\in F\restriction_C \vdash (x,y)\in F
}{
  \DeltaRow{Mengen}{x \dsep y \dsep A \dsep B \dsep C}
  \DeltaRow{Funktionen}{F\colon A\to B}
}
\begin{tabproof}
  \proofstep{1}{C\subseteq A}{\rA}
  \proofstep{2}{(x,y)\in F\restriction_C}{\rA}
  \proofstep{1,2}{(x,y)\in \{\, (x,y) \in F \mid x \in C \,\}}%
    {\rIE{\FormulaRefAuto{C\subseteq A \vdash F\restriction_C := \{\, (x,y) \in F \mid x \in C \,\}}{1},2}}
  \proofstep{1,2}{(x,y)\in F}%
    {\FormulaRefAuto{x\in \{x\in A\mid P(x)\}\vdash x\in A}{3}}
\end{tabproof}

% — Projektion auf die zweite Komponente: y liegt in B (über F) —
\FormulaThmDelta{%
C\subseteq A \dsep (x,y)\in F\restriction_C \vdash y\in B
}{
  \DeltaRow{Mengen}{x \dsep y \dsep A \dsep B \dsep C}
  \DeltaRow{Funktionen}{F\colon A\to B}
}
\begin{tabproof}
  \proofstep{1}{C\subseteq A}{\rA}
  \proofstep{2}{(x,y)\in F\restriction_C}{\rA}
  \proofstep{1,2}{(x,y)\in F}{\FormulaRefAuto{C\subseteq A \dsep (x,y)\in F\restriction_C \vdash (x,y)\in F}{1,2}}
  \proofstep{1,2}{y\in B}{\FormulaRefAuto{(a,b)\in A\times B\vdash b\in B}{3}}
\end{tabproof}


% — Charakterisierung über F und C: (x,y) in F|_C <-> x in C und (x,y) in F —

\FormulaThmDelta{%
C\subseteq A \dsep (x,y)\in F\restriction_C \vdash x\in C\land (x,y)\in F
}{
  \DeltaRow{Mengen}{x \dsep y \dsep A \dsep B \dsep C}
  \DeltaRow{Funktionen}{F\colon A\to B}
}
\begin{tabproof}
  \proofstep{1}{C\subseteq A}{\rA}
  \proofstep{2}{(x,y)\in F\restriction_C}{\rA}
  \proofstep{1,2}{x\in C}{\FormulaRefAuto{C\subseteq A \dsep (x,y)\in F\restriction_C \vdash x\in C}{1,2}}
  \proofstep{1,2}{(x,y)\in F}{\FormulaRefAuto{C\subseteq A \dsep (x,y)\in F\restriction_C \vdash (x,y)\in F}{1,2}}
  \proofstep{1,2}{x\in C\land (x,y)\in F}{\rAI{3,4}}
\end{tabproof}

\FormulaThmDelta{%
C\subseteq A \dsep x\in C\dsep (x,y)\in F \vdash (x,y)\in F\restriction_C
}{
  \DeltaRow{Mengen}{x \dsep y \dsep A \dsep B \dsep C}
  \DeltaRow{Funktionen}{F\colon A\to B}
}
\begin{tabproof}
  \proofstep{1}{C\subseteq A}{\rA}
  \proofstep{2}{x\in C}{\rA}
  \proofstep{3}{(x,y)\in F}{\rA}
  \proofstep{2,3}{(x,y)\in \{\, (x,y) \in F \mid x \in C \,\}}%
    {\FormulaRefAuto{x\in A, P(x)\vdash x\in\{x\in A\mid P(x)\}}{3,2}}
  \proofstep{1,2,3}{(x,y)\in F\restriction_C}{\rIE{\FormulaRefAuto{C\subseteq A \vdash F\restriction_C := \{\, (x,y) \in F \mid x \in C \,\}}{1},4}}
\end{tabproof}

\FormulaThmDelta{%
(x,y)\in F\restriction_C \eqvdash x\in C\land (x,y)\in F
}{
  \DeltaRow{Mengen}{x \dsep y \dsep A \dsep B \dsep C}
  \DeltaRow{Funktionen}{F\colon A\to B}
  \DeltaRow{Teilmengen}{C\subseteq A}
}
\begin{tabproofsplit}
  \proofpart{\(\vdash\)}
    \proofstep{1}{(x,y)\in F\restriction_C}{\rA}
    \proofstep{1}{x\in C\land (x,y)\in F}{\FormulaRefAuto{C\subseteq A \dsep (x,y)\in F\restriction_C \vdash x\in C\land (x,y)\in F}{1}}
  \closeproofpart
  \proofpart{\(\dashv\)}
    \proofstep{1}{x\in C\land (x,y)\in F}{\rA}
    \proofstep{1}{x\in C}{\rAEa{1}}
    \proofstep{1}{(x,y)\in F}{\rAEb{1}}
    \proofstep{1}{(x,y)\in F\restriction_C}{\FormulaRefAuto{C\subseteq A \dsep x\in C\dsep (x,y)\in F \vdash (x,y)\in F\restriction_C}{2,3}}
  \closeproofpart
\end{tabproofsplit}


% — Funktionswerte der Einschränkung und Übereinstimmung mit F —

\FormulaThmDelta{%
C\subseteq A \dsep x\in C \vdash (x,F(x))\in F\restriction_C
}{
  \DeltaRow{Mengen}{x \dsep A \dsep B \dsep C}
  \DeltaRow{Funktionen}{F\colon A\to B}
}
\begin{tabproof}
  \proofstep{1}{C\subseteq A}{\rA}
  \proofstep{2}{x\in C}{\rA}
  \proofstep{1,2}{x\in A}{\FormulaRefAuto{A\subseteq B,x\in A\vdash x\in B}{1,2}}
  \proofstep{1,2}{(x,F(x))\in F}{\FormulaRefAuto{x\in A\dsep y\in A\dsep x=y\vdash F(x)=F(y)}{3}}
  \proofstep{1,2}{(x,F(x))\in F\restriction_C}{\FormulaRefAuto{C\subseteq A \dsep x\in C\dsep (x,y)\in F \vdash (x,y)\in F\restriction_C}{1,2,4}}
\end{tabproof}

\FormulaThmDelta[Existenz als Menge geordneter Paare der Einschränkung]{%
F\restriction_C \subseteq C\times B
}{
  \DeltaRow{Mengen}{x \dsep y \dsep A \dsep B \dsep C}
  \DeltaRow{Teilmengen}{C\subseteq A}
  \DeltaRow{Funktionen}{F\colon A\to B}
}
\begin{tabproof}
  \proofstep{1}{(x,y)\in F\restriction_C}{\rA}

  \proofstep{1}{(x,y)\in F}{%
    \FormulaRefAuto{C\subseteq A \dsep (x,y)\in F\restriction_C \vdash (x,y)\in F}{1}
  }
  \proofstep{1}{x\in C}{%
    \FormulaRefAuto{C\subseteq A \dsep (x,y)\in F\restriction_C \vdash x\in C}{2}
  }

 \proofstep{1}{y\in B}{%
    \FormulaRefAuto{C\subseteq A \dsep (x,y)\in F\restriction_C \vdash y\in B}{2}
  }

 \proofstep{1}{(x,y)\in C\times B}{%
    \FormulaRefAuto{a\in A\dsep b\in B\vdash (a,b)\in A\times B}{3,4}
  }

  \proofstep{}{\forall x,y\, ((x,y)\in F\restriction_C\rightarrow (x,y)\in C\times B)}{%
    \rUI{\rRI{1,5}}
  }

  \proofstep{}{F\restriction_C \subseteq C\times B}{
  \FormulaRefAuto{A \subseteq B := \forall x\,(x\in A \rightarrow x\in B)}{6}
  }
\end{tabproof}

\FormulaThmDeltaK[Funktionale Eindeutigkeit der Einschränkung]{%
(x,y_1)\in F\restriction_C \dsep (x,y_2)\in F\restriction_C \vdash y_1 = y_2
}{%
C\subseteq A \dsep F\colon A\to B \dsep (x,y_1)\in F\restriction_C \dsep (x,y_2)\in F\restriction_C 
\vdash y_1 = y_2
}{
  \DeltaRow{Mengen}{x \dsep y_1 \dsep y_2 \dsep A \dsep B \dsep C}
  \DeltaRow{Funktionen}{F\colon A\to B}
  \DeltaRow{Teilmengen}{C\subseteq A}
}
\begin{tabproof}
  % Annahmen
  \proofstep{1}{(x,y_1)\in F\restriction_C}{\rA}
  \proofstep{2}{(x,y_2)\in F\restriction_C}{\rA}

  % Von der Einschränkung zurück zur ursprünglichen Funktion
  \proofstep{1}{(x,y_1)\in F}{%
    \FormulaRefAuto{C\subseteq A \dsep (x,y)\in F\restriction_C \vdash (x,y)\in F}{1}
  }
  \proofstep{2}{(x,y_2)\in F}{%
    \FormulaRefAuto{C\subseteq A \dsep (x,y)\in F\restriction_C \vdash (x,y)\in F}{2}
  }

  % Funktionale Eindeutigkeit von F nutzen
  \proofstep{1,2}{y_1 = y_2}{%
    \FormulaRefAuto{(x,y)\in F\dsep (x,z)\in F \vdash y=z}{3,4}
  }
\end{tabproof}

\FormulaThmDeltaK[Totalität der Einschränkung]{%
C\subseteq A\dsep x\in C \vdash \exists y\, (x,y)\in F\restriction_C
}{%
C\subseteq A \dsep \forall y\in B\,\exists x\in C\,F(x)=y \dsep y\in B \vdash \exists x\, (x,y)\in F\restriction_C
}{
  \DeltaRow{Mengen}{x \dsep y \dsep A \dsep B \dsep C}
  \DeltaRow{Funktionen}{F\colon A\to B}
}
\begin{tabproof}
  % Annahmen
  \proofstep{1}{C\subseteq A}{\rA}
  \proofstep{2}{x\in C}{\rA}
  \proofstep{1,2}{x\in A}{\FormulaRefAuto{A\subseteq B, x\in A\vdash x\in B}{1,2}}
  \proofstep{1,2}{\exists y\,(x,y)\in F}{%
  \FormulaRefAuto{x\in A \vdash \exists y\,(x,y)\in F}{3}%
  }
  \proofstep{5}{(x,b)\in F}{\rA}
  \proofstep{1,2,5}{(x,b)\in F\restriction_C}{\FormulaRefAuto{C\subseteq A \dsep x\in C\dsep (x,y)\in F \vdash (x,y)\in F\restriction_C}{1,2,5}}
  \proofstep{1,2,5}{\exists y\,(x,y)\in F\restriction_C}{\rEI{6}}
  \proofstep{1,2}{\exists y\,(x,y)\in F\restriction_C}{\rEE{4,5,7}}
\end{tabproof}

% — F\restriction_C ist eine Funktion C -> B —

\FormulaThmDelta[\(F\restriction_C\) als Funktion]{%
 F\restriction_C\colon C\to B
}{
  \DeltaRow{Mengen}{A \dsep B \dsep C}
  \DeltaRow{Teilmengen}{C\subseteq A}
  \DeltaRow{Funktionen}{F\colon A\to B}
}
\begin{tabproof}
  \proofstep{}{F\restriction_C \subseteq C\times B}{\FormulaRefAuto{F\restriction_C\subseteq C\times B}}
  \proofstep{}{\forall (x,y_1),(x,y_2)\in F\restriction_C\, y_1 = y_2}{\FormulaRefAuto{(x,y_1)\in F\restriction_C \dsep (x,y_2)\in F\restriction_C \vdash y_1 = y_2}}
  \proofstep{}{\forall x\in C\, \exists y\, (x,y)\in F\restriction_C}{\FormulaRefAuto{C\subseteq A\dsep x\in C \vdash \exists y\, (x,y)\in F\restriction_C}}
  \proofstep{}{F\restriction_C\colon C\to B}{\FormulaRefAuto{Funktion}{1,2,3}}
\end{tabproof}

\subsubsection{Weitere Eigenschaften}

\FormulaThmDelta[Funktionswerte der Einschränkung]{%
C\subseteq A \dsep x\in C \vdash (F\restriction_C)(x)=F(x)
}{
  \DeltaRow{Mengen}{x \dsep A \dsep B \dsep C}
  \DeltaRow{Funktionen}{F\colon A\to B}
}
\begin{tabproof}
  \proofstep{1}{C\subseteq A}{\rA}
  \proofstep{2}{x\in C}{\rA}
  \proofstep{1,2}{(x,F(x))\in F\restriction_C}{\FormulaRefAuto{C\subseteq A \dsep x\in C \vdash (x,F(x))\in F\restriction_C}{1,2}}
  \proofstep{1,2}{(F\restriction_C)(x)=F(x)}{\FormulaRefAuto{(x,y)\in F \vdash F(x)=y}{3}}
\end{tabproof}

\subsection{Umkehrfunktion}

\subsubsection{Definition}

\FormulaDefDelta[Inverse Relation]{%
F^{-1} := \{\, (y,x) \in B \times A \mid (x,y)\in F \}
}
{
\DeltaRow{Mengen}{x\dsep y\dsep A\dsep B}
\DeltaRow{Funktionen}{F\colon A\to B}
}

\subsubsection{Die Funktionseigenschaften}


\FormulaThmDelta[Existenz als Menge geordneter Paare]{%
F^{-1} \subseteq B\times A}
{
\DeltaRow{Mengen}{x\dsep y\dsep A\dsep B}
\DeltaRow{Funktionen}{F\colon A\to B}
}

\begin{tabproof}
\proofstep{}{F^{-1} = \{\, (y,x) \in B \times A \mid (x,y)\in F \}}{\FormulaRefAuto{F^{-1} := \{\, (y,x) \in B \times A \mid (x,y)\in F \}}}
\proofstep{}{\{\, (y,x) \in B \times A \mid (x,y)\in F \}\subseteq B\times A}{\FormulaRefAuto{\{ x \in A \mid P(x) \} \subseteq A}}
\proofstep{}{F^{-1}\subseteq B\times A}{\rIE{1,2}}
\end{tabproof}


\FormulaThmDelta{(x,y)\in F\eqvdash (y,x)\in F^{-1}}
{
\DeltaRow{Mengen}{x\dsep y}
\DeltaRow{Funktionen}{F}
}
\begin{tabproofsplit}
\proofpart{\(\vdash\)}
\proofstep{1}{(x,y)\in F}{\rA}
\proofstep{1}{(x,y)\in A\times B}{\FormulaRefAuto{A\subseteq B,\, x\in A \vdash x\in B}{1,\FormulaRefAuto{F \subseteq A \times B}}}
\proofstep{1}{(y,x)\in B\times A}{\FormulaRefAuto{(a,b)\in A\times B\eqvdash (b,a)\in B\times A}}
\proofstep{1}{(y,x)\in F^{-1}}{\rIE{\FormulaRefAuto{F^{-1} := \{\, (y,x) \in B \times A \mid (x,y)\in F \}},\FormulaRefAuto{x \in A, P(x)\vdash x \in \{x \in A \mid P(x)\}}{3,1}}}
\closeproofpart
\proofpart{\(\dashv\)}
\proofstep{1}{(x,y)\in F^{-1}}{\rA}
\proofstep{1}{(x,y)\in F}{\FormulaRefAuto{x \in \{x \in A \mid P(x)\}\vdash P(x)}{1}}
\closeproofpart
\end{tabproofsplit}

\FormulaThmDeltaK[Funktionale Eindeutigkeit von \(F^{-1}\)]{(y,x_1)\in F^{-1}\dsep (y,x_2)\in F^{-1} \vdash x_1 = x_2}{F injektiv\dsep (y,x_1)\in F^{-1}\dsep (y,x_2)\in F^{-1} \vdash x_1 = x_2}
{
\DeltaRow{Mengen}{x_1\dsep x_2\dsep y}
\DeltaRow{Injektive Funktionen}{F}
}
\begin{tabproof}
\proofstep{1}{(y,x_1)\in F^{-1}}{\rA}
\proofstep{2}{(y,x_2)\in F^{-1}}{\rA}
\proofstep{1}{(x_1,y)\in F}{\FormulaRefAuto{(x,y)\in F\eqvdash (y,x)\in F^{-1}}{1}}
\proofstep{2}{(x_2,y)\in F}{\FormulaRefAuto{(x,y)\in F\eqvdash (y,x)\in F^{-1}}{2}}
\proofstep{1}{y=F(x_1)}{\FormulaRefAuto{(x,y)\in F\vdash y=F(x)}{3}}
\proofstep{2}{y=F(x_2)}{\FormulaRefAuto{(x,y)\in F\vdash y=F(x)}{4}}
\proofstep{1,2}{F(x_1)=F(x_2)}{\FormulaRefAuto{a = b,\, a = c \vdash b = c}{5,6}}
\proofstep{1,2}{x_1=x_2}{\FormulaRefAuto{(x,z_1)\in F\dsep (y,z_2)\in F\dsep F(x)=F(y)\vdash x=y}{3,4,7}}
\end{tabproof}

\FormulaThmDeltaK[Totalität auf \(B\)]%
{y\in B\vdash \exists x\, (y,x)\in F^{-1}}{F surjektiv\dsep y\in B\vdash \exists x\, (y,x)\in F^{-1}}
{
\DeltaRow{Mengen}{x\dsep y\dsep A\dsep B}
\DeltaRow{Surjektive Funktionen}{F\colon A\to B}
}
\begin{tabproof}
  \proofstep{1}{y\in B}{\rA}
  \proofstep{1}{\exists x\in A\, F(x)=y}{\FormulaRefAuto{y\in B\vdash \exists x\in A F(x)=y}{1}}
  \proofstep{3}{x\in A\land F(x)=y}{\rA}
  \proofstep{3}{x\in A}{\rAEa{3}}
  \proofstep{3}{F(x)=y}{\rAEb{3}}
  \proofstep{3}{(x,y)\in F}{\FormulaRefAuto{x\in A\dsep F(x)=y\vdash (x,y)\in F}{4,5}}
  \proofstep{3}{(y,x)\in F^{-1}}{\FormulaRefAuto{(x,y)\in F\eqvdash (y,x)\in F^{-1}}{6}}
  \proofstep{3}{\exists x\,(y,x)\in F^{-1}}{\rEI{7}}
  \proofstep{1}{\exists x\,(y,x)\in F^{-1}}{\rEE{2,3,8}}
\end{tabproof}

\FormulaThmDelta[\(F^{-1}\) als Funktion]%
{F^{-1}\colon B\to A}
{
\DeltaRow{Mengen}{A\dsep B}
\DeltaRow{Bijektive Funktionen}{F\colon A\bij B}
}
\begin{tabproof}
    \proofstep{}{F^{-1} \subseteq B\times A}{\FormulaRefAuto{F^{-1} \subseteq B\times A}}
    \proofstep{}{\forall (y,x_1),(y,x_2)\in F^{-1}\, x_1=x_2}{\FormulaRefAuto{F injektiv\dsep (y,x_1)\in F^{-1}\dsep (y,x_2)\in F^{-1} \vdash x_1 = x_2}}
    \proofstep{}{\forall y\in B\exists x\, (y,x)\in F^{-1}}{\FormulaRefAuto{F surjektiv\dsep y\in B\vdash \exists x\, (y,x)\in F^{-1}}}
    \proofstep{}{F^{-1}\colon B\to A}{\FormulaRefAuto{Funktion}{1,2,3}}
\end{tabproof}

\subsubsection{Die Bijektionseigeschaften}

\FormulaThmDelta[Injektivität von \(F^{-1}\)]%
{x\in B\dsep y\in B\dsep F^{-1}(x)=F^{-1}(y)\vdash x=y}%
{
\DeltaRow{Mengen}{x\dsep y\dsep A\dsep B}
\DeltaRow{Bijektive Funktionen}{F\colon A\to B}
}
\begin{tabproof}
  \proofstep{1}{F^{-1}(x)=F^{-1}(y)}{\rA}
  \proofstep{2}{x\in B}{\rA}
  \proofstep{3}{y\in B}{\rA}
  \proofstep{1,2}{(x,F^{-1}(y))\in F^{-1}}{\FormulaRefAuto{x\in A\dsep F(x)=y\vdash (x,y)\in F}{2,1}}
  \proofstep{1,3}{(y,F^{-1}(x))\in F^{-1}}{\FormulaRefAuto{x\in A\dsep y=F(x)\vdash (x,y)\in F}{3,1}}
  \proofstep{1,2}{(F^{-1}(y),x)\in F}{\FormulaRefAuto{(x,y)\in F\eqvdash (y,x)\in F^{-1}}{4}}
  \proofstep{1,3}{(F^{-1}(x),y)\in F}{\FormulaRefAuto{(x,y)\in F\eqvdash (y,x)\in F^{-1}}{5}}
  \proofstep{1,3}{(F^{-1}(y),y)\in F}{\rIE{1,7}}
  \proofstep{1,2,3}{x=y}{\FormulaRefAuto{(x,y)\in F\dsep (x,z)\in F \vdash y=z}{6,8}}
\end{tabproof}

\FormulaThmDelta[Surjektivität von \(F^{-1}\)]%
{x\in A\vdash \exists y\in B\; F^{-1}(y)=x}%
{
\DeltaRow{Mengen}{x\dsep y\dsep A\dsep B}
\DeltaRow{Bijektive Funktionen}{F\colon A\to B}
}
\begin{tabproof}
  \proofstep{1}{x\in A}{\rA}
  \proofstep{1}{\exists y\,(x,y)\in F}{\FormulaRefAuto{x\in A \vdash \exists y\,(x,y)\in F}{1}}
  \proofstep{3}{(x,y)\in F}{\rA}
  \proofstep{3}{(y,x)\in F^{-1}}{\FormulaRefAuto{(x,y)\in F\eqvdash (y,x)\in F^{-1}}{3}}
  \proofstep{3}{F^{-1}(y)=x}{\FormulaRefAuto{(x,y)\in F\vdash F(x)=y}{4}}
  \proofstep{3}{y\in B}{\FormulaRefAuto{(x,y)\in F\vdash x\in A}{4}}
  \proofstep{3}{\exists y\in B\, F^{-1}(y)=x}{\rEI{\rAI{6,5}}}
  \proofstep{1}{\exists y\in B\, F^{-1}(y)=x}{\rEE{2,3,7}}
\end{tabproof}

\FormulaThmDelta[\(F^{-1}\) als bijektive Funktion]%
{F^{-1}\colon B\bij A}%
{
\DeltaRow{Mengen}{A\dsep B}
\DeltaRow{Bijektive Funktionen}{F\colon A\bij B}
}
\begin{tabproof}
  \proofstep{}{F^{-1}\colon B\to A}{\FormulaRefAuto{F^{-1}\colon B\to A}}
  \proofstep{}{\forall x,y\in B\, (F^{-1}(x)=F^{-1}(y)\rightarrow x=y)}{\FormulaRefAuto{x\in B\dsep y\in B\dsep F^{-1}(x)=F^{-1}(y)\vdash x=y}}
  \proofstep{}{\forall x\in A\exists y\in B\, F^{-1}(y)=x}{\FormulaRefAuto{x\in A\vdash \exists y\in B\; F^{-1}(y)=x}}
  \proofstep{}{F^{-1}\colon A\bij B}{\FormulaRefAuto{Bijektive Funktion}{1,2,3}}
\end{tabproof}

\subsubsection{Weitere Eigenschaften}

\FormulaThmDelta
{x\in A\dsep F(x)=y\vdash F^{-1}(y)=x}%
{
\DeltaRow{Mengen}{x\dsep y\dsep A\dsep B}
\DeltaRow{Bijektive Funktionen}{F\colon A\to B}
}
\begin{tabproof}
    \proofstep{1}{x\in A}{\rA}
    \proofstep{2}{F(x)=y}{\rA}
    \proofstep{1,2}{(x,y)\in F}{\FormulaRefAuto{x\in A\dsep F(x)=y\vdash (x,y)\in F}{1,2}}
    \proofstep{1,2}{(y,x)\in F^{-1}}{\FormulaRefAuto{(x,y)\in F\eqvdash (y,x)\in F^{-1}}{3}}
    \proofstep{1,2}{F^{-1}(y)=x}{\FormulaRefAuto{(x,y)\in F\vdash F(x)=y}{4}}
\end{tabproof}

\FormulaThmDelta{%
y\in B \dsep F^{-1}(y)=x \vdash F(x)=y
}{
  \DeltaRow{Mengen}{x \dsep y \dsep A \dsep B}
  \DeltaRow{Bijektive Funktionen}{F\colon A\to B}
}
\begin{tabproof}
  \proofstep{1}{y\in B}{\rA}
  \proofstep{2}{F^{-1}(y)=x}{\rA}

  % Aus Funktionswert folgt geordnetes Paar in der (Umkehr-)Relation
  \proofstep{1,2}{(y,x)\in F^{-1}}{\FormulaRefAuto{x\in A\dsep F(x)=y\vdash (x,y)\in F}{1,2}}

  % Inversen-Charakterisierung
  \proofstep{1,2}{(x,y)\in F}{\FormulaRefAuto{(x,y)\in F\eqvdash (y,x)\in F^{-1}}{3}}

  % Aus Paar in F folgt Funktionswert
  \proofstep{1,2}{F(x)=y}{\FormulaRefAuto{(x,y)\in F\vdash F(x)=y}{4}}
\end{tabproof}

\FormulaThmDelta{%
y\in B \vdash F\bigl(F^{-1}(y)\bigr)=y
}{
  \DeltaRow{Mengen}{x \dsep y \dsep A \dsep B}
  \DeltaRow{Bijektive Funktionen}{F\colon A\to B}
}
\begin{tabproof}
  \proofstep{1}{y\in B}{\rA}
  \proofstep{}{F^{-1}(y)=F^{-1}(y)}{\rII}

  % Aus Funktionswert folgt geordnetes Paar in der (Umkehr-)Relation
  \proofstep{1}{(y,F^{-1}(y))\in F^{-1}}{%
    \FormulaRefAuto{x\in A \dsep R(x)=y \vdash (x,y)\in R}{1,2}
  }

  % Inversen-Charakterisierung
  \proofstep{}{(F^{-1}(y),y)\in F}{%
    \FormulaRefAuto{(x,y)\in F \eqvdash (y,x)\in F^{-1}}{3}
  }

  % Aus Paar in F folgt Funktionswert
  \proofstep{}{F(F^{-1}(y))=y}{%
    \FormulaRefAuto{(x,y)\in F \vdash F(x)=y}{4}
  }
\end{tabproof}

\FormulaThmDelta{%
x\in A \vdash F^{-1}\bigl(F(x)\bigr)=x
}{
  \DeltaRow{Mengen}{x \dsep y \dsep A \dsep B}
  \DeltaRow{Bijektive Funktionen}{F\colon A\to B}
}
\begin{tabproof}
  \proofstep{1}{x\in A}{\rA}
  \proofstep{}{F(x)=F(x)}{\rII}

  % Aus Funktionswert folgt (x, F(x)) in F
  \proofstep{1,2}{(x,F(x))\in F}{%
    \FormulaRefAuto{x\in A \dsep R(x)=y \vdash (x,y)\in R}{1,2}
  }

  % Charakterisierung der Inversen
  \proofstep{}{(F(x),x)\in F^{-1}}{%
    \FormulaRefAuto{(x,y)\in F \eqvdash (y,x)\in F^{-1}}{3}
  }

  % Aus Paar in F^{-1} folgt Funktionswert von F^{-1}
  \proofstep{}{F^{-1}(F(x))=x}{%
    \FormulaRefAuto{(x,y)\in F \vdash F(x)=y}{4}
  }
\end{tabproof}

\FormulaThmDelta{%
\forall y\in B\,\bigl(F(G(y))=y\bigr) \vdash G = F^{-1}
}{
  \DeltaRow{Mengen}{x \dsep y \dsep A \dsep B}
  \DeltaRow{Bijektive Funktionen}{F\colon A\to B}
  \DeltaRow{Funktionen}{G\colon B\to A}
}
\begin{tabproof}
  \proofstep{1}{\forall y\in B\,\bigl(F(G(y))=y\bigr)}{\rA}
  \proofstep{2}{y\in B}{\rA}
  \proofstep{1,2}{F(G(y))=y}{\rRE{\rUE{1},2}}
  \proofstep{1,2}{F(F^{-1}(y))=y}{\FormulaRefAuto{y\in B \vdash F\bigl(F^{-1}(y)\bigr)=y}{2}}
  \proofstep{2}{G(y)\in A}{\FormulaRefAuto{x\in A\vdash F(x)\in B}{2}}
  \proofstep{2}{F^{-1}(y)\in A}{\FormulaRefAuto{x\in A\vdash F(x)\in B}{2}}
  \proofstep{1,2}{G(y)=F^{-1}(y)}{\FormulaRefAuto{x\in A\dsep y\in A\dsep F(x)=F(y)\vdash x=y}{5,6,3,4}}
  \proofstep{1}{\forall y\,\Bigl(G(y)=F^{-1}(y)\Bigr)}{\rUI{\rRI{2,7}}}
  \proofstep{1}{G=F^{-1}}{\FormulaRefAuto{\forall x\in A (F(x)=G(x)) \eqvdash F=G}{8}}
\end{tabproof}

\FormulaThmDelta{%
\forall x\in A\,\bigl(F^{-1}(G(x))=x\bigr) \vdash G = F
}{
  \DeltaRow{Mengen}{x \dsep A \dsep B}
  \DeltaRow{Bijektive Funktionen}{F\colon A\to B}
  \DeltaRow{Funktionen}{G\colon A\to B}
}
\begin{tabproof}
  \proofstep{1}{\forall x\in A\,\bigl(F^{-1}(G(x))=x\bigr)}{\rA}
  \proofstep{2}{x\in A}{\rA}
  \proofstep{1,2}{F^{-1}(G(x))=x}{\rRE{\rUE{1},2}}
  \proofstep{2}{F^{-1}\bigl(F(x)\bigr)=x}{\FormulaRefAuto{x\in A \vdash F^{-1}\bigl(F(x)\bigr)=x}{2}}
  \proofstep{2}{G(x)\in B}{\FormulaRefAuto{x\in A \vdash F(x)\in B}{2}}
  \proofstep{2}{F(x)\in B}{\FormulaRefAuto{x\in A \vdash F(x)\in B}{2}}
  \proofstep{1,2}{G(y)=F(y)}{\FormulaRefAuto{x\in A\dsep y\in A\dsep F(x)=F(y)\vdash x=y}{5,6,3,4}}
  \proofstep{1}{\forall y\,\Bigl(G(y)=F(y)\Bigr)}{\rUI{\rRI{2,7}}}
  \proofstep{1}{G=F}{\FormulaRefAuto{\forall x\in A (F(x)=G(x)) \eqvdash F=G}{8}}
\end{tabproof}


\subsection{Identität}

\subsubsection{Definition der Identität}
% — Definition —
\FormulaDefDeltaK[Identitätsrelation auf \(A\)]{%
\Id_A := \{\, (x,y) \in A \times A \mid x = y \,\}
}{Identität}{
  \DeltaRow{Mengen}{x \dsep y \dsep A}
}

\subsubsection{Die Funktionseigenschaften}

% — Grundlegende Eigenschaften der Menge —
\FormulaThmDelta[Existenz als Menge geordneter Paare]{%
\Id_A \subseteq A \times A
}{
  \DeltaRow{Mengen}{x \dsep y \dsep A}
}
\begin{tabproof}
  \proofstep{}{%
    \Id_A = \{\, (x,y) \in A \times A \mid x = y \,\}
  }{\FormulaRefAuto{\Id_A := \{\, (x,y) \in A \times A \mid x = y \,\}}}
  \proofstep{}{%
    \{\, (x,y) \in A \times A \mid x = y \,\} \subseteq A \times A
  }{\FormulaRefAuto{\{ x \in A \mid P(x) \} \subseteq A}}
  \proofstep{}{%
    \Id_A \subseteq A \times A
  }{\rIE{1,2}}
\end{tabproof}

\FormulaThmDelta{%
(x,y)\in \Id_A \vdash x\in A
}{
  \DeltaRow{Mengen}{x \dsep y \dsep A}
}
\begin{tabproof}
  \proofstep{}{%
    (x,y)\in \Id_A
  }{\rA}
  \proofstep{1}{(x,y)\in \{\, (x,y) \in A \times A \mid x = y \,\}}{\rIE{\FormulaRefAuto{\Id_A := \{\, (x,y) \in A \times A \mid x = y \,\}},1}}
  \proofstep{1}{(x,y)\in A\times A}{\FormulaRefAuto{x \in \{x \in A \mid P(x)\}\vdash x\in A}{2}}
  \proofstep{1}{x\in A}{\FormulaRefAuto{(a,b)\in A\times B\vdash a\in A}{3}}
\end{tabproof}

\FormulaThmDelta{%
(x,y)\in \Id_A \vdash y\in A
}{
  \DeltaRow{Mengen}{x \dsep y \dsep A}
}
\begin{tabproof}
  \proofstep{}{%
    (x,y)\in \Id_A
  }{\rA}
  \proofstep{1}{(x,y)\in \{\, (x,y) \in A \times A \mid x = y \,\}}{\rIE{\FormulaRefAuto{\Id_A := \{\, (x,y) \in A \times A \mid x = y \,\}},1}}
  \proofstep{1}{(x,y)\in A\times A}{\FormulaRefAuto{x \in \{x \in A \mid P(x)\}\vdash x\in A}{2}}
  \proofstep{1}{x\in A}{\FormulaRefAuto{(a,b)\in A\times B\vdash b\in B}{3}}
\end{tabproof}


% — Charakterisierung der Elemente —
\FormulaThmDelta{%
(x,y)\in \Id_A \vdash x = y
}{
  \DeltaRow{Mengen}{x \dsep y \dsep A}
}
\begin{tabproof}
    \proofstep{1}{(x,y)\in \Id_A}{\rA}
    \proofstep{1}{x=y}{\FormulaRefAuto{x \in \{x \in A \mid P(x)\}\vdash P(x)}{1}}
\end{tabproof}

\FormulaThmDelta{%
(x,y)\in \Id_A \vdash y = x
}{
  \DeltaRow{Mengen}{x \dsep y \dsep A}
}
\begin{tabproof}
    \proofstep{1}{(x,y)\in \Id_A}{\rA}
    \proofstep{1}{x=y}{\FormulaRefAuto{(x,y)\in \Id_A \vdash x = y}{1}}
    \proofstep{1}{y=x}{\FormulaRefAuto{a = b \vdash b = a}{1}}
\end{tabproof}

% — Charakterisierung der Elemente —
\FormulaThmDelta{%
x\in A\dsep x=y \vdash (x,y)\in \Id_A
}{
  \DeltaRow{Mengen}{x \dsep y \dsep A}
}
\begin{tabproof}
    \proofstep{1}{x\in A}{\rA}
    \proofstep{1}{x=y}{\rA}
    \proofstep{1}{(x,y)\in A\times A}{\FormulaRefAuto{a\in A\dsep b\in B\vdash (a,b)\in A\times B}{1,1}}
    \proofstep{1}{(x,y)\in  \{\, (x,y) \in A \times A \mid x = y \,\}}{\FormulaRefAuto{x \in A, P(x)\vdash x \in \{x \in A \mid P(x)\}}{3,2}}
    \proofstep{1}{(x,y)\in  \Id_A}{\rIE{\FormulaRefAuto{%
\Id_A := \{\, (x,y) \in A \times A \mid x = y \,\}
},4}}
\end{tabproof}

\FormulaThmDelta{%
 (x,y)\in \Id_A\eqvdash x\in A\land x=y 
}{
  \DeltaRow{Mengen}{x \dsep y \dsep A}
}
\begin{tabproofsplit}
  \proofpart{\(\vdash\)}
    \proofstep{1}{(x,y)\in \Id_A}{\rA}
    \proofstep{1}{x\in A}{\FormulaRefAuto{(x,y)\in \Id_A \vdash x\in A}{1}}
    \proofstep{1}{x=y}{\FormulaRefAuto{(x,y)\in \Id_A \vdash x = y}{1}}
    \proofstep{1}{x\in A\land x=y}{\rAI{2,3}}
  \closeproofpart
    \proofpart{\(\dashv\)}
    \proofstep{1}{x\in A\land x=y}{\rA}
    \proofstep{1}{x\in A}{\rAEa{1}}
    \proofstep{1}{x=y}{\rAEb{1}}
    \proofstep{1}{(x,y)\in \Id_A}{\FormulaRefAuto{x\in A\dsep x=y \vdash (x,y)\in \Id_A}{2,3}}
  \closeproofpart
\end{tabproofsplit}

\FormulaThmDelta{%
 x\in A\land x=y\eqvdash y\in A\land y=x 
}{
  \DeltaRow{Mengen}{x \dsep y \dsep A}
}
\begin{tabproofsplit}
    \proofpart{\(\vdash\)}
    \proofstep{1}{x\in A\land x=y}{\rA}
    \proofstep{1}{x\in A}{\rAEa{1}}
    \proofstep{1}{x=y}{\rAEb{1}}
    \proofstep{1}{y\in A}{\rIE{3,2}}
    \proofstep{1}{y=x}{\FormulaRefAuto{a = b \vdash b = a}{3}}
    \proofstep{1}{y\in A\land y=x}{\rAI{4,5}}
  \closeproofpart
    \proofpart{\(\dashv\)}
    \proofstep{1}{y\in A\land y=x}{\rA}
    \proofstep{1}{y\in A}{\rAEa{1}}
    \proofstep{1}{y=x}{\rAEb{1}}
    \proofstep{1}{x\in A}{\rIE{3,2}}
    \proofstep{1}{x=y}{\FormulaRefAuto{a = b \vdash b = a}{3}}
    \proofstep{1}{x\in A\land x=y}{\rAI{4,5}}
  \closeproofpart
\end{tabproofsplit}

\FormulaThmDelta{%
 (x,y)\in \Id_A\eqvdash (y,x)\in \Id_A
}{
  \DeltaRow{Mengen}{x \dsep y \dsep A}
}
\begin{tabproofwide}
  \proofstepwide{(x,y)\in \Id_A}{\leftrightarrow}{x\in A\land x=y}%
    {\FormulaRefAuto{(x,y)\in \Id_A\eqvdash x\in A\land x=y}}
  \proofstepwide{}{\leftrightarrow}{y\in A\land y=x}%
    {\FormulaRefAuto{x\in A\land x=y\eqvdash y\in A\land y=x}}
   \proofstepwide{}{\leftrightarrow}{(y,x)\in \Id_A}%
    {\FormulaRefAuto{(x,y)\in \Id_A\eqvdash x\in A\land x=y}}   
   \proofstepwide{(x,y)\in \Id_A}{\leftrightarrow}{(y,x)\in \Id_A}%
    {\rChain{1,3}}  
\end{tabproofwide}


\FormulaThmDelta{%
x\in A \vdash (x,x)\in \Id_A
}{
  \DeltaRow{Mengen}{x \dsep A}
}
\begin{tabproof}
    \proofstep{1}{x\in A}{\rA}
    \proofstep{1}{x=x}{\rII}
    \proofstep{1}{(x,x)\in \Id_A}{\FormulaRefAuto{x\in A\dsep x=y \vdash (x,y)\in \Id_A}{1,2}}
\end{tabproof}

% — Totalität (Zeugen durch Gleichheit) —
\FormulaThmDelta[Totalität auf \(A\)]{%
x\in A \vdash \exists y\, (x,y)\in \Id_A
}{
  \DeltaRow{Mengen}{x \dsep y \dsep A}
}
\begin{tabproof}
  \proofstep{1}{x\in A}{\rA}
  \proofstep{1}{(x,x)\in \Id_A}{\FormulaRefAuto{x\in A \vdash (x,x)\in \Id_A}{1}}
  \proofstep{1}{\exists y\, (x,y)\in \Id_A}{\rEI{2}}
\end{tabproof}

% — Funktionale Eindeutigkeit —
\FormulaThmDelta[Funktionale Eindeutigkeit der \(\Id_A\)]{%
(x,y_1)\in \Id_A \dsep (x,y_2)\in \Id_A \vdash y_1 = y_2
}{
  \DeltaRow{Mengen}{x \dsep y_1 \dsep y_2 \dsep A}
}
\begin{tabproof}
  \proofstep{1}{(x,y_1)\in \Id_A}{\rA}
  \proofstep{2}{(x,y_2)\in \Id_A}{\rA}
  \proofstep{1}{x=y_1}{\FormulaRefAuto{(x,y)\in \Id_A \vdash x = y}{1}}
  \proofstep{2}{x=y_2}{\FormulaRefAuto{(x,y)\in \Id_A \vdash x = y}{2}}
  \proofstep{1,2}{y_1=y_2}{\FormulaRefAuto{a=b, a=c \vdash b=c}{3,4}}
\end{tabproof}

\FormulaThmDelta[\(\Id_A\) als Funktion]{%
\Id_A\colon A\to A
}{
  \DeltaRow{Mengen}{A}
}
\begin{tabproof}
    \proofstep{}{\Id_A \subseteq A \times A}{\FormulaRefAuto{\Id_A \subseteq A\times A}}
    \proofstep{}{\forall (x,y_1)\dsep (x,y_2)\in \Id_A y_1=y_2}{\FormulaRefAuto{(x,y_1)\in \Id_A \dsep (x,y_2)\in \Id_A \vdash y_1=y_2}}
    \proofstep{}{\forall x\in A\exists y\,(x,y)\in \Id_A}{\FormulaRefAuto{x\in A \vdash \exists y\, (x,y)\in \Id_A}}
    \proofstep{}{\Id_A\colon A\to A}{\FormulaRefAuto{Funktion}{1,2,3}}
\end{tabproof}

\subsubsection{Die Bijektionseigenschaften}


% — Injektivität und Surjektivität —
\FormulaThmDelta[Injektivität von \(\Id_A\)]{%
x\in A \dsep y\in A \dsep \Id_A(x)=\Id_A(y) \vdash x=y
}{
  \DeltaRow{Mengen}{x \dsep y \dsep A}
}
\begin{tabproof}
  \proofstep{1}{\Id_A(x)=\Id_A(y)}{\rA}
  \proofstep{2}{x\in A}{\rA}
  \proofstep{3}{y\in A}{\rA}
  \proofstep{2}{(x,\Id_A(x))\in \Id_A}{\FormulaRefAuto{x\in A \vdash (x,x)\in \Id_A}{2}}
  \proofstep{3}{(y,\Id_A(y))\in \Id_A}{\FormulaRefAuto{x\in A \vdash (x,x)\in \Id_A}{3}}
  \proofstep{2}{x=\Id_A(x)}{\FormulaRefAuto{(x,y)\in \Id_A \vdash x = y}{4}}
  \proofstep{3}{y=\Id_A(y)}{\FormulaRefAuto{(x,y)\in \Id_A \vdash x = y}{5}}
  \proofstep{1,2,3}{x=y}{\FormulaRefAuto{a = b\dsep c = a\dsep d = b \vdash c = d}{1,6,7}}
\end{tabproof}

\FormulaThmDelta[Surjektivität von \(\Id_A\)]{%
x\in A \vdash \exists y\in A\; \Id_A(y)=x
}{
  \DeltaRow{Mengen}{x \dsep y \dsep A}
}
\begin{tabproof}
  \proofstep{1}{x\in A}{\rA}
  \proofstep{1}{(x,x)\in \Id_A}{\FormulaRefAuto{x\in A \vdash (x,x)\in \Id_A}{1}}
  \proofstep{1}{\Id_A(x)=x}{\FormulaRefAuto{(x,y)\in F\vdash F(x)=y}{2}}
  \proofstep{1}{\exists y\in A\; \Id_A(y)=x}{\rEI{\rAI{1,3}}}
\end{tabproof}

\FormulaThmDelta[\(\Id_A\) als bijektive Funktion]{%
\Id_A\colon A\bij A
}{
  \DeltaRow{Mengen}{A}
}
\begin{tabproof}
    \proofstep{}{\Id_A\colon A\to A}{\FormulaRefAuto{\Id_A\colon A\to A}}
    \proofstep{}{\forall x,y\in A\,(\Id_A(x)=\Id_A(y) \rightarrow x=y)}{\FormulaRefAuto{x\in A \dsep y\in A \dsep \Id_A(x)=\Id_A(y) \vdash x=y}}
    \proofstep{}{\forall x\in A\exists y\in A\,\Id_A(y)=x}{\FormulaRefAuto{x\in A \vdash \exists y\in A\; \Id_A(y)=x}}
    \proofstep{}{\Id_A\colon A\bij A}{\FormulaRefAuto{Bijektive Funktion}{1,2,3}}
\end{tabproof}

\subsubsection{Weitere Eigenschaften}

\FormulaThmDelta{%
(x,y)\in \Id_A^{-1}\eqvdash (x,y)\in \Id_A
}{
  \DeltaRow{Mengen}{x \dsep y \dsep A}
}
\begin{tabproofwide}
  \proofstepwide{(x,y)\in \Id_A^{-1}}{\leftrightarrow}{(y,x)\in \Id_A}%
    {\FormulaRefAuto{(x,y)\in F\eqvdash (y,x)\in F^{-1}}}
  \proofstepwide{}{\leftrightarrow}{(x,y)\in \Id_A}%
    {\FormulaRefAuto{(x,y)\in \Id_A\eqvdash (y,x)\in \Id_A}} 
  \proofstepwide{(x,y)\in \Id_A^{-1}}{\leftrightarrow}{(x,y)\in \Id_A}%
    {\rChain{1,2}} 
\end{tabproofwide}

% — Eigeninvers-Eigenschaft —
\FormulaThmDelta[Eigeninversität von \(\Id_A\)]{%
\Id_A^{-1}=\Id_A
}{
  \DeltaRow{Mengen}{x \dsep y \dsep A}
}
\begin{tabproofwide}
  \proofstepwide{(x,y)\in \Id_A^{-1}}{\leftrightarrow}{(x,y)\in \Id_A}{\FormulaRefAuto{(x,y)\in \Id_A^{-1}\eqvdash (x,y)\in \Id_A}}
  \proofstepwide{\Id_A^{-1}}{=}{\Id_A}{\FormulaRefAuto{\forall x\, (x \in A \leftrightarrow x \in B) \eqvdash A = B}{\rUI{1}}}
\end{tabproofwide}

% — Fixpunkteigenschaft der Identität —
\FormulaThmDelta[Fixpunkteigenschaft]{%
x\in A \vdash \Id_A(x)=x
}{
  \DeltaRow{Mengen}{x \dsep A}
}
\begin{tabproof}
  \proofstep{1}{x\in A}{\rA}
  \proofstep{1}{(x,x)\in \Id_A}{\FormulaRefAuto{x\in A \vdash (x,x)\in \Id_A}{1}}
  \proofstep{1}{\Id_A(x)=x}{\FormulaRefAuto{(x,y)\in F \vdash F(x)=y}{2}}
\end{tabproof}

\subsection{Inklusionsabbildung}

\subsubsection{Definition der Inklusionsabbildung}
% — Definition —
\FormulaDefDeltaK[Inklusionsabbildung von \(C\) in \(A\)]{%
C\subseteq A \vdash \iota_{C,A} := \{\, (x,y) \in C \times A \mid x = y \,\}
}{Inklusionsabbildung}{
  \DeltaRow{Mengen}{x \dsep y \dsep A \dsep C}
  \DeltaRow{Teilmengen}{C\subseteq A}
}

\subsubsection{Die Funktionseigenschaften}

% — Grundlegende Eigenschaft der Menge —
\FormulaThmDelta[Existenz als Menge geordneter Paare]{%
C\subseteq A \vdash \iota_{C,A} \subseteq C \times A
}{
  \DeltaRow{Mengen}{x \dsep y \dsep A \dsep C}
  \DeltaRow{Teilmengen}{C\subseteq A}
}
\begin{tabproof}
  \proofstep{1}{C\subseteq A}{\rA}
  \proofstep{1}{%
    \iota_{C,A} = \{\, (x,y) \in C \times A \mid x = y \,\}
  }{\FormulaRefAuto{C\subseteq A \vdash \iota_{C,A} := \{\, (x,y) \in C \times A \mid x = y \,\}}{1}}
  \proofstep{}{%
    \{\, (x,y) \in C \times A \mid x = y \,\} \subseteq C \times A
  }{\FormulaRefAuto{\{ x \in A \mid P(x) \} \subseteq A}}
  \proofstep{1}{%
    \iota_{C,A} \subseteq C \times A
  }{\rIE{2,3}}
\end{tabproof}

% — Projektion auf die erste Komponente: x liegt in C —
\FormulaThmDelta{%
(x,y)\in \iota_{C,A} \vdash x\in C
}{
  \DeltaRow{Mengen}{x \dsep y \dsep A \dsep C}
  \DeltaRow{Teilmengen}{C\subseteq A}
}
\begin{tabproof}
  \proofstep{1}{(x,y)\in \iota_{C,A}}{\rA}
  \proofstep{2}{C\subseteq A}{\rA}
  \proofstep{1,2}{(x,y)\in \{\, (x,y) \in C \times A \mid x = y \,\}}{%
    \rIE{\FormulaRefAuto{C\subseteq A \vdash \iota_{C,A} := \{\, (x,y) \in C \times A \mid x = y \,\}}{2},1}
  }
  \proofstep{1,2}{(x,y)\in C\times A}{%
    \FormulaRefAuto{x \in \{x \in A \mid P(x)\}\vdash x\in A}{3}
  }
  \proofstep{1,2}{x\in C}{%
    \FormulaRefAuto{(a,b)\in A\times B\vdash a\in A}{4}
  }
\end{tabproof}

% — Projektion auf die zweite Komponente: y liegt in A —
\FormulaThmDelta{%
(x,y)\in \iota_{C,A} \vdash y\in A
}{
  \DeltaRow{Mengen}{x \dsep y \dsep A \dsep C}
  \DeltaRow{Teilmengen}{C\subseteq A}
}
\begin{tabproof}
  \proofstep{1}{(x,y)\in \iota_{C,A}}{\rA}
  \proofstep{2}{C\subseteq A}{\rA}
  \proofstep{1,2}{(x,y)\in \{\, (x,y) \in C \times A \mid x = y \,\}}{%
    \rIE{\FormulaRefAuto{C\subseteq A \vdash \iota_{C,A} := \{\, (x,y) \in C \times A \mid x = y \,\}}{2},1}
  }
  \proofstep{1,2}{(x,y)\in C\times A}{%
    \FormulaRefAuto{x \in \{x \in A \mid P(x)\}\vdash x\in A}{3}
  }
  \proofstep{1,2}{y\in A}{%
    \FormulaRefAuto{(a,b)\in A\times B\vdash b\in B}{4}
  }
\end{tabproof}

% — Charakterisierung der Elemente: (x,y) in iota_{C,A} -> x=y —
\FormulaThmDelta{%
(x,y)\in \iota_{C,A} \vdash x = y
}{
  \DeltaRow{Mengen}{x \dsep y \dsep A \dsep C}
  \DeltaRow{Teilmengen}{C\subseteq A}
}
\begin{tabproof}
  \proofstep{1}{(x,y)\in \iota_{C,A}}{\rA}
  \proofstep{2}{C\subseteq A}{\rA}
  \proofstep{1,2}{(x,y)\in \{\, (x,y) \in C \times A \mid x = y \,\}}{%
    \rIE{\FormulaRefAuto{C\subseteq A \vdash \iota_{C,A} := \{\, (x,y) \in C \times A \mid x = y \,\}}{2},1}
  }
  \proofstep{1,2}{x=y}{%
    \FormulaRefAuto{x \in \{x \in A \mid P(x)\}\vdash P(x)}{3}
  }
\end{tabproof}

% — Umkehrung: aus x\in C und y=x folgt (x,y)\in iota_{C,A} —
\FormulaThmDelta{%
C\subseteq A \dsep x\in C \dsep y=x \vdash (x,y)\in \iota_{C,A}
}{
  \DeltaRow{Mengen}{x \dsep y \dsep A \dsep C}
  \DeltaRow{Teilmengen}{C\subseteq A}
}
\begin{tabproof}
  \proofstep{1}{C\subseteq A}{\rA}
  \proofstep{2}{x\in C}{\rA}
  \proofstep{3}{y=x}{\rA}
  \proofstep{1,2}{x\in A}{%
    \FormulaRefAuto{A\subseteq B,x\in A\vdash x\in B}{1,2}
  }
  \proofstep{1,2,3}{y\in A}{\rIE{3,4}}
  \proofstep{2,5}{(x,y)\in C\times A}{%
    \FormulaRefAuto{a\in A\dsep b\in B\vdash (a,b)\in A\times B}{2,5}
  }
  \proofstep{2,3,6}{(x,y)\in \{\, (x,y) \in C \times A \mid x = y \,\}}{%
    \FormulaRefAuto{x\in A, P(x)\vdash x\in\{x\in A\mid P(x)\}}{6,3}
  }
  \proofstep{1,2,3}{(x,y)\in \iota_{C,A}}{%
    \rIE{\FormulaRefAuto{C\subseteq A \vdash \iota_{C,A} := \{\, (x,y) \in C \times A \mid x = y \,\}}{1},7}
  }
\end{tabproof}

% — Äquivalente Charakterisierung der Elemente —
\FormulaThmDelta{%
(x,y)\in \iota_{C,A}\eqvdash x\in C\land y=x
}{
  \DeltaRow{Mengen}{x \dsep y \dsep A \dsep C}
  \DeltaRow{Teilmengen}{C\subseteq A}
}
\begin{tabproofsplit}
  \proofpart{\(\vdash\)}
    \proofstep{1}{(x,y)\in \iota_{C,A}}{\rA}
    \proofstep{2}{C\subseteq A}{\rA}
    \proofstep{1,2}{x\in C}{\FormulaRefAuto{(x,y)\in \iota_{C,A} \vdash x\in C}{1}}
    \proofstep{1,2}{x=y}{\FormulaRefAuto{(x,y)\in \iota_{C,A} \vdash x=y}{1}}
    \proofstep{1,2}{y=x}{\FormulaRefAuto{a=b\vdash b=a}{4}}
    \proofstep{1,2}{x\in C\land y=x}{\rAI{3,5}}
  \closeproofpart
  \proofpart{\(\dashv\)}
    \proofstep{1}{x\in C\land y=x}{\rA}
    \proofstep{1}{x\in C}{\rAEa{1}}
    \proofstep{1}{y=x}{\rAEb{1}}
    \proofstep{1}{x=y}{\FormulaRefAuto{a=b\vdash b=a}{3}}
    \proofstep{1}{(x,y)\in \iota_{C,A}}{%
      \FormulaRefAuto{C\subseteq A \dsep x\in C \dsep y=x \vdash (x,y)\in \iota_{C,A}}{1,2,4}
    }
  \closeproofpart
\end{tabproofsplit}

% — Diagonalelemente —
\FormulaThmDelta{%
x\in C \vdash (x,x)\in \iota_{C,A}
}{
  \DeltaRow{Mengen}{x \dsep A \dsep C}
  \DeltaRow{Teilmengen}{C\subseteq A}
}
\begin{tabproof}
  \proofstep{1}{C\subseteq A}{\rA}
  \proofstep{2}{x\in C}{\rA}
  \proofstep{2}{x=x}{\rII}
  \proofstep{1,2}{(x,x)\in \iota_{C,A}}{%
    \FormulaRefAuto{C\subseteq A \dsep x\in C \dsep y=x \vdash (x,y)\in \iota_{C,A}}{1,2,3}
  }
\end{tabproof}

% — Totalität auf C —
\FormulaThmDelta[Totalität auf \(C\)]{%
x\in C \vdash \exists y\, (x,y)\in \iota_{C,A}
}{
  \DeltaRow{Mengen}{x \dsep y \dsep A \dsep C}
  \DeltaRow{Teilmengen}{C\subseteq A}
}
\begin{tabproof}
  \proofstep{1}{x\in C}{\rA}
  \proofstep{1}{(x,x)\in \iota_{C,A}}{\FormulaRefAuto{x\in C \vdash (x,x)\in \iota_{C,A}}{1}}
  \proofstep{1}{\exists y\,(x,y)\in \iota_{C,A}}{\rEI{2}}
\end{tabproof}

% — Funktionale Eindeutigkeit —
\FormulaThmDelta[Funktionale Eindeutigkeit von \(\iota_{C,A}\)]{%
(x,y_1)\in \iota_{C,A} \dsep (x,y_2)\in \iota_{C,A} \vdash y_1 = y_2
}{
  \DeltaRow{Mengen}{x \dsep y_1 \dsep y_2 \dsep A \dsep C}
  \DeltaRow{Teilmengen}{C\subseteq A}
}
\begin{tabproof}
  \proofstep{1}{(x,y_1)\in \iota_{C,A}}{\rA}
  \proofstep{2}{(x,y_2)\in \iota_{C,A}}{\rA}
  \proofstep{1}{x=y_1}{\FormulaRefAuto{(x,y)\in \iota_{C,A} \vdash x=y}{1}}
  \proofstep{2}{x=y_2}{\FormulaRefAuto{(x,y)\in \iota_{C,A} \vdash x=y}{2}}
  \proofstep{1,2}{y_1=y_2}{\FormulaRefAuto{a=b, a=c \vdash b=c}{3,4}}
\end{tabproof}

% — \iota_{C,A} als Funktion C -> A —
\FormulaThmDelta[\(\iota_{C,A}\) als Funktion]{%
\iota_{C,A}\colon C\to A
}{
  \DeltaRow{Mengen}{A \dsep C}
  \DeltaRow{Teilmengen}{C\subseteq A}
}
\begin{tabproof}
  \proofstep{}{%
    \iota_{C,A} \subseteq C \times A
  }{\FormulaRefAuto{C\subseteq A \vdash \iota_{C,A} \subseteq C \times A}}
  \proofstep{}{%
    \forall (x,y_1)\dsep (x,y_2)\in \iota_{C,A} \; y_1=y_2
  }{\FormulaRefAuto{(x,y_1)\in \iota_{C,A} \dsep (x,y_2)\in \iota_{C,A} \vdash y_1=y_2}}
  \proofstep{}{%
    \forall x\in C\,\exists y\, (x,y)\in \iota_{C,A}
  }{\FormulaRefAuto{x\in C \vdash \exists y\, (x,y)\in \iota_{C,A}}}
  \proofstep{}{\iota_{C,A}\colon C\to A}{\FormulaRefAuto{Funktion}{1,2,3}}
\end{tabproof}

% — Funktionswert der Inklusionsabbildung (Fixpunkteigenschaft) —
\FormulaThmDelta[Funktionswert der Inklusionsabbildung]{%
C\subseteq A \dsep x\in C \vdash \iota_{C,A}(x)=x
}{
  \DeltaRow{Mengen}{x \dsep A \dsep C}
  \DeltaRow{Teilmengen}{C\subseteq A}
  \DeltaRow{Funktionen}{\iota_{C,A}\colon C\to A}
}
\begin{tabproof}
  \proofstep{1}{C\subseteq A}{\rA}
  \proofstep{2}{x\in C}{\rA}
  \proofstep{2}{(x,x)\in \iota_{C,A}}{\FormulaRefAuto{x\in C \vdash (x,x)\in \iota_{C,A}}{2}}
  \proofstep{1,2}{\iota_{C,A}(x)=x}{%
    \FormulaRefAuto{(x,y)\in F \vdash F(x)=y}{3}
  }
\end{tabproof}

\subsubsection{Injektivitätseigenschaft}

\FormulaThmDelta[Injektivität von \(\iota_{C,A}\)]{%
x_1\in C \dsep x_2\in C \dsep \iota_{C,A}(x_1)=\iota_{C,A}(x_2) \vdash x_1=x_2
}{
  \DeltaRow{Mengen}{x_1 \dsep x_2 \dsep A \dsep C}
  \DeltaRow{Teilmengen}{C\subseteq A}
  \DeltaRow{Funktionen}{\iota_{C,A}\colon C\to A}
}
\begin{tabproof}
  \proofstep{1}{x_1\in C}{\rA}
  \proofstep{2}{x_2\in C}{\rA}
  \proofstep{3}{\iota_{C,A}(x_1)=\iota_{C,A}(x_2)}{\rA}
  \proofstep{1}{\iota_{C,A}(x_1)=x_1}{%
    \FormulaRefAuto{C\subseteq A \dsep x\in C \vdash \iota_{C,A}(x)=x}{1}
  }
  \proofstep{2}{\iota_{C,A}(x_2)=x_2}{%
    \FormulaRefAuto{C\subseteq A \dsep x\in C \vdash \iota_{C,A}(x)=x}{2}
  }
  \proofstep{1,2,3}{x_1=x_2}{%
    \FormulaRefAuto{a = b\dsep a = c\dsep b=d\vdash c = d}{3,4,5}
  }
\end{tabproof}


\subsection{Erste Projektion}

\subsubsection{Definition der Projektion}

% — Definition der ersten Projektion —
\FormulaDefDeltaK[Erste Projektion auf \(A\)]{%
  \pi_1 := \{\, ((x,y),z) \in (A \times B) \times A \mid (x,y)\in A\times B \land z = x \,\}
}{Projektion1}{
  \DeltaRow{Mengen}{x \dsep y \dsep z \dsep A \dsep B}
}

\subsubsection{Die Funktionseigenschaften}

% — Grundlegende Eigenschaft: Menge geordneter Paare —
\FormulaThmDelta[Existenz als Menge geordneter Paare]{%
  \pi_1 \subseteq (A \times B) \times A
}{
  \DeltaRow{Mengen}{x \dsep y \dsep z \dsep A \dsep B}
}
\begin{tabproof}
  \proofstep{}{%
    \pi_1 \subseteq (A\times B)\times A
  }{\FormulaRefAuto{A=\{ x \in B \mid P(x) \}\vdash  A\subseteq B}{\FormulaRefAuto{\pi_1 := \{\, ((x,y),z) \in (A \times B) \times A \mid (x,y)\in A\times B \land z = x \,\}}}}
\end{tabproof}

% — Elemente der Projektion liegen im Produktraum —
\FormulaThmDelta{%
  ((x,y),z)\in \pi_1 \vdash (x,y)\in A\times B
}{
  \DeltaRow{Mengen}{x \dsep y \dsep z \dsep A \dsep B}
}
\begin{tabproof}
  \proofstep{1}{((x,y),z)\in \pi_1}{\rA}
  \proofstep{1}{((x,y),z)\in (A\times B)\times A}{%
    \FormulaRefAuto{x\in \{x\in A\mid P(x)\}\vdash x\in A}{1}}
  \proofstep{1}{(x,y)\in A\times B}{%
    \FormulaRefAuto{(a,b)\in A\times B\vdash a\in A}{2}}
\end{tabproof}

\FormulaThmDelta{%
  ((x,y),z)\in \pi_1 \vdash z\in A
}{
  \DeltaRow{Mengen}{x \dsep y \dsep z \dsep A \dsep B}
}
\begin{tabproof}
  \proofstep{1}{((x,y),z)\in \pi_1}{\rA}
  \proofstep{1}{((x,y),z)\in (A\times B)\times A}{%
    \FormulaRefAuto{x\in \{x\in A\mid P(x)\}\vdash x\in A}{1}}
  \proofstep{1}{z\in A}{%
    \FormulaRefAuto{(a,b)\in A\times B\vdash b\in B}{2}}
\end{tabproof}

% — Charakterisierung: z = x —
\FormulaThmDelta{%
  ((x,y),z)\in \pi_1 \vdash z = x
}{
  \DeltaRow{Mengen}{x \dsep y \dsep z \dsep A \dsep B}
}
\begin{tabproof}
  \proofstep{1}{((x,y),z)\in \pi_1}{\rA}
  \proofstep{1}{(x,y)\in A\times B \land z = x}{%
    \FormulaRefAuto{x \in A\dsep A=\{x \in B \mid P(x)\}\vdash P(x)}{1,\FormulaRefAuto{\pi_1 := \{\, ((x,y),z) \in (A \times B) \times A \mid (x,y)\in A\times B \land z = x \,\}}}}
  \proofstep{1}{z=x}{\rAEb{3}}
\end{tabproof}

% — Einführungsrichtung: aus (x,y)\in A\times B folgt Element der Projektion —
\FormulaThmDelta{%
  (x,y)\in A\times B \vdash ((x,y),x)\in \pi_1
}{
  \DeltaRow{Mengen}{x \dsep y \dsep z \dsep A \dsep B}
}
\begin{tabproof}
  \proofstep{1}{(x,y)\in A\times B}{\rA}
  \proofstep{1}{x\in A}{\FormulaRefAuto{(a,b)\in A\times B\vdash a\in A}{1}}
  \proofstep{1}{(x,y)\in A\times B \land x=x}{\rAI{1,\rII}}
  \proofstep{1}{((x,y),x)\in (A\times B)\times A}{\FormulaRefAuto{a\in A\dsep b\in B\vdash (a,b)\in A\times B}{1,2}}
  \proofstep{1}{((x,y),x)\in \pi_1}{\FormulaRefAuto{x \in A\dsep P(x)\dsep B=\{x \in A \mid P(x)\}\vdash x \in B}{4,3,\FormulaRefAuto{\pi_1 := \{\, ((x,y),z) \in (A \times B) \times A \mid (x,y)\in A\times B \land z = x \,\}}}}
\end{tabproof}

% — Totalität: jeder Punkt im Produktraum hat ein Bild —
\FormulaThmDelta[Totalität auf \(A\times B\)]{%
  (x,y)\in A\times B \vdash \exists z\, ((x,y),z)\in \pi_1
}{
  \DeltaRow{Mengen}{x \dsep y \dsep z \dsep A \dsep B}
}
\begin{tabproof}
  \proofstep{1}{(x,y)\in A\times B}{\rA}
  \proofstep{1}{((x,y),x)\in \pi_1}{\FormulaRefAuto{(x,y)\in A\times B \vdash ((x,y),x)\in \pi_1}{1}}
  \proofstep{1}{\exists z\,((x,y),z)\in \pi_1}{\rEI{2}}
\end{tabproof}

% — Funktionale Eindeutigkeit —
\FormulaThmDelta[Funktionale Eindeutigkeit von \(\pi_1\)]{%
  ((x,y),z_1)\in \pi_1 \dsep ((x,y),z_2)\in \pi_1 \vdash z_1 = z_2
}{
  \DeltaRow{Mengen}{x \dsep y \dsep z_1 \dsep z_2 \dsep A \dsep B}
}
\begin{tabproof}
  \proofstep{1}{((x,y),z_1)\in \pi_1}{\rA}
  \proofstep{2}{((x,y),z_2)\in \pi_1}{\rA}
  \proofstep{1}{z_1=x}{\FormulaRefAuto{((x,y),z)\in \pi_1 \vdash z = x}{1}}
  \proofstep{2}{z_2=x}{\FormulaRefAuto{((x,y),z)\in \pi_1 \vdash z = x}{2}}
  \proofstep{1,2}{z_1=z_2}{\FormulaRefAuto{a=b, c=b \vdash a=c}{3,4}}
\end{tabproof}

% — Projektion als Funktion —
\FormulaThmDelta[\(\pi_1\) als Funktion]{%
  \pi_1\colon A\times B \to A
}{
  \DeltaRow{Mengen}{A \dsep B}
}
\begin{tabproof}
  \proofstep{}{%
    \pi_1 \subseteq (A\times B)\times A
  }{\FormulaRefAuto{\pi_1 \subseteq (A \times B) \times A}}
  \proofstep{}{%
    \forall (x,y)\in A\times B\,\exists z\,((x,y),z)\in \pi_1
  }{\FormulaRefAuto{(x,y)\in A\times B \vdash \exists z\, ((x,y),z)\in \pi_1}}
  \proofstep{}{%
    \forall \,(((x,y),z_1), ((x,y),z_2)\in \pi_1\,z_1=z_2)
  }{\FormulaRefAuto{((x,y),z_1)\in \pi_1 \dsep ((x,y),z_2)\in \pi_1 \vdash z_1=z_2}}
  \proofstep{}{%
    \pi_1\colon A\times B\to A
  }{\FormulaRefAuto{Funktion}{1,2,3}}
\end{tabproof}

\subsubsection{Weitere Eigenschaften}

\paragraph{Fixpunkteigenschaft der Projektion}

% — Fixpunkteigenschaft der Projektion —
\FormulaThmDelta[Fixpunkteigenschaft der Projektion]{%
  (x,y)\in A\times B \vdash \pi_1(x,y) = x
}{
  \DeltaRow{Mengen}{x \dsep y \dsep A \dsep B}
}
\begin{tabproof}
  \proofstep{1}{(x,y)\in A\times B}{\rA}
  \proofstep{1}{((x,y),x)\in \pi_1}{\FormulaRefAuto{(x,y)\in A\times B \vdash ((x,y),x)\in \pi_1}{1}}
  \proofstep{1}{\pi_1(x,y)=x}{\FormulaRefAuto{(x,y)\in F \vdash F(x)=y}{2}}
\end{tabproof}

\paragraph{Surjektivität der Projektion}

\FormulaThmDelta[Existenz eines Urbildpaares]{%
  B \neq \emptyset \dsep x\in A \vdash \exists (x,y)\in A\times B\,\bigl(\pi_1(x,y)=x\bigr)
}{
  \DeltaRow{Mengen}{x \dsep y \dsep A \dsep B}
}
\begin{tabproof}
  % 1. Annahme: B ist nicht leer
  \proofstep{1}{B \neq \emptyset}{\rA}
  % 2. Annahme: x \in A
  \proofstep{2}{x\in A}{\rA}
  % 3. Aus B \neq \emptyset folgt: es gibt ein y \in B
  \proofstep{1}{\exists y\in B}{%
    \FormulaRefAuto{A \neq \emptyset \vdash \exists x\in A}{1}}
  % 4. Existenzauswahl: wähle ein beliebiges y \in B
  \proofstep{3}{y\in B}{\rA}
  % 5. Mit x\in A und y\in B liegt (x,y) im Produktraum
  \proofstep{2,3}{(x,y)\in A\times B}{%
    \FormulaRefAuto{a\in A \dsep b\in B \vdash (a,b)\in A\times B}{2,3}}
  % 6. Fixpunkteigenschaft: pi_1(x,y) = x
  \proofstep{2,3}{\pi_1(x,y)=x}{%
    \FormulaRefAuto{(x,y)\in A\times B \vdash \pi_1(x,y)=x}{5}}
  % 7. Existenz eines passenden Paares (x,y)
  \proofstep{2,3}{\exists (x,y)\in A\times B\,\bigl(\pi_1(x,y)=x\bigr)}{\rEI{6}}
  % 8. Beseitigung der Existenzannahme über y
  \proofstep{1,2}{\exists (x,y)\in A\times B\,\bigl(\pi_1(x,y)=x\bigr)}{\rEE{3,7}}
\end{tabproof}

\subsection{Zweite Projektion}

\subsubsection{Definition der Projektion}

% — Definition der zweiten Projektion —
\FormulaDefDeltaK[Zweite Projektion auf \(B\)]{%
  \pi_2 := \{\, ((x,y),z) \in (A \times B) \times B
              \mid (x,y)\in A\times B \land z = y \,\}
}{Projektion2}{
  \DeltaRow{Mengen}{x \dsep y \dsep z \dsep A \dsep B}
}

\subsubsection{Die Funktionseigenschaften}

% — Grundlegende Eigenschaft: Menge geordneter Paare —
\FormulaThmDelta[Existenz als Menge geordneter Paare]{%
  \pi_2 \subseteq (A \times B) \times B
}{
  \DeltaRow{Mengen}{x \dsep y \dsep z \dsep A \dsep B}
}
\begin{tabproof}
  \proofstep{}{%
    \pi_2 \subseteq (A\times B)\times B
  }{%
    \FormulaRefAuto{A = \{ x \in B \mid P(x) \} \vdash A \subseteq B}{%
      \FormulaRefAuto{\pi_2 := \{\, ((x,y),z) \in (A \times B) \times B
        \mid (x,y)\in A\times B \land z = y \,\}}%
    }%
  }
\end{tabproof}

% — Elemente der Projektion liegen im Produktraum —
\FormulaThmDelta{%
  ((x,y),z)\in \pi_2 \vdash (x,y)\in A\times B
}{
  \DeltaRow{Mengen}{x \dsep y \dsep z \dsep A \dsep B}
}
\begin{tabproof}
  \proofstep{1}{((x,y),z)\in \pi_2}{\rA}
  \proofstep{1}{((x,y),z)\in (A\times B)\times B}{%
    \FormulaRefAuto{x\in \{x\in A\mid P(x)\}\vdash x\in A}{1}}
  \proofstep{1}{(x,y)\in A\times B}{%
    \FormulaRefAuto{(a,b)\in A\times B\vdash a\in A}{2}}
\end{tabproof}

\FormulaThmDelta{%
  ((x,y),z)\in \pi_2 \vdash z\in B
}{
  \DeltaRow{Mengen}{x \dsep y \dsep z \dsep A \dsep B}
}
\begin{tabproof}
  \proofstep{1}{((x,y),z)\in \pi_2}{\rA}
  \proofstep{1}{((x,y),z)\in (A\times B)\times B}{%
    \FormulaRefAuto{x\in \{x\in A\mid P(x)\}\vdash x\in A}{1}}
  \proofstep{1}{z\in B}{%
    \FormulaRefAuto{(a,b)\in A\times B\vdash b\in B}{2}}
\end{tabproof}

% — Charakterisierung: z = y —
\FormulaThmDelta{%
  ((x,y),z)\in \pi_2 \vdash z = y
}{
  \DeltaRow{Mengen}{x \dsep y \dsep z \dsep A \dsep B}
}
\begin{tabproof}
  \proofstep{1}{((x,y),z)\in \pi_2}{\rA}
  \proofstep{1}{(x,y)\in A\times B \land z = y}{%
    \FormulaRefAuto{x \in A\dsep A=\{x \in B \mid P(x)\}\vdash P(x)}{%
      1,\FormulaRefAuto{\pi_2 := \{\, ((x,y),z) \in (A \times B) \times B
        \mid (x,y)\in A\times B \land z = y \,\}}%
    }%
  }
  \proofstep{1}{z=y}{\rAEb{2}}
\end{tabproof}

% — Einführungsrichtung: aus (x,y)\in A\times B folgt Element der Projektion —
\FormulaThmDelta{%
  (x,y)\in A\times B \vdash ((x,y),y)\in \pi_2
}{
  \DeltaRow{Mengen}{x \dsep y \dsep z \dsep A \dsep B}
}
\begin{tabproof}
  \proofstep{1}{(x,y)\in A\times B}{\rA}
  \proofstep{1}{y\in B}{\FormulaRefAuto{(a,b)\in A\times B\vdash b\in B}{1}}
  \proofstep{1}{(x,y)\in A\times B \land y=y}{\rAI{1,\rII}}
  \proofstep{1}{((x,y),y)\in (A\times B)\times B}{%
    \FormulaRefAuto{a\in A\dsep b\in B\vdash (a,b)\in A\times B}{1,2}}
  \proofstep{1}{((x,y),y)\in \pi_2}{%
    \FormulaRefAuto{x \in A\dsep P(x)\dsep B=\{x \in A \mid P(x)\}\vdash x \in B}{4,3,\FormulaRefAuto{\pi_2 := \{\, ((x,y),z) \in (A \times B) \times B
      \mid (x,y)\in A\times B \land z = y \,\}}}}
\end{tabproof}

% — Totalität: jeder Punkt im Produktraum hat ein Bild —
\FormulaThmDelta[Totalität auf \(A\times B\)]{%
  (x,y)\in A\times B \vdash \exists z\, ((x,y),z)\in \pi_2
}{
  \DeltaRow{Mengen}{x \dsep y \dsep z \dsep A \dsep B}
}
\begin{tabproof}
  \proofstep{1}{(x,y)\in A\times B}{\rA}
  \proofstep{1}{((x,y),y)\in \pi_2}{%
    \FormulaRefAuto{(x,y)\in A\times B \vdash ((x,y),y)\in \pi_2}{1}}
  \proofstep{1}{\exists z\,((x,y),z)\in \pi_2}{\rEI{2}}
\end{tabproof}

% — Funktionale Eindeutigkeit —
\FormulaThmDelta[Funktionale Eindeutigkeit von \(\pi_2\)]{%
  ((x,y),z_1)\in \pi_2 \dsep ((x,y),z_2)\in \pi_2 \vdash z_1 = z_2
}{
  \DeltaRow{Mengen}{x \dsep y \dsep z_1 \dsep z_2 \dsep A \dsep B}
}
\begin{tabproof}
  \proofstep{1}{((x,y),z_1)\in \pi_2}{\rA}
  \proofstep{2}{((x,y),z_2)\in \pi_2}{\rA}
  \proofstep{1}{z_1=y}{\FormulaRefAuto{((x,y),z)\in \pi_2 \vdash z = y}{1}}
  \proofstep{2}{z_2=y}{\FormulaRefAuto{((x,y),z)\in \pi_2 \vdash z = y}{2}}
  \proofstep{1,2}{z_1=z_2}{\FormulaRefAuto{a=b, c=b \vdash a=c}{3,4}}
\end{tabproof}

% — Projektion als Funktion —
\FormulaThmDelta[\(\pi_2\) als Funktion]{%
  \pi_2\colon A\times B \to B
}{
  \DeltaRow{Mengen}{A \dsep B}
}
\begin{tabproof}
  \proofstep{}{%
    \pi_2 \subseteq (A\times B)\times B
  }{\FormulaRefAuto{\pi_2 \subseteq (A \times B) \times B}}
  \proofstep{}{%
    \forall (x,y)\in A\times B\,\exists z\,((x,y),z)\in \pi_2
  }{\FormulaRefAuto{(x,y)\in A\times B \vdash \exists z\, ((x,y),z)\in \pi_2}}
  \proofstep{}{%
    \forall ((x,y),z_1),((x,y),z_2)\in \pi_2\, z_1=z_2
  }{\FormulaRefAuto{((x,y),z_1)\in \pi_2 \dsep ((x,y),z_2)\in \pi_2 \vdash z_1=z_2}}
  \proofstep{}{%
    \pi_2\colon A\times B\to B
  }{\FormulaRefAuto{Funktion}{1,2,3}}
\end{tabproof}

\subsubsection{Weitere Eigenschaften}

% — Fixpunkteigenschaft der Projektion —
\FormulaThmDelta[Fixpunkteigenschaft der Projektion]{%
  (x,y)\in A\times B \vdash \pi_2(x,y) = y
}{
  \DeltaRow{Mengen}{x \dsep y \dsep A \dsep B}
}
\begin{tabproof}
  \proofstep{1}{(x,y)\in A\times B}{\rA}
  \proofstep{1}{((x,y),y)\in \pi_2}{%
    \FormulaRefAuto{(x,y)\in A\times B \vdash ((x,y),y)\in \pi_2}{1}}
  \proofstep{1}{\pi_2(x,y)=y}{%
    \FormulaRefAuto{(x,y)\in F \vdash F(x)=y}{2}}
\end{tabproof}

\subsection{Komposition}

\subsubsection{Definition der Komposition}

% — Definition —
\FormulaDefDelta[Komposition]{%
G\circ F := \bigl\{\, (x,z) \in A \times C \mid z = G(F(x)) \,\bigr\}
}{
  \DeltaRow{Mengen}{x \dsep y \dsep z \dsep A \dsep B \dsep C}
  \DeltaRow{Funktionen}{F\colon A\to B \dsep G\colon B\to C}
}

\subsubsection{Funktionseigenschaften}

% — Existenz als Menge geordneter Paare —
\FormulaThmDelta[Existenz als Menge geordneter Paare]{%
G\circ F \subseteq A \times C
}{
  \DeltaRow{Mengen}{x \dsep z \dsep A \dsep C}
  \DeltaRow{Funktionen}{F\colon A\to B \dsep G\colon B\to C}
}
\begin{tabproof}
  \proofstep{}{%
    G\circ F = \{\, (x,z) \in A \times C \mid z = G(F(x)) \,\}
  }{\FormulaRefAuto{G\circ F := \{\, (x,z) \in A \times C \mid z = G(F(x)) \,\}}}
  \proofstep{}{%
    \{\, (x,z) \in A \times C \mid z = G(F(x)) \,\} \subseteq A \times C
  }{\FormulaRefAuto{\{ x \in A \mid P(x) \} \subseteq A}}
  \proofstep{}{G\circ F \subseteq A \times C}{\rIE{1,2}}
\end{tabproof}

% — Typisierung der Elemente —
\FormulaThmDelta{%
(x,z)\in G\circ F \vdash x\in A \land z\in C
}{
  \DeltaRow{Mengen}{x \dsep z \dsep A \dsep C}
  \DeltaRow{Funktionen}{F\colon A\to B \dsep G\colon B\to C}
}
\begin{tabproof}
  \proofstep{1}{(x,z)\in G\circ F}{\rA}
  \proofstep{1}{(x,z)\in A\times C}{\FormulaRefAuto{x \in \{x \in A \mid P(x)\}\vdash x\in A}{1}}
  \proofstep{1}{x\in A}{\FormulaRefAuto{(a,b)\in A\times B\vdash a\in A}{2}}
  \proofstep{1}{z\in C}{\FormulaRefAuto{(a,b)\in A\times B\vdash b\in B}{2}}
  \proofstep{1}{x\in A\land z\in C}{\rAI{3,4}}
\end{tabproof}

% — Charakterisierung des Bildpaares —
\FormulaThmDelta{%
(x,z)\in G\circ F \vdash z = G(F(x))
}{
  \DeltaRow{Mengen}{x \dsep z \dsep A \dsep C}
  \DeltaRow{Funktionen}{F\colon A\to B \dsep G\colon B\to C}
}
\begin{tabproof}
  \proofstep{1}{(x,z)\in G\circ F}{\rA}
  \proofstep{1}{z=G(F(x))}{\FormulaRefAuto{x \in \{x \in A \mid P(x)\}\vdash P(x)}{1}}
\end{tabproof}

\FormulaThmDelta{%
x\in A \vdash (x,G(F(x)))\in G\circ F
}{
  \DeltaRow{Mengen}{x \dsep A \dsep C}
  \DeltaRow{Funktionen}{F\colon A\to B \dsep G\colon B\to C}
}
\begin{tabproof}
  \proofstep{1}{x\in A}{\rA}
  \proofstep{1}{F(x)\in B}{\FormulaRefAuto{x\in A\vdash F(x)\in B}{1}}
  \proofstep{1}{G(F(x))\in C}{\FormulaRefAuto{x\in A\vdash F(x)\in B}{2}}
  \proofstep{1}{(x,G(F(x)))\in A\times C}{\FormulaRefAuto{a\in A\dsep b\in B\vdash (a,b)\in A\times B}{1,3}}
  \proofstep{}{G(F(x))=G(F(x))}{\rII}
  \proofstep{1}{(x,G(F(x)))\in \{\, (x,z) \in A \times C \mid z = G(F(x)) \,\}}{\FormulaRefAuto{x \in A, P(x)\vdash x \in \{x \in A \mid P(x)\}}{4,5}}
  \proofstep{1}{(x,G(F(x)))\in G\circ F}{\rIE{\FormulaRefAuto{G\circ F := \{\, (x,z) \in A \times C \mid z = G(F(x)) \,\}},3}}
\end{tabproof}
4
% — Funktionale Eindeutigkeit der Komposition —
\FormulaThmDelta[Funktionale Eindeutigkeit von \(G\circ F\)]{%
(x,z_1)\in G\circ F \dsep (x,z_2)\in G\circ F \vdash z_1=z_2
}{
  \DeltaRow{Mengen}{x \dsep z_1 \dsep z_2 \dsep A \dsep C}
  \DeltaRow{Funktionen}{F\colon A\to B \dsep G\colon B\to C}
}
\begin{tabproof}
  \proofstep{1}{(x,z_1)\in G\circ F}{\rA}
  \proofstep{2}{(x,z_2)\in G\circ F}{\rA}
  \proofstep{1}{z_1=G(F(x))}{\FormulaRefAuto{(x,z)\in G\circ F \vdash z = G(F(x))}{1}}
  \proofstep{2}{z_2=G(F(x))}{\FormulaRefAuto{(x,z)\in G\circ F \vdash z = G(F(x))}{2}}
  \proofstep{1,2}{z_1=z_2}{\FormulaRefAuto{a = b,\, c = b \vdash a = c}{3,4}}
\end{tabproof}

% — Totalität auf A —
\FormulaThmDelta[Totalität auf \(A\)]{%
x\in A \vdash \exists z\, (x,z)\in G\circ F
}{
  \DeltaRow{Mengen}{x \dsep z \dsep A}
  \DeltaRow{Funktionen}{F\colon A\to B \dsep G\colon B\to C}
}
\begin{tabproof}
  \proofstep{1}{x\in A}{\rA}
  \proofstep{1}{(x,G(F(x)))\in G\circ F}{\FormulaRefAuto{x\in A \vdash (x,G(F(x)))\in G\circ F}{1}}
  \proofstep{1}{\exists z\, (x,z)\in G\circ F}{\rEI{2}}
\end{tabproof}

\FormulaThmDelta[Komposition als Funktion]{%
G\circ F\colon A\to C
}{
  \DeltaRow{Mengen}{A\dsep B\dsep C}
  \DeltaRow{Funktionen}{F\colon A\to B \dsep G\colon B\to C}
}
\begin{tabproof}
  \proofstep{}{G\circ F \subseteq A\times C}{\FormulaRefAuto{G\circ F \subseteq A\times C}}
  \proofstep{}{\forall (x,y_1)\dsep (x,y_2)\in G\circ F\,y_1=y_2}{\FormulaRefAuto{(x,z_1)\in G\circ F \dsep (x,z_2)\in G\circ F \vdash z_1=z_2}}
  \proofstep{}{\forall x\in A\exists y\, (x,y)\in G\circ F}{\FormulaRefAuto{x\in A \vdash \exists z\, (x,z)\in G\circ F}}
  \proofstep{}{G\circ F\colon A\to C}{\FormulaRefAuto{Funktion}{1,2,3}}
\end{tabproof}

\subsubsection{Die Auswertungseigenschaft}

\FormulaThmDelta{%
x\in A\dsep y\in A\dsep G(F(x))=G(F(y))\vdash (G\circ F)(x)=(G\circ F)(y)
}{
  \DeltaRow{Mengen}{x \dsep A \dsep B \dsep C}
  \DeltaRow{Funktionen}{F\colon A\to B \dsep G\colon B\to C}
}
\begin{tabproof}
\proofstep{1}{x\in A}{\rA}
\proofstep{2}{y\in A}{\rA}
\proofstep{3}{G(F(x))=G(F(y))}{\rA}
\proofstep{1}{(G\circ F)(x)=G(F(x))}{\FormulaRefAuto{x\in A \vdash (G\circ F)(x)=G(F(x))}{1}}
\proofstep{2}{(G\circ F)(y)=G(F(y))}{\FormulaRefAuto{x\in A \vdash (G\circ F)(x)=G(F(x))}{2}}
\proofstep{3}{(G\circ F)(x)=G(F(y))}{\rIE{4,3}}
\proofstep{3}{(G\circ F)(x)=(G\circ F)(y)}{\rIE{5,6}}
\end{tabproof}

\FormulaThmDelta{%
x\in A\dsep y\in A\dsep (G\circ F)(x)=(G\circ F)(y) \vdash G(F(x))=G(F(y))
}{
  \DeltaRow{Mengen}{x \dsep A \dsep B \dsep C}
  \DeltaRow{Funktionen}{F\colon A\to B \dsep G\colon B\to C}
}
\begin{tabproof}
  \proofstep{1}{x\in A}{\rA}
  \proofstep{2}{y\in A}{\rA}
  \proofstep{3}{(G\circ F)(x)=(G\circ F)(y)}{\rA}

  \proofstep{1}{(G\circ F)(x)=G(F(x))}{\FormulaRefAuto{x\in A \vdash (G\circ F)(x)=G(F(x))}{1}}
  \proofstep{2}{(G\circ F)(y)=G(F(y))}{\FormulaRefAuto{x\in A \vdash (G\circ F)(x)=G(F(x))}{2}}

  \proofstep{3}{G(F(x))=(G\circ F)(y)}{\rIE{4,3}}
  \proofstep{3}{G(F(x))=G(F(y))}{\rIE{5,6}}
\end{tabproof}

\FormulaThmDelta{%
x\in A \vdash (G\circ F)(x)=G(F(x))
}{
  \DeltaRow{Mengen}{x \dsep A \dsep B \dsep C}
  \DeltaRow{Funktionen}{F\colon A\to B \dsep G\colon B\to C}
}
\begin{tabproof}
  \proofstep{1}{x\in A}{\rA}
  \proofstep{1}{(x,G(F(x)))\in G\circ F}{\FormulaRefAuto{x\in A \vdash (x,G(F(x)))\in G\circ F}{1}}
  \proofstep{1}{(G\circ F)(x)=G(F(x))}{\FormulaRefAuto{(x,y)\in F\vdash F(x)=y}{2}}
\end{tabproof}

\subsubsection{Erhalt der Injektivität}

% — Erhalt von Injektivität / Surjektivität / Bijektivität —
\FormulaThmDelta[Erhalt der Injektivität]{%
x\in A \dsep y\in A \dsep (G\circ F)(x)=(G\circ F)(y) \vdash x=y
}{
  \DeltaRow{Mengen}{x \dsep y \dsep A \dsep B \dsep C}
  \DeltaRow{Injektive Funktionen}{F\colon A\to B \dsep G\colon B\to C}
}
\begin{tabproof}
  \proofstep{1}{(G\circ F)(x)=(G\circ F)(y)}{\rA}
  \proofstep{2}{x\in A}{\rA}
  \proofstep{3}{y\in A}{\rA}
  \proofstep{2}{F(x)\in B}{\FormulaRefAuto{x\in A\vdash F(x)\in B}{2}}
  \proofstep{3}{F(y)\in B}{\FormulaRefAuto{x\in A\vdash F(x)\in B}{3}}
  \proofstep{1}{G(F(x))=G(F(y))}{\rIE{\FormulaRefAuto{x\in A \vdash (G\circ F)(x)=G(F(x))},1}}
  \proofstep{1,2,3}{F(x)=F(y)}{\FormulaRefAuto{Injektivität}{4,5,6}}
  \proofstep{1,2,3}{x=y}{\FormulaRefAuto{Injektivität}{2,3,7}}
\end{tabproof}

\FormulaThmDelta[Die Komposition als injektive Funktion]{%
G\circ F\colon A\inj C
}{
  \DeltaRow{Mengen}{A\dsep B\dsep C}
  \DeltaRow{Injektive Funktionen}{F\colon A\to B \dsep G\colon B\to C}
}
\begin{tabproof}
    \proofstep{}{G\circ F\colon A\to C}{\FormulaRefAuto{G\circ F\colon A\to C}}
    \proofstep{}{\forall x,y\in A\,((G\circ F)(x)=(G\circ F)(y) \rightarrow x=y)}{\FormulaRefAuto{x\in A \dsep y\in A \dsep (G\circ F)(x)=(G\circ F)(y) \vdash x=y}}
    \proofstep{}{G\circ F\colon A\inj C}{\FormulaRefAuto{Injektive Funktion}{1,2}}
\end{tabproof}

\subsubsection{Erhalt der Surjektivität}

\FormulaThmDelta[Erhalt der Surjektivität]{%
z\in C \vdash \exists x\in A\, (G\circ F)(x)=z
}{
  \DeltaRow{Mengen}{x \dsep y \dsep z \dsep A \dsep B \dsep C}
  \DeltaRow{Surjektive Funktionen}{F\colon A\to B \dsep G\colon B\to C}
}
\begin{tabproof}
  \proofstep{1}{z\in C}{\rA}
  \proofstep{1}{\exists y\in B\, G(y)=z}{\FormulaRefAuto{Surjektivität}{1}} % (Schema: Surj. von G)
  \proofstep{3}{y\in B \land G(y)=z}{\rA}
  \proofstep{3}{y\in B}{\rAEa{3}}
  \proofstep{3}{G(y)=z}{\rAEb{3}}
  \proofstep{3}{\exists x\in A\, F(x)=y}{\FormulaRefAuto{y\in B \vdash \exists x\in A\, F(x)=y}{4}}
  \proofstep{7}{x\in A \land F(x)=y}{\rA}
  \proofstep{7}{x\in A}{\rAEa{7}}
  \proofstep{7}{F(x)=y}{\rAEb{7}}
  \proofstep{3,7}{G(F(x))=z}{\rIE{9,5}}
  \proofstep{3,7}{(G\circ F)(x)=G(F(x))}{\FormulaRefAuto{x\in A \vdash (G\circ F)(x)=G(F(x))}{8}}
  \proofstep{3,7}{(G\circ F)(x)=z}{\rIE{11,10}}
  \proofstep{3,7}{\exists x\in A\,(G\circ F)(x)=z}{\rEI{\rAI{8,12}}}
  \proofstep{3}{\exists x\in A\,(G\circ F)(x)=z}{\rEE{6,7,13}}
  \proofstep{1}{\exists x\in A\,(G\circ F)(x)=z}{\rEE{2,3,14}}
\end{tabproof}

\FormulaThmDelta[Die Komposition als surjektive Funktion]{%
G\circ F\colon A\sur C
}{
  \DeltaRow{Mengen}{A\dsep B\dsep C}
  \DeltaRow{Surjektive Funktionen}{F\colon A\to B \dsep G\colon B\to C}
}
\begin{tabproof}
    \proofstep{}{G\circ F\colon A\to C}{\FormulaRefAuto{G\circ F\colon A\to C}}
    \proofstep{}{\forall z\in C\exists x\in A\, (G\circ F)(x)=z}{\FormulaRefAuto{z\in C \vdash \exists x\in A\, (G\circ F)(x)=z}}
    \proofstep{}{G\circ F\colon A\sur C}{\FormulaRefAuto{Surjektive Funktion}{1,2}}
\end{tabproof}

\subsubsection{Erhalt der Bijektivität}

\FormulaThmDelta[Die Komposition als bijektive Funktion]{%
F\colon A\bij B \dsep G\colon B\bij C\vdash G\circ F\colon A\bij C
}{
  \DeltaRow{Mengen}{A\dsep B\dsep C}
  \DeltaRow{Funktionen}{F\colon A\to B \dsep G\colon B\to C}
}
\begin{tabproof}
    \proofstep{1}{F\colon A\bij B}{\rA}
    \proofstep{2}{G\colon B\bij C}{\rA}
    \proofstep{1}{F\colon A\sur B}{\FormulaRefAuto{F\colon A\bij B\vdash F\colon A\sur B}{1}}
    \proofstep{1}{F\colon A\inj B}{\FormulaRefAuto{F\colon A\bij B\vdash F\colon A\inj B}{1}}
    \proofstep{2}{G\colon B\sur C}{\FormulaRefAuto{F\colon A\bij B\vdash F\colon A\sur B}{2}}
    \proofstep{2}{G\colon B\inj C}{\FormulaRefAuto{F\colon A\bij B\vdash F\colon A\inj B}{2}}
    \proofstep{1,2}{ G\circ F\colon A\inj C}{\FormulaRefAuto{G\circ F\colon A\inj C}{4,6}}
    \proofstep{1,2}{ G\circ F\colon A\sur C}{\FormulaRefAuto{G\circ F\colon A\sur C}{3,5}}
    \proofstep{1,2}{ G\circ F\colon A\bij C}{\FormulaRefAuto{F\colon A\inj B\dsep F\colon A\sur B\vdash F\colon A\bij B}{7,8}}
\end{tabproof}

\subsubsection{Rechenregeln}

\FormulaThmDelta{%
G = H \vdash F\circ G = F\circ H
}{
  \DeltaRow{Mengen}{A \dsep B \dsep C \dsep D}
  \DeltaRow{Funktionen}{G,H\colon A\to B \dsep F\colon B\to C}
}
\begin{tabproof}
  \proofstep{1}{G=H}{\rA}
  \proofstep{}{F\circ G=F\circ G}{\rII}
  \proofstep{}{F\circ G=F\circ H}{\rIE{1,2}}
\end{tabproof}

\FormulaThmDelta{%
G = H \vdash G\circ F = H\circ F
}{
  \DeltaRow{Mengen}{x \dsep A \dsep B \dsep C}
  \DeltaRow{Funktionen}{F\colon A\to B \dsep G,H\colon B\to C}
}
\begin{tabproof}
  \proofstep{1}{G=H}{\rA}
  \proofstep{}{G\circ F=G\circ F}{\rII}
  \proofstep{}{G\circ F=H\circ F}{\rIE{1,2}}
\end{tabproof}

% — Assoziativität der Komposition —
\FormulaThmDelta[Assoziativität]{%
H\circ (G\circ F) = (H\circ G)\circ F
}{
  \DeltaRow{Mengen}{A \dsep B \dsep C \dsep D}
  \DeltaRow{Funktionen}{F\colon A\to B \dsep G\colon B\to C \dsep H\colon C\to D}
}
\begin{tabproofwide}
  \proofstepwidestar[1]{x\in A}{\rA}
  \proofstepwidestar[1]{F(x)\in B}{\FormulaRefAuto{x\in A\vdash F(x)\in B}{1}}
  \proofstepwidestar[1]{(G\circ F)(x)=G(F(x))}{\FormulaRefAuto{x\in A \vdash (G\circ F)(x)=G(F(x))}{1}}
  \proofstepwide[1]{(H\circ(G\circ F))(x)}{=}{H((G\circ F)(x))}{\FormulaRefAuto{x\in A \vdash (G\circ F)(x)=G(F(x))}{1}}
  \proofstepwide[1]{}{=}{H(G(F(x)))}{\rIE{3,4}}
  \proofstepwide[1]{}{=}{(H\circ G)(F(x))}{\FormulaRefAuto{x\in A \vdash (G\circ F)(x)=G(F(x))}{2}}
  \proofstepwide[1]{}{=}{((H\circ G)\circ F)(x)}{\FormulaRefAuto{x\in A \vdash (G\circ F)(x)=G(F(x))}{1}}
  \proofstepwide[1]{(H\circ(G\circ F))(x)}{=}{((H\circ G)\circ F)(x)}{\rChain{4,7}}
  \proofstepwide[]{\forall x\in A\,((H\circ(G\circ F))(x)}{=}{((H\circ G)\circ F)(x))}{\rUI{\rRI{1,8}}}
  \proofstepwide[]{H\circ (G\circ F)}{=}{(H\circ G)\circ F}{\FormulaRefAuto{\forall x\in A (F(x)=G(x)) \eqvdash F=G}{9}}
\end{tabproofwide}

% — Linksneutralität —
\FormulaThmDelta[Linksneutralität]{%
\Id_B \circ F = F
}{
  \DeltaRow{Mengen}{A \dsep B}
  \DeltaRow{Funktionen}{F\colon A\to B}
}
\begin{tabproofwide}
  \proofstepwidestar[1]{x\in A}{\rA}
  \proofstepwidestar[1]{F(x)\in B}{\FormulaRefAuto{x\in A \vdash F(x)\in B}{1}}

  \proofstepwide[1]{(\Id_B\circ F)(x)}{=}{\Id_B(F(x))}{\FormulaRefAuto{x\in A \vdash (G\circ F)(x)=G(F(x))}{1}}
  \proofstepwide[1]{}{=}{F(x)}{\FormulaRefAuto{x\in A \vdash \Id_A(x)=x}{3}}
  \proofstepwide[1]{(\Id_B\circ F)(x)}{=}{F(x)}{\rChain{3,4}}
  
  \proofstepwide[]{\forall x\in A\,\bigl((\Id_B\circ F)(x)}{=}{F(x)\bigr)}{\rUI{\rRI{1,5}}}
  \proofstepwide[]{\Id_B\circ F}{=}{F}{\FormulaRefAuto{\forall x\in A\,(F(x)=G(x)) \eqvdash F=G}{6}}
\end{tabproofwide}

% — Rechtsneutralität —
\FormulaThmDelta[Rechtsneutralität]{%
F \circ \Id_A = F
}{
  \DeltaRow{Mengen}{A \dsep B}
  \DeltaRow{Funktionen}{F\colon A\to B}
}
\begin{tabproofwide}
  \proofstepwidestar[1]{x\in A}{\rA}

  % Fixpunkteigenschaft der Identität auf A
  \proofstepwidestar[1]{\Id_A(x)=x}{\FormulaRefAuto{x\in A \vdash \Id_A(x)=x}{1}}

  % Komposition auswerten
  \proofstepwide[1]{(F\circ \Id_A)(x)}{=}{F(\Id_A(x))}{\FormulaRefAuto{x\in A \vdash (G\circ F)(x)=G(F(x))}{1}}

  % Ersetze \Id_A(x) durch x
  \proofstepwide[1]{}{=}{F(x)}{\rIE{2,3}}

  \proofstepwide[1]{(F\circ \Id_A)(x)}{=}{F(x)}{\rChain{3,4}}
  
  % Verallgemeinern und Extensionalität
  \proofstepwide[]{\forall x\in A\,\bigl((F\circ \Id_A)(x)}{=}{F(x)\bigr)}{\rUI{\rRI{1,5}}}
  \proofstepwide[]{F\circ \Id_A}{=}{F}{\FormulaRefAuto{\forall x\in A\,(F(x)=G(x)) \eqvdash F=G}{6}}
\end{tabproofwide}

\FormulaThmDelta{%
F\circ F^{-1} = \Id_B
}{
  \DeltaRow{Mengen}{A \dsep B}
  \DeltaRow{Bijektive Funktionen}{F\colon A\to B}
}
\begin{tabproof}
    \proofstep{1}{y\in B}{\rA}
    \proofstep{1}{F(F^{-1}(y))=y}{\FormulaRefAuto{y\in B \vdash F\bigl(F^{-1}(y)\bigr)=y}{1}}
    \proofstep{1}{(F\circ F^{-1})(y)=F(F^{-1}(y))}{\FormulaRefAuto{x\in A \vdash (G\circ F)(x)=G(F(x))}{2}}
    \proofstep{1}{(F\circ F^{-1})(y)=y}{\rIE{2,3}}
    \proofstep{1}{\Id_B(y)=y}{\FormulaRefAuto{x\in A \vdash \Id_A(x)=x}{1}}
    \proofstep{1}{(F\circ F^{-1})(y)=\Id_B(y)}{\rIE{4,5}}
    \proofstep{1}{\forall y\in B\,(F\circ F^{-1})(y)=\Id_B(y)}{\rUI{\rRI{1,6}}}
    \proofstep{1}{F\circ F^{-1}=\Id_B}{\FormulaRefAuto{\forall x\in A (F(x)=G(x)) \eqvdash F=G}{7}}    
\end{tabproof}

\FormulaThmDelta{%
F^{-1}\circ F = \Id_A
}{
  \DeltaRow{Mengen}{x \dsep A \dsep B}
  \DeltaRow{Bijektive Funktionen}{F\colon A\to B}
}
\begin{tabproof}
  \proofstep{1}{x\in A}{\rA}
  \proofstep{1}{F^{-1}(F(x))=x}{\FormulaRefAuto{x\in A \vdash F^{-1}(F(x))=x}{1}}
  \proofstep{1}{(F^{-1}\circ F)(x)=F^{-1}(F(x))}{\FormulaRefAuto{x\in A \vdash (G\circ F)(x)=G(F(x))}{1}}
  \proofstep{1}{(F^{-1}\circ F)(x)=x}{\rIE{2,3}}
  \proofstep{1}{\Id_A(x)=x}{\FormulaRefAuto{x\in A \vdash \Id_A(x)=x}{1}}
  \proofstep{1}{(F^{-1}\circ F)(x)=\Id_A(x)}{\rIE{4,5}}
  \proofstep{ }{\forall x\in A\,\bigl((F^{-1}\circ F)(x)=\Id_A(x)\bigr)}{\rUI{\rRI{1,6}}}
  \proofstep{ }{F^{-1}\circ F=\Id_A}{\FormulaRefAuto{\forall x\in A (F(x)=G(x)) \eqvdash F=G}{7}}
\end{tabproof}

% Inversenformel für Kompositionen
\FormulaThmDelta[Inverse der Komposition]{%
 G^{-1}\circ F^{-1} = (F\circ G)^{-1}
}{
  \DeltaRow{Mengen}{x \dsep y \dsep A \dsep B \dsep C}
  \DeltaRow{Bijektive Funktionen}{G\colon A\to B \dsep F\colon B\to C}
}
\begin{tabproofwide}
  \proofstepwidestar[1]{x\in A}{\rA}
  \proofstepwidestar[1]{G(x)\in B}{\FormulaRefAuto{x\in A \vdash F(x)\in B}{1}}
  \proofstepwidestar[1]{F(G(x))\in C}{\FormulaRefAuto{x\in A \vdash F(x)\in B}{2}}

  \proofstepwidestar[1]{(F\circ G)(x)=F(G(x))}{\FormulaRefAuto{x\in A \vdash (G\circ F)(x)=G(F(x))}{1}}

  \proofstepwidestar[1]{F^{-1}(F(G(x)))=G(x)}{\FormulaRefAuto{x\in A \vdash F^{-1}\bigl(F(x)\bigr)=x}{2}}

  \proofstepwidestar[1]{G^{-1}(G(x))=x}{\FormulaRefAuto{x\in A \vdash F^{-1}\bigl(F(x)\bigr)=x}{1}}

    
  \proofstepwide[1]{(G^{-1}\circ F^{-1})((F\circ G)(x))}{=}{(G^{-1}\circ F^{-1})(F(G(x)))}{\rIE{4,7}}
  \proofstepwide[1]{}{=}{G^{-1}(F^{-1}(F(G(x))))}{\FormulaRefAuto{x\in A \vdash (G\circ F)(x)=G(F(x))}{3}}

  \proofstepwide[1]{}{=}{G^{-1}(G(x))}{\rIE{5,9}}
  \proofstepwide[1]{}{=}{x}{\rIE{6,10}}
  \proofstepwide[1]{(G^{-1}\circ F^{-1})((F\circ G)(x))}{=}{x}{\rChain{7,10}}
  \proofstepwidestar[]{\forall x\in A\, \bigl((F\circ G)(G^{-1}\circ F^{-1})(y)=y\bigr)}{\rUI{\rRI{1,11}}}

  \proofstepwidestar[]{\forall x\in A\, \bigl((F\circ G)(G^{-1}\circ F^{-1})(y)=y\bigr)}{\rUI{\rRI{1,11}}}

  \proofstepwidestar[]{(G^{-1}\circ F^{-1})=(F\circ G)^{-1}}{\FormulaRefAuto{\forall y\in B\,\bigl(F(G(y))=y\bigr) \vdash G = F^{-1}}{13}}
\end{tabproofwide}

\subsubsection{Eigenschaften der Komponenten}

\FormulaThmDeltaK{%
x\in A\dsep y\in A\dsep F(x)=F(y)\vdash x=y
}{G\circ F\colon A\to C bijektiv\dsep x\in A\dsep y\in A\dsep F(x)=F(y)\vdash x=y}{
  \DeltaRow{Mengen}{A\dsep B\dsep C}
  \DeltaRow{Funktionen}{F\colon A\to B\dsep G\colon B\to C}
  \DeltaRow{Bijektive Funktionen}{G\circ F\colon A\to C }
}
\begin{tabproof}
    \proofstep{1}{x\in A}{\rA}
    \proofstep{2}{y\in A}{\rA}
    \proofstep{3}{F(x)=F(y)}{\rA}
    \proofstep{1}{F(x)\in B}{\FormulaRefAuto{x\in A\vdash F(x)\in B}{1}}
    \proofstep{2}{F(y)\in B}{\FormulaRefAuto{x\in A\vdash F(x)\in B}{2}}
    \proofstep{1,2,3}{G(F(x))=G(F(y))}{\FormulaRefAuto{x\in A\dsep y\in A\dsep x=y\vdash F(x)=F(y)}{4,5,3}}
    \proofstep{1,2,3}{(G\circ F)(x)=(G\circ F)(y)}{\FormulaRefAuto{x\in A\dsep y\in A\dsep G(F(x))=G(F(y))\vdash (G\circ F)(x)=(G\circ F)(y)}{6}}
    \proofstep{1,2,3}{x=y}{\FormulaRefAuto{x\in A\dsep y\in A\dsep F(x)=F(y)\vdash x=y}{7}}
\end{tabproof}

\FormulaThmDeltaK{%
y\in C\vdash \exists x\in B\; G(x)=y
}{G\circ F\colon A\to C bijektiv\dsep y\in C\vdash \exists x\in B\; G(x)=y}{
  \DeltaRow{Mengen}{A\dsep B\dsep C}
  \DeltaRow{Funktionen}{F\colon A\to B\dsep G\colon B\to C}
  \DeltaRow{Bijektive Funktionen}{G\circ F\colon A\to C }
}
\begin{tabproof}
    \proofstep{1}{y\in C}{\rA}
    \proofstep{1}{\exists x\in A\, (G\circ F)(x)=y}{\FormulaRefAuto{y\in B\vdash \exists x\in A\; F(x)=y}{1}}
    \proofstep{3}{x\in A\land (G\circ F)(x)=y}{\rA}
    \proofstep{3}{x\in A}{\rAEa{3}}
    \proofstep{3}{(G\circ F)(x)=y}{\rAEb{3}}
    \proofstep{3}{(G\circ F)(x)=G(F(x))}{\FormulaRefAuto{x\in A \vdash (G\circ F)(x)=G(F(x))}{4}}
    \proofstep{3}{G(F(x))=y}{\rEE{6,5}}
    \proofstep{3}{F(x)\in B}{\FormulaRefAuto{x\in A\vdash F(x)\in B}{4}}
    \proofstep{3}{F(x)\in B\land G(F(x))=y}{\rAI{8,7}}
    \proofstep{3}{\exists x\in B\, G(x)=y}{\rEI{9}}
    \proofstep{1}{\exists x\in B\, G(x)=y}{\rEE{2,3,10}}
\end{tabproof}

\FormulaThmDeltaK{%
x\in B\dsep y\in B\dsep G(x)=G(y)\vdash x=y
}{G\circ F\colon A\to C bijektiv\dsep F\colon A\to B surjektiv\dsep x\in B\dsep y\in B\dsep G(x)=G(y)\vdash x=y}{
  \DeltaRow{Mengen}{A\dsep B\dsep C}
  \DeltaRow{Funktionen}{G\colon B\to C}
  \DeltaRow{Surjektive Funktionen}{F\colon A\to B}
  \DeltaRow{Bijektive Funktionen}{G\circ F\colon A\to C }
}
\begin{tabproof}
    \proofstep{1}{x\in B}{\rA}
    \proofstep{2}{y\in B}{\rA}
    \proofstep{3}{G(x)=G(y)}{\rA}
    \proofstep{1}{\exists u\in A\,F(u)=x}{\FormulaRefAuto{y\in B\vdash \exists x\in A\; F(x)=y}{1}}
    \proofstep{2}{\exists v\in A\,F(v)=y}{\FormulaRefAuto{y\in B\vdash \exists x\in A\; F(x)=y}{2}}
    \proofstep{6}{u\in A\land F(u)=x}{\rA}
    \proofstep{6}{u\in A}{\rAEa{6}}
    \proofstep{6}{F(u)=x}{\rAEb{6}}
    \proofstep{9}{v\in A\land F(v)=y}{\rA}
    \proofstep{9}{v\in A}{\rAEa{9}}
    \proofstep{9}{F(v)=y}{\rAEb{9}}
    \proofstep{3,6,9}{G(F(u))=G(F(v))}{\rIE{9,\rIE{8,3}}}
    \proofstep{3,6,9}{(G\circ F)(u)=(G\circ F)(v)}{\FormulaRefAuto{x\in A\dsep y\in A\dsep (G\circ F)(x)=(G\circ F)(y) \vdash G(F(x))=G(F(y))}{7,10,13}}
    \proofstep{3,6,9}{u=v}{\FormulaRefAuto{x\in A \dsep y\in A \dsep (G\circ F)(x)=(G\circ F)(y) \vdash x=y}{7,10,14}}
    \proofstep{3,6,9}{F(u)=F(v)}{\FormulaRefAuto{x\in A\dsep y\in A\dsep x=y\vdash F(x)=F(y)}{7,10,14}}
    \proofstep{3,6,9}{x=y}{\rIE{11,\rIE{8,15}}}
    \proofstep{1,3,9}{x=y}{\rEE{4,6,16}}
    \proofstep{1,2,3}{x=y}{\rEE{5,9,17}}
\end{tabproof}

\FormulaThmDelta{%
 G\circ F\colon A\bij C\vdash F\colon A\inj B
}{
  \DeltaRow{Mengen}{A \dsep B \dsep C}
  \DeltaRow{Funktionen}{F\colon A\to B \dsep G\colon B\to C}
}
\begin{tabproof}
    \proofstep{1}{ G\circ F\colon A\bij C}{\rA}
    \proofstep{1}{ \forall x,y\in A\, (F(x)=F(y)\rightarrow x=y)}{\FormulaRefAuto{G\circ F\colon A\to C bijektiv\dsep x\in A\dsep y\in A\dsep F(x)=F(y)\vdash x=y}{1}}
    \proofstep{1}{F\colon A\inj B}{\FormulaRefAuto{Injektive Funktion}{2}}
\end{tabproof}

\FormulaThmDelta{%
 G\circ F\colon A\bij C\vdash G\colon B\sur C
}{
  \DeltaRow{Mengen}{A \dsep B \dsep C}
  \DeltaRow{Funktionen}{F\colon A\to B \dsep G\colon B\to C}
}
\begin{tabproof}
    \proofstep{1}{ G\circ F\colon A\bij C}{\rA}
    \proofstep{1}{\forall y\in C\exists x\in B\; G(x)=y}{\FormulaRefAuto{G\circ F\colon A\to C bijektiv\dsep y\in C\vdash \exists x\in B\; G(x)=y}{1}}
    \proofstep{1}{F\colon A\inj B}{\FormulaRefAuto{Surjektive Funktion}{2}}
\end{tabproof}


\FormulaThmDelta{%
 F\colon A\sur B\dsep G\circ F\colon A\bij C\vdash G\colon B\bij C
}{
  \DeltaRow{Mengen}{A \dsep B \dsep C}
  \DeltaRow{Funktionen}{F\colon A\to B \dsep G\colon B\to C}
}
\begin{tabproof}
    \proofstep{1}{ F\colon A\sur B}{\rA}
    \proofstep{2}{ G\circ F\colon A\bij C}{\rA}
    \proofstep{2}{G\colon B\sur C}{\FormulaRefAuto{ G\circ F\colon A\bij C\vdash G\colon B\sur C}{2}}
    \proofstep{1,2}{\forall x,y\in B\, (G(x)=G(y)\rightarrow x=y)}{\FormulaRefAuto{G\circ F\colon A\to C bijektiv\dsep F\colon A\to B surjektiv\dsep x\in B\dsep y\in B\dsep G(x)=G(y)\vdash x=y}{1,2}}
    \proofstep{1,2}{G\colon B\inj C}{\FormulaRefAuto{Injektive Funktion}{4}}
    \proofstep{1,2}{G\colon B\bij C}{\FormulaRefAuto{F\colon A\inj B\dsep F\colon A\sur B\vdash F\colon A\bij B}{5,3}}
\end{tabproof}

\FormulaThmDelta{%
F\circ G = \Id_A\dsep G\circ F =\Id_B \vdash F\colon B\bij A
}{
  \DeltaRow{Mengen}{A}
  \DeltaRow{Funktionen}{G\colon A\to B \dsep F\colon B\to A}
}
\begin{tabproof}
    \proofstep{1}{ F\circ G = \Id_A}{\rA}
    \proofstep{2}{ G\circ F =\Id_B}{\rA}
    \proofstep{}{\Id_A\colon A\bij A}{\FormulaRefAuto{\Id_A\colon A\bij A}}
    \proofstep{}{\Id_B\colon B\bij B}{\FormulaRefAuto{\Id_A\colon A\bij A}}
    \proofstep{1}{F\circ G\colon A\bij A}{\rIE{1,3}}
    \proofstep{2}{G\circ F\colon B\bij B}{\rIE{2,4}}
    \proofstep{1}{F\colon B\sur A}{\FormulaRefAuto{G\circ F\colon A\bij C\vdash G\colon B\sur C}{5}}
    \proofstep{2}{F\colon B\inj A}{\FormulaRefAuto{G\circ F\colon A\bij C\vdash F\colon A\inj B}{6}}
    \proofstep{1,2}{F\colon B\bij A}{\FormulaRefAuto{F\colon A\inj B\dsep F\colon A\sur B\vdash F\colon A\bij B}{8,7}}
\end{tabproof}

\FormulaThmDelta{%
F\circ G = \Id_A\dsep G\circ F =\Id_B \vdash G\colon A\bij B
}{
  \DeltaRow{Mengen}{A}
  \DeltaRow{Funktionen}{G\colon A\to B \dsep F\colon B\to A}
}
\begin{tabproof}
    \proofstep{1}{ F\circ G = \Id_A}{\rA}
    \proofstep{2}{ G\circ F =\Id_B}{\rA}
    \proofstep{}{\Id_A\colon A\bij A}{\FormulaRefAuto{\Id_A\colon A\bij A}}
    \proofstep{}{\Id_B\colon B\bij B}{\FormulaRefAuto{\Id_A\colon A\bij A}}
    \proofstep{1}{F\circ G\colon A\bij A}{\rIE{1,3}}
    \proofstep{2}{G\circ F\colon B\bij B}{\rIE{2,4}}
    \proofstep{1}{G\colon A\sur B}{\FormulaRefAuto{G\circ F\colon A\bij C\vdash G\colon B\sur C}{6}}
    \proofstep{2}{G\colon A\inj B}{\FormulaRefAuto{G\circ F\colon A\bij C\vdash F\colon A\inj B}{5}}
    \proofstep{1,2}{F\colon B\bij A}{\FormulaRefAuto{F\colon A\inj B\dsep F\colon A\sur B\vdash F\colon A\bij B}{8,7}}
\end{tabproof}

\FormulaThmDelta{%
F\circ G = \Id_A\dsep G\circ F =\Id_B \vdash G=F^{-1}
}{
  \DeltaRow{Mengen}{A}
  \DeltaRow{Funktionen}{G\colon A\to B \dsep F\colon B\to A}
}
\begin{tabproofwide}
    \proofstepwidestar[1]{ F\circ G = \Id_A}{\rA}
    \proofstepwidestar[2]{ G\circ F =\Id_B}{\rA}
    \proofstepwidestar[1,2]{F\colon B\bij A}{\FormulaRefAuto{F\circ G = \Id_A\dsep G\circ F =\Id_B \vdash F\colon B\bij A}{1,2}}
    \proofstepwide[]{G}{=}{G\circ \Id_A}{\FormulaRefAuto{F \circ \Id_A = F}}
    \proofstepwide[1,2]{}{=}{G\circ (F\circ F^{-1})}{\rIE{\FormulaRefAuto{F\circ F^{-1} = \Id_B}{3},4}}
    \proofstepwide[1,2]{}{=}{(G\circ F)\circ F^{-1}}{\FormulaRefAuto{H\circ (G\circ F) = (H\circ G)\circ F}{5}}
    \proofstepwide[1,2]{}{=}{\Id_B\circ F^{-1}}{\rIE{2,6}}
    \proofstepwide[1,2]{}{=}{F^{-1}}{\FormulaRefAuto{\Id_B\circ F = F}}
    \proofstepwide[1,2]{G}{=}{F^{-1}}{\rChain{4,8}}
\end{tabproofwide}

\chapter{Das Auswahlaxiom}

\FormulaAxiomDelta[Auswahlaxiom (AC – Schnittform)]{
  \exists F\colon A \to B \, (G \circ F = \Id_A)
}{
  \DeltaRow{Mengen}{A \dsep B}
  \DeltaRow{Surjektive Funktionen}{G\colon B \sur A}
}

\section{Äquivalente Formulierungen}

\FormulaDefDeltaK[Begriff der totalen Relation]{F \subseteq A \times B}{Totale Relation}{
  \DeltaRow{Mengen}{A \dsep B \dsep F \dsep x \dsep y}
  \DeltaRow{\textbf{Axiome}}{}
  %
  % — Existenz als Menge geordneter Paare —
  \DeltaRow{Menge geordneter Paare}
           {F \subseteq A \times B}
           [\FormulaRefAuto{F \subseteq A \times B}]
  %
  % — Totalität —
  \DeltaRow{Totalität}
           {x \in A \vdash \exists y\,(x,y)\in F}
           [\FormulaRefAuto{x \in A \vdash \exists y\,(x,y)\in F}]
  %
  \DeltaRow{\textbf{Neue Symbole}}{}
  \DeltaRow{Totale Relationen}{ }
}

\FormulaThmDelta[Auswahlaxiom (AC – Relationsform)]{%
  \exists F\colon A\to B\,\forall x\in A\,\bigl((x,F(x))\in R\bigr)
}{%
  \DeltaRow{Mengen}{A \dsep B}
  \DeltaRow{Totale Relation}{R \text{ totale Relation auf } A\dsep B}
  \DeltaRow{Funktion}{F\colon A\to B}
}



\chapter{Konstruktion der natürlichen Zahlen}

Wir konstruieren die Menge der natürlichen Zahlen allein aus den Axiomen der Mengenlehre, insbesondere dem Unendlichkeitsaxiom.  Dieser Ansatz kommt ohne eine explizite Nachfolgerfunktion als primitives Symbol aus; diese wird vielmehr aus der Mengenoperation \(x \cup \{x\}\) gewonnen.

\section{Axiom der Unendlichkeit}


\FormulaAxiomAuto[Unendlichkeit]{ \exists A\;\bigl(\emptyset \in A \land \forall x \in A\,(x \cup \{x\} \in A)\bigr) }

% === Kapitel: Konstruktion der natürlichen Zahlen (umformuliert – Fokus: Funktion erster Ordnung auf A) ===

\section{Nachfolger und induktive Mengen}

\subsection{Induktive Mengen}

\FormulaDefAuto[induktive Menge]{\mathrm{Induktiv}(A) := \emptyset \in A \,\land\, \forall x\in A\,(x \cup \{x\} \in A)}[Für eine Menge \(A\) definieren wir:]

% Existenz einer induktiven Menge (aus dem Unendlichkeitsaxiom)
\FormulaThmAuto{\exists A(\mathrm{Induktiv}(A))}
\begin{tabproof}
  \proofstep{}{ \exists A\bigl(\emptyset \in A \land \forall x \in A\,(x \cup \{x\} \in A)\bigr) }{\FormulaRefAuto{\exists A\bigl(\emptyset \in A \land \forall x \in A\,(x \cup \{x\} \in A)\bigr)}}
  \proofstep{}{\exists A(\mathrm{Induktiv}(A)) }{\rIE{\FormulaRefAuto{\mathrm{Induktiv}(A) := \emptyset \in A \,\land\, \forall x\in A\,(x \cup \{x\} \in A)},1}}
\end{tabproof}

% Sofortige Konsequenzen aus der Definition
\FormulaThmAuto{\mathrm{Induktiv}(A)\vdash \emptyset\in A}
\begin{tabproof}
  \proofstep{1}{\mathrm{Induktiv}(A)}{\rA}
  \proofstep{1}{\emptyset \in A \land \forall x \in A\,(x \cup \{x\} \in A)}{\rIE{\FormulaRefAuto{\mathrm{Induktiv}(A) := \emptyset \in A \,\land\, \forall x\in A\,(x \cup \{x\} \in A)},1}}
  \proofstep{1}{\emptyset \in A}{\rAEa{2}}
\end{tabproof}

\FormulaThmAuto{\mathrm{Induktiv}(A)\vdash \forall x \in A\,(x \cup \{x\} \in A)}
\begin{tabproof}
  \proofstep{1}{\mathrm{Induktiv}(A)}{\rA}
  \proofstep{1}{\emptyset \in A \land \forall x \in A\,(x \cup \{x\} \in A)}{\rIE{\FormulaRefAuto{\mathrm{Induktiv}(A) := \emptyset \in A \,\land\, \forall x\in A\,(x \cup \{x\} \in A)},1}}
  \proofstep{1}{\forall x \in A\,(x \cup \{x\} \in A)}{\rAEb{2}}
\end{tabproof}

\FormulaThmAuto{\mathrm{Induktiv}(A),\,x\in A\vdash x \cup \{x\} \in A}
\begin{tabproof}
  \proofstep{1}{\mathrm{Induktiv}(A)}{\rA}
  \proofstep{2}{x\in A}{\rA}
  \proofstep{1}{\forall u \in A\,(u \cup \{u\} \in A)}{\FormulaRefAuto{\mathrm{Induktiv}(A)\vdash \forall x \in A\,(x \cup \{x\} \in A)}{1}}
  \proofstep{1,2}{x \cup \{x\} \in A}{\rRE{\rUE{3},2}}
\end{tabproof}

% Schnitt induktiver Mengen ist induktiv
\FormulaThmAuto{\mathrm{Induktiv}(A),\, \mathrm{Induktiv}(B)\vdash \mathrm{Induktiv}(A\cap B)}
\begin{tabproof}
  \proofstep{1}{\mathrm{Induktiv}(A)}{\rA}
  \proofstep{2}{\mathrm{Induktiv}(B)}{\rA}

  % Null in A∩B
  \proofstep{1}{\,\emptyset\in A}{\FormulaRefAuto{\mathrm{Induktiv}(A)\vdash \emptyset\in A}{1}}
  \proofstep{2}{\,\emptyset\in B}{\FormulaRefAuto{\mathrm{Induktiv}(A)\vdash \emptyset\in A}{2}}
  \proofstep{1,2}{\,\emptyset\in A\cap B}{\FormulaRefAuto{x\in A,\, x\in B \vdash x\in A\cap B}{3,4}}

  % Abschluss unter Nachfolger-Operation für A∩B
  \proofstep{1}{\forall x\in A\,(x\cup\{x\}\in A)}{\FormulaRefAuto{\mathrm{Induktiv}(A)\vdash \forall x \in A\,(x \cup \{x\} \in A)}{1}}
  \proofstep{2}{\forall x\in B\,(x\cup\{x\}\in B)}{\FormulaRefAuto{\mathrm{Induktiv}(A)\vdash \forall x \in A\,(x \cup \{x\} \in A)}{2}}
  \proofstep{8}{x\in A\cap B}{\rA}
  \proofstep{8}{x\in A}{\FormulaRefAuto{x\in A\cap B \vdash x\in A}{9}}
  \proofstep{8}{x\in B}{\FormulaRefAuto{x\in A\cap B \vdash x\in B}{9}}
  \proofstep{1,8}{x\cup\{x\}\in A}{\rRE{\rUE{6},9}}
  \proofstep{2,8}{x\cup\{x\}\in B}{\rRE{\rUE{7},10}}
  \proofstep{1,2,8}{x\cup\{x\}\in A\cap B}{\FormulaRefAuto{x\in A,\, x\in B \vdash x\in A\cap B}{11,12}}
  \proofstep{1,2}{\,\forall x\in A\cap B\,(x\cup\{x\}\in A\cap B)}{\rUI{\rRI{8,13}}}

  % Schluss: Induktivität von A∩B
  \proofstep{1,2}{\,\emptyset\in A\cap B \;\land\; \forall x\in A\cap B\,(x\cup\{x\}\in A\cap B)}{\rAI{5,14}}
  \proofstep{1,2}{\,\mathrm{Induktiv}(A\cap B)}%
    {\rIE{\FormulaRefAuto{\mathrm{Induktiv}(A) := \emptyset \in A \,\land\, \forall x\in A\,(x \cup \{x\} \in A)},15}}
\end{tabproof}

\subsection{Die Nachfolger-Funktion}

\FormulaDefAuto
{\mathrm{succ}_A := \{\, (x,y) \in A \times A \mid y = x \cup \{x\} \,\}}%
[Sei \(A\) eine Menge. Dann definieren wir:]

\FormulaThmAuto{\mathrm{succ}_A \subseteq A \times A}
\begin{tabproof}
  \proofstep{}{\mathrm{succ}_A\subseteq A\times A}{\FormulaRefAuto{\{ x \in A \mid P(x) \} \subseteq A}{\FormulaRefAuto
{\mathrm{succ}_A := \{\, (x,y) \in A \times A \mid y = x \cup \{x\} \,\}}}}
\end{tabproof}

\FormulaThmAuto[Funktionale Eindeutigkeit von \(\mathrm{succ}_A\)]%
{(x,y)\in \mathrm{succ}_A,\,(x,z)\in \mathrm{succ}_A \vdash y=z}[Seien \(x,y,z\in A\), dann gilt:]
\begin{tabproof}
  \proofstep{1}{(x,y)\in \mathrm{succ}_A}{\rA}
  \proofstep{2}{(x,z)\in \mathrm{succ}_A}{\rA}
  \proofstep{1}{y = x\cup\{x\}}{\FormulaRefAuto{x \in \{x \in A \mid P(x)\} \eqvdash x \in A \land P(x)}{\FormulaRefAuto
{\mathrm{succ}_A := \{\, (x,y) \in A \times A \mid y = x \cup \{x\} \,\}}{1}}}
  \proofstep{2}{z = x\cup\{x\}}{\FormulaRefAuto{x \in \{x \in A \mid P(x)\} \eqvdash x \in A \land P(x)}{\FormulaRefAuto
{\mathrm{succ}_A := \{\, (x,y) \in A \times A \mid y = x \cup \{x\} \,\}}{2}}}
  \proofstep{1,2}{y=z}{\rIE{4,3}}
\end{tabproof}

\FormulaThmAuto{\mathrm{Induktiv}(A), x\in A\vdash (x,x\cup \{x\})\in \mathrm{succ}_A}
\begin{tabproof}
  \proofstep{1}{ \mathrm{Induktiv}(A) }{\rA}
  \proofstep{2}{ x\in A }{\rA}
  \proofstep{1,2}{ x\cup\{x\} \in A }{\FormulaRefAuto{\mathrm{Induktiv}(A),\,x\in A\vdash x \cup \{x\} \in A}{1,2}}
  \proofstep{1,2}{ (x,x\cup\{x\}) \in A\times A}{\FormulaRefAuto{a\in A\dsep b\in B\vdash (a,b)\in A\times B}{3}}
  \proofstep{}{ (x,x\cup\{x\})=(x,x\cup\{x\})}{\rII}
  \proofstep{1,2}{(x,x\cup\{x\}) \in \mathrm{succ}_A}{\FormulaRefAuto{x \in \{x \in A \mid P(x)\} \eqvdash x \in A \land P(x)}{4,5}}
\end{tabproof}



\FormulaThmAuto[Totalität von \(\mathrm{succ}_A\) auf induktiven Mengen]%
{\mathrm{Induktiv}(A) \vdash \forall x\in A\,\exists y\in A\;((x,y)\in \mathrm{succ}_A)}
\begin{tabproof}
  \proofstep{1}{ \mathrm{Induktiv}(A) }{\rA}
  \proofstep{2}{ x\in A }{\rA}
  \proofstep{1,2}{ x\cup\{x\} \in A }{\FormulaRefAuto{\mathrm{Induktiv}(A),\,x\in A\vdash x \cup \{x\} \in A}{1,2}}
  \proofstep{1,2}{ (x,x\cup \{x\})\in \mathrm{succ}_A }{\FormulaRefAuto{\mathrm{Induktiv}(A), x\in A\vdash (x,x\cup \{x\})\in \mathrm{succ}_A}{1,2]}}
  \proofstep{1,2}{ \exists y\in A((x,y)\in \mathrm{succ}_A) }{\rEI{\rAI{3,4}}}  
  \proofstep{1,2}{\forall x\in A\exists y\in A((x,y)\in \mathrm{succ}_A) }{\rUI{\rRI{2,5}}}  
\end{tabproof}

\begin{remark}
Diese Sätze zeigen genau die Funktionseigenschaften von \(\mathrm{succ}_A\)  und führen zu folgender Definition.
\end{remark}

\FormulaDefAuto[Nachfolger]{\mathrm{succ}(x) := x \cup \{x\}}[Sei A eine induktive Menge, dann definieren wir die Funktion \(\mathrm{succ}_A:A\mapsto A\)  für alle \(x\in A\) durch:]

\section{Konstruktion der natürlichen Zahlen}

\FormulaThmAuto{
  \exists! C\; \forall B \Bigl(\mathrm{Induktiv}(B) \rightarrow 
    C = \{ x \in B \mid \forall A (\mathrm{Induktiv}(A) \rightarrow x \in A) \}\Bigr)
}
\begin{tabproofwide}
  % Schritt 1: Existenz einer induktiven Menge
  \proofstepwidestar[]{\exists A(\mathrm{Induktiv}(A))}%
    {\FormulaRefAuto{\exists A(\mathrm{Induktiv}(A))}}

  % Schritt 2: Einzigartige Existenz des Schnitts aller induktiven Mengen
  \proofstepwide{}{}{\exists! C\; \forall B(\mathrm{Induktiv}(B)\rightarrow }%
    {\multirow{2}{*}{\FormulaRefAuto{P(D)\vdash \exists! C\forall B(P(B)\rightarrow C= \{ x \in B \mid \forall A (P(A) \rightarrow x \in A) \})}{1}}}
  \proofstepwide{}{}%
    {C=\{x\in B \mid \forall A (\mathrm{Induktiv}(A)\rightarrow x\in A)\})}%
    {}
\end{tabproofwide}


\FormulaDefAuto[Menge der natürlichen Zahlen]{\mathbb{N} := \bigcap \{ A \mid \mathrm{Induktiv}(A) \}}[Sei \(\mathrm{Induktiv}(A)\) wie oben definiert.  Wir setzen:]

\begin{remark}
  Die Definition sagt: Ein Element \(x\) gehört genau dann zu \(\mathbb{N}\), wenn es zu jeder induktiven Menge gehört.  Damit ist \(\mathbb{N}\) die \emph{kleinste} induktive Menge, denn für jede andere induktive Menge \(B\) gilt \(\mathbb{N} \subseteq B\).
\end{remark}

\FormulaThmAuto{x\in\mathbb{N}\eqvdash \forall \mathrm{Induktiv}(N)(x\in N)}
\begin{tabproofwide}
    \proofstepwide[]{x\in \mathbb{N}}{\leftrightarrow}{x\in \bigcap \{ A \mid \mathrm{Induktiv}(A) \}}{\FormulaRefAuto{\mathbb{N} := \bigcap \{ A \mid \mathrm{Induktiv}(A) \}}}
    \proofstepwide[]{}{\leftrightarrow}{\forall \mathrm{Induktiv}(N)(x\in N)}{\FormulaRefAuto{\exists A(P(A)) \vdash x \in \bigcap_{P(B)} B \leftrightarrow \forall C\, (P(C) \rightarrow x \in C)}{\FormulaRefAuto{\exists A(\mathrm{Induktiv}(A))}}}
\end{tabproofwide}

\section{Eigenschaften der Menge der natürlichen Zahlen}

Im Folgenden werden wir einige grundlegende Eigenschaften der so konstruierten Menge \(\mathbb{N}\) beweisen.

\subsection{Induktivität der Menge der natürlichen Zahlen}


\FormulaThmAuto[Null-Axiom]{\emptyset\in\mathbb{N}}
\begin{tabproof}
    \proofstep{1}{\mathrm{Induktiv}(N)}{\rA}
    \proofstep{1}{\emptyset\in N}{\FormulaRefAuto{\mathrm{Induktiv}(A)\vdash \emptyset\in A}{1}}
    \proofstep{}{\forall \mathrm{Induktiv}(N)(x\in N)}{\rUI{\rRI{1,2}}}
    \proofstep{}{\emptyset\in\mathbb{N}}{\FormulaRefAuto{x\in\mathbb{N}\eqvdash \forall \mathrm{Induktiv}(N)(x\in N)}{3}}
\end{tabproof}

\FormulaThmAuto[Nachfolger-Axiom]{\forall x \in \mathbb{N}\,(\mathrm{succ}(x) \in \mathbb{N})}
\begin{tabproof}
  % Nehme ein beliebiges x \in \mathbb{N} an und zeige succ(x) \in \mathbb{N}
  \proofstep{1}{x \in \mathbb{N}}{\rA}

  % Charakterisierung von \mathbb{N} benutzen
  \proofstep{1}{\forall \mathrm{Induktiv}(N)(x\in N)}%
    {\FormulaRefAuto{x\in\mathbb{N}\ \eqvdash\ \forall \mathrm{Induktiv}(N)(x\in N)}{1}}

  % Sei N induktiv; daraus folgt x \in N und dann succ(x) \in N
  \proofstep{3}{\mathrm{Induktiv}(N)}{\rA}
  \proofstep{1}{\mathrm{Induktiv}(N)\rightarrow x\in N}{\rUE{2}}
  \proofstep{1,3}{x\in N}{\rRE{4,3}}
  \proofstep{1,3}{\mathrm{succ}(x)\in N}{\FormulaRefAuto{\mathrm{Induktiv}(A),x\in A\vdash \mathrm{succ}(x) \in A}{3,5}}

  % Verallgemeinere auf alle induktiven N
  \proofstep{1}{\,\forall \mathrm{Induktiv}(N)\,(\mathrm{succ}(x)\in N)}{\rUI{\rRI{3,6}}}

  % Charakterisierung von \mathbb{N} rückwärts auf succ(x) anwenden
  \proofstep{1}{\,\mathrm{succ}(x)\in \mathbb{N}}%
    {\FormulaRefAuto{x\in\mathbb{N}\eqvdash \forall \mathrm{Induktiv}(N)(x\in N)}{7}}

  % Schluss: All-Intro über x in \mathbb{N}
  \proofstep{}{\,\forall x\in \mathbb{N}\,(\mathrm{succ}(x)\in \mathbb{N})}{\rUI{\rRI{1,8}}}
\end{tabproof}

\FormulaThmAuto{\mathrm{Induktiv}(\mathbb{N})}
\begin{tabproof}
  \proofstep{}{\,\emptyset\in\mathbb{N}}%
    {\FormulaRefAuto{\emptyset\in\mathbb{N}}{}}
  \proofstep{}{\,\forall x\in\mathbb{N}\,(\mathrm{succ}(x)\in\mathbb{N})}%
    {\FormulaRefAuto{\forall x \in \mathbb{N}\,(\mathrm{succ}(x) \in \mathbb{N})}{}}
  \proofstep{}{\,\emptyset\in\mathbb{N}\ \land\ \forall x\in\mathbb{N}\,(\mathrm{succ}(x)\in\mathbb{N})}%
    {\rAI{1,2}}
  \proofstep{}{\,\mathrm{Induktiv}(\mathbb{N})}%
    {\rIE{\FormulaRefAuto{\mathrm{Induktiv}(A) := \emptyset \in A \,\land\, \forall x\in A\,(\mathrm{succ}(x) \in A)},3}}
\end{tabproof}

\subsection{Minimalität der Menge der natürlichen Zahlen}

\FormulaThmAuto[Minimalität]{\mathrm{Induktiv}(A) \vdash \mathbb{N} \subseteq A}[Sei \(A\) eine Menge, dann gilt:]
\begin{tabproof}
  \proofstep{1}{ \mathrm{Induktiv}(A) }{\rA}
  \proofstep{2}{ \mathbb{N} \subseteq A }{\FormulaRefAuto{P(C)\vdash \bigcap_{P(A)} A \subseteq C}{1}}
\end{tabproof}

\section{Herleitung der Peano-Axiome}

Die folgenden Abschnitte zeigen, wie sich die Peano-Axiome aus den zuvor eingeführten Definitionen und Sätzen ergeben.

\subsection{Das Null-Axiom}

Aus \FormulaRefAuto{\emptyset\in\mathbb{N}} folgt, dass die Null in den natürlichen Zahlen enthalten ist.

\subsection{Nachfolger‑Axiom}

Aus \FormulaRefAuto{\forall x \in \mathbb{N}\,(\mathrm{succ}(x) \in \mathbb{N})} ergibt sich, dass die Menge der natürlichen Zahlen mit der Nachfolgerfunktion abgeschlossen ist.

\subsection{Kein Vorgänger von Null}

\FormulaThmAuto{\mathrm{succ}(x) \neq \emptyset}
\begin{tabproof}
  \proofstep{}{ x \cup \{x\}\neq\emptyset }{\FormulaRefAuto{A\cup \{a\}\neq\emptyset}}
  \proofstep{}{\mathrm{succ}(x)\neq\emptyset}{\rIE{\FormulaRefAuto{\mathrm{succ}(x) := x \cup \{x\}},2}}
\end{tabproof}

\subsection{Injektivität der Nachfolgerfunktion}

\FormulaThmAuto{\mathrm{succ}(x) = \mathrm{succ}(y) \vdash x = y}

\begin{tabproof}
  \proofstep{1}{ \mathrm{succ}(x) = \mathrm{succ}(y) }{\rA}
  \proofstep{1}{ x \cup \{x\} = y \cup \{y\} }{\rIE{\FormulaRefAuto{\mathrm{succ}(x) := x \cup \{x\}},1}}
  \proofstep{1}{ x = y }{\FormulaRefAuto{a\cup\{a\}=b\cup\{b\}\eqvdash a=b}{2}}
\end{tabproof}

\subsection{Induktionsprinzip}

\FormulaThmAuto{P(\emptyset),\, \forall x\in\mathbb{N}\,(P(x) \rightarrow P(\mathrm{succ}(x))) \vdash \mathrm{Induktiv}(\{x \in \mathbb{N} \mid P(x)\})}[Für jedes einstellige Prädikat \(P\) gilt:]
\begin{notation*}
    Im Beweis bezeichnen wir mit \(S := \{x \in \mathbb{N} \mid P(x)\}\).
\end{notation*}
\begin{tabproof}
  \proofstep{1}{ P(\emptyset) }{\rA}
  \proofstep{2}{ \forall x \in \mathbb{N}\,(P(x) \rightarrow P(\mathrm{succ}(x))) }{\rA}
  \proofstep{1}{\emptyset\in S}{\FormulaRefAuto{x \in \{x \in A \mid P(x)\} \eqvdash x \in A \land P(x)}{\FormulaRefAuto{\emptyset\in\mathbb{N}},1}}
  \proofstep{4}{n\in S}{\rA}
  \proofstep{4}{n\in \mathbb{N}}{\rAEa{\FormulaRefAuto{x \in \{x \in A \mid P(x)\} \eqvdash x \in A \land P(x)}{4}}}
  \proofstep{4}{P(n)}{\rAEb{\FormulaRefAuto{x \in \{x \in A \mid P(x)\} \eqvdash x \in A \land P(x)}{4}}}
  \proofstep{4}{\mathrm{succ}(n)\in\mathbb{N}}{\rRE{\rUE{\FormulaRefAuto{\forall x \in \mathbb{N}\,(\mathrm{succ}(x) \in \mathbb{N})}},5}}
  \proofstep{2,4}{P(\mathrm{succ}(n))}{\rRE{\rRE{\rUE{2},5},6}}
  \proofstep{2,4}{\mathrm{succ}(n)\in S}{\FormulaRefAuto{x \in \{x \in A \mid P(x)\} \eqvdash x \in A \land P(x)}{7,8}}
  \proofstep{2}{\forall n\in S(\mathrm{succ}(n)\in S)}{\rUI{\rRI{4,9}}}
  \proofstep{1,2}{ \mathrm{Induktiv}(S) }{\FormulaRefAuto{\mathrm{Induktiv}(A) := \emptyset \in A \,\land\, \forall x\in A\,(\mathrm{succ}(x) \in A)}{\rAI{3,10}}}
\end{tabproof}

\FormulaThmAuto[Induktion auf \(\mathbb{N}\)]{P(\emptyset),\, \forall x\in\mathbb{N}\,(P(x) \rightarrow P(\mathrm{succ}(x))) \vdash \forall x\in\mathbb{N}\,(P(x))}[Für jedes einstellige Prädikat \(P\) gilt:]
\begin{tabproof}
  \proofstep{1}{ P(\emptyset) }{\rA}
  \proofstep{2}{ \forall x \in \mathbb{N}\,(P(x) \rightarrow P(\mathrm{succ}(x))) }{\rA}
  \proofstep{1,2}{ \mathrm{Induktiv}(\{x \in \mathbb{N} \mid P(x)\}) }{\FormulaRefAuto{P(\emptyset),\, \forall x\in\mathbb{N}\,(P(x) \rightarrow P(\mathrm{succ}(x))) \vdash \mathrm{Induktiv}(\{x \in \mathbb{N} \mid P(x)\})}{1,2}}
  \proofstep{}{ \mathbb{N} \subseteq \{x \in \mathbb{N} \mid P(x)\} }{\FormulaRefAuto{\mathrm{Induktiv}(A) \vdash \mathbb{N} \subseteq A}{3}}
  \proofstep{1,2}{ \forall x\in\mathbb{N}\,(P(x)) }{\FormulaRefAuto{A\subseteq \{x\in B\mid P(x)\}\vdash \forall x\in A(P(x))}{4}}
\end{tabproof}


\chapter{Auswahlaxiom}

\begin{definition}[Paarweise disjunkte Familie nichtleerer Mengen]
Seien \(A\) eine Menge, dann heißt \(A\) eine \textbf{paarweise disjunkte Familie nichtleerer Mengen}, falls die Menge \(A\) die folgenden Axiome erfüllt.

\noindent\textbf{Axiome für paarweise disjunkte Familien nichtleerer Mengen}
\begin{AxTable}
  \AxLine{Nichtleerheit}{x \in A \vdash x \neq \emptyset}
  \AxLine{Paarweise Disjunktheit}{x \neq y,\, x,y \in A \vdash x \cap y = \emptyset}
\end{AxTable}
\end{definition}

% Nichtleerheit jedes Elements
\FormulaAxiomDelta[Nichtleerheit]{x \in A \vdash x \neq \emptyset}{
  \DeltaRow{Mengen}{x}%
  \DeltaRow{Paarweise disjunkte Familie nichtleerer Mengen}{A}%
}

% Paarweise Disjunktheit (Implikationsform, empfohlen)
\FormulaAxiomDelta[Paarweise Disjunktheit]{x \neq y,\, x,y \in A \vdash x \cap y = \emptyset}{
  \DeltaRow{Mengen}{x \dsep y}%
  \DeltaRow{Paarweise disjunkte Familie nichtleerer Mengen}{A}%
}

Hieraus ergibt sich nun das Auswahlaxiom der Zermelo-Fränkel-Axiome.

\FormulaAxiomDelta[Auswahlaxiom (AC)]{
\exists B\,\forall x \in A\,\exists z\,B\cap x = \{z\}
}{
\DeltaRow{Paarweise disjunkte Familie nichtleerer Mengen}{A}
}



Auf Methaebene definieren wir für beliebige Mengen \(x\) und \(A\): \[x\not\in A:=\neg(x\in A)\]

\begin{remark}
In folgenden Abschnitten bezeichnen wir \(A,\, B,\, C\) als Mengen. Für die Elemente von Mengen verwenden wir häufig Kleinbuchstaben als Symbole wie \(x,\, y,\, z\).
\end{remark}

\PrintFormulaStack



\end{document}

