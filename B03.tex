%============================================================
%  Bd. 03 - Mengenlehre %============================================================

\documentclass[main.tex]{subfiles}


\ifSubfilesClassLoaded{
    \usepackage{xr}
    \externaldocument{registry/_B01}
  \externaldocument{registry/_B02}
  \externaldocument{registry/_B03}
}{
   % Code für als Subfile eingebunden
}

\title{Bd. 03 - Mengenlehre}
\author{Martin Kunze}
\date{}

% Standalone-B03: feste Registry/Debug-Dateien + Cross-Band-Load
% Standalone-B03: feste Registry/Debug-Dateien + Cross-Band-Load
\ifSubfilesClassLoaded{
  \directlua{
    thmlookup.registry_path = "registry/_B03.registry.tsv";
    thmlookup.debug_path    = "registry/_B03.debug.log";
    thmlookup.prepare_run();
    thmlookup.load_registry_file("registry/_B01.registry.tsv");
    thmlookup.load_registry_file("registry/_B02.registry.tsv");
    texio.write_nl("thmlookup: loaded B02 registry")
  }
}{}


\begin{document}

\maketitle
\tableofcontents
%\listoftheorems
\setcounter{file}{3}
\renewcommand{\FormulaBandID}{3}


\chapter{Einführung}

Die Mengenlehre ist ein fundamentaler Teil der Mathematik, der die Grundlage für viele andere Bereiche bildet. In diesem Kapitel werden wir die Zermelo-Fraenkel (ZF) Axiome der Mengenlehre einführen und diskutieren. Dabei bezeichnen \( A \), \( B \), \( C \) und \( D \) stets Mengen, es sei denn, es wird ausdrücklich etwas anderes angegeben. Alle Variablen, die als Mengen bezeichnet werden, sind implizit durch Allquantoren gebunden, es sei denn, es wird ein anderer Quantor verwendet. Das bedeutet, dass Aussagen wie „\( A = B \)“ oder „\( A \neq B \)“ für alle Mengen \( A \) und \( B \) gelten, ohne dass dies explizit angegeben werden muss.

\chapter{Die Zermelo-Fraenkel-Axiome}

Nachdem wir die grundlegenden Begriffe und Notationen eingeführt haben, wenden wir uns nun den Zermelo-Fraenkel-Axiomen zu, die das Fundament der modernen Mengenlehre bilden. Diese Axiome definieren, wie Mengen gebildet werden können und welche Eigenschaften sie besitzen.

\FormulaDefDeltaK[Begriff der Menge (Zermelo-Fraenkel-Axiome)]{\in \text{(zweistelliges Prädikat)}}{Menge}{
  \DeltaRow{\textbf{Axiome}}{}
  % — Notations-/Meta-Zeilen —
  % — Extensionalität —
  \DeltaRow{Extensionalität}
           {\forall x\, (x \in A \leftrightarrow x \in B) \vdash A = B}
           [\FormulaRefAutoFwd{\forall x\, (x \in A \leftrightarrow x \in B) \vdash A = B}]
  %
  % — Leere Menge —
  \DeltaRow{Leere Menge}
           {\exists O\;\bigl(\forall x\,(x \not\in O)\bigr)}
           [\FormulaRefAutoFwd{\exists O\;\bigl(\forall x\,(x \not\in O)\bigr)}]
  %
  % — Aussonderung (Schema) —
  \DeltaRow{Aussonderung}
           {\exists B\;\forall x\;\bigl(x \in B \;\leftrightarrow\; x \in A \,\land\, P(x)\bigr)}
           [\FormulaRefAutoFwd{\exists B\;\forall x\;\bigl(x \in B \;\leftrightarrow\; x \in A \,\land\, P(x)\bigr)}]
  %
  % — Paarmenge —
  \DeltaRow{Paarmenge}
           {\exists C\;\bigl(\forall x\,(x \in C \leftrightarrow x=A \lor x=B)\bigr) }
           [\FormulaRefAutoFwd{\exists C\;\bigl(\forall x\,(x \in C \leftrightarrow x=A \lor x=B)\bigr)}]
  %
  % — Vereinigung —
  \DeltaRow{Vereinigung}
    {\exists C\;\forall x\;\bigl(x \in C \;\leftrightarrow\;\exists B\in A\,(x \in B)\bigr)}
    [\FormulaRefAutoFwd{\exists C\;\forall x\;\bigl(x \in C \;\leftrightarrow\;\exists B\in A\,(x \in B)\bigr)}]
  %
  % — Regularität (Fundierung) —
  \DeltaRow{Regularität}
           {A \neq \varnothing \vdash \exists x \in A \,(x \cap A = \varnothing)}
           [\FormulaRefAutoFwd{A \neq \varnothing \vdash \exists x \in A \,(x \cap A = \varnothing)}]
  %
  % — Potenzmenge —
  \DeltaRow{Potenzmenge}
           {\exists B\forall x\bigl(x \in B \leftrightarrow x \subseteq A\bigr)}
           [\FormulaRefAutoFwd{\exists B\forall x\bigl(x \in B \leftrightarrow x \subseteq A\bigr)}]
  %
  % — Ersetzung (Schema) —
  \DeltaRow{Ersetzung}
           {\exists B\;\forall y\;\bigl( y\in B\;\leftrightarrow\; \exists x\in A\;y=F(x) \bigr)}
           [\FormulaRefAutoFwd{\exists B\;\forall y\;\bigl( y\in B\;\leftrightarrow\; \exists x\in A\;y=F(x) \bigr)}]
  %
  % — Auswahlaxiom (eine Form) —
  \DeltaRow{Auswahlaxiom}
           {\exists F\colon A \to B \, (G \circ F = \Id_A)}
           [\FormulaRefAutoFwd{\exists F\colon A \to B\,G\circ F = \Id_A}]
  %
  % — Unendlichkeit —
  \DeltaRow{Unendlichkeit}
           {\exists A\,\bigl(\varnothing\in A \land \forall x\in A\,(\,x\cup\{x\}\in A\,)\bigr)}
           [\FormulaRefAutoFwd{\exists A\,(\varnothing \in A \land \forall x \in A\,(x \cup \{x\} \in A))}]
%
    \DeltaRow{\textbf{Neue Symbole}}{}
    \DeltaRow{Mengen}{A\dsep B\dsep C\dsep O\dsep x}
    %%begin novalidate
    \DeltaRow{Leere Menge}{\varnothing}[\FormulaRefAutoFwd{\varnothing := \iota O\bigl(\forall x\,(x \not\in O)\bigr)}]
    %%end novalidate
    \DeltaPrem{Funktionen}{F\colon A\to B}[\FormulaRefAutoFwd{Funktion}]
    \DeltaPrem{\makecell[l]{Surjektive\\ Funktionen}}{G\colon B\sur A}[\FormulaRefAutoFwd{Surjektive Funktion}]
    \DeltaRow{Identität}{\Id_A\colon A\to A}[\FormulaRefAutoFwd{\Id_A\colon A\to A}]
    \DeltaRow{Element}{ }
}

\chapter{Extensionalität}

\section{Axiom der Extensionalität}

\FormulaAxiomDelta[Extensionalität]{\forall x\, (x \in A \leftrightarrow x \in B) \vdash A = B}%
{%
\DeltaRow{Mengen}{A\dsep B}% 
}%

\FormulaThmDelta[Extensionalität]{\forall x\, (x \in A \leftrightarrow x \in B) \eqvdash A = B}%
{%
\DeltaRow{Mengen}{A\dsep B}% 
}%
\begin{tabproofsplit}
    \proofpart{\(\vdash\)}
        \proofstep{}{\forall x\, (x \in A \leftrightarrow x \in B) \rightarrow A = B}{\FormulaRefAuto{\forall x\, (x \in A \leftrightarrow x \in B) \vdash A = B}}
    \closeproofpart
    \proofpart{\(\dashv\)}
        \proofstep{1}{A=B}{\rA}
        \proofstep{2}{x\in A}{\rA}
        \proofstep{1,2}{x\in B}{\rIE{1,2}}
        \proofstep{1}{x\in A\rightarrow x\in B}{\rRI{2,3}}
        \proofstep{5}{x\in B}{\rA}
        \proofstep{1,5}{x\in A}{\rIE{1,5}}
        \proofstep{1}{x\in B\rightarrow x\in A}{\rRI{5,6}}
        \proofstep{1}{x\in A\leftrightarrow x\in A}{\rLRI{4,7}}
        \proofstep{1}{\forall x\, (x\in A\leftrightarrow x\in A)}{\rUI{8}}
    \closeproofpart
\end{tabproofsplit}

\FormulaThmDelta[Ersetzung in der Elementbeziehung]{%
  a=b \dsep b\in A \vdash a\in A
}{
  \DeltaRow{Mengen}{A \dsep a \dsep b}
}
\begin{tabproof}
  \proofstep{1}{a=b}{\rA}
  \proofstep{2}{b\in A}{\rA}
  \proofstep{1}{b=a}{\FormulaRefAuto{a=b\vdash b=a}{1}}
  \proofstep{1,2}{a\in A}{\rIE{2,3}}
\end{tabproof}


\section{Ungleichheit von Mengen}

\FormulaThmAuto{
A \neq B \eqvdash \exists x (x \not\in A\land x\in B) \lor  \exists x (x \in A\land x\not\in B)
}
\begin{tabproofwide}
  \proofstepwide{A \neq B}{\leftrightarrow}{\neg(\forall x(x\in A\leftrightarrow x\in B))}{\FormulaRefAuto{P \leftrightarrow Q \eqvdash \neg P \leftrightarrow \neg Q}{\FormulaRefAuto{\forall x\, (x \in A \leftrightarrow x \in B) \eqvdash A = B}}}
  \proofstepwide{}{\leftrightarrow}{\exists x(x \not\in A\land x\in B)}{\multirow{2}{*}{\FormulaRefAuto{\neg\forall x(P(x)\leftrightarrow Q(x)) \eqvdash \exists x (P(x)\land \neg Q(x))\lor \exists x (Q(x)\land \neg P(x))}{1}}}
  \proofstepwide*{}{\lor}{\exists x(x \in A\land x\not\in B)}{}
  \proofstepwide{A \neq B}{\leftrightarrow}{\exists x(x \not\in A\land x\in B)}{\multirow{2}{*}{\rChain{1,2}}}
  \proofstepwide*{}{\lor}{\exists x(x \in A\land x\not\in B)}{}
\end{tabproofwide}

\FormulaThmAuto{x\in A, x\not\in B\vdash A\neq B}
\begin{tabproof}
  \proofstep{1}{x\in A}{\rA}
  \proofstep{2}{x\not\in B}{\rA}
  \proofstep{1}{\exists x(x\in A\land x\not\in B)}{\rEI{\rAI{1,2}}}
  \proofstep{1}{A\neq B}{\FormulaRefAuto{A \neq B \eqvdash \exists x (x \not\in A\land x\in B) \lor \exists x (x \in A\land x\not\in B)}{\rOIa{3}}}
\end{tabproof}


\chapter{Teilmengen}

\section{Definition der Teilmenge}

\FormulaDefDelta[Teilmenge]{ A \subseteq B := \forall x\,(x\in A \rightarrow x\in B) }%
{%
\DeltaRow{Mengen}{A\dsep B}%
}
\begin{remark}
    In Worten: \(A\) ist Teilmenge von \(B\).
\end{remark}

\FormulaDefDelta[Echte Teilmenge]{%
  A \subset B := A \subseteq B \land \neg(A=B)%
}{%
  \DeltaRow{Mengen}{A\dsep B}%
}
\begin{remark}
  In Worten: \(A\) ist eine echte Teilmenge von \(B\).
\end{remark}


\section{Grundlegende Eigenschaften}

\FormulaThmAuto{ A\subseteq B,\, x\in A \vdash x\in B }
\begin{tabproof}
  \proofstep{1}{ A\subseteq B }{ \rA }
  \proofstep{2}{ x\in A }{ \rA }
  \proofstep{1}{ x\in A \rightarrow x\in B }{ \rUE{ \FormulaRefAuto{ A \subseteq B := \forall x\,(x\in A \rightarrow x\in B) }{1} } }
  \proofstep{1,2}{ x\in B }{ \rRE{2,3} }
\end{tabproof}

\FormulaThmAuto{ A = B,\, x \in A \vdash x \in B }
\begin{tabproof}
  \proofstep{1}{ A = B }{ \rA }
  \proofstep{2}{ x \in A }{ \rA }
  \proofstep{1}{ x \in A \leftrightarrow x \in B }{ \rUE{ \FormulaRefAuto{ \forall x\, (x \in A \leftrightarrow x \in B) \eqvdash A = B } } }
  \proofstep{1,2}{ x \in B }{ \FormulaRefAuto{ P \leftrightarrow Q, P \vdash Q }{3,2} }
\end{tabproof}

\FormulaThmAuto{ a \in A,\; b \not\in A \vdash a \neq b }
\begin{tabproof}
  \proofstep{1}{ a \in A }{ \rA }
  \proofstep{2}{ b \not\in A }{ \rA }
  \proofstep{3}{ a = b }{ \rA }
  \proofstep{1,3}{ b \in A }{ \rIE{3,1} }
  \proofstep{1,2,3}{ \bot }{ \rAI{4,2} }
  \proofstep{1,2}{ a \neq b }{ \rCI{3,5} }
\end{tabproof}

\FormulaThmAuto{A\subseteq C,\, B\subseteq C,\, z\in A\lor z\in B \vdash z\in C}
\begin{tabproofwide}
  \proofstepwidestar[1]{A \subseteq C}{\rA}
  \proofstepwidestar[2]{B \subseteq C}{\rA}
  \proofstepwidestar[3]{z \in A \lor z \in B}{\rA}

  \proofstepwide[1]{z \in A}{\rightarrow}{z \in C}%
    {\rUE{\FormulaRefAuto{A \subseteq B := \forall x\,(x\in A \rightarrow x\in B)}{1}}}
  \proofstepwide[2]{z \in B}{\rightarrow}{z \in C}%
    {\rUE{\FormulaRefAuto{A \subseteq B := \forall x\,(x\in A \rightarrow x\in B)}{2}}}

  \proofstepwidestar[1,2,3]{z \in C}%
    {\FormulaRefAuto{P \rightarrow Q, R \rightarrow Q, P \lor R \vdash Q}{4,5,3}}
\end{tabproofwide}
\section{Ordnungsrelation}

\FormulaThmAuto[Reflexivität von Teilmengen]{ A \subseteq A }
\begin{tabproof}
  \proofstep{}{ x \in A \rightarrow x \in A }{ \FormulaRefAuto{ P \rightarrow P } }
  \proofstep{}{ \forall x(x \in A \rightarrow x \in A) }{ \rUI{1} }
  \proofstep{}{ A \subseteq A }{ \FormulaRefAuto{ A \subseteq B := \forall x\,(x \in A \rightarrow x \in B) }{2} }
\end{tabproof}


\FormulaThmAuto[Antisymmetrie von Teilmengen]{ A \subseteq B \land B \subseteq A \eqvdash A = B }
\begin{tabproofwide}
  \proofstepwide{A \subseteq B \land B \subseteq A}{\leftrightarrow}{\forall x(x \in A \rightarrow x \in B)}{\multirow{2}{*}{\FormulaRefAuto{ A \subseteq B := \forall x\,(x \in A \rightarrow x \in B) }}}
  \proofstepwide{}{ \land }{\forall x(x \in B \rightarrow x \in A)}{}
  \proofstepwide*{}{ \leftrightarrow }{\forall x(x \in A \leftrightarrow x \in B)}{\FormulaRefAuto{\forall x (P(x) \leftrightarrow Q(x)) \eqvdash \forall x (P(x) \rightarrow Q(x)) \land \forall x (Q(x) \rightarrow P(x))}{1}}
  \proofstepwide{}{ \leftrightarrow }{A = B}{\FormulaRefAuto{ \forall x\, (x \in A \leftrightarrow x \in B) \eqvdash A = B }{2}}
  \proofstepwide*{A \subseteq B \land B \subseteq A}{ \leftrightarrow }{A = B}{\rChain{1,3}}
\end{tabproofwide}

\FormulaThmAuto{ A=B\vdash A\subseteq B }
\begin{tabproof}
  \proofstep{1}{A=B}{\rA}
  \proofstep{1}{A\subseteq B \land B\subseteq A}{%
    \FormulaRefAuto{ A \subseteq B \land B \subseteq A \eqvdash A = B }{1}%
  }
  \proofstep{1}{A\subseteq B}{\rAEa{2}}
\end{tabproof}

\FormulaThmAuto{ A=B\vdash B\subseteq A }
\begin{tabproof}
  \proofstep{1}{A=B}{\rA}
  \proofstep{1}{A\subseteq B \land B\subseteq A}{%
    \FormulaRefAuto{ A \subseteq B \land B \subseteq A \eqvdash A = B }{1}%
  }
  \proofstep{1}{B\subseteq A}{\rAEb{2}}
\end{tabproof}

\FormulaThmAuto{ A \subseteq B, B \subseteq A \vdash A = B }
\begin{tabproof}
    \proofstep{1}{A\subseteq B}{\rA}
    \proofstep{2}{B\subseteq A}{\rA}
    \proofstep{1,2}{A\subseteq B\land B\subseteq A}{\rAI{1,2}}
    \proofstep{1,2}{A\subseteq B\land B\subseteq A}{\FormulaRefAuto{ A \subseteq B \land B \subseteq A \eqvdash A = B }{3}}
\end{tabproof}

\FormulaThmAuto[Transitivität von Teilmengen]{
A \subseteq B, B \subseteq C \vdash A \subseteq C
}
\begin{tabproof}
  \proofstep{1}{A \subseteq B}{\rA}
  \proofstep{2}{B \subseteq C}{\rA}
  \proofstep{1}{\forall x(x \in A \rightarrow x \in B)}{\FormulaRefAuto{A \subseteq B := \forall x(x \in A \rightarrow x \in B)}{1}}
  \proofstep{1}{\forall x(x \in B \rightarrow x \in C)}{\FormulaRefAuto{A \subseteq B := \forall x(x \in A \rightarrow x \in B)}{2}}
  \proofstep{1,2}{\forall x(x \in A \rightarrow x \in C)}{\FormulaRefAuto{\forall x(P(x) \rightarrow Q(x)), \forall x(Q(x) \rightarrow R(x)) \vdash \forall x(P(x) \rightarrow R(x))}{3,4}}
  \proofstep{1,2}{A \subseteq C}{\FormulaRefAuto{A \subseteq B := \forall x(x \in A \rightarrow x \in B)}{5}}
\end{tabproof}

\FormulaThmAuto[Rechtsverträglichkeit von \(\subseteq\) und \(=\)]{
A \subseteq B, B = C \vdash A \subseteq C
}
\begin{tabproof}
  \proofstep{1}{A \subseteq B}{\rA}
  \proofstep{2}{B = C}{\rA}
  \proofstep{3}{B \subseteq C}{\rAEa{\FormulaRefAuto{A \subseteq B \land B \subseteq A \eqvdash A = B}{2}}}
  \proofstep{4}{A \subseteq C}{\FormulaRefAuto{A \subseteq B, B \subseteq C \vdash A \subseteq C}{1,3}}
\end{tabproof}

\begin{remark}[Gemischte Kettenregel]
Auf Basis des vorangegangenen Theorems können \(\subseteq\) und \(=\) nun in einer \emph{Kette} \((\subseteq,=^{*})\) kombiniert werden,  da sie \emph{rechts-verträglich} sind. Ebenso ist \(=\) wegen \FormulaRefAuto{a = b \vdash b = a}{} außerdem Symmetrisch, was mit dem Stern in der Kette illustriert wird.
\end{remark}



\chapter{Leere Menge}
\section{Axiom der leeren Menge}

\FormulaAxiomAuto[Leere Menge]{\exists O\;\bigl(\forall x\,(x \not\in O)\bigr)}

\section{Definition der leeren Menge}


\FormulaThmAuto{ \exists! O\forall x (x \not\in O) }
\begin{tabproof}
  \proofstep{}{ \exists O\forall x (x \not\in O) }{ \FormulaRefAuto{\exists O\;\bigl(\forall x\,(x \not\in O)\bigr)} }
  \proofstep{2}{ \forall x (x \not\in O) }{ \rA }
  \proofstep{3}{ \forall x (x \not\in P) }{ \rA }
  \proofstep{2}{ \forall x (x \not\in O \lor x \in P) }{ \FormulaRefAuto{\forall x(F(x))\lor\forall x(G(x))\vdash\forall x(F(x)\lor G(x))} }
  \proofstep{3}{ \forall x (x \not\in P \lor x \in O) }{ \FormulaRefAuto{ \forall x(F(x))\lor\forall x(G(x))\vdash\forall x(F(x)\lor G(x))} }
  \proofstep{2,3}{ \forall x (x \in O \leftrightarrow x \in P) }{ \FormulaRefAuto{ \forall x (P(x) \leftrightarrow Q(x)) \eqvdash \forall x (\neg P(x) \lor Q(x)) \land \forall x (\neg Q(x) \lor P(x)) }{4,5} }
  \proofstep{2,3}{ O = P }{ \FormulaRefAuto{ \forall x\, (x \in A \leftrightarrow x \in B) \eqvdash A = B } }
  \proofstep{}{ \exists! O (\forall x (x \not\in O)) }{ \UEI{1,2,3,6} }
\end{tabproof}

%%begin novalidate
\FormulaDefAuto[Leere Menge]{\varnothing := \iota O\bigl(\forall x\,(x \not\in O)\bigr)}
%%end novalidate

\FormulaThmDelta[Leere Menge]{x \not\in \varnothing}%
{%
\DeltaRow{Mengen}{x}% 
}
\begin{tabproof}
  \proofstep{}{ \forall x\,\bigl(x \not\in O\bigr) }{\FormulaRefAuto{\varnothing := \iota O\bigl(\forall x\,(x \not\in O)\bigr)}}
  \proofstep{}{ x \not\in O }{\rUE{1}}
\end{tabproof}

\FormulaThmDelta{\forall x\, x\notin A\vdash A=\varnothing}{
\DeltaRow{Mengen}{x\dsep A}
}
\begin{tabproof}
    \proofstep{1}{\forall x\, x\notin A}{\rA}
    \proofstep{}{\forall x\, x\notin \varnothing}{\FormulaRefAuto{x \not\in \varnothing}}
    \proofstep{}{\exists! O\forall x (x \not\in O)}{\FormulaRefAuto{\exists! O\forall x (x \not\in O)}}
    \proofstep{}{A=\varnothing}{\FormulaRefAuto{\exists! x\,P(x),\; P(a),\; P(b) \vdash a = b}{3,2,1}}
    
\end{tabproof}

\section{Grundlegende Eigenschaften}


\FormulaThmAuto{\forall A\,(\varnothing\subseteq A)}
\begin{tabproof}
  \proofstep{}{x \not\in \varnothing}{\FormulaRefAuto{\varnothing := \iota O\bigl(\forall x\,(x \not\in O)\bigr)}{}}
  \proofstep{}{x \not\in A \rightarrow x \not\in \varnothing}{\FormulaRefAuto{Q \vdash P \rightarrow Q}{}}
  \proofstep{}{x \in \varnothing \rightarrow x \in A}{\FormulaRefAuto{P \rightarrow Q \eqvdash \neg Q \rightarrow \neg P}{2}}
  \proofstep{}{\forall x(x \in \varnothing \rightarrow x \in A)}{\rUI{3}}
  \proofstep{}{\varnothing \subseteq A}{\FormulaRefAuto{A \subseteq B := \forall x\,(x\in A \rightarrow x\in B)}{4}}
  \proofstep{}{\forall A(\varnothing \subseteq A)}{\rUI{5}}
\end{tabproof}

\FormulaThmDelta[Leere Menge ist Teilmenge]{%
  \varnothing\subseteq A
}{%
  \DeltaRow{Mengen}{A}%
}
\begin{tabproof}
  \proofstep{}{\forall A\,(\varnothing\subseteq A)}{\FormulaRefAuto{\forall A\,(\varnothing\subseteq A)}}
  \proofstep{}{\varnothing\subseteq A}{\rUE{1}}
\end{tabproof}


\FormulaThmDelta[Teilmenge der leeren Menge]{%
  A\subseteq\varnothing\vdash A=\varnothing
}{%
  \DeltaRow{Mengen}{A}%
}
\begin{tabproof}
  \proofstep{1}{A\subseteq\varnothing}{\rA}
  \proofstep{2}{x\in A}{\rA}
  \proofstep{1,2}{x\in\varnothing}{\FormulaRefAuto{A \subseteq B,\, x \in A \vdash x \in B}{1,2}}
  \proofstep{}{x\not\in\varnothing}{\FormulaRefAuto{x \not\in \varnothing}}
  \proofstep{1,2}{\bot}{\rBI{3,4}}
  \proofstep{1}{x\not\in A}{\rCI{2,5}}
  \proofstep{1}{\forall x\, (x\not\in A)}{\rUI{6}}
  \proofstep{1}{A=\varnothing}{\FormulaRefAuto{\forall x\, x\notin A\vdash A=\varnothing}{7}}
\end{tabproof}


\FormulaThmAuto{
A \subseteq B,\, \forall x \in B\,(x \not\in A) \vdash A = \varnothing
}
\begin{tabproof}
  \proofstep{1}{A \subseteq B}{\rA}
  \proofstep{2}{\forall x \in B\,(x \not\in A)}{\rA}
  \proofstep{3}{x \in A}{\rA}
  \proofstep{1,3}{x \in B}{\FormulaRefAuto{A \subseteq B,\, x \in A \vdash x \in B}{1,3}}
  \proofstep{2}{x \in B \rightarrow x \not\in A}{\rUE{2}}
  \proofstep{2}{x \in A \rightarrow x \not\in B}{\FormulaRefAuto{P \rightarrow Q \eqvdash \neg Q \rightarrow \neg P}{5}}
  \proofstep{2,3}{x \not\in B}{\rRE{3,6}}
  \proofstep{1,2,3}{\bot}{\rBI{4,7}}
  \proofstep{1,2}{x \not\in A}{\rCI{3,8}}
  \proofstep{1,2}{\forall x\,(x \not\in A)}{\rUI{9}}
  \proofstep{1,2}{A = \varnothing}{\FormulaRefAuto{\varnothing := \iota O\,(\forall x\,(x \not\in O))}{10}}
\end{tabproof}

\FormulaThmAuto{a \in A \vdash A \neq \varnothing}
\begin{tabproof}
  \proofstep{1}{a \in A}{\rA}
  \proofstep{}{a \not\in \varnothing}{\FormulaRefAuto{\varnothing := \iota O\,(\forall x\,(x \not\in O))}}
  \proofstep{1}{A \neq \varnothing}{\FormulaRefAuto{x \in A,\, x \not\in B \vdash A \neq B}{1,2}}
\end{tabproof}

\FormulaThmAuto{\exists x\,(x \in S) \vdash S \neq \varnothing}
\begin{tabproof}
  \proofstep{1}{\exists x\,(x \in S)}{\rA}
  \proofstep{2}{a \in S}{\rA}
  \proofstep{2}{S \neq \varnothing}{\FormulaRefAuto{a \in A \vdash A \neq \varnothing}{2}}
  \proofstep{1}{S \neq \varnothing}{\rEE{1,2,3}}
\end{tabproof}

\FormulaThmDelta{%
  A\neq\varnothing \eqvdash \exists x\,(x \in A)
}{
  \DeltaRow{Mengen}{x\dsep A}
}
\begin{tabproofsplit}
  \proofpart{\(\vdash\)}
    \proofstep{1}{A\neq\varnothing}{\rA}
    \proofstep{2}{\neg\exists x\,(x\in A)}{\rA}
    \proofstep{2}{\forall x\, x\notin A}{\FormulaRefAuto{\forall x(\neg P(x)) \eqvdash \neg\exists x (P(x))}{2}}
    \proofstep{2}{A=\varnothing}{\FormulaRefAuto{\forall x\, x\notin A\vdash A=\varnothing}{3}}
    \proofstep{1,2}{\bot}{\rBI{1,4}}
    \proofstep{1}{\exists x\,(x\in A)}{\rCE{2,5}}
  \closeproofpart
  \proofpart{\(\dashv\)}
    \proofstep{1}{\exists x\,(x \in A)}{\rA}
    \proofstep{2}{a \in A}{\rA}
    \proofstep{2}{A \neq \varnothing}{\FormulaRefAuto{a \in A \vdash A \neq \varnothing}{2}}
    \proofstep{1}{A \neq \varnothing}{\rEE{1,2,3}}
  \closeproofpart
\end{tabproofsplit}

\chapter{Aussonderung}

\section{Axiom der Aussonderung}

\FormulaAxiomDelta[Aussonderung]{\exists B\;\forall x\;\bigl(x \in B \;\leftrightarrow\; x \in A \,\land\, P(x)\bigr)}%
{%
\DeltaRow{Mengen}{A}% 
\DeltaRow{Einstellige Prädikate}{P}%
}%

\section{Definition der ausgesonderten Menge}

\FormulaThmDelta[Zur Eindeutigkeit]{ \forall x (x \in A \leftrightarrow P(x)), \forall x (x \in B \leftrightarrow P(x)) \vdash A = B }{%
\DeltaRow{Mengen}{A\dsep B}% 
\DeltaRow{Einstellige Prädikate}{P}%
}
\begin{tabproof}
  \proofstep{1}{ \forall x (x \in A \leftrightarrow P(x)) }{ \rA }
  \proofstep{2}{ \forall x (x \in B \leftrightarrow P(x)) }{ \rA }
  \proofstep{1,2}{ \forall x (x \in A \leftrightarrow x \in C) }{ \FormulaRefAuto{ \forall x(P(x)\leftrightarrow Q(x)), \forall x(R(x)\leftrightarrow Q(x))\vdash \forall x(P(x)\leftrightarrow R(x)) }{1,2} }
  \proofstep{1,2}{ A = B }{ \FormulaRefAuto{ \forall x\, (x \in A \leftrightarrow x \in B) \eqvdash A = B }{3} }
\end{tabproof}

\FormulaThmDelta[Eindeutigkeit der Komprehension]{ \exists A(\forall x(x \in A\leftrightarrow P(x)))\vdash \exists! A(\forall x(x \in A\leftrightarrow P(x))) }{%
\DeltaRow{Mengen}{A}% 
\DeltaRow{Einstellige Prädikate}{P}%
}
\begin{tabproof}
  \proofstep{1}{ \exists A(\forall x(x \in A\leftrightarrow P(x))) }{ \rA }
  \proofstep{2}{ \forall x(x \in A\leftrightarrow P(x)) }{ \rA }
  \proofstep{3}{ \forall x(x \in B\leftrightarrow P(x)) }{ \rA }
  \proofstep{2,3}{ A = B }{ \FormulaRefAuto{ \forall x (x \in A \leftrightarrow P(x)), \forall x (x \in B \leftrightarrow P(x)) \vdash A = B}{2,3} }
  \proofstep{1}{ \exists! A(\forall x(x \in A\leftrightarrow P(x))) }{ \UEI{1,2,3,4} }
\end{tabproof}

\FormulaThmDelta{%
  \forall x(P(x)\rightarrow x\in A)\vdash \exists B\bigl(\forall x(x\in B\leftrightarrow P(x))\bigr)
}{%
\DeltaRow{Mengen}{A\dsep B}%
\DeltaRow{Einstellige Prädikate}{P}%
}
\begin{tabproof}
  \proofstep{1}{ \forall x (P(x) \rightarrow x \in A) }{ \rA }
  \proofstep{1}{ \forall x\bigl((x \in A \land P(x)) \leftrightarrow P(x)\bigr) }{ \FormulaRefAuto{ \forall x(P(x)\rightarrow Q(x))\vdash \forall x((Q(x)\land P(x))\leftrightarrow P(x)) } }
  \proofstep{}{ \exists B\;\forall x\;\bigl(x \in B \leftrightarrow x \in A \land P(x)\bigr) }{ \FormulaRefAuto{ \exists B\;\forall x\;\bigl(x \in B \leftrightarrow x \in A \,\land\, P(x)\bigr) } }
  \proofstep{4}{ \forall x\bigl(x \in C \leftrightarrow x \in A \land P(x)\bigr) }{ \rA }
  \proofstep{1,4}{ \forall x\bigl(x \in C \leftrightarrow P(x)\bigr) }{ \FormulaRefAuto{ \forall x(P(x)\leftrightarrow Q(x)), \forall x(Q(x)\leftrightarrow R(x))\vdash \forall x(P(x)\leftrightarrow R(x)) }{4,2} }
  \proofstep{1,4}{ \exists B\bigl(\forall x(x \in B \leftrightarrow P(x))\bigr) }{ \rEI{5} }
  \proofstep{1}{ \exists B\bigl(\forall x(x \in B \leftrightarrow P(x))\bigr) }{ \rEE{3,4,6} }
\end{tabproof}

\FormulaThmDelta{%
  \forall x(P(x)\rightarrow x\in A)\vdash \exists! B(\forall x(x\in B\leftrightarrow P(x)))
}{%
\DeltaRow{Mengen}{A\dsep B}%
\DeltaRow{Einstellige Prädikate}{P}%
}
\begin{tabproof}
  \proofstep{1}{ \forall x(P(x)\rightarrow x\in A) }{ \rA }
  \proofstep{1}{ \exists B\bigl(\forall x(x\in B\leftrightarrow P(x))\bigr) }{ \FormulaRefAuto{ \forall x(P(x)\rightarrow x\in A)\vdash \exists B\bigl(\forall x(x\in B\leftrightarrow P(x))\bigr) } }
  \proofstep{1}{ \exists! B\bigl(\forall x(x\in B\leftrightarrow P(x))\bigr) }{ \FormulaRefAuto{ \exists A(\forall x(x \in A\leftrightarrow P(x)))\vdash \exists! A(\forall x(x \in A\leftrightarrow P(x))) } }
\end{tabproof}


%%begin novalidate
\FormulaDefDelta[Aussonderung]{\{x\in A \,\mid\, P(x)\} := \iota B\bigl(\forall x\,(x\in B \leftrightarrow (x\in A \land P(x)))\bigr)}%
{%
\DeltaRow{Mengen}{A}% 
\DeltaRow{Einstellige Prädikate}{P}%
}%
%%end novalidate

\FormulaThmDelta[Aussonderung]{%
  x \in \{u \in A \mid P(u)\} \eqvdash x \in A \land P(x)
}{%
  \DeltaRow{Mengen}{A\dsep x}%
  \DeltaRow{Einstellige Prädikate}{P}%
}%
\begin{tabproof}
  \proofstep{}{%
    \forall x\,\bigl(x\in \{u\in A \mid P(u)\} \leftrightarrow (x\in A \land P(x))\bigr)
  }{%
    \FormulaRefAuto{\{x\in A \,\mid\, P(x)\} := \iota B\bigl(\forall x\,(x\in B \leftrightarrow (x\in A \land P(x)))\bigr)}%
  }

  \proofstep{}{%
    x\in \{u\in A \mid P(u)\} \leftrightarrow (x\in A \land P(x))
  }{%
    \rUE{1}%
  }
\end{tabproof}


\FormulaThmDelta{x \in A\dsep P(x)\vdash x \in \{x \in A \mid P(x)\}}%
{%
\DeltaRow{Mengen}{A\dsep x}% 
\DeltaRow{Einstellige Prädikate}{P}%
}
\begin{tabproof}
    \proofstep{1}{ x \in A }{ \rA }
    \proofstep{2}{ P(x) }{ \rA }
    \proofstep{1,2}{ x \in A\land P(x) }{ \rAI{1,2} }
    \proofstep{1,2}{ x \in \{x \in A \mid P(x)\} }{ \FormulaRefAuto{x \in \{u \in A \mid P(u)\} \eqvdash x \in A \land P(x)}{3} }
\end{tabproof}

\FormulaThmDelta{x \in A\dsep P(x)\dsep B=\{x \in A \mid P(x)\}\vdash x \in B}
{%
\DeltaRow{Mengen}{A\dsep B\dsep x}% 
\DeltaRow{Einstellige Prädikate}{P}%
}
\begin{tabproof}
    \proofstep{1}{ x \in A }{ \rA }
    \proofstep{2}{ P(x) }{ \rA }
    \proofstep{3}{ B=\{x \in A \mid P(x)\} }{ \rA }
    \proofstep{1,2}{ x \in \{x \in A \mid P(x)\} }{ \FormulaRefAuto{x \in A\dsep P(x)\vdash x \in \{x \in A \mid P(x)\}}{1,2} }
    \proofstep{1,2,3}{ x \in B }{ \rIE{3,4}}
\end{tabproof}

\section{Grundlegende Eigenschaften}

\FormulaThmAuto{x \in \{x \in A \mid P(x)\}\vdash x\in A}
\begin{tabproof}
  \proofstep{1}{x \in \{x \in A \mid P(x)\}}{\rA}
  \proofstep{1}{x\in A\land P(x)}{\FormulaRefAuto{x \in \{u \in A \mid P(u)\} \eqvdash x \in A \land P(x)}{1}}
  \proofstep{1}{P(x)}{\rAEa{2}}
\end{tabproof}

\FormulaThmAuto{x \in A\dsep A=\{x \in B \mid P(x)\}\vdash x\in B}
\begin{tabproof}
  \proofstep{1}{x \in A}{\rA}
  \proofstep{2}{A=\{x \in B \mid P(x)\}}{\rA}
  \proofstep{1,2}{x\in \{x \in B \mid P(x)\}}{\rIE{2,1}}
  \proofstep{1,2}{x\in B}{\FormulaRefAuto{x \in \{x \in A \mid P(x)\}\vdash x\in A}{3}}
\end{tabproof}

\FormulaThmAuto{x \in \{x \in A \mid P(x)\}\vdash P(x)}
\begin{tabproof}
  \proofstep{1}{x \in \{x \in A \mid P(x)\}}{\rA}
  \proofstep{1}{x\in A\land P(x)}{\FormulaRefAuto{x \in \{u \in A \mid P(u)\} \eqvdash x \in A \land P(x)}{1}}
  \proofstep{1}{P(x)}{\rAEb{2}}
\end{tabproof}

\FormulaThmAuto{x \in A\dsep A=\{x \in B \mid P(x)\}\vdash P(x)}
\begin{tabproof}
  \proofstep{1}{x \in A}{\rA}
  \proofstep{2}{A=\{x \in B \mid P(x)\}}{\rA}
  \proofstep{1,2}{x\in \{x \in B \mid P(x)\}}{\rIE{2,1}}
  \proofstep{1,2}{P(x)}{\FormulaRefAuto{x \in \{x \in A \mid P(x)\}\vdash P(x)}{3}}
\end{tabproof}

\FormulaThmAuto{x \not\in \{x \in A \mid P(x)\}\eqvdash (x \not\in A \lor \neg P(x))}
\begin{tabproofwide}
  \proofstepwide{x \not\in \{x \in A \mid P(x)\}}{\leftrightarrow}{\neg(x \in A \land P(x))}{\FormulaRefAuto{P \leftrightarrow Q \eqvdash \neg P \leftrightarrow \neg Q}{\FormulaRefAuto{\{x\in A \mid P(x)\} := \iota B(\forall x\,(x\in B \leftrightarrow (x\in A \land P(x)) ))}}}
  \proofstepwide{\neg(x \in A \land P(x))}{\leftrightarrow}{x \not\in A \lor \neg P(x)}{\FormulaRefAuto{\neg(P \land Q) \eqvdash \neg P \lor \neg Q}{1}}
  \proofstepwide{x \not\in \{x \in A \mid P(x)\}}{\leftrightarrow}{x \not\in A \lor \neg P(x)}{\rChain{1,2}}
\end{tabproofwide}

\FormulaThmAuto{\{ x \in A \mid P(x) \} \subseteq A}
\begin{tabproof}
  \proofstep{1}{x \in \{ x \in A \mid P(x) \}}{\rA}
  \proofstep{1}{x \in A \land P(x)}{\FormulaRefAuto{\{x\in A \,\mid\, P(x)\} := \iota B(\forall x\,(x\in B \leftrightarrow (x\in A \land P(x))))}{1}}
  \proofstep{1}{x \in A}{\rAEa{2}}
  \proofstep{}{ \{ x \in A \mid P(x) \} \subseteq A }{\FormulaRefAuto{A \subseteq B := \forall x\,(x \in A \rightarrow x \in B)}{\rUI{\rRI{1,3}}}}
\end{tabproof}

\FormulaThmAuto{A=\{ x \in B \mid P(x) \}\vdash  A\subseteq B}
\begin{tabproof}
  \proofstep{1}{A=\{ x \in B \mid P(x) \}}{\rA}
  \proofstep{1}{\{x \in B \mid P(x)\}\subseteq B}{\FormulaRefAuto{\{ x \in A \mid P(x) \} \subseteq A}{1}}
\proofstep{1}{A\subseteq B}{\rIE{1,2}}
\end{tabproof}

\FormulaThmAuto{\forall x \in A\,(P(x)),\, y \in A \vdash P(y)}
\begin{tabproof}
  \proofstep{1}{\forall x \in A\,(P(x))}{\rA}
  \proofstep{2}{y \in A}{\rA}
  \proofstep{1}{y \in A \rightarrow P(y)}{\rUE{1}}
  \proofstep{1,2}{P(y)}{\rRE{3,2}}
\end{tabproof}

\FormulaThmAuto{\forall x \in M(P(x)) \eqvdash M = \{x \in M \mid P(x)\}}
\begin{tabproofsplit}
\proofpart{\(\vdash\)}
  \proofstep{1}{\forall x \in M(P(x))}{\rA}
  \proofstep{1}{x \in M \rightarrow P(x)}{\rUE{1}}
  \proofstep{1}{x \in M \leftrightarrow (x \in M \land P(x))}{\FormulaRefAuto{P \rightarrow Q \vdash P \leftrightarrow (P \land Q)}{2}}
  \proofstep{1}{x \in M \leftrightarrow x \in \{x \in M \mid P(x)\}}{\FormulaRefAuto{\{x\in A \,\mid\, P(x)\} := \iota B\bigl(\forall x\,(x\in B \leftrightarrow (x\in A \land P(x)))\bigr)}{3}}
  \proofstep{1}{\forall x (x \in M \leftrightarrow x \in \{x \in M \mid P(x)\})}{\rUI{4}}
  \proofstep{1}{M = \{x \in M \mid P(x)\}}{\FormulaRefAuto{\forall x\, (x \in A \leftrightarrow x \in B) \eqvdash A = B}{5}}
\closeproofpart

\proofpart{\(\dashv\)}
  \proofstep{1}{M = \{x \in M \mid P(x)\}}{\rA}
  \proofstep{2}{y \in M}{\rA}
  \proofstep{1,2}{y \in \{x \in M \mid P(x)\}}{\FormulaRefAuto{A = B,\, x \in A \vdash x \in B}{1,2}}
  \proofstep{1,2}{P(y)}{\rAEb{\FormulaRefAuto{\{x\in A \,\mid\, P(x)\} := \iota B\bigl(\forall x\,(x\in B \leftrightarrow (x\in A \land P(x)))\bigr)}{3}}}
  \proofstep{1}{y \in M \rightarrow P(y)}{\rRI{2,4}}
  \proofstep{1}{\forall x \in M(P(x))}{\rUI{5}}
\closeproofpart
\end{tabproofsplit}

\FormulaThmAuto{A\subseteq \{x\in B\mid P(x)\}\vdash \forall x\in A(P(x))}
\begin{tabproof}
\proofstep{1}{A\subseteq \{x\in B\mid P(x)\}}{\rA}
\proofstep{2}{x\in A}{\rA}
\proofstep{1,2}{x\in \{x\in B\mid P(x)\}}{\FormulaRefAuto{ A\subseteq B,\, x\in A \vdash x\in B }{1,2}}
\proofstep{1,2}{P(x)}{\rAEb{\FormulaRefAuto{x \in \{u \in A \mid P(u)\} \eqvdash x \in A \land P(x)}{3}}}
\proofstep{1,2}{\forall x\in A(P(x))}{\rUI{\rRI{2,4}}}
\end{tabproof}

\chapter{Russel Paradoxon und die universelle Menge}

\FormulaThmAuto[Russells Paradoxon in der ZF-Mengenlehre]{\neg \exists U \forall A (A \in U \leftrightarrow A \not\in A)}
\begin{tabproof}
\proofstep{1}{\exists U \forall A (A \in U \leftrightarrow A \notin A)}{\rA}
\proofstep{2}{\forall A (A \in U \leftrightarrow A \notin A)}{\rA}
\proofstep{2}{U \in U \leftrightarrow U \notin U}{\rUE{2}}
\proofstep{}{\neg(U \in U \leftrightarrow U \notin U)}{\FormulaRefAuto{\neg (P\leftrightarrow \neg P)}}
\proofstep{2}{\bot}{\rBI{3,4}}
\proofstep{1}{\bot}{\rEE{1,2,5}}
\proofstep{}{\neg\exists U \forall A (A \in U \leftrightarrow A \notin A)}{\rCI{1,6}}
\end{tabproof}



Angenommen, es gibt eine universelle Menge \( U \) in der ZF-Mengenlehre, dann führt dies aufgrund des nachstehenden Satzes zu einem Widerspruch. 


\FormulaThmAuto{\exists U \forall A (A \in U)\vdash \forall A(A\not\in A\leftrightarrow A\in U\land A\not\in A)}
\begin{tabproof}
\proofstep{1}{\exists U \forall A (A \in U)}{\rA}
\proofstep{2}{\forall A (A \in U)}{\rA}
\proofstep{2}{\forall A (A \not\in A\leftrightarrow A\in U\land A\notin A)}{\FormulaRefAuto{\forall x(P(x)) \vdash \forall x(Q(x)\leftrightarrow P(x)\land Q(x))}{2}}
\proofstep{1}{\forall A (A \not\in A\leftrightarrow A\in U\land A\notin A)}{\rEE{1,2,3}}
\end{tabproof}

\FormulaThmAuto{\neg \exists U \forall A (A \in U)}
\begin{tabproof}
\proofstep{1}{\exists U \forall A (A \in U)}{\rA}
\proofstep{1}{\forall A(A\not\in A\leftrightarrow A\in U\land A\not\in A)}{\FormulaRefAuto{\exists U \forall A (A \in U)\vdash \forall A(A\not\in A\leftrightarrow A\in U\land A\not\in A)}{1}}
\proofstep{1}{\exists B\forall A(A\in B\leftrightarrow A\in U\land A\notin A)}{\FormulaRefAuto{\exists B\;\forall x\;\bigl(x \in B \;\leftrightarrow\; x \in A \,\land\, P(x)\bigr)}{2}}
\proofstep{4}{\forall A(A\in B\leftrightarrow A\in U\land A\notin A)}{\rA}
\proofstep{1,4}{\forall A(A\in B\leftrightarrow A\notin A)}{\FormulaRefAuto{\forall x(P(x)\leftrightarrow Q(x)), \forall x(R(x)\leftrightarrow Q(x))\vdash \forall x(P(x)\leftrightarrow R(x))}{4,2}}
\proofstep{1,4}{\exists B\forall A(A\in B\leftrightarrow A\notin A)}{\rEI{5}}
\proofstep{}{\neg\exists B\forall A(A\in B\leftrightarrow A\notin A)}{\FormulaRefAuto{\neg \exists U \forall A (A \in U \leftrightarrow A \not\in A)}}
\proofstep{1,4}{\bot}{\rBI{6,7}}
\proofstep{1}{\bot}{\rEE{3,4,8}}
\proofstep{}{\neg \exists U \forall A (A \in U)}{\rCI{1,9}}
\end{tabproof}

\chapter{Schnittmengen}

\section{Definition der Schnittmenge}

\FormulaDefDelta[Schnitt]{A \cap B := \{ x \in A \mid x \in B \}}{%
\DeltaRow{Mengen}{A\dsep B}%
}

\section{Grundlegende Eigenschaften}

\FormulaThmAuto{x \in A \cap B \vdash x \in A}
\begin{tabproof}
\proofstep{1}{x \in A \cap B}{\rA}
\proofstep{1}{x \in A}{\rAEa{\FormulaRefAuto{\{x\in A \,\mid\, P(x)\} := \iota B\bigl(\forall x\,(x\in B \leftrightarrow (x\in A \land P(x)))\bigr)}{\FormulaRefAuto{A \cap B := \{ x \in A \mid x \in B \}}}}}
\end{tabproof}

\FormulaThmAuto{x \in A \cap B \vdash x \in B}
\begin{tabproof}
\proofstep{1}{x \in A \cap B}{\rA}
\proofstep{1}{x \in B}{\rAEb{\FormulaRefAuto{\{x\in A \,\mid\, P(x)\} := \iota B\bigl(\forall x\,(x\in B \leftrightarrow (x\in A \land P(x)))\bigr)}{\FormulaRefAuto{A \cap B := \{ x \in A \mid x \in B \}}}}}
\end{tabproof}

\FormulaThmAuto{x \in A \cap B \eqvdash x \in A \land x \in B}
\begin{tabproof}
\proofstepstar{}{x \in A \cap B \leftrightarrow x \in A \land x \in B}{\FormulaRefAuto{\{x\in A \,\mid\, P(x)\} := \iota B\bigl(\forall x\,(x\in B \leftrightarrow (x\in A \land P(x)))\bigr)}{\FormulaRefAuto{A \cap B := \{ x \in A \mid x \in B \}}}}
\end{tabproof}

\FormulaThmAuto{x \in A, x\in B \vdash x \in A\cap B}
\begin{tabproof}
\proofstep{1}{x \in A}{\rA}
\proofstep{2}{x \in B}{\rA}
\proofstep{1,2}{x \in A\land x\in B}{\rAI{1,2}}
\proofstep{1,2}{x \in A\cap B}{\FormulaRefAuto{x \in A \cap B \eqvdash x \in A \land x \in B}{3}}
\end{tabproof}

\FormulaThmAuto{x \in (A \cap B) \cap C \eqvdash (x \in A \land x \in B) \land x \in C}
\begin{tabproofwide}
  \proofstepwide{x \in (A \cap B)}{\leftrightarrow}{x \in A \land x \in B}{\FormulaRefAuto{x \in A \cap B \eqvdash x \in A \land x \in B}}
  \proofstepwide{x \in (A \cap B) \cap C}{\leftrightarrow}{x \in (A \cap B) \land x \in C}{\FormulaRefAuto{x \in A \cap B \eqvdash x \in A \land x \in B}}
  \proofstepwide{}{\leftrightarrow}{(x \in A \land x \in B) \land x \in C}{\rLRS{1}}
  \proofstepwide{x \in (A \cap B) \cap C}{\leftrightarrow}{(x \in A \land x \in B) \land x \in C}{\rChain{2,3}}
\end{tabproofwide}

\FormulaThmAuto{x \in A \cap (B \cap C) \eqvdash x \in A \land (x \in B \land x \in C)}
\begin{tabproofwide}
  \proofstepwide{x \in B \cap C}{\leftrightarrow}{x \in B \land x \in C}{\FormulaRefAuto{x \in A \cap B \eqvdash x \in A \land x \in B}}
  \proofstepwide{x \in A \cap (B \cap C)}{\leftrightarrow}{x \in A \land x \in (B \cap C)}{\FormulaRefAuto{x \in A \cap B \eqvdash x \in A \land x \in B}}
  \proofstepwide{}{\leftrightarrow}{x \in A \land (x \in B \land x \in C)}{\rLRS{1}}
  \proofstepwide{x \in A \cap (B \cap C)}{\leftrightarrow}{x \in A \land (x \in B \land x \in C)}{\rChain{2,3}}
\end{tabproofwide}


\FormulaThmAuto[Idempotenz des Schnitts]{A = A \cap A}
\begin{tabproofwide}
  \proofstepwide{x \in A \cap A}{\leftrightarrow}{x \in A}{\rAEa{\FormulaRefAuto{\{x\in A \mid P(x)\} := \iota B(\forall x\,(x\in B \leftrightarrow (x\in A \land P(x))))}{\FormulaRefAuto{A \cap B := \{ x \in A \mid x \in B \}}}}}
  \proofstepwide{A}{=}{A \cap A}{\FormulaRefAuto{\forall x\, (x \in A \leftrightarrow x \in B) \eqvdash A = B}{\rUI{1}}}
\end{tabproofwide}

\FormulaThmAuto[Kommutativität des Schnitts]{A \cap B = B \cap A}
\begin{tabproofwide}
  \proofstepwide{x \in A \cap B}{\leftrightarrow}{x \in A \land x \in B}{\FormulaRefAuto{x \in A \cap B \eqvdash x \in A \land x \in B}}
  \proofstepwide{}{ \leftrightarrow }{x \in B \land x \in A}{\FormulaRefAuto{P \land Q \vdash Q \land P}{1}}
  \proofstepwide{}{ \leftrightarrow }{x \in B \cap A}{\FormulaRefAuto{x \in A \cap B \eqvdash x \in A \land x \in B}{2}}
  \proofstepwide{x \in A \cap B}{\leftrightarrow}{x \in B \cap A}{\rChain{1,3}}
  \proofstepwide{A \cap B}{=}{B \cap A}{\FormulaRefAuto{\forall x\, (x \in A \leftrightarrow x \in B) \eqvdash A = B}{\rUI{4}}}
\end{tabproofwide}


\FormulaThmAuto[Assoziativität des Schnitts]{(A \cap B) \cap C = A \cap (B \cap C)}
\begin{tabproofwide}
  \proofstepwide{x \in (A \cap B) \cap C}{\leftrightarrow}{(x \in A \land x \in B) \land x \in C}{\FormulaRefAuto{x \in (A \cap B) \cap C \eqvdash (x \in A \land x \in B) \land x \in C}}
  \proofstepwide{}{\leftrightarrow}{x \in A \land (x \in B \land x \in C)}{\FormulaRefAuto{P \land (Q \land R) \eqvdash (P \land Q) \land R}{1}}
  \proofstepwide{}{\leftrightarrow}{x \in A \cap (B \cap C)}{\FormulaRefAuto{x \in A \cap (B \cap C) \eqvdash x \in A \land (x \in B \land x \in C)}{}}
  \proofstepwide{x \in (A \cap B) \cap C}{\leftrightarrow}{x \in A \cap (B \cap C)}{\rChain{1,3}}
  \proofstepwide{(A \cap B) \cap C}{=}{A \cap (B \cap C)}{\FormulaRefAuto{\forall x\, (x \in A \leftrightarrow x \in B) \eqvdash A = B}{\rUI{4}}}
\end{tabproofwide}


\FormulaThmAuto{A \cap B \subseteq A}
\begin{tabproof}
  \proofstep{1}{x \in A \cap B}{\rA}
  \proofstep{1}{x \in A}{\rAEa{\FormulaRefAuto{\{x\in A \,\mid\, P(x)\} := \iota B\bigl(\forall x\,(x\in B \leftrightarrow (x\in A \land P(x)))\bigr)}{\FormulaRefAuto{A \cap B := \{ x \in A \mid x \in B \}}{1}}}}
  \proofstep{}{A \cap B \subseteq A}{\FormulaRefAuto{ A \subseteq B := \forall x\,(x\in A \rightarrow x\in B) }{\rUI{\rRI{1,2}}}}
\end{tabproof}

\FormulaThmAuto{A \cap B \subseteq B}
\begin{tabproof}
  \proofstep{1}{x \in A \cap B}{\rA}
  \proofstep{1}{x \in B}{\rAEb{\FormulaRefAuto{\{x\in A \,\mid\, P(x)\} := \iota B\bigl(\forall x\,(x\in B \leftrightarrow (x\in A \land P(x)))\bigr)}{\FormulaRefAuto{A \cap B := \{ x \in A \mid x \in B \}}{1}}}}
  \proofstep{}{A \cap B \subseteq B}{\FormulaRefAuto{ A \subseteq B := \forall x\,(x\in A \rightarrow x\in B) }{\rUI{\rRI{1,2}}}}
\end{tabproof}

\FormulaThmAuto{A \subseteq B \vdash A \cap C \subseteq B \cap C}
\begin{tabproof}
  \proofstep{1}{A \subseteq B}{\rA}
  \proofstep{2}{x \in A \cap C}{\rA}
  \proofstep{2}{x \in A}{\FormulaRefAuto{x \in A \cap B \vdash x \in A}{2}}
  \proofstep{2}{x \in C}{\FormulaRefAuto{x \in A \cap B \vdash x \in B}{2}}
  \proofstep{1,2}{x \in B}{\FormulaRefAuto{A \subseteq B,\, x \in A \vdash x \in B}{1,3}}
  \proofstep{1,2}{x \in B \cap C}{\FormulaRefAuto{x \in A \cap B \eqvdash x \in A \land x \in B}{\rAI{5,4}}}
  \proofstep{1}{A \cap C \subseteq B \cap C}{\FormulaRefAuto{A \subseteq B := \forall x\,(x \in A \rightarrow x \in B)}{\rUI{\rRI{2,6}}}}
\end{tabproof}


\FormulaThmAuto{A \subseteq B \eqvdash A \cap B = A}
\begin{tabproofsplitwide}
  \proofpartwide{\(\vdash\)}
    \proofstepwidestar[1]{A \subseteq B}{\rA}
    \proofstepwidestar[]{A \cap B \subseteq A}{\FormulaRefAuto{A \cap B \subseteq A}}
    \proofstepwide{A}{=}{A \cap A}{\FormulaRefAuto{A = A \cap A}}
    \proofstepwide[1]{}{\subseteq}{A \cap B}{\FormulaRefAuto{A \subseteq B \vdash A \cap C \subseteq B \cap C}{1}}
    \proofstepwide[1]{A}{\subseteq}{A \cap B}{\rChain{3,4}}
    \proofstepwide[1]{A \cap B}{=}{A}{\FormulaRefAuto{A \subseteq B \land B \subseteq A \eqvdash A = B}{\rAI{2,5}}}
  \closeproofpartwide

  \proofpartwide{\(\dashv\)}
    \proofstepwidestar[1]{A \cap B = A}{\rA}
    \proofstepwide[1]{A}{\subseteq}{A \cap B}{\rAEb{\FormulaRefAuto{A \subseteq B \land B \subseteq A \eqvdash A = B}{1}}}
    \proofstepwide[1]{A \cap B}{\subseteq}{B}{\FormulaRefAuto{A \cap B \subseteq B}}
    \proofstepwide[1]{A}{\subseteq}{B}{\FormulaRefAuto{A \subseteq B, B \subseteq C \vdash A \subseteq C}{2,3}}
  \closeproofpartwide
\end{tabproofsplitwide}

\FormulaThmAuto{B \subseteq A \eqvdash A \cap B = B}
\begin{tabproof}
  \proofstep{}{B \subseteq A \leftrightarrow B \cap A = B}{\FormulaRefAuto{A \subseteq B \eqvdash A \cap B = A}}
  \proofstep{}{B \cap A = A \cap B}{\FormulaRefAuto{A \cap B = B \cap A}}
  \proofstep{}{B \subseteq A \leftrightarrow A \cap B = B}{\rLRS{2,1}}
\end{tabproof}

\FormulaThmAuto{\varnothing \cap A = \varnothing}
\begin{tabproof}
  \proofstep{}{ \varnothing \subseteq A }{ \rUE{\FormulaRefAuto{\forall A\,(\varnothing \subseteq A)} } }
  \proofstep{}{ \varnothing \cap A = \varnothing }{ \FormulaRefAuto{ A \subseteq B \eqvdash A \cap B = A }{1} }
\end{tabproof}


\FormulaThmAuto{A \cap \varnothing = \varnothing}
\begin{tabproof}
  \proofstep{}{ \varnothing \subseteq A }{ \rUE{\FormulaRefAuto{\forall A\,(\varnothing \subseteq A)} } }
  \proofstep{}{ A \cap \varnothing = \varnothing }{ \FormulaRefAuto{ B \subseteq A \eqvdash A \cap B = B }{1} }
\end{tabproof}

\FormulaThmDelta{%
  X=U \dsep Y=V \dsep U\cap V=W \vdash X\cap Y=W%
}{
  \DeltaRow{Mengen}{U\dsep V\dsep W\dsep X\dsep Y}
}
\begin{tabproof}
  \proofstep{1}{X=U}{\rA}
  \proofstep{2}{Y=V}{\rA}
  \proofstep{3}{U\cap V=W}{\rA}
  \proofstep{1}{U=X}{\FormulaRefAuto{a=b \vdash b=a}{1}}
  \proofstep{1}{X\cap V=W}{\rIE{4,3}}
  \proofstep{2}{V=Y}{\FormulaRefAuto{a=b \vdash b=a}{2}}
  \proofstep{1,2}{X\cap Y=W}{\rIE{6,5}}
\end{tabproof}

\FormulaThmAuto{A \cap B = \varnothing,\ x \in A \vdash x \notin B}
\begin{tabproof}
  \proofstep{1}{A \cap B = \varnothing}{\rA}
  \proofstep{2}{x \in A}{\rA}
  \proofstep{3}{x \in B}{\rA}
  \proofstep{2,3}{x \in A \cap B}{\FormulaRefAuto{x \in A \cap B \eqvdash x \in A \land x \in B}{\rAI{2,3}}}
  \proofstep{1,2,3}{x \in \varnothing}{\rIE{1,4}}
  \proofstep{}{x \notin \varnothing}{\rUE{\FormulaRefAuto{\varnothing := \iota O\bigl(\forall x\,(x \not\in O)\bigr)}{}}}
  \proofstep{}{ \bot }{\rBI{5,6}}
  \proofstep{1,2}{x \notin B}{\rCE{1,2}}
\end{tabproof}

\FormulaThmAuto{A \cap B = \varnothing,\ x \in B \vdash x \notin A}
\begin{tabproof}
  \proofstep{1}{A \cap B = \varnothing}{\rA}
  \proofstep{2}{x \in B}{\rA}
  \proofstep{}{B \cap A = A \cap B}{\FormulaRefAuto{A \cap B = B \cap A}}
  \proofstep{1}{B \cap A = \varnothing}{\rIE{1,3}}
  \proofstep{1,2}{x \notin A}{\FormulaRefAuto{A \cap B = \varnothing,\ x \in A \vdash x \notin B}{4,2}}
\end{tabproof}


\section{Der unendliche Schnitt}

In diesem Abschnitt sei \( P \) ein Prädikat, das einer Menge \( A \) eine Eigenschaft zuweist.

\FormulaThmAuto{P(C) \vdash \{ x \in B \mid \forall A (P(A) \rightarrow x \in A) \} \subseteq \{ x \in C \mid \forall A (P(A) \rightarrow x \in A) \}}
\begin{notation*}
Wir bezeichnen mit \( I_B := \{ x \in B \mid \forall A (P(A) \rightarrow x \in A) \} \) und entsprechend \( I_C := \{ x \in C \mid \forall A (P(A) \rightarrow x \in A) \} \).
\end{notation*}
\begin{tabproof}
  \proofstep{1}{P(C)}{\rA}
  \proofstep{2}{x \in I_B}{\rA}
  \proofstep{2}{\forall A (P(A) \rightarrow x \in A)}{\rAEb{\FormulaRefAuto{\{x\in A \,\mid\, P(x)\} := \iota B\bigl(\forall x\,(x\in B \leftrightarrow (x\in A \land P(x)))\bigr)}{2}}}
  \proofstep{2}{P(C) \rightarrow x \in C}{\rUE{4}}
  \proofstep{1,2}{x \in C}{\rRE{1,4}}
  \proofstep{1,2}{x \in I_C}{\FormulaRefAuto{\{x\in A \,\mid\, P(x)\} := \iota B\bigl(\forall x\,(x\in B \leftrightarrow (x\in A \land P(x)))\bigr)}{\rAI{5,3}}}
  \proofstep{1}{I_B \subseteq I_C}{\FormulaRefAuto{A \subseteq B := \forall x\,(x \in A \rightarrow x \in B)}{\rUI{\rRI{2,6}}}}
\end{tabproof}

\FormulaThmDeltaR{%
  \begin{aligned}[t]
    &P(B),\,P(C)\ \vdash\\
    &\{\,x \in B \mid \forall A\,(P(A) \rightarrow x \in A)\,\}
      \;=\;
      \{\,x \in C \mid \forall A\,(P(A) \rightarrow x \in A)\,\}
  \end{aligned}%
}{P(B),\, P(C) \vdash \{ x \in B \mid \forall A (P(A) \rightarrow x \in A) \} = \{ x \in C \mid \forall A (P(A) \rightarrow x \in A) \}}{%
  \DeltaRow{Mengen}{A\dsep B\dsep C\dsep x}%
  \DeltaRow{Einstellige Prädikate}{P}%
}
\begin{notation*}
Wir bezeichnen mit \( I_B := \{ x \in B \mid \forall A (P(A) \rightarrow x \in A) \} \) und entsprechend \( I_C := \{ x \in C \mid \forall A (P(A) \rightarrow x \in A) \} \).
\end{notation*}
\begin{tabproof}
  \proofstep{1}{P(B)}{\rA}
  \proofstep{2}{P(C)}{\rA}
  \proofstep{2}{I_B \subseteq I_C}{\FormulaRefAuto{P(C) \vdash \{ x \in B \mid \forall A (P(A) \rightarrow x \in A) \} \subseteq \{ x \in C \mid \forall A (P(A) \rightarrow x \in A) \}}{2}}
  \proofstep{1}{I_C \subseteq I_B}{\FormulaRefAuto{P(C) \vdash \{ x \in B \mid \forall A (P(A) \rightarrow x \in A) \} \subseteq \{ x \in C \mid \forall A (P(A) \rightarrow x \in A) \}}{1}}
  \proofstep{1,2}{I_B = I_C}{\FormulaRefAuto{A \subseteq B \land B \subseteq A \eqvdash A = B}{3,4}}
\end{tabproof}

\FormulaThmAuto{\exists C\forall B\Bigl(P(B)\rightarrow C = \{ x \in B \mid \forall A (P(A) \rightarrow x \in A) \}\Bigr)}
\begin{notation*}
Wir bezeichnen mit \( I_B := \{ x \in B \mid \forall A (P(A) \rightarrow x \in A) \} \) und entsprechend \( I_D := \{ x \in D \mid \forall A (P(A) \rightarrow x \in A) \} \).
\end{notation*}
\begin{tabproofsplit}
  \proofpart{Fall 1: \( \exists D(P(D)) \vdash \exists C\forall B\bigl(P(B)\rightarrow C = I_B\bigr) \)}
    \proofstep{1}{\exists D(P(D))}{\rA}
    \proofstep{2}{P(D)}{\rA}
    \proofstep{3}{P(B)}{\rA}
    \proofstep{2,3}{I_D = I_B}{\FormulaRefAuto{P(B),\, P(C) \vdash \{ x \in B \mid \forall A (P(A) \rightarrow x \in A) \} = \{ x \in C \mid \forall A (P(A) \rightarrow x \in A) \}}}
    \proofstep{2}{\exists C\,\forall B\,\bigl(P(B) \rightarrow C = I_B\bigr)}{\rEI{\rUI{\rRI{3,4}}}}
  \closeproofpart

  \proofpart{Fall 2: \( \forall D(\neg P(D)) \vdash \exists C\forall B\bigl(P(B)\rightarrow C = I_B\bigr) \)}
    \proofstep{1}{\forall D(\neg P(D))}{\rA}
    \proofstep{1}{\forall B\,\bigl(\neg P(B) \lor C = I_B\bigr)}{\FormulaRefAuto{\forall x(F(x)) \lor \forall x(G(x)) \vdash \forall x(F(x) \lor G(x))}}
    \proofstep{1}{\forall B\,\bigl(P(B) \rightarrow C = I_B\bigr)}{\rLRS{\FormulaRefAuto{\neg(P \lor Q) \eqvdash \neg P \land \neg Q}{}, 2}}
    \proofstep{1}{\exists C\,\forall B\,\bigl(P(B) \rightarrow C = I_B\bigr)}{\rEI{3}}
  \closeproofpart

  \proofpart{Fallunterscheidung über das klassische Prinzip \( P \lor \neg P \)}
    \proofstep{}{ \exists D(P(D)) \lor \forall D(\neg P(D)) }{\rLRS{\FormulaRefAuto{\forall x(\neg P(x)) \eqvdash \neg \exists x(P(x))}{}, \FormulaRefAuto{P \lor \neg P}}}
    \proofstep{}{ \exists C\,\forall B\,\bigl(P(B) \rightarrow C = I_B\bigr) }{\rOE{1,1,5,1,4}}
  \closeproofpart
\end{tabproofsplit}


\FormulaThmDeltaR{
  \begin{aligned}
    &P(B_0),\;
    \forall B\Bigl(P(B)\rightarrow C= \{ x \in B \mid \forall A (P(A) \rightarrow x \in A) \}\Bigr),\\
    &\forall B\Bigl(P(B)\rightarrow D= \{ x \in B \mid \forall A (P(A) \rightarrow x \in A) \}\Bigr)
    \vdash C=D
  \end{aligned}
}{P(B_0),\forall B\bigl(P(B)\rightarrow C= \{ x \in B \mid \forall A (P(A) \rightarrow x \in A) \}\bigr),\, \forall B\bigl(P(B)\rightarrow D= \{ x \in B \mid \forall A (P(A) \rightarrow x \in A) \}\bigr) \vdash C=D}{}
\begin{notation*}
Wir bezeichnen mit \( I_B := \{ x \in B \mid \forall A (P(A) \rightarrow x \in A) \} \).
\end{notation*}
\begin{tabproof}
    \proofstep{1}{\forall B\bigl(P(B)\rightarrow C=I_B\bigr)}{\rA}
    \proofstep{2}{\forall B\bigl(P(B)\rightarrow D=I_B\bigr)}{\rA}
    \proofstep{3}{P(B_0)}{\rA}
    \proofstep{1}{P(B_0) \rightarrow C=I_{B_0}}{\rUE{1}}
    \proofstep{1,3}{C = I_{B_0}}{\rRE{3,4}}
    \proofstep{2}{P(B_0) \rightarrow D=I_{B_0}}{\rUE{2}}
    \proofstep{2,3}{D = I_{B_0}}{\rRE{3,6}}
    \proofstep{1,2,3}{C = D}{\FormulaRefAuto{a = b,\, c = b \vdash a = c}{5,7}}
\end{tabproof}


\FormulaThmAuto{P(D)\vdash \exists! C\forall B(P(B)\rightarrow C= \{ x \in B \mid \forall A (P(A) \rightarrow x \in A) \})}
\begin{notation*}
Es sei \( I_B := \{ x \in B \mid \forall A (P(A) \rightarrow x \in A) \} \).
\end{notation*}
\begin{tabproof}
  \proofstep{}{ \exists C\forall B\bigl(P(B)\rightarrow C=I_B\bigr) }{ \FormulaRefAuto{\exists C\forall B\Bigl(P(B)\rightarrow C = \{ x \in B \mid \forall A (P(A) \rightarrow x \in A) \}\Bigr)} }
  \proofstep{2}{\forall B\bigl(P(B)\rightarrow C=I_B\bigr)}{\rA}
  \proofstep{3}{\forall B\bigl(P(B)\rightarrow D=I_B\bigr)}{\rA}
  \proofstep{4}{P(D)}{\rA}
  \proofstep{2,3,4}{C = D}{\FormulaRefAuto{P(B_0),\forall B\bigl(P(B)\rightarrow C= \{ x \in B \mid \forall A (P(A) \rightarrow x \in A) \}\bigr),\, \forall B\bigl(P(B)\rightarrow D= \{ x \in B \mid \forall A (P(A) \rightarrow x \in A) \}\bigr) \vdash C=D}{4,2,3}}
  \proofstep{4}{\exists! C\,\forall B\bigl(P(B)\rightarrow C=I_B\bigr)}{\UEI{1,2,3,5}}
\end{tabproof}
%%begin novalidate
\FormulaDefAuto[Der unendliche Schnitt]{\exists A\,P(A)\rightarrow \bigcap_{P(B)} B := \iota C\,\forall D\Bigl(P(D)\rightarrow C = \{ x \in D \mid \forall A (P(A)\rightarrow x \in A) \}\Bigr)}
%%end novalidate
% Gleichbedeutende Notation mittels Mengenschreibweise
% (nur eingeführt, wenn der Schnitt definiert ist)
\FormulaDefAuto[Notationserweiterung]{\exists A\,P(A)\rightarrow\bigcap \{\,B \mid P(B)\,\} := \bigcap_{P(B)} B}


\FormulaThmAuto{P(A) \vdash \bigcap_{P(B)} B = \{ x \in A \mid \forall D (P(D) \rightarrow x \in D) \}}
\begin{notation*}
Wir bezeichnen mit \( I_A := \{ x \in A \mid \forall A (P(A) \rightarrow x \in A) \} \) und entsprechend \( I_D := \{ x \in D \mid \forall A (P(A) \rightarrow x \in A) \} \).
\end{notation*}
\begin{tabproof}
  \proofstep{1}{P(A)}{\rA}
  \proofstep{1}{\exists M(P(M))}{\rEI{1}}
  \proofstep{1}{\forall D\Bigl(P(D) \rightarrow \bigcap_{P(B)} B = I_D\Bigr)}{\FormulaRefAuto{\exists A\,P(A)\rightarrow \bigcap_{P(B)} B := \iota C\,\forall D\bigl(P(D)\rightarrow C = \{ x \in D \mid \forall A (P(A)\rightarrow x \in A) \}\bigr)}{2}}
  \proofstep{1}{P(A)\rightarrow \bigcap_{P(B)} B = I_A}{\rUE{3}}
  \proofstep{1}{\bigcap_{P(B)} B = I_A}{\rRE{4,1}}
\end{tabproof}

\FormulaThmAuto{P(A) \vdash x \in \bigcap_{P(B)} B\leftrightarrow x \in \{\,x \in A \mid \forall D\,(P(D) \rightarrow x \in D)\,\}}
\begin{notation*}
Wir bezeichnen mit \( I_A := \{ x \in A \mid \forall A (P(A) \rightarrow x \in A) \} \).
\end{notation*}
\begin{tabproof}
  \proofstep{1}{P(A)}{\rA}
  \proofstep{1}{x\in \bigcap_{P(B)} B\leftrightarrow x\in I_A}{\rUE{\FormulaRefAuto{\forall x\, (x \in A \leftrightarrow x \in B) \eqvdash A = B}{\FormulaRefAuto{P(A) \vdash \bigcap_{P(B)} B = \{ x \in A \mid \forall D\, (P(D) \rightarrow x \in D) \}}{1}}}}
\end{tabproof}

\FormulaThmAuto{P(A) \vdash x \in \bigcap_{P(B)} B \leftrightarrow \forall C\, (P(C) \rightarrow x \in C)}
\begin{notation*}
Wir bezeichnen mit \( I_A := \{ x \in A \mid \forall A (P(A) \rightarrow x \in A) \} \).    
\end{notation*}
\begin{tabproofwide}
  \proofstepwidestar[1]{P(A)}{\rA}
  \proofstepwidestar[2]{\forall C\, (P(C)\rightarrow x\in C)}{\rA}
  \proofstepwidestar[]{\forall C\, (P(C)\rightarrow x\in C) \rightarrow x\in A}{\rRI{2,\FormulaRefAuto{P(a), \forall x\, (P(x) \rightarrow Q(x)) \vdash Q(a)}{1,2}}}
  \proofstepwide[1]{x\in \bigcap_{P(B)} B}{\leftrightarrow}{x\in I_A}{\FormulaRefAuto{P(A) \vdash x \in \bigcap_{P(B)} B\leftrightarrow x \in \{\,x \in A \mid \forall D\,(P(D) \rightarrow x \in D)\,\}}{1}}

  % ---- hier die aufgesplittete Zeile (2 Zeilen) ----
  \proofstepwide[1]{}{\leftrightarrow}{x\in A}{\multirow{2}{*}{\FormulaRefAuto{x \in \{u \in A \mid P(u)\} \eqvdash x \in A \land P(x)}{4}}}
  \proofstepwide*{}{\land}{\forall C\, (P(C)\rightarrow x\in C)}{}
  % --------------------------------------------------

  \proofstepwide[1]{}{\leftrightarrow}{\forall C\, (P(C)\rightarrow x\in C)}{\FormulaRefAuto{P \rightarrow Q \vdash P \leftrightarrow (Q \land P)}{3}}
  \proofstepwide[1]{x\in \bigcap_{P(B)} B}{\leftrightarrow}{\forall C\, (P(C)\rightarrow x\in C)}{\rChain{4,6}}
\end{tabproofwide}

\FormulaThmAuto{\exists A(P(A)) \vdash x \in \bigcap_{P(B)} B \leftrightarrow \forall C\, (P(C) \rightarrow x \in C)}
\begin{tabproof}
    \proofstep{1}{\exists A(P(A))}{\rA}
    \proofstep{2}{P(A)}{\rA}
    \proofstep{2}{x \in \bigcap_{P(B)} B \leftrightarrow \forall C\, (P(C) \rightarrow x \in C)}{\FormulaRefAuto{P(A) \vdash x \in \bigcap_{P(B)} B \leftrightarrow \forall C\, (P(C) \rightarrow x \in C)}{2}}
    \proofstep{1}{x \in \bigcap_{P(B)} B \leftrightarrow \forall C\, (P(C) \rightarrow x \in C)}{\rEE{1,2,3}}
\end{tabproof}



\FormulaThmAuto{P(C)\vdash \bigcap_{P(A)} A \subseteq C}
\begin{tabproof}
  \proofstep{1}{P(C)}{\rA}
  \proofstep{2}{x \in \bigcap_{P(A)} A}{\rA}
  \proofstep{1,2}{\forall A\,\bigl(P(A)\rightarrow x\in A\bigr)}{\FormulaRefAuto{P \leftrightarrow Q, P \vdash Q}{\FormulaRefAuto{P(A) \vdash x \in \bigcap_{P(B)} B \leftrightarrow \forall C\, (P(C) \rightarrow x \in C)}{1},\,2}}
  \proofstep{1,2}{x \in C}{\FormulaRefAuto{P(a), \forall x\,\bigl(P(x) \rightarrow Q(x)\bigr) \vdash Q(a)}{1,3}}
  \proofstep{1}{\forall x\,\bigl(x\in \bigcap_{P(A)} A \rightarrow x\in C\bigr)}{\rUI{\rRI{2,4}}}
  \proofstep{1}{\bigcap_{P(A)} A \subseteq C}{\FormulaRefAuto{A \subseteq B := \forall x\,\bigl(x\in A \rightarrow x\in B\bigr)}{5}}
\end{tabproof}

\chapter{Paarmenge}

\section{Axiom der Paarmenge}

\FormulaAxiomDelta[Paarmenge]{\exists C\;\bigl(\forall x\,(x \in C \leftrightarrow x=A \lor x=B)\bigr) }{%
\DeltaRow{Mengen}{A\dsep B}%
}

\section{Definition der Paarmenge}
%%begin novalidate
\FormulaDefDelta[Paarmenge]{\{A,B\} := \iota C\Bigl(\forall x\;\bigl(x \in C \;\leftrightarrow\; x = A \lor x = B\bigr)\Bigr)}{%
\DeltaRow{Mengen}{A\dsep B}%
}
%%end novalidate
\FormulaThmDelta{x \in \{A,B\}\;\eqvdash\;(x = A \lor x = B)}{%
\DeltaRow{Mengen}{x\dsep A\dsep B}%
}
\begin{tabproof}
  \proofstep{}{ \forall x\;\bigl(x \in C \;\leftrightarrow\; x = A \lor x = B\bigr) }{\FormulaRefAuto{\{A,B\} := \iota C\Bigl(\forall x\;\bigl(x \in C \;\leftrightarrow\; x = A \lor x = B\bigr)\Bigr)}}
  \proofstep{}{ x \in C \;\leftrightarrow\; x = A \lor x = B }{\rUE{1}}
\end{tabproof}

\FormulaDefDelta[Geordnetes Paar]{(a, b) := \{ \{ a \}, \{ a, b \} \}}{
\DeltaRow{Mengen}{a\dsep b}%
}

\section{Grundlegende Eigenschaften}

\FormulaThmAuto{x \notin \{a,b\} \eqvdash x \neq a \land x \neq b}
\begin{tabproofwide}
  \proofstepwide{x \notin \{a,b\}}{\leftrightarrow}{\neg(x = a \lor x = b)}%
    {\FormulaRefAuto{\{A,B\} := \iota C\Bigl(\forall x\;\bigl(x \in C \;\leftrightarrow\; x = A \lor x = B\bigr)\Bigr)}}
  \proofstepwide{}{\leftrightarrow}{x \neq a \land x \neq b}%
    {\FormulaRefAuto{\neg(P \lor Q) \eqvdash \neg P \land \neg Q}{1}}
  \proofstepwide{x \notin \{a,b\}}{\leftrightarrow}{x \neq a \land x \neq b}%
    {\rChain{1,2}}
\end{tabproofwide}

\FormulaThmAuto{a \in \{a,b\}}
\begin{tabproof}
  \proofstep{}{a = a}{\rIE{}}
  \proofstep{}{a = a \lor a = b}{\rOIa{1}}
  \proofstep{}{a \in \{a,b\}}{\FormulaRefAuto{\{A,B\} := \iota C\Bigl(\forall x\,\bigl(x \in C \leftrightarrow x = A \lor x = B\bigr)\Bigr)}{2}}
\end{tabproof}

\FormulaThmAuto{b \in \{a,b\}}
\begin{tabproof}
  \proofstep{}{b = b}{\rIE{}}
  \proofstep{}{b = a \lor b = b}{\rOIb{1}}
  \proofstep{}{b \in \{a,b\}}{\FormulaRefAuto{\{A,B\} := \iota C\Bigl(\forall x\,\bigl(x \in C \leftrightarrow x = A \lor x = B\bigr)\Bigr)}{2}}
\end{tabproof}

\FormulaThmAuto{\{a,b\} = \{b,a\}}
\begin{tabproofwide}
  \proofstepwide{x \in \{a,b\}}{\leftrightarrow}{x = a \lor x = b}%
    {\FormulaRefAuto{\{A,B\} := \iota C\Bigl(\forall x\,\bigl(x \in C \leftrightarrow x = A \lor x = B\bigr)\Bigr)}{}}
  \proofstepwide{}{\leftrightarrow}{x = b \lor x = a}%
    {\FormulaRefAuto{P \lor Q \vdash Q \lor P}{1}}
  \proofstepwide{}{\leftrightarrow}{x \in \{b,a\}}%
    {\FormulaRefAuto{\{A,B\} := \iota C\Bigl(\forall x\,\bigl(x \in C \leftrightarrow x = A \lor x = B\bigr)\Bigr)}{2}}
  \proofstepwide{x \in \{a,b\}}{\leftrightarrow}{x \in \{b,a\}}%
    {\rChain{1,3}}
  \proofstepwide{\{a,b\}}{=}{\{b,a\}}%
    {\FormulaRefAuto{\forall x\, (x \in A \leftrightarrow x \in B) \eqvdash A = B}{\rUI{4}}}
\end{tabproofwide}

\FormulaThmAuto{\{a,b\} \neq \varnothing}
\begin{tabproof}
  \proofstep{}{a \in \{a,b\}}{\FormulaRefAuto{a \in \{a,b\}}{}}
  \proofstep{}{\{a,b\} \neq \varnothing}{\FormulaRefAuto{\exists x\,(x \in S) \vdash S \neq \varnothing}{1}}
\end{tabproof}

\FormulaDefAuto[Einermenge]{\{a\} := \{a,a\}}

\FormulaThmAuto{a \in \{a\}}
\begin{tabproof}
  \proofstep{}{a \in \{a,a\}}{\FormulaRefAuto{a \in \{a,b\}}{}}
  \proofstep{}{a \in \{a\}}{\FormulaRefAuto{\{a\} := \{a,a\}}{1}}
\end{tabproof}

\FormulaThmAuto{x \in \{a\} \eqvdash x = a}
\begin{tabproofwide}
  \proofstepwide{x \in \{a\}}{\leftrightarrow}{x \in \{a,a\}}%
    {\rIE{\FormulaRefAuto{\{a\} := \{a,a\}}{}, 1}}
  \proofstepwide{}{\leftrightarrow}{x = a \lor x = a}%
    {\rIE{\FormulaRefAuto{\{A,B\} := \iota C\Bigl(\forall x\,\bigl(x \in C \leftrightarrow x = A \lor x = B\bigr)\Bigr)}{}, 1}}
  \proofstepwide{}{\leftrightarrow}{x = a}%
    {\FormulaRefAuto{P \lor P \eqvdash P}{2}}
  \proofstepwide{x \in \{a\}}{\leftrightarrow}{x = a}%
    {\rChain{1,3}}
\end{tabproofwide}

\FormulaThmAuto{x \notin \{a\} \eqvdash x \neq a}
\begin{tabproof}
  \proofstep{}{x \notin \{a\} \leftrightarrow x \neq a}%
    {\FormulaRefAuto{P \leftrightarrow Q \eqvdash \neg P \leftrightarrow \neg Q}{\FormulaRefAuto{x \in \{a\} \eqvdash x = a}{}}}
\end{tabproof}

\FormulaThmDelta{%
  \{x\in A\mid x\in \{a\}\}=\{x\in A\mid x=a\}%
}{
  \DeltaRow{Mengen}{A\dsep a\dsep x}
}
\begin{tabproof}
  \proofstep{1}{x\in \{x\in A\mid x\in \{a\}\}}{\rA}
  \proofstep{1}{x\in A\land x\in \{a\}}{\FormulaRefAuto{x \in \{u \in A \mid P(u)\} \eqvdash x \in A \land P(x)}{1}}
  \proofstep{1}{x\in A}{\rAEa{2}}
  \proofstep{1}{x\in \{a\}}{\rAEb{2}}
  \proofstep{1}{x=a}{\FormulaRefAuto{x \in \{a\} \eqvdash x = a}{4}}
  \proofstep{1}{x\in \{x\in A\mid x=a\}}{\FormulaRefAuto{x \in A\dsep P(x)\vdash x \in \{x \in A \mid P(x)\}}{3,5}}
  \proofstep{}{\{x\in A\mid x\in \{a\}\}\subseteq \{x\in A\mid x=a\}}{%
    \FormulaRefAuto{A \subseteq B := \forall x\,(x\in A \rightarrow x\in B)}{\rUI{\rRI{1,6}}}}

  \proofstep{8}{x\in \{x\in A\mid x=a\}}{\rA}
  \proofstep{8}{x\in A\land x=a}{\FormulaRefAuto{x \in \{u \in A \mid P(u)\} \eqvdash x \in A \land P(x)}{8}}
  \proofstep{8}{x\in A}{\rAEa{9}}
  \proofstep{8}{x=a}{\rAEb{9}}
  \proofstep{8}{x\in \{a\}}{\FormulaRefAuto{x \in \{a\} \eqvdash x = a}{11}}
  \proofstep{8}{x\in \{x\in A\mid x\in \{a\}\}}{\FormulaRefAuto{x \in A\dsep P(x)\vdash x \in \{x \in A \mid P(x)\}}{10,12}}
  \proofstep{}{\{x\in A\mid x=a\}\subseteq \{x\in A\mid x\in \{a\}\}}{%
    \FormulaRefAuto{A \subseteq B := \forall x\,(x\in A \rightarrow x\in B)}{\rUI{\rRI{8,13}}}}

  \proofstep{}{\{x\in A\mid x\in \{a\}\}=\{x\in A\mid x=a\}}{\FormulaRefAuto{A \subseteq B,\, B \subseteq A \vdash A = B}{7,14}}
\end{tabproof}



\FormulaThmAuto{\{a\}=\{b,c\}\vdash a=b\land a=c}
\begin{tabproof}
  \proofstep{1}{\{a\}=\{b,c\}}{\rA}
  \proofstep{1}{b\in\{a\}}{\rIE{1,\FormulaRefAuto{a\in\{a,b\}}}}
  \proofstep{1}{a=b}{\FormulaRefAuto{a = b \vdash b = a}{\FormulaRefAuto{x \in \{a\} \eqvdash x = a}{2}}}
  \proofstep{1}{c\in\{a\}}{\rIE{1,\FormulaRefAuto{b\in\{a,b\}}}}
  \proofstep{1}{a=c}{\FormulaRefAuto{a = b \vdash b = a}{\FormulaRefAuto{x \in \{a\} \eqvdash x = a}{2}}}
  \proofstep{1}{a=b\land a=c}{\rAI{3,5}}
\end{tabproof}

\FormulaThmAuto{\exists x\in \{a,b\} P(x)\vdash P(a)\lor P(b)}
\begin{tabproof}
  \proofstep{1}{\exists x\in \{a,b\} P(x)}{\rA}
  \proofstep{2}{x\in \{a,b\}\land P(x)}{\rA}
  \proofstep{2}{x\in \{a,b\}}{\rAEa{2}}
  \proofstep{2}{P(x)}{\rAEb{2}}
  \proofstep{2}{x=a\lor x=b}{\FormulaRefAuto{x \in \{A,B\}\;\eqvdash\;(x = A \lor x = B)}{3}}
  \proofstep{6}{x=a}{\rA}
  \proofstep{2,6}{P(a)}{\rIE{6,4}}
  \proofstep{2,6}{P(a)\lor P(b)}{\rOIa{7}}
  \proofstep{9}{x=b}{\rA}
  \proofstep{2,9}{P(b)}{\rIE{9,4}}
  \proofstep{2,9}{P(a)\lor P(b)}{\rOIb{10}}
  \proofstep{2}{P(a)\lor P(b)}{\rOE{5,6,8,9,11}}
  \proofstep{1}{P(a)\lor P(b)}{\rEE{1,2,12}}
\end{tabproof}

\FormulaThmAuto{a \in A \vdash \{a\} \subseteq A}
\begin{tabproof}
  \proofstep{1}{x \in \{a\}}{\rA}
  \proofstep{2}{a \in A}{\rA}
  \proofstep{1}{x = a}{\FormulaRefAuto{x \in \{a\} \eqvdash x = a}{1}}
  \proofstep{1,2}{x \in A}{\rIE{3,2}}
  \proofstep{2}{\{a\} \subseteq A}{\FormulaRefAuto{A \subseteq B := \forall x\,(x \in A \rightarrow x \in B)}{\rUI{\rRI{1,4}}}}
\end{tabproof}

\FormulaThmAuto{a\in A\vdash A\cap \{A,a\}\neq\varnothing}
\begin{tabproof}
  \proofstep{1}{a \in A}{\rA}
  \proofstep{}{a \in \{A,a\}}{\FormulaRefAuto{b \in \{a,b\}}}
  \proofstep{1}{a \in A\cap \{A,a\}}{\FormulaRefAuto{x \in A, x\in B \vdash x \in A\cap B}{1,2}}
  \proofstep{1}{A\cap \{A,a\}\neq\varnothing}{\FormulaRefAuto{a \in A \vdash A \neq \varnothing}{3}}
\end{tabproof}

\FormulaThmAuto{a\in A\vdash A\cap \{a,A\}\neq\varnothing}
\begin{tabproof}
  \proofstep{1}{a \in A}{\rA}
  \proofstep{}{a \in \{a,A\}}{\FormulaRefAuto{a \in \{a,b\}}}
  \proofstep{1}{a \in A\cap \{a,A\}}{\FormulaRefAuto{x \in A, x\in B \vdash x \in A\cap B}{1,2}}
  \proofstep{1}{A\cap \{a,A\}\neq\varnothing}{\FormulaRefAuto{a \in A \vdash A \neq \varnothing}{3}}
\end{tabproof}

\FormulaThmAuto{\{a\}=\{b\}\eqvdash a=b}
\begin{tabproofsplit}
    \proofpart{\(\vdash\)}
    \proofstep{1}{\{a\}=\{b\}}{ \rA}
    \proofstep{1}{\forall x(x\in\{a\}\leftrightarrow x\in\{b\})}{\FormulaRefAuto{\forall x\, (x \in A \leftrightarrow x \in B) \eqvdash A = B}}
    \proofstep{}{a\in\{a\}}{\FormulaRefAuto{a \in \{a\}}}
    \proofstep{1}{a\in\{b\}}{\FormulaRefAuto{P\leftrightarrow Q, P\vdash Q}{\rUE{2},3}}
    \proofstep{1}{a=b}{\FormulaRefAuto{x \in \{a\} \eqvdash x = a}{4}}
    \closeproofpart
    \proofpart{\(\dashv\)}
    \proofstep{1}{a=b}{ \rA}
    \proofstep{}{\{a\}=\{a\}}{ \rII}
    \proofstep{1}{\{a\}=\{b\}}{ \rIE{1,2}}
    \closeproofpart
\end{tabproofsplit}

\FormulaThmAuto{a=c,b=d\vdash \{a,b\}\subseteq \{c,d\}}
\begin{tabproof}
  \proofstep{1}{a=c}{\rA}
  \proofstep{2}{b=d}{\rA}
  \proofstep{3}{x\in\{a,b\}}{\rA}
  \proofstep{3}{x=a\lor x=b}{\FormulaRefAuto{x \in \{A,B\}\;\eqvdash\;(x = A \lor x = B)}{3}}

  \proofcase[1]{x=a \vdash x\in\{c,d\}}
  \proofstep{5}{x=a}{\rA}
  \proofstep{1,5}{x=c}{\FormulaRefAuto{a = b,\, b = c \vdash a = c}{5,1}}
  \proofstep{1,5}{x=c\lor x=d}{\rOIa{6}}
  \proofstep{1,5}{x\in \{c,d\}}{\FormulaRefAuto{x \in \{A,B\}\;\eqvdash\;(x = A \lor x = B)}{7}}

  \proofcase[2]{x=b \vdash x\in\{c,d\}}
  \proofstep{9}{x=b}{\rA}
  \proofstep{2,9}{x=d}{\FormulaRefAuto{a = b,\, b = c \vdash a = c}{9,2}}
  \proofstep{2,9}{x=c\lor x=d}{\rOIb{10}}
  \proofstep{2,9}{x\in \{c,d\}}{\FormulaRefAuto{x \in \{A,B\}\;\eqvdash\;(x = A \lor x = B)}{11}}
  
  \proofcasesummary[1]{\{a,b\}\subseteq\{c,d\}}
  \proofstep{1,2,3}{x\in \{c,d\}}{\rOE{4,5,8,9,12}}
  \proofstep{1,2}{\{a,b\}\subseteq\{c,d\}}%
    {\FormulaRefAuto{ A \subseteq B := \forall x\,(x\in A \rightarrow x\in B)}{\rUI{\rRI{3,13}}}}
\end{tabproof}

\FormulaThmAuto{a=d,b=c\vdash \{a,b\}\subseteq \{c,d\}}
\begin{tabproof}
  \proofstep{1}{a=d}{\rA}
  \proofstep{2}{b=c}{\rA}
  \proofstep{3}{x\in\{a,b\}}{\rA}
  \proofstep{3}{x=a\lor x=b}{\FormulaRefAuto{x \in \{A,B\}\;\eqvdash\;(x = A \lor x = B)}{3}}

  \proofcase[1]{x=a \vdash x\in\{c,d\}}
  \proofstep{5}{x=a}{\rA}
  \proofstep{1,5}{x=d}{\FormulaRefAuto{a = b,\, b = c \vdash a = c}{5,1}}
  \proofstep{1,5}{x=c\lor x=d}{\rOIb{6}}
  \proofstep{1,5}{x\in \{c,d\}}{\FormulaRefAuto{x \in \{A,B\}\;\eqvdash\;(x = A \lor x = B)}{7}}

  \proofcase[2]{x=b \vdash x\in\{c,d\}}
  \proofstep{9}{x=b}{\rA}
  \proofstep{2,9}{x=c}{\FormulaRefAuto{a = b,\, b = c \vdash a = c}{9,2}}
  \proofstep{2,9}{x=c\lor x=d}{\rOIa{10}}
  \proofstep{2,9}{x\in \{c,d\}}{\FormulaRefAuto{x \in \{A,B\}\;\eqvdash\;(x = A \lor x = B)}{11}}
  
  \proofcasesummary[1]{\{a,b\}\subseteq\{c,d\}}
  \proofstep{1,2,3}{x\in \{c,d\}}{\rOE{4,5,8,9,12}}
  \proofstep{1,2}{\{a,b\}\subseteq\{c,d\}}%
    {\FormulaRefAuto{ A \subseteq B := \forall x\,(x\in A \rightarrow x\in B)}{\rUI{\rRI{3,13}}}}
\end{tabproof}

\FormulaThmAuto{a=c,b=d\vdash \{a,b\}=\{c,d\}}
% --- Beweis ---
\begin{tabproof}
  \proofstep{1}{a=c}{\rA}
  \proofstep{2}{b=d}{\rA}
  \proofstep{2}{c=a}{\FormulaRefAuto{a=b\vdash b=a}{1}}
  \proofstep{2}{b=d}{\FormulaRefAuto{a=b\vdash b=a}{2}}

  \proofstep{1,2}{\{a,b\}\subseteq\{c,d\}}{\FormulaRefAuto{a=c,b=d\vdash \{a,b\}\subseteq \{c,d\}}{1,2}}
  \proofstep{1,2}{\{c,d\}\subseteq\{a,b\}}{\FormulaRefAuto{a=c,b=d\vdash \{a,b\}\subseteq \{c,d\}}{3,4}}

  \proofstep{1,2}{\{a,b\}=\{c,d\}}{\FormulaRefAuto{ A \subseteq B, B \subseteq A \vdash A = B }{5,6}}
\end{tabproof}

\FormulaThmAuto{a=c\land b=d\vdash \{a,b\}=\{c,d\}}[Sei \(A\) eine Menge und \(a,b,c,d\in A\), dann gilt:]
\begin{tabproof}
    \proofstep{1}{a=c\land b=d}{\rA}
    \proofstep{1}{a=c}{\rAEa{1}}
    \proofstep{1}{b=d}{\rAEb{1}}
    \proofstep{1}{\{a,b\}=\{c,d\}}{\FormulaRefAuto{a=c,b=d\vdash \{a,b\}=\{c,d\}}{2,3}}
\end{tabproof}

\FormulaThmAuto{a=d, b=c\vdash \{a,b\}=\{c,d\}}
% --- Beweis ---
\begin{tabproof}
  \proofstep{1}{a=d}{\rA}
  \proofstep{2}{b=c}{\rA}
  \proofstep{3}{d=a}{\FormulaRefAuto{a=b\vdash b=a}{1}}   % Symmetrie aus 1
  \proofstep{4}{c=b}{\FormulaRefAuto{a=b\vdash b=a}{2}}   % Symmetrie aus 2

  % Erstes Inklusionsziel: {a,b} ⊆ {c,d}
  \proofstep{1,2}{\{a,b\}\subseteq\{c,d\}}{%
    \FormulaRefAuto{a=d,b=c\vdash \{a,b\}\subseteq \{c,d\}}{1,2}}

  % Zweites Inklusionsziel: {c,d} ⊆ {a,b}
  % Anwenden des selben Teilmengen-Theorems mit Umbenennung (a',b',c',d')=(c,d,a,b)
  % Voraussetzungen dafür sind c=b (Zeile 4) und d=a (Zeile 3)
  \proofstep{4,3}{\{c,d\}\subseteq\{a,b\}}{%
    \FormulaRefAuto{a=d,b=c\vdash \{a,b\}\subseteq \{c,d\}}{4,3}}

  % Gleichheit aus beidseitiger Inklusion
  \proofstep{1,2}{\{a,b\}=\{c,d\}}{%
    \FormulaRefAuto{ A \subseteq B, B \subseteq A \vdash A = B }{5,6}}
\end{tabproof}

\FormulaThmAuto{a=d\land b=c\vdash \{a,b\}=\{c,d\}}
\begin{tabproof}
    \proofstep{1}{a=d\land b=c}{\rA}
    \proofstep{1}{a=d}{\rAEa{1}}
    \proofstep{1}{b=c}{\rAEb{1}}
    \proofstep{1}{\{a,b\}=\{c,d\}}{\FormulaRefAuto{a=d, b=c\vdash \{a,b\}=\{c,d\}}{2,3}}
\end{tabproof}

\FormulaThmAuto{\{a,b\}=\{c,d\}\vdash (a=c\lor a=d)\land (b=c\lor b=d)}
\begin{tabproof}
    \proofstep{1}{\{a,b\}=\{c,d\}}{ \rA}
    \proofstep{}{a\in \{a,b\}}{\FormulaRefAuto{a\in \{a,b\}}}
    \proofstep{1}{a\in \{c,d\}}{\FormulaRefAuto{A=B, x\in A\vdash x\in B}{1,2}}
    \proofstep{1}{a=c\lor a=d}{\FormulaRefAuto{x \in \{A,B\}\;\eqvdash\;(x = A \lor x = B)}{3}}
    \proofstep{}{b\in \{a,b\}}{\FormulaRefAuto{b\in \{a,b\}}}
    \proofstep{1}{b\in \{c,d\}}{\FormulaRefAuto{A=B, x\in A\vdash x\in B}{1,5}}
    \proofstep{1}{b=c\lor b=d}{\FormulaRefAuto{x \in \{A,B\}\;\eqvdash\;(x = A \lor x = B)}{3}}
    \proofstep{1}{(a=c\lor a=d)\land (b=c\lor b=d)}{\rAI{4,7}}
\end{tabproof}

\FormulaThmAuto{\{a,b\}=\{c,d\},a=c,b=c\vdash d=c}
\begin{tabproof}
\proofstep{1}{\{a,b\}=\{c,d\}}{\rA}
\proofstep{2}{a=c}{\rA}
\proofstep{3}{b=c}{\rA}
\proofstep{}{d\in\{c,d\}}{\FormulaRefAuto{b\in\{a,b\}}}
\proofstep{1}{d\in\{a,b\}}{\rIE{1,4}}
\proofstep{1}{d=a\lor d=b}{\FormulaRefAuto{x \in \{A,B\}\;\eqvdash\;(x = A \lor x = B)}{5}}
\proofstep{1,2,3}{d=c}{\FormulaRefAuto{a = c,\, b = c, d=a\lor d=b \vdash d = c}{2,3,5}}
\end{tabproof}

\FormulaThmAuto{\{a,b\}=\{c,d\},a=d,b=d\vdash c=d}
\begin{tabproof}
\proofstep{1}{\{a,b\}=\{c,d\}}{\rA}
\proofstep{2}{a=d}{\rA}
\proofstep{3}{b=d}{\rA}
\proofstep{}{c\in\{c,d\}}{\FormulaRefAuto{a\in\{a,b\}}}
\proofstep{1}{c\in\{a,b\}}{\rIE{1,4}}
\proofstep{1}{c=a\lor c=b}{\FormulaRefAuto{x \in \{A,B\}\;\eqvdash\;(x = A \lor x = B)}{5}}
\proofstep{1,2,3}{c=d}{\FormulaRefAuto{a = c,\, b = c, d=a\lor d=b \vdash d = c}{2,3,5}}
\end{tabproof}

\FormulaThmAuto{\{a,b\}=\{c,d\} \;\eqvdash\; \bigl((a=c\land b=d)\;\lor\;(a=d\land b=c)\bigr)}
\begin{tabproofsplit}
  %%%%%%%%%%%%%%%%%%%%%%%%%%%%%
  \proofpart{\(\vdash\)}
  \proofstep{1}{\{a,b\}=\{c,d\}}{\rA}
  \proofstep{1}{(a=c\lor a=d)\land (b=c\lor b=d)}%
    {\FormulaRefAuto{\{a,b\}=\{c,d\}\vdash (a=c\lor a=d)\land (b=c\lor b=d)}{1}}
    \proofstep{1}{(a=c\land b=c)\lor (a=c\land b=d)\lor}%
  {\multirow{2}{*}{$\FormulaRefAuto{(P \lor Q) \land (R \lor S) \eqvdash
   (P \land R) \lor (P \land S) \lor (Q \land R) \lor (Q \land S)}{2}$}}
\proofstepstar{}{(a=d\land b=c)\lor (a=d\land b=d)}{}
  \proofstep{4}{a=c\land b=c}{\rA}
  \proofstep{4}{a=c}{\rAEa{4}}
  \proofstep{4}{b=c}{\rAEb{4}}
  \proofstep{1,4}{d=c}{\FormulaRefAuto{\{a,b\}=\{c,d\},a=c,b=c\vdash d=c}{1,5,6}}
  \proofstep{1,4}{b=d}{\FormulaRefAuto{a = b, c = b\vdash a = c}{5,6}}
  \proofstep{1,4}{(a=c\land b=d) \lor (a=d\land b=c)}{\rOIa{\rAI{5,8}}}
  \proofstep{10}{(a=c\land b=d) \lor (a=d\land b=c)}{\rA}
  \proofstep{11}{a=d\land b=d}{\rA}
  \proofstep{11}{a=d}{\rAEa{12}}
  \proofstep{11}{b=d}{\rAEb{12}}
  \proofstep{1,11}{c=d}{\FormulaRefAuto{\{a,b\}=\{c,d\},a=d,b=d\vdash c=d}{1,12,13}}
  \proofstep{1,11}{b=c}{\FormulaRefAuto{a = b, c = b\vdash a = c}{14,16}}
  \proofstep{1,11}{(a=c\land b=d) \lor (a=d\land b=c)}{\rOIb{\rAI{12,15}}}
  \proofstep{1}{(a=c\land b=d)\lor (a=d\land b=c)}{\FormulaRefAuto{P_1\lor P_2\lor P_3\lor P_4\dsep [P_1]\vdots R\dsep [P_2]\vdots R\dsep [P_3]\vdots R\dsep [P_4]\vdots R\vdash R}{3,4,9,10,10,11,16}}
  %%%%%%%%%%%%%%%%%%%%%%%%%%%%%
  \closeproofpart
  \proofpart{\(\dashv\)}
  \proofstep{1}{(a=c\land b=d)\lor (a=d\land b=c)}{\rA}
  \proofstep{2}{a=c\land b=d}{\rA}
  \proofstep{2}{a=c}{\rAEa{2}}
  \proofstep{2}{b=d}{\rAEb{2}}
  \proofstep{2}{\{a,b\}=\{c,d\}}{\FormulaRefAuto{a=c,b=d\vdash \{a,b\}=\{c,d\}}{3,4}}
  \proofstep{6}{a=d\land b=d}{\rA}
  \proofstep{6}{a=d}{\rAEa{2}}
  \proofstep{6}{b=c}{\rAEb{2}}
  \proofstep{6}{\{a,b\}=\{d,c\}}{\FormulaRefAuto{a=c,b=d\vdash \{a,b\}=\{c,d\}}{7,8}}
  \proofstep{6}{\{a,b\}=\{c,d\}}{\rIE{\FormulaRefAuto{\{a,b\} = \{b,a\}},9}}
  \proofstep{1}{\{a,b\}=\{c,d\}}{\rOE{1,2,5,6,10}}
  \closeproofpart
\end{tabproofsplit}

\FormulaThmAuto{\{a,b\}=\{c,d\}, a=c \vdash b=d}
\begin{tabproof}
\proofstep{1}{\{a,b\}=\{c,d\}}{\rA}
\proofstep{2}{a=c}{\rA}
\proofstep{1}{(a=c\land b=d)\;\lor\;(a=d\land b=c)}{\FormulaRefAuto{\{a,b\}=\{c,d\} \;\eqvdash\; \bigl((a=c\land b=d)\;\lor\;(a=d\land b=c)\bigr)}{1}}
\proofstep{4}{a=c\land b=d}{\rA}
\proofstep{4}{b=d}{\rAEb{4}}
\proofstep{6}{a=d\land b=c}{\rA}
\proofstep{6}{a=d}{\rAEa{6}}
\proofstep{6}{b=c}{\rAEb{6}}
\proofstep{2}{c=a}{\FormulaRefAuto{a=b \vdash b=a}{2}}
\proofstep{9,7}{c=d}{\FormulaRefAuto{a=b,\, b=c \vdash a=c}{9,7}}
\proofstep{8,10}{b=d}{\FormulaRefAuto{a=b,\, b=c \vdash a=c}{8,10}}
% Disjunktionselimination
\proofstep{1,2}{b=d}{\rOE{3,4,5,6,11}}
\end{tabproof}


\section{Geordnete Paare}

\FormulaThmAuto{(a,b) = (c,d)\eqvdash a=c\land b=d}[Sei \(A\) eine Menge und \(a,b,c,d\in A\), dann gilt:]
\begin{tabproofsplit}
    \proofpart{\(\vdash\)}
    \proofstep{1}{(a,b) = (c,d)}{\rA}
    \proofstep{1}{\{\{a\},\{a,b\}\}=\{\{c\},\{c,d\}\}}{\rIE{\FormulaRefAuto{(a, b) := \{ \{ a \}, \{ a, b \} \}},1}}
    \proofstep{1}{\{a\}=\{c\}\land \{a,b\}=\{c,d\}}{\multirow{2}{*}{$\FormulaRefAuto{\{a,b\}=\{c,d\} \;\eqvdash\; \bigl((a=c\land b=d)\;\lor\;(a=d\land b=c)\bigr)}{2}$}}
    \proofstepstar{}{\lor \{a\}=\{c,d\}\land \{a,b\}=\{c\}}{}
    \proofstep{4}{\{a\}=\{c\}\land \{a,b\}=\{c,d\}}{\rA}
    \proofstep{4}{\{a\}=\{c\}}{\rAEa{4}}
    \proofstep{4}{\{a,b\}=\{c,d\}}{\rAEb{4}}
    \proofstep{4}{a=c}{\FormulaRefAuto{\{a\}=\{b\}\eqvdash a=b}{5}}
    \proofstep{4}{b=d}{\FormulaRefAuto{\{a,b\}=\{c,d\}, a=c \vdash b=d}{6,7}}
    \proofstep{4}{a=c\land b=d}{\rAI{7,8}}
    \proofstep{10}{\{a\}=\{c,d\}\land \{a,b\}=\{c\}}{\rA}
    \proofstep{10}{\{a\}=\{c,d\}}{\rAEa{13}}
    \proofstep{10}{\{a,b\}=\{c\}}{\rAEb{13}}
    \proofstep{10}{a=c\land a=d}{\FormulaRefAuto{\{a\}=\{b,c\}\vdash a=b\land a=c}{11}}
    \proofstep{10}{c=a\land c=b}{\FormulaRefAuto{\{a\}=\{b,c\}\vdash a=b\land a=c}{\FormulaRefAuto{a = b \vdash b = a}{12}}}
    \proofstep{10}{a=c}{\rAEa{13}}
    \proofstep{10}{a=d}{\rAEb{13}}
    \proofstep{10}{c=b}{\rAEb{14}}
    \proofstep{10}{b=d}{\FormulaRefAuto{a = b,\, a = c \vdash b = c}{17,\FormulaRefAuto{a = b,\, a = c \vdash b = c}{15,16}}}
    \proofstep{10}{a=c\land b=d}{\rAI{15,18}}
    \proofstep{1}{a=c\land b=d}{\rOE{3,4,9,10,19}}
    \closeproofpart
    \proofpart{\(\dashv\)}
    \proofstep{1}{a=c\land b=d}{\rA}
    \proofstep{1}{\{a,b\}=\{c,d\}}{\FormulaRefAuto{a=c\land b=d\vdash \{a,b\}=\{c,d\}}{1}}
    \proofstep{1}{a=c}{\rAEa{1}}
    \proofstep{1}{{a}={c}}{\FormulaRefAuto{\{a\}=\{b\}\eqvdash a=b}{3}}
    \proofstep{1}{\{\{a\},\{a,b\}\}=\{\{c\},\{c,d\}\}}{\FormulaRefAuto{a=c, b=d\vdash \{a,b\}=\{c,d\}}{4,2}}
    \closeproofpart
\end{tabproofsplit}

\FormulaThmAuto{a=c,\, b=d\vdash (a,b) = (c,d)}[Sei \(A\) eine Menge und \(a,b,c,d\in A\), dann gilt:]
\begin{tabproof}
  \proofstep{1}{a=c}{\rA}
  \proofstep{2}{b=d}{\rA}
  \proofstep{1,2}{a=c\land b=d}{\rAI{1,2}}
  \proofstep{1,2}{(a,b) = (c,d)}{\FormulaRefAuto{(a,b) = (c,d)\eqvdash a=c\land b=d}{3}}
\end{tabproof}

\FormulaThmAuto{a\neq b\vdash (c,a)\neq (d,b)}[Sei \(A\) eine Menge und \(a,b,c,d\in A\), dann gilt:]
\begin{tabproof}
  \proofstep{1}{a\neq b}{\rA}
  \proofstep{2}{(c,a)=(d,b)}{\rA}
  \proofstep{2}{a=b}{\rAEb{\FormulaRefAuto{(a,b) = (c,d)\eqvdash a=c\land b=d}{2}}}
  \proofstep{1,2}{\bot}{\rBI{1,3}}
  \proofstep{1}{(c,a)\neq (d,b)}{\rCI{2,4}}
\end{tabproof}

\FormulaThmAuto{a\neq b\vdash (a,c)\neq (b,d)}[Sei \(A\) eine Menge und \(a,b,c,d\in A\), dann gilt:]
\begin{tabproof}
  \proofstep{1}{a\neq b}{\rA}
  \proofstep{2}{(a,c)=(b,d)}{\rA}
  \proofstep{2}{a=b}{\rAEa{\FormulaRefAuto{(a,b) = (c,d)\eqvdash a=c\land b=d}{2}}}
  \proofstep{1,2}{\bot}{\rBI{1,3}}
  \proofstep{1}{(a,c)\neq (b,d)}{\rCI{2,4}}
\end{tabproof}

\subsection{Verschachtelte Paare und Tripel}

\FormulaDefAuto[Tripel (rechtsassoziiert)]{(x,y,z) := (x,(y,z))}[Seien \(x,y,z\) Elemente einer Menge \(A\). Dann definieren wir:]


% --- Kernsatz: Komponenten-Eindeutigkeit des Tripels ---
\FormulaThmAuto{(a,b,c)=(a',b',c') \eqvdash a=a' \land b=b' \land c=c'}[Seien \(a,b,c,a',b',c'\) Elemente einer Menge \(A\). Dann gilt:]
\begin{tabproofsplit}
  \proofpart{\(\vdash\)}
    \proofstep{1}{(a,b,c)=(a',b',c')}{\rA}
    \proofstep{1}{(a,(b,c))=(a',(b',c'))}{\rIE{\FormulaRefAuto{(x,y,z) := (x,(y,z))},1}}
    \proofstep{1}{a=a'\land (b,c)=(b',c')}{\FormulaRefAuto{(a,b) = (c,d)\eqvdash a=c\land b=d}{2}}
    \proofstep{1}{a=a'}{\rAEa{3}}
    \proofstep{1}{(b,c)=(b',c')}{\rAEb{3}}
    \proofstep{1}{b=b'\land c=c'}{\FormulaRefAuto{(a,b) = (c,d)\eqvdash a=c\land b=d}{5}}
    \proofstep{1}{b=b'}{\rAEa{6}}
    \proofstep{1}{c=c'}{\rAEb{6}}
    \proofstep{1}{a=a'\land b=b'}{\rAI{4,7}}
    \proofstep{1}{a=a'\land b=b'\land c=c'}{\rAI{9,8}}
  \closeproofpart

  \proofpart{\(\dashv\)}
    \proofstep{1}{a=a'\land b=b'\land c=c'}{\rA}
    \proofstep{1}{a=a'}{\rAEa{1}}
    \proofstep{1}{b=b'}{\FormulaRefAuto{P\land Q\land R\vdash Q}{1}}
    \proofstep{1}{c=c'}{\rAEb{1}}
    \proofstep{1}{(b,c)=(b',c')}{\FormulaRefAuto{a=c,\, b=d\vdash (a,b) = (c,d)}{3,4}}
    \proofstep{1}{(a,(b,c))=(a',(b',c'))}{\FormulaRefAuto{a=c,\, b=d\vdash (a,b) = (c,d)}{2,5}}
    \proofstep{1}{(a,b,c)=(a',b',c')}{\rIE{\FormulaRefAuto{(x,y,z) := (x,(y,z))},6}}
  \closeproofpart
\end{tabproofsplit}

% 1) Aus komponentenweiser Gleichheit folgt Tripelgleichheit (rechtsassoziiert)
\FormulaThmAuto{a=a',\, b=b',\, c=c' \vdash (a,b,c)=(a',b',c')}[Seien \(a,b,c,a',b',c'\) Elemente einer Menge \(A\). Dann gilt:]
\begin{tabproof}
  \proofstep{1}{a=a'}{\rA}
  \proofstep{2}{b=b'}{\rA}
  \proofstep{3}{c=c'}{\rA}
  \proofstep{1,2,3}{(a,b,c)=(a',b',c')}{\FormulaRefAuto{(a,b,c)=(a',b',c') \eqvdash a=a' \land b=b' \land c=c'}\rAI{\rAI{1,2},3}}
\end{tabproof}

% 2) Komponenten-Eindeutigkeit für linksassoziierte Kodierung (Äquivalenz)
\FormulaThmAuto{((a,b),c)=((a',b'),c') \eqvdash a=a' \land b=b' \land c=c'}[Seien \(a,b,c,a',b',c'\) Elemente einer Menge \(A\). Dann gilt:]
\begin{tabproofsplit}
  \proofpart{\(\vdash\)}
    \proofstep{1}{((a,b),c)=((a',b'),c')}{\rA}
    \proofstep{1}{(a,b)=(a',b')\land c=c'}{\FormulaRefAuto{(a,b) = (c,d)\eqvdash a=c\land b=d}{1}}
    \proofstep{1}{(a,b)=(a',b')}{\rAEa{2}}
    \proofstep{1}{c=c'}{\rAEb{2}}
    \proofstep{1}{a=a'\land b=b'}{\FormulaRefAuto{(a,b) = (c,d)\eqvdash a=c\land b=d}{3}}
    \proofstep{1}{a=a'}{\rAEa{5}}
    \proofstep{1}{b=b'}{\rAEb{5}}
    \proofstep{1}{a=a'\land b=b'}{\rAI{6,7}}
    \proofstep{1}{a=a'\land b=b'\land c=c'}{\rAI{8,4}}
  \closeproofpart

  \proofpart{\(\dashv\)}
    \proofstep{1}{a=a'\land b=b'\land c=c'}{\rA}
    \proofstep{1}{a=a'}{\rAEa{1}}
    \proofstep{1}{b=b'}{\FormulaRefAuto{P\land Q\land R\vdash Q}{1}}
    \proofstep{1}{c=c'}{\rAEb{1}}
    \proofstep{2,3}{(a,b)=(a',b')}{\FormulaRefAuto{a=c,\, b=d\vdash (a,b) = (c,d)}{2,3}}
    \proofstep{5,4}{((a,b),c)=((a',b'),c')}{\FormulaRefAuto{a=c,\, b=d\vdash (a,b) = (c,d)}{5,4}}
  \closeproofpart
\end{tabproofsplit}

% 3) Aus komponentenweiser Gleichheit folgt linksassoziierte Paar-Gleichheit (kurz)
\FormulaThmAuto{a=a',\, b=b',\, c=c' \vdash ((a,b),c)=((a',b'),c')}[Seien \(a,b,c,a',b',c'\) Elemente einer Menge \(A\). Dann gilt:]
\begin{tabproof}
  \proofstep{1}{a=a'}{\rA}
  \proofstep{2}{b=b'}{\rA}
  \proofstep{3}{c=c'}{\rA}
  \proofstep{1,2,3}{((a,b),c)=((a',b'),c')}{\FormulaRefAuto{((a,b),c)=((a',b'),c') \eqvdash a=a' \land b=b' \land c=c'}\rAI{\rAI{1,2},3}}
\end{tabproof}

\chapter{Die Differenz}

\FormulaDefAuto{A \setminus B := \{ x \in A \mid x \notin B \}}

\FormulaThmAuto{x \in A \setminus B \eqvdash x \in A \land x \notin B}
\begin{tabproofwide}
  \proofstepwide{x \in A \setminus B}{\leftrightarrow}{x \in \{x \in A \mid x \notin B\}}%
    {\rUE{\FormulaRefAuto{\forall x\, (x \in A \leftrightarrow x \in B) \eqvdash A = B}{\FormulaRefAuto{A \setminus B := \{ x \in A \mid x \notin B \}}{}}}}
  \proofstepwide{}{\leftrightarrow}{x \in A \land x \notin B}%
    {\FormulaRefAuto{\{x \in A \mid P(x)\} := \iota B\bigl(\forall x\,(x \in B \leftrightarrow (x \in A \land P(x)))\bigr)}{1}}
\end{tabproofwide}

\FormulaThmAuto{x \in A \setminus B \vdash x \in A}
\begin{tabproof}
  \proofstep{1}{x \in A \setminus B}{\rA}
  \proofstep{1}{x \in A \land x \notin B}{\FormulaRefAuto{x \in A \setminus B \eqvdash x \in A \land x \notin B}{1}}
  \proofstep{1}{x \in A}{\rAEa{2}}
\end{tabproof}

\FormulaThmAuto{c \in A \setminus \{a\} \vdash c \neq a}
\begin{tabproof}
  \proofstep{1}{c \in A \setminus \{a\}}{\rA}
  \proofstep{1}{c \notin \{a\}}{\rAEb{\FormulaRefAuto{x \in A \setminus B \eqvdash x \in A \land x \notin B}{1}}}
  \proofstep{1}{c \neq a}{\FormulaRefAuto{x \notin \{a\} \eqvdash x \neq a}{2}}
\end{tabproof}

\FormulaThmAuto{c \in A \setminus \{a,b\} \vdash c \neq a}
\begin{tabproof}
  \proofstep{1}{c \in A \setminus \{a,b\}}{\rA}
  \proofstep{1}{c \notin \{a,b\}}{\rAEb{\FormulaRefAuto{x \in A \setminus B \eqvdash x \in A \land x \notin B}{1}}}
  \proofstep{1}{c \neq a}{\rAEa{\FormulaRefAuto{x \notin \{a,b\} \eqvdash x \neq a \land x \neq b}{2}}}
\end{tabproof}

\FormulaThmAuto{c \in A \setminus \{a,b\} \vdash c \neq b}
\begin{tabproof}
  \proofstep{1}{c \in A \setminus \{a,b\}}{\rA}
  \proofstep{1}{c \notin \{a,b\}}{\rAEb{\FormulaRefAuto{x \in A \setminus B \eqvdash x \in A \land x \notin B}{1}}}
  \proofstep{1}{c \neq b}{\rAEb{\FormulaRefAuto{x \notin \{a,b\} \eqvdash x \neq a \land x \neq b}{2}}}
\end{tabproof}

\FormulaThmAuto{c \in A \setminus B,\, b \in B \vdash c \neq b}
\begin{tabproof}
  \proofstep{1}{c \in A \setminus B}{\rA}
  \proofstep{2}{b \in B}{\rA}
  \proofstep{1}{c \notin B}{\rAEb{\FormulaRefAuto{x \in A \setminus B \eqvdash x \in A \land x \notin B}{1}}}
  \proofstep{1,2}{c \neq b}{\FormulaRefAuto{ a \in A,\; b \not\in A \vdash a \neq b }{2,3}}
\end{tabproof}

\FormulaThmAuto{a \notin A \setminus \{a\}}
\begin{tabproof}
  \proofstep{1}{a \in A \setminus \{a\}}{\rA}
  \proofstep{}{a = a}{\rII}
  \proofstep{1}{a \neq a}{\FormulaRefAuto{c \in A \setminus \{a\} \vdash c \neq a}{1}}
  \proofstep{1}{\bot}{\rBI{2,3}}
  \proofstep{}{a \notin A \setminus \{a\}}{\rCI{1,4}}
\end{tabproof}

\FormulaThmAuto{a \notin A \setminus \{a,b\}}
\begin{tabproof}
  \proofstep{1}{a \in A \setminus \{a,b\}}{\rA}
  \proofstep{}{a = a}{\rII}
  \proofstep{1}{a \neq a}{\FormulaRefAuto{c \in A \setminus \{a,b\} \vdash c \neq a}{1}}
  \proofstep{1}{\bot}{\rBI{2,3}}
  \proofstep{}{a \notin A \setminus \{a,b\}}{\rCI{1,4}}
\end{tabproof}

\FormulaThmAuto{b \notin A \setminus \{a,b\}}
\begin{tabproof}
  \proofstep{1}{b \in A \setminus \{a,b\}}{\rA}
  \proofstep{}{b = b}{\rII}
  \proofstep{1}{b \neq b}{\FormulaRefAuto{c \in A \setminus \{a,b\} \vdash c \neq b}{1}}
  \proofstep{1}{\bot}{\rBI{2,3}}
  \proofstep{}{b \notin A \setminus \{a,b\}}{\rCI{1,4}}
\end{tabproof}

\FormulaThmAuto{a \in A,\, a \neq b \vdash a \in A \setminus \{b\}}
\begin{tabproof}
  \proofstep{1}{a \in A}{\rA}
  \proofstep{2}{a \neq b}{\rA}
  \proofstep{2}{a \notin \{b\}}{\FormulaRefAuto{x \notin \{a\} \eqvdash x \neq a}{2}}
  \proofstep{1,2}{a \in A \setminus \{b\}}{\FormulaRefAuto{x \in A \setminus B \eqvdash x \in A \land x \notin B}{\rAI{1,3}}}
\end{tabproof}

\FormulaThmAuto{a \in A,\, b \neq a \vdash a \in A \setminus \{b\}}
\begin{tabproof}
  \proofstep{1}{a \in A}{\rA}
  \proofstep{2}{b \neq a}{\rA}
  \proofstep{2}{a \neq b}{\FormulaRefAuto{a \neq b \vdash b \neq a}{2}}
  \proofstep{1,2}{a \in A \setminus \{b\}}{\FormulaRefAuto{a \in A,\, a \neq b \vdash a \in A \setminus \{b\}}{1,3}}
\end{tabproof}

\FormulaThmAuto{A\setminus B\subseteq A}
\begin{tabproof}
  \proofstep{1}{x\in A\setminus B}{\rA}
  \proofstep{1}{x\in A}{\FormulaRefAuto{x \in A \setminus B \vdash x \in A}{2}}
  \proofstep{1}{A\setminus B\subseteq A}{\FormulaRefAuto{A \subseteq B := \forall x\,(x \in A \rightarrow x \in B)}{\rUI{\rRI{1,2}}}}
\end{tabproof}


\FormulaThmAuto{A\subseteq C\vdash A\setminus B\subseteq C}
\begin{tabproof}
  \proofstep{1}{A\subseteq C}{\rA}
  \proofstep{2}{x\in A\setminus B}{\rA}
  \proofstep{2}{x\in A}{\FormulaRefAuto{x \in A \setminus B \vdash x \in A}{2}}
  \proofstep{1,2}{x\in C}{\FormulaRefAuto{A\subseteq B, x\in A\vdash x\in B}{1,3}}
  \proofstep{1,2}{A\setminus B\subseteq C}{\FormulaRefAuto{A \subseteq B := \forall x\,(x \in A \rightarrow x \in B)}{\rUI{\rRI{2,4}}}}
\end{tabproof}

\FormulaThmAuto{A=A\setminus \{a\}\vdash a\notin A}
\begin{tabproof}
  \proofstep{1}{A=A\setminus \{a\}}{\rA}
  \proofstep{}{a\notin A\setminus \{a\}}{\FormulaRefAuto{a \notin A \setminus \{a\}}}
  \proofstep{1}{a\notin A}{\rIE{1,2}}
\end{tabproof}


\chapter{Vereinigung}

\section{Axiom der Vereinigung}

\FormulaAxiomDelta[Vereinigung]{%
  \exists C\;\forall x\;\bigl(x \in C \;\leftrightarrow\;\exists B\in A\,(x \in B)\bigr)%
}{%
  \DeltaRow{Mengen}{A}%
}


\section{Definition der Vereinigung}
%%begin novalidate
\FormulaDefDelta[Vereinigung]{%
  \bigcup A := \iota C\Bigl(\forall x\;\bigl(x \in C \;\leftrightarrow\;\exists B\in A\,(x \in B)\bigr)\Bigr)%
}{%
  \DeltaRow{Mengen}{A}%
}
%%end novalidate

\FormulaThmDelta{%
  x \in \bigcup A \;\eqvdash\;\exists B\in A\,(x \in B)%
}{%
  \DeltaRow{Mengen}{x\dsep A\dsep B}%
}
\begin{tabproof}
  \proofstep{}{%
    \forall x\;\bigl(x \in \bigcup A \;\leftrightarrow\;\exists B\in A\,(x \in B)\bigr)%
  }{%
    \FormulaRefAuto{\bigcup A := \iota C\Bigl(\forall x\;\bigl(x \in C \;\leftrightarrow\;\exists B\in A\,(x \in B)\bigr)\Bigr)}%
  }
  \proofstep{}{%
    x \in \bigcup A \;\leftrightarrow\;\exists B\in A\,(x \in B)%
  }{%
    \rUE{1}%
  }
\end{tabproof}

\FormulaThmAuto{\bigcup \varnothing = \varnothing}
\begin{tabproof}
  \proofstep{1}{x \in \bigcup \varnothing}{\rA}
  \proofstep{1}{\exists B\in \varnothing\,(x \in B)}{%
    \FormulaRefAuto{x \in \bigcup A \;\eqvdash\;\exists B\in A\,(x \in B)}{1}%
  }

  \proofstep{3}{B\in \varnothing \land x\in B}{\rA}
  \proofstep{3}{B\in \varnothing}{\rAEa{3}}
  \proofstep{3}{B \not\in \varnothing}{\FormulaRefAuto{x \not\in \varnothing}}
  \proofstep{3}{\bot}{\rBI{4,5}}
  \proofstep{1}{\bot}{\rEE{2,3,6}}
  \proofstep{}{x \not\in \bigcup \varnothing}{\rCI{1,7}}

  \proofstep{}{\forall x\,\bigl(x \not\in \bigcup \varnothing\bigr)}{\rUI{8}}
  \proofstep{}{\bigcup \varnothing = \varnothing}{%
    \FormulaRefAuto{\forall x\, x\notin A\vdash A=\varnothing}{9}%
  }
\end{tabproof}

\FormulaThmDelta{%
  B\in A\dsep x\in B\vdash x \in \bigcup A%
}{%
  \DeltaRow{Mengen}{x\dsep A\dsep B}%
}
\begin{tabproof}
  \proofstep{1}{B\in A}{\rA}
  \proofstep{2}{x\in B}{\rA}
  \proofstep{1,2}{B\in A\land x\in B}{\rAI{1,2}}
  \proofstep{1,2}{\exists B\in A\,(x\in B)}{\rEI{3}}
  \proofstep{1,2}{x \in \bigcup A}{%
    \FormulaRefAuto{x \in \bigcup A \;\eqvdash\;\exists B\in A\,(x \in B)}{4}%
  }
\end{tabproof}

\FormulaDefAuto[Vereinigung zweier Mengen]{A \cup B := \bigcup \{A, B\}}

\FormulaThmAuto{x\in A \cup B \eqvdash x\in \bigcup \{A, B\}}
\begin{tabproof}
\proofstep{}{z \in A \cup B\leftrightarrow z \in \bigcup \{A, B\}}{\rUE{\FormulaRefAuto{\forall x\, (x \in A \leftrightarrow x \in B) \eqvdash A = B}{\FormulaRefAuto{A \cup B := \bigcup \{A, B\}}}}}
\end{tabproof}

\FormulaThmAuto{A \subseteq B \vdash \bigcup A \subseteq \bigcup B}
\begin{tabproof}
  \proofstep{1}{A \subseteq B}{\rA}
  \proofstep{2}{x \in \bigcup A}{\rA}
  \proofstep{2}{\exists X\in A\,(x \in X)}{%
    \FormulaRefAuto{x \in \bigcup A \;\eqvdash\;\exists B\in A\,(x \in B)}{2}%
  }
  \proofstep{4}{X \in A \land x \in X}{\rA}
  \proofstep{4}{X \in A}{\rAEa{4}}
  \proofstep{4}{x \in X}{\rAEb{4}}
  \proofstep{1,4}{X \in B}{\FormulaRefAuto{A \subseteq B,\, x \in A \vdash x \in B}{5,1}}
  \proofstep{1,4}{x \in \bigcup B}{\FormulaRefAuto{B\in A\dsep x\in B\vdash x \in \bigcup A}{7,6}}
  \proofstep{1,2}{x \in \bigcup B}{\rEE{3,4,8}}
  \proofstep{1}{\bigcup A \subseteq \bigcup B}{\FormulaRefAuto{A \subseteq B := \forall x\,(x \in A \rightarrow x \in B)}{\rUI{\rRE{2,9}}}}
\end{tabproof}

\FormulaThmAuto{z \in A \cup B \eqvdash z \in A \lor z \in B}
\begin{tabproofwide}
  \proofstepwide{z \in A \cup B}{\leftrightarrow}{z \in \bigcup \{A, B\}}%
    {\FormulaRefAuto{x\in A \cup B \eqvdash x\in \bigcup \{A, B\}}}
  \proofstepwide{}{\leftrightarrow}{\exists C\,(C \in \{A, B\} \land z \in C)}%
    {\FormulaRefAuto{\bigcup A := \iota C\bigl(\forall x\,(x \in C \leftrightarrow \exists B\in A\,(x \in B))\bigr)}{1}}
  \proofstepwide{}{\leftrightarrow}{\exists C\,((C = A \lor C = B) \land z \in C)}%
    {\rLRS{\FormulaRefAuto{\{A,B\} := \iota C\bigl(\forall x\,(x \in C \leftrightarrow x = A \lor x = B)\bigr)}}}
  \proofstepwide{}{\leftrightarrow}{z \in A \lor z \in B}%
    {\FormulaRefAuto{\exists c\,((c = a \lor c = b) \land P(c)) \eqvdash P(a) \lor P(b)}{3}}
\end{tabproofwide}


\section{Grundlegende Eigenschaften}

\subsection{Elementare Eigenschaften}

\FormulaThmAuto{z \in A \vdash z \in A \cup B}
\begin{tabproof}
  \proofstep{1}{z \in A}{\rA}
  \proofstep{1}{z \in A \lor z \in B}{\rOIa{1}}
  \proofstep{1}{z \in A \cup B}{\FormulaRefAuto{z \in A \cup B \eqvdash z \in A \lor z \in B}{2}}
\end{tabproof}

\FormulaThmAuto{z \in B \vdash z \in A \cup B}
\begin{tabproof}
  \proofstep{1}{z \in B}{\rA}
  \proofstep{1}{z \in A \lor z \in B}{\rOIb{1}}
  \proofstep{1}{z \in A \cup B}{\FormulaRefAuto{z \in A \cup B \eqvdash z \in A \lor z \in B}{2}}
\end{tabproof}

\FormulaThmAuto{x \in \{a,b\} \eqvdash x \in \{a\} \cup \{b\}}
\begin{tabproofwide}
  \proofstepwide{x \in \{a\}}{\leftrightarrow}{x = a}%
    {\FormulaRefAuto{x \in \{a\} \eqvdash x = a}}
  \proofstepwide{x \in \{b\}}{\leftrightarrow}{x = b}%
    {\FormulaRefAuto{x \in \{a\} \eqvdash x = a}}
  \proofstepwide{B \in \{\{a\},\{b\}\}}{\leftrightarrow}{B = \{a\} \lor B = \{b\}}%
    {\FormulaRefAuto{\{A,B\} := \iota C\bigl(\forall x\,(x \in C \leftrightarrow x = A \lor x = B)\bigr)}}
  \proofstepwide{x \in \{a,b\}}{\leftrightarrow}{x = a \lor x = b}%
    {\FormulaRefAuto{\{A,B\} := \iota C\bigl(\forall x\,(x \in C \leftrightarrow x = A \lor x = B)\bigr)}}
  \proofstepwide{}{\leftrightarrow}{x \in \{a\} \lor x = b}{\rLRS{1}}
  \proofstepwide{}{\leftrightarrow}{x \in \{a\} \lor x \in \{b\}}{\rLRS{2}}

  % --- Zeile 7: auf zwei Zeilen gesplittet ---
  \proofstepwide{}{\leftrightarrow}{\exists B\,\bigl(B = \{a\} \lor B = \{b\}\bigr)}%
    {\multirow{2}{*}{\FormulaRefAuto{\exists c\,((c = a \lor c = b) \land P(c)) \eqvdash P(a) \lor P(b)}{6}}}
  \proofstepwide*{}{\land}{x \in B}{}
  % -------------------------------------------

  \proofstepwide{}{\leftrightarrow}{\exists B\in \{\{a\},\{b\}\}\,x \in B}{\rLRS{3}}
  \proofstepwide{}{\leftrightarrow}{x \in \bigcup \{\{a\},\{b\}\}}%
    {\FormulaRefAuto{\bigcup A := \iota C\bigl(\forall x\,(x \in C \leftrightarrow \exists B\in A\,(x \in B))\bigr)}{8}}
  \proofstepwide{}{\leftrightarrow}{x \in \{a\} \cup \{b\}}%
    {\FormulaRefAuto{A \cup B := \bigcup \{A, B\}}{8}}
  \proofstepwide{x \in \{a,b\}}{\leftrightarrow}{x \in \{a\} \cup \{b\}}{\rChain{4,10}}
\end{tabproofwide}

\FormulaThmAuto{\{a,b\} = \{a\} \cup \{b\}}
\begin{tabproof}
  \proofstep{}{x \in \{a,b\} \leftrightarrow x \in \{a\} \cup \{b\}}{\FormulaRefAuto{x \in \{a,b\} \eqvdash x \in \{a\} \cup \{b\}}}
  \proofstep{}{\{a,b\} = \{a\} \cup \{b\}}{\FormulaRefAuto{\forall x\,(x \in A \leftrightarrow x \in B) \eqvdash A = B}{\rUI{1}}}
\end{tabproof}

\FormulaThmAuto{a \in A \cup \{a\}}
\begin{tabproof}
  \proofstep{}{a \in \{a\}}{\FormulaRefAuto{a \in \{a\}}}
  \proofstep{}{a \in A \cup \{a\}}{\FormulaRefAuto{z \in B \vdash z \in A \cup B}{1}}
\end{tabproof}

\FormulaThmAuto{a \in \{a\} \cup A}
\begin{tabproof}
  \proofstep{}{a \in \{a\}}{\FormulaRefAuto{a \in \{a\}}}
  \proofstep{}{a \in \{a\} \cup A}{\FormulaRefAuto{z \in A \vdash z \in A \cup B}{1}}
\end{tabproof}

\FormulaThmAuto{\{a\} \cup A\neq\varnothing}
\begin{tabproof}
  \proofstep{}{a \in \{a\} \cup A}{\FormulaRefAuto{a \in \{a\} \cup A}}
  \proofstep{}{\{a\} \cup A\neq\varnothing}{\FormulaRefAuto{a\in S\vdash S\neq\varnothing}}
\end{tabproof}

\FormulaThmAuto{A\cup \{a\}\neq\varnothing}
\begin{tabproof}
  \proofstep{}{a \in A\cup \{a\}}{\FormulaRefAuto{a \in A\cup \{a\}}}
  \proofstep{}{\{a\} \cup A\neq\varnothing}{\FormulaRefAuto{a\in S\vdash S\neq\varnothing}}
\end{tabproof}


\FormulaThmAuto{A\cup \{a\}=A\eqvdash a\in A}
\begin{tabproofsplit}
  \proofpart{$\vdash$}
    \proofstep{1}{A \cup \{a\} = A}{\rA}
    \proofstep{}{a \in A \cup \{a\}}{\FormulaRefAuto{a \in A \cup \{a\}}}
    \proofstep{}{a \in A}{\rIE{1,2}}
  \closeproofpart

  \proofpart{$\dashv$}
    % Annahme
    \proofstep{1}{a \in A}{\rA}

    % (i) A \cup {a} \subseteq A
    \proofstep{2}{x \in A \cup \{a\}}{\rA}
    \proofstep{2}{x \in A \lor x \in \{a\}}%
      {\FormulaRefAuto{z \in A \cup B \eqvdash z \in A \lor z \in B}{2}}

    % Fall 1: x \in A
    \proofstep{3}{x \in A}{\rA}

    % Fall 2: x \in {a}  =>  x = a  =>  x \in A
    \proofstep{5}{x \in \{a\}}{\rA}
    \proofstep{5}{x = a}{\FormulaRefAuto{x \in \{a\} \eqvdash x = a}{5}}
    \proofstep{1,5}{x \in A}{\rIE{6,1}}

    % Fälle zusammenführen
    \proofstep{1,2}{x \in A}{\rOE{3,4,4,5,7}}

    % Universalisierung: Teilmengenrichtung
    \proofstep{}{A \cup \{a\} \subseteq A}{\FormulaRefAuto{A \subseteq B := \forall x\,(x \in A \rightarrow x \in B)}{\rUI{\rRI{2,8}}}}

    % (ii) A \subseteq A \cup {a}
    \proofstep{}{A \subseteq A \cup \{a\}}%
      {\FormulaRefAuto{z \in A \vdash z \in A \cup B}}

    % (iii) Gleichheit aus beidseitiger Inklusion
    \proofstep{}{A \cup \{a\} = A}{\FormulaRefAuto{A \subseteq B \land B \subseteq A \eqvdash A = B}{9,10}}
  \closeproofpart
\end{tabproofsplit}


\FormulaThmAuto{a \notin B,\, A = B \cup \{a\} \vdash A \not\subseteq B}
\begin{tabproof}
  \proofstep{1}{a \notin B}{\rA}
  \proofstep{2}{A = B \cup \{a\}}{\rA}
  \proofstep{3}{A \subseteq B}{\rA}
  \proofstep{}{a \in B \cup \{a\}}{\FormulaRefAuto{a \in A \cup \{a\}}}
  \proofstep{2}{a \in A}{\rIE{2,4}}
  \proofstep{2,3}{a \in B}{\FormulaRefAuto{A \subseteq B,\, x \in A \vdash x \in B}{5,3}}
  \proofstep{1,2,3}{\bot}{\rBI{1,6}}
  \proofstep{1,2}{A \not\subseteq B}{\rCI{3,7}}
\end{tabproof}

\subsection{Idempotenz, Kommutativität, Assoziativität}

\FormulaThmAuto{x \in A \eqvdash x \in A \cup A}
\begin{tabproofwide}
  \proofstepwide{x \in A}{\leftrightarrow}{x \in A \lor x \in A}%
    {\FormulaRefAuto{P \lor P \eqvdash P}}
  \proofstepwide{}{\leftrightarrow}{x \in A \cup A}%
    {\FormulaRefAuto{z \in A \cup B \eqvdash z \in A \lor z \in B}}
  \proofstepwide{x \in A}{\leftrightarrow}{x \in A \cup A}%
    {\rChain{1,2}}
\end{tabproofwide}

\FormulaThmAuto[Idempotenz]{A = A \cup A}
\begin{tabproofwide}
  \proofstepwide{x \in A}{\leftrightarrow}{x \in A \cup A}%
    {\FormulaRefAuto{x \in A \eqvdash x \in A \cup A}}
  \proofstepwide{A}{=}{A \cup A}%
    {\FormulaRefAuto{\forall x\, (x \in A \leftrightarrow x \in B) \eqvdash A = B}{\rUI{1}}}
\end{tabproofwide}

\FormulaThmAuto{x \in A \cup B \eqvdash x \in B \cup A}
\begin{tabproofwide}
  \proofstepwide{x \in A \cup B}{\leftrightarrow}{x \in A \lor x \in B}%
    {\FormulaRefAuto{z \in A \cup B \eqvdash z \in A \lor z \in B}}
  \proofstepwide{}{\leftrightarrow}{x \in B \lor x \in A}%
    {\FormulaRefAuto{P \lor Q \vdash Q \lor P}{1}}
  \proofstepwide{}{\leftrightarrow}{x \in B \cup A}%
    {\FormulaRefAuto{z \in A \cup B \eqvdash z \in A \lor z \in B}{2}}
  \proofstepwide{x \in A \cup B}{\leftrightarrow}{x \in B \cup A}%
    {\rChain{1,3}}
\end{tabproofwide}

\FormulaThmAuto[Kommutativität]{A \cup B = B \cup A}
\begin{tabproofwide}
  \proofstepwide{x \in A \cup B}{\leftrightarrow}{x \in B \cup A}%
    {\FormulaRefAuto{x \in A \cup B \eqvdash x \in B \cup A}}
  \proofstepwide{A \cup B}{=}{B \cup A}%
    {\FormulaRefAuto{\forall x\, (x \in A \leftrightarrow x \in B) \eqvdash A = B}{\rUI{1}}}
\end{tabproofwide}

\FormulaThmAuto{x \in A \eqvdash x \in A \cup \varnothing}
\begin{tabproofwide}
  \proofstepwide{x \in A}{\leftrightarrow}{x \in A \lor x \in \varnothing}%
    {\FormulaRefAuto{\forall x(\neg Q(x)) \vdash P \leftrightarrow P \lor Q(a)}{\FormulaRefAuto{\varnothing := \iota O\bigl(\forall x\,(x \not\in O)\bigr)}}}
  \proofstepwide{}{\leftrightarrow}{x \in A \cup \varnothing}%
    {\FormulaRefAuto{z \in A \cup B \eqvdash z \in A \lor z \in B}{1}}
  \proofstepwide{x \in A}{\leftrightarrow}{x \in A \cup \varnothing}%
    {\rChain{1,2}}
\end{tabproofwide}

\FormulaThmAuto{A = A \cup \varnothing}
\begin{tabproofwide}
  \proofstepwide{x \in A}{\leftrightarrow}{x \in A \cup \varnothing}%
    {\FormulaRefAuto{x \in A \eqvdash x \in A \cup \varnothing}}
  \proofstepwide{A}{=}{A \cup \varnothing}%
    {\FormulaRefAuto{\forall x\, (x \in A \leftrightarrow x \in B) \eqvdash A = B}{\rUI{1}}}
\end{tabproofwide}

\FormulaThmAuto{A = \varnothing \cup A}
\begin{tabproofwide}
  \proofstepwide{A}{=}{A \cup \varnothing}%
    {\FormulaRefAuto{A = A \cup \varnothing}}
  \proofstepwide{A \cup \varnothing}{=}{\varnothing \cup A}%
    {\FormulaRefAuto{A \cup B = B \cup A}}
  \proofstepwide{A}{=}{\varnothing \cup A}%
    {\rIE{2,1}}
\end{tabproofwide}

\FormulaThmAuto{A \cup \{A\} = \{\varnothing\} \eqvdash A = \varnothing}
\begin{tabproofsplit}
\proofpart{\(\vdash\)}
  \proofstep{1}{\{\varnothing\} = A \cup \{A\}}{\rA}
  \proofstep{}{A \in A \cup \{A\}}{\FormulaRefAuto{a \in A \cup \{a\}}}
  \proofstep{1}{A \in \{\varnothing\}}{\FormulaRefAuto{A = B,\, x \in A \vdash x \in B}{1,2}}
  \proofstep{1}{A = \varnothing}{\FormulaRefAuto{x \in \{a\} \eqvdash x = a}{3}}
\closeproofpart

\proofpart{\(\dashv\)}
  \proofstep{1}{A = \varnothing}{\rA}
  \proofstep{}{\{\varnothing\} = \varnothing \cup \{\varnothing\}}{\FormulaRefAuto{A = \varnothing \cup A}}
  \proofstep{1}{A \cup \{A\} = \{\varnothing\}}{\rIE{1,\FormulaRefAuto{a = b \vdash b = a}{2}}}
\closeproofpart
\end{tabproofsplit}



\FormulaThmAuto{z \in A \cup B \eqvdash z \not\in A \rightarrow z \in B}
\begin{tabproofwide}
  \proofstepwide{z \in A \cup B}{\leftrightarrow}{z \in A \lor z \in B}%
    {\FormulaRefAuto{z \in A \cup B \eqvdash z \in A \lor z \in B}}
  \proofstepwide{}{ \leftrightarrow }{z \not\in A \rightarrow z \in B}%
    {\FormulaRefAuto{P \rightarrow Q \eqvdash \neg P \lor Q}{1}}
\end{tabproofwide}

\FormulaThmAuto{z \in A \cup B \eqvdash z \not\in B \rightarrow z \in A}
\begin{tabproofwide}
  \proofstepwide{z \in A \cup B}{\leftrightarrow}{z \in A \lor z \in B}%
    {\FormulaRefAuto{z \in A \cup B \eqvdash z \in A \lor z \in B}}
  \proofstepwide{}{\leftrightarrow}{z \not\in B \rightarrow z \in A}%
    {\FormulaRefAuto{P \rightarrow Q \eqvdash \neg P \lor Q}{1}}
\end{tabproofwide}

\FormulaThmAuto{A=B\vdash A\cup C = B\cup C}
\begin{tabproof}
    \proofstep{1}{A=B}{\rA}
    \proofstep{1}{x\in A\cup C\leftrightarrow x\in B\cup C}{\rIE{1,\FormulaRefAuto{P\leftrightarrow P}}}
    \proofstep{1}{A\cup C=B\cup C}{\FormulaRefAuto{\forall x\, (x \in A \leftrightarrow x \in B) \eqvdash A = B}{\rUI{2}}}
\end{tabproof}


\FormulaThmAuto{z \in (A \cup B) \cup C \eqvdash (z \in A \lor z \in B) \lor z \in C}
\begin{tabproofwide}
  % Zeile 1 -> zwei Zeilen
  \proofstepwide{z \in (A \cup B) \cup C}{\leftrightarrow}{z \in (A \cup B)}%
    {\multirow{2}{*}{\FormulaRefAuto{z \in A \cup B \eqvdash z \in A \lor z \in B}}}
  \proofstepwide*{}{\lor}{z \in C}{}

  % Zeile 2 -> zwei Zeilen
  \proofstepwide{}{\leftrightarrow}{(z \in A \lor z \in B)}%
    {\multirow{2}{*}{\rLRS{\FormulaRefAuto{z \in A \cup B \eqvdash z \in A \lor z \in B}{},1}}}
  \proofstepwide*{}{\lor}{z \in C}{}
\end{tabproofwide}

\FormulaThmAuto{z \in A \cup (B \cup C) \eqvdash z \in A \lor (z \in B \lor z \in C)}
\begin{tabproofwide}
  % Zeile 1 -> zwei Zeilen
  \proofstepwide{z \in A \cup (B \cup C)}{\leftrightarrow}{z \in A}%
    {\multirow{2}{*}{\FormulaRefAuto{z \in A \cup B \eqvdash z \in A \lor z \in B}}}
  \proofstepwide*{}{\lor}{z \in (B \cup C)}{}

  % Zeile 2 -> zwei Zeilen
  \proofstepwide{}{\leftrightarrow}{z \in A}%
    {\multirow{2}{*}{\rLRS{\FormulaRefAuto{z \in A \cup B \eqvdash z \in A \lor z \in B}{},1}}}
  \proofstepwide*{}{\lor}{(z \in B \lor z \in C)}{}
\end{tabproofwide}

\FormulaThmAuto{z \in (A \cup B) \cup C \eqvdash z \in A \cup (B \cup C)}
\begin{tabproofwide}
  \proofstepwide{z \in (A \cup B) \cup C}{\leftrightarrow}{(z \in A \lor z \in B) \lor z \in C}%
    {\FormulaRefAuto{z \in (A \cup B) \cup C \eqvdash (z \in A \lor z \in B) \lor z \in C}}
  \proofstepwide{}{\leftrightarrow}{z \in A \lor (z \in B \lor z \in C)}%
    {\FormulaRefAuto{P \lor (Q \lor R) \eqvdash (P \lor Q) \lor R}{1}}
  \proofstepwide{}{\leftrightarrow}{z \in A \cup (B \cup C)}%
    {\FormulaRefAuto{z \in A \cup (B \cup C) \eqvdash z \in A \lor (z \in B \lor z \in C)}{2}}
\end{tabproofwide}

\FormulaThmAuto[Assoziativität der Vereinigung]{(A \cup B) \cup C = A \cup (B \cup C)}
\begin{tabproofwide}
  \proofstepwide{z \in (A \cup B) \cup C}{\leftrightarrow}{z \in A \cup (B \cup C)}%
    {\FormulaRefAuto{z \in (A \cup B) \cup C \eqvdash z \in A \cup (B \cup C)}}
  \proofstepwide{(A \cup B) \cup C}{=}{A \cup (B \cup C)}%
    {\FormulaRefAuto{\forall x\, (x \in A \leftrightarrow x \in B) \eqvdash A = B}{\rUI{1}}}
\end{tabproofwide}

\subsection{ Distributivgesetze}

\FormulaThmAuto{z \in A \cup (B \cap C) \eqvdash z \in A \lor (z \in B \land z \in C)}
\begin{tabproofwide}
  % Zeile 1 -> zwei Zeilen
  \proofstepwide{z \in A \cup (B \cap C)}{\leftrightarrow}{z \in A}%
    {\multirow{2}{*}{\FormulaRefAuto{z \in A \cup B \eqvdash z \in A \lor z \in B}}}
  \proofstepwide*{}{\lor}{z \in (B \cap C)}{}

  % Zeile 2 -> zwei Zeilen
  \proofstepwide{}{\leftrightarrow}{z \in A}%
    {\multirow{2}{*}{\rLRS{\FormulaRefAuto{x \in A \cap B \eqvdash x \in A \land x \in B}{},1}}}
  \proofstepwide*{}{\lor}{(z \in B \land z \in C)}{}
\end{tabproofwide}

\FormulaThmAuto{z \in (A \cap B) \cup C \eqvdash (z \in A \land z \in B) \lor z \in C}
\begin{tabproofwide}
  % Zeile 1 -> zwei Zeilen
  \proofstepwide{z \in (A \cap B) \cup C}{\leftrightarrow}{z \in (A \cap B)}%
    {\multirow{2}{*}{\FormulaRefAuto{z \in A \cup B \eqvdash z \in A \lor z \in B}}}
  \proofstepwide*{}{\lor}{z \in C}{}

  % Zeile 2 -> zwei Zeilen
  \proofstepwide{}{\leftrightarrow}{(z \in A \land z \in B)}%
    {\multirow{2}{*}{\rLRS{\FormulaRefAuto{x \in A \cap B \eqvdash x \in A \land x \in B}{},1}}}
  \proofstepwide*{}{\lor}{z \in C}{}
\end{tabproofwide}

\FormulaThmAuto{z \in A \cap (B \cup C) \eqvdash z \in A \land (z \in B \lor z \in C)}
\begin{tabproofwide}
  % Zeile 1 -> zwei Zeilen
  \proofstepwide{z \in A \cap (B \cup C)}{\leftrightarrow}{z \in A}%
    {\multirow{2}{*}{\FormulaRefAuto{x \in A \cap B \eqvdash x \in A \land x \in B}}}
  \proofstepwide*{}{\land}{z \in (B \cup C)}{}

  % Zeile 2 -> zwei Zeilen
  \proofstepwide{}{\leftrightarrow}{z \in A}%
    {\multirow{2}{*}{\rLRS{\FormulaRefAuto{z \in A \cup B \eqvdash z \in A \lor z \in B}{},1}}}
  \proofstepwide*{}{\land}{(z \in B \lor z \in C)}{}
\end{tabproofwide}

\FormulaThmAuto{z \in (A \cup B) \cap C \eqvdash (z \in A \lor z \in B) \land z \in C}
\begin{tabproofwide}
  % Zeile 1 -> zwei Zeilen
  \proofstepwide{z \in (A \cup B) \cap C}{\leftrightarrow}{z \in (A \cup B)}%
    {\multirow{2}{*}{\FormulaRefAuto{x \in A \cap B \eqvdash x \in A \land x \in B}}}
  \proofstepwide*{}{\land}{z \in C}{}

  % Zeile 2 -> zwei Zeilen
  \proofstepwide{}{\leftrightarrow}{(z \in A \lor z \in B)}%
    {\multirow{2}{*}{\rLRS{\FormulaRefAuto{z \in A \cup B \eqvdash z \in A \lor z \in B}{},1}}}
  \proofstepwide*{}{\land}{z \in C}{}
\end{tabproofwide}

\FormulaThmAuto{z \in (A \cup B) \cap (C \cup D) \eqvdash (z \in A \lor z \in B) \land (z \in C \lor z \in D)}
\begin{tabproofwide}
  % Zeile 1 -> zwei Zeilen
  \proofstepwide{z \in (A \cup B) \cap (C \cup D)}{\leftrightarrow}{z \in (A \cup B)}%
    {\multirow{2}{*}{\FormulaRefAuto{x \in A \cap B \eqvdash x \in A \land x \in B}}}
  \proofstepwide*{}{\land}{z \in (C \cup D)}{}

  % Zeile 2 -> zwei Zeilen
  \proofstepwide{}{\leftrightarrow}{(z \in A \lor z \in B)}%
    {\multirow{2}{*}{\rLRS{\FormulaRefAuto{z \in A \cup B \eqvdash z \in A \lor z \in B}{},1}}}
  \proofstepwide*{}{\land}{(z \in C \lor z \in D)}{}
\end{tabproofwide}

\FormulaThmAuto{z \in (A \cap B) \cup (C \cap D) \eqvdash (z \in A \land z \in B) \lor (z \in C \land z \in D)}
\begin{tabproofwide}
  % Zeile 1 -> zwei Zeilen
  \proofstepwide{z \in (A \cap B) \cup (C \cap D)}{\leftrightarrow}{z \in (A \cap B)}%
    {\multirow{2}{*}{\FormulaRefAuto{z \in A \cup B \eqvdash z \in A \lor z \in B}}}
  \proofstepwide*{}{\lor}{z \in (C \cap D)}{}

  % Zeile 2 -> zwei Zeilen
  \proofstepwide{}{\leftrightarrow}{(z \in A \land z \in B)}%
    {\multirow{2}{*}{\rLRS{\FormulaRefAuto{x \in A \cap B \eqvdash x \in A \land x \in B}{},1}}}
  \proofstepwide*{}{\lor}{(z \in C \land z \in D)}{}
\end{tabproofwide}

\FormulaThmAuto{z \in A \cap (B \cup C) \eqvdash z \in (A \cap B) \cup (A \cap C)}
\begin{tabproofwide}
  % Zeile 1 -> zwei Zeilen
  \proofstepwide{z \in A \cap (B \cup C)}{\leftrightarrow}{z \in A}%
    {\multirow{2}{*}{\FormulaRefAuto{z \in A \cap (B \cup C) \eqvdash z \in A \land (z \in B \lor z \in C)}}}
  \proofstepwide*{}{\land}{(z \in B \lor z \in C)}{}

  % Zeile 2 -> zwei Zeilen
  \proofstepwide{}{\leftrightarrow}{(z \in A \land z \in B)}%
    {\multirow{2}{*}{\FormulaRefAuto{P \land (Q \lor R) \eqvdash (P \land Q) \lor (P \land R)}{1}}}
  \proofstepwide*{}{\lor}{(z \in A \land z \in C)}{}

  % Zeile 3 -> zwei Zeilen
  \proofstepwide{}{\leftrightarrow}{z \in (A \cap B)}%
    {\multirow{2}{*}{\FormulaRefAuto{z \in (A \cap B) \cup (C \cap D) \eqvdash (z \in A \land z \in B) \lor (z \in C \land z \in D)}{2}}}
  \proofstepwide*{}{\cup}{z \in (A \cap C)}{}

  % Abschluss (unverändert)
  \proofstepwide{z \in A \cap (B \cup C)}{\leftrightarrow}{z \in (A \cap B) \cup (A \cap C)}%
    {\rChain{1,3}}
\end{tabproofwide}

\FormulaThmAuto{z \in (A \cup B) \cap C \eqvdash z \in (A \cap C) \cup (B \cap C)}
\begin{tabproofwide}
  % Zeile 1 -> zwei Zeilen
  \proofstepwide{z \in (A \cup B) \cap C}{\leftrightarrow}{(z \in A \lor z \in B)}%
    {\multirow{2}{*}{\FormulaRefAuto{z \in (A \cup B) \cap C \eqvdash (z \in A \lor z \in B) \land z \in C}}}
  \proofstepwide*{}{\land}{z \in C}{}

  % Zeile 2 -> zwei Zeilen
  \proofstepwide{}{\leftrightarrow}{(z \in A \land z \in C)}%
    {\multirow{2}{*}{\FormulaRefAuto{(P \lor Q) \land R \eqvdash (P \land R) \lor (Q \land R)}{1}}}
  \proofstepwide*{}{\lor}{(z \in B \land z \in C)}{}

  % Zeile 3 -> zwei Zeilen
  \proofstepwide{}{\leftrightarrow}{z \in (A \cap C)}%
    {\multirow{2}{*}{\FormulaRefAuto{z \in (A \cap B) \cup (C \cap D) \eqvdash (z \in A \land z \in B) \lor (z \in C \land z \in D)}{2}}}
  \proofstepwide*{}{\cup}{z \in (B \cap C)}{}

  % Abschluss
  \proofstepwide{z \in (A \cup B) \cap C}{\leftrightarrow}{z \in (A \cap C) \cup (B \cap C)}%
    {\rChain{1,3}}
\end{tabproofwide}

\FormulaThmAuto{z \in (A \cap B) \cup C \eqvdash z \in (A \cup C) \cap (B \cup C)}
\begin{tabproofwide}
  % Zeile 1 -> zwei Zeilen
  \proofstepwide{z \in (A \cap B) \cup C}{\leftrightarrow}{(z \in A \land z \in B)}%
    {\multirow{2}{*}{\FormulaRefAuto{z \in (A \cap B) \cup C \eqvdash (z \in A \land z \in B) \lor z \in C}}}
  \proofstepwide*{}{\lor}{z \in C}{}

  % Zeile 2 -> zwei Zeilen
  \proofstepwide{}{\leftrightarrow}{(z \in A \lor z \in C)}%
    {\multirow{2}{*}{\FormulaRefAuto{(P \land Q) \lor R \eqvdash (P \lor R) \land (Q \lor R)}{1}}}
  \proofstepwide*{}{\land}{(z \in B \lor z \in C)}{}

  % Zeile 3 -> zwei Zeilen
  \proofstepwide{}{\leftrightarrow}{z \in (A \cup C)}%
    {\multirow{2}{*}{\FormulaRefAuto{z \in (A \cup B) \cap (C \cup D) \eqvdash (z \in A \lor z \in B) \land (z \in C \lor z \in D)}{2}}}
  \proofstepwide*{}{\cap}{z \in (B \cup C)}{}

  % Abschluss
  \proofstepwide{z \in (A \cap B) \cup C}{\leftrightarrow}{z \in (A \cup C) \cap (B \cup C)}%
    {\rChain{1,3}}
\end{tabproofwide}

\FormulaThmAuto{z \in A \cup (B \cap C) \eqvdash z \in (A \cup B) \cap (A \cup C)}
\begin{tabproofwide}
  % Zeile 1 -> zwei Zeilen
  \proofstepwide{z \in A \cup (B \cap C)}{\leftrightarrow}{z \in A}%
    {\multirow{2}{*}{\FormulaRefAuto{z \in A \cup (B \cap C) \eqvdash z \in A \lor (z \in B \land z \in C)}}}
  \proofstepwide*{}{\lor}{(z \in B \land z \in C)}{}

  % Zeile 2 -> zwei Zeilen
  \proofstepwide{}{\leftrightarrow}{(z \in A \lor z \in B)}%
    {\multirow{2}{*}{\FormulaRefAuto{P \lor (Q \land R) \eqvdash (P \lor Q) \land (P \lor R)}{1}}}
  \proofstepwide*{}{\land}{(z \in A \lor z \in C)}{}

  % Zeile 3 -> zwei Zeilen
  \proofstepwide{}{\leftrightarrow}{z \in (A \cup B)}%
    {\multirow{2}{*}{\FormulaRefAuto{z \in (A \cup B) \cap (C \cup D) \eqvdash (z \in A \lor z \in B) \land (z \in C \lor z \in D)}{2}}}
  \proofstepwide*{}{\cap}{z \in (A \cup C)}{}

  % Abschluss
  \proofstepwide{z \in A \cup (B \cap C)}{\leftrightarrow}{z \in (A \cup B) \cap (A \cup C)}%
    {\rChain{1,3}}
\end{tabproofwide}

\FormulaThmAuto{A \cup (B \cap C) = (A \cup B) \cap (A \cup C)}
\begin{tabproofwide}
  \proofstepwide{z \in A \cup (B \cap C)}{\leftrightarrow}{z \in (A \cup B) \cap (A \cup C)}%
    {\FormulaRefAuto{z \in A \cup (B \cap C) \eqvdash z \in (A \cup B) \cap (A \cup C)}}
  \proofstepwide{A \cup (B \cap C)}{=}{(A \cup B) \cap (A \cup C)}%
    {\FormulaRefAuto{\forall x\, (x \in A \leftrightarrow x \in B) \eqvdash A = B}{\rUI{1}}}
\end{tabproofwide}

\FormulaThmAuto{(A \cap B) \cup C = (A \cup C) \cap (B \cup C)}
\begin{tabproofwide}
  \proofstepwide{z \in (A \cap B) \cup C}{\leftrightarrow}{z \in (A \cup C) \cap (B \cup C)}%
    {\FormulaRefAuto{z \in (A \cap B) \cup C \eqvdash z \in (A \cup C) \cap (B \cup C)}}
  \proofstepwide{(A \cap B) \cup C}{=}{(A \cup C) \cap (B \cup C)}%
    {\FormulaRefAuto{\forall x\, (x \in A \leftrightarrow x \in B) \eqvdash A = B}{\rUI{1}}}
\end{tabproofwide}

\FormulaThmAuto{A \cap (B \cup C) = (A \cap B) \cup (A \cap C)}
\begin{tabproofwide}
  \proofstepwide{z \in A \cap (B \cup C)}{\leftrightarrow}{z \in (A \cap B) \cup (A \cap C)}%
    {\FormulaRefAuto{z \in A \cap (B \cup C) \eqvdash z \in (A \cap B) \cup (A \cap C)}}
  \proofstepwide{A \cap (B \cup C)}{=}{(A \cap B) \cup (A \cap C)}%
    {\FormulaRefAuto{\forall x\, (x \in A \leftrightarrow x \in B) \eqvdash A = B}{\rUI{1}}}
\end{tabproofwide}

\FormulaThmAuto{(A \cup B) \cap C = (A \cap C) \cup (B \cap C)}
\begin{tabproofwide}
  \proofstepwide{z \in (A \cup B) \cap C}{\leftrightarrow}{z \in (A \cap C) \cup (B \cap C)}%
    {\FormulaRefAuto{z \in (A \cup B) \cap C \eqvdash z \in (A \cap C) \cup (B \cap C)}}
  \proofstepwide{(A \cup B) \cap C}{=}{(A \cap C) \cup (B \cap C)}%
    {\FormulaRefAuto{\forall x\, (x \in A \leftrightarrow x \in B) \eqvdash A = B}{\rUI{1}}}
\end{tabproofwide}

\subsection{Eigenschaften in Bezug auf Teilmengen}


\FormulaThmAuto{A \subseteq A \cup B}
\begin{tabproofwide}
  \proofstepwide{x \in A}{\rightarrow}{x \in A \lor x \in B}%
    {\FormulaRefAuto{P \rightarrow P \lor Q}}
  \proofstepwide{}{ \rightarrow}{x \in A \cup B}%
    {\FormulaRefAuto{z \in A \cup B \eqvdash z \in A \lor z \in B}{1}}
  \proofstepwide{x \in A}{\rightarrow}{x \in A \cup B}%
    {\rChain{1,2}}
  \proofstepwidestar{A \subseteq A \cup B}%
    {\FormulaRefAuto{ A \subseteq B := \forall x\,(x\in A \rightarrow x\in B) }{\rUI{3}}}
\end{tabproofwide}

\FormulaThmAuto{A \subseteq B \cup A}
\begin{tabproofwide}
  \proofstepwide{x \in A}{\rightarrow}{x \in B \lor x \in A}%
    {\FormulaRefAuto{P \rightarrow Q \lor P}}
  \proofstepwide{}{\rightarrow}{x \in B \cup A}%
    {\FormulaRefAuto{z \in A \cup B \eqvdash z \in A \lor z \in B}{1}}
  \proofstepwide{x \in A}{\rightarrow}{x \in B \cup A}%
    {\rChain{1,2}}
  \proofstepwidestar{A \subseteq B \cup A}%
    {\FormulaRefAuto{A \subseteq B := \forall x\,(x\in A \rightarrow x\in B)}{\rUI{3}}}
\end{tabproofwide}

\FormulaThmAuto{A \subseteq C,\, B \subseteq C \vdash A \cup B \subseteq C}
\begin{tabproofwide}
  \proofstepwidestar[1]{A \subseteq C}{\rA}
  \proofstepwidestar[2]{B \subseteq C}{\rA}
  \proofstepwide{z \in A \cup B}{\rightarrow}{z \in A \lor z \in B}%
    {\FormulaRefAuto{z \in A \cup B \eqvdash z \in A \lor z \in B}}
  \proofstepwide[1,2]{}{ \rightarrow}{z \in C}%
    {\FormulaRefAuto{A\subseteq C,\, B\subseteq C,\, z\in A\lor z\in B \vdash z\in C}{1,2,3}}
  \proofstepwide[1,2]{z \in A \cup B}{\rightarrow}{z \in C}%
    {\rChain{3,4}}
  \proofstepwidestar[1,2]{A \cup B \subseteq C}%
    {\FormulaRefAuto{A \subseteq B := \forall x\,(x\in A \rightarrow x\in B)}{\rUI{5}}}
\end{tabproofwide}


\FormulaThmAuto{A \subseteq B \vdash A \cup C \subseteq B \cup C}
\begin{tabproofwide}
  \proofstepwidestar[1]{A \subseteq B}{\rA}
  \proofstepwide{z \in A \cup C}{\rightarrow}{z \in A \lor z \in C}%
    {\FormulaRefAuto{z \in A \cup B \eqvdash z \in A \lor z \in B}}
  \proofstepwide[1]{}{ \rightarrow}{z \in B \lor z \in C}%
    {\FormulaRefAuto{P \rightarrow Q,\, P \lor R \vdash Q \lor R}%
      {\rUE{\FormulaRefAuto{A \subseteq B := \forall x\,(x\in A \rightarrow x\in B)}{1}},2}}
  \proofstepwide[1]{}{ \rightarrow}{z \in B \cup C}%
    {\FormulaRefAuto{z \in A \cup B \eqvdash z \in A \lor z \in B}{3}}
  \proofstepwide[1]{z \in A \cup C}{\rightarrow}{z \in B \cup C}%
    {\rChain{2,4}}
  \proofstepwidestar[1]{A \cup C \subseteq B \cup C}%
    {\FormulaRefAuto{A \subseteq B := \forall x\,(x\in A \rightarrow x\in B)}{\rUI{5}}}
\end{tabproofwide}

\FormulaThmAuto{A \subseteq B \vdash C \cup A \subseteq C \cup B}
\begin{tabproofwide}
  \proofstepwidestar[1]{A \subseteq B}{\rA}
  \proofstepwide{C \cup A}{=}{A \cup C}%
    {\FormulaRefAuto{A \cup B = B \cup A}}
  \proofstepwide[1]{}{ \subseteq}{B \cup C}%
    {\FormulaRefAuto{A \subseteq B \vdash A \cup C \subseteq B \cup C}{1}}
  \proofstepwide[1]{}{=}{C \cup B}%
    {\FormulaRefAuto{A \cup B = B \cup A}{3}}
  \proofstepwide[1]{C \cup A}{\subseteq}{C \cup B}%
    {\rChain{2,4}}
\end{tabproofwide}

\FormulaThmAuto{A \subseteq B,\, C \subseteq D \vdash A \cup C \subseteq B \cup D}
\begin{tabproofwide}
  \proofstepwidestar[1]{A \subseteq B}{\rA}
  \proofstepwidestar[2]{C \subseteq D}{\rA}
  \proofstepwide[1]{A \cup C}{\subseteq}{B \cup C}%
    {\FormulaRefAuto{A \subseteq B \vdash A \cup C \subseteq B \cup C}{1}}
  \proofstepwide[2]{}{ \subseteq}{B \cup D}%
    {\FormulaRefAuto{A \subseteq B \vdash C \cup A \subseteq C \cup B}{2}}
  \proofstepwide[1,2]{A \cup C}{\subseteq}{B \cup D}%
    {\rChain{3,4}}
\end{tabproofwide}

\FormulaThmAuto{A \subseteq B,\, C \subseteq D \vdash A \cap C \subseteq B \cap D}
\begin{tabproofwide}
  \proofstepwidestar[1]{A \subseteq B}{\rA}
  \proofstepwidestar[2]{C \subseteq D}{\rA}
  \proofstepwide{x \in A \cap C}{\rightarrow}{x \in A \land x \in C}%
    {\FormulaRefAuto{x \in A \cap B \eqvdash x \in A \land x \in B}}
  \proofstepwide[1]{}{ \rightarrow}{x \in B \land x \in C}%
    {\FormulaRefAuto{P \rightarrow Q,\, P \land R \vdash Q \land R}{\rUE{\FormulaRefAuto{A \subseteq B := \forall x\,(x \in A \rightarrow x \in B)}{1}},3}}
  \proofstepwide[1,2]{}{ \rightarrow}{x \in B \land x \in D}%
    {\FormulaRefAuto{P \rightarrow Q,\, R \land P \vdash R \land Q}{\rUE{\FormulaRefAuto{A \subseteq B := \forall x\,(x \in A \rightarrow x \in B)}{2}},4}}
  \proofstepwide[1,2]{}{ \rightarrow}{x \in B \cap D}%
    {\FormulaRefAuto{x \in A \cap B \eqvdash x \in A \land x \in B}{5}}
  \proofstepwide[1,2]{x \in A \cap C}{\rightarrow}{x \in B \cap D}%
    {\rChain{3,6}}
  \proofstepwidestar[1,2]{A \cap C \subseteq B \cap D}%
    {\FormulaRefAuto{A \subseteq B := \forall x\,(x \in A \rightarrow x \in B)}{\rUI{7}}}
\end{tabproofwide}

\FormulaThmAuto{a \in A,\, b \in B \vdash \{a,b\} \subseteq A \cup B}
\begin{tabproofwide}
  \proofstepwidestar[1]{a \in A}{\rA}
  \proofstepwidestar[2]{b \in B}{\rA}
  \proofstepwide[1]{\{a\}}{\subseteq}{A}%
    {\FormulaRefAuto{a \in A \vdash \{a\} \subseteq A}{1}}
  \proofstepwide[2]{\{b\}}{\subseteq}{B}%
    {\FormulaRefAuto{a \in A \vdash \{a\} \subseteq A}{2}}
  \proofstepwide[1,2]{\{a\} \cup \{b\}}{\subseteq}{A \cup B}%
    {\FormulaRefAuto{A \subseteq B,\, C \subseteq D \vdash A \cup C \subseteq B \cup D}{3,4}}
  \proofstepwide{\{a,b\}}{=}{\{a\} \cup \{b\}}%
    {\FormulaRefAuto{\{a,b\} = \{a\} \cup \{b\}}}
  \proofstepwidestar{\{a,b\} \subseteq A \cup B}%
    {\rIE{6,5}}
\end{tabproofwide}

\FormulaThmAuto{a \in A,\, b \in A \vdash \{a,b\} \subseteq A}
\begin{tabproofwide}
  \proofstepwidestar[1]{a \in A}{\rA}
  \proofstepwidestar[2]{b \in A}{\rA}
  \proofstepwide[1,2]{\{a,b\}}{\subseteq}{A \cup A}%
    {\FormulaRefAuto{a \in A,\, b \in B \vdash \{a,b\} \subseteq A \cup B}{1,2}}
  \proofstepwide[1,2]{}{=}{A}%
    {\FormulaRefAuto{A = A \cup A}{3}}
  \proofstepwidestar[1,2]{\{a,b\} \subseteq A}%
    {\rChain{3,4}}
\end{tabproofwide}

% =========================================================
% Einfügestelle-Vorschlag:
% direkt nach dem Abschnitt zur Vereinigung (bigcup)
% =========================================================

\chapter{Mengenfamilien}

\section{Grundbegriffe}

\FormulaDefDelta[Überdeckung einer Menge]{%
  \CoverFam(M,A) \coloneqq A \subseteq \bigcup M
}{
  \DeltaRow{Mengen}{A \dsep M}
  \DeltaRow{Neue Symbole}{\CoverFam}
}

\FormulaDefDelta[Partition einer Menge]{%
  \PartFam(M,A) \coloneqq \DisjFam(M)\land A=\bigcup M
}{
  \DeltaRow{Mengen}{A \dsep M}
  \DeltaRow{Neue Symbole}{\PartFam}
}

\section{Paarweise disjunkte nichtleere Familien}

\FormulaAxiomDelta[Nichtleere Mengen]{%
  X \in M \vdash X \neq \varnothing
}{
  \DeltaRow{Mengen}{M \dsep X}
}

\FormulaAxiomDelta[Paarweise disjunkt]{%
  X \in M \dsep Y \in M \dsep X \neq Y \vdash X \cap Y = \varnothing
}{
  \DeltaRow{Mengen}{M \dsep X \dsep Y}
}

\FormulaDefDeltaK[Begriff der paarweise disjunkten nichtleeren Familie]{\DisjFam(M)}{Disjunkte Familie}{
  \DeltaRow{Menge von Mengen}{M \dsep X \dsep Y}
  \DeltaRow{\textbf{Axiome}}{}
  \DeltaRow{Nichtleere Mengen}
           {X \in M \vdash X \neq \varnothing}
           [\parbox[t]{3cm}{\FormulaRefAuto{X \in M \vdash X \neq \varnothing}}]
  \DeltaRow{Paarweise disjunkt}
           {\begin{aligned}
              X \in M \dsep Y \in M \dsep X \neq Y \\
              \vdash X \cap Y = \varnothing
            \end{aligned}}
           [\parbox[t]{3cm}{\FormulaRefAuto{X \in M \dsep Y \in M \dsep X \neq Y \vdash X \cap Y = \varnothing}}]
  \DeltaRow{\textbf{Neue Symbole}}{}
  \DeltaRow{\makecell[l]{Paarweise disjunkte\\ nichtleere Familie}}{}
}

\FormulaThmDelta[Elemente verschiedener Mengen sind verschieden]{%
  X \in M \dsep Y \in M \dsep X \neq Y \dsep a \in X
  \vdash a \notin Y
}{
  \DeltaRow{Mengen}{M \dsep X \dsep Y \dsep a}
  \DeltaPrem{\makecell[l]{Paarweise disjunkte\\ nichtleere Familie}}{\DisjFam(M)}
}
\begin{tabproof}
  \proofstep{1}{X\in M}{\rA}
  \proofstep{2}{Y\in M}{\rA}
  \proofstep{3}{X\neq Y}{\rA}
  \proofstep{4}{a\in X}{\rA}

  \proofstep{1,2,3}{X\cap Y=\varnothing}{%
    \FormulaRefAuto{X \in M \dsep Y \in M \dsep X \neq Y \vdash X \cap Y = \varnothing}{1,2,3}}

  \proofstep{1,2,3,4}{a\notin Y}{%
    \FormulaRefAuto{A \cap B = \varnothing,\ x \in A \vdash x \notin B}{5,4}}
\end{tabproof}

\FormulaThmDelta[Gemeinsames Element impliziert Gleichheit]{%
  X \in M \dsep Y \in M \dsep a \in X \dsep a \in Y
  \vdash X = Y
}{
  \DeltaRow{Mengen}{M \dsep X \dsep Y \dsep a}
  \DeltaPrem{\makecell[l]{Paarweise disjunkte\\ nichtleere Familie}}{\DisjFam(M)}
}
\begin{tabproof}
  \proofstep{1}{X\in M}{\rA}
  \proofstep{2}{Y\in M}{\rA}
  \proofstep{3}{a\in X}{\rA}
  \proofstep{4}{a\in Y}{\rA}

  \proofstep{5}{X\neq Y}{\rA}
  \proofstep{1,2,3,5}{a\notin Y}{%
    \FormulaRefAuto{X \in M \dsep Y \in M \dsep X \neq Y \dsep a \in X \vdash a \notin Y}{1,2,5,3}}
  \proofstep{1,2,3,4,5}{\bot}{\rBI{4,6}}

  \proofstep{1,2,3,4}{X=Y}{\rCE{5,7}}
\end{tabproof}

% ============================================================
% Vereinigungsabgeschlossene Familien
% (einfügen z.B. nach \section{Paarweise disjunkte nichtleere Familien})
% ============================================================

% ============================================================
% Vereinigungsabgeschlossene Familien
% ============================================================

\section{Vereinigungsabgeschlossene Familien}

\subsection{Axiom der Vereinigungsschließung}

\FormulaAxiomDelta[Vereinigungsschließung]{%
  X \in M \dsep Y \in M \vdash X \cup Y \in M%
}{%
  \DeltaRow{Mengen}{M \dsep X \dsep Y}%
}

\subsection{Definition der vereinigungsabgeschlossenen Familie}

\FormulaDefDeltaK[Begriff der vereinigungsabgeschlossenen Familie]{\UnionClosedFam(M)}{Vereinigungsabgeschlossene Familie}{%
  \DeltaRow{Mengen}{M \dsep X \dsep Y}%
  \DeltaRow{\textbf{Axiome}}{}%
  \DeltaRow{Vereinigungsschließung}{%
    X \in M \dsep Y \in M \vdash X \cup Y \in M%
  }[\FormulaRefAuto{X \in M \dsep Y \in M \vdash X \cup Y \in M}]%
  \DeltaRow{\textbf{Neue Symbole}}{}%
  \DeltaRow{\makecell[l]{Vereinigungsabgeschlossene\\ Familie}}{}%
}


\chapter{Regularität}

\section{Axiom der Regulariät}

\FormulaAxiomDelta[Regularität (Fundamentalsatz)]{ A \neq \varnothing \vdash \exists x \in A \,(x \cap A = \varnothing) }{
\DeltaRow{Mengen}{A}%
}
\begin{remark}
Dieses Axiom verhindert zyklische Mitgliedschaften, indem jede nicht-leere Menge ein 
\textbf{Minimalelement} enthält.    
\end{remark}

\section{Ausschluss gegenseitiger Mitgliedschaft}

\FormulaThmAuto{a\in b\vdash b\not\in a}
\begin{tabproof}
  \proofstep{1}{a \in b}{\rA}
  \proofstep{2}{b \in a}{\rA}
  \proofstep{ }{\{a, b\} \neq \varnothing}{\FormulaRefAuto{\{a,b\} \neq \varnothing}}
  \proofstep{ }{\exists x \in \{a, b\}\,(x \cap \{a, b\} = \varnothing)}{\FormulaRefAuto{ A \neq \varnothing \vdash \exists x \in A \,(x \cap A = \varnothing) }{3}}
  \proofstep{ }{a \cap \{a, b\} = \varnothing \;\lor\; b \cap \{a, b\} = \varnothing}{\FormulaRefAuto{\exists x\in \{a,b\} P(x)\vdash P(a)\lor P(b)}{4}}
  \proofstep{6}{a \cap \{a, b\} = \varnothing}{\rA}
  \proofstep{2}{a \cap \{a, b\} \neq \varnothing}{\FormulaRefAuto{a\in A\vdash A\cap \{A,a\}\neq\varnothing}{2}}
  \proofstep{2,6}{\bot}{\rBI{6,7}}
  \proofstep{9}{b \cap \{a, b\} = \varnothing}{\rA}
  \proofstep{1}{b \cap \{a, b\} \neq \varnothing}{\FormulaRefAuto{a\in A\vdash A\cap \{A,a\}\neq\varnothing}{1}}
  \proofstep{1,9}{\bot}{\rBI{9,10}}
  \proofstep{1,2}{\bot}{\rOE{5,6,8,9,11}}
  \proofstep{1}{b\notin a}{\rCI{2,12}}
\end{tabproof}

\FormulaThmAuto{a\not\in a}
\begin{tabproof}
  \proofstep{1}{a \in a}{\rA}
  \proofstep{1}{a \notin a}{\FormulaRefAuto{a\in b\vdash b\not\in a}}
  \proofstep{1}{\bot}{\rBI{1,2}}
  \proofstep{}{a\notin a}{\rCE{1,3}}
\end{tabproof}

\FormulaThmAuto{a\cup\{a\}=b\cup\{b\}\eqvdash a=b}
\begin{tabproofsplit}
\proofpart{\(\vdash\)}
  \proofstep{1}{a \cup \{a\} = b \cup \{b\}}{\rA}

  % aus der Gleichheit folgen die beiden Disjunktionen
  \proofstep{}{a \in a \cup \{a\}}{\FormulaRefAuto{a \in A \cup \{a\}}}
  \proofstep{1}{a \in b \cup \{b\}}{\rIE{1,2}}
  \proofstep{1}{a \in b \lor a \in \{b\}}{\FormulaRefAuto{z \in A \cup B \eqvdash z \in A \lor z \in B}{3}}
  \proofstep{1}{a \in b \lor a = b}{\rLRS{\FormulaRefAuto{x \in \{a\} \eqvdash x = a}{},4}}

  \proofstep{}{b \in b \cup \{b\}}{\FormulaRefAuto{a \in A \cup \{a\}}}
  \proofstep{1}{b \in a \cup \{a\}}{\rIE{1,6}}
  \proofstep{1}{b \in a \lor b \in \{a\}}{\FormulaRefAuto{z \in A \cup B \eqvdash z \in A \lor z \in B}{7}}
  \proofstep{1}{b \in a \lor b = a}{\rLRS{\FormulaRefAuto{x \in \{a\} \eqvdash x = a}{},8}}

  % indirekt: a \neq b  ⇒  a \in b und b \in a; Widerspruch mit a∈b ⟹ b∉a
  \proofstep{10}{a \neq b}{\rA}

  % aus (a∈b ∨ a=b) und a≠b folgt a∈b
  \proofstep{1,10}{a \in b}{\FormulaRefAuto{P\lor Q, \neg Q\vdash P}{5,10}}

  % aus (b∈a ∨ b=a) und b≠a (Symmetrie der Ungleichheit) folgt b∈a
  \proofstep{1,10}{b \in a}{\FormulaRefAuto{P\lor Q, \neg Q\vdash P}{9,10}}

  % Widerspruch mit dem Lemma a∈b ⟹ b∉a
  \proofstep{1,10}{\bot}{\rBI{11,12}}
  \proofstep{1}{a = b}{\rCE{10,13}}
\closeproofpart

\proofpart{\(\dashv\)}
  \proofstep{1}{a = b}{\rA}
  \proofstep{1}{a \cup \{a\} = b \cup \{b\}}{\FormulaRefAuto{A=B\vdash A\cup C = B\cup C}{1}}
\closeproofpart
\end{tabproofsplit}

\chapter{Potenzmenge}

\section{Axiom der Potenzmenge}

\FormulaAxiomDelta[Potenzmenge]{\exists B\forall x\bigl(x \in B \leftrightarrow x \subseteq A\bigr)}%
{%
\DeltaRow{Mengen}{A}%
}

\section{Definition der Potenzmenge}
%%begin novalidate
\FormulaDefDelta[Potenzmenge]{\powerset(A) := \iota B\Bigl(\forall x\;\bigl(x \in B \leftrightarrow x \subseteq A\bigr)\Bigr)}%
{%
\DeltaRow{Mengen}{A}%
}
%%end novalidate

\FormulaThmDelta[Potenzmenge]{x \in \powerset(A)\;\eqvdash\; x \subseteq A}%
{%
\DeltaRow{Mengen}{x\dsep A}%
}
\begin{tabproof}
  \proofstep{}{ \forall x\;\bigl(x \in B \leftrightarrow x \subseteq A\bigr) }{\FormulaRefAuto{\powerset(A) := \iota B\Bigl(\forall x\;\bigl(x \in B \leftrightarrow x \subseteq A\bigr)\Bigr)}}
  \proofstep{}{ x \in \powerset(A)\;\leftrightarrow\; x \subseteq A }{\rUE{1}}
\end{tabproof}

\section{Grundlegende Eigenschaften}

\FormulaThmAuto{A \subseteq B \vdash \powerset(A) \subseteq \powerset(B)}
\begin{tabproofwide}
  \proofstepwidestar[1]{A \subseteq B}{\rA}
  \proofstepwide{x \in \powerset(A)}{\rightarrow}{x \subseteq A}%
    {\FormulaRefAuto{\powerset(A) := \iota B\Bigl(\forall x\;\bigl(x \in B \leftrightarrow x \subseteq A\bigr)\Bigr)}}
  \proofstepwide[1]{}{ \rightarrow}{x \subseteq B}%
    {\FormulaRefAuto{A \subseteq B,\, B \subseteq C \vdash A \subseteq C}{2,1}}
  \proofstepwide[1]{}{ \rightarrow}{x \in \powerset(B)}%
    {\FormulaRefAuto{\powerset(A) := \iota B\Bigl(\forall x\;\bigl(x \in B \leftrightarrow x \subseteq A\bigr)\Bigr)}{3}}
  \proofstepwide[1]{x \in \powerset(A)}{\rightarrow}{x \in \powerset(B)}%
    {\rChain{2,4}}
  \proofstepwidestar[1]{\powerset(A) \subseteq \powerset(B)}%
    {\FormulaRefAuto{A \subseteq B := \forall x\,(x\in A \rightarrow x\in B)}{\rUI{5}}}
\end{tabproofwide}

\FormulaThmAuto{a \in A \vdash \{a\} \in \powerset(A)}
\begin{tabproof}
  \proofstep{1}{a \in A}{\rA}
  \proofstep{1}{\{a\} \subseteq A}{\FormulaRefAuto{a \in A \vdash \{a\} \subseteq A}{1}}
  \proofstep{1}{\{a\} \in \powerset(A)}{\FormulaRefAuto{\powerset(A) := \iota B\Bigl(\forall x\;\bigl(x \in B \leftrightarrow x \subseteq A\bigr)\Bigr)}{2}}
\end{tabproof}

\FormulaThmAuto{A \subseteq B,\, a \in \powerset(A) \vdash a \in \powerset(B)}
\begin{tabproof}
  \proofstep{1}{A \subseteq B}{\rA}
  \proofstep{2}{a \in \powerset(A)}{\rA}
  \proofstep{1}{\powerset(A) \subseteq \powerset(B)}{\FormulaRefAuto{A \subseteq B \vdash \powerset(A) \subseteq \powerset(B)}{1}}
  \proofstep{1,2}{a \in \powerset(B)}{\rRE{\rUE{\FormulaRefAuto{A \subseteq B := \forall x\,(x \in A \rightarrow x \in B)}{3}},2}}
\end{tabproof}


\FormulaThmAuto{a \in \powerset(A) \vdash \forall B\,(a \in \powerset(A \cup B))}
\begin{tabproof}
  \proofstep{1}{a \in \powerset(A)}{\rA}
  \proofstep{}{A \subseteq A \cup B}{\FormulaRefAuto{A \subseteq A \cup B}{}}
  \proofstep{1}{a \in \powerset(A \cup B)}%
    {\FormulaRefAuto{A \subseteq B,\, a \in \powerset(A) \vdash a \in \powerset(B)}{2,1}}
  \proofstep{1}{\forall B\,(a \in \powerset(A \cup B))}{\rUI{3}}
\end{tabproof}

\section{Das kartesische Produkt}

\subsection{Existenz des karthesischen Produktes}

\FormulaThmAuto{a \in A,\, b \in B \vdash (a,b) \in \powerset(\powerset(A \cup B))}
\begin{tabproof}
  \proofstep{1}{a \in A}{\rA}
  \proofstep{2}{b \in B}{\rA}
  \proofstep{1,2}{\{a,b\} \subseteq A \cup B}{\FormulaRefAuto{a \in A,\, b \in B \vdash \{a,b\} \subseteq A \cup B}{1,2}}
  \proofstep{1,2}{\{a,b\} \in \powerset(A \cup B)}{\FormulaRefAuto{\powerset(A) := \iota B\bigl(\forall x\,(x \in B \leftrightarrow x \subseteq A)\bigr)}{3}}
  \proofstep{1}{a \in A \cup B}{\FormulaRefAuto{z \in A \vdash z \in A \cup B}{1}}
  \proofstep{1}{\{a\} \in \powerset(A \cup B)}{\FormulaRefAuto{a \in A \vdash \{a\} \in \powerset(A)}{5}}
  \proofstep{1,2}{\{\{a\},\{a,b\}\} \subseteq \powerset(A \cup B)}{\FormulaRefAuto{a \in A,\, b \in B \vdash \{a,b\} \subseteq A \cup B}{6,4}}
  \proofstep{1,2}{\{\{a\},\{a,b\}\} \in \powerset(\powerset(A \cup B))}{\FormulaRefAuto{\powerset(A) := \iota B\bigl(\forall x\,(x \in B \leftrightarrow x \subseteq A)\bigr)}{7}}
  \proofstep{1,2}{(a,b) \in \powerset(\powerset(A \cup B))}{\rIE{\FormulaRefAuto{(a, b) := \{ \{ a \}, \{ a, b \} \}},8}}
\end{tabproof}

\FormulaThmAuto{\exists a \in A\, \exists b \in B \big(x=(a,b)\big)\vdash x \in \powerset(\powerset(A \cup B))}
\begin{tabproof}
  \proofstep{1}{\exists a \in A\, \exists b \in B \big(x=(a,b)\big)}{\rA}
  \proofstep{2}{a \in A\land b \in B\land x=(a,b)}{\rA}
  \proofstep{2}{a \in A}{\rAEa{2}}
  \proofstep{2}{b \in B}{\FormulaRefAuto{P\land Q\land R\vdash Q}{2}}
  \proofstep{2}{x=(a,b)}{\rAEb{2}}
  \proofstep{2}{(a,b) \in \powerset(\powerset(A \cup B))}{\FormulaRefAuto{a \in A,\, b \in B \vdash (a,b) \in \powerset(\powerset(A \cup B))}{3,4}}
  \proofstep{2}{x \in \powerset(\powerset(A \cup B))}{\rIE{5,6}}  
  \proofstep{1}{x \in \powerset(\powerset(A \cup B))}{\rEE{1,2,7}}  
\end{tabproof}

\FormulaThmAuto[Eindeutige Existenz des kartesischen Produkts]{\exists! C\, \forall (a,b) \bigl(x \in C \leftrightarrow \exists a \in A\, \exists b \in B \big(x=(a,b)\big)\bigr)}
\begin{tabproof}
  \proofstep{}{ \exists! C\, \forall (a,b) \bigl((a,b) \in C \leftrightarrow \exists a \in A\, \exists b \in B}{\multirow{2}{*}{\rEI{\FormulaRefAuto{\forall x(P(x)\rightarrow x\in A)\vdash \exists! B(\forall x(x\in B\leftrightarrow P(x)))}{\rUI{\FormulaRefAuto{\exists a \in A\, \exists b \in B \big(x=(a,b)\big)\vdash x \in \powerset(\powerset(A \cup B))}}}}}}
  \proofstepstar{}{ \big(x=(a,b)\big)\bigr)}{}
\end{tabproof}
%%begin novalidate
\FormulaDefAuto[Kartesisches Produkt ($A \times B$)]{A \times B := \iota C\, \forall x\, \bigl(x \in C \leftrightarrow \exists a \in A\, \exists b \in B \big(x=(a,b)\big)\bigr)}
%%end novalidate

\FormulaThmAuto{x\in A\times B\eqvdash \exists a \in A\, \exists b \in B\,x=(a,b)}
\begin{tabproof}
    \proofstep{}{x\in A\times B\leftrightarrow \exists a \in A\, \exists b \in B \big(x=(a,b)\big)}{\FormulaRefAuto{A \times B := \iota C\, \forall x\, \bigl(x \in C \leftrightarrow \exists a \in A\, \exists b \in B \big(x=(a,b)\big)\bigr)}}
\end{tabproof}



\subsection{Grundlegende Eigenschaften}

\FormulaThmAuto[Kartesisches Produkt ($A \times B$)]{(a,b)\in A\times B\eqvdash a\in A\land b\in B}
\begin{tabproofsplit}
\proofpart{\(\vdash\)}
  \proofstep{1}{(a,b)\in A\times B}{\rA}
  \proofstep{1}{\exists c\in A\exists d\in B((a,b)=(c,d))}{\FormulaRefAuto{A \times B := \iota C\, \forall x\, \bigl(x \in C \leftrightarrow \exists a \in A\, \exists b \in B \big(x=(a,b)\big)\bigr)}{1}}
  \proofstep{3}{c\in A\land d\in B\land (c,d)=(a,b)}{\rA}
  \proofstep{3}{c\in A}{\rAEa{3}}
  \proofstep{3}{d\in B}{\FormulaRefAuto{P\land Q\land R\vdash Q}{3}}
  \proofstep{3}{(c,d)=(a,b)}{\rAEb{3}}
  \proofstep{3}{c=a\land d=b}{\FormulaRefAuto{(a,b)=(c,d)\eqvdash a=c\land b=d}{6}}
  \proofstep{3}{c=a}{\rAEa{7}}
  \proofstep{3}{d=b}{\rAEb{7}}
  \proofstep{3}{a\in A}{\rIE{8,4}}
  \proofstep{3}{b\in B}{\rIE{9,5}}
  \proofstep{3}{a\in A\land b\in B}{\rAI{10,11}}
  \proofstep{1}{a\in A\land b\in B}{\rEE{2,3,12}}
\closeproofpart
\proofpart{\(\dashv\)}
  \proofstep{1}{a\in A\land b\in B}{\rA}
  \proofstep{}{(a,b)=(a,b)}{\rII}
  \proofstep{1}{a\in A\land b\in B\land (a,b)=(a,b)}{\rAI{1,2}}
  \proofstep{1}{\exists a\in A\exists b\in B((a,b)=(a,b))}{\rEI{3}}
  \proofstep{1}{(a,b)\in A\times B}{\FormulaRefAuto{A \times B := \iota C\, \forall x\, \bigl(x \in C \leftrightarrow \exists a \in A\, \exists b \in B \big(x=(a,b)\big)\bigr)}{4}}
\closeproofpart
\end{tabproofsplit}

\FormulaThmAuto{(a,b)\in A\times B\eqvdash (b,a)\in B\times A}
\begin{tabproofwide}
  \proofstepwide{(a,b)\in A\times B}{\leftrightarrow}{a\in A\land b\in B}%
    {\FormulaRefAuto{(a,b)\in A\times B\eqvdash a\in A\land b\in B}}
  \proofstepwide{}{\leftrightarrow}{b\in B\land a\in A}%
    {\FormulaRefAuto{P\land Q\eqvdash Q\land P}{1}}
  \proofstepwide{}{\leftrightarrow}{(b,a)\in B\times A}%
    {\FormulaRefAuto{(a,b)\in A\times B\eqvdash a\in A\land b\in B}{2}}
  \proofstepwide{(a,b)\in A\times B}{\leftrightarrow}{(b,a)\in B\times A}%
    {\rChain{1,3}}
\end{tabproofwide}


\FormulaThmAuto{(a,b)\in A\times B\vdash a\in A}
\begin{tabproof}
      \proofstep{1}{(a,b)\in A\times B}{\rA}
      \proofstep{1}{a\in A\land b\in B}{\FormulaRefAuto{(a,b)\in A\times B\eqvdash a\in A\land b\in B}{1}}
      \proofstep{1}{a\in A}{\rAEa{2}}
\end{tabproof}

\FormulaThmAuto{(a,b)\in A\times B\vdash b\in B}
\begin{tabproof}
      \proofstep{1}{(a,b)\in A\times B}{\rA}
      \proofstep{1}{a\in A\land b\in B}{\FormulaRefAuto{(a,b)\in A\times B\eqvdash a\in A\land b\in B}{1}}
      \proofstep{1}{a\in B}{\rAEb{2}}
\end{tabproof}


\FormulaThmAuto{a\in A\dsep b\in B\vdash (a,b)\in A\times B}
\begin{tabproof}
      \proofstep{1}{a\in A}{\rA}
      \proofstep{2}{b\in B}{\rA}
      \proofstep{1,2}{a\in A\land b\in B}{\rAI{1,2}}
      \proofstep{1,2}{(a,b)\in A\times B}{\FormulaRefAuto{(a,b)\in A\times B\eqvdash a\in A\land b\in B}{3}}
\end{tabproof}

% ============================================================
%  Dreifache Produkte als verschachtelte Paare
% ============================================================

% ((a,b),c) ∈ (A×B)×C  ⇒  a∈A ∧ b∈B ∧ c∈C
\FormulaThmAuto{((a,b),c)\in (A\times B)\times C\ \eqvdash\ a\in A\land b\in B\land c\in C}
\begin{tabproofsplit}
\proofpart{\(\vdash\)}
  \proofstep{1}{((a,b),c)\in (A\times B)\times C}{\rA}
  % Erst Projektion aus (A×B)×C
  \proofstep{1}{(a,b)\in A\times B\ \land\ c\in C}{\FormulaRefAuto{(a,b)\in A\times B\eqvdash a\in A\land b\in B}{1}}
  \proofstep{1}{(a,b)\in A\times B}{\rAEa{2}}
  \proofstep{1}{c\in C}{\rAEb{2}}
  % Dann Projektion aus A×B
  \proofstep{1}{a\in A\land b\in B}{\FormulaRefAuto{(a,b)\in A\times B\eqvdash a\in A\land b\in B}{3}}
  \proofstep{1}{a\in A}{\rAEa{5}}
  \proofstep{1}{b\in B}{\rAEb{5}}
  \proofstep{1}{a\in A\land b\in B\land c\in C}{\rAI{\rAI{6,7},4}}
\closeproofpart
\proofpart{\(\dashv\)}
  \proofstep{1}{a\in A\land b\in B\land c\in C}{\rA}
  \proofstep{1}{a\in A\land b\in B}{\rAEa{1}}
  \proofstep{1}{c\in C}{\rAEb{1}}
  \proofstep{1}{(a,b)\in A\times B}{\FormulaRefAuto{a\in A\dsep b\in B\vdash (a,b)\in A\times B}{2}}
  \proofstep{1}{((a,b),c)\in (A\times B)\times C}{\FormulaRefAuto{a\in A, b\in B\vdash (a,b)\in A\times B}{4,3}}
\closeproofpart
\end{tabproofsplit}

% (a,(b,c)) ∈ A×(B×C)  ⇒  a∈A ∧ b∈B ∧ c∈C
\FormulaThmAuto{(a,(b,c))\in A\times (B\times C)\ \eqvdash\ a\in A\land b\in B\land c\in C}
\begin{tabproofsplit}
\proofpart{\(\vdash\)}
  \proofstep{1}{(a,(b,c))\in A\times (B\times C)}{\rA}
  % Erst Projektion aus A×(B×C)
  \proofstep{1}{a\in A\ \land\ (b,c)\in B\times C}{\FormulaRefAuto{(a,b)\in A\times B\eqvdash a\in A\land b\in B}{1}}
  \proofstep{1}{a\in A}{\rAEa{2}}
  \proofstep{1}{(b,c)\in B\times C}{\rAEb{2}}
  % Dann Projektion aus B×C
  \proofstep{1}{b\in B\land c\in C}{\FormulaRefAuto{(a,b)\in A\times B\eqvdash a\in A\land b\in B}{4}}
  \proofstep{1}{b\in B}{\rAEa{5}}
  \proofstep{1}{c\in C}{\rAEb{5}}
  \proofstep{1}{a\in A\land b\in B\land c\in C}{\rAI{3,\rAI{6,7}}}
\closeproofpart
\proofpart{\(\dashv\)}
  \proofstep{1}{a\in A\land b\in B\land c\in C}{\rA}
  \proofstep{1}{a\in A}{\rAEa{1}}
  \proofstep{1}{b\in B\land c\in C}{\rAEb{1}}
  \proofstep{1}{(b,c)\in B\times C}{\FormulaRefAuto{a\in A\dsep b\in B\vdash (a,b)\in A\times B}{3}}
  \proofstep{1}{(a,(b,c))\in A\times (B\times C)}{\FormulaRefAuto{(a,b)\in A\times B\eqvdash a\in A\land b\in B}{2,4}}
\closeproofpart
\end{tabproofsplit}

% Einführungsrichtungen getrennt (falls du sie einzeln haben willst):

% a∈A, b∈B, c∈C  ⊢  (a,(b,c))∈A×(B×C)
\FormulaThmAuto{a\in A,\, b\in B,\, c\in C\ \vdash\ (a,(b,c))\in A\times (B\times C)}
\begin{tabproof}
  \proofstep{1}{a\in A}{\rA}
  \proofstep{2}{b\in B}{\rA}
  \proofstep{3}{c\in C}{\rA}
  \proofstep{2,3}{(b,c)\in B\times C}{\FormulaRefAuto{a\in A\dsep b\in B\vdash (a,b)\in A\times B}{2,3}}
  \proofstep{1,4}{(a,(b,c))\in A\times (B\times C)}{\FormulaRefAuto{(a,b)\in A\times B\eqvdash a\in A\land b\in B}{1,4}}
\end{tabproof}

% a∈A, b∈B, c∈C  ⊢  ((a,b),c)∈(A×B)×C
\FormulaThmAuto{a\in A,\, b\in B,\, c\in C\ \vdash\ ((a,b),c)\in (A\times B)\times C}
\begin{tabproof}
  \proofstep{1}{a\in A}{\rA}
  \proofstep{2}{b\in B}{\rA}
  \proofstep{3}{c\in C}{\rA}
  \proofstep{1,2}{(a,b)\in A\times B}{\FormulaRefAuto{a\in A\dsep b\in B\vdash (a,b)\in A\times B}{1,2}}
  \proofstep{4,3}{((a,b),c)\in (A\times B)\times C}{\FormulaRefAuto{(a,b)\in A\times B\eqvdash a\in A\land b\in B}{4,3}}
\end{tabproof}

\chapter{Relationen}

\FormulaDefDelta[Relation (Menge geordneter Paare)]{F \subseteq A \times B}{
\DeltaRow{Mengen}{A\dsep B\dsep F}
}

\section{Begriff der totalen Relation}

\subsection{Axiom der Totalität}

\FormulaAxiomDelta[Totalität]
{x\in A \vdash \exists y\,(x,y)\in F}
{
\DeltaRow{Mengen}{x\dsep y\dsep A\dsep F}
}
\subsection{Definition der totalen Relation}

\FormulaDefDeltaK[Begriff der totalen Relation]{\TotRel{R,A,B}}{Totale Relation}{
  \DeltaRow{Mengen}{A \dsep B \dsep R \dsep x \dsep y}
  \DeltaRow{\textbf{Axiome}}{}
  \DeltaRow{Menge geordneter Paare}
           {R \subseteq A \times B}
           [\FormulaRefAuto{F \subseteq A \times B}]
  \DeltaRow{Totalität}
           {x \in A \vdash \exists y\,(x,y)\in R}
           [\FormulaRefAuto{x\in A \vdash \exists y\,(x,y)\in F}]
  \DeltaRow{\textbf{Neue Symbole}}{}
  \DeltaPrem{Totale Relationen}{}
}

\subsection{Beispiele}

\subsubsection{Mitgliedschaftsrelation}

% ------------------------------------------------------------
% 0) Definition: Mitgliedschaftsrelation der Familie
% ------------------------------------------------------------
\FormulaDefDelta[Mitgliedschaftsrelation]{%
  \MemRel(M) \coloneqq \{\, (X,a)\in M\times \bigcup M \mid a\in X \,\}
}{
  \DeltaRow{Mengen}{M \dsep X \dsep a}
}

% ------------------------------------------------------------
% 1) Lemma: \MemRel(M) ist Teilmenge des Produkts
% ------------------------------------------------------------
\FormulaThmDelta[Relation]
{\MemRel(M) \subseteq M\times \bigcup M}
{
  \DeltaRow{Mengen}{M \dsep X \dsep a}
}
\begin{tabproof}
  \proofstep{}{\MemRel(M)=\{\, (X,a)\in M\times \bigcup M \mid a\in X \,\}}{\rA}
  \proofstep{}{\MemRel(M) \subseteq M\times \bigcup M}{%
    \FormulaRefAuto{\{\,x\in A \mid P(x)\,\}\subseteq A}{}}
\end{tabproof}

% ------------------------------------------------------------
% 2) Lemma: Aus (X,a)\in \MemRel(M) folgt X\in M, a\in \bigcup M und a\in X
% ------------------------------------------------------------
\FormulaThmDelta
{(X,a)\in \MemRel(M) \vdash X\in M \land a\in \bigcup M \land a\in X}
{
  \DeltaRow{Mengen}{M \dsep X \dsep a}
}
\begin{tabproof}
  \proofstep{1}{(X,a)\in \MemRel(M)}{\rA}
  \proofstep{1}{(X,a)\in M\times \bigcup M \land a\in X}{%
    \FormulaRefAuto{x \in \{u \in A \mid P(u)\} \eqvdash x \in A \land P(x)}{1}}
  \proofstep{1}{(X,a)\in M\times \bigcup M}{\rAEa{2}}
  \proofstep{1}{a\in X}{\rAEb{2}}
  \proofstep{1}{X\in M \land a\in \bigcup M}{%
    \FormulaRefAuto{(a,b)\in A\times B \eqvdash a\in A \land b\in B}{3}}
  \proofstep{1}{X\in M}{\rAEa{5}}
  \proofstep{1}{a\in \bigcup M}{\rAEb{5}}
  \proofstep{1}{X\in M \land a\in \bigcup M \land a\in X}{\FormulaRefAuto{P\dsep Q\dsep R\vdash P\land Q\land R}{6,7,4}}
\end{tabproof}

\FormulaThmDelta
{(X,a)\in \MemRel(M) \vdash X\in M}
{
  \DeltaRow{Mengen}{M \dsep X \dsep a}
}
\begin{tabproof}
  \proofstep{1}{(X,a)\in \MemRel(M)}{\rA}
  \proofstep{1}{X\in M \land a\in \bigcup M \land a\in X}{\FormulaRefAuto{(X,a)\in \MemRel(M) \vdash X\in M \land a\in \bigcup M \land a\in X}{1}}
  \proofstep{1}{X\in M}{\rAEa{2}}
\end{tabproof}

% ------------------------------------------------------------
% 3) Lemma: Aus X\in M und a\in X folgt (X,a)\in \MemRel(M)
% ------------------------------------------------------------
\FormulaThmDelta
{X\in M \dsep a\in X \vdash (X,a)\in \MemRel(M)}
{
  \DeltaRow{Mengen}{M \dsep X \dsep a}
}
\begin{tabproof}
  \proofstep{1}{X\in M}{\rA}
  \proofstep{2}{a\in X}{\rA}
  \proofstep{1,2}{a\in \bigcup M}{%
    \FormulaRefAuto{B\in A\dsep x\in B\vdash x \in \bigcup A}{1,2}}
  \proofstep{1,2}{(X,a)\in M\times \bigcup M}{%
    \FormulaRefAuto{(a,b)\in A\times B \eqvdash a\in A \land b\in B}{1,3}}
  \proofstep{}{\{\, (X,a)\in M\times \bigcup M \mid a\in X \,\}=\MemRel(M)}{\FormulaRefAuto{a=b\vdash b=a}{\FormulaRefAuto{\MemRel(M) \coloneqq \{\, (X,a)\in M\times \bigcup M \mid a\in X \,\}}}}
  \proofstep{1,2}{(X,a)\in \{\, (X,a)\in M\times \bigcup M \mid a\in X \,\}}{%
    \FormulaRefAuto{x\in A \dsep P(x) \vdash x\in \{\,x\in A \mid P(x)\,\}}{4,2}}
  \proofstep{1,2}{(X,a)\in \MemRel(M)}{%
    \rIE{5,6}}
\end{tabproof}

% ============================================================
% (1) Totalität von \MemRel(M) (als eigenes Lemma)
% ============================================================

\FormulaThmDelta[Totalität von $\MemRel(M)$]{%
  X\in M \vdash \exists a\,\bigl((X,a)\in \MemRel(M)\bigr)
}{
  \DeltaRow{Mengen}{M \dsep X \dsep a}
  \DeltaPrem{\makecell[l]{Paarweise disjunkte\\ nichtleere Familie}}{\DisjFam(M)}
}
\begin{tabproof}
  \proofstep{1}{X\in M}{\rA}

  \proofstep{1}{X\neq\varnothing}{%
    \FormulaRefAuto{X\in A \vdash X\neq\varnothing}{1}}

  \proofstep{1}{\exists a\,(a\in X)}{%
    \FormulaRefAuto{A\neq\varnothing \eqvdash \exists x\,(x\in A)}{2}}

  \proofstep{4}{a\in X}{\rA}

  \proofstep{1,4}{(X,a)\in \MemRel(M)}{%
    \FormulaRefAuto{X\in M \dsep a\in X \vdash (X,a)\in \MemRel(M)}{1,4}}

  \proofstep{1,4}{\exists a\,\bigl((X,a)\in \MemRel(M)\bigr)}{\rEI{5}}

  \proofstep{1}{\exists a\,\bigl((X,a)\in \MemRel(M)\bigr)}{\rEE{3,4,6}}

\end{tabproof}


% ============================================================
% (2) TotRel(\MemRel(M),M,⋃M) als "Definition-Style"-Theorem (ohne tabproof)
% ============================================================

\FormulaThmDelta[Totale Relation]{\TotRel{\MemRel(M),M,\bigcup M}}{
  \DeltaRow{Mengen}{M \dsep X \dsep a}
  \DeltaPrem{\makecell[l]{Paarweise disjunkte\\ nichtleere Familie}}{\DisjFam(M)}
  \DeltaRow{\textbf{Begründung}}{}
  \DeltaRow{Menge geordneter Paare}{\MemRel(M) \subseteq M\times \bigcup M}
    [\FormulaRefAuto{\MemRel(M) \subseteq M\times \bigcup M}]
  \DeltaRow{Totalität}{X\in M \vdash \exists a\,\bigl((X,a)\in \MemRel(M)\bigr)}
    [\FormulaRefAuto{X\in M \vdash \exists a\,\bigl((X,a)\in \MemRel(M)\bigr)}]
}{}

\section{Begriff der Ordnung}

\subsection{Axiome einer partiellen Ordnung}

\FormulaAxiomDelta[Reflexivität]{%
  x\in A \vdash (x,x)\in R%
}{
  \DeltaRow{Mengen}{A \dsep R \dsep x}
}

\FormulaAxiomDelta[Transitivität]{%
  (x,y)\in R \dsep (y,z)\in R \vdash (x,z)\in R%
}{
  \DeltaRow{Mengen}{A \dsep R \dsep x \dsep y \dsep z}
}

\FormulaAxiomDelta[Antisymmetrie]{%
  (x,y)\in R \dsep (y,x)\in R \vdash x=y%
}{
  \DeltaRow{Mengen}{A \dsep R \dsep x \dsep y}
}

\subsection{Definition der partiellen Ordnung}

\FormulaDefDeltaK[Begriff der partiellen Ordnung]{\PartOrd{R,A}}{Partielle Ordnung}{
  \DeltaRow{Mengen}{A \dsep R \dsep x \dsep y \dsep z}
  \DeltaRow{\textbf{Axiome}}{}
  \DeltaRow{Relation auf \(A\)}
           {R \subseteq A \times A}
           [\FormulaRefAuto{F \subseteq A \times B}]
  \DeltaRow{Reflexivität}
           {x\in A \vdash (x,x)\in R}
           [\FormulaRefAuto{x\in A \vdash (x,x)\in R}]
  \DeltaRow{Transitivität}
           {(x,y)\in R \dsep (y,z)\in R \vdash (x,z)\in R}
           [\FormulaRefAuto{(x,y)\in R \dsep (y,z)\in R \vdash (x,z)\in R}]
  \DeltaRow{Antisymmetrie}
           {(x,y)\in R \dsep (y,x)\in R \vdash x=y}
           [\FormulaRefAuto{(x,y)\in R \dsep (y,x)\in R \vdash x=y}]
  \DeltaRow{\textbf{Neue Symbole}}{}
  \DeltaPrem{Partielle Ordnungen}{}
}

\subsection{Axiom einer totalen Ordnung}

\FormulaAxiomDelta[Vergleichbarkeit]{%
  x\in A \dsep y\in A \vdash (x,y)\in R \lor (y,x)\in R%
}{
  \DeltaRow{Mengen}{A \dsep R \dsep x \dsep y}
}

\subsection{Definition der totalen Ordnung}

\FormulaDefDeltaK[Begriff der totalen Ordnung]{\TotOrd{R,A}}{Totale Ordnung}{
  \DeltaRow{Mengen}{A \dsep R \dsep x \dsep y \dsep z}
  \DeltaRow{\textbf{Axiome}}{}
  \DeltaRow{Partielle Ordnung}
           {\PartOrd{R,A}}
           [\FormulaRefAuto{Partielle Ordnung}]
  \DeltaRow{Vergleichbarkeit}
           {x\dsep y\in A \vdash (x,y)\in R \lor (y,x)\in R}
           [\FormulaRefAuto{x\in A \dsep y\in A \vdash (x,y)\in R \lor (y,x)\in R}]
  \DeltaRow{\textbf{Neue Symbole}}{}
  \DeltaPrem{Totale Ordnungen}{}
}

% ============================================================
% Kleinste / groesste Elemente, Minimum / Maximum
% (auf Basis einer partiellen Ordnung)
% ============================================================

% ============================================================
% Minimum / Maximum (als eindeutige Terme via \iota)
% Kleinstes/Groesstes Element als Spezialfall T=A
% ============================================================

\subsection{Minimum und Maximum}

% ------------------------------------------------------------
% (1) Eindeutigkeit des Minimum-Kriteriums
% ------------------------------------------------------------
\FormulaThmDelta[Eindeutigkeit eines Minimums]{%
  \exists m\in T\,\forall x\in T\,\bigl((m,x)\in R\bigr)
  \vdash \exists! m\in T\,\forall x\in T\,\bigl((m,x)\in R\bigr)%
}{
  \DeltaRow{Mengen}{A \dsep R \dsep T \dsep m \dsep n \dsep x}
  \DeltaPrem{Partielle Ordnung}{\PartOrd{R,A}}
}
\begin{tabproof}
  \proofstep{1}{\exists m\in T\,\forall x\in T\,\bigl((m,x)\in R\bigr)}{\rA}

  \proofstep{2}{m\in T \land \forall x\in T\,\bigl((m,x)\in R\bigr)}{\rA}
  \proofstep{3}{n\in T \land \forall x\in T\,\bigl((n,x)\in R\bigr)}{\rA}

  \proofstep{2}{m\in T}{\rAEa{2}}
  \proofstep{2}{\forall x\in T\,\bigl((m,x)\in R\bigr)}{\rAEb{2}}
  \proofstep{3}{n\in T}{\rAEa{3}}
  \proofstep{3}{\forall x\in T\,\bigl((n,x)\in R\bigr)}{\rAEb{3}}

  \proofstep{2,3}{(m,n)\in R}{\rRE{6,\rUE{5}}}
  \proofstep{2,3}{(n,m)\in R}{\rRE{4,\rUE{7}}}

  \proofstep{2,3}{m=n}{%
    \FormulaRefAuto{(x,y)\in R \dsep (y,x)\in R \vdash x=y}{8,9}}

  \proofstep{1}{\exists! m\in T\,\forall x\in T\,\bigl((m,x)\in R\bigr)}{%
    \UEI{1,2,3,10}}
\end{tabproof}


% ------------------------------------------------------------
% (2) Definition: Minimum als \iota-Term (partielle Definition)
% ------------------------------------------------------------
\FormulaDefDelta[Minimum]{%
  \begin{aligned}
    &\exists m\in T\,\forall x\in T\,\bigl((m,x)\in R\bigr)\\
    &\vdash \Min{R,A}(T)\coloneqq
      \iota m\in T\,\forall x\in T\,\bigl((m,x)\in R\bigr)
  \end{aligned}%
}{
  \DeltaRow{Mengen}{A \dsep R \dsep T \dsep m \dsep x}
  \DeltaPrem{Partielle Ordnung}{\PartOrd{R,A}}
}

% ------------------------------------------------------------
% (3) Eindeutigkeit des Maximum-Kriteriums (kurz mit  \exists m\in T ...)
% ------------------------------------------------------------
\FormulaThmDelta[Eindeutigkeit eines Maximums]{%
  \exists m\in T\,\forall x\in T\,\bigl((x,m)\in R\bigr)
  \vdash \exists! m\in T\,\forall x\in T\,\bigl((x,m)\in R\bigr)%
}{
  \DeltaRow{Mengen}{A \dsep R \dsep T \dsep m \dsep n \dsep x}
  \DeltaPrem{Partielle Ordnung}{\PartOrd{R,A}}
}
\begin{tabproof}
  \proofstep{1}{\exists m\in T\,\forall x\in T\,\bigl((x,m)\in R\bigr)}{\rA}

  \proofstep{2}{m\in T \land \forall x\in T\,\bigl((x,m)\in R\bigr)}{\rA}
  \proofstep{3}{n\in T \land \forall x\in T\,\bigl((x,n)\in R\bigr)}{\rA}

  \proofstep{2}{m\in T}{\rAEa{2}}
  \proofstep{2}{\forall x\in T\,\bigl((x,m)\in R\bigr)}{\rAEb{2}}
  \proofstep{3}{n\in T}{\rAEa{3}}
  \proofstep{3}{\forall x\in T\,\bigl((x,n)\in R\bigr)}{\rAEb{3}}

  \proofstep{2,3}{(m,n)\in R}{\rRE{4,\rUE{7}}}
  \proofstep{2,3}{(n,m)\in R}{\rRE{6,\rUE{5}}}

  \proofstep{2,3}{m=n}{%
    \FormulaRefAuto{(x,y)\in R \dsep (y,x)\in R \vdash x=y}{8,9}}

  \proofstep{1}{\exists! m\in T\,\forall x\in T\,\bigl((x,m)\in R\bigr)}{%
    \UEI{1,2,3,10}}
\end{tabproof}

% ------------------------------------------------------------
% (4) Definition: Maximum als \iota-Term (partielle Definition)
% ------------------------------------------------------------
\FormulaDefDelta[Maximum]{%
  \begin{aligned}
    &\exists m\in T\,\forall x\in T\,\bigl((x,m)\in R\bigr)\\
    &\vdash \Max{R,A}(T)\coloneqq
      \iota m\in T\,\forall x\in T\,\bigl((x,m)\in R\bigr)
  \end{aligned}%
}{
  \DeltaRow{Mengen}{A \dsep R \dsep T \dsep m \dsep x}
  \DeltaPrem{Partielle Ordnung}{\PartOrd{R,A}}
}



% ============================================================
% Schranken
% ============================================================

\subsection{Schranken}

\subsubsection{Untere und obere Schrankenmengen}

\FormulaDefDelta[Untere Schrankenmenge]{%
  T\subseteq A \vdash
  \LB_{R,A}(T)\coloneqq
  \{\, a\in A \mid \forall x\in T\,\bigl((a,x)\in R\bigr)\,\}%
}{
  \DeltaRow{Mengen}{A \dsep R \dsep T \dsep a \dsep x}
  \DeltaPrem{Partielle Ordnung}{\PartOrd{R,A}}
}

\FormulaDefDelta[Obere Schrankenmenge]{%
  T\subseteq A \vdash
  \UB_{R,A}(T)\coloneqq
  \{\, a\in A \mid \forall x\in T\,\bigl((x,a)\in R\bigr)\,\}%
}{
  \DeltaRow{Mengen}{A \dsep R \dsep T \dsep a \dsep x}
  \DeltaPrem{Partielle Ordnung}{\PartOrd{R,A}}
}

\subsubsection{Infimum und Supremum}

\FormulaDefDelta[Infimum]{%
  \begin{aligned}
    &T\subseteq A \dsep \exists m\in \LB_{R,A}(T)\,
      \forall a\in \LB_{R,A}(T)\,\bigl((a,m)\in R\bigr)\\
    &\vdash \inf_{R,A}(T)\coloneqq \Max{R,A}\bigl(\LB_{R,A}(T)\bigr)
  \end{aligned}%
}{
  \DeltaRow{Mengen}{A \dsep R \dsep T \dsep m \dsep a}
  \DeltaPrem{Partielle Ordnung}{\PartOrd{R,A}}
}

\FormulaDefDelta[Supremum]{%
  \begin{aligned}
    &T\subseteq A \dsep \exists m\in \UB_{R,A}(T)\,
      \forall a\in \UB_{R,A}(T)\,\bigl((m,a)\in R\bigr)\\
    &\vdash \sup_{R,A}(T)\coloneqq \Min{R,A}\bigl(\UB_{R,A}(T)\bigr)
  \end{aligned}%
}{
  \DeltaRow{Mengen}{A \dsep R \dsep T \dsep m \dsep a}
  \DeltaPrem{Partielle Ordnung}{\PartOrd{R,A}}
}


\subsection{Restriktion einer partiellen Ordnung}

% ============================================================
% Restriktion einer partiellen Ordnung auf T
% ============================================================

% ------------------------------------------------------------
% (0) Lemma: Relation auf T
% ------------------------------------------------------------
\FormulaThmDelta[Relation auf $T$]{%
  R\cap (T\times T)\subseteq T\times T%
}{
  \DeltaRow{Mengen}{R \dsep T}
}
\begin{tabproof}
  \proofstep{}{\;R\cap (T\times T)\subseteq T\times T}{%
    \FormulaRefAuto{A \cap B \subseteq B}}
\end{tabproof}

% ------------------------------------------------------------
% (1) Lemma: Reflexivität der Restriktion
% ------------------------------------------------------------
\FormulaThmDelta[Reflexivität der Restriktion]{%
  T\subseteq A \dsep x\in T
  \vdash (x,x)\in R\cap (T\times T)%
}{
  \DeltaRow{Mengen}{A \dsep R \dsep T \dsep x}
  \DeltaPrem{Teilmenge}{T\subseteq A}
  \DeltaPrem{Partielle Ordnung}{\PartOrd{R,A}}
}
\begin{tabproof}
  \proofstep{1}{T\subseteq A}{\rA}
  \proofstep{2}{x\in T}{\rA}

  \proofstep{1,2}{x\in A}{\FormulaRefAuto{A\subseteq B, x\in A \vdash x\in B}{1,2}}

  \proofstep{1,2}{(x,x)\in R}{%
    \FormulaRefAuto{x\in A \vdash (x,x)\in R}{3}}

  \proofstep{2}{(x,x)\in T\times T}{%
    \FormulaRefAuto{a\in A \dsep b\in B \vdash (a,b)\in A\times B}{2,2}}

  \proofstep{1,2}{(x,x)\in R\cap (T\times T)}{%
    \FormulaRefAuto{x\in A \dsep x\in B \vdash x\in A\cap B}{4,5}}
\end{tabproof}

% ------------------------------------------------------------
% (2) Lemma: Transitivität der Restriktion
% ------------------------------------------------------------
\FormulaThmDelta[Transitivität der Restriktion]{%
  (x,y)\in R\cap (T\times T) \dsep (y,z)\in R\cap (T\times T)
  \vdash (x,z)\in R\cap (T\times T)%
}{
  \DeltaRow{Mengen}{A \dsep R \dsep T \dsep x \dsep y \dsep z}
  \DeltaPrem{Partielle Ordnung}{\PartOrd{R,A}}
}
\begin{tabproof}
  \proofstep{1}{(x,y)\in R\cap (T\times T)}{\rA}
  \proofstep{2}{(y,z)\in R\cap (T\times T)}{\rA}

  \proofstep{1}{(x,y)\in R}{\FormulaRefAuto{x\in A\cap B \vdash x\in A}{1}}
  \proofstep{2}{(y,z)\in R}{\FormulaRefAuto{x\in A\cap B \vdash x\in A}{2}}

  \proofstep{1,2}{(x,z)\in R}{%
    \FormulaRefAuto{(x,y)\in R \dsep (y,z)\in R \vdash (x,z)\in R}{3,4}}

  \proofstep{1}{(x,y)\in T\times T}{\FormulaRefAuto{x\in A\cap B \vdash x\in B}{1}}
  \proofstep{2}{(y,z)\in T\times T}{\FormulaRefAuto{x\in A\cap B \vdash x\in B}{2}}

  \proofstep{1}{x\in T}{\FormulaRefAuto{(a,b)\in A\times B \vdash a\in A}{6}}
  \proofstep{2}{z\in T}{\FormulaRefAuto{(a,b)\in A\times B \vdash b\in B}{7}}

  \proofstep{1,2}{(x,z)\in T\times T}{%
    \FormulaRefAuto{a\in A \dsep b\in B \vdash (a,b)\in A\times B}{8,9}}

  \proofstep{1,2}{(x,z)\in R\cap (T\times T)}{%
    \FormulaRefAuto{x\in A \dsep x\in B \vdash x\in A\cap B}{5,10}}
\end{tabproof}


% ------------------------------------------------------------
% (3) Lemma: Antisymmetrie der Restriktion
% ------------------------------------------------------------
\FormulaThmDelta[Antisymmetrie der Restriktion]{%
  (x,y)\in R\cap (T\times T) \dsep (y,x)\in R\cap (T\times T)
  \vdash x=y%
}{
  \DeltaRow{Mengen}{A \dsep R \dsep T \dsep x \dsep y}
  \DeltaPrem{Partielle Ordnung}{\PartOrd{R,A}}
}
\begin{tabproof}
  \proofstep{1}{(x,y)\in R\cap (T\times T)}{\rA}
  \proofstep{2}{(y,x)\in R\cap (T\times T)}{\rA}

  \proofstep{1}{(x,y)\in R}{\FormulaRefAuto{x\in A\cap B \vdash x\in A}{1}}
  \proofstep{2}{(y,x)\in R}{\FormulaRefAuto{x\in A\cap B \vdash x\in A}{2}}

  \proofstep{1,2}{x=y}{%
    \FormulaRefAuto{(x,y)\in R \dsep (y,x)\in R \vdash x=y}{3,4}}
\end{tabproof}


% ------------------------------------------------------------
% (4) Hauptsatz: Restriktion ist wieder partielle Ordnung
% ------------------------------------------------------------
\FormulaThmDelta[Restriktion einer partiellen Ordnung]{%
  T\subseteq A \dsep \PartOrd{R,A}\vdash \PartOrd{R\cap (T\times T),T}%
}{
  \DeltaRow{Mengen}{A \dsep R \dsep T \dsep x \dsep y \dsep z}
  \DeltaPrem{Teilmenge}{T\subseteq A}
  \DeltaPrem{Partielle Ordnung}{\PartOrd{R,A}}
  \DeltaRow{\textbf{Begründung}}{}
  \DeltaRow{Relation auf \(T\)}{R\cap (T\times T)\subseteq T\times T}
    [\FormulaRefAuto{R\cap (T\times T)\subseteq T\times T}]
  \DeltaRow{Reflexivität}{x\in T \vdash (x,x)\in R\cap (T\times T)}
    [\FormulaRefAuto{T\subseteq A \dsep x\in T \vdash (x,x)\in R\cap (T\times T)}]
  \DeltaRow{Transitivität}{%
    \begin{aligned}
      &(x,y)\dsep (y,z)\in R\cap (T\times T)\\
      &\vdash (x,z)\in R\cap (T\times T)
    \end{aligned}}
    [\FormulaRefAuto{(x,y)\in R\cap (T\times T) \dsep (y,z)\in R\cap (T\times T) \vdash (x,z)\in R\cap (T\times T)}]
  \DeltaRow{Antisymmetrie}{%
    \begin{aligned}
      &(x,y)\dsep (y,x)\in R\cap (T\times T)\\
      &\vdash x=y
    \end{aligned}}
    [\FormulaRefAuto{(x,y)\in R\cap (T\times T) \dsep (y,x)\in R\cap (T\times T) \vdash x=y}]
}{}



% ============================================================
% Wohlordnung
% (totale Ordnung + jedes nichtleere T ⊆ A besitzt ein Minimum)
% ============================================================

\subsection{Wohlordnung}
\subsubsection{Axiom einer Wohlordnung}

\FormulaAxiomDelta[Wohlordnungsprinzip]{%
  T\subseteq A \dsep T\neq\varnothing
  \vdash \exists m\in T\,\forall x\in T\,\bigl((m,x)\in R\bigr)%
}{
  \DeltaRow{Mengen}{A \dsep R \dsep T \dsep m \dsep x}
}

\subsubsection{Definition der Wohlordnung}

\FormulaDefDeltaK[Begriff der Wohlordnung]{\WellOrd{R,A}}{Wohlordnung}{
  \DeltaRow{Mengen}{A \dsep R \dsep T \dsep m \dsep x \dsep y \dsep z}
  \DeltaRow{\textbf{Axiome}}{}
  \DeltaRow{Totale Ordnung}
           {\TotOrd{R,A}}
           [\FormulaRefAuto{Totale Ordnung}]
  \DeltaRow{Wohlordnungsprinzip}{%
    \begin{aligned}
      &T\subseteq A \dsep T\neq\varnothing\\
      &\vdash \exists m\in T\,\forall x\in T\,\bigl((m,x)\in R\bigr)
    \end{aligned}}
           [\FormulaRefAuto{T\subseteq A \dsep T\neq\varnothing \vdash \exists m\in T\,\forall x\in T\,\bigl((m,x)\in R\bigr)}]
  \DeltaRow{\textbf{Neue Symbole}}{}
  \DeltaPrem{Wohlordnungen}{}
}




\section{Begriff der Funktion}

\subsection{Axiom der funktionalen Eindeutigkeit}

\FormulaAxiomDelta[Funktionale Eindeutigkeit]
{(x,y)\in F\dsep (x,z)\in F \vdash y=z}
{
\DeltaRow{Mengen}{x\dsep y\dsep z\dsep F}
}

\subsection{Definition der Funktion}


\FormulaDefDeltaK[Begriff der Funktion]{F\colon A\to B}{Funktion}{
  \DeltaRow{Mengen}{A\dsep B\dsep F\dsep x\dsep y\dsep z}
  \DeltaRow{\textbf{Axiome}}{}
  %
  % — Existenz als Menge geordneter Paare —
  \DeltaRow{\makecell[l]{Menge geordneter\\Paare}}
           {F \subseteq A \times B}
           [\FormulaRefAuto{F \subseteq A \times B}]
  %
  % — Funktionale Eindeutigkeit —
  \DeltaRow{\makecell[l]{Funktionale \\Eindeutigkeit}}
           {(x,y)\in F\dsep (x,z)\in F \vdash y=z}
           [\FormulaRefAuto{(x,y)\in F\dsep (x,z)\in F \vdash y=z}]
  %
  % — Aussonderung (Schema) —
  \DeltaRow{Totalität}
           {x\in A \vdash \exists y\,(x,y)\in F}
           [\FormulaRefAuto{x\in A \vdash \exists y\,(x,y)\in F}]
%
    \DeltaRow{\textbf{Neue Symbole}}{}
    \DeltaPrem{Funktionen}{ }
}

\section{Grundlegende Eigenschaften}

\FormulaThmDelta{F\colon A\to B \vdash \TotRel{F,A,B}}{
\DeltaRow{Mengen}{A\dsep B\dsep F}
}
\begin{tabproof}
    \proofstep{1}{F\colon A\to B}{\rA}
    \proofstep{1}{F \subseteq A \times B}{\FormulaRefAuto{R \subseteq A \times B}{1}}
    \proofstep{1}{x \in A \vdash \exists y\,\bigl((x,y)\in R\bigr)}{\FormulaRefAuto{x\in A \vdash \exists y\,(x,y)\in F}{1}}
    \proofstep{1}{\TotRel{F,A,B}}{\FormulaRefAuto{Totale Relation}{2,3}}
\end{tabproof}


\FormulaThmDelta{(x,y)\in F\vdash (x,y)\in A\times B}{
\DeltaRow{Mengen}{x\dsep y\dsep A\dsep B}
\DeltaPrem{Funktionen}{F\colon A\to B}
}
\begin{tabproof}
  \proofstep{1}{(x,y)\in F}{\rA}
  \proofstep{}{F\subseteq A\times B}{\FormulaRefAuto{F \subseteq A \times B}}
  \proofstep{1}{(x,y)\in A\times B}{\FormulaRefAuto{A\subseteq B,\, x\in A \vdash x\in B}{2,1}}
\end{tabproof}

\FormulaThmDelta{(x,y)\in F\vdash x\in A}{
\DeltaRow{Mengen}{x\dsep y\dsep A\dsep B}
\DeltaPrem{Funktionen}{F\colon A\to B}
}
\begin{tabproof}
  \proofstep{1}{(x,y)\in F}{\rA}
  \proofstep{1}{(x,y)\in A\times B}{\FormulaRefAuto{(x,y)\in F\vdash (x,y)\in A\times B}{1}}
  \proofstep{1}{x\in A}{\FormulaRefAuto{(a,b)\in A\times B\vdash a\in A}{2}}
\end{tabproof}

\FormulaThmDelta{(x,y)\in F\vdash y\in B}{
\DeltaRow{Mengen}{x\dsep y\dsep A\dsep B}
\DeltaPrem{Funktionen}{F\colon A\to B}
}
\begin{tabproof}
  \proofstep{1}{(x,y)\in F}{\rA}
  \proofstep{1}{(x,y)\in A\times B}{\FormulaRefAuto{(x,y)\in F\vdash (x,y)\in A\times B}{1}}
  \proofstep{1}{y\in B}{\FormulaRefAuto{(a,b)\in A\times B\vdash b\in B}{2}}
\end{tabproof}

\section{Funktionswert}

\FormulaThmDelta[Eindeutigkeit des Funktionswertes]{x\in A\vdash \exists! y\;(x,y)\in F}{
\DeltaRow{Mengen}{x\dsep A}
\DeltaPrem{Funktionen}{F\colon A\to B}
}
\begin{tabproof}
  \proofstep{1}{x\in A}{\rA}
  \proofstep{1}{\exists y(x,y)\in F}{\FormulaRefAuto{x\in A \vdash \exists y\,(x,y)\in F}{1}}
  \proofstep{2}{(x,y)\in F}{\rA}
  \proofstep{3}{(x,z)\in F}{\rA}
  \proofstep{2,3}{y=z}{\FormulaRefAuto{(x,y)\in F\dsep (x,z)\in F \vdash y=z}{2,3}}
  \proofstep{1}{\exists! y(x,y)\in F}{\UEI{1,2,3,4}}
\end{tabproof}
%%begin novalidate
\FormulaDefDelta[Funktionswert]{x\in A\vdash (F(x) \coloneqq \iota y\,\bigl((x,y)\in F\bigr)}{
\DeltaRow{Mengen}{x\dsep A}
\DeltaPrem{Funktionen}{F\colon A\to B}
}
%%end novalidate

\FormulaThmDelta{x\in A\vdash (x,F(x))\in F}{\DeltaRow{Mengen}{x\dsep A}
\DeltaPrem{Funktionen}{F\colon A\to B}
}
\begin{tabproof}
    \proofstep{1}{x\in A}{\rA}
    \proofstep{1}{(x,F(x))\in F}{\FormulaRefAuto{x\in A\vdash (F(x) \coloneqq \iota y\,\bigl((x,y)\in F\bigr)}{1}}
\end{tabproof}

\subsection{Eigenschaften des Funktionswertes}

\FormulaThmDelta[Ersetzung im Funktionsargument]{%
  x = y \dsep a = F(y) \vdash a = F(x)
}{
  \DeltaRow{Mengen}{A \dsep B \dsep x \dsep y \dsep a}
  \DeltaPrem{Funktionen}{F\colon A\to B}
}
\begin{tabproof}
  \proofstep{1}{x = y}{\rA}
  \proofstep{2}{a = F(y)}{\rA}
  \proofstep{1}{y = x}{\FormulaRefAuto{a = b \vdash b = a}{1}}
  \proofstep{1,2}{a = F(x)}{\rIE{3,2}}
\end{tabproof}



\FormulaThmDelta{x\in A\dsep y\in A\dsep x=y\vdash F(x)=F(y)}
{
\DeltaRow{Mengen}{x\dsep y\dsep A}
\DeltaPrem{Funktionen}{F\colon A\to B}
}
\begin{tabproof}
    \proofstep{1}{x\in A}{\rA}
    \proofstep{2}{y\in A}{\rA}
    \proofstep{3}{x=y}{\rA}
    \proofstep{1}{(x,F(x))\in F}{\FormulaRefAuto{x\in A\vdash (F(x) \coloneqq \iota y\,\bigl((x,y)\in F\bigr)}{1}}
    \proofstep{2}{(y,F(y))\in F}{\FormulaRefAuto{x\in A\vdash (F(x) \coloneqq \iota y\,\bigl((x,y)\in F\bigr)}{2}}
    \proofstep{1,3}{(y,F(x))\in F}{\rIE{3,4}}
    \proofstep{1,2,3}{F(x)=F(y)}{\FormulaRefAuto{(x,y)\in F\dsep (x,z)\in F \vdash y=z}{6,5}}
\end{tabproof}

\FormulaThmDelta{%
  x_1\in A \dsep x_2\in A \dsep F(x_1)\neq F(x_2) \vdash x_1\neq x_2%
}{
  \DeltaRow{Mengen}{A\dsep B\dsep x_1\dsep x_2}
  \DeltaPrem{Funktionen}{F\colon A\to B}
}
\begin{tabproof}
  \proofstep{1}{x_1\in A}{\rA}
  \proofstep{2}{x_2\in A}{\rA}
  \proofstep{3}{F(x_1)\neq F(x_2)}{\rA}

  \proofstep{4}{x_1=x_2}{\rA}
  \proofstep{1,2,4}{F(x_1)=F(x_2)}{%
    \FormulaRefAuto{x\in A\dsep y\in A\dsep x=y\vdash F(x)=F(y)}{1,2,4}}
  \proofstep{1,2,3,4}{\bot}{\rBI{3,5}}
  \proofstep{1,2,3}{x_1\neq x_2}{\rCI{4,6}}
\end{tabproof}

\FormulaThmDelta{x\in A\dsep F(x)=y\vdash (x,y)\in F}
{
\DeltaRow{Mengen}{x\dsep y\dsep A}
\DeltaPrem{Funktionen}{F\colon A\to B}
}
\begin{tabproof}
  \proofstep{1}{x\in A}{\rA}
  \proofstep{2}{R(x)=y}{\rA}
  \proofstep{1}{(x,F(x))\in F}{\FormulaRefAuto{x\in A\vdash (F(x) \coloneqq \iota y\,\bigl((x,y)\in F\bigr)}{1}}
  \proofstep{1,2}{(x,y)\in F}{\rIE{2,3}}
\end{tabproof}

\FormulaThmDelta{x\in A\dsep y=F(x)\vdash (x,y)\in F}
{
\DeltaRow{Mengen}{x\dsep y\dsep A}
\DeltaPrem{Funktionen}{F\colon A\to B}
}
\begin{tabproof}
  \proofstep{1}{x\in A}{\rA}
  \proofstep{2}{y=F(x)}{\rA}
  \proofstep{2}{F(x)=y}{\FormulaRefAuto{a=b\vdash b=a}{2}}
  \proofstep{1,2}{(x,y)\in F}{\FormulaRefAuto{x\in A\dsep F(x)=y\vdash (x,y)\in F}{1,3}}
\end{tabproof}

\FormulaThmDelta{x\in A\dsep (x,y)\in F\vdash y=F(x)}{
\DeltaRow{Mengen}{x\dsep y\dsep A}
\DeltaPrem{Funktionen}{F\colon A\to B}
}
\begin{tabproof}
  \proofstep{1}{(x,y)\in F}{\rA}
  \proofstep{2}{x\in A}{\rA}
  \proofstep{2}{(x,F(x))\in F}{\FormulaRefAuto{x\in A\vdash (F(x) \coloneqq \iota y\,\bigl((x,y)\in F\bigr)}{2}}
  \proofstep{1}{y=F(x)}{\FormulaRefAuto{(x,y)\in F\dsep (x,z)\in F \vdash y=z}{1,2}}
\end{tabproof}

\FormulaThmDelta{x\in A\dsep (x,y)\in F\vdash F(x)=y}{
\DeltaRow{Mengen}{x\dsep y\dsep A}
\DeltaPrem{Funktionen}{F\colon A\to B}
}
\begin{tabproof}
  \proofstep{1}{(x,y)\in F}{\rA}
  \proofstep{2}{x\in A}{\rA}
  \proofstep{2}{(x,F(x))\in F}{\FormulaRefAuto{x\in A\vdash (F(x) \coloneqq \iota y\,\bigl((x,y)\in F\bigr)}{2}}
  \proofstep{1}{y=F(x)}{\FormulaRefAuto{(x,y)\in F\dsep (x,z)\in F \vdash y=z}{2,1}}
\end{tabproof}


\FormulaThmDelta{x\in A\vdash F(x)\in B}{
\DeltaRow{Mengen}{A\dsep B}
\DeltaPrem{Funktionen}{F\colon A\to B}
}
\begin{tabproof}
  \proofstep{1}{x\in A}{\rA}
  \proofstep{}{F\subseteq A\times B}{\FormulaRefAuto{F \subseteq A \times B}}
  \proofstep{1}{(x,F(x))\in F}{\FormulaRefAuto{x\in A\vdash (F(x) \coloneqq \iota y\,\bigl((x,y)\in F\bigr)}{1}}
  \proofstep{1}{(x,F(x))\in A\times B}{\FormulaRefAuto{ A\subseteq B,\, x\in A \vdash x\in B }{2,3}}
  \proofstep{1}{F(x)\in B}{\FormulaRefAuto{(a,b)\in A\times B\vdash b\in B}{4}}
\end{tabproof}

\FormulaThmDelta{(x,y)\in F\vdash y=F(x)}{
\DeltaRow{Mengen}{x\dsep y\dsep A\dsep B}
\DeltaPrem{Funktionen}{F\colon A\to B}
}
\begin{tabproof}
  \proofstep{1}{(x,y)\in F}{\rA}
  \proofstep{1}{x\in A}{\FormulaRefAuto{(x,y)\in F\vdash x\in A}{1}}
  \proofstep{1}{(x,F(x))\in F}{\FormulaRefAuto{x\in A\vdash (F(x) \coloneqq \iota y\,\bigl((x,y)\in F\bigr)}{2}}
  \proofstep{1}{y=F(x)}{\FormulaRefAuto{(x,y)\in F\dsep (x,z)\in F \vdash y=z}{1,3}}
\end{tabproof}

\FormulaThmDelta{(x,y)\in F\vdash F(x)=y}{
\DeltaRow{Mengen}{x\dsep y\dsep A\dsep B}
\DeltaPrem{Funktionen}{F\colon A\to B}
}
\begin{tabproof}
  \proofstep{1}{(x,y)\in F}{\rA}
  \proofstep{1}{x\in A}{\FormulaRefAuto{(x,y)\in F\vdash x\in A}{1}}
  \proofstep{1}{(x,F(x))\in F}{\FormulaRefAuto{x\in A\vdash (F(x) \coloneqq \iota y\,\bigl((x,y)\in F\bigr)}{2}}
  \proofstep{1}{y=F(x)}{\FormulaRefAuto{(x,y)\in F\dsep (x,z)\in F \vdash y=z}{3,1}}
\end{tabproof}

\FormulaThmDelta{F(x)=G(x)\dsep (x,y)\in F\vdash (x,y)\in G}{
\DeltaRow{Mengen}{x\dsep A\dsep B}
\DeltaPrem{Funktionen}{F,G\colon A\to B}
}
\begin{tabproof}
  \proofstep{1}{F(x)=G(x)}{\rA}
  \proofstep{2}{(x,y)\in F}{\rA}
  \proofstep{2}{y=F(x)}{\FormulaRefAuto{(x,y)\in F\vdash F(x)=y}{2}}
  \proofstep{1,2}{y=G(x)}{\FormulaRefAuto{a=b, b=c \vdash a=c}{3,1}}
  \proofstep{2}{x\in A}{\FormulaRefAuto{(x,y)\in F\vdash x\in A}{2}}
  \proofstep{1,2}{(x,y)\in G}{\FormulaRefAuto{x\in A\dsep y=F(x)\vdash (x,y)\in F}{5,4}}
\end{tabproof}

\FormulaThmDelta{%
  \{x\in A\mid F(x)\in \{a\}\}=\{x\in A\mid F(x)=a\}%
}{
  \DeltaRow{Mengen}{A\dsep a\dsep x}
  \DeltaPrem{Funktionen}{F}
}
\begin{tabproof}
  \proofstep{1}{x\in \{x\in A\mid F(x)\in \{a\}\}}{\rA}
  \proofstep{1}{x\in A\land F(x)\in \{a\}}{%
    \FormulaRefAuto{x \in \{u \in A \mid P(u)\} \eqvdash x \in A \land P(x)}{1}}
  \proofstep{1}{x\in A}{\rAEa{2}}
  \proofstep{1}{F(x)\in \{a\}}{\rAEb{2}}
  \proofstep{1}{F(x)=a}{\FormulaRefAuto{x \in \{a\} \eqvdash x = a}{4}}
  \proofstep{1}{x\in \{x\in A\mid F(x)=a\}}{%
    \FormulaRefAuto{x \in A\dsep P(x)\vdash x \in \{x \in A \mid P(x)\}}{3,5}}
  \proofstep{}{\{x\in A\mid F(x)\in \{a\}\}\subseteq \{x\in A\mid F(x)=a\}}{%
    \FormulaRefAuto{A \subseteq B := \forall x\,(x\in A \rightarrow x\in B)}{\rUI{\rRI{1,6}}}}

  \proofstep{8}{x\in \{x\in A\mid F(x)=a\}}{\rA}
  \proofstep{8}{x\in A\land F(x)=a}{%
    \FormulaRefAuto{x \in \{u \in A \mid P(u)\} \eqvdash x \in A \land P(x)}{8}}
  \proofstep{8}{x\in A}{\rAEa{9}}
  \proofstep{8}{F(x)=a}{\rAEb{9}}
  \proofstep{8}{F(x)\in \{a\}}{\FormulaRefAuto{x \in \{a\} \eqvdash x = a}{11}}
  \proofstep{8}{x\in \{x\in A\mid F(x)\in \{a\}\}}{%
    \FormulaRefAuto{x \in A\dsep P(x)\vdash x \in \{x \in A \mid P(x)\}}{10,12}}
  \proofstep{}{\{x\in A\mid F(x)=a\}\subseteq \{x\in A\mid F(x)\in \{a\}\}}{%
    \FormulaRefAuto{A \subseteq B := \forall x\,(x\in A \rightarrow x\in B)}{\rUI{\rRI{8,13}}}}

  \proofstep{}{\{x\in A\mid F(x)\in \{a\}\}=\{x\in A\mid F(x)=a\}}{%
    \FormulaRefAuto{A \subseteq B, B \subseteq A \vdash A = B}{7,14}}
\end{tabproof}


\subsection{Zur Gleichheit von Funktionen}

\FormulaThmDelta{\forall x\in A (F(x)=G(x)) \vdash \forall (x,y)((x,y)\in F\rightarrow (x,y)\in G)}{
\DeltaRow{Mengen}{A\dsep B}
\DeltaPrem{Funktionen}{F,G\colon A\to B}
}
\begin{tabproof}
  \proofstep{1}{\forall x\in A (F(x)=G(x))}{\rA}
  \proofstep{2}{(x,y)\in F}{\rA}
  \proofstep{2}{x\in A}{\FormulaRefAuto{(x,y)\in F\vdash x\in A}{2}}
  \proofstep{1,2}{F(x)=G(x)}{\rRE{\rUE{1},3}}
  \proofstep{1,2}{(x,y)\in G}{\FormulaRefAuto{F(x)=G(x)\dsep (x,y)\in F\vdash (x,y)\in G}{4,2}}
  \proofstep{1}{(x,y)\in F\rightarrow (x,y)\in G}{\rRI{2,5}}
  \proofstep{1}{\forall (x,y)((x,y)\in F\rightarrow (x,y)\in G)}{\rUI{6}}
\end{tabproof}



\FormulaThmDelta{\forall x\in A (F(x)=G(x)) \vdash \forall (x,y)((x,y)\in G\rightarrow (x,y)\in F)}{
\DeltaRow{Mengen}{A\dsep B}
\DeltaPrem{Funktionen}{F,G\colon A\to B}
}
\begin{tabproof}
  \proofstep{1}{\forall x\in A (F(x)=G(x))}{\rA}
  \proofstep{2}{x\in A}{\rA}
  \proofstep{1,2}{F(x)=G(x)}{\rRE{\rUE{1},2}}
  \proofstep{1,2}{G(x)=F(x)}{\FormulaRefAuto{a=b\vdash b=a}{3}}
  \proofstep{1}{\forall x\in A(G(x)=F(x))}{\rUI{\rRI{2,4}}}
  \proofstep{1}{\forall (x,y)((x,y)\in G\rightarrow (x,y)\in F)}{\FormulaRefAuto{\forall x\in A (F(x)=G(x)) \vdash \forall (x,y)((x,y)\in F\rightarrow (x,y)\in G)}{5}}
\end{tabproof}

\FormulaThmDelta{\forall x\in A (F(x)=G(x)) \eqvdash \forall (x,y)((x,y)\in F\leftrightarrow (x,y)\in G)}{
\DeltaRow{Mengen}{A\dsep B}
\DeltaPrem{Funktionen}{F,G\colon A\to B}
}
\begin{tabproofsplit}
\proofpart{\(\vdash\)}
  \proofstep{1}{\forall x\in A (F(x)=G(x))}{\rA}
  \proofstep{1}{\forall (x,y)((x,y)\in F\rightarrow (x,y)\in G)}{\FormulaRefAuto{\forall x\in A (F(x)=G(x)) \vdash \forall (x,y)((x,y)\in F\rightarrow (x,y)\in G)}{1}}
  \proofstep{1}{\forall (x,y)((x,y)\in G\rightarrow (x,y)\in F)}{\FormulaRefAuto{\forall x\in A (F(x)=G(x)) \vdash \forall (x,y)((x,y)\in G\rightarrow (x,y)\in F)}{1}}
  \proofstep{1}{\forall (x,y)((x,y)\in F\leftrightarrow (x,y)\in G)}{\FormulaRefAuto{\forall x (P(x) \leftrightarrow Q(x)) \eqvdash \forall x (P(x) \rightarrow Q(x)) \land \forall x (Q(x) \rightarrow P(x))}{2,3}}
\closeproofpart
\proofpart{\(\dashv\)}
  \proofstep{1}{\forall (x,y)((x,y)\in F\leftrightarrow (x,y)\in G)}{\rA}
  \proofstep{2}{x\in A}{\rA}
  \proofstep{2}{(x,F(x))\in F}{\FormulaRefAuto{x\in A\vdash (F(x) \coloneqq \iota y\,\bigl((x,y)\in F\bigr)}{2}}
  \proofstep{1}{(x,F(x))\in F\leftrightarrow (x,F(x))\in G}{\rUI{1}}
  \proofstep{1,2}{(x,F(x))\in G}{\FormulaRefAuto{P \leftrightarrow Q, P \vdash Q}{4,3}}
  \proofstep{2}{(x,G(x))\in G}{\FormulaRefAuto{x\in A\vdash (F(x) \coloneqq \iota y\,\bigl((x,y)\in F\bigr)}{2}}
  \proofstep{1,2}{F(x)=G(x)}{\FormulaRefAuto{(x,y)\in F\dsep (x,z)\in F \vdash y=z}{5,6}}
  \proofstep{1,2}{\forall x\in A(F(x)=G(x))}{\rUI{\rRI{2,7}}}
\closeproofpart
\end{tabproofsplit}

\FormulaThmDelta{\forall x\in A (F(x)=G(x)) \eqvdash F=G}{
\DeltaRow{Mengen}{A\dsep B}
\DeltaPrem{Funktionen}{F,G\colon A\to B}
}
\begin{tabproofwide}
  \proofstepwide{\forall x\in A (F(x)=G(x))}{\leftrightarrow}{\forall (x,y)((x,y)\in F\leftrightarrow (x,y)\in G)}%
    {\FormulaRefAuto{\forall x\in A (F(x)=G(x)) \eqvdash \forall (x,y)((x,y)\in F\leftrightarrow (x,y)\in G)}}
  \proofstepwide{}{\leftrightarrow}{F=G}%
    {\FormulaRefAuto{\forall x\, (x \in A \leftrightarrow x \in B) \eqvdash A = B}{1}}
   \proofstepwide{\forall x\in A (F(x)=G(x))}{\leftrightarrow}{F=G}%
    {\rChain{1,2}}
\end{tabproofwide}

\FormulaThmDelta{x\in A\dsep F=G\vdash F(x)=G(x)}{
\DeltaRow{Mengen}{A\dsep B}
\DeltaPrem{Funktionen}{F,G\colon A\to B}
}
\begin{tabproof}
    \proofstep{1}{x\in A}{\rA}
    \proofstep{2}{F=G}{\rA}
    \proofstep{2}{\forall x\in A (F(x)=G(x))}{\FormulaRefAuto{\forall x\in A (F(x)=G(x)) \eqvdash F=G}{2}}
    \proofstep{1,2}{\forall x\in A (F(x)=G(x))}{\rRE{1,\rUI{3}}}
\end{tabproof}


\section{Axiom der Ersetzung}

\FormulaAxiomDelta[Ersetzung]{
\exists B\;\forall y\;\bigl( y\in B\;\leftrightarrow\; \exists x\in A\;y=F(x) \bigr)
}{
\DeltaRow{Mengen}{A}
\DeltaPrem{Funktionen}{F\colon A\to B}
}

\section{Die Bildmenge}

%%begin novalidate
\FormulaDefDelta[Bildmenge]
{F[A] \coloneqq \iota C\Bigl(\forall y\;\bigl( y\in C \leftrightarrow \exists x\in A\,y=F(x) \bigr)\Bigr)}
{
  \DeltaRow{Mengen}{A}
  \DeltaPrem{Funktionen}{F\colon A\to B}
}
%%end novalidate

\FormulaThmDelta{y\in F[A]\eqvdash \exists x\in A\,y=F(x)}
{
  \DeltaRow{Mengen}{x\dsep y\dsep A}
  \DeltaPrem{Funktionen}{F\colon A\to B }
}
\begin{tabproof}
  \proofstep{}{ \forall y\;\bigl( y\in R[A] \leftrightarrow \exists x\in A\,y=R(x)\bigr) }{\FormulaRefAuto{F[A] \coloneqq \iota C\Bigl(\forall y\;\bigl( y\in C \leftrightarrow \exists x\in A\,y=F(x) \bigr)\Bigr)}}
  \proofstep{}{ y\in R[A] \leftrightarrow \exists x\in A\,y=R(x) }{\rUE{1}}
\end{tabproof}

\FormulaThmDelta{x\in A\vdash F(x)\in F[A]}
{
  \DeltaRow{Mengen}{A\dsep B}
  \DeltaPrem{Funktionen}{F\colon A\to B }
}
\begin{tabproof}
  \proofstep{1}{x\in A}{\rA}
  \proofstep{}{F(x)=F(x)}{\rII}
  \proofstep{1}{x\in A\land F(x)=F(x)}{\rAI{1,2}}
  \proofstep{1}{\exists x\in A\, F(x)=F(x)}{\rEI{3}}
  \proofstep{1}{F(x)\in F[A]}{\FormulaRefAuto{y\in F[A]\eqvdash \exists x\in A\,y=F(x)}{4}}
\end{tabproof}

\FormulaThmDelta{F\subseteq A\times F[A]}
{
  \DeltaRow{Mengen}{A\dsep B}
  \DeltaPrem{Funktionen}{F\colon A\to B }
}
\begin{tabproof}
    \proofstep{1}{(x,y)\in F}{\rA}
    \proofstep{1}{x\in A}{\FormulaRefAuto{(x,y)\in F\vdash x\in A}{1}}
    \proofstep{1}{F(x)\in F[A]}{\FormulaRefAuto{x\in A\vdash F(x)\in F[A]}{2}}
    \proofstep{1}{y=F(x)}{\FormulaRefAuto{x\in A\dsep (x,y)\in F\vdash F(x)=y}{2,1}}
    \proofstep{1}{y\in F[A]}{\rIE{4,3}}
    \proofstep{1}{(x,y)\in A\times F[A]}{\FormulaRefAuto{a\in A\dsep b\in B\vdash (a,b)\in A\times B}{2,5}}
    \proofstep{1}{F\subseteq A\times F[A]}{\FormulaRefAuto{A \subseteq B := \forall x\,(x\in A \rightarrow x\in B)}{\rUI{\rRI{1,6}}}}
\end{tabproof}



\section{Urbild}
%%begin novalidate
\FormulaDefDelta[Urbildmenge]{%
C\subseteq B \vdash F^{-1}[C] \coloneqq \{\,x\in A \mid F(x)\in C\,\}%
}{
  \DeltaRow{Mengen}{A\dsep B\dsep C\dsep x}
  \DeltaPrem{Funktionen}{F\colon A\to B}
}
%%end novalidate

\FormulaThmDelta{%
C\subseteq B \vdash \bigl(x\in F^{-1}[C] \leftrightarrow (x\in A \land F(x)\in C)\bigr)
}{
  \DeltaRow{Mengen}{A\dsep B\dsep C\dsep x}
  \DeltaPrem{Funktionen}{F\colon A\to B}
}
\begin{tabproof}
  \proofstep{1}{C\subseteq B}{\rA}

  \proofstep{1}{F^{-1}[C]=\{\,u\in A \mid F(u)\in C\,\}}{%
    \FormulaRefAuto{C\subseteq B \vdash F^{-1}[C] \coloneqq \{\,u\in A \mid F(u)\in C\,\}}{1}}

  \proofstep{}{x\in \{\,u\in A \mid F(u)\in C\,\}\leftrightarrow (x\in A \land F(x)\in C)}{%
    \FormulaRefAuto{x\in \{u\in A \mid P(u)\}\eqvdash x\in A \land P(x)}}

  \proofstep{1}{x\in F^{-1}[C]\leftrightarrow (x\in A \land F(x)\in C)}{\rIE{2,3}}
\end{tabproof}

\section{Graph einer Funktion}

\FormulaDefDelta[Graph einer Funktion]{\Graph(F):=\{(x,y) \in A \times B \mid y=F(x)\}}%
{
\DeltaRow{Mengen}{x\dsep y\dsep A\dsep B}
\DeltaRow{Funktionensymbol}{F}
}

\FormulaThmDelta{\Graph(F)\subseteq A\times B}%
{
\DeltaRow{Mengen}{x\dsep y\dsep A\dsep B}
\DeltaRow{Funktionensymbol}{F}
}
\begin{tabproof}
    \proofstep{1}{\Graph(F):=\{(x,y) \in A \times B \mid y=F(x)\}}{\FormulaRefAuto{\Graph(F):=\{(x,y) \in A \times B \mid y=F(x)\}}}
    \proofstep{1}{\{(x,y) \in A \times B \mid y=F(x)\}\subseteq A\times B}{\FormulaRefAuto{\{ x \in A \mid P(x)\} \subseteq A}}
    \proofstep{1}{\Graph(F)\subseteq A\times B}{\rIE{1,2}}
\end{tabproof}

\FormulaThmDelta{(x,y)\in \Graph(F)\vdash y=F(x)}%
{
\DeltaRow{Mengen}{x\dsep y\dsep A\dsep B}
\DeltaRow{Funktionensymbol}{F}
}
\begin{tabproof}
  \proofstep{1}{(x,y)\in \Graph(F)}{\rA}
  \proofstep{}{\Graph(F)=\{(x,y) \in A \times B \mid y=F(x)\}}{\FormulaRefAuto{\Graph(F):=\{(x,y) \in A \times B \mid y=F(x)\}}}
  \proofstep{1}{y=F(x)}{\FormulaRefAuto{x \in A\dsep A=\{x \in B \mid P(x)\}\vdash P(x)}{1,2}}
\end{tabproof}

\FormulaThmDelta[Funktionale Eindeutigkeit von \(\Graph(F)\)]{(x,y)\in \Graph(F)\dsep (x,z)\in \Graph(F)\vdash y=z}%
{
\DeltaRow{Mengen}{x\dsep y\dsep z}
\DeltaRow{Funktionensymbol}{F}
}
\begin{tabproof}
  \proofstep{1}{(x,y)\in \Graph(F)}{\rA}
  \proofstep{2}{(x,z)\in \Graph(F)}{\rA}
  \proofstep{1}{y = F(x)}{\FormulaRefAuto{(x,y)\in Graph(F)\vdash y=F(x)}{1}}
  \proofstep{2}{z = F(x)}{\FormulaRefAuto{(x,y)\in Graph(F)\vdash y=F(x)}{2}}
  \proofstep{5}{y = z}{\FormulaRefAuto{a = b,\, c = b \vdash a = c}{3,4}}
\end{tabproof}

\FormulaThmDelta
{\forall u\in A\,F(u)\in B\dsep x\in A \vdash \exists y\,(x,y)\in\Graph(F)}
{
\DeltaRow{Mengen}{u\dsep x\dsep y}
\DeltaRow{Funktionensymbol}{F}
}
\begin{tabproof}
  \proofstep{1}{\forall u\in A\,F(u)\in B}{\rA}
  \proofstep{2}{x\in A}{\rA}
  \proofstep{1,2}{F(x)\in B}{\rRE{\rUE{1},2}}
  \proofstep{1,2}{(x,F(x))\in A\times B}{\FormulaRefAuto{a\in A\dsep b\in B\vdash (a,b)\in A\times B}{2,3}}
  \proofstep{}{F(x)=F(x)}{\rII}
  \proofstep{1,2}{(x,F(x))\in\{(x,y)\in A \times B\mid y=F(x)\}}{\FormulaRefAuto{x \in A\dsep P(x)\vdash x \in \{x \in A \mid P(x)\}}{4,5}}
  \proofstep{1,2}{(x,F(x))\in\Graph(F)}{\rIE{\FormulaRefAuto{\Graph(F):=\{(x,y) \in A \times B \mid y=F(x)\}},6}}
  \proofstep{1,2}{\exists y\,(x,y)\in\Graph(F)}{\rEI{7}}
\end{tabproof}

\FormulaThmDelta[Graph eines Funktionensymbols ist eine Funktion]
{\forall u\in A\,F(u)\in B \vdash \Graph(F)\colon A\to B}
{
  \DeltaRow{Mengen}{A\dsep B}
  \DeltaRow{Funktionensymbol}{F}
}
\begin{tabproof}
  \proofstep{1}{\forall u\in A\,F(u)\in B}{\rA}
  \proofstep{}{\Graph(F) \subseteq A\times B }%
    {\FormulaRefAuto{\Graph(F)\subseteq A\times B}}
  \proofstep{}{\forall (x,y),(x,z)\in \Graph(F)\, y=z }%
    {\FormulaRefAuto{(x,y)\in \Graph(F)\dsep (x,z)\in \Graph(F)\vdash y=z}}
  \proofstep{1}{\forall x\in A \exists y\,(x,y)\in \Graph(F)}%
    {\FormulaRefAuto{\forall u\in A\,F(u)\in B\dsep x\in A \vdash \exists y\,(x,y)\in\Graph(F)}{1}}
  \proofstep{1}{\Graph(F)\colon A\to B}%
    {\FormulaRefAuto{Funktion}{1}}
\end{tabproof}

\FormulaThmDelta[Funktionswerte von \(\Graph(F)\)]
{\forall u\in A\,F(u)\in B \vdash \forall x\in A\,\Graph(F)(x) = F(x)}
{
  \DeltaRow{Mengen}{A\dsep B\dsep x}
  \DeltaRow{Funktionensymbol}{F}
}
\begin{tabproof}
  \proofstep{1}{\forall u\in A\,F(u)\in B}{\rA}
  \proofstep{2}{x\in A}{\rA}
  \proofstep{1}{\Graph(F)\colon A\to B}{\FormulaRefAuto{\forall u\in A\,F(u)\in B \vdash \Graph(F)\colon A\to B}{1}}
  \proofstep{1,2}{(x,\Graph(x))\in\Graph(F)}{\FormulaRefAuto{x\in A\vdash (x,F(x))\in F}{3,2}}
  \proofstep{1,2}{\Graph(F)(x) = F(x)}{\FormulaRefAuto{(x,y)\in \Graph(F)\vdash y=F(x)}{4}}
  \proofstep{1}{\forall x\in A\,\Graph(F)(x) = F(x)}{\rUI{\rRI{2,5}}}
\end{tabproof}

\FormulaThmDelta
{\forall u\in A\,t(u)\in B \vdash \exists F\colon A\to B\,\forall x\in A\,F(x) = t(x)}
{
  \DeltaRow{Mengen}{A\dsep B\dsep x}
  \DeltaRow{Funktionensymbol}{t}
}
\begin{tabproof}
  \proofstep{1}{\forall u\in A\,t(u)\in B}{\rA}
  \proofstep{1}{\forall x\in A\,\Graph(t)(x) = t(x)}{\FormulaRefAuto{\forall u\in A\,F(u)\in B \vdash \forall x\in A\,\Graph(F)(x) = F(x)}{1}}
  \proofstep{1}{\Graph(t)\colon A\to B}{\FormulaRefAuto{\forall u\in A\,F(u)\in B \vdash \Graph(F)\colon A\to B}{1}}
  \proofstep{1}{\exists F\colon A\to B\,\forall x\in A\,F(x) = t(x)}{\rEI{\rAI{3,2}}}
\end{tabproof}

\FormulaThmDeltaR[Funktion]{\exists! F\colon A\to B\,\forall x\in A\,F(x) = t(x)}{x\in A\vdash t(x):=y\dsep x\in A\vdash t(x)\in B\exists! F\colon A\to B\,\forall x\in A\,F(x) = t(x)}{
  \DeltaRow{Mengen}{A\dsep B\dsep F\dsep x}
  \DeltaRow{Funktionensymbol}{t}
  %
  \DeltaRow{\textbf{Bedingungen}}{}
  \DeltaRow{Definition einer Funktionsvorschrift}{x\in A\vdash t(x):=y}
  \DeltaRow{Typisierungsaxiom}{x\in A\vdash t(x)\in B}
  \DeltaRow{\textbf{Folgerung}}{}
}
\begin{tabproof}
  \proofstep{1}{\forall u\in A\,t(u)\in B}{\rA}
  \proofstep{1}{\exists F\colon A\to B\,\forall x\in A\,F(x) = t(x)}{\FormulaRefAuto{\forall u\in A\,t(u)\in B \vdash \exists F\colon A\to B\,\forall x\in A\,F(x) = t(x)}{1}}
  \proofstep{3}{F\colon A\to B\land \forall x\in A\, F(x)=t(x)}{\rA}
  \proofstep{4}{G\colon A\to B\land \forall x\in A\, G(x)=t(x)}{\rA}
  \proofstep{3}{F\colon A\to B}{\rAEa{3}}
  \proofstep{3}{\forall x\in A\, F(x)=t(x)}{\rAEb{3}}
  \proofstep{4}{G\colon A\to B}{\rAEa{4}}
  \proofstep{4}{\forall x\in A\, G(x)=t(x)}{\rAEb{4}}
  \proofstep{3,4}{\forall x\in A\, F(x)=G(x)}{\FormulaRefAuto{\forall x\, F(x)\rightarrow P(x)=Q(x)\dsep \forall x\, F(x)\rightarrow P(x)=R(x)\vdash \forall x\, F(x)\rightarrow Q(x)=R(x)}{6,8}}
  \proofstep{3,4}{F=G}{\FormulaRefAuto{\forall x\in A (F(x)=G(x)) \eqvdash F=G}{9}}
  \proofstep{3,4}{\exists! F\colon A\to B\,\forall x\in A\,F(x) = t(x)}{\UEI{2,3,4,10}}
\end{tabproof}



\section{Eigenschaften von Funktionen}
\subsection{Injektivität}

\FormulaDefDeltaK[Begriff der injektiven Funktion]{F\colon A\rightarrowtail B}{Injektive Funktion}{
  \DeltaRow{Mengen}{A\dsep B\dsep x\dsep y}
  \DeltaPrem{Funktionen}{F\colon A\to B}[\FormulaRefAuto{Funktion}]
  %
  \DeltaRow{\textbf{Axiome}}{}
  %
  % — Injektivität —
  \DeltaRow{Injektivität}
           {x,y\in A\dsep F(x)=F(y)\vdash x=y}
           [\FormulaRefAutoFwd{x\in A\dsep y\in A\dsep F(x)=F(y)\vdash x=y}]
  \DeltaRow{\textbf{Neue Symbole}}{}
  \DeltaRow{Injektive Funktionen}{}
}

\FormulaAxiomDeltaK[Injektivität]%
{x\in A\dsep y\in A\dsep F(x)=F(y)\vdash x=y}%
{Injektivität}{
\DeltaRow{Mengen}{x\dsep y\dsep A\dsep B}
\DeltaPrem{Funktionen}{F\colon A\to B}
}


\FormulaThmDelta%
{(x,z_1)\in F\dsep (y,z_2)\in F\dsep F(x)=F(y)\vdash x=y}%
{
\DeltaRow{Mengen}{x\dsep y\dsep z_1\dsep z_2\dsep A\dsep B}
\DeltaRow{Injektive Funktionen}{F\colon A\to B}
}
\begin{tabproof}
  \proofstep{1}{(x,z_1)\in F}{\rA}
  \proofstep{2}{(y,z_2)\in F}{\rA}
  \proofstep{3}{F(x)=F(y)}{\rA}
  \proofstep{1}{x\in A}{\FormulaRefAuto{(x,y)\in F\vdash x\in A}{1}}
  \proofstep{2}{y\in A}{\FormulaRefAuto{(x,y)\in F\vdash x\in A}{2}}
  \proofstep{1,2,3}{x=y}{\FormulaRefAuto{x\in A\dsep y\in A\dsep F(x)=F(y)\vdash x=y}{4,5,3}}
\end{tabproof}


\subsection{Surjektivität}

\FormulaDefDeltaK[Begriff der surjektiven Funktion]{F\colon A\twoheadrightarrow B}{Surjektive Funktion}{
  \DeltaRow{Mengen}{A\dsep B\dsep x\dsep y}
  \DeltaPrem{Funktionen}{F\colon A\to B}[\FormulaRefAuto{Funktion}]
%
  \DeltaRow{\textbf{Axiome}}{}
  %
  % — Surjekvität —
  \DeltaRow{Surjektivität}
           {y\in B\vdash \exists x\in A\; F(x)=y}
           [\FormulaRefAutoFwd{y\in B\vdash \exists x\in A\; F(x)=y}]
  \DeltaRow{\textbf{Neue Symbole}}{}
  \DeltaPrem{Surjektive Funktionen}{ }
}

\FormulaAxiomDeltaK[Surjektivität]%
{y\in B\vdash \exists x\in A\; F(x)=y}{Surjektivität}
{
\DeltaRow{Mengen}{y\dsep A\dsep B}
\DeltaPrem{Funktionen}{F\colon A\to B}
}

\FormulaThmDelta{F\colon A\to B \vdash F\colon A\twoheadrightarrow F[A]}{
\DeltaRow{Mengen}{A\dsep B\dsep F}
}
\begin{tabproof}
    \proofstep{1}{F\colon A\to B}{\rA}
    \proofstep{1}{F\subseteq A\times F[A]}{\FormulaRefAuto{F\subseteq A\times F[A]}{1}}
    \proofstep{1}{\forall x,y\,((x,y)\in F\dsep (x,z)\in F\rightarrow y=z)}{\FormulaRefAuto{(x,y)\in F\dsep (x,z)\in F \vdash y=z}{1}}
    \proofstep{1}{\forall x\,(x\in A\rightarrow \exists y (x,y)\in F)}{\FormulaRefAuto{x\in A \vdash \exists y\,(x,y)\in F}{1}}
    \proofstep{1}{\forall y\,(y\in F[A]\rightarrow \exists x\in A (F(x)=y))}{\FormulaRefAuto{y\in F[A]\eqvdash \exists x\in A\,y=F(x)}{1}}
    \proofstep{1}{F\colon A\twoheadrightarrow F[A]}{\FormulaRefAuto{Surjektive Funktion}{2,3,4,5}}
\end{tabproof}

% ------------------------------------------------------------
% Hilfslemma: der breite Äquivalenz-Block (Diagonal-Schritt)
% ------------------------------------------------------------
\FormulaThmDelta{%
  a\in A \dsep F(a)=\{x\in A\mid x\notin F(x)\}\vdash a\in F(a)\leftrightarrow a\notin F(a)%
}{
  \DeltaRow{Mengen}{A\dsep a\dsep x}
  \DeltaPrem{Funktionen}{F\colon A\to \powerset(A)}
}
\begin{tabproofwide}
  \proofstepwidestar[1]{a\in A}{\rA}
  \proofstepwidestar[2]{F(a)=\{x\in A\mid x\notin F(x)\}}{\rA}

  \proofstepwide[1,2]{a\in F(a)}{\leftrightarrow}{%
    \begin{aligned}[t]
      a\in \{x\in A\mid {}\\
      x\notin F(x)\}
    \end{aligned}
  }{\rUE{\FormulaRefAuto{\forall x\, (x \in A \leftrightarrow x \in B) \eqvdash A = B}{2}}}

  \proofstepwide[1,2]{}{\leftrightarrow}{a\in A \land a\notin F(a)}{%
    \FormulaRefAuto{x \in \{u \in A \mid P(u)\} \eqvdash x \in A \land P(x)}{3}}

  \proofstepwide[1,2]{}{\leftrightarrow}{a\notin F(a)}{%
    \FormulaRefAuto{Q\vdash (P\leftrightarrow Q\land R)\leftrightarrow (P\leftrightarrow R)}{1}}

  \proofstepwide[1,2]{a\in F(a)}{\leftrightarrow}{a\notin F(a)}{\rChain{3,5}}
\end{tabproofwide}


% ------------------------------------------------------------
% Hauptsatz: Cantor (keine Surjektion A -> P(A))
% ------------------------------------------------------------
\FormulaThmDelta{%
  F\colon A\to \powerset(A) \vdash \neg(F\colon A\sur \powerset(A))%
}{
  \DeltaRow{Mengen}{A\dsep F}
}
\begin{tabproofwide}
  \proofstepwidestar[1]{F\colon A\sur \powerset(A)}{\rA}
  \proofstepwidestar[1]{\forall y\in \powerset(A)\,\exists x\in A\,(F(x)=y)}{%
    \FormulaRefAuto{y\in B\vdash \exists x\in A\; F(x)=y}{1}}
  \proofstepwidestar[1]{\exists x\in A\,\bigl(F(x)=\{x\in A\mid x\notin F(x)\}\bigr)}{\rUE{2}}

  \proofstepwidestar[4]{a\in A\land F(a)=\{x\in A\mid x\notin F(x)\}}{\rA}
  \proofstepwidestar[4]{a\in A}{\rAEa{4}}
  \proofstepwidestar[4]{F(a)=\{x\in A\mid x\notin F(x)\}}{\rAEb{4}}

  \proofstepwide[4]{a\in F(a)}{\leftrightarrow}{a\notin F(a)}{%
    \FormulaRefAuto{a\in A \dsep F(a)=\{x\in A\mid x\notin F(x)\}\vdash a\in F(a)\leftrightarrow a\notin F(a)}{5,6}}

  \proofstepwidestar[]{\neg (a\in F(a)\leftrightarrow a\notin F(a))}{%
    \FormulaRefAuto{\neg(P\leftrightarrow\neg P)}}

  \proofstepwidestar[4]{\bot}{\rBI{7,8}}
  \proofstepwidestar[1]{\bot}{\rEE{3,4,9}}
  \proofstepwidestar[]{\neg(F\colon A\sur \powerset(A))}{\rCI{1,10}}
\end{tabproofwide}


% ------------------------------------------------------------
% Surjektion aus Injektion (bei Nichtleere)
% (Einfügen nach \subsection{Surjektivität}, vor \subsection{Bijektivität})
% ------------------------------------------------------------


\subsubsection{Surjektion aus Injektion bei Nichtleere}
% ------------------------------------------------------------
% Konstruktion: Surjektion aus Injektion bei Nichtleere
% ------------------------------------------------------------

\FormulaDefDelta{%
  \Gsurjfrominj{F}{a_0} \coloneqq
  \{\, (b,a)\in B\times A \mid (b\in F[A]\land F(a)=b)\lor (b\notin F[A]\land a=a_0)\,\}%
}{
  \DeltaRow{Mengen}{A\dsep B\dsep a_0}
  \DeltaPrem{Injektive Funktionen}{F\colon A\inj B}
}

% --- (1a) Definition-Elimination: b liegt in B ---
\FormulaThmDelta{%
  (b,a)\in \Gsurjfrominj{F}{a_0}\vdash b\in B%
}{
  \DeltaRow{Mengen}{A\dsep B\dsep a_0\dsep a\dsep b}
  \DeltaPrem{Injektive Funktionen}{F\colon A\inj B}
}
\begin{tabproof}
  \proofstep{1}{(b,a)\in \Gsurjfrominj{F}{a_0}}{\rA}

  \proofstep{1}{%
    \begin{aligned}[t]
      (b,a)\in \{\, (b,a)\in B\times A \mid {}&
        (b\in F[A]\land F(a)=b)\\
        &\lor (b\notin F[A]\land a=a_0)\,\}
    \end{aligned}%
  }{%
    \rIE{\FormulaRefAuto{\Gsurjfrominj{F}{a_0} \coloneqq
      \{\, (b,a)\in B\times A \mid (b\in F[A]\land F(a)=b)\lor (b\notin F[A]\land a=a_0)\,\}}{1}}%
  }

  \proofstep{1}{%
    \begin{aligned}[t]
      (b,a)\in B\times A \land \bigl({}&
        (b\in F[A]\land F(a)=b)\\
        &\lor (b\notin F[A]\land a=a_0)\bigr)
    \end{aligned}%
  }{%
    \FormulaRefAuto{x \in \{u \in A \mid P(u)\} \eqvdash x \in A \land P(x)}{2}%
  }

  \proofstep{1}{(b,a)\in B\times A}{\rAEa{3}}
  \proofstep{1}{b\in B\land a\in A}{\FormulaRefAuto{(a,b)\in A\times B\eqvdash a\in A\land b\in B}{4}}
  \proofstep{1}{b\in B}{\rAEa{5}}
\end{tabproof}


% --- (1b) Definition-Elimination: a liegt in A ---
\FormulaThmDelta{%
  (b,a)\in \Gsurjfrominj{F}{a_0}\vdash a\in A%
}{
  \DeltaRow{Mengen}{A\dsep B\dsep a_0\dsep a\dsep b}
  \DeltaPrem{Injektive Funktionen}{F\colon A\inj B}
}
\begin{tabproof}
  \proofstep{1}{(b,a)\in \Gsurjfrominj{F}{a_0}}{\rA}

  \proofstep{1}{%
    \begin{aligned}[t]
      (b,a)\in \{\, (b,a)\in B\times A \mid {}&
        (b\in F[A]\land F(a)=b)\\
        &\lor (b\notin F[A]\land a=a_0)\,\}
    \end{aligned}%
  }{%
    \rIE{\FormulaRefAuto{\Gsurjfrominj{F}{a_0} \coloneqq
      \{\, (b,a)\in B\times A \mid (b\in F[A]\land F(a)=b)\lor (b\notin F[A]\land a=a_0)\,\}}{1}}%
  }

  \proofstep{1}{%
    \begin{aligned}[t]
      (b,a)\in B\times A \land \bigl({}&
        (b\in F[A]\land F(a)=b)\\
        &\lor (b\notin F[A]\land a=a_0)\bigr)
    \end{aligned}%
  }{%
    \FormulaRefAuto{x \in \{u \in A \mid P(u)\} \eqvdash x \in A \land P(x)}{2}%
  }

  \proofstep{1}{(b,a)\in B\times A}{\rAEa{3}}
  \proofstep{1}{b\in B\land a\in A}{\FormulaRefAuto{(a,b)\in A\times B\eqvdash a\in A\land b\in B}{4}}
  \proofstep{1}{a\in A}{\rAEb{5}}
\end{tabproof}


% --- (1c) Definition-Elimination: der entscheidende Disjunktions-Teil ---
\FormulaThmDelta{%
  (b,a)\in \Gsurjfrominj{F}{a_0}\vdash (b\in F[A]\land F(a)=b)\lor (b\notin F[A]\land a=a_0)%
}{
  \DeltaRow{Mengen}{A\dsep B\dsep a_0\dsep a\dsep b}
  \DeltaPrem{Injektive Funktionen}{F\colon A\inj B}
}
\begin{tabproof}
  \proofstep{1}{(b,a)\in \Gsurjfrominj{F}{a_0}}{\rA}

  \proofstep{1}{%
    \begin{aligned}[t]
      (b,a)\in \{\, (b,a)\in B\times A \mid {}&
        (b\in F[A]\land F(a)=b)\\
        &\lor (b\notin F[A]\land a=a_0)\,\}
    \end{aligned}%
  }{%
    \rIE{\FormulaRefAuto{\Gsurjfrominj{F}{a_0} \coloneqq
      \{\, (b,a)\in B\times A \mid (b\in F[A]\land F(a)=b)\lor (b\notin F[A]\land a=a_0)\,\}}{1}}%
  }

  \proofstep{1}{%
    \begin{aligned}[t]
      (b,a)\in B\times A \land \bigl({}&
        (b\in F[A]\land F(a)=b)\\
        &\lor (b\notin F[A]\land a=a_0)\bigr)
    \end{aligned}%
  }{%
    \FormulaRefAuto{x \in \{u \in A \mid P(u)\} \eqvdash x \in A \land P(x)}{2}%
  }

  \proofstep{1}{(b\in F[A]\land F(a)=b)\lor (b\notin F[A]\land a=a_0)}{\rAEb{3}}
\end{tabproof}




% --- (2) Definition zusammenbauen: Introduction ---
\FormulaThmDelta{%
  b\in B \dsep a\in A \dsep \bigl((b\in F[A]\land F(a)=b)\lor (b\notin F[A]\land a=a_0)\bigr)
  \vdash (b,a)\in \Gsurjfrominj{F}{a_0}%
}{
  \DeltaRow{Mengen}{A\dsep B\dsep a_0\dsep a\dsep b}
  \DeltaPrem{Injektive Funktionen}{F\colon A\inj B}
}
\begin{tabproof}
  \proofstep{1}{b\in B}{\rA}
  \proofstep{2}{a\in A}{\rA}
  \proofstep{3}{\bigl((b\in F[A]\land F(a)=b)\lor (b\notin F[A]\land a=a_0)\bigr)}{\rA}

  \proofstep{1,2}{(b,a)\in B\times A}{\FormulaRefAuto{a\in A\dsep b\in B\vdash (a,b)\in A\times B}{1,2}}

  \proofstep{1,2,3}{(b,a)\in \Gsurjfrominj{F}{a_0}}{%
    \rIE{\FormulaRefAuto{\Gsurjfrominj{F}{a_0} \coloneqq
    \{\, (b,a)\in B\times A \mid (b\in F[A]\land F(a)=b)\lor (b\notin F[A]\land a=a_0)\,\}},4,3}}
\end{tabproof}

% --- (3) Fall-Lemma: b liegt im Bild ---
\FormulaThmDelta{%
  b\in F[A] \dsep (b,a)\in \Gsurjfrominj{F}{a_0}\vdash F(a)=b%
}{
  \DeltaRow{Mengen}{A\dsep B\dsep a_0\dsep a\dsep b}
  \DeltaPrem{Injektive Funktionen}{F\colon A\inj B}
}
\begin{tabproof}
  \proofstep{1}{b\in F[A]}{\rA}
  \proofstep{2}{(b,a)\in \Gsurjfrominj{F}{a_0}}{\rA}

  \proofstep{2}{(b\in F[A]\land F(a)=b)\lor (b\notin F[A]\land a=a_0)}{%
    \FormulaRefAuto{(b,a)\in \Gsurjfrominj{F}{a_0}\vdash (b\in F[A]\land F(a)=b)\lor (b\notin F[A]\land a=a_0)}{2}}

  \proofstep{4}{b\notin F[A]\land a=a_0}{\rA}
  \proofstep{4}{b\notin F[A]}{\rAEa{4}}
  \proofstep{1,4}{\bot}{\rBI{1,5}}
  \proofstep{1}{\neg(b\notin F[A]\land a=a_0)}{\rCI{4,6}}

  \proofstep{1,2}{(b\in F[A]\land F(a)=b)}{%
    \FormulaRefAuto{P \lor Q, \neg Q \vdash P}{3,7}}
  \proofstep{1,2}{F(a)=b}{\rAEb{8}}
\end{tabproof}

% --- (4) Fall-Lemma: b liegt nicht im Bild ---
\FormulaThmDelta{%
  b\notin F[A] \dsep (b,a)\in \Gsurjfrominj{F}{a_0}\vdash a=a_0%
}{
  \DeltaRow{Mengen}{A\dsep B\dsep a_0\dsep a\dsep b}
  \DeltaPrem{Injektive Funktionen}{F\colon A\inj B}
}
\begin{tabproof}
  \proofstep{1}{b\notin F[A]}{\rA}
  \proofstep{2}{(b,a)\in \Gsurjfrominj{F}{a_0}}{\rA}

  \proofstep{2}{(b\in F[A]\land F(a)=b)\lor (b\notin F[A]\land a=a_0)}{%
    \FormulaRefAuto{(b,a)\in \Gsurjfrominj{F}{a_0}\vdash (b\in F[A]\land F(a)=b)\lor (b\notin F[A]\land a=a_0)}{2}}

  \proofstep{4}{b\in F[A]\land F(a)=b}{\rA}
  \proofstep{4}{b\in F[A]}{\rAEa{4}}
  \proofstep{1,4}{\bot}{\rBI{5,1}}
  \proofstep{1}{\neg(b\in F[A]\land F(a)=b)}{\rCI{4,6}}

  \proofstep{1,2}{(b\notin F[A]\land a=a_0)}{%
    \FormulaRefAuto{P \lor Q, \neg P \vdash Q}{3,7}}
  \proofstep{1,2}{a=a_0}{\rAEb{8}}
\end{tabproof}

% --- (5) Funktionalität (punktweise) ---
\FormulaThmDelta{%
  (b,a)\in \Gsurjfrominj{F}{a_0} \dsep (b,c)\in \Gsurjfrominj{F}{a_0}\vdash a=c%
}{
  \DeltaRow{Mengen}{A\dsep B\dsep a_0\dsep a\dsep b\dsep c}
  \DeltaPrem{Injektive Funktionen}{F\colon A\inj B}
}
\begin{tabproof}
  \proofstep{1}{(b,a)\in \Gsurjfrominj{F}{a_0}}{\rA}
  \proofstep{2}{(b,c)\in \Gsurjfrominj{F}{a_0}}{\rA}

  \proofstep{1}{a\in A}{%
    \FormulaRefAuto{(b,a)\in \Gsurjfrominj{F}{a_0}\vdash a\in A}{1}}
  \proofstep{2}{c\in A}{%
    \FormulaRefAuto{(b,a)\in \Gsurjfrominj{F}{a_0}\vdash a\in A}{2}}

  \proofstep{}{b\in F[A]\lor b\notin F[A]}{\FormulaRefAuto{P \lor \neg P}}

  \proofstep{6}{b\in F[A]}{\rA}
  \proofstep{1,6}{F(a)=b}{\FormulaRefAuto{b\in F[A]\dsep (b,a)\in \Gsurjfrominj{F}{a_0}\vdash F(a)=b}{6,1}}
  \proofstep{2,6}{F(c)=b}{\FormulaRefAuto{b\in F[A]\dsep (b,a)\in \Gsurjfrominj{F}{a_0}\vdash F(a)=b}{6,2}}
  \proofstep{1,2,6}{F(a)=F(c)}{\FormulaRefAuto{a=b,\, c=b \vdash a=c}{7,8}}
  \proofstep{1,2,6}{a=c}{\FormulaRefAuto{x\in A\dsep y\in A\dsep F(x)=F(y)\vdash x=y}{3,4,9}}

  \proofstep{11}{b\notin F[A]}{\rA}
  \proofstep{1,11}{a=a_0}{\FormulaRefAuto{b\notin F[A]\dsep (b,a)\in \Gsurjfrominj{F}{a_0}\vdash a=a_0}{11,1}}
  \proofstep{2,11}{c=a_0}{\FormulaRefAuto{b\notin F[A]\dsep (b,a)\in \Gsurjfrominj{F}{a_0}\vdash a=a_0}{11,2}}
  \proofstep{1,2,11}{a=c}{\FormulaRefAuto{a=b,\, c=b \vdash a=c}{12,13}}

  \proofstep{1,2}{a=c}{\rOE{5,6,10,11,14}}
\end{tabproof}

% --- (6) Totalität (punktweise) ---
\FormulaThmDelta{%
   a_0\in A \dsep b\in B \vdash \exists a\in A\,\bigl((b,a)\in \Gsurjfrominj{F}{a_0}\bigr)%
}{
  \DeltaRow{Mengen}{A\dsep B\dsep a_0\dsep a\dsep b}
  \DeltaPrem{Injektive Funktionen}{F\colon A\inj B}
}
\begin{tabproof}
  \proofstep{1}{a_0\in A}{\rA}
  \proofstep{2}{b\in B}{\rA}

  \proofstep{}{b\in F[A]\lor b\notin F[A]}{\FormulaRefAuto{P \lor \neg P}}

  \proofstep{4}{b\in F[A]}{\rA}
  \proofstep{4}{\exists x\in A\,b=F(x)}{\FormulaRefAuto{y\in F[A]\eqvdash \exists x\in A\,y=F(x)}{4}}

  \proofstep{6}{x\in A\land b=F(x)}{\rA}
  \proofstep{6}{x\in A}{\rAEa{6}}
  \proofstep{6}{b=F(x)}{\rAEb{6}}
  \proofstep{6}{F(x)=b}{\FormulaRefAuto{a=b\vdash b=a}{8}}
  \proofstep{4,6}{(b\in F[A]\land F(x)=b)}{\rAI{4,9}}
  \proofstep{4,6}{(b\in F[A]\land F(x)=b)\lor (b\notin F[A]\land x=a_0)}{\rOIa{10}}
  \proofstep{2,4,6}{(b,x)\in \Gsurjfrominj{F}{a_0}}{%
    \FormulaRefAuto{b\in B \dsep a\in A \dsep \bigl((b\in F[A]\land F(a)=b)\lor (b\notin F[A]\land a=a_0)\bigr)
    \vdash (b,a)\in \Gsurjfrominj{F}{a_0}}{2,7,11}}
  \proofstep{2,4,6}{\exists a\in A\,((b,a)\in \Gsurjfrominj{F}{a_0})}{\rEI{\rAI{7,12}}}
  \proofstep{2,4}{\exists a\in A\,((b,a)\in \Gsurjfrominj{F}{a_0})}{\rEE{5,6,13}}
  \proofstep{15}{b\notin F[A]}{\rA}
  \proofstep{}{a_0=a_0}{\rII}
  \proofstep{15}{b\notin F[A]\land a_0=a_0}{\rAI{15,16}}
  \proofstep{15}{(b\in F[A]\land F(a_0)=b)\lor (b\notin F[A]\land a_0=a_0)}{\rOIb{17}}
  \proofstep{1,2,15}{(b,a_0)\in \Gsurjfrominj{F}{a_0}}{%
    \FormulaRefAuto{b\in B \dsep a\in A \dsep \bigl((b\in F[A]\land F(a)=b)\lor (b\notin F[A]\land a=a_0)\bigr)
    \vdash (b,a)\in \Gsurjfrominj{F}{a_0}}{2,1,18}}
  \proofstep{1,15}{\exists a\in A\,((b,a)\in \Gsurjfrominj{F}{a_0})}{\rEI{\rAI{1,29}}}
  \proofstep{1,2}{\exists a\in A\,((b,a)\in \Gsurjfrominj{F}{a_0})}{\rOE{3,4,14,15,20}}
\end{tabproof}

% --- (7) Jetzt kurz: G ist Funktion ---
\FormulaThmDelta{%
  a_0\in A \vdash \Gsurjfrominj{F}{a_0}\colon B\to A%
}{
  \DeltaRow{Mengen}{A\dsep B\dsep a_0\dsep a\dsep b\dsep c}
  \DeltaPrem{Injektive Funktionen}{F\colon A\inj B}
}
\begin{tabproof}
  \proofstep{1}{a_0\in A}{\rA}
  \proofstep{}{%
    \Gsurjfrominj{F}{a_0}\subseteq B\times A}{%
    \FormulaRefAuto{\{\,x\in A \mid P(x)\,\}\subseteq A}}
  \proofstep{3}{b\in B}{\rA}
  \proofstep{1,3}{\exists a\in A\,((b,a)\in \Gsurjfrominj{F}{a_0})}{%
    \FormulaRefAuto{a_0\in A \dsep b\in B \vdash \exists a\in A\,\bigl((b,a)\in \Gsurjfrominj{F}{a_0}\bigr)}{1,3}} 
  \proofstep{1}{b\in B\rightarrow \exists a\in A\,((b,a)\in \Gsurjfrominj{F}{a_0})}{\rRI{3,4}}
  \proofstep{6}{(b,a)\in \Gsurjfrominj{F}{a_0}\land (b,c)\in \Gsurjfrominj{F}{a_0}}{\rA}
  \proofstep{6}{(b,a)\in \Gsurjfrominj{F}{a_0}}{\rAEa{6}}
  \proofstep{6}{(b,c)\in \Gsurjfrominj{F}{a_0}}{\rAEb{6}}
  \proofstep{6}{a=c}{%
    \FormulaRefAuto{(b,a)\in \Gsurjfrominj{F}{a_0} \dsep (b,c)\in \Gsurjfrominj{F}{a_0}\vdash a=c}{7,8}}
  \proofstep{}{((b,a)\in \Gsurjfrominj{F}{a_0}\land (b,c)\in \Gsurjfrominj{F}{a_0})\rightarrow a=c}{\rRI{6,9}}
  \proofstep{1}{\Gsurjfrominj{F}{a_0}\colon B\to A}{\FormulaRefAuto{Funktion}{2,5,10}}
\end{tabproof}


% --- (8) Kurz: Surjektivität ---
\FormulaThmDelta{%
  a_0\in A \vdash \Gsurjfrominj{F}{a_0}\colon B\sur A%
}{
  \DeltaRow{Mengen}{A\dsep B\dsep a_0\dsep b\dsep y}
  \DeltaPrem{Injektive Funktionen}{F\colon A\inj B}
}
\begin{tabproof}
  \proofstep{1}{a_0\in A}{\rA}
  \proofstep{1}{\Gsurjfrominj{F}{a_0}\colon B\to A}{%
    \FormulaRefAuto{a_0\in A \vdash \Gsurjfrominj{F}{a_0}\colon B\to A}{1}}
  \proofstep{3}{y\in A}{\rA}
  \proofstep{3}{(y,F(y))\in F}{\FormulaRefAuto{x\in A\vdash (x,F(x))\in F}{3}}
  \proofstep{3}{F(y)\in B}{\FormulaRefAuto{(x,y)\in F\vdash y\in B}{4}}
  \proofstep{3}{F(y)\in F[A]}{\FormulaRefAuto{x\in A\vdash F(x)\in F[A]}{3}}
  \proofstep{}{F(y)=F(y)}{\rII}
  \proofstep{3}{F(y)\in F[A]\land F(y)=F(y)}{\rAI{6,7}}
  \proofstep{3}{%
    \begin{aligned}[t]
      &(F(y)\in F[A]\land F(y)=F(y))\\
      &\lor (F(y)\notin F[A]\land y=a_0)
    \end{aligned}
  }{\rOIa{8}}
  \proofstep{3}{(F(y),y)\in \Gsurjfrominj{F}{a_0}}{%
    \FormulaRefAuto{b\in B \dsep a\in A \dsep \bigl((b\in F[A]\land F(a)=b)\lor (b\notin F[A]\land a=a_0)\bigr)
    \vdash (b,a)\in \Gsurjfrominj{F}{a_0}}{5,3,9}}
  \proofstep{3}{\Gsurjfrominj{F}{a_0}(F(y))=y}{\FormulaRefAuto{x\in A\dsep (x,y)\in F\vdash F(x)=y}{5,10}}
  
  \proofstep{3}{\exists b\in B\,\bigl(\Gsurjfrominj{F}{a_0}(b)=y\bigr)}{\rEI{\rAI{5,11}}}

  \proofstep{}{y\in A\rightarrow \exists b\in B\,\bigl(\Gsurjfrominj{F}{a_0}(b)=y\bigr)}{\rRI{3,12}}

  \proofstep{1}{\Gsurjfrominj{F}{a_0}\colon B\sur A}{\FormulaRefAuto{Surjektive Funktion}{2,13}}
\end{tabproof}

% --- (9) Optional: Existenzsatz mit A≠∅ ---
\FormulaThmDelta{%
  A\neq \varnothing \vdash \exists G\colon B\sur A%
}{
  \DeltaRow{Mengen}{A\dsep B}
  \DeltaPrem{Injektive Funktionen}{F\colon A\inj B}
}
\begin{tabproof}
  \proofstep{1}{A\neq \varnothing}{\rA}

  \proofstep{1}{\exists a_0\,(a_0\in A)}{\FormulaRefAuto{A\neq\varnothing \eqvdash \exists x\,(x \in A)}{1}}

  \proofstep{3}{a_0\in A}{\rA}
  \proofstep{3}{\Gsurjfrominj{F}{a_0}\colon B\sur A}{\FormulaRefAuto{a_0\in A \vdash \Gsurjfrominj{F}{a_0}\colon B\sur A}{3}}
  \proofstep{3}{\exists G\colon B\sur A}{\rEI{4}}

  \proofstep{1}{\exists G\colon B\sur A}{\rEE{2,3,5}}
\end{tabproof}



\subsection{Bijektivität}

\FormulaDefDeltaK[Begriff der bijektiven Funktion]{F\colon A\bij B}{Bijektive Funktion}{
  \DeltaRow{Mengen}{A\dsep B\dsep x\dsep y}
  \DeltaPrem{Funktionen}{F\colon A\to B}[\FormulaRefAuto{Funktion}]
%
  \DeltaRow{\textbf{Axiome}}{}
  %
  % — Injektivität —
  \DeltaRow{Injektivität}
           {x,y\in A\dsep F(x)=F(y)\vdash x=y}
           [\FormulaRefAuto{x\in A\dsep y\in A\dsep F(x)=F(y)\vdash x=y}]
  %
  % — Surjekvität —
  \DeltaRow{Surjektivität}
           {y\in B\vdash \exists x\in A\; F(x)=y}
           [\FormulaRefAuto{y\in B\vdash \exists x\in A\; F(x)=y}]
  \DeltaRow{\textbf{Neue Symbole}}{}
  \DeltaPrem{Bijektive Funktionen}{ }
}

\FormulaThmDelta{F\colon A\inj B\dsep F\colon A\sur B \vdash F\colon A\bij B}{
\DeltaRow{Mengen}{A\dsep B}
\DeltaPrem{Funktionen}{F\colon A\to B}
}
\begin{tabproof}
    \proofstep{1}{F\colon A\inj B}{\rA}
    \proofstep{2}{F\colon A\sur B}{\rA}
    \proofstep{1}{\forall x,y\in A\,(F(x)=F(y)\rightarrow x=y)}{\FormulaRefAuto{x\in A\dsep y\in A\dsep F(x)=F(y)\vdash x=y}{1}}
    \proofstep{1}{\forall y\in B\,\exists x\in A\,(F(x)=y)}{\FormulaRefAuto{y\in B\vdash \exists x\in A\; F(x)=y}{2}}
    \proofstep{1,2}{F\colon A\bij B}{\FormulaRefAuto{Bijektive Funktion}{3,4}}
\end{tabproof}

\FormulaThmDelta{F\colon A\bij B \vdash F\colon A\inj B}{
\DeltaRow{Mengen}{A\dsep B}
}
\begin{tabproof}
    \proofstep{1}{F\colon A\bij B}{\rA}
    \proofstep{1}{F\colon A\to B}{\FormulaRefAuto{Bijektive Funktion}{1}}
    \proofstep{1}{\forall x,y\in A\, (F(x)=F(y)\rightarrow x=y)}{\FormulaRefAuto{Bijektive Funktion}{1}}
    \proofstep{1}{F\colon A\inj B}{\FormulaRefAuto{Injektive Funktion}{2,3}}
\end{tabproof}

\FormulaThmDelta{F\colon A\bij B \vdash F\colon A\sur B}{
\DeltaRow{Mengen}{A\dsep B}
}
\begin{tabproof}
    \proofstep{1}{F\colon A\bij B}{\rA}
    \proofstep{1}{F\colon A\to B}{\FormulaRefAuto{Bijektive Funktion}{1}}
    \proofstep{1}{\forall y\in B\exists x\in A\; F(x)=y}{\FormulaRefAuto{Bijektive Funktion}{1}}
    \proofstep{1}{F\colon A\inj B}{\FormulaRefAuto{Injektive Funktion}{2,3}}
\end{tabproof}

\FormulaThmDelta{F\colon A\inj B\land F\colon A\sur B \eqvdash F\colon A\bij B }{
\DeltaRow{Mengen}{A\dsep B}
}
\begin{tabproofsplit}
    \proofpart{\(\vdash\)}
    \proofstep{1}{F\colon A\inj B\land F\colon A\sur B}{\rA}
    \proofstep{1}{F\colon A\inj B}{\rAEa{1}}
    \proofstep{1}{F\colon A\sur B}{\rAEb{2}}
    \proofstep{1}{F\colon A\bij B}{\FormulaRefAuto{F\colon A\inj B\dsep F\colon A\sur B \vdash F\colon A\bij B}{2,3}}
    \closeproofpart
    \proofpart{\(\dashv\)}
    \proofstep{1}{F\colon A\bij B}{\rA}
    \proofstep{1}{F\colon A\inj B}{\FormulaRefAuto{F\colon A\bij B \vdash F\colon A\inj B}{1}}
    \proofstep{1}{F\colon A\sur B}{\FormulaRefAuto{F\colon A\bij B \vdash F\colon A\sur B}{1}}
    \proofstep{1}{F\colon A\inj B\land F\colon A\sur B}{\rAI{2,3}}
    \closeproofpart
\end{tabproofsplit}

\FormulaThmDelta[Eindeutigkeit des Urbildes bei Bijektivität]{%
F\colon A\bij B \dsep y\in B \vdash \exists! x\in A\; F(x)=y
}{
  \DeltaRow{Mengen}{A \dsep B \dsep F\dsep x \dsep y \dsep z}
}
\begin{tabproof}
  % Annahmen
  \proofstep{1}{F\colon A\bij B}{\rA}
  \proofstep{2}{y\in B}{\rA}

  % Existenz eines Urbildes aus Surjektivität
  \proofstep{1,2}{\exists x\in A\,F(x)=y}{%
    \FormulaRefAuto{y\in B\vdash \exists x\in A\; F(x)=y}{1,2}}

  % Uniqueness-Teil: angenommen zwei Urbilder x,z
  \proofstep{3}{x\in A \land F(x)=y}{\rA}
  \proofstep{4}{z\in A \land F(z)=y}{\rA}

  % Aus den Konjunktionen die Einzelteile holen
  \proofstep{3}{x\in A}{\rAEa{3}}
  \proofstep{3}{F(x)=y}{\rAEb{3}}
  \proofstep{4}{z\in A}{\rAEa{4}}
  \proofstep{4}{F(z)=y}{\rAEb{4}}

  % F(x)=y und F(z)=y ⇒ F(x)=F(z) (über Gleichheitseigenschaften)
  \proofstep{4}{y=F(z)}{\FormulaRefAuto{a=b \vdash b=a}{9}}
  \proofstep{3,4}{F(x)=F(z)}{\FormulaRefAuto{a = b,\, b = c \vdash a = c}{7,10}}

  % Injektivität verwenden: aus x,z∈A und F(x)=F(z) folgt x=z
  \proofstep{1,3,4}{x=z}{%
    \FormulaRefAuto{x\in A\dsep y\in A\dsep F(x)=F(y)\vdash x=y}{6,8,11}}

  % ∃!-Einführung: aus Existenz (3) und Eindeutigkeit (Subbeweis 3–11)
  \proofstep{1,2}{\exists! x\in A\,F(x)=y}{\UEI{3,4,5,12}}
\end{tabproof}

\subsubsection{Gleichmächtigkeit}

\FormulaDefDeltaK[Begriff der gleichmächtigen Mengen]{A \EqCard B \coloneqq \exists F\colon A\bij B}{Gleichmächtigkeit}{
  \DeltaRow{Mengen}{A\dsep B}
  \DeltaRow{\textbf{Neue Symbole}}{}
  \DeltaRow{Gleichmächtigkeit}{}
}


\section{Beispiele von Funktionen}

\subsection{Die Identitätsfunktion}

\subsubsection{Definition der Identitätsfunktion}

\FormulaDefDelta[Funktionsvorschrift für die Identitätsfunktion auf \(A\)]
{x\in A \vdash t_{\Id_A}(x) \coloneqq x}
{
\DeltaRow{Mengen}{A\dsep x}
\DeltaRow{Funktionensymbol}{t_{\Id_A}}
}


% — Typisierungsaxiom / Wertebereich —
\FormulaThmDelta[Typisierungsaxiom]
{x\in A \vdash t_{\Id_A}(x)\in A}
{
  \DeltaRow{Mengen}{A\dsep x}
\DeltaRow{Funktionensymbol}{t_{\Id_A}}
}
\begin{tabproof}
    \proofstep{1}{x\in A}{\rA}
    \proofstep{1}{t_{\Id_A}(x)=x}{\FormulaRefAuto{x\in A \vdash t_{\Id_A}(x) \coloneqq x}{1}}
    \proofstep{1}{t_{\Id_A}(x)\in A}{\rIE{2,1}}
\end{tabproof}


\FormulaThmDelta[Identität als Funktion]{\exists! F\colon A\to A\,\forall x\in A\,F(x) = t_{\Id_A}(x)}{
  \DeltaRow{Mengen}{A\dsep x}
  \DeltaRow{Funktionensymbol}{t_{\Id_A}}
}
\begin{tabproof}
    \proofstep{}{\forall x\in A\,t_{\Id_A}(x)= x}{\FormulaRefAuto{x\in A \vdash t_{\Id_A}(x) \coloneqq x}}
    \proofstep{}{\forall x\in A\,t_{\Id_A}(x)\in A}{\FormulaRefAuto{x\in A \vdash t_{\Id_A}(x)\in A}}
    \proofstep{}{\exists! F\colon A\to A\,\forall x\in A\,F(x) = t_{\Id_A}(x)}{\FormulaRefAuto{x\in A\vdash t(x):=y\dsep x\in A\vdash t(x)\in B\exists! F\colon A\to B\,\forall x\in A\,F(x) = t(x)}{1,2}}
\end{tabproof}

\FormulaDefDelta[Identitätsfunktion auf \(A\)]
{\Id_A\coloneqq\iota F\left(F\colon A\to A \land \forall x\in A\, F(x)=t_{\Id_A}(x)\right)}
{
\DeltaRow{Mengen}{A}
\DeltaRow{Funktionensymbol}{t_{\Id_A}}
}

\FormulaThmDeltaK[Identitätsfunktion]
{\Id_A\colon A\to A}{IdA}
{
\DeltaRow{Mengen}{A}
}
\begin{tabproof}
    \proofstep{}{\Id_A\colon A\to A}{\rAEa{\FormulaRefAuto{\Id_A\coloneqq\iota F\left(F\colon A\to A \land \forall x\in A\, F(x)=t_{\Id_A}(x)\right)}}}
\end{tabproof}


\FormulaThmDelta
{\TotRel{\Id_A,A,A}}
{
\DeltaRow{Mengen}{A}
}
\begin{tabproof}
    \proofstep{}{\Id_A\colon A\to A}{\FormulaRefAuto{IdA}}
    \proofstep{}{\TotRel{\Id_A,A,A}}{\FormulaRefAuto{F\colon A\to B\vdash\TotRel{F,A,B}}{1}}
\end{tabproof}


\FormulaThmDelta[Identitätsfunktion auf \(A\)]
{x\in A\vdash \Id_A(x)=x}
{
\DeltaRow{Mengen}{A\dsep x}
}
\begin{tabproof}
    \proofstep{1}{x\in A}{\rA}
    \proofstep{}{\forall x\in A\, \Id_A(x)=t_{\Id_A}(x)}{\rAEb{\FormulaRefAuto{\Id_A\coloneqq\iota F\left(F\colon A\to A \land \forall x\in A\, F(x)=t_{\Id_A}(x)\right)}}}
    \proofstep{1}{\Id_A(x)=t_{\Id_A}(x)}{\rRE{1,\rUE{2}}}
    \proofstep{1}{t_{\Id_A}(x)=x}{\FormulaRefAuto{x\in A \vdash t_{\Id_A}(x) \coloneqq x}{1}}
    \proofstep{1}{\Id_A(x)=x}{\FormulaRefAuto{a = b,\, b = c \vdash a = c}{3,4}}
\end{tabproof}

\FormulaThmDelta{%
  x\in A\vdash x=\Id_A(x)%
}{
  \DeltaRow{Mengen}{A\dsep x}
}
\begin{tabproof}
  \proofstep{1}{x\in A}{\rA}
  \proofstep{1}{\Id_A(x)=x}{%
    \FormulaRefAuto{x\in A\vdash \Id_A(x)=x}}
  \proofstep{1}{x=\Id_A(x)}{%
    \FormulaRefAuto{a=b\vdash b=a}}
\end{tabproof}

\subsubsection{Die Bijektionseigenschaften}

\paragraph{Injektivität}
\FormulaThmDelta[Injektivität von \(\Id_A\)]{%
x\in A \dsep y\in A \dsep \Id_A(x)=\Id_A(y) \vdash x=y
}{
  \DeltaRow{Mengen}{x \dsep y \dsep A}
}
\begin{tabproof}
  \proofstep{1}{\Id_A(x)=\Id_A(y)}{\rA}
  \proofstep{2}{x\in A}{\rA}
  \proofstep{3}{y\in A}{\rA}
  \proofstep{2}{\Id_A(x)=x}{\FormulaRefAuto{x\in A\vdash \Id_A(x)=x}{2}}
  \proofstep{3}{\Id_A(y)=y}{\FormulaRefAuto{x\in A\vdash \Id_A(x)=x}{3}}
  \proofstep{1,2,3}{x=y}{\FormulaRefAuto{a = b\dsep a = c\dsep b=d\vdash c = d}{1,4,5}}
\end{tabproof}

\paragraph{Surjektivität}
\FormulaThmDelta[Surjektivität von \(\Id_A\)]{%
x\in A \vdash \exists y\in A\; \Id_A(y)=x
}{
  \DeltaRow{Mengen}{x \dsep y \dsep A}
}
\begin{tabproof}
  \proofstep{1}{x\in A}{\rA}
  \proofstep{1}{\Id_A(x)=x}{\FormulaRefAuto{x\in A\vdash \Id_A(x)=x}{1}}
  \proofstep{1}{\exists y\in A\; \Id_A(y)=x}{\rEI{\rAI{1,2}}}
\end{tabproof}

\paragraph{Bijektivität}
\FormulaThmDelta[\(\Id_A\) als bijektive Funktion]{%
\Id_A\colon A\bij A
}{
  \DeltaRow{Mengen}{A}
}
\begin{tabproof}
    \proofstep{}{\Id_A\colon A\to A}{\FormulaRefAuto{IdA}}
    \proofstep{}{\forall x,y\in A\,(\Id_A(x)=\Id_A(y) \rightarrow x=y)}{\FormulaRefAuto{x\in A \dsep y\in A \dsep \Id_A(x)=\Id_A(y) \vdash x=y}}
    \proofstep{}{\forall x\in A\exists y\in A\,\Id_A(y)=x}{\FormulaRefAuto{x\in A \vdash \exists y\in A\; \Id_A(y)=x}}
    \proofstep{}{\Id_A\colon A\bij A}{\FormulaRefAuto{Bijektive Funktion}{1,2,3}}
\end{tabproof}

\subsection{Die Projektionen auf die Komponenten}

\subsubsection{Projektion auf die erste Komponente}

\FormulaDefDelta[Funktionsvorschrift für die Projektion auf die erste Komponente]
{(x,y)\in A\times B \vdash t_{\pi_1}(x,y) \coloneqq x}
{
  \DeltaRow{Mengen}{A\dsep B\dsep x\dsep y}
  \DeltaRow{Funktionensymbol}{t_{\pi_1}}
}

% — Typisierungsaxiom / Wertebereich —
\FormulaThmDelta[Typisierungsaxiom für \(\pi_1\)]
{(x,y)\in A\times B \vdash t_{\pi_1}(x,y)\in A}
{
  \DeltaRow{Mengen}{A\dsep B\dsep x\dsep y}
  \DeltaRow{Funktionensymbol}{t_{\pi_1}}
}
\begin{tabproof}
  \proofstep{1}{(x,y)\in A\times B}{\rA}
  \proofstep{1}{x\in A \land y\in B}{%
    \FormulaRefAuto{(a,b)\in A\times B\eqvdash a\in A\land b\in B}{1}}
  \proofstep{1}{x\in A}{\rAEa{2}}
  \proofstep{1}{t_{\pi_1}(x,y)=x}{%
    \FormulaRefAuto{(x,y)\in A\times B \vdash t_{\pi_1}(x,y) \coloneqq x}{1}}
  \proofstep{1}{t_{\pi_1}(x,y)\in A}{\rIE{4,3}}
\end{tabproof}



\FormulaThmDelta[Projektion auf die erste Komponente als Funktion]{%
\exists! F\colon A\times B\to A\,\forall x\in A\,\forall y\in B\,F(x,y) = t_{\pi_1}(x,y)%
}{
  \DeltaRow{Mengen}{A\dsep B\dsep x\dsep y}
  \DeltaRow{Funktionensymbol}{t_{\pi_1}}
}
\begin{tabproof}
  \proofstep{}{\forall (x,y)\in A\times B\,t_{\pi_1}(x,y)= x}{\FormulaRefAuto{(x,y)\in A\times B \vdash t_{\pi_1}(x,y) \coloneqq x}}

  \proofstep{}{\forall (x,y)\in A\times B\,t_{\pi_1}(x,y)\in A}{\FormulaRefAuto{(x,y)\in A\times B \vdash t_{\pi_1}(x,y)\in A}}
    
  \proofstep{}{\exists! F\colon A\times B\to A\,\forall x\in A\,\forall y\in B\,F(x,y) = t_{\pi_1}(x,y)}{%
    \FormulaRefAuto{x\in A\vdash t(x):=y\dsep x\in A\vdash t(x)\in B\exists! F\colon A\to B\,\forall x\in A\,F(x) = t(x)}{1,2}}
\end{tabproof}

\FormulaDefDelta[Projektion auf die erste Komponente]
{\pi_1\coloneqq\iota F\left(F\colon A\times B\to A \land \forall x\in A\,\forall y\in B\, F(x,y)=t_{\pi_1}(x,y)\right)}
{
  \DeltaRow{Mengen}{A\dsep B}
  \DeltaRow{Funktionensymbol}{t_{\pi_1}}
}

\FormulaThmDelta[Projektion auf die erste Komponente]
{\pi_1\colon A\times B\to A}
{
  \DeltaRow{Mengen}{A\dsep B}
}
\begin{tabproof}
  \proofstep{}{\pi_1\colon A\times B\to A}{%
    \rAEa{\FormulaRefAuto{\pi_1\coloneqq\iota F\left(F\colon A\times B\to A \land \forall x\in A\,\forall y\in B\, F(x,y)=t_{\pi_1}(x,y)\right)}}}
\end{tabproof}

\FormulaThmDelta
{\TotRel{\pi_1,A\times B,A}}
{
  \DeltaRow{Mengen}{A\dsep B}
}
\begin{tabproof}
    \proofstep{}{\pi_1\colon A\times B\to A}{\FormulaRefAuto{IdA}}
    \proofstep{}{\TotRel{\pi_1,A\times B,A}}{\FormulaRefAuto{F\colon A\to B\vdash\TotRel{F,A,B}}{1}}
\end{tabproof}

\FormulaThmDelta[Grundgleichung der Projektion auf die erste Komponente]
{(x,y)\in A\times B \vdash \pi_1(x,y)=x}
{
  \DeltaRow{Mengen}{A\dsep B\dsep x\dsep y}
}
\begin{tabproof}
  \proofstep{1}{(x,y)\in A\times B}{\rA}
  \proofstep{}{%
    \forall x\in A\,\forall y\in B\, \pi_1(x,y)=t_{\pi_1}(x,y)}{%
    \rAEb{\FormulaRefAuto{\pi_1\coloneqq\iota F\left(F\colon A\times B\to A \land \forall x\in A\,\forall y\in B\, F(x,y)=t_{\pi_1}(x,y)\right)}}}
  \proofstep{1}{\pi_1(x,y)=t_{\pi_1}(x,y)}{\rRE{1,\rUE{2}}}
  \proofstep{1}{t_{\pi_1}(x,y)=x}{%
    \FormulaRefAuto{(x,y)\in A\times B \vdash t_{\pi_1}(x,y) \coloneqq x}{1}}
  \proofstep{1}{\pi_1(x,y)=x}{\FormulaRefAuto{a = b,\, b = c \vdash a = c}{3,4}}
\end{tabproof}

\FormulaThmDelta
{x\in A\dsep y\in B \vdash \pi_1(x,y)=x}
{
  \DeltaRow{Mengen}{A\dsep B\dsep x\dsep y}
  \DeltaRow{Funktionensymbol}{t_{\pi_1}}
}
\begin{tabproof}
  \proofstep{1}{x\in A}{\rA}
  \proofstep{2}{y\in B}{\rA}
  \proofstep{1,2}{(x,y)\in A\times B}{%
    \FormulaRefAuto{(a,b)\in A\times B\eqvdash a\in A\land b\in B}{1,2}}
  \proofstep{1,2}{\pi_1(x,y)=x}{%
    \FormulaRefAuto{(x,y)\in A\times B\vdash \pi_1(x,y)=x}{3}}
\end{tabproof}

\paragraph{Surjektivität}

\FormulaThmDelta[Surjektivität von \(\pi_1\)]{%
x\in A \dsep b_0\in B \vdash \exists z\in A\times B\; \pi_1(z)=x
}{
  \DeltaRow{Mengen}{A \dsep B \dsep x \dsep b_0 \dsep z}
}
\begin{tabproof}
  \proofstep{1}{x\in A}{\rA}
  \proofstep{2}{b_0\in B}{\rA}
  % Paarbildung
  \proofstep{1,2}{(x,b_0)\in A\times B}{%
    \FormulaRefAuto{(a,b)\in A\times B\eqvdash a\in A\land b\in B}{1,2}}
  % Grundgleichung der Projektion: (x,b0) in A×B ⇒ π1(x,b0)=x
  \proofstep{1,2}{\pi_1(x,b_0)=x}{%
    \FormulaRefAuto{(x,y)\in A\times B \vdash \pi_1(x,y)=x}{3}}
  % Existenzzeuge z = (x,b0) für Surjektivität
  \proofstep{1,2}{\exists z\in A\times B\; \pi_1(z)=x}{%
    \rEI{\rAI{3,4}}}
\end{tabproof}

\subsubsection{Projektion auf die zweite Komponente}

\FormulaDefDelta[Funktionsvorschrift für die Projektion auf die zweite Komponente]
{(x,y)\in A\times B \vdash t_{\pi_2}(x,y) \coloneqq y}
{
  \DeltaRow{Mengen}{A\dsep B\dsep x\dsep y}
  \DeltaRow{Funktionensymbol}{t_{\pi_2}}
}

% — Typisierungsaxiom / Wertebereich —
\FormulaThmDelta[Typisierungsaxiom für \(\pi_2\)]
{(x,y)\in A\times B \vdash t_{\pi_2}(x,y)\in B}
{
  \DeltaRow{Mengen}{A\dsep B\dsep x\dsep y}
  \DeltaRow{Funktionensymbol}{t_{\pi_2}}
}
\begin{tabproof}
  \proofstep{1}{(x,y)\in A\times B}{\rA}
  \proofstep{1}{x\in A \land y\in B}{%
    \FormulaRefAuto{(a,b)\in A\times B\eqvdash a\in A\land b\in B}{1}}
  \proofstep{1}{y\in B}{\rAEb{2}}
  \proofstep{1}{t_{\pi_2}(x,y)=y}{%
    \FormulaRefAuto{(x,y)\in A\times B \vdash t_{\pi_2}(x,y) \coloneqq y}{1}}
  \proofstep{1}{t_{\pi_2}(x,y)\in B}{\rIE{4,3}}
\end{tabproof}

\FormulaThmDelta[Projektion auf die zweite Komponente als Funktion]{%
\exists! F\colon A\times B\to B\,\forall x\in A\,\forall y\in B\,F(x,y) = t_{\pi_2}(x,y)%
}{
  \DeltaRow{Mengen}{A\dsep B\dsep x\dsep y}
  \DeltaRow{Funktionensymbol}{t_{\pi_2}}
}
\begin{tabproof}
  \proofstep{}{\forall (x,y)\in A\times B\,t_{\pi_2}(x,y)= y}{%
    \FormulaRefAuto{(x,y)\in A\times B \vdash t_{\pi_2}(x,y) \coloneqq y}}

  \proofstep{}{\forall (x,y)\in A\times B\,t_{\pi_2}(x,y)\in B}{%
    \FormulaRefAuto{(x,y)\in A\times B \vdash t_{\pi_2}(x,y)\in B}}
    
  \proofstep{}{\exists! F\colon A\times B\to B\,\forall x\in A\,\forall y\in B\,F(x,y) = t_{\pi_2}(x,y)}{%
    \FormulaRefAuto{x\in A\vdash t(x):=y\dsep x\in A\vdash t(x)\in B\exists! F\colon A\to B\,\forall x\in A\,F(x) = t(x)}{1,2}}
\end{tabproof}

\FormulaDefDelta[Projektion auf die zweite Komponente]
{\pi_2\coloneqq\iota F\left(F\colon A\times B\to B \land \forall x\in A\,\forall y\in B\, F(x,y)=t_{\pi_2}(x,y)\right)}
{
  \DeltaRow{Mengen}{A\dsep B}
  \DeltaRow{Funktionensymbol}{t_{\pi_2}}
}

\FormulaThmDelta[Projektion auf die zweite Komponente]
{\pi_2\colon A\times B\to B}
{
  \DeltaRow{Mengen}{A\dsep B}
}
\begin{tabproof}
  \proofstep{}{\pi_2\colon A\times B\to B}{%
    \rAEa{\FormulaRefAuto{\pi_2\coloneqq\iota F\left(F\colon A\times B\to B \land \forall x\in A\,\forall y\in B\, F(x,y)=t_{\pi_2}(x,y)\right)}}}
\end{tabproof}

\FormulaThmDelta
{\TotRel{\pi_2,A\times B,B}}
{
  \DeltaRow{Mengen}{A\dsep B}
}
\begin{tabproof}
    \proofstep{}{\pi_2\colon A\times B\to B}{\FormulaRefAuto{IdA}}
    \proofstep{}{\TotRel{\pi_2,A\times B,B}}{\FormulaRefAuto{F\colon A\to B\vdash\TotRel{F,A,B}}{1}}
\end{tabproof}

\FormulaThmDelta[Grundgleichung der Projektion auf die zweite Komponente]
{(x,y)\in A\times B \vdash \pi_2(x,y)=y}
{
  \DeltaRow{Mengen}{A\dsep B\dsep x\dsep y}
}
\begin{tabproof}
  \proofstep{1}{(x,y)\in A\times B}{\rA}
  \proofstep{}{%
    \forall x\in A\,\forall y\in B\, \pi_2(x,y)=t_{\pi_2}(x,y)}{%
    \rAEb{\FormulaRefAuto{\pi_2\coloneqq\iota F\left(F\colon A\times B\to B \land \forall x\in A\,\forall y\in B\, F(x,y)=t_{\pi_2}(x,y)\right)}}}
  \proofstep{1}{\pi_2(x,y)=t_{\pi_2}(x,y)}{\rRE{1,\rUE{2}}}
  \proofstep{1}{t_{\pi_2}(x,y)=y}{%
    \FormulaRefAuto{(x,y)\in A\times B \vdash t_{\pi_2}(x,y) \coloneqq y}{1}}
  \proofstep{1}{\pi_2(x,y)=y}{\FormulaRefAuto{a = b,\, b = c \vdash a = c}{3,4}}
\end{tabproof}

\FormulaThmDelta
{x\in A\dsep y\in B\,\pi_2(x,y)=y}
{
  \DeltaRow{Mengen}{A\dsep B\dsep x\dsep y}
  \DeltaRow{Funktionensymbol}{t_{\pi_2}}
}
\begin{tabproof}
  \proofstep{1}{x\in A}{\rA}
  \proofstep{2}{y\in B}{\rA}
  \proofstep{1,2}{(x,y)\in A\times B}{%
    \FormulaRefAuto{(a,b)\in A\times B\eqvdash a\in A\land b\in B}{1,2}}
  \proofstep{1,2}{\pi_2(x,y)=y}{%
    \FormulaRefAuto{(x,y)\in A\times B \vdash \pi_2(x,y)=y}{3}}
\end{tabproof}

\paragraph{Surjektivität}

\FormulaThmDelta[Surjektivität von \(\pi_2\)]{%
y\in B \dsep a_0\in A \vdash \exists z\in A\times B\; \pi_2(z)=y
}{
  \DeltaRow{Mengen}{A \dsep B \dsep y \dsep a_0 \dsep z}
}
\begin{tabproof}
  \proofstep{1}{y\in B}{\rA}
  \proofstep{2}{a_0\in A}{\rA}
  % Paarbildung
  \proofstep{1,2}{(a_0,y)\in A\times B}{%
    \FormulaRefAuto{a\in A \dsep b\in B \vdash (a,b)\in A\times B}{2,1}}
  % Grundgleichung der Projektion: (a0,y) in A×B ⇒ π2(a0,y)=y
  \proofstep{1,2}{\pi_2(a_0,y)=y}{%
    \FormulaRefAuto{(x,y)\in A\times B \vdash \pi_2(x,y)=y}{3}}
  % Existenzzeuge z = (a0,y) für Surjektivität
  \proofstep{1,2}{\exists z\in A\times B\; \pi_2(z)=y}{%
    \rEI{\rAI{3,4}}}
\end{tabproof}

\subsubsection{Eigenschaften von Projektionsabbildungen}

\FormulaThmDelta[Rekonstruktion aus Projektionen]{%
  (x,y)\in A\times B \vdash (x,y) = \bigl(\pi_1(x,y),\pi_2(x,y)\bigr)
}{
  \DeltaRow{Mengen}{A\dsep B\dsep x\dsep y}
  \DeltaRow{Projektion auf erste Komponente}{\pi_1:A\times B\to A}
  \DeltaRow{Projektion auf zweite Komponente}{\pi_2:A\times B\to B}
}
\begin{tabproofwide}
  \proofstepwidestar[1]{(x,y)\in A\times B}{\rA}
  \proofstepwidestar[1]{x=\pi_1(x,y)}{%
    \FormulaRefAuto{a=b\vdash b=a}{\FormulaRefAuto{(x,y)\in A\times B \vdash \pi_1(x,y)=x}{1}}}
  \proofstepwidestar[1]{y=\pi_2(x,y)}{%
    \FormulaRefAuto{a=b\vdash b=a}{\FormulaRefAuto{(x,y)\in A\times B \vdash \pi_2(x,y)=y}{1}}}
  \proofstepwide[1]{(x,y)}{=}{(x,y)}{\rII}
  \proofstepwide[1]{}{=}{(\pi_1(x,y),y)}{\rIE{2,4}}
  \proofstepwide[1]{}{=}{(\pi_1(x,y),\pi_2(x,y))}{\rIE{3,5}}
  \proofstepwide[1]{(x,y)}{=}{(\pi_1(x,y),\pi_2(x,y))}{\rChain{4,6}}
\end{tabproofwide}

\FormulaThmDelta[Element des Produkts als Paar]{%
  z\in A\times B \vdash z = \bigl(\pi_1(z),\pi_2(z)\bigr)
}{
  \DeltaRow{Mengen}{A\dsep B\dsep z}
  \DeltaRow{Projektion auf erste Komponente}{\pi_1:A\times B\to A}
  \DeltaRow{Projektion auf zweite Komponente}{\pi_2:A\times B\to B}
}
\begin{tabproof}
  \proofstep{1}{z\in A\times B}{\rA}
  
  \proofstep{1}{\exists x\in A\,\exists y\in B\, z=(x,y)}{%
    \FormulaRefAuto{x\in A\times B\eqvdash \exists a \in A\, \exists b \in B\,x=(a,b)}{1}}
  
  \proofstep{3}{x\in A\land y\in B\land z=(x,y)}{\rA}
  
  \proofstep{3}{x\in A}{\rAEa{3}}
  
  \proofstep{3}{y\in B}{\FormulaRefAuto{P\land Q\land R\vdash Q}{3}}
  
  \proofstep{3}{z=(x,y)}{\rAEb{3}}
  
  \proofstep{3}{(x,y)\in A\times B}{%
    \FormulaRefAuto{(a,b)\in A\times B\eqvdash a\in A\land b\in B}{4,5}}
  
  \proofstep{3}{(x,y)=\bigl(\pi_1(x,y),\pi_2(x,y)\bigr)}{%
    \FormulaRefAuto{(x,y)\in A\times B \vdash (x,y)=\bigl(\pi_1(x,y),\pi_2(x,y)\bigr)}{7}}

  \proofstep{3}{z=\bigl(\pi_1(z),\pi_2(z)\bigr)}{%
    \rIE{6,8}}
  
  \proofstep{1}{z=\bigl(\pi_1(z),\pi_2(z)\bigr)}{%
    \rEE{2,3,9}}
\end{tabproof}

\subsection{Die Inklusionsabbildung}

\subsubsection{Definition der Inklusionsabbildung}

\FormulaDefDelta[Funktionsvorschrift für die Inklusionsabbildung von \(A\) nach \(B\)]
{A\subseteq B \dsep x\in A \vdash t_{\iota_{A,B}}(x) \coloneqq x}
{
  \DeltaRow{Mengen}{A\dsep B\dsep x}
  \DeltaRow{Funktionensymbol}{t_{\iota_{A,B}}}
}

% — Typisierungsaxiom / Wertebereich —
\FormulaThmDelta[Typisierungsaxiom für \(t_{\iota_{A,B}}\)]
{A\subseteq B \dsep x\in A \vdash t_{\iota_{A,B}}(x)\in B}
{
  \DeltaRow{Mengen}{A\dsep B\dsep x}
  \DeltaRow{Funktionensymbol}{t_{\iota_{A,B}}}
}
\begin{tabproof}
  \proofstep{1}{A\subseteq B}{\rA}
  \proofstep{2}{x\in A}{\rA}
  \proofstep{1,2}{x\in B}{%
    \FormulaRefAuto{A\subseteq B, x\in A \vdash x\in B}{1,2}}
  \proofstep{1,2}{t_{\iota_{A,B}}(x)=x}{%
    \FormulaRefAuto{A\subseteq B \dsep x\in A \vdash t_{\iota_{A,B}}(x) \coloneqq x}{1,2}}
  \proofstep{1,2}{t_{\iota_{A,B}}(x)\in B}{\rIE{4,3}}
\end{tabproof}

\FormulaThmDelta[Inklusionsabbildung als Funktion]{%
A\subseteq B\vdash \exists! F\colon A\to B\,\forall x\in A\,F(x) = t_{\iota_{A,B}}(x)%
}{
  \DeltaRow{Mengen}{A\dsep B\dsep x}
  \DeltaRow{Funktionensymbol}{t_{\iota_{A,B}}}
}
\begin{tabproof}
  \proofstep{1}{A\subseteq B}{\rA}
  \proofstep{1}{\forall x\in A\,t_{\iota_{A,B}}(x)=x}{\FormulaRefAuto{A\subseteq B \dsep x\in A \vdash t_{\iota_{A,B}}(x) \coloneqq x}{1}}
  \proofstep{1}{\forall x\in A\,t_{\iota_{A,B}}(x)\in B}{\FormulaRefAuto{A\subseteq B \dsep x\in A \vdash t_{\iota_{A,B}}(x)\in B}{1}}
  \proofstep{1}{\exists! F\colon A\to B\,\forall x\in A\,F(x) = t_{\iota_{A,B}}(x)}{%
    \FormulaRefAuto{x\in A\vdash t(x):=y\dsep x\in A\vdash t(x)\in B\exists! F\colon A\to B\,\forall x\in A\,F(x) = t(x)}{2,3}}
\end{tabproof}

\FormulaDefDelta[Inklusionsabbildung von \(A\) nach \(B\)]
{A\subseteq B\vdash \iota_{A,B}\coloneqq\iota F\left(F\colon A\to B \land \forall x\in A\, F(x)=t_{\iota_{A,B}}(x)\right)}
{
  \DeltaRow{Mengen}{A\dsep B}
  \DeltaRow{Funktionensymbol}{t_{\iota_{A,B}}}
}

\FormulaThmDelta[Inklusionsabbildung von \(A\) nach \(B\)]
{A\subseteq B\vdash \iota_{A,B}\colon A\to B}
{
  \DeltaRow{Mengen}{A\dsep B}
}
\begin{tabproof}
  \proofstep{1}{A\subseteq B}{\rA}
  \proofstep{1}{\iota_{A,B}\colon A\to B}{%
    \rAEa{\FormulaRefAuto{A\subseteq B\vdash \iota_{A,B}\coloneqq\iota F\left(F\colon A\to B \land \forall x\in A\, F(x)=t_{\iota_{A,B}}(x)\right)}{1}}}
\end{tabproof}

\FormulaThmDelta
{A\subseteq B\vdash \TotRel{\iota_{A,B},A,B}}
{
  \DeltaRow{Mengen}{A\dsep B}
}
\begin{tabproof}
    \proofstep{1}{A\subseteq B}{\rA}
    \proofstep{}{\iota_{A,B}\colon A\to B}{\FormulaRefAuto{IdA}{1}}
    \proofstep{}{\TotRel{\iota_{A,B},A,B}}{\FormulaRefAuto{F\colon A\to B\vdash\TotRel{F,A,B}}{2}}
\end{tabproof}

\FormulaThmDelta[Grundgleichung der Inklusionsabbildung]
{A\subseteq B \dsep x\in A \vdash \iota_{A,B}(x)=x}
{
  \DeltaRow{Mengen}{A\dsep B\dsep x}
}
\begin{tabproof}
  \proofstep{1}{A\subseteq B}{\rA}
  \proofstep{2}{x\in A}{\rA}
  \proofstep{1}{%
    \forall x\in A\, \iota_{A,B}(x)=t_{\iota_{A,B}}(x)}{%
    \rAEb{\FormulaRefAuto{A\subseteq B\vdash \iota_{A,B}\coloneqq\iota F\left(F\colon A\to B \land \forall x\in A\, F(x)=t_{\iota_{A,B}}(x)\right)}{1}}}
  \proofstep{2}{\iota_{A,B}(x)=t_{\iota_{A,B}}(x)}{\rRE{2,\rUE{3}}}
  \proofstep{1,2}{t_{\iota_{A,B}}(x)=x}{%
    \FormulaRefAuto{A\subseteq B \dsep x\in A \vdash t_{\iota_{A,B}}(x) \coloneqq x}{1,2}}
  \proofstep{1,2}{\iota_{A,B}(x)=x}{\FormulaRefAuto{a = b,\, b = c \vdash a = c}{4,5}}
\end{tabproof}

\FormulaThmDelta
{A\subseteq B \dsep x\in A \vdash x=\iota_{A,B}(x)}%
{
  \DeltaRow{Mengen}{A\dsep B\dsep x}
}
\begin{tabproof}
  \proofstep{1}{A\subseteq B}{\rA}
  \proofstep{2}{x\in A}{\rA}

  \proofstep{1,2}{\iota_{A,B}(x)=x}{%
    \FormulaRefAuto{A\subseteq B \dsep x\in A \vdash \iota_{A,B}(x)=x}{1,2}}

  \proofstep{1,2}{x=\iota_{A,B}(x)}{%
    \FormulaRefAuto{a=b\vdash b=a}{3}}
\end{tabproof}


\subsubsection{Die Injektivität der Inklusionsabbildung}

\paragraph{Injektivität}
\FormulaThmDelta[Injektivität von \(\iota_{A,B}\)]{%
x\in A \dsep y\in A \dsep \iota_{A,B}(x)=\iota_{A,B}(y) \vdash x=y
}{
  \DeltaRow{Mengen}{A \dsep B \dsep x \dsep y}
}
\begin{tabproof}
  \proofstep{1}{\iota_{A,B}(x)=\iota_{A,B}(y)}{\rA}
  \proofstep{2}{x\in A}{\rA}
  \proofstep{3}{y\in A}{\rA}
  \proofstep{2}{\iota_{A,B}(x)=x}{%
    \FormulaRefAuto{A\subseteq B \dsep x\in A \vdash \iota_{A,B}(x)=x}{2}}
  \proofstep{3}{\iota_{A,B}(y)=y}{%
    \FormulaRefAuto{A\subseteq B \dsep x\in A \vdash \iota_{A,B}(x)=x}{3}}
  \proofstep{1,2,3}{x=y}{%
    \FormulaRefAuto{a=b\dsep a=c\dsep b=d\vdash c=d}{1,4,5}}
\end{tabproof}

\subsection{Die Faserabbildung}

\subsubsection{Funktionsvorschrift}

\FormulaDefDelta[Funktionsvorschrift der Faser]{%
  y\in B \vdash t_{\Fib_F}(y) \coloneqq F^{-1}[\{y\}]%
}{
  \DeltaRow{Mengen}{A\dsep B\dsep y}
  \DeltaPrem{Funktionen}{F\colon A\to B}
  \DeltaRow{Funktionensymbol}{t_{\Fib_F}}
}

\FormulaThmDelta[Typisierungsaxiom für $t_{\Fib_F}$]{%
  y\in B \vdash t_{\Fib_F}(y)\in \powerset(A)%
}{
  \DeltaRow{Mengen}{A\dsep B\dsep y}
  \DeltaPrem{Funktionen}{F\colon A\to B}
  \DeltaRow{Funktionensymbol}{t_{\Fib_F}}
}
\begin{tabproof}
  \proofstep{1}{y\in B}{\rA}
  \proofstep{1}{t_{\Fib_F}(y)=F^{-1}[\{y\}]}{%
    \FormulaRefAuto{y\in B \vdash t_{\Fib_F}(y) \coloneqq F^{-1}[\{y\}]}{1}}
  \proofstep{1}{\{y\}\subseteq B}{%
    \FormulaRefAuto{x\in A\vdash \{x\}\subseteq A}{1}}
  \proofstep{1}{F^{-1}[\{y\}]=\{\,u\in A \mid F(u)\in \{y\}\,\}}{%
    \FormulaRefAuto{C\subseteq B \vdash F^{-1}[C] \coloneqq \{\,x\in A \mid F(x)\in C\,\}}{3}}
  \proofstep{1}{\{\,u\in A \mid F(u)\in \{y\}\,\}\subseteq A}{%
    \FormulaRefAuto{\{\,x\in A \mid P(x)\,\}\subseteq A}}
  \proofstep{1}{F^{-1}[\{y\}]\subseteq A}{\rIE{4,5}}
  \proofstep{1}{t_{\Fib_F}(y)\subseteq A}{\rIE{2,6}}
  \proofstep{1}{t_{\Fib_F}(y)\in \powerset(A)}{%
    \FormulaRefAuto{x \in \powerset(A)\;\eqvdash\; x \subseteq A}{7}}
\end{tabproof}

\subsubsection{Die Faser als Abbildung}

\FormulaThmDelta[Faser als Funktion]{%
  \exists! G\colon B\to \powerset(A)\,\forall y\in B\,G(y)=t_{\Fib_F}(y)%
}{
  \DeltaRow{Mengen}{A\dsep B\dsep y}
  \DeltaPrem{Funktionen}{F\colon A\to B}
  \DeltaRow{Funktionensymbol}{t_{\Fib_F}}
}
\begin{tabproof}
  \proofstep{}{\forall y\in B\,t_{\Fib_F}(y)\in \powerset(A)}{%
    \FormulaRefAuto{y\in B \vdash t_{\Fib_F}(y)\in \powerset(A)}}
  \proofstep{}{%
    \exists! G\colon B\to \powerset(A)\,\forall y\in B\,G(y)=t_{\Fib_F}(y)}{%
    \FormulaRefAuto{x\in A\vdash t(x):=y\dsep x\in A\vdash t(x)\in B\exists! F\colon A\to B\,\forall x\in A\,F(x) = t(x)}{1}}
\end{tabproof}

\FormulaDefDelta[Faser zu $F$]{%
  \Fib_F\coloneqq\iota G\left(G\colon B\to \powerset(A) \land \forall y\in B\, G(y)=t_{\Fib_F}(y)\right)%
}{
  \DeltaRow{Mengen}{A\dsep B}
  \DeltaPrem{Funktionen}{F\colon A\to B}
  \DeltaRow{Funktionensymbol}{t_{\Fib_F}}
}

\FormulaThmDelta[Faser ist eine Funktion]{%
  \Fib_F\colon B\to \powerset(A)%
}{
  \DeltaRow{Mengen}{A\dsep B}
  \DeltaPrem{Funktionen}{F\colon A\to B}
}
\begin{tabproof}
  \proofstep{}{\Fib_F\colon B\to \powerset(A)}{%
    \rAEa{\FormulaRefAuto{\Fib_F\coloneqq\iota G\left(G\colon B\to \powerset(A) \land \forall y\in B\, G(y)=t_{\Fib_F}(y)\right)}}}
\end{tabproof}

\FormulaThmDelta[Grundgleichung der Faser]{%
  y\in B \vdash \Fib_F(y)=F^{-1}[\{y\}]%
}{
  \DeltaRow{Mengen}{A\dsep B\dsep y}
  \DeltaPrem{Funktionen}{F\colon A\to B}
  \DeltaRow{Funktionensymbol}{t_{\Fib_F}}
}
\begin{tabproof}
  \proofstep{1}{y\in B}{\rA}
  \proofstep{}{%
    \forall y\in B\, \Fib_F(y)=t_{\Fib_F}(y)}{%
    \rAEb{\FormulaRefAuto{\Fib_F\coloneqq\iota G\left(G\colon B\to \powerset(A) \land \forall y\in B\, G(y)=t_{\Fib_F}(y)\right)}}}
  \proofstep{1}{\Fib_F(y)=t_{\Fib_F}(y)}{\rRE{1,\rUE{2}}}
  \proofstep{1}{t_{\Fib_F}(y)=F^{-1}[\{y\}]}{%
    \FormulaRefAuto{y\in B \vdash t_{\Fib_F}(y) \coloneqq F^{-1}[\{y\}]}{1}}
  \proofstep{1}{\Fib_F(y)=F^{-1}[\{y\}]}{%
    \FormulaRefAuto{a = b,\, b = c \vdash a = c}{3,4}}
\end{tabproof}

\subsubsection{Die Faser als Menge}

\FormulaThmDelta[Faser als Urbild eines Singletons]{%
  y\in B \vdash \Fib_F(y) = \{\,x\in A \mid F(x)=y\,\}%
}{
  \DeltaRow{Mengen}{A\dsep B\dsep x\dsep y}
  \DeltaPrem{Funktionen}{F\colon A\to B}
}
\begin{tabproofwide}
  \proofstepwidestar[1]{y\in B}{\rA}
  \proofstepwidestar[1]{\{y\}\subseteq B}{%
    \FormulaRefAuto{x\in A\vdash \{x\}\subseteq A}{1}}

  \proofstepwide[1]{\Fib_F(y)}{=}{F^{-1}[\{y\}]}{%
    \FormulaRefAuto{y\in B \vdash \Fib_F(y)=F^{-1}[\{y\}]}{1}}

  \proofstepwide[1]{}{=}{\{\,u\in A \mid F(u)\in \{y\}\,\}}{%
    \FormulaRefAuto{C\subseteq B \vdash F^{-1}[C] \coloneqq \{\,x\in A \mid F(x)\in C\,\}}{2}}

  \proofstepwide[1]{}{=}{\{\,u\in A \mid F(u)=y\,\}}{%
    \FormulaRefAuto{\{x\in A\mid F(x)\in \{a\}\}=\{x\in A\mid F(x)=a\}}}

  \proofstepwide[1]{\Fib_F(y)}{=}{\{\,u\in A \mid F(u)=y\,\}}{%
    \rChain{3,4,5}}
\end{tabproofwide}


\FormulaThmDelta{y\in B\dsep x\in \Fib_F(y) \vdash x\in A\land F(x)=y}
{
  \DeltaRow{Mengen}{A\dsep B\dsep x\dsep y}
  \DeltaPrem{Funktionen}{F\colon A\to B}
}
\begin{tabproof}
  \proofstep{1}{y\in B}{\rA}
  \proofstep{2}{x\in \Fib_F(y)}{\rA}
  \proofstep{1}{\Fib_F(y)=\{\,t\in A \mid F(t)=y\,\}}{%
    \FormulaRefAuto{y\in B \vdash \Fib_F(y) = \{\,x\in A \mid F(x)=y\,\}}{1}}
  \proofstep{1,2}{x\in \{\,t\in A \mid F(t)=y\,\}}{\rIE{3,2}}
  \proofstep{1,2}{x\in A \land F(x)=y}{%
    \FormulaRefAuto{x\in\{u\in A\mid P(u)\}\eqvdash x\in A\land P(x)}{4}}
\end{tabproof}

\FormulaThmDelta{y\in B\dsep x\in \Fib_F(y) \vdash x\in A}
{
  \DeltaRow{Mengen}{A\dsep B\dsep x\dsep y}
  \DeltaPrem{Funktionen}{F\colon A\to B}
}
\begin{tabproof}
  \proofstep{1}{y\in B}{\rA}
  \proofstep{2}{x\in \Fib_F(y)}{\rA}
  \proofstep{1,2}{x\in A\land F(x)=y}{\FormulaRefAuto{y\in B\dsep x\in \Fib_F(y) \vdash x\in A\land F(x)=y}{1,2}}
  \proofstep{1,2}{x\in A}{\rAEa{3}}
\end{tabproof}

\FormulaThmDelta{y\in B\dsep x\in \Fib_F(y) \vdash F(x)=y}
{
  \DeltaRow{Mengen}{A\dsep B\dsep x\dsep y}
  \DeltaPrem{Funktionen}{F\colon A\to B}
}
\begin{tabproof}
  \proofstep{1}{y\in B}{\rA}
  \proofstep{2}{x\in \Fib_F(y)}{\rA}
  \proofstep{1,2}{x\in A\land F(x)=y}{\FormulaRefAuto{y\in B\dsep x\in \Fib_F(y) \vdash x\in A\land F(x)=y}{1,2}}
  \proofstep{1,2}{F(x)=y}{\rAEb{3}}
\end{tabproof}

\FormulaThmDelta
{y\in B\dsep x\in A\dsep F(x)=y\vdash x\in \Fib_F(y)}
{
  \DeltaRow{Mengen}{A\dsep B\dsep x\dsep y}
  \DeltaPrem{Funktionen}{F\colon A\to B}
}
\begin{tabproof}
  \proofstep{1}{y\in B}{\rA}
  \proofstep{2}{x\in A}{\rA}
  \proofstep{3}{F(x)=y}{\rA}
  \proofstep{2,3}{x\in A\land F(x)=y}{\rAI{2,3}}
  \proofstep{2,3}{x\in \{\,t\in A \mid F(t)=y\,\}}{%
    \FormulaRefAuto{x\in\{u\in A\mid P(u)\}\eqvdash x\in A\land P(x)}{4}}
  \proofstep{1}{\Fib_F(y)=\{\,t\in A \mid F(t)=y\,\}}{%
    \FormulaRefAuto{y\in B \vdash \Fib_F(y) = \{\,x\in A \mid F(x)=y\,\}}{1}}
  \proofstep{1,2,3}{x\in \Fib_F(y)}{\rIE{6,5}}
\end{tabproof}


\FormulaThmDelta
{y\in B\vdash x\in\Fib_F(y)\leftrightarrow x\in A\land F(x)=y}
{
  \DeltaRow{Mengen}{A\dsep B\dsep x\dsep y}
  \DeltaPrem{Funktionen}{F\colon A\to B}
}
\begin{tabproof}
  \proofstep{1}{y\in B}{\rA}

  \proofstep{2}{x\in \Fib_F(y)}{\rA}
  \proofstep{1,2}{x\in A\land F(x)=y}{%
    \FormulaRefAuto{y\in B\dsep x\in \Fib_F(y) \vdash x\in A\land F(x)=y}{1,2}}
  \proofstep{1}{x\in \Fib_F(y)\rightarrow (x\in A\land F(x)=y)}{\rRI{2,3}}

  \proofstep{5}{x\in A\land F(x)=y}{\rA}
  \proofstep{5}{x\in A}{\rAEa{5}}
  \proofstep{5}{F(x)=y}{\rAEb{5}}
  \proofstep{1,5}{x\in \Fib_F(y)}{%
    \FormulaRefAuto{y\in B\dsep x\in A\dsep F(x)=y\vdash x\in \Fib_F(y)}{1,6,7}}
  \proofstep{1}{(x\in A\land F(x)=y)\rightarrow x\in \Fib_F(y)}{\rRI{5,8}}

  \proofstep{1}{x\in\Fib_F(y)\leftrightarrow (x\in A\land F(x)=y)}{\rLRI{4,9}}
\end{tabproof}

\FormulaThmDelta[Verschiedene Fasern sind disjunkt]{%
  y\in B \dsep z\in B \dsep y\neq z \vdash \Fib_F(y)\cap \Fib_F(z)=\varnothing%
}{
  \DeltaRow{Mengen}{A\dsep B\dsep y\dsep z\dsep a}
  \DeltaPrem{Funktionen}{F\colon A\to B}
}
\begin{tabproof}
  \proofstep{1}{y\in B}{\rA}
  \proofstep{2}{z\in B}{\rA}
  \proofstep{3}{y\neq z}{\rA}

  \proofstep{4}{a\in \Fib_F(y)\cap \Fib_F(z)}{\rA}
  \proofstep{4}{a\in \Fib_F(y)}{%
    \FormulaRefAuto{x \in A \cap B \vdash x \in A}{4}}
  \proofstep{4}{a\in \Fib_F(z)}{%
    \FormulaRefAuto{x \in A \cap B \vdash x \in B}{4}}

  \proofstep{1,4}{F(a)=y}{%
    \FormulaRefAuto{y\in B\dsep x\in \Fib_F(y) \vdash F(x)=y}{1,5}}
  \proofstep{2,4}{F(a)=z}{%
    \FormulaRefAuto{y\in B\dsep x\in \Fib_F(y) \vdash F(x)=y}{2,6}}

  \proofstep{1,2,4}{y=z}{%
    \FormulaRefAuto{a=b \dsep a=c \vdash b=c}{7,8}}
  \proofstep{1,2,3,4}{\bot}{\rBI{3,9}}

  \proofstep{1,2,3}{a\notin \Fib_F(y)\cap \Fib_F(z)}{\rCE{4,10}}
  \proofstep{1,2,3}{\forall a\,\bigl(a\notin \Fib_F(y)\cap \Fib_F(z)\bigr)}{\rUI{11}}
  \proofstep{1,2,3}{\Fib_F(y)\cap \Fib_F(z)=\varnothing}{%
    \FormulaRefAuto{\forall b\,b\notin X\vdash X=\varnothing}{12}}
\end{tabproof}

\FormulaThmDelta{%
  y\in B \dsep z\in B \dsep X=\Fib_F(y) \dsep Y=\Fib_F(z) \dsep X\neq Y
  \vdash X\cap Y=\varnothing%
}{
  \DeltaRow{Mengen}{A\dsep B\dsep X\dsep Y\dsep y\dsep z}
  \DeltaPrem{Funktionen}{F\colon A\to B}
}
\begin{tabproof}
  \proofstep{1}{y\in B}{\rA}
  \proofstep{2}{z\in B}{\rA}
  \proofstep{3}{X=\Fib_F(y)}{\rA}
  \proofstep{4}{Y=\Fib_F(z)}{\rA}
  \proofstep{5}{X\neq Y}{\rA}

  \proofstep{6}{y=z}{\rA}
  \proofstep{4,6}{Y=\Fib_F(y)}{%
    \FormulaRefAuto{x = y \dsep a = F(y) \vdash a = F(x)}{6,4}}

  \proofstep{3,4,6}{X=Y}{%
    \FormulaRefAuto{a = b,\, c = b \vdash a = c}{3,7}}

  \proofstep{3,4,5,6}{\bot}{\rBI{5,8}}
  \proofstep{1,2,3,4,5}{y\neq z}{\rCI{6,9}}

  \proofstep{1,2,3,4,5}{\Fib_F(y)\cap \Fib_F(z)=\varnothing}{%
    \FormulaRefAuto{y\in B \dsep z\in B \dsep y\neq z \vdash \Fib_F(y)\cap \Fib_F(z)=\varnothing}{1,2,10}}

  \proofstep{1,2,3,4,5}{X\cap Y=\varnothing}{%
    \FormulaRefAuto{X=U \dsep Y=V \dsep U\cap V=W \vdash X\cap Y=W}{3,4,11}}
\end{tabproof}



\FormulaThmDelta{%
  X\in \Fib_F[B] \dsep Y\in \Fib_F[B] \dsep X\neq Y \vdash X\cap Y=\varnothing%
}{
  \DeltaRow{Mengen}{A\dsep B\dsep X\dsep Y\dsep y\dsep z}
  \DeltaPrem{Funktionen}{F\colon A\to B}
}
\begin{tabproof}
  \proofstep{1}{X\in \Fib_F[B]}{\rA}
  \proofstep{2}{Y\in \Fib_F[B]}{\rA}
  \proofstep{3}{X\neq Y}{\rA}

  \proofstep{1}{\exists y\in B\,X=\Fib_F(y)}{%
    \FormulaRefAuto{X\in F[A] \eqvdash \exists x\in A\,X=F(x)}{1}}

  \proofstep{2}{\exists z\in B\,Y=\Fib_F(z)}{%
    \FormulaRefAuto{X\in F[A] \eqvdash \exists x\in A\,X=F(x)}{2}}

  \proofstep{6}{y\in B \land X=\Fib_F(y)}{\rA}
  \proofstep{6}{y\in B}{\rAEa{6}}
  \proofstep{6}{X=\Fib_F(y)}{\rAEb{6}}

  \proofstep{10}{z\in B \land Y=\Fib_F(z)}{\rA}
  \proofstep{10}{z\in B}{\rAEa{10}}
  \proofstep{10}{Y=\Fib_F(z)}{\rAEb{10}}

  \proofstep{3,6,10}{X\cap Y=\varnothing}{%
    \FormulaRefAuto{y\in B \dsep z\in B \dsep X=\Fib_F(y) \dsep Y=\Fib_F(z) \dsep X\neq Y \vdash X\cap Y=\varnothing}{7,11,8,12,3}}

  \proofstep{3,6}{X\cap Y=\varnothing}{\rEE{5,10,13}}
  \proofstep{3}{X\cap Y=\varnothing}{\rEE{4,6,14}}
\end{tabproof}

\subsubsection{Fasern surjektiver Abbildungen}

\FormulaThmDeltaR{y\in B \vdash \Fib_F(y)\neq\varnothing}
{F\colon A\sur B \dsep y\in B \vdash \Fib_F(y)\neq\varnothing}
{
  \DeltaRow{Mengen}{A\dsep B\dsep x\dsep y}
  \DeltaRow{Surjektive Funktion}{F\colon A\sur B}
}
\begin{tabproof}
  \proofstep{1}{y\in B}{\rA}
  \proofstep{1}{\exists x\in A\,F(x)=y}{\FormulaRefAuto{Surjektivität}{1}}
  \proofstep{3}{x\in A \land F(x)=y}{\rA}
  \proofstep{3}{x\in A}{\rAEa{3}}
  \proofstep{3}{F(x)=y}{\rAEb{3}}
  \proofstep{1,3}{x\in \Fib_F(y)}{%
    \FormulaRefAuto{y\in B\dsep x\in A\dsep F(x)=y \vdash x\in\Fib_F(y)}{1,4,5}}
  \proofstep{1,3}{\Fib_F(y)\neq\varnothing}{\FormulaRefAuto{a\in A\vdash A\neq\varnothing}{6}}
  \proofstep{1}{\Fib_F(y)\neq\varnothing}{\rEE{2,3,7}}
\end{tabproof}

\FormulaThmDelta
{X\in \Fib_F[B] \vdash X\neq\varnothing}
{
  \DeltaRow{Mengen}{A\dsep B\dsep X}
  \DeltaRow{Surjektive Funktion}{F\colon A\sur B}
}
\begin{tabproof}
  \proofstep{1}{X\in \Fib_F[B]}{\rA}

  \proofstep{1}{\exists y\in B\,X=\Fib_F(y)}{%
    \FormulaRefAuto{X\in F[A] \eqvdash \exists x\in A\,X=F(x)}{1}}

  \proofstep{3}{y\in B \land X=\Fib_F(y)}{\rA}
  \proofstep{3}{y\in B}{\rAEa{3}}
  \proofstep{3}{X=\Fib_F(y)}{\rAEb{3}}

  \proofstep{3}{\Fib_F(y)\neq\varnothing}{%
    \FormulaRefAuto{F\colon A\sur B \dsep y\in B \vdash \Fib_F(y)\neq\varnothing}{4}}

  \proofstep{3}{X\neq\varnothing}{\rIE{5,6}}
  \proofstep{1}{X\neq\varnothing}{\rEE{2,3,7}}
\end{tabproof}

\FormulaThmDelta
{\DisjFam(\Fib_F[B])}
{
  \DeltaRow{Mengen}{A\dsep B\dsep X\dsep Y}
  \DeltaRow{Surjektive Funktion}{F\colon A\sur B}
}
\begin{tabproof}
  \proofstep{}{X\in \Fib_F[B] \vdash X\neq\varnothing}{%
    \FormulaRefAuto{X\in \Fib_F[B] \vdash X\neq\varnothing}}

  \proofstep{}{X\in \Fib_F[B]\dsep Y\in \Fib_F[B]\dsep X\neq Y \vdash X\cap Y=\varnothing}{%
    \FormulaRefAuto{X\in \Fib_F[B] \dsep Y\in \Fib_F[B] \dsep X\neq Y \vdash X\cap Y=\varnothing}}

  \proofstep{}{\DisjFam(\Fib_F[B])}{%
    \FormulaRefAuto{Disjunkte Familie}{1,2}}
\end{tabproof}



\subsection{Die Einschränkung einer Funktion}

\subsubsection{Definition der eingeschränkten Funktionsvorschrift}

\FormulaDefDelta[Funktionsvorschrift für die Einschränkung von \(F\) auf \(C\)]
{C\subseteq A \dsep x\in C \vdash t_{F\restriction_C}(x) \coloneqq F(x)}
{
  \DeltaRow{Mengen}{A\dsep B\dsep C\dsep x}
  \DeltaPrem{Funktionen}{F\colon A\to B}
  \DeltaRow{Funktionensymbol}{t_{F\restriction_C}}
}

% — Typisierungsaxiom / Wertebereich —
\FormulaThmDelta[Typisierungsaxiom für \(t_{F\restriction_C}\)]
{C\subseteq A \dsep x\in C \vdash t_{F\restriction_C}(x)\in B}
{
  \DeltaRow{Mengen}{A\dsep B\dsep C\dsep x}
  \DeltaPrem{Funktionen}{F\colon A\to B}
  \DeltaRow{Funktionensymbol}{t_{F\restriction_C}}
}
\begin{tabproof}
  \proofstep{1}{C\subseteq A}{\rA}
  \proofstep{2}{x\in C}{\rA}
  % aus C⊆A und x∈C folgt x∈A
  \proofstep{1,2}{x\in A}{\FormulaRefAuto{A\subseteq B, x\in A \vdash x\in B}{1,2}}
  % Typisierungsaxiom für F selbst: x∈A ⇒ F(x)∈ B
  \proofstep{3}{F(x)\in B}{\FormulaRefAuto{x\in A\vdash F(x)\in B}{3}}
  % Definition der Funktionsvorschrift der Einschränkung
  \proofstep{1,2,3}{t_{F\restriction_C}(x)=F(x)}{%
    \FormulaRefAuto{C\subseteq A \dsep x\in C \vdash t_{F\restriction_C}(x) \coloneqq F(x)}{1,2,3}}
  % Über Gleichung auf den Wertebereich schließen
  \proofstep{1,2,3}{t_{F\restriction_C}(x)\in B}{\rIE{6,5}}
\end{tabproof}

\subsubsection{Die Einschränkung als Funktion}

\FormulaThmDelta[Einschränkung als Funktion]{%
C\subseteq A \vdash
\exists! G\colon C\to B\,\forall x\in C\,G(x) = t_{F\restriction_C}(x)%
}{
  \DeltaRow{Mengen}{A\dsep B\dsep C\dsep x}
  \DeltaPrem{Funktionen}{F\colon A\to B}
  \DeltaRow{Funktionensymbol}{t_{F\restriction_C}}
}
\begin{tabproof}
  \proofstep{1}{C\subseteq A}{\rA}

  \proofstep{1}{\forall x\in C\, t_{F\restriction_C}(x) = F(x)}{\FormulaRefAuto{C\subseteq A \dsep x\in C \vdash t_{F\restriction_C}(x) \coloneqq F(x)}{1}}

  \proofstep{1}{\forall x\in C\, t_{F\restriction_C}(x)\in B}{\FormulaRefAuto{C\subseteq A \dsep x\in C \vdash t_{F\restriction_C}(x)\in B}{1}}
  % Existenz und Eindeutigkeit einer Funktion G: C→B mit dieser Vorschrift
  \proofstep{1}{\exists! G\colon C\to B\,\forall x\in C\,G(x) = t_{F\restriction_C}(x)}{%
    \FormulaRefAuto{x\in A\vdash t(x):=y\dsep x\in A\vdash t(x)\in B\exists! F\colon A\to B\,\forall x\in A\,F(x) = t(x)}{2,3}}
\end{tabproof}

\FormulaDefDelta[Einschränkung von \(F\) auf \(C\)]
{C\subseteq A \vdash F\restriction_C\coloneqq\iota G\left(G\colon C\to B \land \forall x\in C\, G(x)=t_{F\restriction_C}(x)\right)}
{
  \DeltaRow{Mengen}{A\dsep B\dsep C}
  \DeltaPrem{Funktionen}{F\colon A\to B}
  \DeltaRow{Funktionensymbol}{t_{F\restriction_C}}
}

\FormulaThmDelta[Einschränkung von \(F\) auf \(C\)]
{C\subseteq A \vdash F\restriction_C\colon C\to B}
{
  \DeltaRow{Mengen}{A\dsep B\dsep C}
  \DeltaPrem{Funktionen}{F\colon A\to B}
}
\begin{tabproof}
  \proofstep{}{\;F\restriction_C\colon C\to B}{%
    \rAEa{\FormulaRefAuto{C\subseteq A \vdash F\restriction_C\coloneqq\iota G\left(G\colon C\to B \land \forall x\in C\, G(x)=t_{F\restriction_C}(x)\right)}}}
\end{tabproof}

\FormulaThmDelta
{C\subseteq A \vdash \TotRel{F\restriction_C,C,B}}
{
  \DeltaRow{Mengen}{A\dsep B}
}
\begin{tabproof}
    \proofstep{1}{C\subseteq A }{\rA}
    \proofstep{1}{F\restriction_C\colon C\to B}{\FormulaRefAuto{C\subseteq A \vdash F\restriction_C\colon C\to B}{1}}
    \proofstep{1}{\TotRel{F\restriction_C,C,B}}{\FormulaRefAuto{F\colon A\to B\vdash\TotRel{F,A,B}}{2}}
\end{tabproof}

\FormulaThmDelta[Grundgleichung der Einschränkung]
{C\subseteq A \dsep x\in C \vdash F\restriction_C(x)=F(x)}
{
  \DeltaRow{Mengen}{A\dsep B\dsep C\dsep x}
  \DeltaPrem{Funktionen}{F\colon A\to B}
}
\begin{tabproof}
  \proofstep{1}{C\subseteq A}{\rA}
  \proofstep{2}{x\in C}{\rA}
  \proofstep{1}{%
    \forall x\in C\, F\restriction_C(x)=t_{F\restriction_C}(x)}{%
    \rAEb{\FormulaRefAuto{C\subseteq A \vdash F\restriction_C\coloneqq\iota G\left(G\colon C\to B \land \forall x\in C\, G(x)=t_{F\restriction_C}(x)\right)}{1}}}
  \proofstep{1,2}{F\restriction_C(x)=t_{F\restriction_C}(x)}{\rRE{2,\rUE{3}}}
  \proofstep{1,2}{t_{F\restriction_C}(x)=F(x)}{%
    \FormulaRefAuto{C\subseteq A \dsep x\in C \vdash t_{F\restriction_C}(x) \coloneqq F(x)}{1,2}}
  \proofstep{1,2}{F\restriction_C(x)=F(x)}{%
    \FormulaRefAuto{a = b,\, b = c \vdash a = c}{4,5}}
\end{tabproof}

\subsubsection{Die Auswertungseigenschaft}

\FormulaThmDelta{%
C\subseteq A \dsep x\in C \dsep y\in C \dsep F(x)=F(y)\vdash F\restriction_C(x)=F\restriction_C(y)
}{
  \DeltaRow{Mengen}{A \dsep B \dsep C \dsep x \dsep y}
  \DeltaPrem{Funktionen}{F\colon A\to B}
}
\begin{tabproof}
  \proofstep{1}{C\subseteq A}{\rA}
  \proofstep{2}{x\in C}{\rA}
  \proofstep{3}{y\in C}{\rA}
  \proofstep{4}{F(x)=F(y)}{\rA}
  \proofstep{1,2}{F\restriction_C(x)=F(x)}{%
    \FormulaRefAuto{C\subseteq A \dsep x\in C \vdash F\restriction_C(x)=F(x)}{1,2}}
  \proofstep{1,3}{F\restriction_C(y)=F(y)}{%
    \FormulaRefAuto{C\subseteq A \dsep x\in C \vdash F\restriction_C(x)=F(x)}{1,3}}
  \proofstep{1,3}{F(y)=F\restriction_C(y)}{%
    \FormulaRefAuto{a=b\vdash b=a}{6}}
  \proofstep{1,3,4}{F(x)=F\restriction_C(y)}{%
    \FormulaRefAuto{a = b,\, b = c \vdash a = c}{4,7}}
  \proofstep{1,2,3,4}{F\restriction_C(x)=F\restriction_C(y)}{%
    \FormulaRefAuto{a = b,\, b = c \vdash a = c}{5,8}}
\end{tabproof}

\FormulaThmDelta{%
C\subseteq A \dsep x\in C \dsep y\in C \dsep F\restriction_C(x)=F\restriction_C(y)\vdash F(x)=F(y)
}{
  \DeltaRow{Mengen}{A \dsep B \dsep C \dsep x \dsep y}
  \DeltaPrem{Funktionen}{F\colon A\to B}
}
\begin{tabproof}
  \proofstep{1}{C\subseteq A}{\rA}
  \proofstep{2}{x\in C}{\rA}
  \proofstep{3}{y\in C}{\rA}
  \proofstep{4}{F\restriction_C(x)=F\restriction_C(y)}{\rA}
  \proofstep{1,2}{F\restriction_C(x)=F(x)}{%
    \FormulaRefAuto{C\subseteq A \dsep x\in C \vdash F\restriction_C(x)=F(x)}{1,2}}
  \proofstep{1,3}{F\restriction_C(y)=F(y)}{%
    \FormulaRefAuto{C\subseteq A \dsep x\in C \vdash F\restriction_C(x)=F(x)}{1,3}}
  \proofstep{1,2}{F(x)=F\restriction_C(x)}{%
    \FormulaRefAuto{a=b\vdash b=a}{5}}
  \proofstep{1,2,4}{F(x)=F\restriction_C(y)}{%
    \FormulaRefAuto{a = b,\, b = c \vdash a = c}{7,4}}
  \proofstep{1,2,3,4}{F(x)=F(y)}{%
    \FormulaRefAuto{a = b,\, b = c \vdash a = c}{8,6}}
\end{tabproof}

\subsubsection{Erhalt von Eigenschaften}

\FormulaThmDelta[Injektivität der Einschränkung]{%
C\subseteq A \dsep x\in C \dsep y\in C \dsep
F\restriction_C(x)=F\restriction_C(y) \vdash x=y
}{
  \DeltaRow{Mengen}{A\dsep B\dsep C\dsep x\dsep y}
  \DeltaRow{Injektive Funktionen}{F\colon A\rightarrowtail B}
  \DeltaPrem{Funktionen}{F\restriction_C\colon C\to B}[\FormulaRefAuto{C\subseteq A \vdash F\restriction_C\colon C\to B}]
}
\begin{tabproof}
  \proofstep{1}{C\subseteq A}{\rA}
  \proofstep{2}{x\in C}{\rA}
  \proofstep{3}{y\in C}{\rA}
  \proofstep{4}{F\restriction_C(x)=F\restriction_C(y)}{\rA}
  \proofstep{1,2,3,4}{F(x)=F(y)}{%
    \FormulaRefAuto{C\subseteq A \dsep x\in C \dsep y\in C \dsep F\restriction_C(x)=F\restriction_C(y)\vdash F(x)=F(y)}{1,2,3,4}}
  \proofstep{2}{x\in A}{\FormulaRefAuto{A\subseteq B, x\in A\vdash x\in B}{2}}
  \proofstep{3}{y\in A}{\FormulaRefAuto{A\subseteq B, x\in A\vdash x\in B}{3}}
  \proofstep{1,6,7,11}{x=y}{%
    \FormulaRefAuto{x\in A\dsep y\in A\dsep F(x)=F(y)\vdash x=y}{6,7,5}}
\end{tabproof}

\FormulaThmDelta[Einschränkung einer injektiven Funktion]{%
 C\subseteq A \vdash F\restriction_C\colon C\rightarrowtail B
}{
  \DeltaRow{Mengen}{A\dsep B\dsep C}
  \DeltaRow{Injektive Funktionen}{F\colon A\rightarrowtail B}
}
\begin{tabproof}
  \proofstep{1}{C\subseteq A}{\rA}
  \proofstep{1}{F\restriction_C\colon C\to B}{%
    \FormulaRefAuto{C\subseteq A \vdash F\restriction_C\colon C\to B}{1}}
  \proofstep{1,2}{\forall x,y\in C\,(F\restriction_C(x)=F\restriction_C(y)\rightarrow x=y)}{%
    \FormulaRefAuto{C\subseteq A \dsep x\in C \dsep y\in C \dsep
F\restriction_C(x)=F\restriction_C(y) \vdash x=y}{1,2}}
  \proofstep{1,2}{F\restriction_C\colon C\rightarrowtail B}{%
    \FormulaRefAuto{Injektive Funktion}{3,4}}
\end{tabproof}

\FormulaThmDelta[Surjektivität der Einschränkung auf das Bild]{%
C\subseteq A \dsep y\in F[C] \vdash \exists x\in C\, F\restriction_C(x)=y
}{
  \DeltaRow{Mengen}{A\dsep B\dsep C\dsep x\dsep y}
  \DeltaPrem{Funktionen}{F\colon A\to B}
}
\begin{tabproof}
  \proofstep{1}{C\subseteq A}{\rA}
  \proofstep{2}{y\in F[C]}{\rA}

  % Bildmengen-Theorem auf C angewandt: y∈F[C] ⇒ ∃x∈C (y=F(x))
  \proofstep{2}{\exists x\in C\, y=F(x)}{%
    \FormulaRefAuto{y\in F[A]\eqvdash \exists x\in A\,y=F(x)}{2}}
  
  \proofstep{3}{x\in C\land y=F(x)}{\rA}
  \proofstep{3}{x\in C}{\rAEa{3}}
  \proofstep{3}{y=F(x)}{\rAEb{3}}

  % Grundgleichung der Einschränkung
  \proofstep{1,3}{F\restriction_C(x)=F(x)}{%
    \FormulaRefAuto{C\subseteq A \dsep x\in C \vdash F\restriction_C(x)=F(x)}{1,4}}

  % Aus F|_C(x)=F(x) und y=F(x) folgt F|_C(x)=y
  \proofstep{3}{F(x)=y}{\FormulaRefAuto{a=b\vdash b=a}{5}}
  \proofstep{3}{F\restriction_C(x)=y}{%
    \FormulaRefAuto{a=b,\, b=c \vdash a=c}{6,7}}

  % Zeugenexistenz
  \proofstep{3}{\exists x\in C\,F\restriction_C(x)=y}{\rEI{\rAI{4,8}}}
  \proofstep{2}{\exists x\in C\,F\restriction_C(x)=y}{\rEE{3,9}}
\end{tabproof}

\FormulaThmDelta[Einschränkung als surjektive Funktion auf das Bild]{%
C\subseteq A \vdash F\restriction_C\colon C\twoheadrightarrow F[C]
}{
  \DeltaRow{Mengen}{A\dsep B\dsep C}
  \DeltaPrem{Funktionen}{F\colon A\to B}
}
\begin{tabproof}
  \proofstep{1}{C\subseteq A}{\rA}
  \proofstep{1}{F\restriction_C\colon C\to B}{%
    \FormulaRefAuto{C\subseteq A \vdash F\restriction_C\colon C\to B}{1}}
  \proofstep{1}{\forall y\in F[C]\;\exists x\in C\,F\restriction_C(x)=y}{%
    \FormulaRefAuto{C\subseteq A \dsep y\in F[C] \vdash \exists x\in C\, F\restriction_C(x)=y}{1}}
  \proofstep{1}{F\restriction_C\colon C\twoheadrightarrow F[C]}{%
    \FormulaRefAuto{Surjektive Funktion}{2,3}}
\end{tabproof}

\subsubsection{Beispiele}

\paragraph{Projektionen}

\FormulaThmDelta[Surjektivität]
{x\in A\vdash \exists y\in R(\pi_1\restriction_R(y)=x)}
{
  \DeltaRow{Mengen}{A\dsep B\dsep x}
  \DeltaPrem{Totale Relationen}{\TotRel{R,A,B}}
}
\begin{tabproofwide}
  \proofstepwidestar[1]{x\in A}{\rA}
  \proofstepwidestar[]{R\subseteq A\times B}{\FormulaRefAuto{F \subseteq A \times B}}
  \proofstepwidestar[1]{\exists x' \, (x,x')\in R}{\FormulaRefAuto{x\in A \vdash \exists y\,(x,y)\in F}{1}}
  \proofstepwidestar[4]{(x,x')\in R}{\rA}
  \proofstepwidestar[4]{(x,x')\in A\times B}{\FormulaRefAuto{A\subseteq B, x\in A\vdash x\in B}{2,4}}
  \proofstepwide[4]{\pi_1\restriction_R(x,x')}{=}{\pi_1(x,x')}{\FormulaRefAuto{C\subseteq A \dsep x\in C \vdash F\restriction_C(x)=F(x)}{2,4}}
  \proofstepwide[4]{}{=}{x}{\FormulaRefAuto{(x,y)\in A\times B \vdash \pi_1(x,y)=x}{5}}
  \proofstepwide[4]{\pi_1\restriction_R(x,x')}{=}{x}{\rChain{6,7}}
  \proofstepwidestar[4]{\exists y\in R \, \pi_1\restriction_R(y)=x}{\rEI{\rAI{4,8}}}
  \proofstepwidestar[1]{\exists y\in R \, \pi_1\restriction_R(y)=x}{\rEE{3,4,9}}
\end{tabproofwide}

\FormulaThmDelta[Surjektive Funktion]{%
\pi_1\restriction_R\colon R\sur A
}{
  \DeltaRow{Mengen}{A\dsep B}
  \DeltaPrem{Totale Relationen}{\TotRel{R,A,B}}
}
\begin{tabproof}
    \proofstep{}{\pi_1\restriction_R\colon R\to A}{\FormulaRefAuto{C\subseteq A \vdash F\restriction_C\colon C\to B}{\FormulaRefAuto{R\subseteq A\times B}}}
    \proofstep{}{\forall x\in A \exists y\in R\,(\pi_1\restriction_R(y)=x)}{\FormulaRefAuto{x\in A\vdash \exists y\in R(\pi_1\restriction_R(y)=x)}}
    \proofstep{}{\pi_1\restriction_R\colon R\sur A}{\FormulaRefAuto{Surjektive Funktion}{1,2}}
\end{tabproof}

\subsection{Die Umkehrfunktion}

\subsubsection{Definition der Umkehrfunktionsvorschrift}

\FormulaDefDelta[Funktionsvorschrift für die Umkehrfunktion von \(F\)]
{y\in B \vdash t_{F^{-1}}(y) \coloneqq \iota x\bigl(x\in A \land F(x)=y\bigr)}
{
  \DeltaRow{Mengen}{A\dsep B\dsep x\dsep y}
  \DeltaPrem{Funktionen}{F\colon A\bij B}
  \DeltaRow{Funktionensymbol}{t_{F^{-1}}}
}

% — Typisierungsaxiom / Wertebereich —
\FormulaThmDelta[Typisierungsaxiom für \(t_{F^{-1}}\)]
{y\in B \vdash t_{F^{-1}}(y)\in A}
{
  \DeltaRow{Mengen}{A\dsep B\dsep x\dsep y}
  \DeltaPrem{Funktionen}{F\colon A\bij B}
  \DeltaRow{Funktionensymbol}{t_{F^{-1}}}
}
\begin{tabproof}
  \proofstep{1}{y\in B}{\rA}
  \proofstep{1}{t_{F^{-1}}(y)\in A}{\rAEa{\FormulaRefAuto{y\in B \vdash t_{F^{-1}}(y) \coloneqq \iota x\bigl(x\in A \land F(x)=y\bigr)}{1}}}
\end{tabproof}

\FormulaThmDelta
{y\in B \vdash F(t_{F^{-1}}(y))=y}
{
  \DeltaRow{Mengen}{A\dsep B\dsep x\dsep y}
  \DeltaPrem{Funktionen}{F\colon A\bij B}
  \DeltaRow{Funktionensymbol}{t_{F^{-1}}}
}
\begin{tabproof}
  \proofstep{1}{y\in B}{\rA}
  \proofstep{1}{F(t_{F^{-1}}(y))=y}{\rAEb{\FormulaRefAuto{y\in B \vdash t_{F^{-1}}(y) \coloneqq \iota x\bigl(x\in A \land F(x)=y\bigr)}{1}}}
\end{tabproof}

\subsubsection{Die Umkehrfunktion als Funktion}

\FormulaThmDelta[Umkehrfunktion als Funktion]{%
\exists! G\colon B\to A\,\forall y\in B\,G(y) = t_{F^{-1}}(y)%
}{
  \DeltaRow{Mengen}{A\dsep B\dsep x\dsep y}
  \DeltaPrem{Funktionen}{F\colon A\bij B}
  \DeltaRow{Funktionensymbol}{t_{F^{-1}}}
}
\begin{tabproof}
  \proofstep{}{\forall y\in B\,t_{F^{-1}}(y)=\iota x\bigl(x\in A\land F(x)=y\bigr)}{\FormulaRefAuto{y\in B \vdash t_{F^{-1}}(y) \coloneqq \iota x\bigl(x\in A \land F(x)=y\bigr)}}
  \proofstep{}{\forall y\in B\,t_{F^{-1}}(y)\in A}{\FormulaRefAuto{y\in B \vdash t_{F^{-1}}(y)\in A}}
  \proofstep{}{\exists! G\colon B\to A\,\forall y\in B\,G(y) = t_{F^{-1}}(y)}{\FormulaRefAuto{x\in A\vdash t(x):=y\dsep x\in A\vdash t(x)\in B\exists! F\colon A\to B\,\forall x\in A\,F(x) = t(x)}{1,2}}
\end{tabproof}

\FormulaDefDelta[Umkehrfunktion von \(F\)]
{F^{-1}\coloneqq\iota G\left(G\colon B\to A \land 
  \forall y\in B\, G(y)=t_{F^{-1}}(y)\right)}
{
  \DeltaRow{Mengen}{A\dsep B}
  \DeltaPrem{Funktionen}{F\colon A\bij B}
  \DeltaRow{Funktionensymbol}{t_{F^{-1}}}
}

\FormulaThmDeltaR[Umkehrfunktion von \(F\)]
{F^{-1}\colon B\to A}{F\colon A\bij B\vdash F^{-1}\colon B\to A}
{
  \DeltaRow{Mengen}{A\dsep B}
  \DeltaPrem{Funktionen}{F\colon A\bij B}
}
\begin{tabproof}
  \proofstep{}{F^{-1}\colon B\to A}{%
    \rAEa{\FormulaRefAuto{\exists! G\colon B\to A\,\forall y\in B\,G(y) = t_{F^{-1}}(y)}}}
\end{tabproof}

\FormulaThmDelta
{\TotRel{F^{-1},B,A)}}
{
  \DeltaRow{Mengen}{A\dsep B}
  \DeltaPrem{Funktionen}{F\colon A\bij B}
}
\begin{tabproof}
    \proofstep{}{F^{-1}\colon B\to A}{\FormulaRefAuto{F\colon A\bij B\vdash F^{-1}\colon B\to A}}
    \proofstep{1}{\TotRel{F^{-1},B,A)}}{\FormulaRefAuto{F\colon A\to B\vdash\TotRel{F,A,B}}{1}}
\end{tabproof}

\FormulaThmDelta[Umkehrfunktion von \(F\)]
{y\in B\vdash F^{-1}(y)=t_{F^{-1}}(y)}
{
  \DeltaRow{Mengen}{A\dsep B}
  \DeltaPrem{Funktionen}{F\colon A\bij B}
}
\begin{tabproof}
  \proofstep{1}{y\in B}{\rA}
  \proofstep{1}{F^{-1}(y)=t_{F^{-1}}(y)}{%
    \rRE{1,\rAEb{\FormulaRefAuto{\exists! G\colon B\to A\,\forall y\in B\,G(y) = t_{F^{-1}}(y)}}}}
\end{tabproof}

\FormulaThmDelta[Typisierungsaxiom für \(t_{F^{-1}}\)]
{y\in B \vdash F^{-1}(y)\in A}
{
  \DeltaRow{Mengen}{A\dsep B\dsep x\dsep y}
  \DeltaPrem{Funktionen}{F\colon A\bij B}
}

\subsubsection{Grundgleichungen der Umkehrfunktion}

\FormulaThmDelta[Links-Umkehrung]
{x\in A \vdash F^{-1}(F(x)) = x}
{
  \DeltaRow{Mengen}{A\dsep B\dsep x}
  \DeltaPrem{Funktionen}{F\colon A\bij B}
}
\begin{tabproof}
  \proofstep{1}{x\in A}{\rA}
  \proofstep{1}{F(x)\in B}{%
    \FormulaRefAuto{x\in A\vdash F(x)\in B}{1}}
  \proofstep{1}{t_{F^{-1}}(F(x))\in A}{%
    \FormulaRefAuto{y\in B \vdash t_{F^{-1}}(y)\in A}{1}}
  \proofstep{1}{F(t_{F^{-1}}(F(x))) = F(x)}{\FormulaRefAuto{y\in B \vdash F(t_{F^{-1}}(y))=y}{2}}
  \proofstep{1}{F^{-1}(F(x))=t_{F^{-1}}(F(x))}{\FormulaRefAuto{y\in B\vdash F^{-1}(y)=t_{F^{-1}}(y)}{2}}
  \proofstep{1}{t_{F^{-1}}(F(x))=x}{\FormulaRefAuto{x\in A\dsep y\in A\dsep F(x)=F(y)\vdash x=y}{3,2,4}}
  \proofstep{1}{F^{-1}(F(x))=x}{\FormulaRefAuto{a=b,b=c\vdash a=c}{5,6}}
\end{tabproof}

\FormulaThmDeltaR[Rechts-Umkehrung]
{y\in B \vdash F(F^{-1}(y)) = y}{F\colon A\bij B \dsep y\in B \vdash F(F^{-1}(y)) = y}
{
  \DeltaRow{Mengen}{A\dsep B\dsep y}
  \DeltaPrem{Funktionen}{F\colon A\bij B}
}
\begin{tabproof}
  \proofstep{1}{y\in B}{\rA}
  \proofstep{1}{t_{F^{-1}}(y)=F^{-1}(y)}{%
    \FormulaRefAuto{a=b\vdash b=a}{\FormulaRefAuto{y\in B\vdash F^{-1}(y)=t_{F^{-1}}(y)}{1}}
  }
  \proofstep{1}{F(t_{F^{-1}}(y))=y}{%
    \FormulaRefAuto{y\in B \vdash F(t_{F^{-1}}(y))=y}{2}}
  \proofstep{1}{F(F^{-1}(y))=y}{%
    \rIE{2,3}}
\end{tabproof}

\subsubsection{Die Bijektionseigenschaften der Umkehrfunktion}

\FormulaThmDelta[Injektivität von \(F^{-1}\)]%
{x\in B\dsep y\in B\dsep F^{-1}(x)=F^{-1}(y)\vdash x=y}%
{
  \DeltaRow{Mengen}{x\dsep y\dsep A\dsep B}
  \DeltaPrem{Bijektive Funktionen}{F\colon A\bij B}
}
\begin{tabproof}
  \proofstep{1}{x\in B}{\rA}
  \proofstep{2}{y\in B}{\rA}
  \proofstep{3}{F^{-1}(x)=F^{-1}(y)}{\rA}
  \proofstep{}{F^{-1}\colon B\to A}{\FormulaRefAuto{F\colon A\bij B\vdash F^{-1}\colon B\to A}}
  \proofstep{1}{F^{-1}(x)\in A}{\FormulaRefAuto{x\in A\vdash F(x)\in B}{1,4}}
  \proofstep{2}{F^{-1}(y)\in A}{\FormulaRefAuto{x\in A\vdash F(x)\in B}{2,4}}
  \proofstep{1}{F(F^{-1}(x))=x}{%
    \FormulaRefAuto{F\colon A\bij B \dsep y\in B \vdash F(F^{-1}(y)) = y}{1}}
  \proofstep{2}{F(F^{-1}(y))=y}{%
    \FormulaRefAuto{F\colon A\bij B \dsep y\in B \vdash F(F^{-1}(y)) = y}{2}}
  \proofstep{1,2,3}{F(F^{-1}(x))=F(F^{-1}(y))}{%
    \FormulaRefAuto{x\in A\dsep y\in A\dsep x=y\vdash F(x)=F(y)}{3}}
  \proofstep{1,2,3}{x=y}{%
    \FormulaRefAuto{a = b\dsep a = c\dsep b=d\vdash c = d}{9,7,8}}
\end{tabproof}

\FormulaThmDelta[Surjektivität von \(F^{-1}\)]%
{x\in A\vdash \exists y\in B\; F^{-1}(y)=x}%
{
  \DeltaRow{Mengen}{x\dsep y\dsep A\dsep B}
  \DeltaPrem{Bijektive Funktionen}{F\colon A\bij B}
}
\begin{tabproof}
  \proofstep{1}{x\in A}{\rA}
  % Typisierung von F:
  \proofstep{1}{F(x)\in B}{%
    \FormulaRefAuto{x\in A\vdash F(x)\in B}{1}}
  % Links-Umkehrung: F^{-1}(F(x))=x
  \proofstep{1}{F^{-1}(F(x))=x}{%
    \FormulaRefAuto{x\in A \vdash F^{-1}(F(x)) = x}{1}}
  % Zeugenbildung y := F(x)
  \proofstep{1}{\exists y\in B\; F^{-1}(y)=x}{%
    \rEI{\rAI{2,3}}}
\end{tabproof}

\FormulaThmDeltaR[\(F^{-1}\) als bijektive Funktion]%
{F^{-1}\colon B\bij A}{F\colon A\bij B\vdash F^{-1}\colon B\bij A}%
{
  \DeltaRow{Mengen}{A\dsep B}
  \DeltaPrem{Bijektive Funktionen}{F\colon A\bij B}
}
\begin{tabproof}
  \proofstep{}{F^{-1}\colon B\to A}{\FormulaRefAuto{F\colon A\bij B\vdash F^{-1}\colon B\to A}}
  \proofstep{}{\forall x,y\in B\, (F^{-1}(x)=F^{-1}(y)\rightarrow x=y)}{%
    \FormulaRefAuto{x\in B\dsep y\in B\dsep F^{-1}(x)=F^{-1}(y)\vdash x=y}}
  \proofstep{}{\forall x\in A\exists y\in B\, F^{-1}(y)=x}{%
    \FormulaRefAuto{x\in A\vdash \exists y\in B\; F^{-1}(y)=x}}
  \proofstep{}{F^{-1}\colon B\bij A}{%
    \FormulaRefAuto{Bijektive Funktion}{1,2,3}}
\end{tabproof}


\FormulaThmDelta[Eigeninversität der Identität]{%
\Id_A^{-1}=\Id_A
}{
  \DeltaRow{Mengen}{A}
}
\begin{tabproof}
  \proofstep{}{%
    \Id_A\colon A\bij A}{%
    \FormulaRefAuto{\Id_A\colon A\bij A}}

  \proofstep{2}{%
    x\in A}{%
    \rA}

  \proofstep{2}{%
    \Id_A^{-1}(\Id_A(x)) = x}{%
    \FormulaRefAuto{x\in A \vdash F^{-1}(F(x)) = x}{1,2}}

  \proofstep{2}{%
    \Id_A(x)=x}{%
    \FormulaRefAuto{x\in A\vdash \Id_A(x)=x}{2}}

  \proofstep{2}{%
    \Id_A^{-1}(x)=x}{%
    \rIE{4,3}}

  \proofstep{2}{%
    \Id_A(x)=\Id_A(x)}{%
    \rII}

  \proofstep{2}{%
    x=\Id_A(x)}{%
    \rIE{4,6}}

  \proofstep{2}{%
    \Id_A^{-1}(x)=\Id_A(x)}{%
    \rIE{7,5}}

  \proofstep{}{%
    x\in A \rightarrow \Id_A^{-1}(x)=\Id_A(x)}{%
    \rRI{2,8}}

  \proofstep{}{%
    \forall x\in A\, \Id_A^{-1}(x)=\Id_A(x)}{%
    \rUI{9}}

  \proofstep{}{%
    \forall x\in A\,(\Id_A^{-1}(x)=\Id_A(x)) \leftrightarrow \Id_A^{-1}=\Id_A}{%
    \FormulaRefAuto{\forall x\in A (F(x)=G(x)) \eqvdash F=G}}

  \proofstep{}{%
    \Id_A^{-1}=\Id_A}{%
    \FormulaRefAuto{P\leftrightarrow Q, P\vdash Q}{11,10}}
\end{tabproof}


\subsubsection{Weitere Eigenschaften}

\FormulaThmDelta{%
\forall y\in B\,\bigl(F(G(y))=y\bigr) \vdash G = F^{-1}
}{
  \DeltaRow{Mengen}{x \dsep y \dsep A \dsep B}
  \DeltaPrem{Bijektive Funktionen}{F\colon A\bij B}
  \DeltaPrem{Funktionen}{G\colon B\to A}
}
\begin{tabproof}
  \proofstep{1}{\forall y\in B\,\bigl(F(G(y))=y\bigr)}{\rA}
  \proofstep{2}{y\in B}{\rA}
  \proofstep{1,2}{F(G(y))=y}{\rRE{\rUE{1},2}}
  \proofstep{1,2}{F(F^{-1}(y))=y}{\FormulaRefAuto{F\colon A\bij B \dsep y\in B \vdash F(F^{-1}(y)) = y}{2}}
  \proofstep{2}{G(y)\in A}{\FormulaRefAuto{x\in A\vdash F(x)\in B}{2}}
  \proofstep{2}{F^{-1}(y)\in A}{\FormulaRefAuto{x\in A\vdash F(x)\in B}{2}}
  \proofstep{1,2}{G(y)=F^{-1}(y)}{\FormulaRefAuto{x\in A\dsep y\in A\dsep F(x)=F(y)\vdash x=y}{5,6,3,4}}
  \proofstep{1}{\forall y\,\Bigl(G(y)=F^{-1}(y)\Bigr)}{\rUI{\rRI{2,7}}}
  \proofstep{1}{G=F^{-1}}{\FormulaRefAuto{\forall x\in A (F(x)=G(x)) \eqvdash F=G}{8}}
\end{tabproof}

\FormulaThmDelta{%
\forall x\in A\,\bigl(F^{-1}(G(x))=x\bigr) \vdash G = F
}{
  \DeltaRow{Mengen}{x \dsep A \dsep B}
  \DeltaPrem{Bijektive Funktionen}{F\colon A\to B}
  \DeltaPrem{Funktionen}{G\colon A\to B}
}
\begin{tabproof}
  \proofstep{1}{\forall x\in A\,\bigl(F^{-1}(G(x))=x\bigr)}{\rA}
  \proofstep{2}{x\in A}{\rA}
  \proofstep{1,2}{F^{-1}(G(x))=x}{\rRE{\rUE{1},2}}
  \proofstep{2}{F^{-1}\bigl(F(x)\bigr)=x}{\FormulaRefAuto{x\in A \vdash F^{-1}(F(x)) = x}{2}}
  \proofstep{2}{G(x)\in B}{\FormulaRefAuto{x\in A \vdash F(x)\in B}{2}}
  \proofstep{2}{F(x)\in B}{\FormulaRefAuto{x\in A \vdash F(x)\in B}{2}}
  \proofstep{1,2}{G(y)=F(y)}{\FormulaRefAuto{x\in A\dsep y\in A\dsep F(x)=F(y)\vdash x=y}{5,6,3,4}}
  \proofstep{1}{\forall y\,\Bigl(G(y)=F(y)\Bigr)}{\rUI{\rRI{2,7}}}
  \proofstep{1}{G=F}{\FormulaRefAuto{\forall x\in A (F(x)=G(x)) \eqvdash F=G}{8}}
\end{tabproof}

\subsection{Die Komposition von Funktionen}

\subsubsection{Funktionsvorschrift und Typisierung}

\FormulaDefDelta[Funktionsvorschrift für die Komposition \(G\circ F\)]
{x\in A \vdash t_{G\circ F}(x) \coloneqq G(F(x))}
{
  \DeltaRow{Mengen}{A\dsep B\dsep C\dsep x}
  \DeltaPrem{Funktionen}{F\colon A\to B\dsep G\colon B\to C}
  \DeltaRow{Funktionensymbol}{t_{G\circ F}}
}

% — Typisierungsaxiom / Wertebereich —
\FormulaThmDelta[Typisierungsaxiom für \(t_{G\circ F}\)]
{x\in A \vdash t_{G\circ F}(x)\in C}
{
  \DeltaRow{Mengen}{A\dsep B\dsep C\dsep x}
  \DeltaPrem{Funktionen}{F\colon A\to B\dsep G\colon B\to C}
  \DeltaRow{Funktionensymbol}{t_{G\circ F}}
}
\begin{tabproof}
  \proofstep{1}{x\in A}{\rA}
  % Typ von F
  \proofstep{1}{F(x)\in B}{%
    \FormulaRefAuto{x\in A\vdash F(x)\in B}{1}}
  % Typ von G
  \proofstep{1}{G(F(x))\in C}{%
    \FormulaRefAuto{x\in A\vdash F(x)\in B}{2}}
  % Definition der Vorschrift
  \proofstep{1}{t_{G\circ F}(x)=G(F(x))}{%
    \FormulaRefAuto{x\in A \vdash t_{G\circ F}(x) \coloneqq G(F(x))}{1}}
  % Wertebereich per Gleichheitselimination
  \proofstep{1}{t_{G\circ F}(x)\in C}{\rIE{4,3}}
\end{tabproof}

\subsubsection{Die Komposition als Funktion}

\FormulaThmDelta[Komposition als Funktion]{%
\exists! H\colon A\to C\,\forall x\in A\,H(x) = t_{G\circ F}(x)%
}{
  \DeltaRow{Mengen}{A\dsep B\dsep C\dsep x}
  \DeltaPrem{Funktionen}{F\colon A\to B\dsep G\colon B\to C}
  \DeltaRow{Funktionensymbol}{t_{G\circ F}}
}
\begin{tabproof}
  % Vorschrift global
  \proofstep{}{\forall x\in A\,t_{G\circ F}(x)=G(F(x))}{%
    \FormulaRefAuto{x\in A \vdash t_{G\circ F}(x) \coloneqq G(F(x))}}
  % Typ global
  \proofstep{}{\forall x\in A\,t_{G\circ F}(x)\in C}{%
    \FormulaRefAuto{x\in A \vdash t_{G\circ F}(x)\in C}}
  % Existenz und Eindeutigkeit der Funktion H: A→C mit dieser Vorschrift
  \proofstep{}{\exists! H\colon A\to C\,\forall x\in A\,H(x) = t_{G\circ F}(x)}{%
    \FormulaRefAuto{x\in A\vdash t(x):=y\dsep x\in A\vdash t(x)\in B\exists! F\colon A\to B\,\forall x\in A\,F(x) = t(x)}{1,2}}
\end{tabproof}

\subsubsection{Definition und Grundgleichung der Komposition}

\FormulaDefDelta[Komposition $G\circ F$]{%
  G\circ F \coloneqq
  \iota H\Bigl(H\colon A\to C \land \forall x\in A\, H(x)=t_{G\circ F}(x)\Bigr)%
}{
  \DeltaRow{Mengen}{A\dsep B\dsep C\dsep x}
  \DeltaPrem{Funktionen}{F\colon A\to B\dsep G\colon B\to C}
  \DeltaRow{Funktionensymbol}{t_{G\circ F}}
}

\FormulaThmDelta[Komposition als Abbildung]{%
  G\circ F\colon A\to C%
}{
  \DeltaRow{Mengen}{A\dsep B\dsep C}
  \DeltaPrem{Funktionen}{F\colon A\to B\dsep G\colon B\to C}
}
\begin{tabproof}
  \proofstep{}{G\circ F\colon A\to C}{%
    \rAEa{\FormulaRefAuto{G\circ F\coloneqq\iota H\left(H\colon A\to C \land \forall x\in A\, H(x)=t_{G\circ F}(x)\right)}}}
\end{tabproof}




% ------------------------------------------------------------
% (1) Grundgleichung der Komposition (vertauschte Namen)
% ------------------------------------------------------------
\FormulaThmDelta[Grundgleichung der Komposition (für $F\circ G$)]%
{x\in A \vdash (F\circ G)(x)=F(G(x))}%
{
  \DeltaRow{Mengen}{A\dsep B\dsep C\dsep x}
  \DeltaPrem{Funktionen}{G\colon A\to B\dsep F\colon B\to C}
}
\begin{tabproofwide}
  \proofstepwidestar[1]{x\in A}{\rA}

  \proofstepwidestar[]{%
    \forall x\in A\, (F\circ G)(x)=t_{F\circ G}(x)}{%
    \rAEb{\FormulaRefAuto{G\circ F \coloneqq
  \iota H\Bigl(H\colon A\to C \land \forall x\in A\, H(x)=t_{G\circ F}(x)\Bigr)}}}

  \proofstepwide[1]{(F\circ G)(x)}{=}{t_{F\circ G}(x)}{\rRE{1,\rUE{2}}}

  \proofstepwide[1]{t_{F\circ G}(x)}{=}{F(G(x))}{%
    \FormulaRefAuto{x\in A \vdash t_{G\circ F}(x) \coloneqq G(F(x))}{1}}

  \proofstepwide[1]{(F\circ G)(x)}{=}{F(G(x))}{%
    \FormulaRefAuto{a=b,\, b=c \vdash a=c}{3,4}}
\end{tabproofwide}


% ------------------------------------------------------------
% (2) Rückwärts-Variante (saubere Gleichungskette / Symmetrie)
% ------------------------------------------------------------
\FormulaThmDelta%
{x\in A \vdash F(G(x))=(F\circ G)(x)}%
{
  \DeltaRow{Mengen}{A\dsep B\dsep C\dsep x}
  \DeltaPrem{Funktionen}{G\colon A\to B\dsep F\colon B\to C}
}
\begin{tabproof}
  \proofstep{1}{x\in A}{\rA}
  \proofstep{1}{(F\circ G)(x)=F(G(x))}{%
    \FormulaRefAuto{x\in A \vdash (F\circ G)(x)=F(G(x))}{1}}
  \proofstep{1}{F(G(x))=(F\circ G)(x)}{%
    \FormulaRefAuto{a=b\vdash b=a}{2}}
\end{tabproof}

\subsubsection{Die Auswertungseigenschaft}

\FormulaThmDelta{%
x\in A\dsep y\in A\dsep G(F(x))=G(F(y))\vdash (G\circ F)(x)=(G\circ F)(y)
}{
  \DeltaRow{Mengen}{x \dsep A \dsep B \dsep C}
  \DeltaPrem{Funktionen}{F\colon A\to B \dsep G\colon B\to C}
}
\begin{tabproof}
\proofstep{1}{x\in A}{\rA}
\proofstep{2}{y\in A}{\rA}
\proofstep{3}{G(F(x))=G(F(y))}{\rA}
\proofstep{1}{(G\circ F)(x)=G(F(x))}{\FormulaRefAuto{x\in A \vdash (G\circ F)(x)=G(F(x))}{1}}
\proofstep{2}{(G\circ F)(y)=G(F(y))}{\FormulaRefAuto{x\in A \vdash (G\circ F)(x)=G(F(x))}{2}}
\proofstep{3}{(G\circ F)(x)=G(F(y))}{\rIE{4,3}}
\proofstep{3}{(G\circ F)(x)=(G\circ F)(y)}{\rIE{5,6}}
\end{tabproof}

\FormulaThmDelta{%
x\in A\dsep y\in A\dsep (G\circ F)(x)=(G\circ F)(y) \vdash G(F(x))=G(F(y))
}{
  \DeltaRow{Mengen}{x \dsep A \dsep B \dsep C}
  \DeltaPrem{Funktionen}{F\colon A\to B \dsep G\colon B\to C}
}
\begin{tabproof}
  \proofstep{1}{x\in A}{\rA}
  \proofstep{2}{y\in A}{\rA}
  \proofstep{3}{(G\circ F)(x)=(G\circ F)(y)}{\rA}

  \proofstep{1}{(G\circ F)(x)=G(F(x))}{\FormulaRefAuto{x\in A \vdash (G\circ F)(x)=G(F(x))}{1}}
  \proofstep{2}{(G\circ F)(y)=G(F(y))}{\FormulaRefAuto{x\in A \vdash (G\circ F)(x)=G(F(x))}{2}}

  \proofstep{3}{G(F(x))=(G\circ F)(y)}{\rIE{4,3}}
  \proofstep{3}{G(F(x))=G(F(y))}{\rIE{5,6}}
\end{tabproof}

\subsubsection{Erhalt der Injektivität}

% — Erhalt von Injektivität / Surjektivität / Bijektivität —
\FormulaThmDelta[Erhalt der Injektivität]{%
x\in A \dsep y\in A \dsep (G\circ F)(x)=(G\circ F)(y) \vdash x=y
}{
  \DeltaRow{Mengen}{x \dsep y \dsep A \dsep B \dsep C}
  \DeltaRow{Injektive Funktionen}{F\colon A\inj B \dsep G\colon B\inj C}
}
\begin{tabproof}
  \proofstep{1}{(G\circ F)(x)=(G\circ F)(y)}{\rA}
  \proofstep{2}{x\in A}{\rA}
  \proofstep{3}{y\in A}{\rA}
  \proofstep{2}{F(x)\in B}{\FormulaRefAuto{x\in A\vdash F(x)\in B}{2}}
  \proofstep{3}{F(y)\in B}{\FormulaRefAuto{x\in A\vdash F(x)\in B}{3}}
  \proofstep{1}{G(F(x))=G(F(y))}{\rIE{\FormulaRefAuto{x\in A \vdash (G\circ F)(x)=G(F(x))},1}}
  \proofstep{1,2,3}{F(x)=F(y)}{\FormulaRefAuto{Injektivität}{4,5,6}}
  \proofstep{1,2,3}{x=y}{\FormulaRefAuto{Injektivität}{2,3,7}}
\end{tabproof}

\FormulaThmDelta[Die Komposition als injektive Funktion]{%
G\circ F\colon A\inj C
}{
  \DeltaRow{Mengen}{A\dsep B\dsep C}
  \DeltaRow{Injektive Funktionen}{F\colon A\to B \dsep G\colon B\to C}
}
\begin{tabproof}
    \proofstep{}{G\circ F\colon A\to C}{\FormulaRefAuto{G\circ F\colon A\to C}}
    \proofstep{}{\forall x,y\in A\,((G\circ F)(x)=(G\circ F)(y) \rightarrow x=y)}{\FormulaRefAuto{x\in A \dsep y\in A \dsep (G\circ F)(x)=(G\circ F)(y) \vdash x=y}}
    \proofstep{}{G\circ F\colon A\inj C}{\FormulaRefAuto{Injektive Funktion}{1,2}}
\end{tabproof}

\subsubsection{Erhalt der Surjektivität}

\FormulaThmDelta[Erhalt der Surjektivität]{%
z\in C \vdash \exists x\in A\, (G\circ F)(x)=z
}{
  \DeltaRow{Mengen}{x \dsep y \dsep z \dsep A \dsep B \dsep C}
  \DeltaPrem{Surjektive Funktionen}{F\colon A\sur B \dsep G\colon B\sur C}
}
\begin{tabproof}
  \proofstep{1}{z\in C}{\rA}
  \proofstep{1}{\exists y\in B\, G(y)=z}{\FormulaRefAuto{Surjektivität}{1}} % (Schema: Surj. von G)
  \proofstep{3}{y\in B \land G(y)=z}{\rA}
  \proofstep{3}{y\in B}{\rAEa{3}}
  \proofstep{3}{G(y)=z}{\rAEb{3}}
  \proofstep{3}{\exists x\in A\, F(x)=y}{\FormulaRefAuto{y\in B \vdash \exists x\in A\, F(x)=y}{4}}
  \proofstep{7}{x\in A \land F(x)=y}{\rA}
  \proofstep{7}{x\in A}{\rAEa{7}}
  \proofstep{7}{F(x)=y}{\rAEb{7}}
  \proofstep{3,7}{G(F(x))=z}{\rIE{9,5}}
  \proofstep{3,7}{(G\circ F)(x)=G(F(x))}{\FormulaRefAuto{x\in A \vdash (G\circ F)(x)=G(F(x))}{8}}
  \proofstep{3,7}{(G\circ F)(x)=z}{\rIE{11,10}}
  \proofstep{3,7}{\exists x\in A\,(G\circ F)(x)=z}{\rEI{\rAI{8,12}}}
  \proofstep{3}{\exists x\in A\,(G\circ F)(x)=z}{\rEE{6,7,13}}
  \proofstep{1}{\exists x\in A\,(G\circ F)(x)=z}{\rEE{2,3,14}}
\end{tabproof}

\FormulaThmDelta[Die Komposition als surjektive Funktion]{%
G\circ F\colon A\sur C
}{
  \DeltaRow{Mengen}{A\dsep B\dsep C}
  \DeltaPrem{Surjektive Funktionen}{F\colon A\sur B \dsep G\colon B\sur C}
}
\begin{tabproof}
    \proofstep{}{G\circ F\colon A\to C}{\FormulaRefAuto{G\circ F\colon A\to C}}
    \proofstep{}{\forall z\in C\exists x\in A\, (G\circ F)(x)=z}{\FormulaRefAuto{z\in C \vdash \exists x\in A\, (G\circ F)(x)=z}}
    \proofstep{}{G\circ F\colon A\sur C}{\FormulaRefAuto{Surjektive Funktion}{1,2}}
\end{tabproof}

\subsubsection{Erhalt der Bijektivität}

\FormulaThmDelta[Die Komposition als bijektive Funktion]{%
F\colon A\bij B \dsep G\colon B\bij C\vdash G\circ F\colon A\bij C
}{
  \DeltaRow{Mengen}{A\dsep B\dsep C}
  \DeltaPrem{Funktionen}{F\colon A\to B \dsep G\colon B\to C}
}
\begin{tabproof}
    \proofstep{1}{F\colon A\bij B}{\rA}
    \proofstep{2}{G\colon B\bij C}{\rA}
    \proofstep{1}{F\colon A\sur B}{\FormulaRefAuto{F\colon A\bij B\vdash F\colon A\sur B}{1}}
    \proofstep{1}{F\colon A\inj B}{\FormulaRefAuto{F\colon A\bij B\vdash F\colon A\inj B}{1}}
    \proofstep{2}{G\colon B\sur C}{\FormulaRefAuto{F\colon A\bij B\vdash F\colon A\sur B}{2}}
    \proofstep{2}{G\colon B\inj C}{\FormulaRefAuto{F\colon A\bij B\vdash F\colon A\inj B}{2}}
    \proofstep{1,2}{ G\circ F\colon A\inj C}{\FormulaRefAuto{G\circ F\colon A\inj C}{4,6}}
    \proofstep{1,2}{ G\circ F\colon A\sur C}{\FormulaRefAuto{G\circ F\colon A\sur C}{3,5}}
    \proofstep{1,2}{ G\circ F\colon A\bij C}{\FormulaRefAuto{F\colon A\inj B\dsep F\colon A\sur B\vdash F\colon A\bij B}{7,8}}
\end{tabproof}


\subsubsection{Komposition mit Inklusionen und Einschränkung}

\FormulaThmDelta{%
A\subseteq B\vdash F\restriction_A = F\circ \iota_{A,B}%
}{
  \DeltaRow{Mengen}{A \dsep B \dsep C \dsep x}
  \DeltaPrem{Funktionen}{F\colon B\to C}
  \DeltaPrem{Inklusionsabbildung}{\iota_{A,B}\colon A\to B}[\FormulaRefAuto{A\subseteq B\vdash \iota_{A,B}\colon A\to B}]
  \DeltaPrem{Komposition}{F\circ \iota_{A,B}\colon A\to C}[\FormulaRefAuto{G\circ F\colon A\to C}]
  \DeltaPrem{Einschränkung}{F\restriction_A\colon A\to C}[\FormulaRefAuto{C\subseteq A \vdash F\restriction_C\colon C\to B}]
}
\begin{tabproofwide}
  \proofstepwidestar[1]{A\subseteq B}{\rA}

  \proofstepwidestar[2]{x\in A}{\rA}

  \proofstepwidestar[1,2]{x=\iota_{A,B}(x)}{%
    \FormulaRefAuto{A\subseteq B \dsep x\in A \vdash x=\iota_{A,B}(x)}{1,2}}

  \proofstepwide[1,2]{F\restriction_A(x)}{=}{F(x)}{
    \FormulaRefAuto{C\subseteq A \dsep x\in C \vdash F\restriction_C(x)=F(x)}{1,2}}

  \proofstepwide[1,2]{}{=}{F(\iota_{A,B}(x))}{
    \rIE{3,4}}    

  \proofstepwide[1,2]{}{=}{(F\circ\iota_{A,B})(x)}{
    \FormulaRefAuto{x\in A \vdash F(G(x))=(F\circ G)(x)}{2}}   
    
  \proofstepwide[1,2]{F\restriction_A(x)}{=}{(F\circ\iota_{A,B})(x)}{
    \rChain{4,6}}

  \proofstepwidestar[1]{\forall x\in A\,\bigl(F\restriction_A(x)=(F\circ \iota_{A,B})(x)\bigr)}{%
    \rUI{\rRI{2,7}}}

  \proofstepwidestar[1]{F\restriction_A = F\circ \iota_{A,B}}{%
    \FormulaRefAuto{\forall x\in A (F(x)=G(x)) \eqvdash F=G}{8}}
\end{tabproofwide}

\subsubsection{Rechenregeln}

\FormulaThmDelta{%
G = H \vdash F\circ G = F\circ H
}{
  \DeltaRow{Mengen}{A \dsep B \dsep C \dsep D}
  \DeltaPrem{Funktionen}{G,H\colon A\to B \dsep F\colon B\to C}
}
\begin{tabproof}
  \proofstep{1}{G=H}{\rA}
  \proofstep{}{F\circ G=F\circ G}{\rII}
  \proofstep{}{F\circ G=F\circ H}{\rIE{1,2}}
\end{tabproof}

\FormulaThmDelta{%
G = H \vdash G\circ F = H\circ F
}{
  \DeltaRow{Mengen}{x \dsep A \dsep B \dsep C}
  \DeltaPrem{Funktionen}{F\colon A\to B \dsep G,H\colon B\to C}
}
\begin{tabproof}
  \proofstep{1}{G=H}{\rA}
  \proofstep{}{G\circ F=G\circ F}{\rII}
  \proofstep{}{G\circ F=H\circ F}{\rIE{1,2}}
\end{tabproof}

% — Assoziativität der Komposition —
\FormulaThmDelta[Assoziativität]{%
H\circ (G\circ F) = (H\circ G)\circ F
}{
  \DeltaRow{Mengen}{A \dsep B \dsep C \dsep D}
  \DeltaPrem{Funktionen}{F\colon A\to B \dsep G\colon B\to C \dsep H\colon C\to D}
}
\begin{tabproofwide}
  \proofstepwidestar[1]{x\in A}{\rA}
  \proofstepwidestar[1]{F(x)\in B}{\FormulaRefAuto{x\in A\vdash F(x)\in B}{1}}
  \proofstepwidestar[1]{(G\circ F)(x)=G(F(x))}{\FormulaRefAuto{x\in A \vdash (G\circ F)(x)=G(F(x))}{1}}
  \proofstepwide[1]{(H\circ(G\circ F))(x)}{=}{H((G\circ F)(x))}{\FormulaRefAuto{x\in A \vdash (G\circ F)(x)=G(F(x))}{1}}
  \proofstepwide[1]{}{=}{H(G(F(x)))}{\rIE{3,4}}
  \proofstepwide[1]{}{=}{(H\circ G)(F(x))}{\FormulaRefAuto{x\in A \vdash (G\circ F)(x)=G(F(x))}{2}}
  \proofstepwide[1]{}{=}{((H\circ G)\circ F)(x)}{\FormulaRefAuto{x\in A \vdash (G\circ F)(x)=G(F(x))}{1}}
  \proofstepwide[1]{(H\circ(G\circ F))(x)}{=}{((H\circ G)\circ F)(x)}{\rChain{4,7}}
  \proofstepwide[]{\forall x\in A\,((H\circ(G\circ F))(x)}{=}{((H\circ G)\circ F)(x))}{\rUI{\rRI{1,8}}}
  \proofstepwide[]{H\circ (G\circ F)}{=}{(H\circ G)\circ F}{\FormulaRefAuto{\forall x\in A (F(x)=G(x)) \eqvdash F=G}{9}}
\end{tabproofwide}

% — Linksneutralität —
\FormulaThmDelta[Linksneutralität]{%
\Id_B \circ F = F
}{
  \DeltaRow{Mengen}{A \dsep B}
  \DeltaPrem{Funktionen}{F\colon A\to B}
}
\begin{tabproofwide}
  \proofstepwidestar[1]{x\in A}{\rA}
  \proofstepwidestar[1]{F(x)\in B}{\FormulaRefAuto{x\in A \vdash F(x)\in B}{1}}

  \proofstepwide[1]{(\Id_B\circ F)(x)}{=}{\Id_B(F(x))}{\FormulaRefAuto{x\in A \vdash (G\circ F)(x)=G(F(x))}{1}}
  \proofstepwide[1]{}{=}{F(x)}{\FormulaRefAuto{x\in A \vdash \Id_A(x)=x}{3}}
  \proofstepwide[1]{(\Id_B\circ F)(x)}{=}{F(x)}{\rChain{3,4}}
  
  \proofstepwide[]{\forall x\in A\,\bigl((\Id_B\circ F)(x)}{=}{F(x)\bigr)}{\rUI{\rRI{1,5}}}
  \proofstepwide[]{\Id_B\circ F}{=}{F}{\FormulaRefAuto{\forall x\in A\,(F(x)=G(x)) \eqvdash F=G}{6}}
\end{tabproofwide}

% — Rechtsneutralität —
\FormulaThmDelta[Rechtsneutralität]{%
F \circ \Id_A = F
}{
  \DeltaRow{Mengen}{A \dsep B}
  \DeltaPrem{Funktionen}{F\colon A\to B}
}
\begin{tabproofwide}
  \proofstepwidestar[1]{x\in A}{\rA}

  % Fixpunkteigenschaft der Identität auf A
  \proofstepwidestar[1]{\Id_A(x)=x}{\FormulaRefAuto{x\in A \vdash \Id_A(x)=x}{1}}

  % Komposition auswerten
  \proofstepwide[1]{(F\circ \Id_A)(x)}{=}{F(\Id_A(x))}{\FormulaRefAuto{x\in A \vdash (G\circ F)(x)=G(F(x))}{1}}

  % Ersetze \Id_A(x) durch x
  \proofstepwide[1]{}{=}{F(x)}{\rIE{2,3}}

  \proofstepwide[1]{(F\circ \Id_A)(x)}{=}{F(x)}{\rChain{3,4}}
  
  % Verallgemeinern und Extensionalität
  \proofstepwide[]{\forall x\in A\,\bigl((F\circ \Id_A)(x)}{=}{F(x)\bigr)}{\rUI{\rRI{1,5}}}
  \proofstepwide[]{F\circ \Id_A}{=}{F}{\FormulaRefAuto{\forall x\in A\,(F(x)=G(x)) \eqvdash F=G}{6}}
\end{tabproofwide}

\FormulaThmDelta{%
F\circ F^{-1} = \Id_B
}{
  \DeltaRow{Mengen}{A \dsep B}
  \DeltaPrem{Bijektive Funktionen}{F\colon A\bij B}
}
\begin{tabproof}
    \proofstep{1}{y\in B}{\rA}
    \proofstep{1}{F(F^{-1}(y))=y}{\FormulaRefAuto{F\colon A\bij B\dsep y\in B \vdash F\bigl(F^{-1}(y)\bigr)=y}{1}}
    \proofstep{1}{(F\circ F^{-1})(y)=F(F^{-1}(y))}{\FormulaRefAuto{x\in A \vdash (G\circ F)(x)=G(F(x))}{2}}
    \proofstep{1}{(F\circ F^{-1})(y)=y}{\rIE{2,3}}
    \proofstep{1}{\Id_B(y)=y}{\FormulaRefAuto{x\in A \vdash \Id_A(x)=x}{1}}
    \proofstep{1}{(F\circ F^{-1})(y)=\Id_B(y)}{\rIE{4,5}}
    \proofstep{1}{\forall y\in B\,(F\circ F^{-1})(y)=\Id_B(y)}{\rUI{\rRI{1,6}}}
    \proofstep{1}{F\circ F^{-1}=\Id_B}{\FormulaRefAuto{\forall x\in A (F(x)=G(x)) \eqvdash F=G}{7}}    
\end{tabproof}

\FormulaThmDelta{%
F^{-1}\circ F = \Id_A
}{
  \DeltaRow{Mengen}{x \dsep A \dsep B}
  \DeltaPrem{Bijektive Funktionen}{F\colon A\to B}
}
\begin{tabproof}
  \proofstep{1}{x\in A}{\rA}
  \proofstep{1}{F^{-1}(F(x))=x}{\FormulaRefAuto{x\in A \vdash F^{-1}(F(x))=x}{1}}
  \proofstep{1}{(F^{-1}\circ F)(x)=F^{-1}(F(x))}{\FormulaRefAuto{x\in A \vdash (G\circ F)(x)=G(F(x))}{1}}
  \proofstep{1}{(F^{-1}\circ F)(x)=x}{\rIE{2,3}}
  \proofstep{1}{\Id_A(x)=x}{\FormulaRefAuto{x\in A \vdash \Id_A(x)=x}{1}}
  \proofstep{1}{(F^{-1}\circ F)(x)=\Id_A(x)}{\rIE{4,5}}
  \proofstep{ }{\forall x\in A\,\bigl((F^{-1}\circ F)(x)=\Id_A(x)\bigr)}{\rUI{\rRI{1,6}}}
  \proofstep{ }{F^{-1}\circ F=\Id_A}{\FormulaRefAuto{\forall x\in A (F(x)=G(x)) \eqvdash F=G}{7}}
\end{tabproof}

% ------------------------------------------------------------
% Hilfslemma: die breite Kettenrechnung auslagern
% ------------------------------------------------------------
\FormulaThmDelta{%
  x\in A \vdash (G^{-1}\circ F^{-1})\bigl((F\circ G)(x)\bigr)=x%
}{
  \DeltaRow{Mengen}{x \dsep A \dsep B \dsep C}
  \DeltaPrem{Bijektive Funktionen}{G\colon A\to B \dsep F\colon B\to C}
}
\begin{tabproofwide}
  \proofstepwidestar[1]{x\in A}{\rA}
  \proofstepwidestar[1]{G(x)\in B}{\FormulaRefAuto{x\in A \vdash F(x)\in B}{1}}
  \proofstepwidestar[1]{F(G(x))\in C}{\FormulaRefAuto{x\in A \vdash F(x)\in B}{2}}

  \proofstepwidestar[1]{(F\circ G)(x)=F(G(x))}{\FormulaRefAuto{x\in A \vdash (G\circ F)(x)=G(F(x))}{1}}

  \proofstepwidestar[1]{F^{-1}\bigl(F(G(x))\bigr)=G(x)}{\FormulaRefAuto{x\in A \vdash F^{-1}\bigl(F(x)\bigr)=x}{2}}
  \proofstepwidestar[1]{G^{-1}\bigl(G(x)\bigr)=x}{\FormulaRefAuto{x\in A \vdash F^{-1}\bigl(F(x)\bigr)=x}{1}}

  \proofstepwide[1]{%
    \begin{aligned}[t]
      (G^{-1}\circ F^{-1})\\
      \bigl((F\circ G)(x)\bigr)
    \end{aligned}
  }{=}{%
    \begin{aligned}[t]
      (G^{-1}\circ F^{-1})\\
      \bigl(F(G(x))\bigr)
    \end{aligned}
  }{\rIE{4,7}}

  \proofstepwide[1]{}{=}{G^{-1}\Bigl(F^{-1}\bigl(F(G(x))\bigr)\Bigr)}{\FormulaRefAuto{x\in A \vdash (G\circ F)(x)=G(F(x))}{3}}
  \proofstepwide[1]{}{=}{G^{-1}\bigl(G(x)\bigr)}{\rIE{5,8}}
  \proofstepwide[1]{}{=}{x}{\rIE{6,9}}

  \proofstepwide[1]{(G^{-1}\circ F^{-1})\bigl((F\circ G)(x)\bigr)}{=}{x}{\rChain{7,10}}
\end{tabproofwide}


% ------------------------------------------------------------
% Inversenformel für Kompositionen (kurzer Beweis: Block ersetzt)
% ------------------------------------------------------------
\FormulaThmDelta[Inverse der Komposition]{%
  G^{-1}\circ F^{-1} = (F\circ G)^{-1}%
}{
  \DeltaRow{Mengen}{x \dsep A \dsep B \dsep C}
  \DeltaPrem{Bijektive Funktionen}{G\colon A\to B \dsep F\colon B\to C}
}
\begin{tabproof}
  \proofstep{1}{x\in A}{\rA}

  \proofstep{1}{(G^{-1}\circ F^{-1})\bigl((F\circ G)(x)\bigr)=x}{%
    \FormulaRefAuto{x\in A \vdash (G^{-1}\circ F^{-1})\bigl((F\circ G)(x)\bigr)=x}{1}%
  }

  \proofstep{}{%
    \forall x\in A\,\Bigl((G^{-1}\circ F^{-1})\bigl((F\circ G)(x)\bigr)=x\Bigr)%
  }{%
    \rUI{\rRI{1,2}}%
  }

  \proofstep{}{%
    G^{-1}\circ F^{-1} = (F\circ G)^{-1}%
  }{%
    \FormulaRefAuto{\forall y\in B\,\bigl(F(G(y))=y\bigr) \vdash G = F^{-1}}{3}%
  }
\end{tabproof}




\subsubsection{Eigenschaften der Komponenten}

\FormulaThmDeltaR{%
x\in A\dsep y\in A\dsep F(x)=F(y)\vdash x=y
}{G\circ F\colon A\to C bijektiv\dsep x\in A\dsep y\in A\dsep F(x)=F(y)\vdash x=y}{
  \DeltaRow{Mengen}{A\dsep B\dsep C}
  \DeltaPrem{Funktionen}{F\colon A\to B\dsep G\colon B\to C}
  \DeltaPrem{Bijektive Funktionen}{G\circ F\colon A\to C }
}
\begin{tabproof}
    \proofstep{1}{x\in A}{\rA}
    \proofstep{2}{y\in A}{\rA}
    \proofstep{3}{F(x)=F(y)}{\rA}
    \proofstep{1}{F(x)\in B}{\FormulaRefAuto{x\in A\vdash F(x)\in B}{1}}
    \proofstep{2}{F(y)\in B}{\FormulaRefAuto{x\in A\vdash F(x)\in B}{2}}
    \proofstep{1,2,3}{G(F(x))=G(F(y))}{\FormulaRefAuto{x\in A\dsep y\in A\dsep x=y\vdash F(x)=F(y)}{4,5,3}}
    \proofstep{1,2,3}{(G\circ F)(x)=(G\circ F)(y)}{\FormulaRefAuto{x\in A\dsep y\in A\dsep G(F(x))=G(F(y))\vdash (G\circ F)(x)=(G\circ F)(y)}{6}}
    \proofstep{1,2,3}{x=y}{\FormulaRefAuto{Injektivität}{1,2,7}}
\end{tabproof}

\FormulaThmDeltaR{%
y\in C\vdash \exists x\in B\; G(x)=y
}{G\circ F\colon A\to C bijektiv\dsep y\in C\vdash \exists x\in B\; G(x)=y}{
  \DeltaRow{Mengen}{A\dsep B\dsep C}
  \DeltaPrem{Funktionen}{F\colon A\to B\dsep G\colon B\to C}
  \DeltaPrem{Bijektive Funktionen}{G\circ F\colon A\to C }
}
\begin{tabproof}
    \proofstep{1}{y\in C}{\rA}
    \proofstep{1}{\exists x\in A\, (G\circ F)(x)=y}{\FormulaRefAuto{y\in B\vdash \exists x\in A\; F(x)=y}{1}}
    \proofstep{3}{x\in A\land (G\circ F)(x)=y}{\rA}
    \proofstep{3}{x\in A}{\rAEa{3}}
    \proofstep{3}{(G\circ F)(x)=y}{\rAEb{3}}
    \proofstep{3}{(G\circ F)(x)=G(F(x))}{\FormulaRefAuto{x\in A \vdash (G\circ F)(x)=G(F(x))}{4}}
    \proofstep{3}{G(F(x))=y}{\rEE{6,5}}
    \proofstep{3}{F(x)\in B}{\FormulaRefAuto{x\in A\vdash F(x)\in B}{4}}
    \proofstep{3}{F(x)\in B\land G(F(x))=y}{\rAI{8,7}}
    \proofstep{3}{\exists x\in B\, G(x)=y}{\rEI{9}}
    \proofstep{1}{\exists x\in B\, G(x)=y}{\rEE{2,3,10}}
\end{tabproof}

\FormulaThmDeltaR{%
x\in B\dsep y\in B\dsep G(x)=G(y)\vdash x=y
}{G\circ F\colon A\to C bijektiv\dsep F\colon A\to B surjektiv\dsep x\in B\dsep y\in B\dsep G(x)=G(y)\vdash x=y}{
  \DeltaRow{Mengen}{A\dsep B\dsep C}
  \DeltaPrem{Funktionen}{G\colon B\to C}
  \DeltaPrem{Surjektive Funktionen}{F\colon A\to B}
  \DeltaPrem{Bijektive Funktionen}{G\circ F\colon A\to C }
}
\begin{tabproof}
    \proofstep{1}{x\in B}{\rA}
    \proofstep{2}{y\in B}{\rA}
    \proofstep{3}{G(x)=G(y)}{\rA}
    \proofstep{1}{\exists u\in A\,F(u)=x}{\FormulaRefAuto{y\in B\vdash \exists x\in A\; F(x)=y}{1}}
    \proofstep{2}{\exists v\in A\,F(v)=y}{\FormulaRefAuto{y\in B\vdash \exists x\in A\; F(x)=y}{2}}
    \proofstep{6}{u\in A\land F(u)=x}{\rA}
    \proofstep{6}{u\in A}{\rAEa{6}}
    \proofstep{6}{F(u)=x}{\rAEb{6}}
    \proofstep{9}{v\in A\land F(v)=y}{\rA}
    \proofstep{9}{v\in A}{\rAEa{9}}
    \proofstep{9}{F(v)=y}{\rAEb{9}}
    \proofstep{3,6,9}{G(F(u))=G(F(v))}{\rIE{9,\rIE{8,3}}}
    \proofstep{3,6,9}{(G\circ F)(u)=(G\circ F)(v)}{\FormulaRefAuto{x\in A\dsep y\in A\dsep (G\circ F)(x)=(G\circ F)(y) \vdash G(F(x))=G(F(y))}{7,10,13}}
    \proofstep{3,6,9}{u=v}{\FormulaRefAuto{x\in A \dsep y\in A \dsep (G\circ F)(x)=(G\circ F)(y) \vdash x=y}{7,10,14}}
    \proofstep{3,6,9}{F(u)=F(v)}{\FormulaRefAuto{x\in A\dsep y\in A\dsep x=y\vdash F(x)=F(y)}{7,10,14}}
    \proofstep{3,6,9}{x=y}{\rIE{11,\rIE{8,15}}}
    \proofstep{1,3,9}{x=y}{\rEE{4,6,16}}
    \proofstep{1,2,3}{x=y}{\rEE{5,9,17}}
\end{tabproof}

\FormulaThmDelta{%
 G\circ F\colon A\bij C\vdash F\colon A\inj B
}{
  \DeltaRow{Mengen}{A \dsep B \dsep C}
  \DeltaPrem{Funktionen}{F\colon A\to B \dsep G\colon B\to C}
}
\begin{tabproof}
    \proofstep{1}{ G\circ F\colon A\bij C}{\rA}
    \proofstep{1}{ \forall x,y\in A\, (F(x)=F(y)\rightarrow x=y)}{\FormulaRefAuto{G\circ F\colon A\to C bijektiv\dsep x\in A\dsep y\in A\dsep F(x)=F(y)\vdash x=y}{1}}
    \proofstep{1}{F\colon A\inj B}{\FormulaRefAuto{Injektive Funktion}{2}}
\end{tabproof}

\FormulaThmDelta{%
 G\circ F\colon A\bij C\vdash G\colon B\sur C
}{
  \DeltaRow{Mengen}{A \dsep B \dsep C}
  \DeltaPrem{Funktionen}{F\colon A\to B \dsep G\colon B\to C}
}
\begin{tabproof}
    \proofstep{1}{ G\circ F\colon A\bij C}{\rA}
    \proofstep{1}{\forall y\in C\exists x\in B\; G(x)=y}{\FormulaRefAuto{G\circ F\colon A\to C bijektiv\dsep y\in C\vdash \exists x\in B\; G(x)=y}{1}}
    \proofstep{1}{F\colon A\inj B}{\FormulaRefAuto{Surjektive Funktion}{2}}
\end{tabproof}


\FormulaThmDelta{%
 F\colon A\sur B\dsep G\circ F\colon A\bij C\vdash G\colon B\bij C
}{
  \DeltaRow{Mengen}{A \dsep B \dsep C}
  \DeltaPrem{Funktionen}{F\colon A\to B \dsep G\colon B\to C}
}
\begin{tabproof}
    \proofstep{1}{ F\colon A\sur B}{\rA}
    \proofstep{2}{ G\circ F\colon A\bij C}{\rA}
    \proofstep{2}{G\colon B\sur C}{\FormulaRefAuto{ G\circ F\colon A\bij C\vdash G\colon B\sur C}{2}}
    \proofstep{1,2}{\forall x,y\in B\, (G(x)=G(y)\rightarrow x=y)}{\FormulaRefAuto{G\circ F\colon A\to C bijektiv\dsep F\colon A\to B surjektiv\dsep x\in B\dsep y\in B\dsep G(x)=G(y)\vdash x=y}{1,2}}
    \proofstep{1,2}{G\colon B\inj C}{\FormulaRefAuto{Injektive Funktion}{4}}
    \proofstep{1,2}{G\colon B\bij C}{\FormulaRefAuto{F\colon A\inj B\dsep F\colon A\sur B\vdash F\colon A\bij B}{5,3}}
\end{tabproof}

\FormulaThmDelta{%
F\circ G = \Id_A\dsep G\circ F =\Id_B \vdash F\colon B\bij A
}{
  \DeltaRow{Mengen}{A}
  \DeltaPrem{Funktionen}{G\colon A\to B \dsep F\colon B\to A}
}
\begin{tabproof}
    \proofstep{1}{ F\circ G = \Id_A}{\rA}
    \proofstep{2}{ G\circ F =\Id_B}{\rA}
    \proofstep{}{\Id_A\colon A\bij A}{\FormulaRefAuto{\Id_A\colon A\bij A}}
    \proofstep{}{\Id_B\colon B\bij B}{\FormulaRefAuto{\Id_A\colon A\bij A}}
    \proofstep{1}{F\circ G\colon A\bij A}{\rIE{1,3}}
    \proofstep{2}{G\circ F\colon B\bij B}{\rIE{2,4}}
    \proofstep{1}{F\colon B\sur A}{\FormulaRefAuto{G\circ F\colon A\bij C\vdash G\colon B\sur C}{5}}
    \proofstep{2}{F\colon B\inj A}{\FormulaRefAuto{G\circ F\colon A\bij C\vdash F\colon A\inj B}{6}}
    \proofstep{1,2}{F\colon B\bij A}{\FormulaRefAuto{F\colon A\inj B\dsep F\colon A\sur B\vdash F\colon A\bij B}{8,7}}
\end{tabproof}

\FormulaThmDelta{%
F\circ G = \Id_A\dsep G\circ F =\Id_B \vdash G\colon A\bij B
}{
  \DeltaRow{Mengen}{A}
  \DeltaPrem{Funktionen}{G\colon A\to B \dsep F\colon B\to A}
}
\begin{tabproof}
    \proofstep{1}{ F\circ G = \Id_A}{\rA}
    \proofstep{2}{ G\circ F =\Id_B}{\rA}
    \proofstep{}{\Id_A\colon A\bij A}{\FormulaRefAuto{\Id_A\colon A\bij A}}
    \proofstep{}{\Id_B\colon B\bij B}{\FormulaRefAuto{\Id_A\colon A\bij A}}
    \proofstep{1}{F\circ G\colon A\bij A}{\rIE{1,3}}
    \proofstep{2}{G\circ F\colon B\bij B}{\rIE{2,4}}
    \proofstep{1}{G\colon A\sur B}{\FormulaRefAuto{G\circ F\colon A\bij C\vdash G\colon B\sur C}{6}}
    \proofstep{2}{G\colon A\inj B}{\FormulaRefAuto{G\circ F\colon A\bij C\vdash F\colon A\inj B}{5}}
    \proofstep{1,2}{F\colon B\bij A}{\FormulaRefAuto{F\colon A\inj B\dsep F\colon A\sur B\vdash F\colon A\bij B}{8,7}}
\end{tabproof}

\FormulaThmDelta{%
F\circ G = \Id_A\dsep G\circ F =\Id_B \vdash G=F^{-1}
}{
  \DeltaRow{Mengen}{A}
  \DeltaPrem{Funktionen}{G\colon A\to B \dsep F\colon B\to A}
}
\begin{tabproofwide}
    \proofstepwidestar[1]{ F\circ G = \Id_A}{\rA}
    \proofstepwidestar[2]{ G\circ F =\Id_B}{\rA}
    \proofstepwidestar[1,2]{F\colon B\bij A}{\FormulaRefAuto{F\circ G = \Id_A\dsep G\circ F =\Id_B \vdash F\colon B\bij A}{1,2}}
    \proofstepwide[]{G}{=}{G\circ \Id_A}{\FormulaRefAuto{F \circ \Id_A = F}}
    \proofstepwide[1,2]{}{=}{G\circ (F\circ F^{-1})}{\rIE{\FormulaRefAuto{F\circ F^{-1} = \Id_B}{3},4}}
    \proofstepwide[1,2]{}{=}{(G\circ F)\circ F^{-1}}{\FormulaRefAuto{H\circ (G\circ F) = (H\circ G)\circ F}{5}}
    \proofstepwide[1,2]{}{=}{\Id_B\circ F^{-1}}{\rIE{2,6}}
    \proofstepwide[1,2]{}{=}{F^{-1}}{\FormulaRefAuto{\Id_B\circ F = F}}
    \proofstepwide[1,2]{G}{=}{F^{-1}}{\rChain{4,8}}
\end{tabproofwide}

% ------------------------------------------------------------
% Hilfssatz: Auswertung der Komposition (rückwärts)
% ------------------------------------------------------------


% ------------------------------------------------------------
% Hauptsatz: Rechtsinverse ist injektiv
% ------------------------------------------------------------
\FormulaThmDelta[Rechtsinverse ist injektiv]{%
  F\circ G=\Id_B\vdash G\colon B\inj A%
}{
  \DeltaRow{Mengen}{A\dsep B\dsep x\dsep y}
  \DeltaPrem{Funktionen}{F\colon A\to B \dsep G\colon B\to A}
}
\begin{tabproofwide}
  \proofstepwidestar[1]{F\circ G=\Id_B}{\rA}

  \proofstepwidestar[2]{x\in B}{\rA}
  \proofstepwidestar[3]{y\in B}{\rA}
  \proofstepwidestar[4]{G(x)=G(y)}{\rA}

  \proofstepwidestar[1]{\Id_B=F\circ G}{%
    \FormulaRefAuto{X=Y\vdash Y=X}}

  \proofstepwide[2]{x}{=}{\Id_B(x)}{%
    \FormulaRefAuto{x\in A\vdash x=\Id_A(x)}}

  \proofstepwide[1,2]{}{=}{(F\circ G)(x)}{%
    \rIE{5,2}}

  \proofstepwide[2]{}{=}{F(G(x))}{%
    \FormulaRefAuto{x\in B \vdash (F\circ G)(x)=F(G(x))}}

  \proofstepwide[2,3,4]{}{=}{F(G(y))}{%
    \rIE{4,8}}

  \proofstepwide[3]{}{=}{(F\circ G)(y)}{%
    \FormulaRefAuto{x\in B \vdash F(G(x))=(F\circ G)(x)}}

  \proofstepwide[1,3]{}{=}{\Id_B(y)}{%
    \rIE{1,3}}

  \proofstepwide[3]{}{=}{y}{%
    \FormulaRefAuto{x\in B\vdash \Id_B(x)=x}}

  \proofstepwide[1,2,3,4]{x}{=}{y}{%
    \rChain{6,12}}

  \proofstepwidestar[1]{G\colon B\inj A}{%
    \FormulaRefAuto{Injektive Funktion}{13}}
\end{tabproofwide}





\chapter{Das Auswahlprinzip}

\section{Von der Surjektionsform zur Relationsform}

\FormulaThmDelta{%
  \pi_1\restriction_R\circ s = \Id_A\dsep x\in A
  \vdash \pi_1(s(x))=x
}{
  \DeltaRow{Mengen}{A\dsep B}
  \DeltaPrem{Totale Relationen}{\TotRel{R,A,B}}
  \DeltaPrem{Funktionen}{s\colon A\to R}
  \DeltaRow{Projektion auf erste Komponente}{\pi_1:A\times B\to A}
}
\begin{tabproofwide}
  \proofstepwidestar[1]{\pi_1\restriction_R\circ s = \Id_A}{\rA}
  \proofstepwidestar[2]{x\in A}{\rA}
  \proofstepwidestar[]{R\subseteq A\times B}{\FormulaRefAuto{F \subseteq A \times B}}
  \proofstepwidestar[2]{s(x)\in R}{\FormulaRefAuto{x\in A\vdash F(x)\in B}{2}}

  \proofstepwide[2]{\pi_1(s(x))}{=}{\pi_1\restriction_R(s(x))}{%
      \FormulaRefAuto{C\subseteq A \dsep x\in C \vdash F\restriction_C(x)=F(x)}{3,4}}

  \proofstepwide[2]{}{=}{(\pi_1\restriction_R\circ s)(x)}{%
      \FormulaRefAuto{x\in A \vdash (G\circ F)(x)=G(F(x))}{2}}

  \proofstepwide[1,2]{}{=}{\Id_A(x)}{%
    \FormulaRefAuto{x\in A\dsep F=G\vdash F(x)=G(x)}{2,1}}

  \proofstepwide[2]{}{=}{x}{%
    \FormulaRefAuto{x\in A\vdash \Id_A(x)=x}{2}}

  \proofstepwide[1,2]{\pi_1(s(x))}{=}{x}{\rChain{6,9}}
\end{tabproofwide}

\FormulaThmDelta{%
  \pi_1\restriction_R\circ s = \Id_A
  \vdash \forall x\in A\,\bigl(x,\pi_2(s(x))\bigr)\in R
}{
  \DeltaRow{Mengen}{A\dsep B}
  \DeltaPrem{Totale Relationen}{\TotRel{R,A,B}}
  \DeltaPrem{Funktionen}{s\colon A\to R}
  \DeltaRow{Projektion auf erste Komponente}{\pi_1\colon A\times B\to A}
  \DeltaRow{Projektion auf zweite Komponente}{\pi_2\colon A\times B\to B}
}
\begin{tabproofwide}
  \proofstepwidestar[1]{\pi_1\restriction_R\circ s = \Id_A}{\rA}
  \proofstepwidestar[]{R\subseteq A\times B}{\FormulaRefAuto{R \subseteq A \times B}}

  \proofstepwidestar[3]{x\in A}{\rA}
  \proofstepwidestar[3]{s(x)\in R}{\FormulaRefAuto{x\in A\vdash F(x)\in B}{3}}
  \proofstepwidestar[3]{s(x)\in A\times B}{\FormulaRefAuto{A\subseteq B, x\in A\vdash x\in B}{2,4}}

  \proofstepwidestar[3]{s(x)=\bigl(\pi_1(s(x)),\pi_2(s(x))\bigr)}{%
    \FormulaRefAuto{z\in A\times B \vdash z=(\pi_1(z),\pi_2(z))}{5}}

  \proofstepwidestar[1,3]{\pi_1(s(x))=x}{%
    \FormulaRefAuto{\pi_1\restriction_R\circ s = \Id_A\dsep x\in A \vdash \pi_1(s(x))=x}{1,3}}

  \proofstepwidestar[1,3]{\bigl(\pi_1(s(x)),\pi_2(s(x))\bigr)=\bigl(x,\pi_2(s(x))\bigr)}{%
    \rIE{7,6}}

  \proofstepwidestar[1,3]{s(x)=\bigl(x,\pi_2(s(x))\bigr)}{%
    \FormulaRefAuto{a=b, b=c\vdash a=c}{6,8}}

  \proofstepwidestar[1,3]{\bigl(x,\pi_2(s(x))\bigr)\in R}{%
    \rIE{9,4}}

  \proofstepwidestar[1]{\forall x\in A\,\bigl(x,\pi_2(s(x))\bigr)\in R}{%
    \rUI{\rRI{3,10}}}
\end{tabproofwide}


\FormulaThmDelta{%
  \pi_1\restriction_R\circ s = \Id_A
  \vdash \forall x\in A\,\bigl(x,\pi_2\restriction_R(s(x))\bigr)\in R
}{
  \DeltaRow{Mengen}{A\dsep B}
  \DeltaPrem{Totale Relationen}{\TotRel{R,A,B}}
  \DeltaPrem{Funktionen}{s\colon A\to R}
  \DeltaRow{Projektion auf zweite Komponente}{\pi_2\colon A\times B\to B}
}
\begin{tabproof}
  \proofstep{1}{\pi_1\restriction_R\circ s = \Id_A}{\rA}
  \proofstep{}{R\subseteq A\times B}{\FormulaRefAuto{R \subseteq A \times B}}

  \proofstep{1}{\forall x\in A\,\bigl(x,\pi_2(s(x))\bigr)\in R}{%
    \FormulaRefAuto{\pi_1\restriction_R\circ G = \Id_A \vdash \forall x\in A\,\bigl(x,\pi_2(G(x))\bigr)\in R}{1}}

  \proofstep{}{x\in A}{\rA}
  \proofstep{}{s(x)\in R}{\FormulaRefAuto{x\in A\vdash F(x)\in B}{4}}

  \proofstep{}{ \pi_2\restriction_R(s(x))=\pi_2(s(x))}{%
    \FormulaRefAuto{C\subseteq A \dsep x\in C \vdash F\restriction_C(x)=F(x)}{2,5}}

  \proofstep{}{ \pi_2(s(x))=\pi_2\restriction_R(s(x))}{%
    \FormulaRefAuto{a=b\vdash b=a}{6}}

  \proofstep{}{(x,\pi_2(s(x)))\in R}{\rUE{3}}
  \proofstep{}{(x,\pi_2\restriction_R(s(x)))\in R}{\rIE{7,8}}

  \proofstep{1}{\forall x\in A\,\bigl(x,\pi_2\restriction_R(s(x))\bigr)\in R}{%
    \rUI{\rRI{4,9}}}
\end{tabproof}

\FormulaThmDelta{%
  \pi_1\restriction_R\circ s = \Id_A
  \vdash \exists F\colon A\to B\,\forall x\in A\,(x,F(x))\in R
}{
  \DeltaRow{Mengen}{A\dsep B}
  \DeltaPrem{Totale Relationen}{\TotRel{R,A,B}}
  \DeltaPrem{Funktionen}{s\colon A\to R}
  \DeltaRow{Projektion auf zweite Komponente}{\pi_2\colon A\times B\to B}
}
\begin{tabproof}
  \proofstep{1}{\pi_1\restriction_R\circ s = \Id_A}{\rA}

  \proofstep{2}{\forall x\in A\,\bigl(x,\pi_2\restriction_R(s(x))\bigr)\in R}{%
    \FormulaRefAuto{\pi_1\restriction_R\circ G = \Id_A \vdash \forall x\in A\,\bigl(x,\pi_2\restriction_R(G(x))\bigr)\in R}{1}}

  \proofstep{3}{\pi_2\restriction_R\circ s\colon A\to B}{%
    \FormulaRefAuto{G\circ F\colon A\to C}}

  \proofstep{4}{x\in A}{\rA}

  \proofstep{4}{\bigl(x,\pi_2\restriction_R(s(x))\bigr)\in R}{\rUE{2}}

  \proofstep{4}{(\pi_2\restriction_R\circ s)(x)=\pi_2\restriction_R(s(x))}{%
    \FormulaRefAuto{x\in A \vdash (G\circ F)(x)=G(F(x))}{4}}

  \proofstep{4}{\pi_2\restriction_R(s(x))=(\pi_2\restriction_R\circ s)(x)}{%
    \FormulaRefAuto{a=b\vdash b=a}{6}}

  \proofstep{4}{\bigl(x,(\pi_2\restriction_R\circ s)(x)\bigr)\in R}{\rIE{7,5}}

  \proofstep{4}{\forall x\in A\,\bigl(x,(\pi_2\restriction_R\circ s)(x)\bigr)\in R}{%
    \rUI{\rRI{4,8}}}

  \proofstep{1}{\exists F\colon A\to B\,\forall x\in A\,(x,F(x))\in R}{%
    \rEI{\rAI{9,3}}}
\end{tabproof}

\FormulaThmDelta{%
  \exists s\colon A\to R\,\bigl(\pi_1\restriction_R\circ s = \Id_A\bigr)
  \vdash \exists F\colon A\to B\,\forall x\in A\,\bigl((x,F(x))\in R\bigr)
}{
  \DeltaRow{Mengen}{A\dsep B}
  \DeltaPrem{Totale Relationen}{\TotRel{R,A,B}}
  \DeltaPrem{Funktionen}{s\colon A\to R}
  \DeltaRow{Projektion auf zweite Komponente}{\pi_2\colon A\times B\to B}
}
\begin{tabproof}
  \proofstep{1}{\exists s\colon A\to R\,(\pi_1\restriction_R\circ s = \Id_A)}{\rA}
  \proofstep{2}{s\colon A\to R\land \pi_1\restriction_R\circ s = \Id_A}{\rA}

  \proofstep{2}{\exists F\colon A\to B\,\forall x\in A\,((x,F(x))\in R)}{%
    \FormulaRefAuto{\pi_1\restriction_R\circ s = \Id_A \vdash \exists F\colon A\to B\,\forall x\in A\,(x,F(x))\in R}{2}}

  \proofstep{1}{\exists F\colon A\to B\,\forall x\in A\,((x,F(x))\in R)}{%
    \rEE{1,2,3}}
\end{tabproof}

\section{Relationsform nach Familienform}

\FormulaThmDelta{%
  \forall x\in M\,\bigl((x,F(x))\in \MemRel(M)\bigr)\dsep X\in M\vdash F(X)\in X
}{
  \DeltaRow{Mengen}{M \dsep X \dsep a}
  \DeltaPrem{\makecell[l]{Paarweise disjunkte\\ nichtleere Familie}}{\DisjFam(M)}
  \DeltaPrem{Funktionen}{F\colon M\to \bigcup M}
  \DeltaRow{Mitgliedschaftsrelation}{\MemRel(M)}[\FormulaRefAuto{\TotRel{\MemRel(M),M,\bigcup M}}]
}
\begin{tabproof}
  \proofstep{1}{%
    \forall x\in M\,\bigl((x,F(x))\in \MemRel(M)\bigr)}{%
    \rA}

  \proofstep{2}{%
    X\in M}{%
    \rA}

  \proofstep{1,2}{%
    (X,F(X))\in \MemRel(M)}{%
    \rRE{2,\rUE{1}}}

  \proofstep{1,2}{%
    X\in M \land F(X)\in \bigcup M \land F(X)\in X}{%
    \FormulaRefAuto{(X,a)\in \MemRel(M) \vdash X\in M \land a\in \bigcup M \land a\in X}{3}}

  \proofstep{1,2}{%
    F(X)\in X}{%
    \rAEb{4}}
\end{tabproof}

\FormulaThmDelta[Existenz eines Auswahlelements in $X$]{%
  \forall x\in M\,\bigl((x,F(x))\in \MemRel(M)\bigr)\dsep X\in M
  \vdash \exists a\in X\,\bigl(a\in F[M]\bigr)
}{
  \DeltaRow{Mengen}{M \dsep X}
  \DeltaPrem{\makecell[l]{Paarweise disjunkte\\ nichtleere Familie}}{\DisjFam(M)}
  \DeltaPrem{Funktionen}{F\colon M\to \bigcup M}
}
\begin{tabproof}
  \proofstep{1}{\forall x\in M\,\bigl((x,F(x))\in \MemRel(M)\bigr)}{\rA}
  \proofstep{2}{X\in M}{\rA}

  \proofstep{1,2}{F(X)\in X}{%
    \FormulaRefAuto{\forall x\in M\,\bigl((x,F(x))\in \MemRel(M)\bigr)\dsep X\in M\vdash F(X)\in X}{1,2}}

  \proofstep{2}{F(X)\in F[M]}{%
    \FormulaRefAuto{x\in A\vdash F(x)\in F[A]}{2}}

  \proofstep{1,2}{\exists a\in X\,\bigl(a\in F[M]\bigr)}{\rEI{\rAI{3,4}}}
\end{tabproof}


\FormulaThmDelta[Auswahlelement in $X$ ist $F(X)$]{%
  \forall x\in M\,\bigl((x,F(x))\in \MemRel(M)\bigr)\dsep X\in M\dsep a\in X\dsep a\in F[M]
  \vdash a=F(X)
}{
  \DeltaRow{Mengen}{M \dsep X \dsep a}
  \DeltaPrem{\makecell[l]{Paarweise disjunkte\\ nichtleere Familie}}{\DisjFam(M)}
  \DeltaPrem{Funktionen}{F\colon M\to \bigcup M}
  \DeltaRow{Mitgliedschaftsrelation}{\MemRel(M)}[\FormulaRefAuto{\TotRel{\MemRel(M),M,\bigcup M}}]
}
\begin{tabproof}
  \proofstep{1}{\forall x\in M\,\bigl((x,F(x))\in \MemRel(M)\bigr)}{\rA}
  \proofstep{2}{X\in M}{\rA}
  \proofstep{3}{a\in X}{\rA}
  \proofstep{4}{a\in F[M]}{\rA}

  \proofstep{4}{\exists Y\in M\,(a=F(Y))}{%
    \FormulaRefAuto{y\in F[A]\eqvdash \exists x\in A\,y=F(x)}{4}}

  \proofstep{6}{Y\in M\land a=F(Y)}{\rA}
  \proofstep{6}{Y\in M}{\rAEa{6}}
  \proofstep{6}{a=F(Y)}{\rAEb{6}}

  \proofstep{1,7}{F(Y)\in Y}{%
    \FormulaRefAuto{\forall x\in M\,\bigl((x,F(x))\in \MemRel(M)\bigr)\dsep X\in M\vdash F(X)\in X}{1,7}}

  \proofstep{1,6}{a\in Y}{%
    \FormulaRefAuto{a=b \dsep b\in A \vdash a\in A}{8,9}}

  \proofstep{1,2,3,6}{X=Y}{%
    \FormulaRefAuto{X \in M \dsep Y \in M \dsep a \in X \dsep a \in Y \vdash X = Y}{2,7,3,10}}

  \proofstep{1,2,3,6}{a=F(X)}{%
    \FormulaRefAuto{x = y \dsep a = F(y) \vdash a = F(x)}{11,8}}

  \proofstep{1,2,3,4}{a=F(X)}{\rEE{5,6,12}}
\end{tabproof}


\FormulaThmDelta[Eindeutige Existenz eines Auswahlelements in $X$]{%
  \forall x\in M\,\bigl((x,F(x))\in \MemRel(M)\bigr)\dsep X\in M
  \vdash \exists! a\in X\,\bigl(a\in F[M]\bigr)
}{
  \DeltaRow{Mengen}{M \dsep X}
  \DeltaPrem{\makecell[l]{Paarweise disjunkte\\ nichtleere Familie}}{\DisjFam(M)}
  \DeltaPrem{Funktionen}{F\colon M\to \bigcup M}
  \DeltaRow{Mitgliedschaftsrelation}{\MemRel(M)}[\FormulaRefAuto{\TotRel{\MemRel(M),M,\bigcup M}}]
}
\begin{tabproof}
  \proofstep{1}{\forall x\in M\,\bigl((x,F(x))\in \MemRel(M)\bigr)}{\rA}
  \proofstep{2}{X\in M}{\rA}

  \proofstep{1,2}{\exists a\in X\,\bigl(a\in F[M]\bigr)}{%
    \FormulaRefAuto{\forall x\in M\,\bigl((x,F(x))\in \MemRel(M)\bigr)\dsep X\in M
    \vdash \exists a\in X\,\bigl(a\in F[M]\bigr)}{1,2}}

  \proofstep{4}{b\in X\land b\in F[M]}{\rA}
  \proofstep{5}{c\in X\land c\in F[M]}{\rA}

  \proofstep{4}{b\in X}{\rAEa{4}}
  \proofstep{4}{b\in F[M]}{\rAEb{4}}
  \proofstep{1,2,4}{b=F(X)}{%
    \FormulaRefAuto{\forall x\in M\,\bigl((x,F(x))\in \MemRel(M)\bigr)\dsep X\in M\dsep a\in X\dsep a\in F[M]
    \vdash a=F(X)}{1,2,6,7}}

  \proofstep{5}{c\in X}{\rAEa{5}}
  \proofstep{5}{c\in F[M]}{\rAEb{5}}
  \proofstep{1,2,5}{c=F(X)}{%
    \FormulaRefAuto{\forall x\in M\,\bigl((x,F(x))\in \MemRel(M)\bigr)\dsep X\in M\dsep a\in X\dsep a\in F[M]
    \vdash a=F(X)}{1,2,9,10}}

  \proofstep{1,2,4,5}{b=c}{\FormulaRefAuto{a=b,c=b\vdash a=c}{8,11}}

  \proofstep{1,2}{\exists! a\in X\,\bigl(a\in F[M]\bigr)}{\UEI{3,4,5,12}}
\end{tabproof}


\FormulaThmDelta[Auswahlfunktion liefert Auswahlmenge]{%
  \exists F\colon M\to \bigcup M\,\forall x\in M\,\bigl((x,F(x))\in \MemRel(M)\bigr)
  \vdash \exists L\,\forall X\in M\,\exists! a\in X\,\bigl(a\in L\bigr)
}{
  \DeltaRow{Mengen}{M}
  \DeltaPrem{\makecell[l]{Paarweise disjunkte\\ nichtleere Familie}}{\DisjFam(M)}
  \DeltaRow{Mitgliedschaftsrelation}{\MemRel(M)}[\FormulaRefAuto{\TotRel{\MemRel(M),M,\bigcup M}}]
}
\begin{tabproof}
  \proofstep{1}{%
    \exists F\colon M\to \bigcup M\,\forall x\in M\,\bigl((x,F(x))\in \MemRel(M)\bigr)
  }{\rA}

  \proofstep{2}{%
    F\colon M\to \bigcup M \land \forall x\in M\,\bigl((x,F(x))\in \MemRel(M)\bigr)
  }{\rA}

  \proofstep{2}{%
    \forall x\in M\,\bigl((x,F(x))\in \MemRel(M)\bigr)
  }{\rAEb{2}}

  \proofstep{4}{X\in M}{\rA}

  \proofstep{2,4}{%
    \exists! a\in X\,\bigl(a\in F[M]\bigr)
  }{%
    \FormulaRefAuto{\forall x\in M\,\bigl((x,F(x))\in \MemRel(M)\bigr)\dsep X\in M
    \vdash \exists! a\in X\,\bigl(a\in F[M]\bigr)}{3,4}
  }

  \proofstep{2}{%
    \forall X\in M\,\exists! a\in X\,\bigl(a\in F[M]\bigr)
  }{\rUI{\rRI{4,5}}}

  \proofstep{2}{%
    \exists L\,\forall X\in M\,\exists! a\in X\,\bigl(a\in L\bigr)
  }{\rEI{6}}

  \proofstep{1}{%
    \exists L\,\forall X\in M\,\exists! a\in X\,\bigl(a\in L\bigr)
  }{\rEE{1,2,7}}
\end{tabproof}



\section{Familienform nach Surjektionsform}

\FormulaThmDeltaR{%
  \begin{aligned}[t]
    &\forall M\,\Bigl(\DisjFam(M)\rightarrow
      \exists L\,\forall X\in M\,\exists! a\in X\,(a\in L)\Bigr)\\
    &\vdash\ \exists L\,\forall X\in \Fib_G[A]\,\exists! a\,(a\in X \land a\in L)
  \end{aligned}%
}{\forall M\,\bigl(\DisjFam(M)\rightarrow \exists L\,\forall X\in M\,\exists! a\in X\, a\in L\bigr) \vdash \exists L\,\forall X\in \Fib_G[A]\,\exists! a\,(a\in X \land a\in L)}{%
  \DeltaDecl{Mengen}{A\dsep B\dsep X\dsep a}%
  \DeltaPrem{Surjektive Funktion}{G\colon B\sur A}%
}

\begin{tabproof}
  \proofstep{1}{\forall M\,\bigl(\DisjFam(M)\rightarrow \exists L\,\forall X\in M\,\exists! a\in X\, a\in L\bigr)}{\rA}
  \proofstep{}{\DisjFam(\Fib_G[A])}{\FormulaRefAuto{\DisjFam(\Fib_F[B])}}
  \proofstep{1}{\exists L\,\forall X\in \Fib_G[A]\,\exists! a\,(a\in X \land a\in L)}{\rRE{\rUE{1},2}}
\end{tabproof}

\FormulaThmDeltaR{%
  \begin{aligned}[t]
    &\forall X\in \Fib_G[A]\,\exists! a\,(a\in X \land a\in L)\dsep x\in A\\
    &\vdash\ \exists! a\,\bigl(a\in \Fib_G(x)\land a\in L\bigr)
  \end{aligned}%
}{%
  \forall X\in \Fib_G[A]\,\exists! a\,(a\in X \land a\in L) \dsep x\in A
  \vdash \exists! a\,\bigl(a\in \Fib_G(x)\land a\in L\bigr)%
}{%
  \DeltaRow{Mengen}{A\dsep B\dsep L\dsep x}%
  \DeltaPrem{Funktionen}{G\colon B\to A}%
}
\begin{tabproof}
  \proofstep{1}{\forall X\in \Fib_G[A]\,\exists! a\,(a\in X \land a\in L)}{\rA}
  \proofstep{2}{x\in A}{\rA}

  \proofstep{2}{\Fib_G(x)\in \Fib_G[A]}{%
    \FormulaRefAuto{x\in A\vdash F(x)\in F[A]}{2}}

  \proofstep{1,2}{\exists! a\,(a\in \Fib_G(x) \land a\in L)}{%
    \rRE{\rUE{1},3}}

\end{tabproof}

% ============================================================
% 3.15.11.x  Konstruktion einer Sektion aus einer Auswahlmenge
% (Einfügen z.B. nach dem Lemma
%  \forall X\in \Fib_G[A]\,\exists! a\,(a\in X \land a\in L) \dsep x\in A
%  \vdash \exists! b\,(b\in \Fib_G(x)\land b\in L)
% und vor dem Hauptsatz \ACsur.)
% ============================================================

\subsection{Die Sektion zu einer surjektiven Funktion}

\subsubsection{Die Auswahltermvorschrift}

\FormulaDefDeltaR{%
  \begin{aligned}[t]
    &\forall X\in \Fib_G[A]\,\exists! a\,(a\in X \land a\in L)\dsep x\in A\\
    &\vdash\ t_{\Sec_{G,L}}(x)\coloneqq \iota b\bigl(b\in \Fib_G(x)\land b\in L\bigr)
  \end{aligned}%
}{%
  \forall X\in \Fib_G[A]\,\exists! a\,(a\in X \land a\in L) \dsep x\in A
  \vdash t_{\Sec_{G,L}}(x)\coloneqq \iota b\bigl(b\in \Fib_G(x)\land b\in L\bigr)%
}{%
  \DeltaRow{Mengen}{A\dsep B\dsep L\dsep X\dsep a\dsep b\dsep x}%
  \DeltaRow{Surjektive Funktion}{G\colon B\sur A}%
  \DeltaRow{Funktionensymbol}{t_{\Sec_{G,L}}}%
}


\FormulaThmDelta[Typisierungsaxiom für $t_{\Sec_{G,L}}$]{%
  \forall X\in \Fib_G[A]\,\exists! a\,(a\in X \land a\in L) \dsep x\in A
  \vdash t_{\Sec_{G,L}}(x)\in B%
}{
  \DeltaRow{Mengen}{A\dsep B\dsep L\dsep X\dsep a\dsep b\dsep x}
  \DeltaRow{Surjektive Funktion}{G\colon B\sur A}
  \DeltaRow{Funktionensymbol}{t_{\Sec_{G,L}}}
}
\begin{tabproof}
  \proofstep{1}{\forall X\in \Fib_G[A]\,\exists! a\,(a\in X \land a\in L)}{\rA}
  \proofstep{2}{x\in A}{\rA}

  \proofstep{1,2}{t_{\Sec_{G,L}}(x)\in \Fib_G(x)}{%
    \rAEa{\FormulaRefAuto{%
      \forall X\in \Fib_G[A]\,\exists! a\,(a\in X \land a\in L) \dsep x\in A
      \vdash t_{\Sec_{G,L}}(x)\coloneqq \iota b\bigl(b\in \Fib_G(x)\land b\in L\bigr)}{1,2}}}

  \proofstep{1,2}{t_{\Sec_{G,L}}(x)\in B}{%
    \FormulaRefAuto{x\in A \dsep b\in \Fib_G(x) \vdash b\in B}{2,3}}
\end{tabproof}

\FormulaThmDelta[Grundgleichung des Auswahlterms]{%
  \forall X\in \Fib_G[A]\,\exists! a\,(a\in X \land a\in L) \dsep x\in A
  \vdash G\bigl(t_{\Sec_{G,L}}(x)\bigr)=x%
}{
  \DeltaRow{Mengen}{A\dsep B\dsep L\dsep X\dsep a\dsep b\dsep x}
  \DeltaRow{Surjektive Funktion}{G\colon B\sur A}
  \DeltaRow{Funktionensymbol}{t_{\Sec_{G,L}}}
}
\begin{tabproof}
  \proofstep{1}{\forall X\in \Fib_G[A]\,\exists! a\,(a\in X \land a\in L)}{\rA}
  \proofstep{2}{x\in A}{\rA}

  \proofstep{1,2}{t_{\Sec_{G,L}}(x)\in \Fib_G(x)}{%
    \rAEa{\FormulaRefAuto{%
      \forall X\in \Fib_G[A]\,\exists! a\,(a\in X \land a\in L) \dsep x\in A
      \vdash t_{\Sec_{G,L}}(x)\coloneqq \iota b\bigl(b\in \Fib_G(x)\land b\in L\bigr)}{1,2}}}

  \proofstep{1,2}{G\bigl(t_{\Sec_{G,L}}(x)\bigr)=x}{%
    \FormulaRefAuto{x\in A \dsep b\in \Fib_G(x) \vdash G(b)=x}{2,3}}
\end{tabproof}

\subsubsection{Die Sektion als Funktion}

\FormulaThmDelta[Sektion als Funktion]{%
  \forall X\in \Fib_G[A]\,\exists! a\,(a\in X \land a\in L)
  \vdash \exists! F\colon A\to B\,\forall x\in A\,F(x)=t_{\Sec_{G,L}}(x)%
}{
  \DeltaRow{Mengen}{A\dsep B\dsep L\dsep X\dsep a\dsep x}
  \DeltaRow{Surjektive Funktion}{G\colon B\sur A}
  \DeltaRow{Funktionensymbol}{t_{\Sec_{G,L}}}
}
\begin{tabproof}
  \proofstep{1}{\forall X\in \Fib_G[A]\,\exists! a\,(a\in X \land a\in L)}{\rA}

  \proofstep{1}{%
    \forall x\in A\,\Bigl(t_{\Sec_{G,L}}(x)=\iota b\bigl(b\in \Fib_G(x)\land b\in L\bigr)\Bigr)}{%
    \FormulaRefAuto{%
      \forall X\in \Fib_G[A]\,\exists! a\,(a\in X \land a\in L) \dsep x\in A
      \vdash t_{\Sec_{G,L}}(x)\coloneqq \iota b\bigl(b\in \Fib_G(x)\land b\in L\bigr)}{1}}

  \proofstep{1}{\forall x\in A\,t_{\Sec_{G,L}}(x)\in B}{%
    \FormulaRefAuto{%
      \forall X\in \Fib_G[A]\,\exists! a\,(a\in X \land a\in L) \dsep x\in A
      \vdash t_{\Sec_{G,L}}(x)\in B}{1}}

  \proofstep{1}{\exists! F\colon A\to B\,\forall x\in A\,F(x)=t_{\Sec_{G,L}}(x)}{%
    \FormulaRefAuto{%
      x\in A\vdash t(x):=y \dsep x\in A\vdash t(x)\in B
      \exists! F\colon A\to B\,\forall x\in A\,F(x)=t(x)}{2,3}}
\end{tabproof}

\FormulaDefDeltaR{%
  \begin{aligned}[t]
    &\forall X\in \Fib_G[A]\,\exists! a\,(a\in X \land a\in L)\\
    &\vdash\ \Sec_{G,L}\coloneqq \iota F\Bigl(F\colon A\to B \land \forall x\in A\,F(x)=t_{\Sec_{G,L}}(x)\Bigr)
  \end{aligned}%
}{%
  \forall X\in \Fib_G[A]\,\exists! a\,(a\in X \land a\in L)
  \vdash \Sec_{G,L}\coloneqq \iota F\left(F\colon A\to B \land \forall x\in A\,F(x)=t_{\Sec_{G,L}}(x)\right)%
}{%
  \DeltaRow{Mengen}{A\dsep B\dsep L}%
  \DeltaRow{Surjektive Funktion}{G\colon B\sur A}%
  \DeltaRow{Funktionensymbol}{t_{\Sec_{G,L}}}%
}


\FormulaThmDelta[Sektion ist eine Funktion]{%
  \forall X\in \Fib_G[A]\,\exists! a\,(a\in X \land a\in L)
  \vdash \Sec_{G,L}\colon A\to B%
}{
  \DeltaRow{Mengen}{A\dsep B\dsep L\dsep X\dsep a}
  \DeltaRow{Surjektive Funktion}{G\colon B\sur A}
}
\begin{tabproof}
  \proofstep{1}{\forall X\in \Fib_G[A]\,\exists! a\,(a\in X \land a\in L)}{\rA}
  \proofstep{1}{\Sec_{G,L}\colon A\to B}{%
    \rAEa{\FormulaRefAuto{%
      \forall X\in \Fib_G[A]\,\exists! a\,(a\in X \land a\in L)
      \vdash \Sec_{G,L}\coloneqq \iota F\left(F\colon A\to B \land \forall x\in A\,F(x)=t_{\Sec_{G,L}}(x)\right)}{1}}}
\end{tabproof}

\FormulaThmDelta[Grundgleichung der Sektion]{%
  \forall X\in \Fib_G[A]\,\exists! a\,(a\in X \land a\in L) \dsep x\in A
  \vdash \Sec_{G,L}(x)=t_{\Sec_{G,L}}(x)%
}{
  \DeltaRow{Mengen}{A\dsep B\dsep L\dsep X\dsep a\dsep x}
  \DeltaRow{Surjektive Funktion}{G\colon B\sur A}
  \DeltaRow{Funktionensymbol}{t_{\Sec_{G,L}}}
}
\begin{tabproof}
  \proofstep{1}{\forall X\in \Fib_G[A]\,\exists! a\,(a\in X \land a\in L)}{\rA}
  \proofstep{2}{x\in A}{\rA}

  \proofstep{1}{%
    \forall x\in A\,\Sec_{G,L}(x)=t_{\Sec_{G,L}}(x)}{%
    \rAEb{\FormulaRefAuto{%
      \forall X\in \Fib_G[A]\,\exists! a\,(a\in X \land a\in L)
      \vdash \Sec_{G,L}\coloneqq \iota F\left(F\colon A\to B \land \forall x\in A\,F(x)=t_{\Sec_{G,L}}(x)\right)}{1}}}

  \proofstep{1,2}{\Sec_{G,L}(x)=t_{\Sec_{G,L}}(x)}{\rRE{2,\rUE{3}}}
\end{tabproof}

\FormulaThmDelta[Sektionsgleichung]{%
  \forall X\in \Fib_G[A]\,\exists! a\,(a\in X \land a\in L) \dsep x\in A
  \vdash G\bigl(\Sec_{G,L}(x)\bigr)=x%
}{
  \DeltaRow{Mengen}{A\dsep B\dsep L\dsep X\dsep a\dsep x}
  \DeltaRow{Surjektive Funktion}{G\colon B\sur A}
  \DeltaRow{Funktionensymbol}{t_{\Sec_{G,L}}}
}
\begin{tabproof}
  \proofstep{1}{\forall X\in \Fib_G[A]\,\exists! a\,(a\in X \land a\in L)}{\rA}
  \proofstep{2}{x\in A}{\rA}

  \proofstep{1,2}{t_{\Sec_{G,L}}(x)\in \Fib_G(x)}{%
    \rAEa{\FormulaRefAuto{%
      \forall X\in \Fib_G[A]\,\exists! a\,(a\in X \land a\in L) \dsep x\in A
      \vdash t_{\Sec_{G,L}}(x)\coloneqq \iota b\bigl(b\in \Fib_G(x)\land b\in L\bigr)}{1,2}}}

  \proofstep{1,2}{\Sec_{G,L}(x)=t_{\Sec_{G,L}}(x)}{%
    \FormulaRefAuto{%
      \forall X\in \Fib_G[A]\,\exists! a\,(a\in X \land a\in L) \dsep x\in A
      \vdash \Sec_{G,L}(x)=t_{\Sec_{G,L}}(x)}{1,2}}

  \proofstep{1,2}{\Sec_{G,L}(x)\in \Fib_G(x)}{\rIE{4,3}}

  \proofstep{1,2}{G\bigl(\Sec_{G,L}(x)\bigr)=x}{%
    \FormulaRefAuto{x\in A \dsep b\in \Fib_G(x) \vdash G(b)=x}{2,5}}
\end{tabproof}

\FormulaThmDelta[Auswahlmenge liefert Rechtsinverse]{%
  \forall X\in \Fib_G[A]\,\exists! a\,(a\in X \land a\in L)
  \vdash \exists F\colon A\to B\,\bigl(G\circ F=\Id_A\bigr)%
}{
  \DeltaRow{Mengen}{A\dsep B\dsep L\dsep X\dsep a\dsep x}
  \DeltaRow{Surjektive Funktion}{G\colon B\sur A}
}
\begin{tabproof}
  \proofstep{1}{\forall X\in \Fib_G[A]\,\exists! a\,(a\in X \land a\in L)}{\rA}

  \proofstep{1}{\Sec_{G,L}\colon A\to B}{%
    \FormulaRefAuto{%
      \forall X\in \Fib_G[A]\,\exists! a\,(a\in X \land a\in L)
      \vdash \Sec_{G,L}\colon A\to B}{1}}

  \proofstep{3}{x\in A}{\rA}

  \proofstep{1,3}{\bigl(G\circ \Sec_{G,L}\bigr)(x)=G\bigl(\Sec_{G,L}(x)\bigr)}{%
    \FormulaRefAuto{x\in A \vdash (G\circ F)(x)=G(F(x))}{2,3}}

  \proofstep{1,3}{G\bigl(\Sec_{G,L}(x)\bigr)=x}{%
    \FormulaRefAuto{%
      \forall X\in \Fib_G[A]\,\exists! a\,(a\in X \land a\in L) \dsep x\in A
      \vdash G\bigl(\Sec_{G,L}(x)\bigr)=x}{1,3}}

  \proofstep{1,3}{\bigl(G\circ \Sec_{G,L}\bigr)(x)=x}{%
    \FormulaRefAuto{a=b,\, b=c \vdash a=c}{4,5}}

  \proofstep{3}{\Id_A(x)=x}{\FormulaRefAuto{x\in A\vdash \Id_A(x)=x}{3}}
  \proofstep{3}{x=\Id_A(x)}{\FormulaRefAuto{a=b\vdash b=a}{7}}

  \proofstep{1,3}{\bigl(G\circ \Sec_{G,L}\bigr)(x)=\Id_A(x)}{%
    \FormulaRefAuto{a=b,\, b=c \vdash a=c}{6,8}}

  \proofstep{1}{x\in A\rightarrow \bigl(G\circ \Sec_{G,L}\bigr)(x)=\Id_A(x)}{\rRI{3,9}}
  \proofstep{1}{\forall x\in A\,\bigl(G\circ \Sec_{G,L}\bigr)(x)=\Id_A(x)}{\rUI{10}}

  \proofstep{1}{G\circ \Sec_{G,L}=\Id_A}{%
    \FormulaRefAuto{\forall x\in A\,(F(x)=G(x)) \eqvdash F=G}{11}}

  \proofstep{1}{\Sec_{G,L}\colon A\to B \land G\circ \Sec_{G,L}=\Id_A}{\rAI{2,12}}
  \proofstep{1}{\exists F\colon A\to B\,\bigl(G\circ F=\Id_A\bigr)}{\rEI{13}}
\end{tabproof}


% -------------------------------------------------
% 10) Hauptsatz: ACfam ⇒ ACsur (für eine gegebene Surjektion G)
% -------------------------------------------------
\FormulaThmDeltaR{%
  \begin{aligned}[t]
    &\forall M\,\Bigl(\DisjFam(M)\rightarrow
      \exists L\,\forall X\in M\,\exists! a\in X\,(a\in L)\Bigr)\\
    &\vdash\ \exists F\colon A\to B\,\bigl(G\circ F=\Id_A\bigr)
  \end{aligned}%
}{%
  \forall M\,\bigl(\DisjFam(M)\rightarrow \exists L\,\forall X\in M\,\exists! a\in X\, a\in L\bigr)
  \vdash \exists F\colon A\to B\,\bigl(G\circ F=\Id_A\bigr)%
}{%
  \DeltaRow{Mengen}{A\dsep B\dsep x}%
  \DeltaRow{Surjektive Funktion}{G\colon B\sur A}%
}

\begin{tabproof}
  \proofstep{1}{\forall M\,\bigl(\DisjFam(M)\rightarrow \exists L\,\forall X\in M\,\exists! a\in X\, a\in L\bigr)}{\rA}

  \proofstep{1}{\exists L\,\forall X\in \Fib_G[A]\,\exists! a\,(a\in X \land a\in L)}{%
    \FormulaRefAuto{\forall M\,\bigl(\DisjFam(M)\rightarrow \exists L\,\forall X\in M\,\exists! a\in X\, a\in L\bigr) \vdash \exists L\,\forall X\in \Fib_G[A]\,\exists! a\,(a\in X \land a\in L)}{1}}

  \proofstep{3}{\forall X\in \Fib_G[A]\,\exists! a\,(a\in X \land a\in L)}{\rA}
  
  \proofstep{3}{\exists F\colon A\to B\,\bigl(G\circ F=\Id_A\bigr)}{%
    \FormulaRefAuto{%
      \forall X\in \Fib_G[A]\,\exists! a\,(a\in X \land a\in L)
      \vdash \exists F\colon A\to B\,\bigl(G\circ F=\Id_A\bigr)}{3}}
  \proofstep{1}{\exists F\colon A\to B\,\bigl(G\circ F=\Id_A\bigr)}{\rEE{2,3,4}}
\end{tabproof}

\subsection{Axiomatische Festlegung}

\FormulaAxiomDeltaK[Auswahlprinzip (AC -- Surjektionsform)]{\exists F\colon A \to B\,G\circ F = \Id_A}{Auswahlaxiom}{
  \DeltaDecl{Mengen}{A \dsep B}
  \DeltaPrem{Surjektive Funktion}{G\colon B \sur A}
}

\subsection{Folgerungen}

\FormulaThmDelta[AC -- Relationsform aus der Surjektionsform]{%
  \exists F\colon A \to B\,\forall x\in A\,\bigl((x,F(x))\in R\bigr)%
}{
  \DeltaDecl{Mengen}{A \dsep B}
  \DeltaPrem{Totale Relation}{\TotRel{R,A,B}}
  \DeltaDecl{Relation}{R}
}
\begin{tabproof}
  \proofstep{}{%
    \pi_1\restriction_R\colon R\sur A%
  }{%
    \FormulaRefAuto{\pi_1\restriction_R\colon R\sur A}%
  }

  \proofstep{1}{%
    \exists s\colon A\to R\,\bigl(\pi_1\restriction_R\circ s=\Id_A\bigr)%
  }{%
    \FormulaRefAuto{\exists F\colon A \to B\,G\circ F = \Id_A}{1}%
  }

  \proofstep{2}{%
    \exists F\colon A \to B\,\forall x\in A\,\bigl((x,F(x))\in R\bigr)%
  }{%
    \FormulaRefAuto{\exists s\colon A\to R\,\bigl(\pi_1\restriction_R\circ s=\Id_A\bigr)\vdash
    \exists F\colon A\to B\,\forall x\in A\,\bigl((x,F(x))\in R\bigr)}{2}%
  }
\end{tabproof}

\FormulaThmDelta[AC -- Familienform aus der Surjektionsform]{%
  \exists L\,\forall X\in M\,\exists! a\in X\,\bigl(a\in L\bigr)%
}{
  \DeltaDecl{Mengen}{M \dsep X \dsep a \dsep L}
  \DeltaPrem{\makecell[l]{Paarweise disjunkte\\ nichtleere Familie}}{\DisjFam(M)}
}
\begin{tabproof}
  \proofstep{}{%
    \TotRel{\MemRel(M),M,\bigcup M}%
  }{%
    \FormulaRefAuto{\TotRel{\MemRel(M),M,\bigcup M}}%
  }

  \proofstep{1}{%
    \exists F\colon M\to \bigcup M\,\forall x\in M\,\bigl((x,F(x))\in \MemRel(M)\bigr)%
  }{%
    \FormulaRefAuto{\exists F\colon A \to B\,\forall x\in A\,\bigl((x,F(x))\in R\bigr)}{1}%
  }

  \proofstep{2}{%
    \exists L\,\forall X\in M\,\exists! a\in X\,\bigl(a\in L\bigr)%
  }{%
    \FormulaRefAuto{\exists F\colon M\to \bigcup M\,\forall x\in M\,\bigl((x,F(x))\in \MemRel(M)\bigr)
    \vdash \exists L\,\forall X\in M\,\exists! a\in X\,\bigl(a\in L\bigr)}{2}%
  }
\end{tabproof}

\subsubsection{Injektion aus Surjektion (via Sektion)}

% --- (1) Korollar: Surjektion + Sektion => Injektion ---
\FormulaThmDelta{%
  \exists H\colon B\inj A%
}{
  \DeltaRow{Mengen}{A\dsep B}
  \DeltaPrem{Surjektive Funktion}{F\colon A\sur B}
  \DeltaPrem{Funktion}{G\colon B\to A}
}
\begin{tabproof}
  \proofstep{1}{\exists S\colon B\to A\,\bigl(F\circ S=\Id_B\bigr)}{%
    \FormulaRefAuto{\exists F\colon A \to B\,G\circ F = \Id_A}}

  \proofstep{2}{S\colon B\to A \land F\circ S=\Id_B}{\rA}
  \proofstep{2}{F\circ S=\Id_B}{\rAEb{2}}

  \proofstep{2}{S\colon B\inj A}{%
    \FormulaRefAuto{F\circ G=\Id_B\vdash G\colon B\inj A}}

  \proofstep{2}{\exists H\colon B\inj A}{\rEI{4}}

  \proofstep{1}{\exists H\colon B\inj A}{\rEE{1,2,5}}
\end{tabproof}


\chapter{Konstruktion der natürlichen Zahlen}

Wir konstruieren die Menge der natürlichen Zahlen allein aus den Axiomen der Mengenlehre, insbesondere dem Unendlichkeitsaxiom.  Dieser Ansatz kommt ohne eine explizite Nachfolgerfunktion als primitives Symbol aus; diese wird vielmehr aus der Mengenoperation \(x \cup \{x\}\) gewonnen.

\section{Axiom der Unendlichkeit}


\FormulaAxiomAuto[Unendlichkeit]{ \exists A\;\bigl(\varnothing \in A \land \forall x \in A\,(x \cup \{x\} \in A)\bigr) }

% === Kapitel: Konstruktion der natürlichen Zahlen (umformuliert – Fokus: Funktion erster Ordnung auf A) ===

\section{Nachfolger und induktive Mengen}

\subsection{Induktive Mengen}

\FormulaDefDelta[induktive Menge]{\Induktiv(A) := \varnothing \in A \,\land\, \forall x\in A\,(x \cup \{x\} \in A)}{
\DeltaDecl{Mengen}{A}
}

% Existenz einer induktiven Menge (aus dem Unendlichkeitsaxiom)
\FormulaThmAuto{\exists A(\Induktiv(A))}
\begin{tabproof}
  \proofstep{}{ \exists A\bigl(\varnothing \in A \land \forall x \in A\,(x \cup \{x\} \in A)\bigr) }{\FormulaRefAuto{\exists A\bigl(\varnothing \in A \land \forall x \in A\,(x \cup \{x\} \in A)\bigr)}}
  \proofstep{}{\exists A(\Induktiv(A)) }{\rIE{\FormulaRefAuto{\Induktiv(A) := \varnothing \in A \,\land\, \forall x\in A\,(x \cup \{x\} \in A)},1}}
\end{tabproof}

% Sofortige Konsequenzen aus der Definition
\FormulaThmDelta{\Induktiv(A)\vdash \varnothing\in A}{
\DeltaDecl{Mengen}{A}
}
\begin{tabproof}
  \proofstep{1}{\Induktiv(A)}{\rA}
  \proofstep{1}{\varnothing \in A \land \forall x \in A\,(x \cup \{x\} \in A)}{\rIE{\FormulaRefAuto{\Induktiv(A) := \varnothing \in A \,\land\, \forall x\in A\,(x \cup \{x\} \in A)},1}}
  \proofstep{1}{\varnothing \in A}{\rAEa{2}}
\end{tabproof}

\FormulaThmDelta{\Induktiv(A)\vdash \forall x \in A\,(x \cup \{x\} \in A)}{
\DeltaDecl{Mengen}{A}
}
\begin{tabproof}
  \proofstep{1}{\Induktiv(A)}{\rA}
  \proofstep{1}{\varnothing \in A \land \forall x \in A\,(x \cup \{x\} \in A)}{\rIE{\FormulaRefAuto{\Induktiv(A) := \varnothing \in A \,\land\, \forall x\in A\,(x \cup \{x\} \in A)},1}}
  \proofstep{1}{\forall x \in A\,(x \cup \{x\} \in A)}{\rAEb{2}}
\end{tabproof}

\FormulaThmDelta{\Induktiv(A),\,x\in A\vdash x \cup \{x\} \in A}{
\DeltaDecl{Mengen}{A}
}
\begin{tabproof}
  \proofstep{1}{\Induktiv(A)}{\rA}
  \proofstep{2}{x\in A}{\rA}
  \proofstep{1}{\forall u \in A\,(u \cup \{u\} \in A)}{\FormulaRefAuto{\Induktiv(A)\vdash \forall x \in A\,(x \cup \{x\} \in A)}{1}}
  \proofstep{1,2}{x \cup \{x\} \in A}{\rRE{\rUE{3},2}}
\end{tabproof}

% Schnitt induktiver Mengen ist induktiv
\FormulaThmDelta{\Induktiv(A),\, \Induktiv(B)\vdash \Induktiv(A\cap B)}{
\DeltaDecl{Mengen}{A\dsep B}
}
\begin{tabproof}
  \proofstep{1}{\Induktiv(A)}{\rA}
  \proofstep{2}{\Induktiv(B)}{\rA}

  % Null in A∩B
  \proofstep{1}{\,\varnothing\in A}{\FormulaRefAuto{\Induktiv(A)\vdash \varnothing\in A}{1}}
  \proofstep{2}{\,\varnothing\in B}{\FormulaRefAuto{\Induktiv(A)\vdash \varnothing\in A}{2}}
  \proofstep{1,2}{\,\varnothing\in A\cap B}{\FormulaRefAuto{x\in A,\, x\in B \vdash x\in A\cap B}{3,4}}

  % Abschluss unter Nachfolger-Operation für A∩B
  \proofstep{1}{\forall x\in A\,(x\cup\{x\}\in A)}{\FormulaRefAuto{\Induktiv(A)\vdash \forall x \in A\,(x \cup \{x\} \in A)}{1}}
  \proofstep{2}{\forall x\in B\,(x\cup\{x\}\in B)}{\FormulaRefAuto{\Induktiv(A)\vdash \forall x \in A\,(x \cup \{x\} \in A)}{2}}
  \proofstep{8}{x\in A\cap B}{\rA}
  \proofstep{8}{x\in A}{\FormulaRefAuto{x\in A\cap B \vdash x\in A}{9}}
  \proofstep{8}{x\in B}{\FormulaRefAuto{x\in A\cap B \vdash x\in B}{9}}
  \proofstep{1,8}{x\cup\{x\}\in A}{\rRE{\rUE{6},9}}
  \proofstep{2,8}{x\cup\{x\}\in B}{\rRE{\rUE{7},10}}
  \proofstep{1,2,8}{x\cup\{x\}\in A\cap B}{\FormulaRefAuto{x\in A,\, x\in B \vdash x\in A\cap B}{11,12}}
  \proofstep{1,2}{\,\forall x\in A\cap B\,(x\cup\{x\}\in A\cap B)}{\rUI{\rRI{8,13}}}

  % Schluss: Induktivität von A∩B
  \proofstep{1,2}{\,\varnothing\in A\cap B \;\land\; \forall x\in A\cap B\,(x\cup\{x\}\in A\cap B)}{\rAI{5,14}}
  \proofstep{1,2}{\,\Induktiv(A\cap B)}%
    {\rIE{\FormulaRefAuto{\Induktiv(A) := \varnothing \in A \,\land\, \forall x\in A\,(x \cup \{x\} \in A)},15}}
\end{tabproof}

\subsection{Die Nachfolger-Funktion}

% ============================================================
% Nachfolger-Funktion (analog zu den Beispielen von Funktionen)
% Voraussetzung: Induktiv(A) wurde bereits definiert.
% ============================================================

\subsubsection{Definition der Nachfolgerfunktion}

\FormulaDefDelta[Funktionsvorschrift für die Nachfolgerfunktion auf \(A\)]%
{x\in A \vdash t_{\Succ_A}(x) \coloneqq x\cup\{x\}}%
{
  \DeltaRow{Mengen}{A\dsep x}
  \DeltaRow{Funktionensymbol}{t_{\Succ_A}}
}

\FormulaThmDelta[Typisierungsaxiom für \(t_{\Succ_A}\)]%
{\Induktiv(A)\dsep x\in A \vdash t_{\Succ_A}(x)\in A}%
{
  \DeltaRow{Mengen}{A\dsep x}
  \DeltaRow{Funktionensymbol}{t_{\Succ_A}}
}
\begin{tabproof}
  \proofstep{1}{\Induktiv(A)}{\rA}
  \proofstep{2}{x\in A}{\rA}
  \proofstep{1,2}{x\cup\{x\}\in A}{%
    \FormulaRefAuto{\Induktiv(A),\,x\in A\vdash x \cup \{x\} \in A}{1,2}}
  \proofstep{2}{t_{\Succ_A}(x)=x\cup\{x\}}{%
    \FormulaRefAuto{x\in A \vdash t_{\Succ_A}(x) \coloneqq x\cup\{x\}}{2}}
  \proofstep{1,2}{t_{\Succ_A}(x)\in A}{\rIE{4,3}}
\end{tabproof}

\FormulaThmDelta[Nachfolgerfunktion als Funktion]%
{\Induktiv(A)\vdash \exists! F\colon A\to A\,\forall x\in A\,F(x)=t_{\Succ_A}(x)}%
{
  \DeltaRow{Mengen}{A\dsep x}
  \DeltaRow{Funktionensymbol}{t_{\Succ_A}}
}
\begin{tabproof}
  \proofstep{1}{\Induktiv(A)}{\rA}
  \proofstep{}{%
    \forall x\in A\,t_{\Succ_A}(x)=x\cup\{x\}}{%
    \FormulaRefAuto{x\in A \vdash t_{\Succ_A}(x) \coloneqq x\cup\{x\}}}
  \proofstep{1}{%
    \forall x\in A\,t_{\Succ_A}(x)\in A}{%
    \rUI{\rRI{2,\FormulaRefAuto{\Induktiv(A)\dsep x\in A \vdash t_{\Succ_A}(x)\in A}{1,2}}}}
  \proofstep{1}{%
    \exists! F\colon A\to A\,\forall x\in A\,F(x)=t_{\Succ_A}(x)}{%
    \FormulaRefAuto{x\in A\vdash t(x):=y\dsep x\in A\vdash t(x)\in B\exists! F\colon A\to B\,\forall x\in A\,F(x) = t(x)}{2,3}}
\end{tabproof}

\FormulaDefDelta[Nachfolgerfunktion auf \(A\)]%
{\Induktiv(A)\vdash \Succ_A\coloneqq\iota F\Bigl(F\colon A\to A \land \forall x\in A\,F(x)=t_{\Succ_A}(x)\Bigr)}%
{
  \DeltaRow{Mengen}{A}
  \DeltaRow{Funktionensymbol}{t_{\Succ_A}}
}

\FormulaThmDelta[Nachfolgerfunktion auf \(A\)]%
{\Induktiv(A)\vdash \Succ_A\colon A\to A}%
{
  \DeltaRow{Mengen}{A}
}
\begin{tabproof}
  \proofstep{1}{\Induktiv(A)}{\rA}
  \proofstep{1}{\Succ_A\colon A\to A}{%
    \rAEa{\FormulaRefAuto{\Induktiv(A)\vdash \Succ_A\coloneqq\iota F\Bigl(F\colon A\to A \land \forall x\in A\,F(x)=t_{\Succ_A}(x)\Bigr)}{1}}}
\end{tabproof}

\FormulaThmDelta%
{\Induktiv(A)\vdash \TotRel{\Succ_A,A,A}}%
{
  \DeltaRow{Mengen}{A}
}
\begin{tabproof}
  \proofstep{1}{\Induktiv(A)}{\rA}
  \proofstep{1}{\Succ_A\colon A\to A}{%
    \FormulaRefAuto{\Induktiv(A)\vdash \Succ_A\colon A\to A}{1}}
  \proofstep{1}{\TotRel{\Succ_A,A,A}}{%
    \FormulaRefAuto{F\colon A\to B\vdash\TotRel{F,A,B}}{2}}
\end{tabproof}

\FormulaThmDelta[Grundgleichung der Nachfolgerfunktion]%
{\Induktiv(A)\dsep x\in A \vdash \Succ_A(x)=x\cup\{x\}}%
{
  \DeltaRow{Mengen}{A\dsep x}
  \DeltaRow{Funktionensymbol}{t_{\Succ_A}}
}
\begin{tabproof}
  \proofstep{1}{\Induktiv(A)}{\rA}
  \proofstep{2}{x\in A}{\rA}

  \proofstep{1}{%
    \forall x\in A\,\Succ_A(x)=t_{\Succ_A}(x)}{%
    \rAEb{\FormulaRefAuto{\Induktiv(A)\vdash \Succ_A\coloneqq\iota F\Bigl(F\colon A\to A \land \forall x\in A\,F(x)=t_{\Succ_A}(x)\Bigr)}{1}}}

  \proofstep{1,2}{\Succ_A(x)=t_{\Succ_A}(x)}{\rRE{2,\rUE{3}}}

  \proofstep{2}{t_{\Succ_A}(x)=x\cup\{x\}}{%
    \FormulaRefAuto{x\in A \vdash t_{\Succ_A}(x) \coloneqq x\cup\{x\}}{2}}

  \proofstep{1,2}{\Succ_A(x)=x\cup\{x\}}{%
    \FormulaRefAuto{a = b,\, b = c \vdash a = c}{4,5}}
\end{tabproof}

\FormulaThmDelta[Bild der Nachfolgerfunktion liegt in \(A\)]{%
  \Induktiv(A)\dsep x\in A \vdash \Succ_A(x)\in A%
}{
  \DeltaDecl{Mengen}{A\dsep x}
}
\begin{tabproof}
  \proofstep{1}{\Induktiv(A)}{\rA}
  \proofstep{2}{x\in A}{\rA}

  \proofstep{1,2}{x\cup\{x\}\in A}{%
    \FormulaRefAuto{\Induktiv(A),\,x\in A\vdash x \cup \{x\} \in A}{1,2}}

  \proofstep{1,2}{\Succ_A(x)=x\cup\{x\}}{%
    \FormulaRefAuto{\Induktiv(A)\dsep x\in A \vdash \Succ_A(x)=x\cup\{x\}}{1,2}}

  \proofstep{1,2}{\Succ_A(x)\in A}{\rIE{4,3}}
\end{tabproof}


\section{Konstruktion der natürlichen Zahlen}

\FormulaThmAuto{
  \exists! C\; \forall B \Bigl(\Induktiv(B) \rightarrow 
    C = \{ x \in B \mid \forall A (\Induktiv(A) \rightarrow x \in A) \}\Bigr)
}
\begin{tabproofwide}
  % Schritt 1: Existenz einer induktiven Menge
  \proofstepwidestar[]{\exists A(\Induktiv(A))}%
    {\FormulaRefAuto{\exists A(\Induktiv(A))}}

  % Schritt 2: Einzigartige Existenz des Schnitts aller induktiven Mengen
  \proofstepwide{}{}{\exists! C\; \forall B(\Induktiv(B)\rightarrow }%
    {\multirow{2}{*}{\FormulaRefAuto{P(D)\vdash \exists! C\forall B(P(B)\rightarrow C= \{ x \in B \mid \forall A (P(A) \rightarrow x \in A) \})}{1}}}
  \proofstepwide{}{}%
    {C=\{x\in B \mid \forall A (\Induktiv(A)\rightarrow x\in A)\})}%
    {}
\end{tabproofwide}


\FormulaDefAuto[Menge der natürlichen Zahlen]{\mathbb{N} := \bigcap \{ A \mid \Induktiv(A) \}}

\begin{remark}
  Die Definition sagt: Ein Element \(x\) gehört genau dann zu \(\mathbb{N}\), wenn es zu jeder induktiven Menge gehört.  Damit ist \(\mathbb{N}\) die \emph{kleinste} induktive Menge, denn für jede andere induktive Menge \(B\) gilt \(\mathbb{N} \subseteq B\).
\end{remark}

\FormulaThmAuto{x\in\mathbb{N}\eqvdash \forall \Induktiv(N)(x\in N)}
\begin{tabproofwide}
    \proofstepwide[]{x\in \mathbb{N}}{\leftrightarrow}{x\in \bigcap \{ A \mid \Induktiv(A) \}}{\FormulaRefAuto{\mathbb{N} := \bigcap \{ A \mid \Induktiv(A) \}}}
    \proofstepwide[]{}{\leftrightarrow}{\forall \Induktiv(N)(x\in N)}{\FormulaRefAuto{\exists A(P(A)) \vdash x \in \bigcap_{P(B)} B \leftrightarrow \forall C\, (P(C) \rightarrow x \in C)}{\FormulaRefAuto{\exists A(\Induktiv(A))}}}
\end{tabproofwide}

\section{Eigenschaften der Menge der natürlichen Zahlen}

Im Folgenden werden wir einige grundlegende Eigenschaften der so konstruierten Menge \(\mathbb{N}\) beweisen.

\subsection{Induktivität der Menge der natürlichen Zahlen}


\FormulaThmAuto[Null-Axiom]{\varnothing\in\mathbb{N}}
\begin{tabproof}
    \proofstep{1}{\Induktiv(N)}{\rA}
    \proofstep{1}{\varnothing\in N}{\FormulaRefAuto{\Induktiv(A)\vdash \varnothing\in A}{1}}
    \proofstep{}{\forall \Induktiv(N)(x\in N)}{\rUI{\rRI{1,2}}}
    \proofstep{}{\varnothing\in\mathbb{N}}{\FormulaRefAuto{x\in\mathbb{N}\eqvdash \forall \Induktiv(N)(x\in N)}{3}}
\end{tabproof}

\FormulaThmAuto[Nachfolger-Axiom]{\forall x \in \mathbb{N}\,(x\cup\{x\} \in \mathbb{N})}
\begin{tabproof}
  \proofstep{1}{x \in \mathbb{N}}{\rA}

  \proofstep{1}{\forall \Induktiv(N)(x\in N)}{%
    \FormulaRefAuto{x\in\mathbb{N}\eqvdash \forall \Induktiv(N)(x\in N)}{1}}

  \proofstep{3}{\Induktiv(N)}{\rA}

  \proofstep{1}{\Induktiv(N)\rightarrow x\in N}{\rUE{2}}
  \proofstep{1,3}{x\in N}{\rRE{4,3}}

  \proofstep{1,3}{x\cup\{x\}\in N}{%
    \FormulaRefAuto{\Induktiv(A),\,x\in A\vdash x \cup \{x\} \in A}{3,5}}

  \proofstep{1}{\Induktiv(N)\rightarrow x\cup\{x\}\in N}{\rRI{3,6}}
  \proofstep{1}{\forall \Induktiv(N)(x\cup\{x\}\in N)}{\rUI{7}}

  \proofstep{1}{x\cup\{x\}\in \mathbb{N}}{%
    \FormulaRefAuto{x\in\mathbb{N}\eqvdash \forall \Induktiv(N)(x\in N)}{8}}

  \proofstep{}{\,\forall x\in \mathbb{N}\,(x\cup\{x\}\in \mathbb{N})}{\rUI{\rRI{1,9}}}
\end{tabproof}


\FormulaThmAuto{\Induktiv(\mathbb{N})}
\begin{tabproof}
  \proofstep{}{\,\varnothing\in\mathbb{N}}{%
    \FormulaRefAuto{\varnothing\in\mathbb{N}}}

  \proofstep{}{\,\forall x\in\mathbb{N}\,(x\cup\{x\}\in\mathbb{N})}{%
    \FormulaRefAuto{\forall x \in \mathbb{N}\,(x\cup\{x\} \in \mathbb{N})}}

  \proofstep{}{\,\varnothing\in\mathbb{N} \;\land\; \forall x\in\mathbb{N}\,(x\cup\{x\}\in\mathbb{N})}{%
    \rAI{1,2}}

  \proofstep{}{\,\Induktiv(\mathbb{N})}{%
    \rIE{\FormulaRefAuto{\Induktiv(A) := \varnothing \in A \,\land\, \forall x\in A\,(x \cup \{x\} \in A)},3}}
\end{tabproof}


\subsection{Die Nachfolgerfunktion}

\FormulaDefAuto[Nachfolgerfunktion]{%
  \Succ \coloneqq \Succ_{\mathbb{N}}%
}

\FormulaThmAuto{%
  \Succ\colon \mathbb{N}\to \mathbb{N}%
}
\begin{tabproof}
  \proofstep{}{\,\Induktiv(\mathbb{N})}{%
    \FormulaRefAuto{\Induktiv(\mathbb{N})}}
  \proofstep{}{\,\Succ_{\mathbb{N}}\colon \mathbb{N}\to \mathbb{N}}{%
    \FormulaRefAuto{\Induktiv(A)\vdash \Succ_A\colon A\to A}}
  \proofstep{}{\,\Succ=\Succ_{\mathbb{N}}}{%
    \FormulaRefAuto{\Succ \coloneqq \Succ_{\mathbb{N}}}}
  \proofstep{}{\,\Succ\colon \mathbb{N}\to \mathbb{N}}{\rIE{3,2}}
\end{tabproof}

\FormulaThmDelta{%
  n\in\mathbb{N}\vdash \Succ(n)\in\mathbb{N}%
}{%
  \DeltaDecl{Mengen}{n}%
}
\begin{tabproof}
  \proofstep{1}{n\in\mathbb{N}}{\rA}
  \proofstep{}{\,\Succ\colon \mathbb{N}\to \mathbb{N}}{%
    \FormulaRefAuto{\Succ\colon \mathbb{N}\to \mathbb{N}}}
  \proofstep{1}{\,\Succ(n)\in \mathbb{N}}{%
    \FormulaRefAuto{x\in A\vdash F(x)\in B}{1,2}}
\end{tabproof}


\FormulaThmDelta{%
  x\in\mathbb{N}\vdash \Succ(x)=x\cup\{x\}%
}{
  \DeltaDecl{Mengen}{x}
}
\begin{tabproof}
  \proofstep{1}{x\in\mathbb{N}}{\rA}
  \proofstep{}{\,\Induktiv(\mathbb{N})}{%
    \FormulaRefAuto{\Induktiv(\mathbb{N})}}
  \proofstep{1}{\,\Succ_{\mathbb{N}}(x)=x\cup\{x\}}{%
    \FormulaRefAuto{\Induktiv(A)\dsep x\in A \vdash \Succ_A(x)=x\cup\{x\}}{2,1}}
  \proofstep{}{\,\Succ=\Succ_{\mathbb{N}}}{%
    \FormulaRefAuto{\Succ \coloneqq \Succ_{\mathbb{N}}}}
  \proofstep{1}{\,\Succ(x)=x\cup\{x\}}{\rIE{4,3}}
\end{tabproof}

\FormulaThmDelta{%
  n\in\mathbb{N}\dsep x\in\Succ(n)\vdash x\in n\lor x=n%
}{%
  \DeltaDecl{Mengen}{n\dsep x}%
}
\begin{tabproof}
  \proofstep{1}{n\in\mathbb{N}}{\rA}
  \proofstep{2}{x\in\Succ(n)}{\rA}
  \proofstep{1}{\Succ(n)=n\cup\{n\}}{%
    \FormulaRefAuto{n\in\mathbb{N}\vdash \Succ(n) = n\cup\{n\}}{1}}
  \proofstep{1,2}{x\in n\cup\{n\}}{\rIE{3,2}}
  \proofstep{1,2}{x\in n\lor x\in\{n\}}{%
    \FormulaRefAuto{z \in A \cup B \eqvdash z \in A \lor z \in B}{2}}

  \proofstep{3}{x\in n}{\rA}
  \proofstep{3}{x\in n\lor x=n}{\rOIa{6}}

  \proofstep{4}{x\in\{n\}}{\rA}
  \proofstep{4}{x=n}{\FormulaRefAuto{x \in \{a\} \eqvdash x = a}{4}}
  \proofstep{4}{x\in n\lor x=n}{\rOIb{9}}

  \proofstep{1,2}{x\in n\lor x=n}{\rOE{5,6,7,8,10}}
\end{tabproof}




\subsection{Minimalität der Menge der natürlichen Zahlen}

\FormulaThmDelta[Minimalität]{\Induktiv(A) \vdash \mathbb{N} \subseteq A}{\DeltaDecl{Mengen}{A}}
\begin{tabproof}
  \proofstep{1}{ \Induktiv(A) }{\rA}
  \proofstep{2}{ \mathbb{N} \subseteq A }{\FormulaRefAuto{P(C)\vdash \bigcap_{P(A)} A \subseteq C}{1}}
\end{tabproof}


\FormulaThmDelta{%
  n\in\mathbb{N}\vdash n\subseteq \Succ(n)%
}{
  \DeltaDecl{Mengen}{n}
}
\begin{tabproof}
  \proofstep{1}{n\in\mathbb{N}}{\rA}

  \proofstep{1}{\Succ(n)=n\cup\{n\}}{%
    \FormulaRefAuto{x\in\mathbb{N}\vdash \Succ(x)=x\cup\{x\}}{1}}

  \proofstep{}{\,n\subseteq n\cup\{n\}}{%
    \FormulaRefAuto{A\subseteq A\cup B}}

  \proofstep{1}{n\subseteq \Succ(n)}{\rIE{2,3}}
\end{tabproof}


\section{Herleitung der Peano-Axiome}

Die folgenden Abschnitte zeigen, wie sich die Peano-Axiome aus den zuvor eingeführten Definitionen und Sätzen ergeben.

\subsection{Das Null-Axiom}

Aus \FormulaRefAuto{\varnothing\in\mathbb{N}} folgt, dass die Null in den natürlichen Zahlen enthalten ist.

\subsection{Nachfolger‑Axiom}

\FormulaThmAuto[Nachfolger-Axiom]{%
  \forall x\in \mathbb{N}\,\Succ(x)\in \mathbb{N}%
}
\begin{tabproof}
  \proofstep{1}{x\in \mathbb{N}}{\rA}

  \proofstep{}{\,\Succ\colon \mathbb{N}\to \mathbb{N}}{%
    \FormulaRefAuto{\Succ\colon \mathbb{N}\to \mathbb{N}}}

  \proofstep{1}{\,\Succ(x)\in \mathbb{N}}{%
    \FormulaRefAuto{x\in A\vdash F(x)\in B}{1,2}}

  \proofstep{}{\,\forall x\in \mathbb{N}\,(\Succ(x)\in \mathbb{N})}{%
    \rUI{\rRI{1,3}}}
\end{tabproof}


\subsection{Kein Vorgänger von Null}

\FormulaThmDelta{x\cup\{x\}\neq \varnothing}{
  \DeltaRow{Mengen}{x}
}
\begin{tabproof}
  \proofstep{1}{x\in \{x\}}{%
    \FormulaRefAuto{x\in \{x\}}}

  \proofstep{1}{x\in x\cup \{x\}}{%
    \FormulaRefAuto{u\in B \vdash u\in A\cup B}{1}}

  \proofstep{}{x\cup\{x\}\neq \varnothing}{%
    \FormulaRefAuto{u\in A \vdash A\neq \varnothing}{2}}
\end{tabproof}


\FormulaThmDelta{%
  x\in\mathbb{N}\vdash \Succ(x) \neq \varnothing%
}{
  \DeltaRow{Mengen}{x}
}

\begin{tabproof}
  \proofstep{1}{x\in\mathbb{N}}{\rA}
  \proofstep{}{x\in \{x\}}{%
    \FormulaRefAuto{x\in \{x\}}}
  \proofstep{}{x\in x\cup \{x\}}{%
    \FormulaRefAuto{u\in B \vdash u\in A\cup B}{2}}
  \proofstep{}{\,\Succ(x)=x\cup \{x\}}{%
    \FormulaRefAuto{x\in\mathbb{N}\vdash \Succ(x)=x\cup\{x\}}}
  \proofstep{1}{x\in \Succ(x)}{\rIE{3,2}}
  \proofstep{}{\,\Succ(x)\neq \varnothing}{%
    \FormulaRefAuto{u\in A \vdash A\neq \varnothing}{4}}
\end{tabproof}

\subsection{Injektivität der Nachfolgerfunktion}

\FormulaThmAuto{%
  x\in\mathbb{N}\dsep y\in\mathbb{N}\dsep \Succ(x)=\Succ(y)\vdash x=y%
}
\begin{tabproof}
  \proofstep{1}{x\in\mathbb{N}}{\rA}
  \proofstep{2}{y\in\mathbb{N}}{\rA}
  \proofstep{3}{\Succ(x)=\Succ(y)}{\rA}

  \proofstep{1}{\Succ(x)=x\cup\{x\}}{%
    \FormulaRefAuto{x\in\mathbb{N}\vdash \Succ(x)=x\cup\{x\}}{1}}
  \proofstep{2}{\Succ(y)=y\cup\{y\}}{%
    \FormulaRefAuto{x\in\mathbb{N}\vdash \Succ(x)=x\cup\{x\}}{2}}

  \proofstep{1}{x\cup\{x\}=\Succ(x)}{%
    \FormulaRefAuto{a=b\vdash b=a}{4}}
  \proofstep{1,3}{x\cup\{x\}=\Succ(y)}{%
    \FormulaRefAuto{a=b,\,b=c\vdash a=c}{6,3}}
  \proofstep{1,2,3}{x\cup\{x\}=y\cup\{y\}}{%
    \FormulaRefAuto{a=b,\,b=c\vdash a=c}{7,5}}

  \proofstep{1,2,3}{x=y}{%
    \FormulaRefAuto{a\cup\{a\}=b\cup\{b\}\eqvdash a=b}{8}}
\end{tabproof}


\subsection{Induktionsprinzip}

\FormulaThmAuto{%
  P(\varnothing),\, \forall x\in\mathbb{N}\,(P(x) \rightarrow P(\Succ(x)))
  \vdash \Induktiv(\{x \in \mathbb{N} \mid P(x)\})%
}
\begin{tabproof}
  \proofstep{1}{P(\varnothing)}{\rA}
  \proofstep{2}{\forall x\in\mathbb{N}\,(P(x)\rightarrow P(\Succ(x)))}{\rA}

  \proofstep{}{\,\varnothing\in\mathbb{N}}{%
    \FormulaRefAuto{\varnothing\in\mathbb{N}}}

  \proofstep{1}{\,\varnothing\in\mathbb{N}\land P(\varnothing)}{\rAI{3,1}}

  \proofstep{1}{\,\varnothing\in \{x\in\mathbb{N}\mid P(x)\}}{%
    \FormulaRefAuto{x \in \{u \in A \mid P(u)\} \eqvdash x \in A \land P(x)}{4}}

  \proofstep{6}{x\in \{x\in\mathbb{N}\mid P(x)\}}{\rA}

  \proofstep{6}{x\in\mathbb{N}\land P(x)}{%
    \FormulaRefAuto{x \in \{u \in A \mid P(u)\} \eqvdash x \in A \land P(x)}{6}}

  \proofstep{6}{x\in\mathbb{N}}{\rAEa{7}}
  \proofstep{6}{P(x)}{\rAEb{7}}

  \proofstep{2,6}{P(x)\rightarrow P(\Succ(x))}{\rRE{\rUE{2},8}}
  \proofstep{2,6}{P(\Succ(x))}{\rRE{10,9}}

  \proofstep{6}{\Succ(x)=x\cup\{x\}}{%
    \FormulaRefAuto{x\in\mathbb{N}\vdash \Succ(x)=x\cup\{x\}}{8}}

  \proofstep{2,6}{P(x\cup\{x\})}{\rIE{12,11}}

  \proofstep{6}{\Succ(x)\in\mathbb{N}}{%
    \FormulaRefAuto{\forall x\in \mathbb{N}\,\Succ(x)\in \mathbb{N}}{8}}

  \proofstep{6}{x\cup\{x\}\in\mathbb{N}}{\rIE{12,14}}

  \proofstep{2,6}{x\cup\{x\}\in\mathbb{N}\land P(x\cup\{x\})}{\rAI{15,13}}

  \proofstep{2,6}{x\cup\{x\}\in \{x\in\mathbb{N}\mid P(x)\}}{%
    \FormulaRefAuto{x \in \{u \in A \mid P(u)\} \eqvdash x \in A \land P(x)}{16}}

  \proofstep{2}{\forall x\in \{x\in\mathbb{N}\mid P(x)\}\,(x\cup\{x\}\in \{x\in\mathbb{N}\mid P(x)\})}{%
    \rUI{\rRI{6,17}}}

  \proofstep{1,2}{%
    \begin{aligned}[t]
      \varnothing\in \{x\in\mathbb{N}\mid P(x)\}\land {}\\
      \forall x\in \{x\in\mathbb{N}\mid P(x)\}\,(x\cup\{x\}\in \{x\in\mathbb{N}\mid P(x)\})
    \end{aligned}%
  }{%
    \rAI{5,18}}

  \proofstep{1,2}{\Induktiv(\{x \in \mathbb{N} \mid P(x)\})}{%
    \rIE{\FormulaRefAuto{\Induktiv(A) := \varnothing \in A \,\land\, \forall x\in A\,(x \cup \{x\} \in A)},19}}
\end{tabproof}


\FormulaThmDelta[Induktion auf \(\mathbb{N}\)]{P(\varnothing)\dsep \forall x\in\mathbb{N}\,(P(x) \rightarrow P(\Succ(x))) \vdash \forall x\in\mathbb{N}\,(P(x))}{
  \DeltaRow{Einstelliges Prädikat}{P}
}
\begin{tabproof}
  \proofstep{1}{ P(\varnothing) }{\rA}
  \proofstep{2}{ \forall x \in \mathbb{N}\,(P(x) \rightarrow P(\Succ(x))) }{\rA}
  \proofstep{1,2}{ \Induktiv(\{x \in \mathbb{N} \mid P(x)\}) }{\FormulaRefAuto{P(\varnothing),\, \forall x\in\mathbb{N}\,(P(x) \rightarrow P(\Succ(x))) \vdash \Induktiv(\{x \in \mathbb{N} \mid P(x)\})}{1,2}}
  \proofstep{}{ \mathbb{N} \subseteq \{x \in \mathbb{N} \mid P(x)\} }{\FormulaRefAuto{\Induktiv(A) \vdash \mathbb{N} \subseteq A}{3}}
  \proofstep{1,2}{ \forall x\in\mathbb{N}\,(P(x)) }{\FormulaRefAuto{A\subseteq \{x\in B\mid P(x)\}\vdash \forall x\in A(P(x))}{4}}
\end{tabproof}

\FormulaThmDelta[Induktion auf \(\mathbb{N}\)]{%
  P(\varnothing)\dsep [x\in\mathbb{N},P(x)]\vdots P(\Succ(x))\dsep n\in\mathbb{N} \vdash P(n)%
}{
  \DeltaRow{Einstelliges Prädikat}{P}
}
\begin{tabproof}
  \proofstep{1}{P(\varnothing)}{\rA}
  \proofstep{2}{n\in\mathbb{N}}{\rA}

  \proofstep{3}{x\in\mathbb{N}}{\rA}
  \proofstep{4}{P(x)}{\rA}
  \proofstep{3,4}{P(\Succ(x))}{[x\in\mathbb{N},P(x)]\vdots P(\Succ(x))(3,4)}
  \proofstep{3}{P(x)\rightarrow P(\Succ(x))}{\rRI{4,5}}
  \proofstep{}{x\in\mathbb{N}\rightarrow \bigl(P(x)\rightarrow P(\Succ(x))\bigr)}{\rRI{3,6}}
  \proofstep{}{ \forall x\in\mathbb{N}\,\bigl(P(x)\rightarrow P(\Succ(x))\bigr) }{\rUI{7}}

  \proofstep{1}{\forall x\in\mathbb{N}\,(P(x))}{%
    \FormulaRefAuto{P(\varnothing)\dsep \forall x\in\mathbb{N}\,(P(x) \rightarrow P(\Succ(x))) \vdash \forall x\in\mathbb{N}\,(P(x))}{1,8}%
  }
  \proofstep{}{n\in\mathbb{N}\rightarrow P(n)}{\rUE{9}}
  \proofstep{2,9}{P(n)}{\rRE{2,10}}
\end{tabproof}


\subsection{Anwendungen des Induktionsprinzips}



\FormulaThmDelta{%
  n\in\mathbb{N}\vdash n\subseteq\mathbb{N}%
}{%
  \DeltaDecl{Mengen}{n\dsep x}%
}
\begin{tabproofsplitwide}
  \proofpartwideind[Induktionsanfang]{\varnothing \subseteq \mathbb{N}}
    \proofstepwidestar{\varnothing \subseteq \mathbb{N}}{%
      \FormulaRefAuto{\varnothing\subseteq A}}
  \closeproofpartwideind

  \proofpartwideind[Induktionsschritt]{%
    n\in\mathbb{N}\dsep n\subseteq\mathbb{N}\vdash \Succ(n)\subseteq\mathbb{N}}
    \proofstepwidestar[1]{n\in\mathbb{N}}{\rA}
    \proofstepwidestar[2]{n\subseteq\mathbb{N}}{\rA}

    \proofstepwidestar[3]{x\in\Succ(n)}{\rA}
    \proofstepwidestar[1,3]{x\in n\lor x=n}{%
      \FormulaRefAuto{n\in\mathbb{N}\dsep x\in\Succ(n)\vdash x\in n\lor x=n}{1,3}}

    \proofstepwidestar[5]{x\in n}{\rA}
    \proofstepwidestar[2,5]{x\in\mathbb{N}}{%
      \FormulaRefAuto{A\subseteq B,\, x\in A\vdash x\in B}{2,5}}

    \proofstepwidestar[7]{x=n}{\rA}
    \proofstepwidestar[1,7]{x\in\mathbb{N}}{\rIE{7,1}}

    \proofstepwidestar[1,2,3]{x\in\mathbb{N}}{\rOE{4,5,6,7,8}}

    \proofstepwidestar[1,2]{x\in\Succ(n)\rightarrow x\in\mathbb{N}}{\rRI{3,9}}
    \proofstepwidestar[1,2]{\forall x\,(x\in\Succ(n)\rightarrow x\in\mathbb{N})}{\rUI{10}}
    \proofstepwidestar[1,2]{\Succ(n)\subseteq\mathbb{N}}{%
      \FormulaRefAuto{A \subseteq B := \forall x\,(x \in A \rightarrow x \in B)}{11}}
  \closeproofpartwideind

  \proofpartwide{Induktionsschluss}
    \proofstepwidestar[1]{n\in\mathbb{N}}{\rA}
    \proofstepwidestar[1]{n\subseteq\mathbb{N}}{%
      \FormulaRefAuto{P(\varnothing)\dsep [x\in\mathbb{N},P(x)]\vdots P(\Succ(x))\dsep n\in\mathbb{N} \vdash P(n)}{%
        \FormulaRefAuto{\varnothing \subseteq \mathbb{N}},%
        \FormulaRefAuto{n\in\mathbb{N}\dsep n\subseteq\mathbb{N}\vdash \Succ(n)\subseteq\mathbb{N}},%
        1}}
  \closeproofpartwide
\end{tabproofsplitwide}




\subsubsection{Transitivitätskette}

\FormulaThmDelta[Transitivität der natürlichen Zahlen]{%
  n\in\mathbb{N}\vdash \bigcup n\subseteq n%
}{%
  \DeltaDecl{Mengen}{n}%
}
\begin{tabproofsplitwide}
  \proofpartwideind[Induktionsanfang]{\bigcup \varnothing \subseteq \varnothing}
    \proofstepwidestar{\bigcup \varnothing = \varnothing}{\FormulaRefAuto{\bigcup \varnothing = \varnothing}}
    \proofstepwidestar{\bigcup \varnothing \subseteq \varnothing}{\FormulaRefAuto{A=B\vdash A\subseteq B}}
  \closeproofpartwideind

  \proofpartwideind[Induktionsschritt]{n\in\mathbb{N}\dsep \bigcup n\subseteq n\vdash \bigcup \Succ(n)\subseteq \Succ(n)}
    \proofstepwidestar[1]{n\in\mathbb{N}}{\rA}
    \proofstepwidestar[2]{\bigcup n\subseteq n}{\rA}
    \proofstepwidestar[1]{\Succ(n)=n\cup\{n\}}{%
      \FormulaRefAuto{n\in\mathbb{N}\vdash \Succ(n) = n\cup\{n\}}{1}}

    \proofstepwidestar[3]{x\in\bigcup\Succ(n)}{\rA}
    \proofstepwidestar[3]{\exists B\in\Succ(n)\,(x\in B)}{%
      \FormulaRefAuto{x \in \bigcup A \eqvdash \exists B\in A\,(x \in B)}{2}}

    \proofstepwidestar[4]{B\in\Succ(n)\land x\in B}{\rA}
    \proofstepwidestar[4]{B\in\Succ(n)}{\rAEa{4}}
    \proofstepwidestar[4]{x\in B}{\rAEb{4}}

    \proofstepwidestar[1,4]{B\in n\lor B=n}{%
      \FormulaRefAuto{n\in\mathbb{N}\dsep x\in\Succ(n)\vdash x\in n\lor x=n}{1,5}}

    \proofstepwidestar[5]{B\in n}{\rA}
    \proofstepwidestar[4,5]{x\in\bigcup n}{%
      \FormulaRefAuto{B\in A\dsep x\in B\vdash x \in \bigcup A}{5,8}}
    \proofstepwidestar[2,4,5]{x\in n}{%
      \FormulaRefAuto{A\subseteq B, x\in A\vdash x\in B}{2,11}}
    \proofstepwidestar[2,4,5]{x\in n\cup\{n\}}{%
      \FormulaRefAuto{z \in A \vdash z \in A \cup B}{13}}
    \proofstepwidestar[1,2,4,5]{x\in\Succ(n)}{\rIE{3,14}}

    \proofstepwidestar[6]{B=n}{\rA}
    \proofstepwidestar[4,6]{x\in n}{\rIE{6,8}}
    \proofstepwidestar[4,6]{x\in n\cup\{n\}}{%
      \FormulaRefAuto{z \in A \vdash z \in A \cup B}{16}}
    \proofstepwidestar[1,4,6]{x\in\Succ(n)}{\rIE{3,17}}

    \proofstepwidestar[1,2,4]{x\in\Succ(n)}{\rOE{9,10,15,16,18}}
    \proofstepwidestar[1,2,3]{x\in\Succ(n)}{\rEE{5,6,19}}

    \proofstepwidestar[1,2]{\bigcup\Succ(n)\subseteq \Succ(n)}{%
      \FormulaRefAuto{A \subseteq B := \forall x\,(x \in A \rightarrow x \in B)}{\rUI{\rRI{4,20}}}}
  \closeproofpartwideind
 \proofpartwide{Induktionsschluss}
    \proofstepwidestar[1]{n\in\mathbb{N}}{\rA}
    \proofstepwidestar[1]{\bigcup n\subseteq n}{%
      \FormulaRefAuto{P(\varnothing)\dsep [x\in\mathbb{N},P(x)]\vdots P(\Succ(x))\dsep n\in\mathbb{N} \vdash P(n)}{\FormulaRefAuto{\bigcup \varnothing \subseteq \varnothing},\FormulaRefAuto{n\in\mathbb{N}\dsep \bigcup n\subseteq n\vdash \bigcup \Succ(n)\subseteq \Succ(n)},1}
    }
  \closeproofpartwide
\end{tabproofsplitwide}



\FormulaThmDelta[Transitivitätskette]{%
  n\in\mathbb{N}\vdash \bigcup n\subseteq n = \bigcup\Succ(n) \subseteq \Succ(n)%
}{%
  \DeltaDecl{Mengen}{n}%
}
\begin{tabproofsplitwide}
  \proofpartwideind[Teil I]{n\in\mathbb{N}\vdash n=\bigcup\Succ(n)}
    \proofstepwidestar[1]{n\in\mathbb{N}}{\rA}

    \proofstepwidestar[2]{x\in n}{\rA}
    \proofstepwidestar[1]{\Succ(n)=n\cup\{n\}}{%
      \FormulaRefAuto{n\in\mathbb{N}\vdash \Succ(n) = n\cup\{n\}}{1}}
    \proofstepwidestar[]{n\in\{n\}}{\FormulaRefAuto{a\in\{a\}}}
    \proofstepwidestar[]{n\in n\cup\{n\}}{%
      \FormulaRefAuto{z\in B\vdash z\in A\cup B}{4}}
    \proofstepwidestar[1]{n\in\Succ(n)}{\rIE{3,5}}
    \proofstepwidestar[1,2]{x\in\bigcup\Succ(n)}{%
      \FormulaRefAuto{B\in A\dsep x\in B\vdash x \in \bigcup A}{6,2}}
    \proofstepwidestar[1]{n\subseteq\bigcup\Succ(n)}{%
      \FormulaRefAuto{A \subseteq B := \forall x\,(x \in A \rightarrow x \in B)}{\rUI{\rRI{2,7}}}}

    \proofstepwidestar[9]{y\in\bigcup\Succ(n)}{\rA}
    \proofstepwidestar[9]{\exists B\in\Succ(n)\,(y\in B)}{%
      \FormulaRefAuto{y \in \bigcup A \eqvdash \exists B\in A\,(y \in B)}{9}}

    \proofstepwidestar[11]{B\in\Succ(n)\land y\in B}{\rA}
    \proofstepwidestar[11]{B\in\Succ(n)}{\rAEa{11}}
    \proofstepwidestar[11]{y\in B}{\rAEb{11}}

    \proofstepwidestar[1,11]{B\in n\cup\{n\}}{\rIE{3,12}}
    \proofstepwidestar[1,11]{B\in n\lor B\in\{n\}}{%
      \FormulaRefAuto{z \in A \cup B \eqvdash z \in A \lor z \in B}{14}}

    \proofstepwidestar[16]{B\in n}{\rA}
    \proofstepwidestar[11,16]{y\in\bigcup n}{%
      \FormulaRefAuto{B\in A\dsep y\in B\vdash y \in \bigcup A}{16,13}}
    \proofstepwidestar[1]{\bigcup n\subseteq n}{%
      \FormulaRefAuto{n\in\mathbb{N}\vdash \bigcup n\subseteq n}{1}}
    \proofstepwidestar[1,11,16]{y\in n}{%
      \FormulaRefAuto{A\subseteq B, y\in A\vdash y\in B}{18,17}}

    \proofstepwidestar[20]{B\in\{n\}}{\rA}
    \proofstepwidestar[20]{B=n}{%
      \FormulaRefAuto{z \in \{a\} \eqvdash z = a}{20}}
    \proofstepwidestar[11,20]{y\in n}{\rIE{21,13}}

    \proofstepwidestar[1,11]{y\in n}{\rOE{15,16,19,20,22}}
    \proofstepwidestar[1,9]{y\in n}{\rEE{10,11,24}}

    \proofstepwidestar[1]{\bigcup\Succ(n)\subseteq n}{%
      \FormulaRefAuto{A \subseteq B := \forall y\,(y \in A \rightarrow y \in B)}{\rUI{\rRI{9,24}}}}

    \proofstepwidestar[1]{n=\bigcup\Succ(n)}{%
      \FormulaRefAuto{A\subseteq B, B\subseteq A\vdash A=B}{8,25}}
  \closeproofpartwide


  \proofpartwideind[Teil II]{n\in\mathbb{N}\vdash \bigcup\Succ(n)\subseteq \Succ(n)}
    \proofstepwidestar[1]{n\in\mathbb{N}}{\rA}
    \proofstepwidestar[]{\forall x\in \mathbb{N}\,\Succ(x)\in \mathbb{N}}{%
      \FormulaRefAuto{\forall x\in \mathbb{N}\,\Succ(x)\in \mathbb{N}}}
    \proofstepwidestar[1]{\Succ(n)\in \mathbb{N}}{\rRE{\rUE{2},1}}
    \proofstepwidestar[1]{\bigcup\Succ(n)\subseteq \Succ(n)}{%
      \FormulaRefAuto{n\in\mathbb{N}\vdash \bigcup n\subseteq n}{3}}
  \closeproofpartwide


  \proofpartwide{Zusammenfassung}
    \proofstepwidestar[1]{n\in\mathbb{N}}{\rA}
    \proofstepwide[1]{\bigcup n}{\subseteq}{n}{
      \FormulaRefAuto{n\in\mathbb{N}\vdash \bigcup n\subseteq n}{1}}
    \proofstepwide[1]{}{=}{\bigcup\Succ(n)}{
      \FormulaRefAuto{n\in\mathbb{N}\vdash n=\bigcup\Succ(n)}{1}}
    \proofstepwide[1]{}{\subseteq}{\Succ(n)}{
      \FormulaRefAuto{n\in\mathbb{N}\vdash \bigcup\Succ(n)\subseteq \Succ(n)}{1}}
  \closeproofpartwide
\end{tabproofsplitwide}

\FormulaThmDelta{%
  u\in\mathbb{N}\vdash \bigcup u\in\mathbb{N}%
}{%
  \DeltaDecl{Mengen}{u\dsep n}%
}
\begin{tabproofsplitwide}

  \proofpartwideind[Induktionsanfang]{\bigcup\varnothing\in\mathbb{N}}
    \proofstepwidestar{\bigcup\varnothing=\varnothing}{%
      \FormulaRefAuto{\bigcup\varnothing=\varnothing}}
    \proofstepwidestar{\varnothing\in\mathbb{N}}{%
      \FormulaRefAuto{\varnothing\in\mathbb{N}}}
    \proofstepwidestar{\bigcup\varnothing\in\mathbb{N}}{\rIE{1,2}}
  \closeproofpartwideind

  \proofpartwideind[Induktionsschritt]{%
    n\in\mathbb{N}\dsep \bigcup n\in\mathbb{N}\vdash \bigcup\Succ(n)\in\mathbb{N}}
    \proofstepwidestar[1]{n\in\mathbb{N}}{\rA}
    \proofstepwidestar[2]{\bigcup n\in\mathbb{N}}{\rA}

    \proofstepwidestar[1]{n=\bigcup\Succ(n)}{%
      \FormulaRefAuto{n\in\mathbb{N}\vdash n=\bigcup\Succ(n)}{1}}
    \proofstepwidestar[1]{\bigcup\Succ(n)=n}{%
      \FormulaRefAuto{a=b\vdash b=a}{3}}
    \proofstepwidestar[1]{\bigcup\Succ(n)\in\mathbb{N}}{\rIE{4,1}}
  \closeproofpartwideind

\end{tabproofsplitwide}

\subsubsection{Die Vorgängerfunktion}


\FormulaDefDelta[Funktionsvorschrift der Vorgängerfunktion]{%
  n\in\mathbb{N}\vdash t_{\Pred}(n) \coloneqq \bigcup n%
}{%
  \DeltaRow{Mengen}{n}%
  \DeltaRow{Funktionensymbol}{t_{\Pred}}%
}




\FormulaThmDelta[Typisierungsaxiom für $t_{\Pred}$]{%
  n\in\mathbb{N}\vdash t_{\Pred}(n)\in\mathbb{N}%
}{%
  \DeltaRow{Mengen}{n}%
  \DeltaRow{Funktionensymbol}{t_{\Pred}}%
}
\begin{tabproof}
  \proofstep{1}{n\in\mathbb{N}}{\rA}
  \proofstep{1}{t_{\Pred}(n)=\bigcup n}{%
    \FormulaRefAuto{u\in\mathbb{N}\vdash t_{\Pred}(u)\coloneqq \bigcup u}{1}}
  \proofstep{1}{\bigcup n\in\mathbb{N}}{%
    \FormulaRefAuto{u\in\mathbb{N}\vdash \bigcup u\in\mathbb{N}}{1}}
  \proofstep{1}{t_{\Pred}(n)\in\mathbb{N}}{\rIE{2,3}}
\end{tabproof}

\FormulaThmDelta[Vorgängerfunktion als Funktion]{%
  \exists! F\colon \mathbb{N}\to\mathbb{N}\,\forall n\in\mathbb{N}\,F(n)=t_{\Pred}(n)%
}{%
  \DeltaRow{Mengen}{n}%
  \DeltaRow{Funktionensymbol}{t_{\Pred}}%
}
\begin{tabproof}
  \proofstep{}{%
    \forall n\in\mathbb{N}\,t_{\Pred}(n)=\bigcup n}{%
    \FormulaRefAuto{u\in\mathbb{N}\vdash t_{\Pred}(u)\coloneqq \bigcup u}}
  \proofstep{}{%
    \forall n\in\mathbb{N}\,t_{\Pred}(n)\in\mathbb{N}}{%
    \FormulaRefAuto{u\in\mathbb{N}\vdash t_{\Pred}(u)\in\mathbb{N}}}
  \proofstep{}{%
    \exists! F\colon \mathbb{N}\to\mathbb{N}\,\forall n\in\mathbb{N}\,F(n)=t_{\Pred}(n)}{%
    \FormulaRefAuto{x\in A\vdash t(x):=y\dsep x\in A\vdash t(x)\in B\exists! F\colon A\to B\,\forall x\in A\,F(x) = t(x)}{1,2}}
\end{tabproof}

\FormulaDefDelta[Vorgängerfunktion]{%
  \Pred\coloneqq \iota F\Bigl(F\colon \mathbb{N}\to\mathbb{N}\land \forall n\in\mathbb{N}\,F(n)=t_{\Pred}(n)\Bigr)%
}{%
  \DeltaRow{Mengen}{n}%
  \DeltaRow{Funktionensymbol}{t_{\Pred}}%
}

\FormulaThmDelta[Vorgängerfunktion]{%
  \Pred\colon \mathbb{N}\to\mathbb{N}%
}{%
  \DeltaRow{Mengen}{n}%
}
\begin{tabproof}
  \proofstep{}{%
    \Pred\colon \mathbb{N}\to\mathbb{N}}{%
    \rAEa{\FormulaRefAuto{\Pred\coloneqq \iota F\Bigl(F\colon \mathbb{N}\to\mathbb{N}\land \forall n\in\mathbb{N}\,F(n)=t_{\Pred}(n)\Bigr)}}}
\end{tabproof}

\FormulaThmDelta[Grundgleichung der Vorgängerfunktion]{%
  n\in\mathbb{N}\vdash \Pred(n)=\bigcup n%
}{%
  \DeltaRow{Mengen}{n}%
}
\begin{tabproof}
  \proofstep{1}{n\in\mathbb{N}}{\rA}
  \proofstep{}{%
    \forall n\in\mathbb{N}\,\Pred(n)=t_{\Pred}(n)}{%
    \rAEb{\FormulaRefAuto{\Pred\coloneqq \iota F\Bigl(F\colon \mathbb{N}\to\mathbb{N}\land \forall n\in\mathbb{N}\,F(n)=t_{\Pred}(n)\Bigr)}}}
  \proofstep{1}{\Pred(n)=t_{\Pred}(n)}{\rRE{1,\rUE{2}}}
  \proofstep{1}{t_{\Pred}(n)=\bigcup n}{%
    \FormulaRefAuto{u\in\mathbb{N}\vdash t_{\Pred}(u)\coloneqq \bigcup u}{1}}
  \proofstep{1}{\Pred(n)=\bigcup n}{%
    \FormulaRefAuto{a=b,\,b=c\vdash a=c}{3,4}}
\end{tabproof}

\FormulaThmDelta[Vorgänger eines Nachfolgers]{%
  n\in\mathbb{N}\vdash \Pred(\Succ(n))=n%
}{%
  \DeltaDecl{Mengen}{n}%
}
\begin{tabproof}
  \proofstep{1}{n\in\mathbb{N}}{\rA}
  \proofstep{1}{\Succ(n)\in\mathbb{N}}{%
    \FormulaRefAuto{n\in\mathbb{N}\vdash \Succ(n)\in\mathbb{N}}{1}}
  \proofstep{1}{\Pred(\Succ(n))=\bigcup\Succ(n)}{%
    \FormulaRefAuto{u\in\mathbb{N}\vdash \Pred(u)=\bigcup u}{2}}
  \proofstep{1}{n=\bigcup\Succ(n)}{%
    \FormulaRefAuto{n\in\mathbb{N}\vdash n=\bigcup\Succ(n)}{1}}
  \proofstep{1}{\bigcup\Succ(n)=n}{%
    \FormulaRefAuto{a=b\vdash b=a}{4}}
  \proofstep{1}{\Pred(\Succ(n))=n}{%
    \FormulaRefAuto{a=b,\,b=c\vdash a=c}{3,5}}
\end{tabproof}

\subsubsection{Keine Selbstmitgliedschaft in den natürlichen Zahlen}

\FormulaThmDelta{%
  n\in\mathbb{N}\vdash n\notin n%
}{%
  \DeltaDecl{Mengen}{n}%
}
\begin{tabproofsplitwide}
  \proofpartwideind[Induktionsanfang]{\varnothing \notin \varnothing}
    \proofstepwidestar[1]{\varnothing\in\varnothing}{\rA}
    \proofstepwidestar[1]{\varnothing\neq\varnothing}{%
      \FormulaRefAuto{u\in A \vdash A\neq \varnothing}{1}}
    \proofstepwidestar{\varnothing=\varnothing}{\rII}
    \proofstepwidestar[1]{\bot}{\rBI{3,2}}
    \proofstepwidestar{\varnothing\notin\varnothing}{\rCI{1,4}}
  \closeproofpartwideind

  \proofpartwideind[Induktionsschritt]{n\in\mathbb{N}\dsep n\notin n\vdash \Succ(n)\notin\Succ(n)}
    \proofstepwidestar[1]{n\in\mathbb{N}}{\rA}
    \proofstepwidestar[2]{n\notin n}{\rA}

    \proofstepwidestar[1]{\Succ(n)=n\cup\{n\}}{%
      \FormulaRefAuto{x\in\mathbb{N}\vdash \Succ(x)=x\cup\{x\}}{1}}

    \proofstepwidestar{n\in\{n\}}{\FormulaRefAuto{a\in\{a\}}}
    \proofstepwidestar{n\in n\cup\{n\}}{%
      \FormulaRefAuto{u\in B \vdash u\in A\cup B}{4}}
    \proofstepwidestar[1]{n\in\Succ(n)}{\rIE{3,5}}

    \proofstepwidestar[7]{\Succ(n)\in\Succ(n)}{\rA}
    \proofstepwidestar[1,7]{\Succ(n)\in n\cup\{n\}}{\rIE{3,7}}
    \proofstepwidestar[1,7]{\Succ(n)\in n \lor \Succ(n)\in\{n\}}{%
      \FormulaRefAuto{z \in A \cup B \eqvdash z \in A \lor z \in B}{8}}

    \proofstepwidestar[10]{\Succ(n)\in n}{\rA}
    \proofstepwidestar[1,10]{n\in\bigcup n}{%
      \FormulaRefAuto{B\in A\dsep x\in B\vdash x \in \bigcup A}{10,6}}
    \proofstepwidestar[1]{\bigcup n\subseteq n}{%
      \FormulaRefAuto{n\in\mathbb{N}\vdash \bigcup n\subseteq n}{1}}
    \proofstepwidestar[1,10]{n\in n}{%
      \FormulaRefAuto{A\subseteq B, x\in A\vdash x\in B}{12,11}}
    \proofstepwidestar[1,2,10]{\bot}{\rBI{13,2}}

    \proofstepwidestar[15]{\Succ(n)\in\{n\}}{\rA}
    \proofstepwidestar[15]{\Succ(n)=n}{%
      \FormulaRefAuto{z \in \{a\} \eqvdash z = a}{15}}
    \proofstepwidestar[1,15]{n\in n}{\rIE{16,6}}
    \proofstepwidestar[1,2,15]{\bot}{\rBI{17,2}}

    \proofstepwidestar[1,2,7]{\bot}{\rOE{9,10,14,15,18}}
    \proofstepwidestar[1,2]{\Succ(n)\notin\Succ(n)}{\rCI{7,19}}
  \closeproofpartwideind

  \proofpartwide{Anwendung des Induktionsprinzips}
    \proofstepwidestar[1]{n\in\mathbb{N}}{\rA}
    \proofstepwidestar[1]{n\notin n}{%
      \FormulaRefAuto{P(\varnothing)\dsep [x\in\mathbb{N},P(x)]\vdots P(\Succ(x))\dsep n\in\mathbb{N} \vdash P(n)}{%
        \FormulaRefAuto{\varnothing \notin \varnothing},%
        \FormulaRefAuto{n\in\mathbb{N}\dsep n\notin n\vdash \Succ(n)\notin\Succ(n)},%
        1}}
  \closeproofpartwide
\end{tabproofsplitwide}

\subsubsection{Weitere Eigenschaften}

\FormulaThmDelta{%
  n\in\mathbb{N}\vdash \varnothing = n \lor \varnothing \in n%
}{%
  \DeltaDecl{Mengen}{n}%
}
\begin{tabproofsplitwide}
  \proofpartwideind[Induktionsanfang]{\varnothing = \varnothing \lor \varnothing \in \varnothing}
    \proofstepwidestar{\varnothing=\varnothing}{\rII}
    \proofstepwidestar{\varnothing = \varnothing \lor \varnothing \in \varnothing}{\rOIa{1}}
  \closeproofpartwideind

  \proofpartwideind[Induktionsschritt]{%
    n\in\mathbb{N}\dsep \varnothing = n \lor \varnothing \in n \vdash
    \varnothing = \Succ(n) \lor \varnothing \in \Succ(n)}
    \proofstepwidestar[1]{n\in\mathbb{N}}{\rA}
    \proofstepwidestar[2]{\varnothing = n \lor \varnothing \in n}{\rA}

    \proofstepwidestar[1]{\Succ(n)=n\cup\{n\}}{%
      \FormulaRefAuto{x\in\mathbb{N}\vdash \Succ(x)=x\cup\{x\}}{1}}

    \proofstepwidestar[4]{\varnothing = n}{\rA}
    \proofstepwidestar[4]{n=\varnothing}{%
      \FormulaRefAuto{a=b\vdash b=a}{4}}
    \proofstepwidestar{n\in\{n\}}{\FormulaRefAuto{a\in\{a\}}}
    \proofstepwidestar[4]{\varnothing\in\{n\}}{\rIE{5,6}}
    \proofstepwidestar[4]{\varnothing\in n\cup\{n\}}{%
      \FormulaRefAuto{u\in B \vdash u\in A\cup B}{7}}
    \proofstepwidestar[1,4]{\varnothing\in\Succ(n)}{\rIE{3,8}}

    \proofstepwidestar[10]{\varnothing \in n}{\rA}
    \proofstepwidestar[10]{\varnothing\in n\cup\{n\}}{%
      \FormulaRefAuto{z \in A \vdash z \in A \cup B}{10}}
    \proofstepwidestar[1,10]{\varnothing\in\Succ(n)}{\rIE{3,11}}

    \proofstepwidestar[1,2]{\varnothing\in\Succ(n)}{\rOE{2,4,9,10,12}}
    \proofstepwidestar[1,2]{\varnothing = \Succ(n) \lor \varnothing \in \Succ(n)}{\rOIb{13}}
  \closeproofpartwideind

  \proofpartwide{Induktionsschluss}
    \proofstepwidestar[1]{n\in\mathbb{N}}{\rA}
    \proofstepwidestar[1]{\varnothing = n \lor \varnothing \in n}{%
      \FormulaRefAuto{P(\varnothing)\dsep [x\in\mathbb{N},P(x)]\vdots P(\Succ(x))\dsep n\in\mathbb{N} \vdash P(n)}{%
        \FormulaRefAuto{\varnothing = \varnothing \lor \varnothing \in \varnothing},%
        \FormulaRefAuto{n\in\mathbb{N}\dsep \varnothing = n \lor \varnothing \in n \vdash \varnothing = \Succ(n) \lor \varnothing \in \Succ(n)},%
        1}}
  \closeproofpartwide
\end{tabproofsplitwide}

\FormulaThmDelta{%
  m\in\mathbb{N}\dsep n\in\mathbb{N}\dsep m\in n\vdash \Succ(m)=n\lor \Succ(m)\in n%
}{%
  \DeltaDecl{Mengen}{m\dsep n}%
}
\begin{tabproofsplitwide}
  \proofpartwideind[Induktionsanfang]{%
    m\in\varnothing \rightarrow \bigl(\Succ(m)=\varnothing \lor \Succ(m)\in\varnothing\bigr)}
    \proofstepwidestar[1]{m\in\varnothing}{\rA}
    \proofstepwidestar[1]{\varnothing\neq\varnothing}{%
      \FormulaRefAuto{u\in A \vdash A\neq \varnothing}{1}}
    \proofstepwidestar{\varnothing=\varnothing}{\rII}
    \proofstepwidestar[1]{\varnothing=\varnothing \land \varnothing\neq\varnothing}{\rAI{3,2}}
    \proofstepwidestar[1]{\Succ(m)=\varnothing \lor \Succ(m)\in\varnothing}{%
      \FormulaRefAuto{P \land \neg P\vdash Q}{4}}
    \proofstepwidestar[]{m\in\varnothing \rightarrow \bigl(\Succ(m)=\varnothing \lor \Succ(m)\in\varnothing\bigr)}{\rRI{1,5}}
  \closeproofpartwideind

  \proofpartwideindDelta[Induktionsschritt]{x\in\mathbb{N}\dsep m\in\Succ(x)\vdash \Succ(m)=\Succ(x)\lor \Succ(m)\in\Succ(x)}{%
    \DeltaPrem{Induktionsvoraussetzung}{x\in\mathbb{N}\dsep m\in x\vdash \Succ(m)=x \lor \Succ(m)\in x}
  }[x\in\mathbb{N}\dsep \bigl(m\in x \rightarrow (\Succ(m)=x \lor \Succ(m)\in x)\bigr)\vdash m\in\Succ(x)\rightarrow\bigl(\Succ(m)=\Succ(x)\lor \Succ(m)\in\Succ(x)\bigr)]

    \proofstepwidestar[1]{x\in\mathbb{N}}{\rA}
    \proofstepwidestar[2]{m\in\Succ(x)}{\rA}

    \proofstepwidestar[1]{\Succ(x)=x\cup\{x\}}{%
      \FormulaRefAuto{x\in\mathbb{N}\vdash \Succ(x)=x\cup\{x\}}{1}}
    \proofstepwidestar[1,2]{m\in x\lor m=x}{%
      \FormulaRefAuto{n\in\mathbb{N}\dsep x\in\Succ(n)\vdash x\in n\lor x=n}{1,2}}

    \proofstepwidestar[5]{m\in x}{\rA}
    \proofstepwidestar[1,5]{\Succ(m)=x \lor \Succ(m)\in x}{IV(1,5)}

    \proofstepwidestar[7]{\Succ(m)=x}{\rA}
    \proofstepwidestar{x\in\{x\}}{\FormulaRefAuto{a\in\{a\}}}
    \proofstepwidestar{ x\in x\cup\{x\} }{%
      \FormulaRefAuto{u\in B \vdash u\in A\cup B}{8}}
    \proofstepwidestar[1]{x\in\Succ(x)}{\rIE{3,9}}
    \proofstepwidestar[7]{x=\Succ(m)}{\FormulaRefAuto{a=b\vdash b=a}{7}}
    \proofstepwidestar[1,7]{\Succ(m)\in\Succ(x)}{\rIE{11,10}}
    \proofstepwidestar[1,7]{\Succ(m)=\Succ(x)\lor \Succ(m)\in\Succ(x)}{\rOIb{12}}

    \proofstepwidestar[14]{\Succ(m)\in x}{\rA}
    \proofstepwidestar[14]{\Succ(m)\in x\cup\{x\}}{%
      \FormulaRefAuto{z \in A \vdash z \in A \cup B}{14}}
    \proofstepwidestar[1,14]{\Succ(m)\in\Succ(x)}{\rIE{3,15}}
    \proofstepwidestar[1,14]{\Succ(m)=\Succ(x)\lor \Succ(m)\in\Succ(x)}{\rOIb{16}}

    \proofstepwidestar[1,5]{\Succ(m)=\Succ(x)\lor \Succ(m)\in\Succ(x)}{\rOE{6,7,13,14,17}}

    \proofstepwidestar[19]{m=x}{\rA}
    \proofstepwidestar[19]{x=m}{\FormulaRefAuto{a=b\vdash b=a}{19}}
    \proofstepwidestar{\Succ(x)=\Succ(x)}{\rII}
    \proofstepwidestar[19]{\Succ(m)=\Succ(x)}{\rIE{20,21}}
    \proofstepwidestar[19]{\Succ(m)=\Succ(x)\lor \Succ(m)\in\Succ(x)}{\rOIa{22}}

    \proofstepwidestar[1,2]{\Succ(m)=\Succ(x)\lor \Succ(m)\in\Succ(x)}{\rOE{4,5,18,19,23}}
    \proofstepwidestar[1]{%
          \begin{aligned}[t]
            m\in\Succ(x)\rightarrow {}\\
            \bigl(\Succ(m)=\Succ(x)\lor \Succ(m)\in\Succ(x)\bigr)
          \end{aligned}%
        }{\rRI{2,24}}
  \closeproofpartwideindDelta

  \proofpartwide{Induktionsschluss}
    \proofstepwidestar[1]{m\in\mathbb{N}}{\rA}
    \proofstepwidestar[2]{n\in\mathbb{N}}{\rA}
    \proofstepwidestar[3]{m\in n}{\rA}

    \proofstepwidestar[2]{m\in n \rightarrow \bigl(\Succ(m)=n \lor \Succ(m)\in n\bigr)}{%
      \FormulaRefAuto{P(\varnothing)\dsep [x\in\mathbb{N},P(x)]\vdots P(\Succ(x))\dsep n\in\mathbb{N} \vdash P(n)}{IA,IS,2}}

    \proofstepwidestar[2,3]{\Succ(m)=n \lor \Succ(m)\in n}{\rRE{4,3}}
  \closeproofpartwide
\end{tabproofsplitwide}


\FormulaThmDelta{%
  m\in\mathbb{N}\dsep n\in\mathbb{N}\vdash m\in n\lor n\in m\lor m=n%
}{%
  \DeltaDecl{Mengen}{m\dsep n}%
}
\begin{tabproofsplitwide}
  \proofpartwideind[Induktionsanfang]{%
    m\in\mathbb{N}\vdash m\in\varnothing \lor \varnothing\in m\lor m=\varnothing}
    \proofstepwidestar[1]{m\in\mathbb{N}}{\rA}
    \proofstepwidestar[1]{\varnothing=m \lor \varnothing\in m}{%
      \FormulaRefAuto{n\in\mathbb{N}\vdash \varnothing = n \lor \varnothing \in n}{1}}

    \proofstepwidestar[3]{\varnothing=m}{\rA}
    \proofstepwidestar[3]{m=\varnothing}{\FormulaRefAuto{a=b\vdash b=a}{3}}
    \proofstepwidestar[3]{m\in\varnothing \lor \varnothing\in m\lor m=\varnothing}{%
      \FormulaRefAuto{R\vdash P\lor Q\lor R}{4}}

    \proofstepwidestar[6]{\varnothing\in m}{\rA}
    \proofstepwidestar[6]{m\in\varnothing \lor \varnothing\in m\lor m=\varnothing}{%
      \FormulaRefAuto{Q\vdash P\lor Q\lor R}{6}}

    \proofstepwidestar[1]{m\in\varnothing \lor \varnothing\in m\lor m=\varnothing}{\rOE{2,3,5,6,7}}
  \closeproofpartwideind

  \proofpartwideindDelta[Induktionsschritt]{%
    m\in\mathbb{N}\dsep x\in\mathbb{N}
    \vdash
    \begin{aligned}[t]
      m\in\Succ(x)\lor {}\\
      \Succ(x)\in m\lor m=\Succ(x)
    \end{aligned}%
  }{%
    \DeltaPrem{Induktionsvoraussetzung}{m\in\mathbb{N}\dsep x\in\mathbb{N}\vdash m\in x\lor x\in m\lor m=x}
  }[m\in\mathbb{N}\dsep x\in\mathbb{N}\vdash m\in\Succ(x)\lor \Succ(x)\in m\lor m=\Succ(x)]

    \proofstepwidestar[1]{m\in\mathbb{N}}{\rA}
    \proofstepwidestar[2]{x\in\mathbb{N}}{\rA}
    \proofstepwidestar[1,2]{m\in x\lor x\in m\lor m=x}{IV(1,2)}

    \proofstepwidestar[2]{\Succ(x)=x\cup\{x\}}{%
      \FormulaRefAuto{x\in\mathbb{N}\vdash \Succ(x)=x\cup\{x\}}{2}}

    \proofstepwidestar[5]{m\in x}{\rA}
    \proofstepwidestar[5]{m\in x\cup\{x\}}{%
      \FormulaRefAuto{u\in A \vdash u\in A \cup B}{5}}
    \proofstepwidestar[2,5]{m\in\Succ(x)}{\rIE{4,6}}
    \proofstepwidestar[2,5]{%
      \begin{aligned}[t]
        m\in\Succ(x)\lor {}\\
        \Succ(x)\in m\lor m=\Succ(x)
      \end{aligned}%
    }{%
      \FormulaRefAuto{P\vdash P\lor Q\lor R}{7}}

    \proofstepwidestar[9]{x\in m}{\rA}
    \proofstepwidestar[1,2,9]{\Succ(x)=m \lor \Succ(x)\in m}{%
      \FormulaRefAuto{m\in\mathbb{N}\dsep n\in\mathbb{N}\dsep m\in n\vdash \Succ(m)=n\lor \Succ(m)\in n}{2,1,9}}

    \proofstepwidestar[11]{\Succ(x)=m}{\rA}
    \proofstepwidestar[11]{m=\Succ(x)}{\FormulaRefAuto{a=b\vdash b=a}{11}}
    \proofstepwidestar[11]{%
      \begin{aligned}[t]
        m\in\Succ(x)\lor {}\\
        \Succ(x)\in m\lor m=\Succ(x)
      \end{aligned}%
    }{%
      \FormulaRefAuto{R\vdash P\lor Q\lor R}{12}}

    \proofstepwidestar[14]{\Succ(x)\in m}{\rA}
    \proofstepwidestar[14]{%
      \begin{aligned}[t]
        m\in\Succ(x)\lor {}\\
        \Succ(x)\in m\lor m=\Succ(x)
      \end{aligned}%
    }{%
      \FormulaRefAuto{Q\vdash P\lor Q\lor R}{14}}

    \proofstepwidestar[1,2,9]{%
      \begin{aligned}[t]
        m\in\Succ(x)\lor {}\\
        \Succ(x)\in m\lor m=\Succ(x)
      \end{aligned}%
    }{\rOE{10,11,13,14,15}}

    \proofstepwidestar[17]{m=x}{\rA}
    \proofstepwidestar[17]{x=m}{\FormulaRefAuto{a=b\vdash b=a}{17}}
    \proofstepwidestar{x\in\{x\}}{\FormulaRefAuto{a\in\{a\}}}
    \proofstepwidestar{x\in x\cup\{x\}}{%
      \FormulaRefAuto{u\in B \vdash u\in A\cup B}{19}}
    \proofstepwidestar[2]{x\in\Succ(x)}{\rIE{4,20}}
    \proofstepwidestar[2,17]{m\in\Succ(x)}{\rIE{18,21}}
    \proofstepwidestar[2,17]{%
      \begin{aligned}[t]
        m\in\Succ(x)\lor {}\\
        \Succ(x)\in m\lor m=\Succ(x)
      \end{aligned}%
    }{%
      \FormulaRefAuto{P\vdash P\lor Q\lor R}{22}}

    \proofstepwidestar[1,2]{%
      \begin{aligned}[t]
        m\in\Succ(x)\lor {}\\
        \Succ(x)\in m\lor m=\Succ(x)
      \end{aligned}%
    }{%
      \FormulaRefAuto{P_1\lor P_2\lor P_3\dsep [P_1]\vdots R\dsep [P_2]\vdots R\dsep [P_3]\vdots R\vdash R}{%
        3,5,8,9,16,17,23}}
  \closeproofpartwideindDelta

  \proofpartwide{Induktionsschluss}
    \proofstepwidestar[1]{m\in\mathbb{N}}{\rA}
    \proofstepwidestar[2]{n\in\mathbb{N}}{\rA}
    \proofstepwidestar[1,2]{m\in n\lor n\in m\lor m=n}{%
      \FormulaRefAuto{P(\varnothing)\dsep [x\in\mathbb{N},P(x)]\vdots P(\Succ(x))\dsep n\in\mathbb{N} \vdash P(n)}{IA,IS,2,1}}
  \closeproofpartwide
\end{tabproofsplitwide}


\subsection{Mitgliedschaft und (echte) Teilmengenrelation}


\FormulaThmDelta[Element ist Teilmenge]{%
  m\in\mathbb{N}\dsep n\in\mathbb{N}\dsep m\in n \vdash m\subseteq n%
}{%
  \DeltaDecl{Mengen}{m\dsep n\dsep x}%
}
\begin{tabproof}
  \proofstep{1}{m\in\mathbb{N}}{\rA}
  \proofstep{2}{n\in\mathbb{N}}{\rA}
  \proofstep{3}{m\in n}{\rA}

  \proofstep{4}{x\in m}{\rA}
  \proofstep{3,4}{x\in\bigcup n}{%
    \FormulaRefAuto{B\in A\dsep x\in B\vdash x \in \bigcup A}{3,4}}
  \proofstep{2}{\bigcup n\subseteq n}{%
    \FormulaRefAuto{n\in\mathbb{N}\vdash \bigcup n\subseteq n}{2}}
  \proofstep{2,3,4}{x\in n}{%
    \FormulaRefAuto{A\subseteq B, x\in A\vdash x\in B}{6,5}}

  \proofstep{2,3}{m\subseteq n}{%
    \FormulaRefAuto{A \subseteq B := \forall x\,(x \in A \rightarrow x \in B)}{\rUI{\rRI{4,7}}}}
\end{tabproof}


\FormulaThmDelta[Teilmenge oder Gleichheit]{%
  m\in\mathbb{N}\dsep n\in\mathbb{N}\dsep m\subseteq n \vdash m\in n\lor m=n%
}{%
  \DeltaDecl{Mengen}{m\dsep n}%
}
\begin{tabproof}
  \proofstep{1}{m\in\mathbb{N}}{\rA}
  \proofstep{2}{n\in\mathbb{N}}{\rA}
  \proofstep{3}{m\subseteq n}{\rA}

  \proofstep{1,2}{m\in n\lor n\in m\lor m=n}{%
    \FormulaRefAuto{m\in\mathbb{N}\dsep n\in\mathbb{N}\vdash m\in n\lor n\in m\lor m=n}{1,2}}

  \proofstep{5}{m\in n}{\rA}
  \proofstep{5}{m\in n\lor m=n}{\rOIa{5}}

  \proofstep{7}{n\in m}{\rA}
  \proofstep{1,2,7}{n\subseteq m}{%
    \FormulaRefAuto{m\in\mathbb{N}\dsep n\in\mathbb{N}\dsep m\in n \vdash m\subseteq n}{2,1,7}}
  \proofstep{1,2,3,7}{m=n}{%
    \FormulaRefAuto{A\subseteq B, B\subseteq A\vdash A=B}{3,8}}
  \proofstep{1,2,3,7}{m\in n\lor m=n}{\rOIb{9}}

  \proofstep{11}{m=n}{\rA}
  \proofstep{11}{m\in n\lor m=n}{\rOIb{11}}

  \proofstep{1,2,3}{m\in n\lor m=n}{%
    \FormulaRefAuto{P_1\lor P_2\lor P_3\dsep [P_1]\vdots R\dsep [P_2]\vdots R\dsep [P_3]\vdots R\vdash R}{4,5,6,7,10,11,12}}
\end{tabproof}

\FormulaThmDelta[Echte Teilmenge ist Element]{%
  m\in\mathbb{N}\dsep n\in\mathbb{N}\dsep m\subseteq n\dsep \neg(m=n) \vdash m\in n%
}{%
  \DeltaDecl{Mengen}{m\dsep n}%
}
\begin{tabproof}
  \proofstep{1}{m\in\mathbb{N}}{\rA}
  \proofstep{2}{n\in\mathbb{N}}{\rA}
  \proofstep{3}{m\subseteq n}{\rA}
  \proofstep{4}{\neg(m=n)}{\rA}

  \proofstep{1,2,3}{m\in n\lor m=n}{%
    \FormulaRefAuto{m\in\mathbb{N}\dsep n\in\mathbb{N}\dsep m\subseteq n \vdash m\in n\lor m=n}{1,2,3}}

  \proofstep{1,2,3,4}{m\in n}{%
    \FormulaRefAuto{P\lor Q,\,\neg Q \vdash P}{5,4}}
\end{tabproof}

% Hilfssatz: aus m∈n folgt m≠n (genauer: ¬(m=n))
\FormulaThmDelta{%
  m\in\mathbb{N}\dsep n\in\mathbb{N}\dsep m\in n \vdash \neg(m=n)%
}{%
  \DeltaDecl{Mengen}{m\dsep n}%
}
\begin{tabproof}
  \proofstep{1}{m\in\mathbb{N}}{\rA}
  \proofstep{2}{n\in\mathbb{N}}{\rA}
  \proofstep{3}{m\in n}{\rA}

  \proofstep{4}{m=n}{\rA}
  \proofstep{3,4}{m\in m}{\rIE{4,3}}
  \proofstep{1}{m\notin m}{\FormulaRefAuto{n\in\mathbb{N}\vdash n\notin n}{1}}
  \proofstep{1,3,4}{\bot}{\rBI{5,6}}
  \proofstep{1,2,3}{\neg(m=n)}{\rCI{4,7}}
\end{tabproof}

% Hauptsatz: m∈n ⇒ m⊂n
\FormulaThmDelta{%
  m\in\mathbb{N}\dsep n\in\mathbb{N}\dsep m\in n \vdash m\subset n%
}{%
  \DeltaDecl{Mengen}{m\dsep n}%
}
\begin{tabproof}
  \proofstep{1}{m\in\mathbb{N}}{\rA}
  \proofstep{2}{n\in\mathbb{N}}{\rA}
  \proofstep{3}{m\in n}{\rA}

  \proofstep{1,2,3}{m\subseteq n}{%
    \FormulaRefAuto{m\in\mathbb{N}\dsep n\in\mathbb{N}\dsep m\in n \vdash m\subseteq n}{1,2,3}}
  \proofstep{1,2,3}{\neg(m=n)}{%
    \FormulaRefAuto{m\in\mathbb{N}\dsep n\in\mathbb{N}\dsep m\in n \vdash \neg(m=n)}{1,2,3}}
  \proofstep{1,2,3}{m\subseteq n \land \neg(m=n)}{\rAI{4,5}}
  \proofstep{1,2,3}{m\subset n}{%
    \FormulaRefAuto{A \subset B := A\subseteq B \land \neg(A=B)}{6}}
\end{tabproof}

\FormulaThmDelta{%
  m\in\mathbb{N}\dsep n\in\mathbb{N}\dsep m\subset n \vdash m\in n%
}{%
  \DeltaDecl{Mengen}{m\dsep n}%
}
\begin{tabproof}
  \proofstep{1}{m\in\mathbb{N}}{\rA}
  \proofstep{2}{n\in\mathbb{N}}{\rA}
  \proofstep{3}{m\subset n}{\rA}

  \proofstep{3}{m\subseteq n \land \neg(m=n)}{%
    \FormulaRefAuto{A \subset B := A\subseteq B \land \neg(A=B)}{3}}
  \proofstep{3}{m\subseteq n}{\rAEa{4}}
  \proofstep{3}{\neg(m=n)}{\rAEb{4}}

  \proofstep{1,2,3}{m\in n}{%
    \FormulaRefAuto{m\in\mathbb{N}\dsep n\in\mathbb{N}\dsep m\subseteq n\dsep \neg(m=n) \vdash m\in n}{1,2,5,6}}
\end{tabproof}

\FormulaThmDelta[Mitgliedschaft genau echte Teilmenge]{%
  m\in\mathbb{N}\dsep n\in\mathbb{N}\vdash (m\in n \leftrightarrow m\subset n)%
}{%
  \DeltaDecl{Mengen}{m\dsep n}%
}
\begin{tabproof}
  \proofstep{1}{m\in\mathbb{N}}{\rA}
  \proofstep{2}{n\in\mathbb{N}}{\rA}

  \proofstep{3}{m\in n}{\rA}
  \proofstep{1,2,3}{m\subset n}{%
    \FormulaRefAuto{m\in\mathbb{N}\dsep n\in\mathbb{N}\dsep m\in n \vdash m\subset n}{1,2,3}}
  \proofstep{1,2}{m\in n \rightarrow m\subset n}{\rRI{3,4}}

  \proofstep{6}{m\subset n}{\rA}
  \proofstep{1,2,6}{m\in n}{%
    \FormulaRefAuto{m\in\mathbb{N}\dsep n\in\mathbb{N}\dsep m\subset n \vdash m\in n}{1,2,6}}
  \proofstep{1,2}{m\subset n \rightarrow m\in n}{\rRI{6,7}}

  \proofstep{1,2}{m\in n \leftrightarrow m\subset n}{\rLRI{5,8}}
\end{tabproof}

\FormulaThmDelta{%
  m\in\mathbb{N}\dsep n\in\mathbb{N}\dsep \neg(m\subseteq n)\vdash n\in m%
}{%
  \DeltaDecl{Mengen}{m\dsep n}%
}
\begin{tabproof}
  \proofstep{1}{m\in\mathbb{N}}{\rA}
  \proofstep{2}{n\in\mathbb{N}}{\rA}
  \proofstep{3}{\neg(m\subseteq n)}{\rA}
  \proofstep{1,2}{m\in n\lor n\in m\lor m=n}{%
    \FormulaRefAuto{m\in\mathbb{N}\dsep n\in\mathbb{N}\vdash m\in n\lor n\in m\lor m=n}{1,2}}

  \proofstep{5}{m\in n}{\rA}
  \proofstep{1,2,5}{m\subseteq n}{%
    \FormulaRefAuto{m\in\mathbb{N}\dsep n\in\mathbb{N}\dsep m\in n \vdash m\subseteq n}{1,2,5}}
  \proofstep{1,2,3,5}{n\in m}{%
    \FormulaRefAuto{P,\,\neg P \vdash Q}{6,3}}

  \proofstep{8}{n\in m}{\rA}

  \proofstep{9}{m=n}{\rA}
  \proofstep{9}{m\subseteq n}{%
    \FormulaRefAuto{A=B\vdash A\subseteq B}{9}}
  \proofstep{3,9}{n\in m}{%
    \FormulaRefAuto{P,\,\neg P \vdash Q}{10,3}}

  \proofstep{1,2,3}{n\in m}{%
    \FormulaRefAuto{P_1\lor P_2\lor P_3\dsep [P_1]\vdots R\dsep [P_2]\vdots R\dsep [P_3]\vdots R\vdash R}{%
      4,5,7,8,8,9,11}}
\end{tabproof}


\subsection{Wohlordnungsprinzip}

\FormulaThmDelta[Wohlordnungsprinzip der natürlichen Zahlen]{%
  B\subseteq\mathbb{N}\dsep B\neq\emptyset
  \vdash \exists a\in B\,\forall b\in B\,a\subseteq b
}{%
  \DeltaDecl{Mengen}{B}%
}
\begin{tabproof}
  \proofstep{1}{B\subseteq\mathbb{N}}{\rA}
  \proofstep{2}{B\neq\varnothing}{\rA}

  \proofstep{2}{\exists x\in B\,\bigl(x\cap B=\varnothing\bigr)}{%
    \FormulaRefAuto{A \neq \varnothing \vdash \exists x \in A \,(x \cap A = \varnothing )}{2}}

  \proofstep{4}{a\in B\land a\cap B=\varnothing}{\rA}
  \proofstep{4}{a\in B}{\rAEa{4}}
  \proofstep{4}{a\cap B=\varnothing}{\rAEb{4}}

  \proofstep{7}{b\in B}{\rA}
  \proofstep{1,4}{a\in\mathbb{N}}{\FormulaRefAuto{A\subseteq B,\, x\in A \vdash x\in B}{1,5}}
  \proofstep{1,7}{b\in\mathbb{N}}{\FormulaRefAuto{A\subseteq B,\, x\in A \vdash x\in B}{1,7}}

  \proofstep{1,4,7}{a\in b\lor b\in a\lor a=b}{%
    \FormulaRefAuto{m\in\mathbb{N}\dsep n\in\mathbb{N}\vdash m\in n\lor n\in m\lor m=n}{8,9}}

  \proofstep{11}{a\in b}{\rA}
  \proofstep{1,4,7,11}{a\subseteq b}{%
    \FormulaRefAuto{m\in\mathbb{N}\dsep n\in\mathbb{N}\dsep m\in n\vdash m\subseteq n}{8,9,11}}

  \proofstep{13}{b\in a}{\rA}
  \proofstep{7,13}{b\in a\cap B}{\FormulaRefAuto{x \in A, x\in B \vdash x \in A\cap B}{13,7}}
  \proofstep{4,7,13}{b\in\varnothing}{\FormulaRefAuto{A = B,\, x \in A \vdash x \in B}{6,14}}
  \proofstep{}{b\not\in\varnothing}{\FormulaRefAuto{x \not\in \varnothing}}
  \proofstep{4,7,13}{b\in\varnothing \land b\not\in\varnothing}{\rAI{15,16}}
  \proofstep{4,7,13}{a\subseteq b}{\FormulaRefAuto{P \land \neg P\vdash Q}{17}}

  \proofstep{19}{a=b}{\rA}
  \proofstep{19}{a\subseteq b}{\FormulaRefAuto{A=B \vdash A\subseteq B}{19}}

  \proofstep{1,4,7}{a\subseteq b}{%
    \FormulaRefAuto{P_1\lor P_2\lor P_3\dsep [P_1]\vdots R\dsep [P_2]\vdots R\dsep [P_3]\vdots R\vdash R}{10,11,12,13,18,19,20}}

  \proofstep{1,4}{\forall b\in B\,a\subseteq b}{\rUI{\rRI{7,21}}}
  \proofstep{1,4}{\exists a\in B\,\forall b\in B\,a\subseteq b}{\rEI{\rAI{5,22}}}

  \proofstep{1,2}{\exists a\in B\,\forall b\in B\,a\subseteq b}{\rEE{3,4,23}}
\end{tabproof}

\subsection{Endliche Mengen}

\subsubsection{Äquivalente Sätze}

\FormulaAxiomDelta[Keine Injektion von $\mathbb{N}$ in eine natürliche Zahl]{%
  n\in\mathbb{N}\vdash \neg\exists F\colon \mathbb{N}\inj n%
}{%
  \DeltaDecl{Mengen}{n}%
}


\FormulaThmDelta{%
    n\in\mathbb{N}\dsep n \EqCard M\vdash \neg\exists F\colon \mathbb{N}\inj M %
}{%
  \DeltaDecl{Mengen}{n\dsep M}%
}
\begin{tabproof}
  \proofstep{1}{n\in\mathbb{N}}{\rA}
  \proofstep{2}{n \EqCard M}{\rA}

  \proofstep{3}{\exists F\colon \mathbb{N}\inj M}{\rA}

  \proofstep{2}{\exists G\colon n\bij M}{\FormulaRefAuto{A \EqCard B \coloneqq \exists F\colon A\bij B}{2}}

  \proofstep{5}{G\colon n\bij M}{\rA}

  \proofstep{5}{G^{-1}\colon M\bij n}{\FormulaRefAuto{F\colon A\bij B \vdash F^{-1}\colon B\bij A}{5}}
  \proofstep{5}{G^{-1}\colon M\inj n}{\FormulaRefAuto{F\colon A\bij B \vdash F\colon A\inj B}{6}}

  \proofstep{6}{F\colon \mathbb{N}\inj M}{\rA}

  \proofstep{5,6}{G^{-1}\circ F\colon \mathbb{N}\inj n}{\FormulaRefAuto{F\colon A\inj B\dsep G\colon B\inj C \vdash G\circ F\colon A\inj C}{6,7}}

  \proofstep{5,6}{\exists H\colon \mathbb{N}\inj n}{\rEI{9}}
  \proofstep{1}{\neg\exists H\colon \mathbb{N}\inj n}{\FormulaRefAuto{n\in\mathbb{N}\vdash \neg\exists F\colon \mathbb{N}\inj n}{1}}
  \proofstep{1,5,6}{\bot}{\rBI{10,11}}

  \proofstep{1,2,3,5}{\bot}{\rEE{3,6,12}}
  \proofstep{1,2,3}{\bot}{\rEE{4,5,13}}

  \proofstep{1,2}{\neg\exists F\colon \mathbb{N}\inj M}{\rCI{3,14}}
\end{tabproof}


\FormulaThmDelta{%
    \neg\exists F\colon \mathbb{N}\inj M\vdash \neg\exists F\colon \mathbb{N}\sur M
}{%
  \DeltaDecl{Mengen}{M}%
}

\FormulaThmDelta{%
    \neg\exists F\colon \mathbb{N}\sur M\vdash \forall F\colon M\inj M\,\bigl(F\colon M\bij M\bigr)
}{%
  \DeltaDecl{Mengen}{M}%
}

\FormulaThmDelta{%
    \forall F\colon M\inj M\,\bigl(F\colon M\bij M\bigr)\vdash\forall F\colon M\sur M\,\bigl(F\colon M\bij M\bigr)
}{%
  \DeltaDecl{Mengen}{M}%
}

\PrintFormulaStack

\PrintFormulaStackSourcesPlainWithType

\end{document}

