%============================================================
%  Bd. 03 - Mengenlehre %============================================================

\documentclass[main.tex]{subfiles}


\ifSubfilesClassLoaded{
    \usepackage{xr}
    \externaldocument{_B01}
  \externaldocument{_B02}
}{
   % Code für als Subfile eingebunden
}

\title{Bd. 03 - Mengenlehre}
\author{Martin Kunze}
\date{}
\setcounter{file}{3}




\begin{document}

\maketitle
\tableofcontents
%\listoftheorems

\FormulaAxiomAuto[Leere Menge]{\exists O\;\bigl(\forall x\,(x \notin O)\bigr)}

\FormulaDefAuto[Leere Menge]{\emptyset := \iota O\bigl(\forall x\,(x \notin O)\bigr)}
\begin{remark}
    Hieraus gewinnen wir für alle \(x\):
    \[
    x \notin \emptyset.
    \]
\end{remark}

\FormulaThmAuto{ \exists! O\forall x (x \notin O) }
\begin{tabproof}
  \proofstep{}{ \exists O\forall x (x \notin O) }{ \FormulaRefAuto{\exists O\;\bigl(\forall x\,(x \notin O)\bigr)} }
  \proofstep{2}{ \forall x (x \notin O) }{ \rA }
  \proofstep{3}{ \forall x (x \notin P) }{ \rA }
  \proofstep{2}{ \forall x (x \notin O \lor x \in P) }{ \FormulaRefAuto{\forall x(F(x))\lor\forall x(G(x))\vdash\forall x(F(x)\lor G(x))} }
  \proofstep{3}{ \forall x (x \notin P \lor x \in O) }{ \FormulaRefAuto{ \forall x(F(x))\lor\forall x(G(x))\vdash\forall x(F(x)\lor G(x))} }
  \proofstep{2,3}{ \forall x (x \in O \leftrightarrow x \in P) }{ \FormulaRefAuto{ \forall x (P(x) \leftrightarrow Q(x)) \dashv \vdash \forall x (\neg P(x) \lor Q(x)) \land \forall x (\neg Q(x) \lor P(x)) } }
  \proofstep{2,3}{ O = P }{ \FormulaRefAuto{ \forall x\, (x \in A \leftrightarrow x \in B) \dashv\vdash A = B } }
  \proofstep{}{ \exists! O (\forall x (x \notin O)) }{ \UEI{1,2,3,6} }
\end{tabproof}

\end{document}