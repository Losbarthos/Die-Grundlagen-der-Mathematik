%============================================================
%  Bd. 04 - Die Menge der natürlichen Zahlen %============================================================

\documentclass{book}
\usepackage{graphicx}
\usepackage[utf8]{inputenc}
\usepackage[ngerman]{babel}
\usepackage{amsmath,amssymb,amsthm}
\usepackage{mathtools}
\usepackage{microtype}
\usepackage{hyperref}
\usepackage{csquotes}
\usepackage{enumitem}
\usepackage{thmtools}
\usepackage{etoolbox}
\usepackage{longtable}
\usepackage{xparse}
\usepackage{imakeidx}
\usepackage{bussproofs}
\tolerance=9999
\MakeAutoQuote{„}{“}
\hypersetup{
    bookmarksopen=false,
    bookmarksnumbered=true
}
% tex/impl/commands.tex
\ProvidesFile{commands.tex}[2026/01/04 Command dispatcher]

% tex/impl/commands/logic.tex
\ProvidesFile{logic.tex}[2026/01/04 Logical separators and symbols]

% ------------------------------------------------------------------
% Trennzeichen für Annahmen (dein \dsep)
% ------------------------------------------------------------------
\providecommand{\dsep}{,\allowbreak\ }

% ------------------------------------------------------------------
% Äquivalenz im Kalkül (dein \eqvdash)
% ------------------------------------------------------------------
\providecommand{\eqvdash}{\dashv\vdash}

% ------------------------------------------------------------------
% XOR (falls du es als Operator nutzt)
% ------------------------------------------------------------------
\providecommand*\lxor{\mathbin{\veebar}}

% ------------------------------------------------------------------
% Optional: kleine Hilfsmakros, die du häufig in Formeln brauchst
% (nur aufnehmen, wenn du sie wirklich verwendest)
% ------------------------------------------------------------------
\providecommand{\st}{\,\middle|\,} % für Mengenschreibweise {x \in A \st P(x)}

% ------------------------------------------------------------------
% Optional: konsistente Klammer-Makros (falls du das magst)
% ------------------------------------------------------------------
\providecommand{\paren}[1]{\ensuremath{\left(#1\right)}}
\providecommand{\brak}[1]{\ensuremath{\left[#1\right]}}
\providecommand{\set}[1]{\ensuremath{\left\{#1\right\}}}
\providecommand{\st}{\ensuremath{\,\middle|\,}}


\input{tex/impl/commands/rules.tex}
% tex/impl/commands/sets.tex
\ProvidesFile{sets.tex}[2026/01/04 Set-theoretic notation]

% Klammerhilfen: funktionieren auch in Textmodus
\providecommand{\paren}[1]{\ensuremath{\left(#1\right)}}
\providecommand{\brak}[1]{\ensuremath{\left[#1\right]}}
\providecommand{\set}[1]{\ensuremath{\left\{#1\right\}}}

% Potenzmenge
\providecommand{\powerset}{\ensuremath{\mathcal{P}}}

\makeatletter
% Mitgliedschaftsrelation: \MemRel  oder \MemRel(M)
\def\MemRel{%
  \@ifnextchar({\MemRel@args}{\MemRel@name}%
}
\def\MemRel@name{%
  \ensuremath{R}%
}
\def\MemRel@args(#1){%
  \ensuremath{R_{#1}}%
}

% (Falls du sie brauchst) CoverFam / PartFam: \CoverFam oder \CoverFam(M,A)
\def\CoverFam{%
  \@ifnextchar({\CoverFam@args}{\CoverFam@name}%
}
\def\CoverFam@name{%
  \ensuremath{\mathsf{CoverFam}}%
}
\def\CoverFam@args(#1,#2){%
  \ensuremath{\mathsf{CoverFam}\!\left(#1,#2\right)}%
}

\def\PartFam{%
  \@ifnextchar({\PartFam@args}{\PartFam@name}%
}
\def\PartFam@name{%
  \ensuremath{\mathsf{PartFam}}%
}
\def\PartFam@args(#1,#2){%
  \ensuremath{\mathsf{PartFam}\!\left(#1,#2\right)}%
}

% Vereinigungsabgeschlossene Familie: \UnionClosedFam oder \UnionClosedFam(M)
\def\UnionClosedFam{%
  \@ifnextchar({\UnionClosedFam@args}{\UnionClosedFam@name}%
}
\def\UnionClosedFam@name{%
  \ensuremath{\mathsf{UnionClosedFam}}%
}
\def\UnionClosedFam@args(#1){%
  \ensuremath{\mathsf{UnionClosedFam}\!\left(#1\right)}%
}
\makeatother

% ------------------------------------------------------------
% Kardinalität / Gleichmächtigkeit
% ------------------------------------------------------------
\providecommand{\EqCard}{\mathrel{\approx}}


% commands/sets.tex
\DeclareMathOperator{\Induktiv}{Induktiv}

\input{tex/impl/commands/relations.tex}
% tex/impl/commands/functions.tex
\ProvidesFile{functions.tex}[2026/01/04 Function-theoretic notation]

% ------------------------------------------------------------
% Pfeile für Abbildungen
% ------------------------------------------------------------
\providecommand{\inj}{\rightarrowtail}           % injektiv
\providecommand{\sur}{\twoheadrightarrow}        % surjektiv
\providecommand{\bij}{\xrightarrow{\sim}}        % bijektiv

% ------------------------------------------------------------
% Standard-Namen / Operatoren
% ------------------------------------------------------------
\providecommand{\Id}{\mathrm{id}}                % Identitäts-Operator (z.B. \Id_A)
\providecommand{\Graph}{\ensuremath{\mathrm{Graph}}}

% ------------------------------------------------------------
% Projekt-spezifische Operatoren (als MathOperator wie bei dir)
% ------------------------------------------------------------
\DeclareMathOperator{\Ausw}{Ausw}                % Auswahlmenge/Operator
\DeclareMathOperator{\Fib}{Fib}                  % Faser (Operator)
\DeclareMathOperator{\FibFam}{FibFam}            % Faserfamilie (falls genutzt)
\DeclareMathOperator{\Sec}{Sec}                  % Sektion / Rechtsinverse (z.B. \Sec_{G,L})

% ------------------------------------------------------------
% Graph einer Surjektion aus einer Injektion (mit Stützelement a_0)
% ------------------------------------------------------------
\DeclareRobustCommand{\Gsurjfrominj}[2]{G_{#1,#2}}

% commands/functions.tex
\DeclareMathOperator{\Succ}{succ} 
\DeclareMathOperator{\Pred}{pred}

% tex/impl/commands/project.tex
\ProvidesFile{project.tex}[2026/01/04 Project specific macros]

\providecommand{\DisjFam}{\ensuremath{\mathsf{DisjFam}}}


\makeindex[name=satz,title=Sätze und Definitionen zu diesem Kapitel]

\theoremstyle{plain}
\newtheorem{notation*}{Notation}
\newtheorem{theorem}{Theorem}
\newtheorem*{theorem*}{Theorem}
\newtheorem{corollary}[theorem]{Korollar}
\newtheorem*{lemma}{Lemma} 
\theoremstyle{remark}
\newtheorem*{remark}{Bemerkung}
\newtheorem*{bemerkung}{Bemerkung}
\theoremstyle{definition}
\newtheorem{definition}{Definition}[section]
\newtheorem{hilfsdefinition}{Hilfsdefinition}[section]
\newtheorem*{tempdefinition}{Temporäre Definition}
\newtheorem*{example}{Beispiel}
\newtheorem*{hint}{Hinweis}

\title{Bd. 04 - Die Menge der natürlichen Zahlen}
\author{Martin Kunze}
\date{}

\begin{document}
\maketitle
\tableofcontents
\listoftheorems

\chapter{Addition von natürlichen Zahlen}

\textbf{Index der Sätze und Definitionen:}


\begin{definition}[Addition]
    Die Addition von zwei natürlichen Zahlen \( a \) und \( b \) ist eine binäre Operation \( +: \mathbb{N} \times \mathbb{N} \to \mathbb{N} \), die rekursiv wie folgt definiert wird:
    
    \begin{itemize}
        \item \textbf{Basisfall}: Für jede natürliche Zahl \( a \) gilt:
        \[
        a + 0 := a.
        \]
        
        \item \textbf{Rekursionsschritt}: Für jede natürliche Zahl \( b \) gilt:
        \[
        a + (b+1) := (a + b) + 1.
        \]
    \end{itemize}
\end{definition}

\begin{remark}
In der Definition von \( a + (b+1) \) sorgt die Klammer um \( b+1 \) dafür, dass \( b \) zuerst um 1 erhöht wird, bevor diese Summe zu \( a \) addiert wird. Es ist wichtig zu beachten, dass die Klammer in \( (a + b) + 1 \) verwendet wird, um die Reihenfolge der Operationen zu verdeutlichen: Zuerst wird \( a + b \) berechnet, und dann wird 1 hinzugefügt. Wird jedoch der gesamte Ausdruck \( (a + b) + 1 \) durch eine Variable \( y \) ersetzt, also \( y = (a + b) + 1 \), dann kann die Klammer um \( y \) weggelassen werden, da \( y \) eine einzelne Zahl darstellt und keine Mehrdeutigkeit besteht.
\end{remark}



\paragraph{Beweisregeln für die Addition}
\label{rule:rAddI} 
Basierend auf diesen Definitionen können wir folgende Regel für die Addition formulieren. 

\[
\begin{array}{llll}
	i & (1) & a \in \mathbb{N} & ... \\
	j & (2) & b \in \mathbb{N} & ... \\
        i & (3) & a = a + 0 & \rAddI{1} \\
	i,j & (4) & a + (b+1) = (a+b)+1 & \rAddI{1,2} \\
            & (5) & +:\mathbb{N}\times\mathbb{N}\rightarrow\mathbb{N} & \rAddI{} \\
\end{array}
\]

\(i\) und \(j\) sind dabei Listen von Annahmen.

\label{aInNaturalwbInNaturalImpaPlusbInNatural}
\begin{theorem}[\(a\in\mathbb{N}, b\in\mathbb{N}\vdash a+b\in\mathbb{N}\) (Wohldefiniertheit der Addition)]
\end{theorem}
\begin{proof}
	\[
	\begin{array}{llll}
		1   &  (1) & a\in\mathbb{N} & \rA \\
            2   &  (2) & b\in\mathbb{N} & \rA \\
            1   &  (3) & a=a+0 & \rAddI{1} \\
            1   &  (4) & a+0\in\mathbb{N} & \rIE{3,1} \\
            5   &  (5) & n\in \mathbb{N} & \rA \\
            6   &  (6) & a+n\in \mathbb{N} & \rA \\
            5,6 &  (7) & a+(n+1)=(a+n)+1 & \rAddI{5,6} \\
            6   &  (8) & (a+n)+1\in\mathbb{N} & \successorIsNaturalNumber{6} \\
            5,6 &  (9) & a+(n+1)\in\mathbb{N} & \rIE{7,8} \\
            1   &  (10) & \forall n\in\mathbb{N}(a+n\in\mathbb{N}) & \rInductionN{4,5,6,9} \\
            1   &  (11) & b\in\mathbb{N}\rightarrow a+b\in\mathbb{N} & \rSetUEb{10} \\
            1,2 &  (12) & a+b\in\mathbb{N} & \rRE{11,2} \\
	\end{array}
	\]
\end{proof}

\label{aInNaturalImpaEqualsZeroPlusa}
\begin{theorem}[\(a\in\mathbb{N}\vdash a=0+a\) (Neutrales Element)]
\end{theorem}
\begin{proof}
        \[
	\begin{array}{llll}
            1   &  (1) & a\in\mathbb{N} & \rA \\
                &  (2) & 0\in\mathbb{N} & \zeroIsNaturalNumber \\
                &  (3) & 0=0+0 & \rAddI{2} \\
            4   &  (4) & n\in\mathbb{N} & \rA \\
            5   &  (5) & n=0+n & \rA \\
            4   &  (6) & 0+(n+1)=(0+n)+1 & \rAddI{2,4} \\
            4,5 &  (7) & 0+(n+1)=n+1 & \rIE{5,6} \\
            4,5 &  (8) & n+1=0+(n+1) & \aIdbImpbIda{7} \\
                &  (9) & \forall n\in\mathbb{N}(n=0+n) & \rInductionN{3,4,5,8} \\
                &  (10) & a\in\mathbb{N}\rightarrow a=0+a & \rSetUEb{9} \\
            1   &  (11) & a=0+a & \rRE{1,10} \\
	\end{array}
	\]
\end{proof}

\label{aInNaturalImpaPlusZeroEqualsZeroPlusa}
\begin{theorem}[\(a\in\mathbb{N}\vdash a+0=0+a\)]
\end{theorem}
\begin{proof}
        \[
	\begin{array}{llll}
            1   &  (1) & a\in\mathbb{N} & \rA \\
            1   &  (2) & a = a+0 & \rAddI{1} \\
            1   &  (3) & a = 0+a & \aInNaturalImpaEqualsZeroPlusa{1} \\
            1   &  (4) & a+0 = 0+a & \rIE{2,3} \\
	\end{array}
	\]
\end{proof}


\label{aInNaturalImpZeroPlusaEqualsaPlusZeroEqualsa}
\begin{theorem}[\(a\in\mathbb{N}\vdash 0+a=a+0=a\) (Neutrales Element)]
\end{theorem}
\begin{proof}
        \[
	\begin{array}{llll}
            1   &  (1) & a\in\mathbb{N} & \rA \\
            1   &  (2) & 0+a=a+0 & \aInNaturalImpaPlusZeroEqualsZeroPlusa{1} \\
            1   &  (3) & a+0=0+a & \aIdbImpbIda{2} \\
            1   &  (4) & a=0+a & \aInNaturalImpaEqualsZeroPlusa{1} \\
            1   &  (5) & 0+a=a & \aIdbImpbIda{4} \\
            1   &  (6) & a+0=a & \rIE{5,3} \\
            1   &  (7) & a+0=0+a\land a+0=a & \rAI{3,6} \\
            1   &  (8) & a+0=0+a=a & \rIIb{7} \\
	\end{array}
	\]
\end{proof}


\label{aInNaturalImpOnePlusaEqualsaPlusOne}
\begin{theorem}[\(a\in\mathbb{N}\vdash 1+a=a+1\)]
\end{theorem}
\begin{proof}
        \[
	\begin{array}{llll}
            1   &  (1) & a\in\mathbb{N} & \rA \\
                &  (2) & 1\in\mathbb{N} & \oneIsNaturalNumber{} \\
                &  (3) & 1+0=0+1 & \aInNaturalImpaPlusZeroEqualsZeroPlusa{2} \\
            4   &  (4) & n\in\mathbb{N} & \rA \\
            5   &  (5) & 1+n=n+1 & \rA \\
            4   &  (6) & 1+(n+1)=(1+n)+1 & \rAddI{4,2} \\
            4,5 &  (7) & 1+(n+1)=(n+1)+1 & \rEE{4,5} \\
                &  (8) & \forall n\in\mathbb{N}(1+n=n+1) & \rInductionN{3,4,5,7} \\
                &  (9) & a\in\mathbb{N}\rightarrow 1+a=a+1 & \rSetUEb{8} \\
            1   &  (10)&1+a=a+1 & \rRE{1,9} \\
	\end{array}
	\]
\end{proof}

\label{aInNaturalwbInNaturalwcInNaturalImpaPlusLpbPluscRpEqualsLpaPlusbRpPlusc}
\begin{theorem}[\(a\in\mathbb{N}, b\in\mathbb{N}, c\in\mathbb{N}\vdash a+(b+c)=(a+b)+c\) (Assoziativgesetz)]
\end{theorem}
\begin{proof}
    \[
	\begin{array}{llll}
            1   &  (1) & a\in\mathbb{N} & \rA \\
            2   &  (2) & b\in\mathbb{N} & \rA \\
            3   &  (3) & c\in\mathbb{N} & \rA \\
            2   &  (4) & b = b + 0 & \aInNaturalImpZeroPlusaEqualsaPlusZeroEqualsa{2}  \\
            1,2 &  (5) & a+b\in\mathbb{N} & \aInNaturalwbInNaturalImpaPlusbInNatural{1,2}  \\
            1,2  &  (6) & a+b = (a+b) + 0 & \aInNaturalImpZeroPlusaEqualsaPlusZeroEqualsa{5}  \\
            1,2  &  (7) & a+(b+0) = (a+b) + 0 & \rIE{4,6}  \\
            8  &  (8) & n\in\mathbb{N} & \rA  \\
            9  &  (9) & a+(b+n)=(a+b)+n & \rA  \\
            1,2,8  &  (10) & (a+b)+(n+1) = ((a+b)+n)+1 & \rAddI{5,8}  \\
            1,2,8,9  &  (11) & (a+b)+(n+1) = (a+(b+n))+1 & \rIE{9,10}  \\
            2,8  &  (12) & b+n\in\mathbb{N} & \aInNaturalwbInNaturalImpaPlusbInNatural{2,8}  \\
            1,2,8  &  (13) & a+((b+n)+1)=(a+(b+n))+1 & \rAddI{1,12}  \\
            1,2,8,9  &  (14) & (a+b)+(n+1)=a+((b+n)+1) & \rIE{13,11}  \\
            2,8    &  (15) & b+(n+1)=(b+n)+1 & \rAddI{2,8}  \\
            1,2,8,9    &  (16) & (a+b)+(n+1)=a+(b+(n+1)) & \rIE{15,14}  \\
            1,2,8,9    &  (17) & a+(b+(n+1))=(a+b)+(n+1) & \aIdbImpbIda{16}  \\
            1,2    &  (18) & \forall n\in\mathbb{N}(a+(b+n)=(a+b)+n) & \rInductionN{7,8,9,17}  \\
            1,2    &  (19) & c\in\mathbb{N}\rightarrow (a+(b+c)=(a+b)+c) & \rSetUEb{18}  \\
            1,2,3   &  (20) & (a+(b+c)=(a+b)+c) & \rRE{3,19}  \\
	\end{array}
	\]
\end{proof}

\label{aInNaturalwbInNaturalImpaPlusbEqualsbPlusa}
\begin{theorem}[\(a\in\mathbb{N}, b\in\mathbb{N}\vdash a+b=b+a\) (Kommutativgesetz)]
\end{theorem}
\begin{proof}
        \[
	\begin{array}{llll}
            1       &  (1) & a\in\mathbb{N} & \rA \\
            2       &  (2) & b\in\mathbb{N} & \rA \\
            1       &  (3) & a+0=0+a & \aInNaturalImpaPlusZeroEqualsZeroPlusa{1} \\
            4       &  (4) & n\in\mathbb{N} & \rA \\
            5       &  (5) & a+n=n+a & \rA \\
            1,4     &  (6) & a+(n+1)=(a+n)+1 & \rAddI{1,4} \\
            1,4,5   &  (7) & a+(n+1)=(n+a)+1 & \rIE{5,6} \\
            1,4     &  (8) & n+a\in\mathbb{N} & \aInNaturalwbInNaturalImpaPlusbInNatural{4,1} \\
            1,4     &  (9) & 1+(n+a)=(n+a)+1 & \aInNaturalImpOnePlusaEqualsaPlusOne{8} \\
            1,4,5   &  (10) & a+(n+1)=1+(n+a) & \rIE{9,7} \\
                    &  (11) & 1\in\mathbb{N} & \oneIsNaturalNumber{} \\
            1,4     &  (12) & 1+(n+a)=(1+n)+a & \aInNaturalwbInNaturalwcInNaturalImpaPlusLpbPluscRpEqualsLpaPlusbRpPlusc{11,4,1} \\
            1,4,5   &  (12) & a+(n+1)=(1+n)+a & \rIE{12,10} \\
            4       &  (13) & 1+n=n+1 & \aInNaturalImpOnePlusaEqualsaPlusOne{4} \\
            1,4,5   &  (14) & a+(n+1)=(n+1)+a & \rIE{13,12} \\
            1       &  (15) & \forall n\in\mathbb{N}(a+n=n+a) & \rInductionN{3,4,5,14} \\
            1       &  (16) & b\in\mathbb{N}\rightarrow a+b=b+a & \rSetUEb{15} \\
            1,2     &  (17) & a+b=b+a & \rRE{2,16} \\
	\end{array}
	\]
\end{proof}

\label{ImpLpNaturalwPluswZeroRpInMonoid}
\begin{theorem}[\(\vdash (\mathbb{N},+,0) \text{ ist ein Monoid.}\)]
\end{theorem}
\begin{proof}
        \[
	\begin{array}{llll}
                &  (1) & +:\mathbb{N}\times\mathbb{N}\rightarrow\mathbb{N} & \rAddI{} \\
                &  (2) & \forall a\in\mathbb{N}\forall b\in\mathbb{N}\forall c\in\mathbb{N}(a+(b+c)=(a+b)+c) & \aInNaturalwbInNaturalwcInNaturalImpaPlusLpbPluscRpEqualsLpaPlusbRpPlusc{} \\
                &  (3) & \forall a\in\mathbb{N}(a+0=0+a=a) & \aInNaturalImpZeroPlusaEqualsaPlusZeroEqualsa{} \\
                &  (4) & (\mathbb{N},+,0) \text{ ist ein Monoid.} & \rMonoidI{1,2,3} \\
	\end{array}
	\]
\end{proof}

\label{ImpLpNaturalwPluswZeroRpInAbelMonoid}
\begin{theorem}[\(\vdash (\mathbb{N},+,0) \text{ ist ein abelscher Monoid.}\)]
\end{theorem}
\begin{proof}
        \[
	\begin{array}{llll}
                &  (1) & (\mathbb{N},+,0) \text{ ist ein Monoid.} & \ImpLpNaturalwPluswZeroRpInMonoid{} \\
                &  (2) & \forall a\in\mathbb{N}\forall b\in\mathbb{N}(a+b=b+a) & \aInNaturalwbInNaturalwcInNaturalImpaPlusLpbPluscRpEqualsLpaPlusbRpPlusc{} \\
                &  (3) & (\mathbb{N},+,0) \text{ ist ein abelscher Monoid.} & \rAbelianMonoidI{1,2} \\
	\end{array}
	\]
\end{proof}

\section{Erweiterte Vertauschungsgesetze der Addition}

\label{aInNaturalwbInNaturalwcInNaturalImpLpaPlusbRpPluscEqualsLpaPluscRpPlusb}
\begin{theorem}[\(a\in\mathbb{N}, b\in\mathbb{N}, c\in\mathbb{N} \vdash (a+b)+c=(a+c)+b\)]
\end{theorem}
\begin{proof}
	\[
	\begin{array}{lll p{5cm}}
		1         &  (1) & a\in\mathbb{N} & \rA \\
		2         &  (2) & b\in\mathbb{N} & \rA \\
		3         &  (3) & c\in\mathbb{N} & \rA \\
		1,2,3     &  (4) & a+(b+c)=(a+b)+c & \aInNaturalwbInNaturalwcInNaturalImpaPlusLpbPluscRpEqualsLpaPlusbRpPlusc{1,2,3} \\
		2,3       &  (5) & b+c=c+b & \aInNaturalwbInNaturalImpaPlusbEqualsbPlusa{2,3} \\        
		             &  (6) & a+(b+c) = a+(b+c) & \rII{} \\       
        2,3       &  (7) & a+(b+c) = a+(c+b) & \rIE{5,6} \\       
        1,2,3     &  (8) & a+(c+b)=(a+c)+b & \aInNaturalwbInNaturalwcInNaturalImpaPlusLpbPluscRpEqualsLpaPlusbRpPlusc{1,3,2} \\       
        1,2,3     &  (9) & a+(b+c)=a+(c+b) & \rIE{7,6} \\       
        1,2,3     &  (10) & a+(b+c)=(a+c)+b & \rIE{8,9} \\       
        1,2,3     &  (11) & (a+b)+c=(a+c)+b & \rIE{6,10} \\       
	\end{array}
	\]
\end{proof}

\label{aInNaturalwbInNaturalwcInNaturalwdInNaturalImpLpaPlusbRpPlusLpcPlusdRpEqualsLpaPluscRpPlusLpbPlusdRp}
\begin{theorem}[\(a\in\mathbb{N}, b\in\mathbb{N}, c\in\mathbb{N}, d\in\mathbb{N} \vdash (a+b)+(c+d)=(a+c)+(b+d)\)]
\end{theorem}
\begin{proof}
	\[
	\begin{array}{lll p{5cm}}
		1         &  (1) & a\in\mathbb{N} & \rA \\
		2         &  (2) & b\in\mathbb{N} & \rA \\
		3         &  (3) & c\in\mathbb{N} & \rA \\
		4         &  (4) & d\in\mathbb{N} & \rA \\
		1,2       &  (5) & a+b \in \mathbb{N} & \aInNaturalwbInNaturalImpaPlusbInNatural{1,2} \\
		1,3       &  (6) & a+c \in \mathbb{N} & \aInNaturalwbInNaturalImpaPlusbInNatural{1,3} \\
		1,2,3,4   &  (7) & (a+b)+(c+d)=((a+b)+c)+d & \aInNaturalwbInNaturalwcInNaturalImpaPlusLpbPluscRpEqualsLpaPlusbRpPlusc{5,3,4} \\
		  1,2,3     &  (8) & (a+b)+c = (a+c)+b & \aInNaturalwbInNaturalwcInNaturalImpLpaPlusbRpPluscEqualsLpaPluscRpPlusb{1,2,3} \\
    	1,2,3     &  (9) & (a+b)+(c+d)=((a+c)+b)+d & \rIE{8,7} \\
        1,2,3     & (10) & ((a+c)+b)+d=(a+c)+(b+d) &  \aInNaturalwbInNaturalwcInNaturalImpaPlusLpbPluscRpEqualsLpaPlusbRpPlusc{6,2,4}\\
        1,2,3     & (11) & (a+b)+(c+d)=(a+c)+(b+d) &  \rIE{10,9}\\
	\end{array}
	\]
\end{proof}

\section{Weitere Eigenschaften der Addition}

\label{aNotEqualsZeroImpaPlusbNotEqualsZero}
\begin{theorem}[\(a\neq 0\vdash a+b\neq 0\)]
Seien \(a,b\in\mathbb{N}\), dann gilt:
\[a\neq 0\vdash a+b\neq 0\]
\end{theorem}
\begin{proof}
Im Beweis nutzen wir die Eigenschaft, dass \((\mathbb{N},+,0)\) ein Monoid ist. Wir führen eine Induktion über \(n\in\mathbb{N}\):
        \[
	\begin{array}{llll}
            1       &  (1)  & a\neq 0 & \rA \\
                    &  (2)  & a+0=a & \rNeutralElementMonoid{} \\
            1       &  (3)  & a+0\neq 0 & \rIE{2,1} \\
            4       &  (4)  & a+n\neq 0 & \rA \\
                    &  (5)  & (a+n)+1\neq 0 & \nInNaturalImpnPlusOneNotEqualsZero{} \\       
                    &  (6)  & (a+n)+1=a+(n+1) & \rAssociativityMonoid{} \\      
                    &  (7)  & a+(n+1)\neq 0 & \rIE{6,5} \\     
            1       &  (8)  & \forall n\in\mathbb{N}(a+n\neq 0) & \rInductionN{3,4,7} \\  
            1       &  (9)  & a+b\neq 0 & \rSetUEc{8} \\  
	\end{array}
	\]
\end{proof}

\label{bNotEqualsZeroImpaPlusbNotEqualsZero}
\begin{theorem}[\(b\neq 0\vdash a+b\neq 0\)]
Seien \(a,b\in\mathbb{N}\), dann gilt:
\[b\neq 0\vdash a+b\neq 0\]
\end{theorem}
\begin{proof}
Im Beweis nutzen wir die Eigenschaft, dass \((\mathbb{N},+,0)\) ein abelscher Monoid ist. 
        \[
	\begin{array}{llll}
            1       &  (1)  & b\neq 0 & \rA \\
            1       &  (2)  & b+a\neq 0 & \aNotEqualsZeroImpaPlusbNotEqualsZero{1} \\
                    &  (3)  & a+b=b+a & \rCommutativeMonoid{} \\
            1       &  (4)  & a+b\neq 0 & \rIE{3,2} \\
	\end{array}
	\]
\end{proof}




\label{aEqualsbEqvaPluscEqualsbPlusc}
\begin{theorem}[\( a=b\dashv \vdash a+c=b+c\)]
Seien \(a,b,c\in\mathbb{N}\), dann gilt:
\[a=b\dashv\vdash a+c=b+c\]
\end{theorem}
\begin{proof}
Seien \(a,b,c\in\mathbb{N}\).
\(\vdash:\)
        \[
	\begin{array}{llll}
            1       &  (1)  & a=b & \rA \\
                    &  (2)  & a+c=a+c & \rII{} \\
            1       &  (3)  & a+c=b+c & \rIE{1,2} \\       
	\end{array}
	\]
 \(\dashv:\)
        Sei \(n\in\mathbb{N}\) eine natürliche Zahl, über die wir die Induktion führen, dann gilt:
        \[
	\begin{array}{llll}
            1       &  (1)  & a+c=b+c & \rA \\
            2       &  (2)  & a+0=b+0 & \rA \\
                    &  (3)  & a+0=a & \rNeutralElementMonoid{} \\
                    &  (4)  & b+0=b & \rNeutralElementMonoid{}  \\
            2       &  (5)  & a=b+0 & \rIE{3,2} \\       
            2       &  (6)  & a=b & \rIE{4,5} \\   
                    &  (7)  & a+0=b+0\rightarrow a=b & \rRI{2,6} \\
            8       &  (8)  & a+n=b+n\rightarrow a=b & \rA \\  
            9       &  (9)  & a+(n+1)=b+(n+1) & \rA \\  
                    &  (10)  & a+(n+1)=(a+n)+1 & \rAssociativityMonoid{} \\
                    &  (11)  & b+(n+1)=(b+n)+1 & \rAssociativityMonoid{} \\
            9       &  (12)  & (a+n)+1=b+(n+1) & \rIE{10,9} \\
            9       &  (13)  & ((a+n)+1)-1=((b+n)+1)-1 & \rPredecessorUniqueness{13}\\ 
                    &  (14)  & a+n=((a+n)+1)-1 & \rPredecessorEc{}\\
                    &  (15)  & b+n=((b+n)+1)-1 & \rPredecessorEc{}\\
            9       &  (16)  & a+n=((b+n)+1)-1 & \rIE{14,13}\\ 
            9       &  (17)  & a+n=b+n & \rIE{15,13}\\ 
            8,9     &  (18)  & a=b & \rRE{8,17}\\ 
            8       &  (19)  & a+(n+1)=b+(n+1)\rightarrow a=b & \rRI{9,18}\\ 
                    &  (20)  &  \forall n\in\mathbb{N}(a+n=b+n\rightarrow a=b) & \rInductionN{7,8,19}\\ 
                    &  (21)  &  a+c=b+c\rightarrow a=b & \rSetUEc{}\\
            1       &  (22)  &   a=b & \rRE{1,21}\\ 
	\end{array}
	\]
\end{proof}


\label{aNotEqualsbEqvaPluscNotEqualsbPlusc}
\begin{theorem}[\(a\neq b\dashv \vdash a+c\neq b+c\)]
Seien \(a,b,c\in\mathbb{N}\), dann gilt:
\[a\neq b\dashv\vdash a+c\neq b+c\]
\end{theorem}
\begin{proof}
Seien \(a,b,c\in\mathbb{N}\).
\(\vdash:\)
        \[
	\begin{array}{llll}
            1       &  (1)  & a\neq b & \rA \\
                    &  (2)  & a=b\leftrightarrow a+c=b+c & \aEqualsbEqvaPluscEqualsbPlusc{} \\
            1       &  (3)  & a+c=b+c & \PLrQwnPImpnQ{2,1} \\       
	\end{array}
	\]
 \(\dashv:\)
        \[
	\begin{array}{llll}
            1       &  (1)  & a+c\neq b+c & \rA \\
                    &  (2)  & a=b\leftrightarrow a+c=b+c & \aEqualsbEqvaPluscEqualsbPlusc{} \\
            1       &  (3)  & a\neq b & \PLrQwnQImpnP{2,1} \\    
	\end{array}
	\]
\end{proof}

\label{aEqualsbEqvcPlusaEqualscPlusb}
\begin{theorem}[\(a=b\dashv \vdash c+a=c+b\)]
Seien \(a,b,c\in\mathbb{N}\), dann gilt:
\[a=b\dashv\vdash c+a=c+b\]
\end{theorem}
\begin{proof}
Im Beweis nutzen wir \(\ImpLpNaturalwPluswZeroRpInAbelMonoid{}\).

\(\vdash:\)
    \[
	\begin{array}{llll}
            1       &  (1)  & a=b & \rA \\
            1       &  (2)  & a+c=b+c & \aEqualsbEqvaPluscEqualsbPlusc{1} \\
                    &  (3)  & a+c=c+a & \rCommutativeMonoid{} \\     
                    &  (4)  & b+c=c+b & \rCommutativeMonoid{} \\ 
            1       &  (5)  & c+a=b+c & \rIE{3,2} \\
            1       &  (6)  & c+a=c+b & \rIE{4,5} \\        
	\end{array}
	\]
 \(\dashv:\)
    \[
	\begin{array}{llll}
            1       &  (1)  & c+a=c+b & \rA \\
                    &  (2)  & a+c=c+a & \rCommutativeMonoid{} \\     
                    &  (3)  & b+c=c+b & \rCommutativeMonoid{} \\ 
            1       &  (4)  & a+c=c+b & \rIE{2,1} \\
            1       &  (5)  & a+c=b+c & \rIE{3,4} \\        
            1       &  (6)  & a=b & \aEqualsbEqvaPluscEqualsbPlusc{5} \\        
	\end{array}
	\]
\end{proof}

\label{aNotEqualsbEqvcPlusaNotEqualscPlusb}
\begin{theorem}[\(a\neq b\dashv \vdash c+a\neq c+b\)]
Seien \(a,b,c\in\mathbb{N}\), dann gilt:
\[a\neq b\dashv\vdash c+a\neq c+b\]
\end{theorem}
\begin{proof}
Seien \(a,b,c\in\mathbb{N}\).
\(\vdash:\)
        \[
	\begin{array}{llll}
            1       &  (1)  & a\neq b & \rA \\
                    &  (2)  & a=b\leftrightarrow c+a=c+b & \aEqualsbEqvcPlusaEqualscPlusb{} \\
            1       &  (3)  & c+a=c+b & \PLrQwnPImpnQ{2,1} \\       
	\end{array}
	\]
 \(\dashv:\)
        \[
	\begin{array}{llll}
            1       &  (1)  & a+c\neq b+c & \rA \\
                    &  (2)  & a=b\leftrightarrow c+a=c+b & \aEqualsbEqvcPlusaEqualscPlusb{} \\
            1       &  (3)  & a\neq b & \PLrQwnQImpnP{2,1} \\    
	\end{array}
	\]
\end{proof}

\label{aInNaturalwbInNaturalwaPlusbEqualsaImpbEqualsZero}
\begin{theorem}[\(a\in\mathbb{N}, b\in\mathbb{N}, a+b=a\vdash b=0\)]
\end{theorem}
\begin{proof}
        \[
	\begin{array}{llll}
            1       &  (1)  & a\in\mathbb{N} & \rA \\
            2       &  (2)  & b\in\mathbb{N} & \rA \\
            3       &  (3)  & a+b=a & \rA \\
            4       &  (4)  & 0+b=0 & \rA \\
            2       &  (5)  & b=0+b & \aInNaturalImpaEqualsZeroPlusa{2} \\
            2,4     &  (6)  & b=0 & \rIE{5,4} \\
            2       &  (7)  & 0+b=0\rightarrow b=0 & \rRI{4,6} \\
            8       &  (8)  & n+b=n\rightarrow b=0 & \rA \\
            9       &  (9)  & n\in\mathbb{N} & \rA \\
            10      &  (10)  & (n+1)+b=n+1 & \rA \\
                    &  (11)  & 1\in\mathbb{N} & \oneIsNaturalNumber{} \\
            2,9     &  (12) & (n+1)+b=(n+b)+1 & \aInNaturalwbInNaturalwcInNaturalImpLpaPlusbRpPluscEqualsLpaPluscRpPlusb{9,11,2} \\
            2,9     &  (13) & n+b\in\mathbb{N} & \aInNaturalwbInNaturalImpaPlusbInNatural{9,2} \\
            2,9     &  (14) & n+b=((n+b)+1)-1 & \rPredecessorEa{13} \\
            2,9     &  (15) & n=(n+1)-1 & \rPredecessorEc{9} \\
            9       &  (16) & n+1\in\mathbb{N} & \aInNaturalwbInNaturalImpaPlusbInNatural{9,11} \\
            2,9     &  (17) & (n+1)+b\in\mathbb{N} & \aInNaturalwbInNaturalImpaPlusbInNatural{16,2} \\
            9       &  (18) & n+1\neq 0 & \nInNaturalImpnPlusOneNotEqualsZero{9} \\
            2,9,10  &  (19) & ((n+1)+b)-1=(n+1)-1 & \rPredecessorUniqueness{16,17,18,10} \\
            2,9,10  &  (20) & ((n+1)+b)-1=n & \rIE{15,19} \\
            2,9,10  &  (21) & ((n+b)+1)-1=n & \rIE{12,20} \\
            2,9,10  &  (22) & n+b=n & \rIE{14,21} \\
            2,8,9,10&  (23) & b=0 & \rRE{22,8} \\
            2,8,9   &  (24) & (n+1)+b=n+1\rightarrow b=0 & \rRI{10,23} \\
            2       &  (25) & \forall n\in\mathbb{N}(n+b=n\rightarrow b=0) & \rInductionN{7,9,8,24} \\
            2       &  (26) & a\in\mathbb{N}\rightarrow (n+b=n\rightarrow b=0) & \rSetUEb{25} \\
            1,2     &  (27) & a+b=a\rightarrow b=0 & \rRE{1,26} \\
            28      &  (28) & a+b=a & \rA \\
            1,2,28  &  (29) & b=0 & \rRE{28,27} \\
	\end{array}
	\]
\end{proof}

\label{aInNaturalwbInNaturalwaPlusbEqualsZeroImpaEqualsZeroAndbEqualsZero}
\begin{theorem}[\(a\in\mathbb{N}, b\in\mathbb{N}, a+b=0\vdash a=0\land b=0\)]
\end{theorem}
\begin{proof}
        \[
	\begin{array}{llll}
            1       &  (1)  & a\in\mathbb{N} & \rA \\
            2       &  (2)  & b\in\mathbb{N} & \rA \\
            3       &  (3)  & a+b=0 & \rA \\
            4       &  (4)  & \neg(a=0\land b=0) & \rA \\
            4       &  (5)  & a\neq 0\lor b\neq 0 & \nLpPAndQRpEqvnPOrnQ{4} \\
            6       &  (6)  & a\neq 0 & \rA \\
            1,6     &  (7)  & \exists x\in\mathbb{N}(x+1=a) & \mInNaturalwmNotEqualsZeroImpExxInNaturalLpxPlusOneEqualsmRp{1,6} \\
            1,6     &  (8)  & n\in\mathbb{N}\land n+1=a & \rSetEEa{7} \\
            1,6     &  (9)  & n\in\mathbb{N} & \rAEa{8} \\
            1,6     &  (10) & n+1=a & \rAEb{8} \\
            1,3,6   &  (11) & (n+1)+b=0 & \rIE{10,3} \\
                    &  (12) & 1\in\mathbb{N} & \oneIsNaturalNumber{} \\
            1,3,6   &  (13) & (n+1)+b=(n+b)+1 & \aInNaturalwbInNaturalwcInNaturalImpLpaPlusbRpPluscEqualsLpaPluscRpPlusb{1,12,2} \\
            1,2,6   &  (14) & n+b\in\mathbb{N} & \aInNaturalwbInNaturalImpaPlusbInNatural{9,2} \\
            1,2,6   &  (15) & (n+b)+1\neq 0 & \nInNaturalImpnPlusOneNotEqualsZero{14} \\
            1,2,3,6 &  (16) & (n+1)+b\neq 0\in & \rIE{13,15} \\
            1,2,3,6 &  (17) & a+b\neq 0 & \rIE{10,16} \\
            1,2,3,6 &  (18) & \bot & \rBI{3,17} \\
            1,2,3   &  (19) & a=0 & \rCE{6,18} \\
            1,2,3   &  (20) & 0+b=0 & \rIE{19,3} \\
            1,2,3   &  (21) & b=0+b & \aInNaturalImpaEqualsZeroPlusa{2} \\
            1,2,3   &  (22) & b=0 & \rIE{21,20} \\
            1,2,3   &  (23) & a=0\land b=0 & \rAI{19,22} \\
	\end{array}
	\]
\end{proof}

\chapter{Ordnungsrelationen und Differenzen für natürliche Zahlen}

\begin{definition}[Kleiner-gleich (\( \leq \))]
Für natürliche Zahlen \( a, b \in \mathbb{N} \) gilt:

\[a\leq b:=\exists c\in\mathbb{N}(a+c=b)\]
\end{definition}

\label{rule:rLeqNI} \label{rule:rLeqNE}
\paragraph{Beweisregeln für \( \leq \)}
Basierend auf der Definition der Kleiner-gleich-Relation (\( \leq \)) können wir die folgenden Regeln formulieren:

\[
\begin{array}{llll}
	i   & (1)      & a \in \mathbb{N}                  & ... \\
	j   & (2)      & b \in \mathbb{N}                  & ... \\ 
	k   & (3)      & \exists c\in\mathbb{N}(a + c = b) & ... \\
	i,j,k & (4)    & a \leq b                          & \rLeqNI{1,2,3} \\
\end{array}
\]

\[
\begin{array}{llll}
	i   & (1)      & a \in \mathbb{N}                  & ... \\
	j   & (2)      & b \in \mathbb{N}                  & ... \\ 
        k   & (3)      & c \in \mathbb{N}                  & ... \\ 
	l   & (3)      & a + c = b & ... \\
	i,j,k,l & (4)    & a \leq b                          & \rLeqNI{1,2,3,4} \\
\end{array}
\]

\[
\begin{array}{llll}
	i      & (1)    & a \in \mathbb{N}                  & ... \\
	j      & (2)    & b \in \mathbb{N}                  & ... \\ 
	k      & (3)    & a\leq b & ... \\
	i,j,k  & (4)    & \exists c\in\mathbb{N}(a + c = b) & \rLeqNE{1,2,3} \\
\end{array}
\]
\( i \), \( j \), \( k \) und \(l\) sind dabei Listen von Annahmen.

\section{Eindeutigkeit der Ordnungsrelation}

\label{ExcSubOnewcSubTwoInNaturalLpaPluscSubOneEqualsbAndaPluscSubTwoEqualsbRpImpcSubOneEqualscSubTwo}
\begin{theorem}[\(\exists c_1, c_2 \in \mathbb{N} (a + c_1 = b \land a + c_2 = b) \vdash c_1 = c_2\) (Eindeutigkeit der Differenz)]
    Seien \( a, b \in \mathbb{N} \) natürliche Zahlen, und es gelte \( a + c_1 = b \) sowie \( a + c_2 = b \) für \( c_1, c_2 \in \mathbb{N} \). Dann folgt, dass \( c_1 = c_2 \).
\end{theorem}
\begin{proof}
    Seien \(a,b\in\mathbb{N}\), dann gilt:
        \[
	\begin{array}{llll}
            1   &  (1) & \exists c_1,c_2\in \mathbb{N}(a+c_1=b\land a+c_2=b) & \rA \\
            1   &  (2) & \exists c_1\exists c_2\in \mathbb{N}(a+c_1=b\land a+c_2=b) & \rSetEEm{1} \\
            3   &  (3) & c_1\in\mathbb{N}\land \exists c_2\in \mathbb{N}(a+c_1=b\land a+c_2=b) & \rA \\
            3   &  (4) & c_1\in\mathbb{N} & \rAEa{3} \\
            3   &  (5) & \exists c_2\in \mathbb{N}(a+c_1=b\land a+c_2=b) & \rAEb{3} \\
            6   &  (6) & c_2\in\mathbb{N}\land (a+c_1=b\land a+c_2=b) & \rA \\
            6   &  (7) & c_2\in\mathbb{N} & \rAEa{6} \\
            6   &  (8) & a+c_1=b\land a+c_2=b & \rAEb{6} \\
            6   &  (9) & a+c_1=b & \rAEa{8} \\
            6   &  (10) & a+c_2=b & \rAEb{8} \\
            6   &  (11) & a+c_1=a+c_2 & \rIE{10,9} \\
            6   &  (12) & c_1=c_2 & \aEqualsbEqvcPlusaEqualscPlusb{11} \\
            3   &  (13) & c_1=c_2 & \rEE{5,6,12} \\
            1   &  (14) & c_1=c_2 & \rEE{2,3,13} \\
    \end{array}
	\]
\end{proof}

\section{Definition der Differenz}

\begin{definition}[Differenz (\( b - a \))]
    Sei \( a, b \in \mathbb{N} \). Eine partielle Definition erlaubt es, die Differenz \( b - a \) wie folgt einzuführen:
    \[
    \forall a, b \in \mathbb{N} \, [ a \leq b \rightarrow ( b - a \coloneqq \iota c \, (c \in \mathbb{N} \land a + c = b) ) ].
    \]
\end{definition}

\subsubsection*{Einführungsregel für die Differenz}
\label{rule:minusI}

\[
\begin{array}{llll}
    i & (1) & a\in\mathbb{N} & ... \\
    j & (2) & b\in\mathbb{N} & ... \\
    k & (3) & a\leq b & ... \\
    i,j,k & (4) & b-a\in\mathbb{N} & \minusI{1,2,3} \\
    i,j,k & (5) & b-a=b-a & \minusI{1,2,3} \\
    i,j,k & (6) & a+(b-a)=b & \minusI{1,2,3} \\
\end{array}
\]

\[
\begin{array}{llll}
    i & (1) & a\in\mathbb{N} & ... \\
    j & (2) & b\in\mathbb{N} & ... \\
    k & (3) & c\in\mathbb{N} & ... \\
    l & (4) & a+c=b & ... \\
    i,j,k,l & (5) & (b-a)=c & \minusI{1,2,3} \\
    i,j,k,l & (6) & c=(b-a) & \minusI{1,2,3} \\
\end{array}
\]
\(i, j\) und \(k\) sind dabei Listen von Annahmen.

\subsubsection*{Eliminierungsregel für die Differenz}
\label{rule:minusE}

\[
\begin{array}{llll}
    i & (1) & a\in\mathbb{N} & ... \\
    j & (2) & b\in\mathbb{N} & ... \\
    k & (3) & a-b\in\mathbb{N} & ... \\
    i,j,k & (4) & a\leq b & \minusE{1,2,3} \\
    i,j,k & (4) & a+(b-a)=b & \minusE{1,2,3} \\
\end{array}
\]

\(i, j\) und \(k\) sind dabei Listen von Annahmen.


\begin{definition}[Subtraktion als Funktion]
    Wir definieren die Subtraktionsfunktion \( - : \{ (a, b) \in \mathbb{N} \times \mathbb{N} \mid b \leq a \} \to \mathbb{N} \) mit
    durch
    \[
    -(a, b) := a-b
    \]
\end{definition}

\section{Eigenschaften von Ordnungsrelation und Differenz}

\label{awbInNaturalwaEqualsbImpaLeqb}
\begin{theorem}[\(a,b\in\mathbb{N},a=b\vdash a\leq b\)]
\end{theorem}
\begin{proof}
    Seien \(a,b\in\mathbb{N}\), \(\ImpLpNaturalwPluswZeroRpInAbelMonoid{}\) und daraus folgt:
        \[
	\begin{array}{lllcll}
                  &  (1) & a+0&=&a & \rNeutralElementMonoid{} \\
                2 &  (2) & &=&b & \rA \\
                2 &  (3) & a+0&=&b & \rTransitivityEqRI{1,2} \\
                2 &  (4) & \multicolumn{3}{l}{a\leq b} & \rLeqNI{3} \\
    \end{array}
	\]
\end{proof}

\label{awbInNaturalwaEqualsbImpbLeqa}
\begin{theorem}[\(a,b\in\mathbb{N},a=b\vdash b\leq a\)]
\end{theorem}
\begin{proof}
    Seien \(a,b\in\mathbb{N}\), \(\ImpLpNaturalwPluswZeroRpInAbelMonoid{}\) und daraus folgt:
        \[
	\begin{array}{lllcll}
                1 &  (1) & \multicolumn{3}{l}{a=b} & \rA \\
                1 &  (2) & \multicolumn{3}{l}{b=a} & \rSymmetryEqRI{1} \\
                1 &  (3) & \multicolumn{3}{l}{b\leq a}& \awbInNaturalwaEqualsbImpaLeqb{2} \\
    \end{array}
	\]
\end{proof}


\label{awbInNaturalaLeqbImpbMinusaLeqb}
\begin{theorem}[\(a,b\in\mathbb{N},a\leq b\vdash b-a\leq b\)]
\end{theorem}
\begin{proof}
    Seien \(a,b\in\mathbb{N}\), \(\ImpLpNaturalwPluswZeroRpInAbelMonoid{}\) und daraus folgt:
        \[
	\begin{array}{lllcll}
                1 &  (1) & \multicolumn{3}{l}{a\leq b} & \rA \\
                  &  (2) & (b-a)+a&=&a+(b-a) & \rCommutativeMonoid{} \\
                1 &  (3) & &=&b & \minusE{1} \\
                1 &  (4) & (b-a)+a&=&b & \rTransitivityEqRI{2,3} \\
                1 &  (5) & \multicolumn{3}{l}{(b-a)\leq b} & \rLeqNI{4} \\
    \end{array}
	\]
\end{proof}


\subsection{Differenz zu Nachfolger und Vorgänger}

\label{aInNaturalImpLpaPlusOneRpMinusaEqualsOne}
\begin{theorem}[\(a\in\mathbb{N}\vdash (a+1)-a=1\)]
\end{theorem}
\begin{proof}
    Seien \(a\in\mathbb{N}\), dann gilt:
        \[
	\begin{array}{llll}
                &  (1) & a=a & \rII{} \\
                &  (2) & a+1=a+1 & \aEqualsbEqvaPluscEqualsbPlusc{1} \\
                &  (3) & (a+1)-a=1 & \minusI{2} \\
    \end{array}
	\]
\end{proof}

\label{aInNaturalwaNotEqualsZeroImpaMinusLpaMinusOneRpEqualsOne}
\begin{theorem}[\(a\in\mathbb{N},a\neq 0\vdash a-(a-1)=1\)]
Seien \(a\in\mathbb{N}\), dann gilt:
\[a\neq 0\vdash a-(a-1)=1\]
\end{theorem}
\begin{proof}
    Sei \(a\in\mathbb{N}\), dann gilt:
        \[
	\begin{array}{llll}
            1   &  (1) & a\neq 0 & \rA \\
            1   &  (2) & a=(a-1)+1 & \rPredecessorI{1} \\
            1   &  (3) & a-(a-1)=1 & \minusI{3} \\
    \end{array}
	\]
\end{proof}

\subsection{Ordnungsrelation mit Nachfolger und Vorgänger}

\label{aInNaturalImpaLeqaPlusOne}
\begin{theorem}[\(a\in\mathbb{N}\vdash a\leq a+1\)]
\end{theorem}
\begin{proof}
    Seien \(a\in\mathbb{N}\), dann gilt:
        \[
	\begin{array}{llll}
                &  (1) & (a+1)-a=1 & \aInNaturalImpLpaPlusOneRpMinusaEqualsOne{} \\
                &  (2) & a\leq a+1 & \minusE{1} \\
    \end{array}
	\]
\end{proof}

\label{aInNaturalwbInNaturalwaEqualsbImpaLeqbPlusOne}
\begin{theorem}[\(a\in\mathbb{N},b\in\mathbb{N},a=b\vdash a\leq b+1\)]
Seien \(a,b\in\mathbb{N}\), dann gilt:
\[a=b\vdash a\leq b+1\]
\end{theorem}
\begin{proof}
    Seien \(a,b\in\mathbb{N}\), dann gilt:
        \[
	\begin{array}{llll}
            1   &  (1) & a=b & \rA \\
                &  (2) & a\leq a+1 & \aInNaturalImpaLeqaPlusOne{} \\
            1   &  (3) &  a\leq b+1 & \rIE{1,2} \\
    \end{array}
	\]
\end{proof}

\label{aInNaturalwbInNaturalwaEqualsbImpbLeqaPlusOne}
\begin{theorem}[\(a\in\mathbb{N},b\in\mathbb{N},a=b\vdash b\leq a+1\)]
Seien \(a,b\in\mathbb{N}\), dann gilt:
\[a=b\vdash b\leq a+1\]
\end{theorem}
\begin{proof}
    Seien \(a,b\in\mathbb{N}\), dann gilt:
        \[
	\begin{array}{llll}
            1   &  (1) & a=b & \rA \\
            1   &  (2) & b=a & \aIdbImpbIda{1} \\
            1   &  (3) & b\leq a+1 & \aInNaturalwbInNaturalwaEqualsbImpaLeqbPlusOne{2} \\
    \end{array}
	\]
\end{proof}

\label{aInNaturalwaNotEqualsZeroImpLpaMinusOneRpLeqa}
\begin{theorem}[\(a\in\mathbb{N},a\neq 0\vdash (a-1)\leq a\)]
Seien \(a\in\mathbb{N}\), dann gilt:
\[(a-1)\leq a\]
\end{theorem}
\begin{proof}
    Sei \(a\in\mathbb{N}\), dann gilt:
        \[
	\begin{array}{llll}
            1   &  (1) & a\neq 0 & \rA \\
            1   &  (2) & a-(a-1)=1 & \rPredecessorI{1} \\
            1   &  (3) & (a-1)\leq a & \minusI{3} \\
    \end{array}
	\]
\end{proof}

\label{aInNaturalwbInNaturalwaNotEqualsZerowaEqualsbImpaMinusOneLeqb}
\begin{theorem}[\(a\in\mathbb{N},b\in\mathbb{N},a\neq 0, a=b\vdash a-1\leq b\)]
Seien \(a,b\in\mathbb{N}\), dann gilt:
\[a\neq 0, a=b\vdash a-1\leq b\]
\end{theorem}
\begin{proof}
    Seien \(a,b\in\mathbb{N}\), dann gilt:
        \[
	\begin{array}{llll}
            1     &  (1) & a\neq 0 & \rA \\
            2     &  (2) & a=b & \rA \\
            1   &  (3) & a-1\leq a\in\mathbb{N} & \aInNaturalwaNotEqualsZeroImpLpaMinusOneRpLeqa{1} \\
            1,2   &  (4) & a-1\leq b & \rIE{2,3} \\
    \end{array}
	\]
\end{proof}

\label{aInNaturalwbInNaturalwaNotEqualsZerowaEqualsbImpbMinusOneLeqa}
\begin{theorem}[\(a\in\mathbb{N},b\in\mathbb{N},a\neq 0, a=b\vdash b-1\leq a\)]
Seien \(a,b\in\mathbb{N}\), dann gilt:
\[a\neq 0, a=b\vdash b-1\leq a\]
\end{theorem}
\begin{proof}
    Seien \(a,b\in\mathbb{N}\), dann gilt:
        \[
	\begin{array}{llll}
            1     &  (1) & a\neq 0 & \rA \\
            2     &  (2) & a=b & \rA \\
            2     &  (3) & b=a & \aIdbImpbIda{2} \\
            2     &  (4) & b\neq 0 & \rIE{2,1} \\
            1,2   &  (5) & b-1\leq a & \aInNaturalwbInNaturalwaNotEqualsZerowaEqualsbImpaMinusOneLeqb{5,4} \\
    \end{array}
	\]
\end{proof}

\subsection{Eigenschaften der Differenz in Bezug auf Null}

\label{aInNaturalImpaMinusZeroEqualsa}
\begin{theorem}[\(a\in\mathbb{N}\vdash a-0=a\)]
\end{theorem}
\begin{proof}
    Sei \(a\in\mathbb{N}\). \(\ImpLpNaturalwPluswZeroRpInAbelMonoid{}\) und daraus folgt:
        \[
	\begin{array}{llll}
            &  (1) & 0\leq a & \rLeqNI{1} \\
            &  (2) & 0+(a-0)=a & \rLeqNI{1} \\
            &  (3) & 0+(a-0)=a-0 & \rNeutralElementMonoid{} \\
            &  (4) & a-0=a & \rIE{3,2} \\
    \end{array}
	\]
\end{proof}



\label{aInNaturalImpaMinusaEqualsZero}
\begin{theorem}[\(a\in\mathbb{N}\vdash a-a=0\)]
\end{theorem}
\begin{proof}
        Sei \(a\in\mathbb{N}\). Wir wissen: \(\LeqIsTotalOrderOnNaturalNumbers{}\), woraus folgt:
        \[
	\begin{array}{llll}
                &  (1) & a\leq a & \rReflexivityOrdRI{} \\
                &  (2) & a+(a-a)=a & \minusI{1} \\
                &  (3) & a-a=0 & \aInNaturalwbInNaturalwaPlusbEqualsaImpbEqualsZero{2} \\
    \end{array}
	\]
\end{proof}

\label{awbInNaturalLpaEqualsbEqvaMinusbEqualsZeroRp}
\begin{theorem}[\(a=b\dashv\vdash a-b=0\)]
Seien \(a,b\in\mathbb{N}\), dann gilt:
\[a=b\dashv\vdash a-b=0\]
\end{theorem}
\begin{proof}
        Seien \(a,b\in\mathbb{N}\), dann gilt:
\(\vdash:\)
	\[
	\begin{array}{llll}
		1 & (1) & a=b & \rA \\
		   & (2) & a-a=0 & \aInNaturalImpaMinusaEqualsZero{1} \\
            1 & (3) & a-b=0 & \rIE{1,2} \\
	\end{array}
	\]
	\(\dashv:\)
    \(\ImpLpNaturalwPluswZeroRpInMonoid{}\) und daraus folgt:
	\[
	\begin{array}{llll}
		1 & (1) & a-b = 0 & \rA \\
            1 & (2) & a+(a-b)=b & \minusE{1} \\
            1 & (3) & a+0=b & \rIE{1,2} \\
              & (4) & a+0=a & \rNeutralElementMonoid{} \\
            1 & (5) & a=b & \rIE{4,3} \\
	\end{array}
	\]
\end{proof}

\label{awbInNaturalLpbEqualsaEqvaMinusbEqualsZeroRp}
\begin{theorem}[\(b=a\dashv\vdash a-b=0\)]
Seien \(a,b\in\mathbb{N}\), dann gilt:
\[a=b\dashv\vdash a-b=0\]
\end{theorem}
\begin{proof}
        Seien \(a,b\in\mathbb{N}\), dann gilt:
\(\vdash:\)
	\[
	\begin{array}{llll}
		1 & (1) & b=a & \rA \\
		  1 & (2) & a=b & \rSymmetryEqRI{1} \\
            1 & (3) & a-b=0 & \awbInNaturalLpaEqualsbEqvaMinusbEqualsZeroRp{1,2} \\
	\end{array}
	\]
	\(\dashv:\)
	\[
	\begin{array}{llll}
		1 & (1) & a-b = 0 & \rA \\
            1 & (2) & a=b & \awbInNaturalLpaEqualsbEqvaMinusbEqualsZeroRp{} \\
            1 & (3) & b=a & \rSymmetryEqRI{2} \\
	\end{array}
	\]
\end{proof}

\label{aNotEqualsbEqvaMinusbNotEqualsZero}
\begin{theorem}[\(a\neq b\dashv\vdash a-b\neq 0\)]
Seien \(a,b\in\mathbb{N}\), dann gilt:
\[a\neq b\dashv\vdash a-b\neq 0\]
\end{theorem}
\begin{proof}
        Seien \(a,b\in\mathbb{N}\), dann gilt:
        \[
	\begin{array}{llll}
		   & (1) & a=b\leftrightarrow a-b=0 & \awbInNaturalLpaEqualsbEqvaMinusbEqualsZeroRp{} \\
		   & (2) & a\neq b\leftrightarrow a-b\neq 0 & \PLrQEqvnPLrnQ{1} \\
	\end{array}
	\]
\end{proof}



\label{bNotEqualsaEqvaMinusbNotEqualsZero}
\begin{theorem}[\(b\neq a\dashv\vdash a-b\neq 0\)]
Seien \(a,b\in\mathbb{N}\), dann gilt:
\[b\neq a\dashv\vdash b-a\neq 0\]
\end{theorem}
\begin{proof}
        Seien \(a,b\in\mathbb{N}\), dann gilt:
        \[
	\begin{array}{llll}
		   & (1) & b=a\leftrightarrow a-b=0 & \awbInNaturalLpaEqualsbEqvaMinusbEqualsZeroRp{} \\
		   & (2) & b\neq a\leftrightarrow a-b\neq 0 & \PLrQEqvnPLrnQ{1} \\
	\end{array}
	\]
\end{proof}

\label{awbInNaturalLpaLeqbEqvZeroLeqbMinusaRp}
\begin{theorem}[\(a\leq b\dashv\vdash 0\leq b-a\)]
Seien \(a,b\in\mathbb{N}\), dann gilt:
\[a\leq b\dashv\vdash 0\leq b-a\]
\end{theorem}
\begin{proof}
        Seien \(a,b\in\mathbb{N}\), dann gilt:
\(\vdash:\)
	\[
	\begin{array}{llll}
		1 & (1) & a\leq b & \rA \\
		  1 & (2) & b-a\in\mathbb{N} & \minusI{1} \\
            1 & (3) & 0\leq b-a & \ImpZeroLeqa{2} \\
	\end{array}
	\]
	\(\dashv:\)
	\[
	\begin{array}{llll}
		1 & (1) & 0\leq b-a & \rA \\
            1 & (2) & ((b-a)-0)\in\mathbb{N} & \minusI{1} \\
            1 & (3) & (b-a)-0=b-a & \aInNaturalImpaMinusZeroEqualsa{2} \\
            1 & (4) & (b-a)\in\mathbb{N} & \rIE{3,2} \\
            1 & (3) & a\leq b & \minusE{4} \\
	\end{array}
	\]
\end{proof}


\label{ImpZeroLeqa}
\begin{theorem}[\(0\leq a\)]
Sei \(a\in\mathbb{N}\), dann gilt:
\[\vdash 0\leq a\]
\end{theorem}
\begin{proof}
    Seien \(a\in\mathbb{N}\). \(\ImpLpNaturalwPluswZeroRpInAbelMonoid{}\) und daraus folgt:
        \[
	\begin{array}{llll}
            &  (1) & 0+a=a & \rNeutralElementMonoid{} \\
            &  (2) & 0\leq a & \rLeqNI{1} \\
    \end{array}
	\]
\end{proof}

\label{aInNaturalwaNotEqualsZerowOneLeqa}
\begin{theorem}[\(a\in\mathbb{N}, a\neq 0, 1\leq a\)]
\end{theorem}
\begin{proof}
    Seien \(a\in\mathbb{N}\). \(\ImpLpNaturalwPluswZeroRpInAbelMonoid{}\) und daraus folgt:
        \[
	\begin{array}{llclll}
         1   &  (1) & \multicolumn{3}{l}{a\neq 0} & \rA \\
         1   &  (2) & a&=&(a-1)+1 & \rPredecessorI{1} \\
             &  (3) & &=&1+(a-1) & \rCommutativeMonoid{} \\
         1   &  (4) & a&=&1+(a-1) & \rTransitivityEqRI{2,3} \\
         1   &  (5) & \multicolumn{3}{l}{1\leq a} & \rLeqNI{4} \\
    \end{array}
	\]
\end{proof}

\label{aInNaturalwaLeqZeroImpaEqualsZero}
\label{aLeqZeroImpaEqualsZero}
\begin{theorem}[\(a\in\mathbb{N},a\leq 0\vdash a=0\)]
Sei \(a\in\mathbb{N}\), dann gilt:
\[a\leq 0\vdash a=0\]
\end{theorem}
\begin{proof}
    Seien \(a\in\mathbb{N}\), dann gilt:
        \[
	\begin{array}{llll}
           1 &  (1) & a\leq 0 & \rA{} \\
             &  (2) & 0\leq a & \ImpZeroLeqa{} \\
           1 &  (3) & a=0 & \rAntisymmetryOrdRI{1,2} \\
    \end{array}
	\]
\end{proof}



\subsection{Eigenschaften der Halbordnung}

\label{aInNaturalwbInNaturalwaLeqbwaNotEqualsbImpaPlusOneLeqb}
\begin{theorem}[\(a\in\mathbb{N},b\in\mathbb{N}, a\leq b,a\neq b\vdash a+1\leq b\)]
Seien \(a,b\in\mathbb{N}\), dann gilt:
\[a\leq b,a\neq b\vdash a+1\leq b\]
\end{theorem}
\begin{proof}
    Seien \(a,b\in\mathbb{N}\). \(\ImpLpNaturalwPluswZeroRpInAbelMonoid{}\) und daraus folgt:
        \[
	\begin{array}{llll}
            1     &  (1) & a\leq b & \rA \\
            2     &  (2) & a\neq b & \rA \\
            1     &  (3) & a+(b-a)=b & \minusI{1} \\
            4     &  (4) & b-a=0 & \rA \\
            1,4   &  (5) & a+0=b & \rIE{4,3} \\
                  &  (6) & a+0=a & \rNeutralElementMonoid{} \\
            1,4   &  (7) & a=b & \rIE{6,5} \\
            1,2,4 &  (8) & \bot & \rBI{2,7} \\
            1,2   &  (9) & b-a\neq 0 & \rCI{4,8} \\
            1,2   &  (10) & b-a=((b-a)-1)+1 & \rPredecessorI{4,8} \\
                  &  (11) & a+(((b-a)-1)+1)=(a+1)+((b-a)-1) & \aInMwbInMwcInMImpaPlusLpbPluscRpEqualsLpaPluscRpPlusb{} \\
            1,2   &  (12) & a+(((b-a)-1)+1)=b & \rIE{2,10} \\
            1,2   &  (13) & (a+1)+((b-a)-1)=b & \rIE{11,12} \\
            1,2   &  (14) & (a+1)\leq b & \rLeqNI{13} \\
    \end{array}
	\]
\end{proof}

\label{aInNaturalImpaLeqa}
\begin{theorem}[\(a\in\mathbb{N}\vdash a\leq a\) (Reflexivität)]
\end{theorem}
\begin{proof}
        Sei \(a\in\mathbb{N}\). \(\ImpLpNaturalwPluswZeroRpInAbelMonoid{}\) und daraus folgt:
        \[
	\begin{array}{llll}
                &  (1) & a+0=a & \rNeutralElementMonoid{} \\
                &  (2) & a\leq a & \rLeqNI{1} \\
    \end{array}
	\]
\end{proof}

\label{aInNaturalwbInNaturalwaLeqbwbLeqaImpaEqualsb}
\begin{theorem}[\(a\in\mathbb{N},b\in\mathbb{N},a\leq b,b\leq a\vdash a = b\) (Antisymmetrie)]
\end{theorem}
\begin{proof}
        Seien \(a,b\in\mathbb{N}\). \(\ImpLpNaturalwPluswZeroRpInAbelMonoid{}\) und daraus folgt:
        \[
	\begin{array}{llll}
            1       &  (1) & a\leq b & \rA \\
            2       &  (2) & b\leq a & \rA \\
            1       &  (3) & a+(b-a)=b & \minusI{1} \\
            2       &  (4) & b+(a-b)=a & \minusI{2} \\
            1,2     &  (5) & (a+(b-a))+(a-b)=a & \rIE{3,4} \\
                    &  (6) & (a+(b-a))+(a-b)=a+((b-a)+(a-b)) & \rAssociativityMonoid{} \\
            1,2     &  (7) & a+((b-a)+(a-b))=a & \rIE{6,5} \\
            1,2     &  (8) & (b-a)+(a-b)=0 & \aInNaturalwbInNaturalwaPlusbEqualsaImpbEqualsZero{7} \\     
            1,2     &  (9) & (b-a)=0\land (a-b)=0 & \aInNaturalwbInNaturalwaPlusbEqualsZeroImpaEqualsZeroAndbEqualsZero{8} \\
            1,2     &  (10) & (b-a)=0 & \rAEa{9} \\
            1,2     &  (11) & a+0=b & \rIE{10,3} \\
            1,2     &  (12) & a+0=b & \rIE{10,3} \\
                    &  (13) & a+0=a & \rNeutralElementMonoid{} \\
            1,2     &  (14) & a=b & \rIE{13,12} \\
    \end{array}
	\]
\end{proof}

\label{aInNaturalwbInNaturalwaLeqbwbLeqcImpaLeqc}
\begin{theorem}[Transitivität der Ordnungsrelation auf \(\mathbb{N}\)]
Seien \(a,b,c\in\mathbb{N}\), dann gilt:
\[a\leq b,b\leq c\vdash a\leq c\]
\end{theorem}
\begin{proof}
    Seien \(a,b,c\in\mathbb{N}\). \(\ImpLpNaturalwPluswZeroRpInAbelMonoid{}\) und daraus folgt:
        \[
	\begin{array}{llll}
            1       &  (1) & a\leq b & \rA \\
            2       &  (2) & b\leq c & \rA \\
            1       &  (3) & a+(b-a)=b & \minusI{1} \\
            2       &  (4) & b+(c-b)=c & \minusI{1} \\
            1,2     &  (5) & (a+(b-a))+(c-b)=c & \rIE{3,4} \\
                    &  (6) & (a+(b-a))+(c-b)=a+((b-a)+(c-b)) & \rAssociativityMonoid{} \\
            1,2     &  (7) & a+((b-a)+(c-b))=c & \rIE{6,5} \\
            1,2     &  (8) & a\leq c & \rLeqNI{7} \\
    \end{array}
	\]
\end{proof}

\label{LeqIsHalfOrderOnNaturalNumbers}
\begin{theorem}[\(\leq\) ist eine Halbordnung auf \(\mathbb{N}\) ]
\end{theorem}
\begin{proof}
        \[
	\begin{array}{llll}
                    & (1) & \forall a \in \mathbb{N}  (a \leq a) & \aInNaturalImpaLeqa{} \\
                    & (2) & \forall a, b \in \mathbb{N}  ((a \leq b \land b \leq a) \rightarrow a = b) & \aInNaturalwbInNaturalwaLeqbwbLeqaImpaEqualsb{} \\
                    & (3) & \forall a, b, c \in \mathbb{N}  ((a \leq b \land b \leq c) \rightarrow a \leq c) & \aInNaturalwbInNaturalwaLeqbwbLeqcImpaLeqc{} \\
                    & (4) & \leq \text{ ist eine Halbordnung auf } \mathbb{N} &  \rPartialOrderRelationI{1,2,3}
    \end{array}
	\]
\end{proof}

\subsection{Invarianz der Ordnungsrelation gegenüber Addition}

\label{awbwcInNaturalLpaLeqbEqvaPluscLeqbPluscRp}
\begin{theorem}[\(a\leq b\dashv\vdash a+c\leq b+c\)]
Seien \(a,b,c\in\mathbb{N}\), dann gilt:
\[a\leq b\dashv\vdash a+c\leq b+c\]
\end{theorem}
\begin{proof}
Wir beweisen die Äquivalenz in drei Schritten:
\begin{enumerate}
    \item Beweis von Lemma 1: \(a+(b-c)=b \vdash (a+c)+(b-c)=b+c\).
    \item Beweis von Lemma 2: \((a+c)+((b+c)-(a+c))=b+c \vdash a+((b+c)-(a+c))=b\).
    \item Beweis der Äquivalenz \(a\leq b \dashv\vdash a+c\leq b+c\) unter Verwendung von Lemma 1 und Lemma 2.
\end{enumerate}

Seien \(a, b, c \in \mathbb{N}\). Im Beweis nutzen wir die Eigenschaft \(\ImpLpNaturalwPluswZeroRpInMonoid{}\).

\begin{longtable}{llclll}
\multicolumn{6}{l}{\textbf{Lemma 1:} \quad \(a + (b - c) = b \vdash (a + c) + (b - c) = b + c\)} \\
1     &  (1) & \(a+(b-c)\) & = & \(b\) & \rA \\
      &  (2) & \((a+c)+(b-c)\) & = & \(a+((b-c)+c)\) & \aInMwbInMwcInMImpaPlusLpbPluscRpEqualsLpaPluscRpPlusb{} \\
      &  (3) &               & = & \((a+(b-c))+c\) & \rAssociativityMonoid{2} \\
1     &  (4) &               & = & \(b+c\) & \rIE{3,1} \\
\\
\multicolumn{6}{l}{\textbf{Lemma 2:} \quad \((a + c) + ((b + c) - (a + c)) = b + c \vdash a + ((b + c) - (a + c)) = b\)} \\
\multicolumn{6}{l}{\text{Im Beweis verwenden wir indirekt die Regel } \aIdbImpbIda{}.} \\
1     &  (1) & \(b+c\) & = & \((a+c)+((b+c)-(a+c))\) & \rA{} \\
1     &  (2) &         & = & \(a+(((b+c)-(a+c))+c)\) & \aInMwbInMwcInMImpaPlusLpbPluscRpEqualsLpaPluscRpPlusb{1} \\
1     &  (3) &         & = & \((a+((b+c)-(a+c)))+c\) & \rAssociativityMonoid{2} \\
1     &  (4) & \(b\)   & = & \(a+((b+c)-(a+c))\) & \aEqualsbEqvaPluscEqualsbPlusc{3} \\
\\
\multicolumn{6}{l}{\textbf{Beweis der Äquivalenz:} \quad \(a \leq b \dashv\vdash a + c \leq b + c\)} \\
\multicolumn{6}{l}{\(\vdash:\)} \\
1     &  (1) & \multicolumn{3}{l}{\(a \leq b\)} & \rA \\
1     &  (2) & \multicolumn{3}{l}{\(a + (b - a) = b\)} & \minusI{1} \\
1     &  (3) & \multicolumn{3}{l}{\((a + c) + (b - c) = b + c\)} & \text{Lemma 1(2)} \\
1     &  (4) & \multicolumn{3}{l}{\(a + c \leq b + c\)} & \rLeqNI{3} \\
\multicolumn{6}{l}{\(\dashv:\)} \\
1     &  (1) & \multicolumn{3}{l}{\(a + c \leq b + c\)} & \rA \\
1     &  (2) & \multicolumn{3}{l}{\(a + c + ((b + c) - (a + c)) = b + c\)} & \minusI{1} \\
1     &  (3) & \multicolumn{3}{l}{\(a + ((b + c) - (a + c)) = b\)} & \text{Lemma 2(2)} \\
1     &  (4) & \multicolumn{3}{l}{\(a \leq b\)} & \rLeqNI{3} \\
\end{longtable}

\end{proof}

\label{awbwcInNaturalLpaLeqbEqvcPlusaLeqcPlusbRp}
\begin{theorem}[\(a\leq b\dashv\vdash c+a\leq c+b\)]
Seien \(a,b,c\in\mathbb{N}\), dann gilt:
\[a\leq b\dashv\vdash c+a\leq c+b\]
\end{theorem}
\begin{proof}
Seien \(a, b, c \in \mathbb{N}\). Im Beweis nutzen wir die Eigenschaft \(\ImpLpNaturalwPluswZeroRpInMonoid{}\).
\(\vdash\):
        \[
	\begin{array}{llclll}
            1     &  (1) & \multicolumn{3}{l}{a\leq b} & \rA \\
                  &  (2) & c+a&=&a+c & \rCommutativeMonoid{} \\
            1     &  (3) &    &\leq& c+b & \awbwcInNaturalLpaLeqbEqvaPluscLeqbPluscRp{1} \\
            1     &  (4) &    &\leq& b+c & \rCommutativeMonoid{} \\
    \end{array}
	\]
\(\dashv\):
        \[
	\begin{array}{llclll}
                  &  (1) & a+c&=&c+a & \rCommutativeMonoid{} \\
            2     &  (2) &    &\leq& c+b & \rA \\
            2     &  (3) &    &\leq& c+b & \rCommutativeMonoid{} \\
            2     &  (4) &   \multicolumn{3}{l}{a\leq b} & \awbwcInNaturalLpaLeqbEqvaPluscLeqbPluscRp{3} \\
        \end{array}
	\]
\end{proof}

\label{awbwcwdInNaturalwaLeqbwcLeqdImpaPluscLeqbPlusd}
\begin{theorem}[\(a,b,c,d\in\mathbb{N},a\leq b, c\leq d\vdash a+c\leq b+d\)]
\end{theorem}
\begin{proof}
Seien \(a, b, c, d \in \mathbb{N}\). Im Beweis nutzen wir die Eigenschaft \(\ImpLpNaturalwPluswZeroRpInMonoid{}\).
        \[
	\begin{array}{llclll}
            1     &  (1) & \multicolumn{3}{l}{a\leq b} & \rA \\
            2     &  (2) & \multicolumn{3}{l}{c\leq d} & \rA \\
            1     &  (3) & a+c&\leq &b+c & \awbwcInNaturalLpaLeqbEqvaPluscLeqbPluscRp{1} \\
            2     &  (4) &    &\leq &b+d & \awbwcInNaturalLpaLeqbEqvcPlusaLeqcPlusbRp{2} \\
            1,2   &  (5) & a+c&\leq &b+d & \rTransitivityOrdRI{3,4} \\
    \end{array}
	\]
\end{proof}



\label{aInNaturalwbInNaturalwcInNaturalwaLeqbImpaLeqbPlusc}
\begin{theorem}[\(a\in\mathbb{N},b\in\mathbb{N},c\in\mathbb{N}, a\leq b\vdash a\leq b+c\)]
Seien \(a,b,c\in\mathbb{N}\), dann gilt:
\[a\leq b\vdash a\leq b+c\]
\end{theorem}
\begin{proof}
Im Beweis nehmen wir an, dass \(a,b,c\in\mathbb{N}\). Der Beweis besteht aus zwei Teilen:
\begin{enumerate}
    \item Wir beweisen ein Lemma, das im weiteren Verlauf verwendet wird.
    \item Wir zeigen, dass aus \(a \leq b\) folgt, dass \(a \leq b+c\).
\end{enumerate}

\paragraph{Lemma: \((a+c)+((b+c)-(a+c))=b+c \vdash a+(c+((b+c)-(a+c)))=b+c\)}

Im Beweis verwenden wir indirekt die Regel \(\aIdbImpbIda{}\).  \(\ImpLpNaturalwPluswZeroRpInMonoid{}\) und daher gilt:\\
        \[
	\begin{array}{llclll}
            1     &  (1) & b+c&=&(a+c)+((b+c)-(a+c)) & \rA \\
            1     &  (2) & &=&a+(c+((b+c)-(a+c)))=b+c & \rAssociativityMonoid{} \\
    \end{array}
	\]
    \\
\paragraph{Beweis des Theorems: \(a \leq b \vdash a \leq b+c\)}
        \[
	\begin{array}{llll}
            1     &  (1) & a\leq b & \rA \\
            1     &  (2) & a+c\leq b+c & \aInNaturalwbInNaturalwaNotEqualsZerowaEqualsbImpbMinusOneLeqa{1} \\
            1     &  (3) & (a+c)+((b+c)-(a+c))=b+c & \minusI{2} \\
            1     &  (4) & a+(c+((b+c)-(a+c)))=b+c & \text{Lemma(3)} \\
            1     &  (5) & a\leq b+c & \rLeqNI{4} \\
    \end{array}
	\]
\end{proof}

\label{aInNaturalwbInNaturalwcInNaturalwaLeqbImpaLeqcPlusb}
\begin{theorem}[\(a\in\mathbb{N},b\in\mathbb{N},c\in\mathbb{N}, a\leq b\vdash a\leq c+b\)]
\end{theorem}
\begin{proof}
Seien \(a, b, c \in \mathbb{N}\). Im Beweis nutzen wir die Eigenschaft \(\ImpLpNaturalwPluswZeroRpInMonoid{}\).
        \[
	\begin{array}{llclll}
            1     &  (1) & \multicolumn{3}{l}{a\leq b} & \rA \\
            1     &  (2) & a&\leq& b+c & \aInNaturalwbInNaturalwcInNaturalwaLeqbImpaLeqbPlusc{1} \\
            1     &  (3) &  &\leq& c+b & \rCommutativeMonoid{} \\
        \end{array}
	\]
\end{proof}


\label{awbInNaturalImpaLeqaPlusb}
\begin{theorem}[\(a,b\in\mathbb{N}\vdash a\leq a+b\)]
\end{theorem}
\begin{proof}
Seien \(a, b \in \mathbb{N}\). Im Beweis nutzen wir die Eigenschaft \(\LeqIsHalfOrderOnNaturalNumbers{}\).
        \[
	\begin{array}{llll}
                  &  (1) & a\leq a & \rReflexivityOrdRI{} \\
                  &  (2) & a\leq a+b & \aInNaturalwbInNaturalwcInNaturalwaLeqbImpaLeqbPlusc{1} \\
        \end{array}
	\]
\end{proof}

\label{awbInNaturalImpaLeqbPlusa}
\begin{theorem}[\(a,b\in\mathbb{N}\vdash a\leq b+a\)]
\end{theorem}
\begin{proof}
Seien \(a, b \in \mathbb{N}\). Im Beweis nutzen wir die Eigenschaft \(\LeqIsHalfOrderOnNaturalNumbers{}\).
        \[
	\begin{array}{llll}
                  &  (1) & a\leq a & \rReflexivityOrdRI{} \\
                  &  (2) & a\leq b+a & \aInNaturalwbInNaturalwcInNaturalwaLeqbImpaLeqcPlusb{1} \\
        \end{array}
	\]
\end{proof}

\label{awbwcInNaturalLpaPlusbLeqcImpaLeqcRp}
\begin{theorem}[\(a,b,c\in\mathbb{N}(a+b\leq c\vdash a\leq c)\)]
\end{theorem}
\begin{proof}
Seien \(a, b, c \in \mathbb{N}\). Im Beweis nutzen wir die Eigenschaft \(\LeqIsHalfOrderOnNaturalNumbers{}\).
        \[
	\begin{array}{llclll}
                  &  (1) & \multicolumn{3}{l}{a\leq a+b} & \awbInNaturalImpaLeqaPlusb{} \\
            2     &  (2) &  &\leq& c & \rA \\
        \end{array}
	\]
\end{proof}



\label{awbwcInNaturalLpaPlusbLeqcImpbLeqcRp}
\begin{theorem}[\(a,b,c\in\mathbb{N}(a+b\leq c\vdash b\leq c)\)]
\end{theorem}
\begin{proof}
Seien \(a, b, c \in \mathbb{N}\). Im Beweis nutzen wir die Eigenschaft \(\LeqIsHalfOrderOnNaturalNumbers{}\).
        \[
	\begin{array}{llclll}
                  &  (1) & \multicolumn{3}{l}{b\leq a+b} & \awbInNaturalImpaLeqbPlusa{} \\
            2     &  (2) &  &\leq& c & \rA \\
        \end{array}
	\]
\end{proof}

\label{bLeqcwaLeqbImpbMinuscLeqc}
\begin{theorem}[\(b\leq c, a\leq b\vdash b-a\leq c\)]
\end{theorem}
\begin{proof}
Seien \(a, b, c \in \mathbb{N}\). 
\[
\begin{array}{llclll}
          1  & (1) & \multicolumn{3}{l}{a\leq b}  & \rA \\
          1  & (2) & \multicolumn{3}{l}{b-a\in\mathbb{N}}  & \minusI{1} \\
          1  & (3) & b-a&\leq& a+(b-a)  & \awbInNaturalImpaLeqaPlusb{2} \\
          1  & (4) & &=& b  & \minusI{1} \\
          5  & (5) & &\leq & c  & \rA \\
          1,5& (6) & b-a&\leq & c  & \rTransitivityOrdRI{3,5} \\
\end{array}
\]
\end{proof}

\subsection{Totalität}

\label{ImpaLeqbOrbLeqa}
\begin{theorem}[Totalität der Ordnungsrelation auf \(\mathbb{N}\)]
Seien \(a,b\in\mathbb{N}\), dann gilt:
\[\vdash a\leq b\lor b\leq a\]
\end{theorem}
\begin{proof}
    Seien \(a,b\in\mathbb{N}\), dann gilt:
        \[
	\begin{array}{llll}
                    &  (1) & 0\leq b & \ImpZeroLeqa{} \\
                    &  (2) & 0\leq b\lor b\leq 0 & \rOIa{1} \\
        3           &  (3) & n\in\mathbb{N} & \rA \\           
        4           &  (4) & n\leq b\lor b\leq n & \rA \\    
        5           &  (5) & n\leq b & \rA \\  
                    &  (6) & n=b\lor n\neq b & \ImpPOrnP{} \\  
        7           &  (7) & n=b & \rA \\  
        7           &  (8) & b\leq n+1 & \aInNaturalwbInNaturalwaEqualsbImpbLeqaPlusOne{7} \\  
        7           &  (9) & n+1\leq b\lor b\leq n+1 & \rOIb{8} \\ 
        10          &  (10) & n\neq b & \rA \\ 
        5,10        &  (11) & n+1\leq b & \aInNaturalwbInNaturalwaLeqbwaNotEqualsbImpaPlusOneLeqb{5,10} \\ 
        5,10        &  (12) & n+1\leq b\lor b\leq n+1 & \rOIa{11} \\ 
        5           &  (13) & n+1\leq b\lor b\leq n+1 & \rOE{6,7,9,10,12} \\ 
        14          &  (14) & b\leq n & \rA \\  
        14          &  (15) & b\leq n+1 & \aInNaturalwbInNaturalwcInNaturalwaLeqbImpaLeqbPlusc{14} \\  
        14          &  (16) & n+1\leq b\lor b\leq n+1 & \rOIb{15} \\ 
         4          &  (17) & n+1\leq b\lor b\leq n+1 & \rOE{4,5,13,14,16} \\ 
                    &  (18) & \forall n\in\mathbb{N}(n\leq b\lor b\leq n) & \rInductionN{2,3,4,17} \\ 
                    &  (19) & a\leq b\lor b\leq a & \rSetUEa{32} \\ 
    \end{array}
	\]
\end{proof}




\label{LeqIsTotalOrderOnNaturalNumbers}
\begin{theorem}[\(\leq\) ist eine totale Ordnung auf \(\mathbb{N}\) ]
\end{theorem}
\begin{proof}
\[
\begin{array}{llll}
            & (1) & \leq \text{ ist eine Halbordnung auf } \mathbb{N}  & \LeqIsHalfOrderOnNaturalNumbers{} \\
            & (2) & \forall a, b \in \mathbb{N}  (a \leq b \lor b \leq a) & \ImpaLeqbOrbLeqa{} \\
          & (3) & \leq \text{ ist eine totale Ordnung auf } \mathbb{N} & \rTotalOrderI{1,2}
\end{array}
\]
\end{proof}

\subsection{Zusammenhänge zwischen Differenzen und Summen}

\label{awbwcInNaturalwcLeqaImpLpaPlusbRpMinuscEqualsbPlusLpaMinuscRp}
\begin{theorem}[\(a,b,c\in\mathbb{N},c\leq a\vdash (a+b)-c=b+(a-c)\)]
\end{theorem}
\begin{proof}
Seien \(a, b, c \in \mathbb{N}\). Im Beweis nutzen wir die Eigenschaft \(\ImpLpNaturalwPluswZeroRpInMonoid{}\).
\[
\begin{array}{llclll}
          1  & (1) & \multicolumn{3}{l}{c\leq a}  & \rA \\
          1  & (2) & \multicolumn{3}{l}{c\leq a+b}  & \aInNaturalwbInNaturalwcInNaturalwaLeqbImpaLeqbPlusc{1} \\
          1  & (3) & \multicolumn{3}{l}{c+((a+b)-c)=a+b} & \minusI{2} \\
          1  & (4) & \multicolumn{3}{l}{c+(a-c)=a} & \minusI{1} \\
          1  & (5) & \multicolumn{3}{l}{c+((a+b)-c)=c+(a-c)+b} & \rIE{4,3} \\
          1  & (6) & (a+b)-c&=&(a-c)+b & \aEqualsbEqvcPlusaEqualscPlusb{5} \\
          1  & (7) & &=&b+(a-c) & \rCommutativeMonoid{6} \\
\end{array}
\]
\end{proof}

\label{awbwcInNaturalwcLeqaImpLpaPlusbRpMinuscEqualsLpaMinuscRpPlusb}
\begin{theorem}[\(a,b,c\in\mathbb{N},c\leq a\vdash (a+b)-c=(a-c)+b\)]
\end{theorem}
\begin{proof}
Seien \(a, b, c \in \mathbb{N}\). Im Beweis nutzen wir die Eigenschaft \(\ImpLpNaturalwPluswZeroRpInAbelMonoid{}\).
\[
\begin{array}{llclll}
          1  & (1) & \multicolumn{3}{l}{c\leq a}  & \rA \\
          1  & (2) & (a+b)-c&=&b+(a-c) & \awbwcInNaturalwcLeqaImpLpaPlusbRpMinuscEqualsbPlusLpaMinuscRp{1} \\
          1  & (3) & &=&(a-c)+b & \rCommutativeMonoid{} \\
\end{array}
\]
\end{proof}

\label{awbwcInNaturalwcLeqbImpLpaPlusbRpMinuscEqualsaPlusLpbMinuscRp}
\begin{theorem}[\(a,b,c\in\mathbb{N}, c\leq b\vdash (a+b)-c=a+(b-c)\)]
\end{theorem}
\begin{proof}
Seien \(a, b, c \in \mathbb{N}\). Im Beweis nutzen wir die Eigenschaft \(\ImpLpNaturalwPluswZeroRpInMonoid{}\).
\[
\begin{array}{llcll p{5cm}}
          1  & (1) & \multicolumn{3}{l}{c\leq b}  & \rA \\
             & (2) & (a+b)-c&=&(b+a)-c & \rCommutativeMonoid{} \\
          1  & (3) & &=&a+(b-c) & \awbwcInNaturalwcLeqaImpLpaPlusbRpMinuscEqualsLpaMinuscRpPlusb{1} \\
\end{array}
\]
\end{proof}

\label{awbwcInNaturalwcLeqbImpLpaPlusbRpMinuscEqualsLpbMinuscRpPlusa}
\begin{theorem}[\(a,b,c\in\mathbb{N}, c\leq b\vdash (a+b)-c=(b-c)+a\)]
\end{theorem}
\begin{proof}
Seien \(a, b, c \in \mathbb{N}\). Im Beweis nutzen wir die Eigenschaft \(\ImpLpNaturalwPluswZeroRpInAbelMonoid{}\).
\[
\begin{array}{llclll}
          1  & (1) & \multicolumn{3}{l}{c\leq b}  & \rA \\
          1  & (2) & (a+b)-c&=&a+(b-c) & \awbwcInNaturalwcLeqbImpLpaPlusbRpMinuscEqualsaPlusLpbMinuscRp{5} \\
          1  & (3) & &=&(a-c)+a & \rCommutativeMonoid{} \\
\end{array}
\]
\end{proof}

\label{awbInNaturalImpLpaPlusbRpMinusbEqualsaPlusLpbMinusbRpEqualsa}
\begin{theorem}[\(a,b\in\mathbb{N}\vdash (a+b)-b=a+(b-b)=a\)]
\end{theorem}
\begin{proof}
Seien \(a, b \in \mathbb{N}\). Im Beweis nutzen wir die Eigenschaft \(\ImpLpNaturalwPluswZeroRpInMonoid{}\).
\[
\begin{array}{llclll}
             & (1) & \multicolumn{3}{l}{b\leq b}  & \rReflexivityOrdRI{} \\
             & (2) & \multicolumn{3}{l}{b\leq a+b}  & \aInNaturalwbInNaturalwcInNaturalwaLeqbImpaLeqcPlusb{1} \\
             & (3) & (a+b)-b&=&a+(b-b)  & \aInNaturalwbInNaturalwcInNaturalwaLeqbImpaLeqcPlusb{1,2} \\
             & (4) &  &=&a+0  & \aInNaturalImpaMinusaEqualsZero{3} \\
             & (5) &  &=&a  & \rNeutralElementMonoid{4} \\
\end{array}
\]
\end{proof}

\label{awbInNaturalImpLpaPlusbRpMinusaEqualsLpaMinusaRpPlusbEqualsb}
\begin{theorem}[\(a,b\in\mathbb{N}\vdash (a+b)-a=(a-a)+b=b\)]
\end{theorem}
\begin{proof}
Seien \(a, b \in \mathbb{N}\). Im Beweis nutzen wir die Eigenschaft \(\ImpLpNaturalwPluswZeroRpInMonoid{}\).
\[
\begin{array}{llclll}
             & (1) & (a+b)-a&=&(b+a)-a  & \rCommutativeMonoid{} \\
             & (2) & &=& b+(a-a)  & \awbInNaturalImpLpaPlusbRpMinusbEqualsaPlusLpbMinusbRpEqualsa{} \\
             & (3) & &=& (a-a)+b  & \rCommutativeMonoid{} \\
             & (4) & &=& b+(a-a)  & \rCommutativeMonoid{} \\
             & (5) & &=& b  & \awbInNaturalImpLpaPlusbRpMinusbEqualsaPlusLpbMinusbRpEqualsa{} \\
             & (6) & \multicolumn{3}{l}{(a+b)-a=(a-a)+b=b}  & \rTransitivityEqRI{1,3,5} \\
\end{array}
\]
\end{proof}

\label{awbInNaturalwbLeqaImpLpaMinusbRpPlusbEqualsa}
\begin{theorem}[\(a,b\in\mathbb{N},b\leq a \vdash (a-b)+b=a\)]
\end{theorem}
\begin{proof}
    Seien \(a,b\in\mathbb{N}\). Daraus folgt:
        \[
	\begin{array}{llclll}
             1 &  (1) & \multicolumn{3}{l}{b\leq a} & \rA \\
             1 &  (2) & (a-b)+b&=&(a+b)-b & \awbwcInNaturalwcLeqaImpLpaPlusbRpMinuscEqualsLpaMinuscRpPlusb{1} \\
             1 &  (3) & &=&a & \awbInNaturalImpLpaPlusbRpMinusbEqualsaPlusLpbMinusbRpEqualsa{2} \\
             1 &  (4) & (a-b)+b&=&a & \rTransitivityEqRI{2,3} \\
    \end{array}
	\]
\end{proof}

\label{awbInNaturalwaMinusbEqualsaImpbEqualsZero}
\begin{theorem}[\(a,b\in\mathbb{N}, a-b=a\vdash b=0\)]
\end{theorem}
\begin{proof}
    Seien \(a,b\in\mathbb{N}\). \(\ImpLpNaturalwPluswZeroRpInAbelMonoid{}\) und daraus folgt:
        \[
	\begin{array}{llclll}
             1 &  (1) & \multicolumn{3}{l}{a-b=a} & \rA \\
             1 &  (2) & \multicolumn{3}{l}{(a-b)+b=a+b} & \aEqualsbEqvaPluscEqualsbPlusc{1} \\
             1 &  (3) & \multicolumn{3}{l}{(a-b)+b=a} & \awbInNaturalwbLeqaImpLpaMinusbRpPlusbEqualsa{1} \\
             1 &  (4) & \multicolumn{3}{l}{a=a+b} & \rIE{3,2} \\
             1 &  (5) & \multicolumn{3}{l}{b=0} & \aInNaturalwbInNaturalwaPlusbEqualsaImpbEqualsZero{4} \\
    \end{array}
	\]
\end{proof}

\label{awbwcInNaturalwcLeqaLpaEqualsbEqvaMinuscEqualsbMinuscRp}
\begin{theorem}[\(a=b\dashv\vdash a-c=b-c\)]
Seien \(a,b,c\in\mathbb{N}\) und \(c\leq a\), dann gilt:
\[a=b\dashv\vdash a-c=b-c\]
\end{theorem}
\begin{proof}
Seien \(a,b,c\in\mathbb{N}\).

	\begin{longtable}{llclll}
               \multicolumn{6}{l}{\(\vdash\):}\\
               1 &  (1) & \multicolumn{3}{l}{\(a=b\)} & \rA \\
               2 &  (2) & \multicolumn{3}{l}{\(c\leq a\)} & \rA \\
               2 &  (3) & \multicolumn{3}{l}{\(a-c=a-c\)} & \minusI{2} \\
               1,2 &  (4) & \multicolumn{3}{l}{\(a-c=b-c\)} & \rIE{1,3} \\
               \multicolumn{6}{l}{\(\dashv\):}\\
               1 &  (1) & \multicolumn{3}{l}{\(c\leq a\)} & \rA \\
               2 &  (2) & \multicolumn{3}{l}{\(a-c=b-c\)} & \rA \\
               1 &  (3) & \(a\)&=&\((a-c)+c\) & \awbInNaturalwbLeqaImpLpaMinusbRpPlusbEqualsa{1} \\
             1,2 &  (4) & &=&\((b-c)+c\) & \rIE{2,3} \\
             1,2 &  (5) & &=&\(b\) & \rIE{1,3} \\
             1,2 &  (6) & \multicolumn{3}{l}{\(a=b\)} & \rTransitivityEqRI{3,5} \\
        \end{longtable}
\end{proof}

\label{awbwcInNaturalwcLeqaLpaNotEqualsbEqvaMinuscNotEqualsbMinuscRp}
\begin{theorem}[\(a\neq b\dashv\vdash a-c\neq b-c\)]
Seien \(a,b,c\in\mathbb{N}\) und \(c\leq a\), dann gilt:
\[a\neq b\dashv\vdash a-c\neq b-c\]
\end{theorem}
\begin{proof}
        Seien \(a,b,c\in\mathbb{N}\) und \(c\leq a\), dann gilt:
        \[
	\begin{array}{llll}
		   & (1) & a=b\leftrightarrow a-c=b-c & \awbInNaturalLpaEqualsbEqvaMinusbEqualsZeroRp{} \\
		   & (2) & a\neq b\leftrightarrow a-c\neq b-c & \PLrQEqvnPLrnQ{1} \\
	\end{array}
	\]
\end{proof}

\label{awbwcInNaturalwaLeqbwbLeqcImpLpbMinusaRpPlusLpcMinusbRpEqualsLpcMinusaRp}
\begin{theorem}[\(a,b,c \in \mathbb{N},a\leq b, b\leq c\vdash (b-a)+(c-b)=(c-a)\)]
\end{theorem}
\begin{proof}
Seien \(a, b,c \in \mathbb{N}\). Im Beweis nutzen wir die Eigenschaften:
\begin{enumerate}
\item \(\ImpLpNaturalwPluswZeroRpInMonoid{}\).
\item \(\LeqIsHalfOrderOnNaturalNumbers{}\).
\item \(\FaSLpEqualsInEquivalencerelationSRp{}\).
\end{enumerate}

\begin{longtable}{llcll p{5cm}}
1 & (1) & \multicolumn{3}{l}{\(a\leq b\)}  & \rA \\
2 & (2) & \multicolumn{3}{l}{\(b\leq c\)}  & \rA \\
1,2 & (3) & \multicolumn{3}{l}{\(a\leq c\)}  & \rTransitivityOrdRI{1,2} \\
1,2 & (4) & \multicolumn{3}{l}{\(a\leq b+c\)}  & \aInNaturalwbInNaturalwcInNaturalwaLeqbImpaLeqbPlusc{1} \\
1,2 & (5) & \((b-a)+(c-b)\)&=&\(((b-a)+c)-b\)  & \awbwcInNaturalwcLeqbImpLpaPlusbRpMinuscEqualsaPlusLpbMinuscRp{2} \\
    & (6) & &=&\((c+(b-a))-b\)  & \rCommutativeMonoid{} \\
1,2 & (7) &  &=&\(((c+b)-a)-b\)  & \awbwcInNaturalwcLeqbImpLpaPlusbRpMinuscEqualsaPlusLpbMinuscRp{1} \\
1,2 & (8) &  &=&\(((b+c)-a)-b\)  & \rCommutativeMonoid{} \\
1,2 & (9) &  &=&\(((b+(c-a))-b\))  & \awbwcInNaturalwcLeqbImpLpaPlusbRpMinuscEqualsaPlusLpbMinuscRp{3} \\
1,2 & (10) &  &=&\(((c-a)+b)-b\)  & \rCommutativeMonoid{} \\
1,2 & (11) &  &=&\(c-a\)  & \awbInNaturalImpLpaPlusbRpMinusbEqualsaPlusLpbMinusbRpEqualsa{} \\
1,2 & (12) &  \((b-a)+(c-b)\)&=&\(c-a\)  & \rTransitivityEqRI{5,11} \\
\end{longtable}

\end{proof}


\label{awbwcInNaturalwaLeqbLpbLeqcEqvbMinusaLeqcMinusaRp}
\begin{theorem}[\(b\leq c \dashv\vdash b-a\leq c-a\)]
Seien \(a,b,c\in\mathbb{N}\) und \(a\leq b\), dann gilt:
\[b\leq c \dashv\vdash b-a\leq c-a\]
\end{theorem}
\begin{proof}
Seien \(a, b,c \in \mathbb{N}\). Im Beweis nutzen wir die Eigenschaften:
\begin{enumerate}
\item \(\ImpLpNaturalwPluswZeroRpInMonoid{}\).
\item \(\LeqIsHalfOrderOnNaturalNumbers{}\).
\item \(\FaSLpEqualsInEquivalencerelationSRp{}\).
\end{enumerate}
\(\vdash:\)
\begin{longtable}{llcll p{5cm}}
1 & (1) & \multicolumn{3}{l}{\(a\leq b\)}  & \rA \\
2 & (2) & \multicolumn{3}{l}{\(b\leq c\)}  & \rA \\
1,2 & (3) & \multicolumn{3}{l}{\((b-a)+(c-b)=(c-a)\)}  & \awbwcInNaturalwaLeqbwbLeqcImpLpbMinusaRpPlusLpcMinusbRpEqualsLpcMinusaRp{1,2} \\
1,2 & (4) & \multicolumn{3}{l}{\((b-a)\leq (c-a)\)}  & \rLeqNI{3} \\
\end{longtable}
\(\dashv:\)
\begin{longtable}{llcll p{5cm}}
1 & (1) & \multicolumn{3}{l}{\(b-a\leq c-a\)}  & \rA \\
2 & (2) & \multicolumn{3}{l}{\(a\leq b\)}  & \rA \\
2 & (3) & \(b\)&=&\((b-a)+a\)  & \awbInNaturalwbLeqaImpLpaMinusbRpPlusbEqualsa{2} \\
1,2 & (4) &        &\(\leq\) &\((c-a)+a\)  & \awbwcInNaturalLpaLeqbEqvaPluscLeqbPluscRp{2} \\
1,2 & (5) &        &=&\(c\)  & \awbInNaturalwbLeqaImpLpaMinusbRpPlusbEqualsa{4} \\
1,2 & (6) & \multicolumn{3}{l}{\(b\leq c\)}  & \rTransitivityOrdRI{3,5} \\
\end{longtable}

\end{proof}

\label{awbwcInNaturalwaPlusbLeqcImpaLeqcMinusb}
\begin{theorem}[\(a,b,c\in\mathbb{N},a+b\leq c \vdash a\leq c-b\)]
\end{theorem}
\begin{proof}
Seien \(a, b, c \in \mathbb{N}\). Im Beweis nutzen wir die Eigenschaft \(\ImpLpNaturalwPluswZeroRpInMonoid{}\).
\[
\begin{array}{llcll p{5cm}}
             1 & (1) & \multicolumn{3}{l}{a+b\leq c}  & \rA \\
             1 & (2) & \multicolumn{3}{l}{b\leq c}  & \rA \\
               & (3) & a&=&(a+b)-b  & \awbInNaturalImpLpaPlusbRpMinusbEqualsaPlusLpbMinusbRpEqualsa{} \\
               & (4) & &\leq& c-b  & \awbwcInNaturalwaLeqbLpbLeqcEqvbMinusaLeqcMinusaRp{1,2} \\
\end{array}
\]
\end{proof}

\label{awbwcInNaturalwaPlusbLeqcImpbLeqcMinusa}
\begin{theorem}[\(a,b,c\in\mathbb{N},a+b\leq c \vdash b\leq c-a\)]
\end{theorem}
\begin{proof}
Seien \(a, b, c \in \mathbb{N}\). Im Beweis nutzen wir die Eigenschaft \(\ImpLpNaturalwPluswZeroRpInMonoid{}\).
\[
\begin{array}{llcll p{5cm}}
               & (1) & b+a&=&a+b  & \rCommutativeMonoid{} \\
             2 & (2) & &\leq& c  & \rA \\       
             2 & (3) & b+a&\leq& c  & \rTransitivityOrdRI{1,2} \\
             2 & (4) & \multicolumn{3}{l}{b\leq c-a}  & \awbwcInNaturalwaPlusbLeqcImpaLeqcMinusb{2} \\
\end{array}
\]
\end{proof}

\label{awbwcInNaturalwbPluscLeqaImpaMinusLpbPluscRpEqualsLpaMinusbRpMinusc}
\begin{theorem}[\(a,b,c\in\mathbb{N},b+c\leq a \vdash a-(b+c)=(a-b)-c\)]
\end{theorem}
\begin{proof}
Seien \(a, b, c \in \mathbb{N}\). Im Beweis nutzen wir die Eigenschaft \(\ImpLpNaturalwPluswZeroRpInMonoid{}\).
\[
\begin{array}{llcll p{5cm}}
          \multicolumn{6}{l}{\textbf{Lemma 1:} \quad b+c\leq a, c\leq a-b \vdash (b+c)+((a-b)-c) = (b+c)+(a-(b+c))} \\
          1  & (1) & \multicolumn{3}{l}{b+c\leq a}  & \rA \\
          1  & (2) & \multicolumn{3}{l}{b\leq a}  & \awbwcInNaturalLpaPlusbLeqcImpaLeqcRp{1} \\
          1  & (3) & \multicolumn{3}{l}{(b+c)+(a-(b+c))=a}  & \minusI{1} \\
          4  & (4) & \multicolumn{3}{l}{c\leq a-b}  & \rA \\
          4  & (5) & \multicolumn{3}{l}{c+((a-b)-c)=a-b}  & \minusI{4} \\
             & (6) & (b+c)+((a-b)-c)&=&b+(c+((a-b)-c))  & \rAssociativityMonoid{} \\
          5  & (7) &                &=&b+(a-b)  & \awbInNaturalwbLeqaImpLpaMinusbRpPlusbEqualsa{4,6} \\
          5  & (8) &                &=&a  & \awbInNaturalwbLeqaImpLpaMinusbRpPlusbEqualsa{2,7} \\
          5  & (9) &                &=&(b+c)+(a-(b+c)) & \rIE{3,9} \\
          \multicolumn{6}{l}{\textbf{Beweis des Theorems:} \quad b+c\leq a \vdash a-(b+c)=(a-b)-c} \\
          
          1  & (1) & \multicolumn{3}{l}{b+c\leq a}  & \rA \\
          1  & (2) & \multicolumn{3}{l}{c\leq a-b}  & \awbwcInNaturalwaPlusbLeqcImpbLeqcMinusa{1} \\
          1  & (3) & \multicolumn{3}{l}{(b+c)+(a-(b+c))=(b+c)+((a-b)-c)}  & \text{Lemma 1(1,2)} \\
          1  & (4) & \multicolumn{3}{l}{a-(b+c)=(a-b)-c}  & \aEqualsbEqvcPlusaEqualscPlusb{3} \\
\end{array}
\]
\end{proof}

\label{awbwcInNaturalwbPluscLeqaImpaMinusLpbPluscRpEqualsLpaMinuscRpMinusb}
\begin{theorem}[\(a,b,c\in\mathbb{N},b+c\leq a \vdash a-(b+c)=(a-c)-b\)]
\end{theorem}
\begin{proof}
Seien \(a, b, c \in \mathbb{N}\). Im Beweis nutzen wir die Eigenschaft \(\ImpLpNaturalwPluswZeroRpInMonoid{}\).
\[
\begin{array}{llcll p{5cm}}
          1  & (1) & \multicolumn{3}{l}{b+c\leq a}  & \rA \\
             & (2) & a-(b+c)&=&a-(c+b)  & \rCommutativeMonoid{} \\
          1  & (3) & &=&(a-c)-b  & \awbwcInNaturalwbPluscLeqaImpaMinusLpbPluscRpEqualsLpaMinusbRpMinusc{1} \\
          1  & (4) & \multicolumn{3}{l}{a-(b+c)=(a-c)-b}  & \rTransitivityEqRI{2,3} \\
\end{array}
\]
\end{proof}

\label{awbwcInNaturalwbPluscLeqaImpLpaMinusbRpMinuscEqualsLpaMinuscRpMinusb}
\begin{theorem}[\(a,b,c\in\mathbb{N},b+c\leq a \vdash (a-b)-c=(a-c)-b\)]
\end{theorem}
\begin{proof}
Seien \(a, b, c \in \mathbb{N}\). Im Beweis nutzen wir die Eigenschaft \(\ImpLpNaturalwPluswZeroRpInMonoid{}\).
\[
\begin{array}{llcll p{5cm}}
          1  & (1) & \multicolumn{3}{l}{b+c\leq a}  & \rA \\
             & (2) & (a-b)-c&=&a-(b+c)  & \awbwcInNaturalwbPluscLeqaImpaMinusLpbPluscRpEqualsLpaMinusbRpMinusc{1} \\
          1  & (3) & &=&(a-c)-b  & \awbwcInNaturalwbPluscLeqaImpaMinusLpbPluscRpEqualsLpaMinuscRpMinusb{1} \\
          1  & (4) & \multicolumn{3}{l}{(a-b)-c=(a-c)-b}  & \rTransitivityEqRI{2,3} \\
\end{array}
\]
\end{proof}



\label{awbwcwdInNaturalwaLeqcwbLeqdImpLpcMinusaRpPlusLpdMinusbRpEqualsLpcPlusdRpMinusLpaPlusbRp}
\begin{theorem}[\(a,b,c,d\in\mathbb{N},a\leq c, b\leq d \vdash (c-a)+(d-b)=(c+d)-(a+b)\)]
\end{theorem}
\begin{proof}
Seien \(a, b, c, d \in \mathbb{N}\). Im Beweis nutzen wir die Eigenschaft \(\ImpLpNaturalwPluswZeroRpInMonoid{}\).
\[
\begin{array}{llcll p{4.3cm}}
          1   & (1) & \multicolumn{3}{l}{a\leq c}  & \rA \\
          2   & (2) & \multicolumn{3}{l}{b\leq d}  & \rA \\
          1,2 & (3) & \multicolumn{3}{l}{a+b\leq c+d}  & \awbwcwdInNaturalwaLeqbwcLeqdImpaPluscLeqbPlusd{1,2} \\
          2 & (3) &  (c-a)+(d-b)&=&((c-a)+d)-b  & \awbwcInNaturalwcLeqbImpLpaPlusbRpMinuscEqualsaPlusLpbMinuscRp{2} \\
          1   & (4) &             &=&((c+d)-a)-b  & \awbwcInNaturalwcLeqaImpLpaPlusbRpMinuscEqualsLpaMinuscRpPlusb{1}  \\
          1,2 & (5) &             &=&(c+d)-(a+b)  & \awbwcInNaturalwbPluscLeqaImpaMinusLpbPluscRpEqualsLpaMinusbRpMinusc{3} \\
          1,2 & (6) &  (c-a)+(d-b)&=&(c+d)-(a+b)  & \rTransitivityOrdRI{3,5} \\
\end{array}
\]
\end{proof}


\label{awbwcwdInNaturalwcLeqawbLeqdLpaPlusbEqualscPlusdEqvaMinuscEqualsdMinusbRp}
\begin{theorem}[\(a+b=c+d \dashv\vdash a-c=d-b\)]
Seien \(a,b,c,d\in\mathbb{N}\), \(c\leq a\) und \(b\leq d\). Dann gilt:
\[a+b=c+d \dashv\vdash a-c=d-b.\]
\end{theorem}
\begin{proof}
Seien \(a,b,c,d\in\mathbb{N}\). Im Beweis nutzen wir die Eigenschaft \(\ImpLpNaturalwPluswZeroRpInMonoid{}\).

\(\vdash\):
    \[
	\begin{array}{llcll p{5cm}}
        1       &  (1)  & \multicolumn{3}{l}{c\leq a} & \rA \\
        2       &  (2)  & \multicolumn{3}{l}{b\leq d} & \rA \\
        3       &  (3)  & \multicolumn{3}{l}{a+b=c+d} & \rA \\
                &  (4)  & \multicolumn{3}{l}{b\leq a+b} & \awbInNaturalImpaLeqbPlusa{} \\
        2       &  (5)  & \multicolumn{3}{l}{c+b\leq c+d} & \awbwcInNaturalLpaLeqbEqvcPlusaLeqcPlusbRp{2} \\                
        3       &  (6)  & \multicolumn{3}{l}{(a+b)-b=(c+d)-b} & \awbwcInNaturalwcLeqaLpaEqualsbEqvaMinuscEqualsbMinuscRp{4,3} \\
                &  (7)  & \multicolumn{3}{l}{(a+b)-b=a} & \awbInNaturalImpLpaPlusbRpMinusbEqualsaPlusLpbMinusbRpEqualsa{} \\
        3       &  (8)  & \multicolumn{3}{l}{a=(c+d)-b} & \rIE{7,6} \\
        1,3     &  (9)  & \multicolumn{3}{l}{a-c=((c+d)-b)-c} & \awbwcInNaturalwcLeqaLpaEqualsbEqvaMinuscEqualsbMinuscRp{1,8} \\
        1,2,3   &  (10)  & ((c+d)-b)-c&=&((c+d)-c)-b & \awbwcInNaturalwbPluscLeqaImpLpaMinusbRpMinuscEqualsLpaMinuscRpMinusb{5} \\
                &  (11)  &     &=&d-b & \awbInNaturalImpLpaPlusbRpMinusbEqualsaPlusLpbMinusbRpEqualsa{} \\       
        1,2,3   &  (12)  &  (c+d)-b)-c &=&d-b & \rTransitivityEqRI{10,11} \\      
        1,2,3   &  (13)  &  \multicolumn{3}{l}{a-c=d-b} & \rIE{12,9} \\  
        \end{array}
    \]
\(\dashv\):
    \[
	\begin{array}{llcll p{5cm}}
        1       &  (1)  & \multicolumn{3}{l}{c\leq a} & \rA \\
        2       &  (2)  & \multicolumn{3}{l}{b\leq d} & \rA \\
        3       &  (3)  & \multicolumn{3}{l}{a-c=d-b} & \rA \\
        3       &  (4)  & \multicolumn{3}{l}{(a-c)+c=(d-b)+c} & \aEqualsbEqvaPluscEqualsbPlusc{3} \\
        1,3     &  (5)  & \multicolumn{3}{l}{(a-c)+c=a} & \awbInNaturalwbLeqaImpLpaMinusbRpPlusbEqualsa{1,4} \\   
        1,3     &  (6)  & \multicolumn{3}{l}{a=(d-b)+c} & \rIE{5,3} \\     
        1,3     &  (7)  & \multicolumn{3}{l}{a+b=((d-b)+c)+b} & \aEqualsbEqvaPluscEqualsbPlusc{6} \\ 
                &  (8)  & ((d-b)+c)+b&=&((d-b)+b)+c & \aInMwbInMwcInMImpLpaPlusbRpPluscEqualsLpaPluscRpPlusb{7} \\ 
        2       &  (9)  & &=&d+c & \awbInNaturalwbLeqaImpLpaMinusbRpPlusbEqualsa{2} \\ 
        2       &  (10)  & ((d-b)+c)+b&=&d+c & \rTransitivityEqRI{8,9} \\ 
        1,2,3   &  (11)  & a+b&=&d+c & \rIE{10,7} \\ 
        \end{array}
    \]
\end{proof}

\section{Induzierte strikte Ordnung der natürlichen Zahlen}

Basierend auf der bereits definierten Halbordnung \(\leq\) auf \(\mathbb{N}\), können wir nun eine strikte Ordnung auf den natürlichen Zahlen einführen. Diese strikte Ordnung, die wir mit dem Symbol \(<\) bezeichnen, wird durch die Beziehung zwischen zwei Elementen \(a\) und \(b\) der Menge \(\mathbb{N}\) definiert, wobei \(a\) echt kleiner als \(b\) ist.

\begin{definition}[Strikte Ordnung \(<\) auf \(\mathbb{N}\)]
Für alle \(a, b \in \mathbb{N}\) definieren wir:
\[
a < b := a \leq b \land a \neq b.
\]
\end{definition}

Diese Definition entspricht der allgemeinen Konstruktion einer induzierten strikten Ordnung aus einer Halbordnung, bei der die Gleichheit ausgeschlossen wird, um eine strikte Vergleichsrelation zu erhalten.

\label{aLneqbEqvExnInNaturalLpnNotEqualsZeroAndaPlusnEqualsbRp}
\begin{theorem}[\(a<b\dashv\vdash a-b\neq 0\land a+(b-a)=b \) ] Seien \(a,b\in\mathbb{N}\), dann gilt:
\[a<b\dashv\vdash a-b\neq 0\land a+(b-a)=b\]
\end{theorem}
\begin{proof}
Seien \(a,b\in\mathbb{N}\). \(\ImpLpNaturalwPluswZeroRpInAbelMonoid{}\), woraus folgt:

\(\vdash:\)
	\[
	\begin{array}{llll}
		1 & (1) & a<b & \rA \\
		1 & (2) & a\leq b & \InducedStrictOrderE{1} \\
		1 & (3) & a\neq b & \InducedStrictOrderE{1} \\
            1 & (4) & a+(b-a)=b & \minusI{2} \\
            5 & (5) & b-a = 0 & \rA \\
              & (6) & a + 0 = a & \rNeutralElementMonoid{} \\
          1,5 & (7) & a + 0 = b& \rIE{5,4} \\
          1,5 & (8) & a = b& \rIE{6,7} \\
        1,3,5 & (9) & \bot & \rBI{3,8} \\
        1,3 & (12) & a-b\neq 0 & \rCI{5,9} \\
        1,5 & (13) & a-b\neq 0\land a + (a-b) = b & \rCI{12,6} \\
	\end{array}
	\]
	\(\dashv:\)
	\[
	\begin{array}{llll}
		1 & (1) & a-b\neq 0\land a + (a-b) = b & \rA \\
            1 & (2) & a-b\neq 0 & \rAEa{1} \\
            1 & (3) & a + (a-b) = b & \rAEb{1} \\
            1 & (4) & a\leq b & \rLeqNI{3} \\
              & (5) & a-b=0\leftrightarrow a=b & \awbInNaturalLpaEqualsbEqvaMinusbEqualsZeroRp{} \\
            1 & (6) & a\neq b & \PLrQwnPImpnQ{5,2} \\
            1 & (7) & a<b & \InducedStrictOrderI{6,4} \\
	\end{array}
	\]
\end{proof}

\label{aInNaturalImpaEqualsZeroOrZeroLneqa}
\begin{theorem}[\(a\in\mathbb{N}\vdash a=0\lor 0<a\)]
\end{theorem}
\begin{proof}
\[
\begin{array}{llll}
            1 & (1) & a\in\mathbb{N}  & \rA{} \\
              & (2) & a=0\lor a\neq 0 & \ImpPOrnP{} \\
            3 & (3) & a=0 & \rA \\
            3 & (4) & a=0\lor 0<a & \rOIa{3} \\
            5 & (5) & a\neq 0 & \rA \\
            1 & (6) & 0\leq a & \ImpZeroLeqa{1} \\
            1,5 & (7) & 0<a & \InducedStrictOrderI{6,5} \\
            1,5 & (8) & a=0\lor 0<a & \rOIb{7} \\
            1  & (9) & a=0\lor 0<a & \rOE{2,3,4,5,8} \\
\end{array}
\]
\end{proof}

\label{aInNaturalwZeroLneqaImpExbInNaturalLpOnePlusbEqualsaRp}
\begin{theorem}[\(a\in\mathbb{N},0<a\vdash\exists b\in\mathbb{N}(1+b=a)\)]
\end{theorem}
\begin{proof}
        Im Beweis benutzen wir die Eigenschaft \(\ImpLpNaturalwPluswZeroRpInMonoid{}\). Hiermit gilt:
\[
\begin{array}{llll}
            1 & (1) & a\in\mathbb{N}  & \rA{} \\
              & (2) & 0\in\mathbb{N}  & \zeroIsNaturalNumber{} \\
            3 & (3) & 0<a & \rA \\
            3 & (4) & 0\leq a & \InducedStrictOrderE{3} \\
            3 & (5) & 0\neq a & \InducedStrictOrderE{3} \\
            3 & (6) & 0+1\leq a & \aInNaturalwbInNaturalwaLeqbwaNotEqualsbImpaPlusOneLeqb{1,2,4,5} \\
            3 & (7) & 0+1=1 & \rNeutralElementMonoid{} \\
            3 & (8) & 1\leq a & \rIE{7,6} \\
            3 & (9) & \exists b\in\mathbb{N}(1+b=a) & \rLeqNE{8} \\
\end{array}
\]
\end{proof}


\label{aLneqbEqvaLeqbMinusOne}
\begin{theorem}[\(a<b\dashv\vdash a\leq b-1\)]
Seien \(a,b\in\mathbb{N}\), dann gilt:
\[a<b\dashv\vdash a\leq b-1\]
\end{theorem}
\begin{proof}
        Seien \(a,b\in\mathbb{N}\). Im Beweis benutzen wir die Eigenschaft \(\ImpLpNaturalwPluswZeroRpInMonoid{}\). Hiermit gilt:        
\(\vdash:\)
\[
\begin{array}{llll}
            1 & (1) & a<b  & \rA{} \\
            1 & (2) & b-a\neq 0\land a+(b-a)=b  & \aLneqbEqvExnInNaturalLpnNotEqualsZeroAndaPlusnEqualsbRp{1} \\
            1 & (3) & b-a\neq 0  & \rAEa{2} \\
            1 & (4) & a+(b-a)=b  & \rAEb{2} \\
            1 & (5) & a+(b-a)\neq 0  & \bNotEqualsZeroImpaPlusbNotEqualsZero{3} \\
            1 & (6) & (a+(b-a))-1=b-1  & \rPredecessorUniqueness{4,5} \\
              & (7) & (a+(b-a))-1=a+((b-a)-1)  & \rAssociativityMonoid{} \\
            1 & (8) & a+((b-a)-1)=b-1  & \rIE{7,6} \\
            1 & (9) & a\leq b-1  & \rLeqNI{8} \\
\end{array}
\]
\(\dashv:\)
\[
\begin{array}{llll}
            1 & (1) & a\leq b-1  & \rA{} \\
            1 & (2) & a+((b-1)-a)=b-1  & \minusI{1} \\
            1 & (3) & (a+((b-1)-a))+1=(b-1)+1  & \aEqualsbEqvaPluscEqualsbPlusc{2} \\
              & (4) & (a+((b-1)-a))+1=a+(((b-1)-a)+1)  & \rAssociativityMonoid{} \\
              & (5) & ((b-1)-a)+1\neq 0  & \nInNaturalImpnPlusOneNotEqualsZero{} \\
              & (6) & (b-1)+1=b  & \rPredecessorEa{} \\
            1 & (7) & (a+((b-1)-a))+1=b  & \rIE{6,3} \\
            1 & (8) & a+(((b-1)-a)+1)=b  & \rIE{4,7} \\
            1 & (9) & a\leq b  & \rLeqNI{8} \\
            1 & (10) & a\leq b  & \InducedStrictOrderI{5,9} \\
\end{array}
\]
\end{proof}


\label{aLneqbPlusOneEqvaLeqb}
\begin{theorem}[\(a<b+1\dashv\vdash a\leq b\)]
Seien \(a,b\in\mathbb{N}\), dann gilt:
\[a<b+1\dashv\vdash a\leq b\]
\end{theorem}
\begin{proof}
        Seien \(a,b\in\mathbb{N}\). Im Beweis benutzen wir die Eigenschaft \(\ImpLpNaturalwPluswZeroRpInMonoid{}\). Hiermit gilt:

        
\(\vdash:\)
\[
\begin{array}{llll}
            1 & (1) & a<b+1  & \rA{} \\
            1 & (2) & a\leq (b+1)-1  & \aLneqbEqvaLeqbMinusOne{1} \\
              & (3) & b=(b+1)-1  & \rPredecessorEc{} \\
            1 & (4) & a\leq b  & \rIE{3,2} \\
\end{array}
\]
\(\dashv:\)
\[
\begin{array}{llll}
            1 & (1) & a\leq b  & \rA{} \\
              & (2) & b=(b+1)-1  & \rPredecessorEc{} \\
            1 & (3) & a\leq (b+1)-1  & \rIE{2,1} \\
            1 & (4) & a<b+1  & \aLneqbEqvaLeqbMinusOne{3} \\
\end{array}
\]
\end{proof}

\label{aInNaturalLpaLneqOneEqvaEqualsZeroRp}
\begin{theorem}[\(a<1\dashv\vdash a=0\)]
Sei \(a\in\mathbb{N}\), dann gilt:
\[a<1\dashv\vdash a=0\]
\end{theorem}
\begin{proof}
Sei \(a\in\mathbb{N}\). Im Beweis verwenden wir \(\LeqIsHalfOrderOnNaturalNumbers{}\):
\(\vdash:\)
	\[
	\begin{array}{llll}
		1 & (1) & a<1 & \rA \\
		1 & (2) & a\leq 1-1 & \aLneqbEqvaLeqbMinusOne{1} \\
		   & (3) & 1-1=0 & \aInNaturalImpaMinusaEqualsZero{} \\
        1 & (4) & a\leq 0 & \rIE{3,2} \\
          & (5) & 0\leq a & \ImpZeroLeqa{} \\
        1 & (6) & a=0 & \rAntisymmetryOrdRI{4,5} \\
	\end{array}
	\]
	\(\dashv:\)
	\[
	\begin{array}{llll}
		1 & (1) & a=0 & \rA \\
          & (2) & 0\leq 1 & \ImpZeroLeqa{} \\
          & (3) & 0\neq 1 & \ImpOneNotEqualsZero{} \\
        1 & (4) & a\neq 1 & \rIE{1,3} \\
        1 & (5) & a\leq 1 & \rIE{1,2} \\
        1 & (6) & a<1 & \InducedStrictOrderI{4,5} \\
	\end{array}
	\]
\end{proof}


\label{aInNaturalwaGeqOneEqvaNotEqualsZero}
\begin{theorem}[\(a\geq 1\dashv\vdash a\neq 0\)]
Sei \(a\in\mathbb{N}\), dann gilt:
\[a\geq 1\dashv\vdash a\neq 0\]
\end{theorem}
\begin{proof}
Sei \(a\in\mathbb{N}\).
\(\vdash:\)
	\[
	\begin{array}{llll}
		1 & (1) & a\geq 1 & \rA \\
		1 & (2) & 1\leq a & \rgeqE{1} \\
            1 & (3) & \neg(a<1) & \nLpaLneqbRpEqvbLeqa{2} \\
              & (4) & a<1\leftrightarrow a=0 & \aInNaturalLpaLneqOneEqvaEqualsZeroRp{} \\
            1 & (5) & a\neq 0 & \PLrQwnPImpnQ{4,3} \\
	\end{array}
	\]
	\(\dashv:\)
	\[
	\begin{array}{llll}
		1 & (1) &  a\neq 0 & \rA \\
            & (2) & a<1\leftrightarrow a=0 & \aInNaturalLpaLneqOneEqvaEqualsZeroRp{} \\
            1 & (3) &  \neg(a<1) & \PLrQwnQImpnP{2,1} \\
            1 & (4) &  1\leq a & \nLpaLneqbRpEqvbLeqa{3} \\
            1 & (5) &  a\geq 1 & \rgeqI{4} \\
	\end{array}
	\]
\end{proof}

\label{aInNaturalwaGneqOneImpaNotEqualsZero}
\begin{theorem}[\(a\in\mathbb{N},a> 1\vdash a\neq 0\)]
\end{theorem}
\begin{proof}
Sei \(a\in\mathbb{N}\).
	\[
	\begin{array}{llll}
		1 & (1) & a>1 & \rA \\
		1 & (2) & a\geq 1 & \aGneqbImpaGeqb{1} \\
            1 & (3) & s\neq 0 & \aInNaturalwaGeqOneEqvaNotEqualsZero{2} \\
	\end{array}
	\]
\end{proof}

\label{nInNaturalwnGneqOneImpZeroLeqnMinusOne}
\begin{theorem}[\(n\in\mathbb{N}, n>1\vdash 0\leq n-1\)]
\end{theorem}
\begin{proof}
Sei \(n\in\mathbb{N}\), dann gilt:
\(\vdash\):
    \[
	\begin{array}{llclll}
            1 &  (1)  & \multicolumn{3}{l}{n>1} & \rA \\
            1 &  (2)  & \multicolumn{3}{l}{n\neq 0} & \aInNaturalwaGneqOneImpaNotEqualsZero{1} \\
            1 &  (3)  & \multicolumn{3}{l}{n-1\in\mathbb{N}} & \rPredecessorI{2} \\
            1 &  (4)  & \multicolumn{3}{l}{0\leq n-1} & \ImpZeroLeqa{3} \\
        \end{array}
    \]
\end{proof}


\subsection{Invarianz der induzierten strikten Ordnungsrelation gegenüber Addition}

\label{awbwcInNaturalLpaLneqbEqvaPluscLneqbPluscRp}
\begin{theorem}[\(a<b\dashv\vdash a+c<b+c\)]
Seien \(a,b,c\in\mathbb{N}\), dann gilt:
\[a<b\dashv\vdash a+c<b+c\]
\end{theorem}
\begin{proof}
        Seien \(a,b,c\in\mathbb{N}\). 
\(\vdash:\)
\[
\begin{array}{llll}
            1 & (1) & a<b  & \rA{} \\
            1 & (2) & a\leq b  & \InducedStrictOrderE{1} \\
            1 & (3) & a\neq b  & \InducedStrictOrderE{1} \\
            1 & (4) & a+c\leq b+c  &  \awbwcInNaturalLpaLeqbEqvaPluscLeqbPluscRp{2} \\
            1 & (5) & a+c\neq b+c  & \aNotEqualsbEqvaPluscNotEqualsbPlusc{3} \\
            1 & (6) & a+c<b+c & \InducedStrictOrderI{4,5} \\
\end{array}
\]
\(\dashv:\)
\[
\begin{array}{llll}
            1 & (1) & a+c<b+c  & \rA{} \\
            1 & (2) & a+c\leq b+c  & \InducedStrictOrderE{1} \\
            1 & (3) & a+c\neq b+c  & \InducedStrictOrderE{1} \\
            1 & (4) & a\leq b  & \awbwcInNaturalLpaLeqbEqvaPluscLeqbPluscRp{2}
            \\
            1 & (5) & a\neq b  & \aNotEqualsbEqvaPluscNotEqualsbPlusc{3}
            \\
            1 & (5) & a< b  & \InducedStrictOrderI{4,5}
            \\
            
\end{array}
\]
\end{proof}

\label{aLneqbEqvcPlusaLneqcPlusb}
\begin{theorem}[\(a<b\dashv\vdash c+a<c+b\)]
Seien \(a,b,c\in\mathbb{N}\), dann gilt:
\[a<b\dashv\vdash c+a<c+b\]
\end{theorem}
\begin{proof}
        Seien \(a,b,c\in\mathbb{N}\). 
\(\vdash:\)
\[
\begin{array}{llll}
            1 & (1) & a<b  & \rA{} \\
            1 & (2) & a\leq b  & \InducedStrictOrderE{1} \\
            1 & (3) & a\neq b  & \InducedStrictOrderE{1} \\
            1 & (4) & c+a\leq c+b  &  \awbwcInNaturalLpaLeqbEqvcPlusaLeqcPlusbRp{2} \\
            1 & (5) & c+a\neq c+b  & \aNotEqualsbEqvcPlusaNotEqualscPlusb{3} \\
            1 & (6) & c+a<c+b & \InducedStrictOrderI{4,5} \\
\end{array}
\]
\(\dashv:\)
\[
\begin{array}{llll}
            1 & (1) & a+c<b+c  & \rA{} \\
            1 & (2) & a+c\leq b+c  & \InducedStrictOrderE{1} \\
            1 & (3) & a+c\neq b+c  & \InducedStrictOrderE{1} \\
            1 & (4) & a\leq b  & \awbwcInNaturalLpaLeqbEqvcPlusaLeqcPlusbRp{2}
            \\
            1 & (5) & a\neq b  & \aNotEqualsbEqvcPlusaNotEqualscPlusb{3}
            \\
            1 & (5) & a<b  & \InducedStrictOrderI{4,5}
            \\
            
\end{array}
\]
\end{proof}

\label{aInNaturalwbInNaturalwcInNaturalwaLeqbwcNotEqualsZeroImpaLneqbPlusc}
\begin{theorem}[\(a\in\mathbb{N},b\in\mathbb{N},c\in\mathbb{N}, a\leq b,c\neq 0\vdash a<b+c\)]
Seien \(a,b,c\in\mathbb{N}\), dann gilt:
\[a\leq b,c\neq 0\vdash a<b+c\]
\end{theorem}
\begin{proof}
        Seien \(a,b,c\in\mathbb{N}\). Im Beweis benutzen wir die Eigenschaften:
\begin{enumerate}
    \item \(\ImpLpNaturalwPluswZeroRpInMonoid{}\).
    \item \(\LeqIsHalfOrderOnNaturalNumbers{}\).
    \item \(\FaSLpEqualsInEquivalencerelationSRp{}\).
\end{enumerate}
Hiermit gilt:
\[
\begin{array}{llclll}
            1 & (1) & \multicolumn{3}{l}{c\neq 0}  & \rA \\
            1 & (2) & \multicolumn{3}{l}{c-1\in\mathbb{N}}  & \rPredecessorI{1} \\
            3 & (3) & a&\leq &b  & \rA  \\
            1,3 & (4) & &\leq &b+(c-1)  &  \aInNaturalwbInNaturalwcInNaturalwaLeqbImpaLeqbPlusc{2,3} \\
            1,3 & (5) & &< &((b+(c-1))+1  &  \aLneqbPlusOneEqvaLeqb{4} \\
            1,3 & (6) & &=& b+((c-1)+1) & \rAssociativityMonoid{} \\
            1,3 & (7) & &=& b+c & \rPredecessorI{2} \\
            1,3 & (8) & \multicolumn{3}{l}{a<b+c} & \rTransitivityEqRI{3,7} \\
\end{array}
\]
\end{proof}

\label{aInNaturalwbInNaturalwcInNaturalwaLeqbwcNotEqualsZeroImpaLneqcPlusb}
\begin{theorem}[\(a\in\mathbb{N},b\in\mathbb{N},c\in\mathbb{N}, a\leq b,c\neq 0 \vdash a<c+b\)]
\end{theorem}
\begin{proof}
        Seien \(a,b,c\in\mathbb{N}\). Im Beweis benutzen wir die Eigenschaften:
\begin{enumerate}
    \item \(\ImpLpNaturalwPluswZeroRpInMonoid{}\).
    \item \(\LeqIsHalfOrderOnNaturalNumbers{}\).
    \item \(\FaSLpEqualsInEquivalencerelationSRp{}\).
\end{enumerate}
Hiermit gilt:
\[
\begin{array}{llclll}
            1 & (1) & \multicolumn{3}{l}{c\neq 0}  & \rA \\
            2 & (2) & a&\leq &b  & \rA  \\
            1,2 & (3) & &<&b+c  &  \aInNaturalwbInNaturalwcInNaturalwaLeqbwcNotEqualsZeroImpaLneqbPlusc{1,2} \\
            1,2 & (4) & &= &c+b  &  \rCommutativeMonoid{} \\
            1,2 & (5) & \multicolumn{3}{l}{a<c+b} & \rTransitivityEqRI{2,4} \\
\end{array}
\]
\end{proof}

\label{awbInNaturalwbNotEqualsZeroImpaLneqaPlusb}
\begin{theorem}[\(a,b\in\mathbb{N},b\neq 0\vdash a<a+b\)]
\end{theorem}
\begin{proof}
        Seien \(a,b\in\mathbb{N}\). Im Beweis benutzen wir die Eigenschaften:
\begin{enumerate}
    \item \(\LeqIsHalfOrderOnNaturalNumbers{}\).
\end{enumerate}
Hiermit gilt:
\[
\begin{array}{llll}
              & (1) & a\leq a  & \rReflexivityOrdRI{} \\
            2 & (2) & b\neq 0  & \rA \\
            2 & (3) & a<a+b & \aInNaturalwbInNaturalwcInNaturalwaLeqbwcNotEqualsZeroImpaLneqbPlusc{1,2} \\
\end{array}
\]
\end{proof}

\label{awbInNaturalbNotEqualsZeroImpaLneqbPlusa}
\begin{theorem}[\(a,b\in\mathbb{N}b\neq 0\vdash a<b+a\)]
\end{theorem}
\begin{proof}
        Seien \(a,b\in\mathbb{N}\). Im Beweis benutzen wir die Eigenschaften:
\begin{enumerate}
    \item \(\LeqIsHalfOrderOnNaturalNumbers{}\).
\end{enumerate}
Hiermit gilt:
\[
\begin{array}{llll}
              & (1) & a\leq a  & \rReflexivityOrdRI{} \\
            2 & (2) & b\neq 0  & \rA \\
            2 & (3) & a<b+a & \aInNaturalwbInNaturalwcInNaturalwaLeqbwcNotEqualsZeroImpaLneqcPlusb{1,2} \\
\end{array}
\]
\end{proof}

\label{awbwcInNaturalLpaPlusbLeqcwbNotEqualsZeroImpaLneqc}
\begin{theorem}[\(a,b,c\in\mathbb{N}(a+b\leq c,b\neq 0 \vdash a<c)\)]
\end{theorem}
\begin{proof}
        Seien \(a,b,c\in\mathbb{N}\). Im Beweis benutzen wir die Eigenschaften:
\begin{enumerate}
    \item \(\LeqIsHalfOrderOnNaturalNumbers{}\).
\end{enumerate}
Hiermit gilt:
\[
\begin{array}{llclll}
            1 & (1) & \multicolumn{3}{l}{b\neq 0}  & \rA \\
            1 & (2) & a&<& a+b  & \awbInNaturalwbNotEqualsZeroImpaLneqaPlusb{1} \\
            1,3 & (3) &  &\leq &c & \rA \\
            1,3 & (4) & \multicolumn{3}{l}{a<c} & \rTransitivityOrdRI{2,3} \\
\end{array}
\]
\end{proof}


\label{awbwcInNaturalLpaPlusbLeqcImpbLneqcRp}
\begin{theorem}[\(a,b,c\in\mathbb{N}(a+b\leq c\vdash b<c)\)]
\end{theorem}
\begin{proof}
        Seien \(a,b,c\in\mathbb{N}\). Im Beweis benutzen wir die Eigenschaften:
\begin{enumerate}
    \item \(\LeqIsHalfOrderOnNaturalNumbers{}\).
\end{enumerate}
Hiermit gilt:
\[
\begin{array}{llclll}
            1 & (1) & \multicolumn{3}{l}{b\neq 0}  & \rA \\
            1 & (2) & b&<& a+b  & \awbInNaturalbNotEqualsZeroImpaLneqbPlusa{1} \\
            1,3 & (3) &  &\leq &c & \rA \\
            1,3 & (4) & \multicolumn{3}{l}{b<c} & \rTransitivityOrdRI{2,3} \\
\end{array}
\]
\end{proof}

\subsection{Eigenschaften der Differenz}

\label{awbInNaturalLpaLneqbEqvZeroLneqbMinusaRp}
\begin{theorem}[\(a,b\in\mathbb{N},a<b\dashv\vdash 0<b-a\)]
Seien \(a,b\in\mathbb{N}\), dann gilt:
\[a<b\dashv\vdash 0<b-a\]
\end{theorem}
\begin{proof}
    \(\vdash\):
    \[
    \begin{array}{llll}
        1 & (1) & a<b  & \rA \\
        1 & (2) & a\leq b  & \rLeqNE{1} \\
        1 & (3) & 0\leq b-a  & \awbInNaturalLpaLeqbEqvZeroLeqbMinusaRp{2} \\
        1 & (4) & a\neq b  & \rLeqNE{1} \\
        1 & (5) & 0\neq a-b  & \aNotEqualsbEqvaMinusbNotEqualsZero{4} \\
        1 & (6) & 0\leq a-b  & \rLeqNI{3,5} \\
    \end{array}
    \]
    \(\dashv\):
    \[
    \begin{array}{llll}
        1 & (1) & 0<b-a  & \rA \\
        1 & (2) & 0\leq b-a  & \rLeqNE{1} \\
        1 & (3) & a\leq b  & \awbInNaturalLpaLeqbEqvZeroLeqbMinusaRp{2} \\
        1 & (4) & 0\neq b-a  & \rLeqNE{1} \\
        1 & (5) & a\neq b  & \aNotEqualsbEqvaMinusbNotEqualsZero{4} \\
        1 & (6) & a<b  & \rLeqNI{3,5} \\
    \end{array}
    \]
\end{proof}


\label{awbwcInNaturalwaLeqbLpbLneqcEqvbMinusaLneqcMinusaRp}
\begin{theorem}[\(b<c \dashv\vdash b-a<c-a\)]
Seien \(a,b,c\in\mathbb{N}\) und \(a\leq b\), dann gilt:
\[b<c \dashv\vdash b-a<c-a\]
\end{theorem}
\begin{proof}
Seien \(a,b,c\in\mathbb{N}\).:
    \(\vdash\):
    \[
    \begin{array}{llll}
        1 & (1) & b<c  & \rA \\
        2 & (2) & a\leq b  & \rA \\
        1 & (3) & b\leq c  & \rLeqNE{1} \\
        1 & (4) & b-a\leq c-a  & \awbwcInNaturalwaLeqbLpbLeqcEqvbMinusaLeqcMinusaRp{1,2} \\
        1 & (5) & b\neq c  & \rLeqNE{1} \\
      1,2 & (6) & b-a\neq c-a  & \awbwcInNaturalwcLeqaLpaNotEqualsbEqvaMinuscNotEqualsbMinuscRp{2,5} \\
    \end{array}
    \]
    \(\dashv\):
    \[
    \begin{array}{llll}
        1 & (1) & b-a<c-a  & \rA \\
        2 & (2) & a\leq b  & \rA \\
        1 & (3) & b-a\leq c-a  & \rLeqNE{1} \\
        1 & (4) & b\leq c  & \awbwcInNaturalwaLeqbLpbLeqcEqvbMinusaLeqcMinusaRp{1,2} \\
        1 & (5) & b-a\neq c-a  & \rLeqNE{1} \\
      1,2 & (6) & b\neq c  & \awbwcInNaturalwcLeqaLpaNotEqualsbEqvaMinuscNotEqualsbMinuscRp{2,5} \\
    \end{array}
    \]
\end{proof}

\label{bLneqcwaLeqbImpbMinusaLneqc}
\begin{theorem}[\(b<c, a\leq b\vdash b-a<c\)]
\end{theorem}
\begin{proof}
Seien \(a, b, c \in \mathbb{N}\). 
\[
\begin{array}{llclll}
          1  & (1) & \multicolumn{3}{l}{a\leq b}  & \rA \\
          1  & (2) & \multicolumn{3}{l}{b-a\in\mathbb{N}}  & \minusI{1} \\
          1  & (3) & b-a&\leq& a+(b-a)  & \awbInNaturalImpaLeqaPlusb{2} \\
          1  & (4) & &=& b  & \minusI{1} \\
          5  & (5) & &< & c  & \rA \\
          1,5& (6) & b-a&<& c  & \rTransitivityOrdRI{3,5} \\
\end{array}
\]
\end{proof}

\section{Endliche Teilmengen der natürlichen Zahlen}

\begin{definition}[Endliches Teilmengen der natürlichen Zahlen]
    Sei \(i, n \in \mathbb{N}\) mit \(i \leq n\). Dann definieren wir das endliche Teilmengen der natürlichen Zahlen von \(i\) bis \(n\) als die Menge
    \[
    \{i, i+1, \dots, n\} := \{ x \in \mathbb{N} \mid i \leq x \land x \leq n \}.
    \]
    Für den Spezialfall \(i = 0\) und \(n\in\mathbb{N}\) mit \(n \neq 0\) schreiben wir \(\{0, 1, \dots, n-1\} := \{ x \in \mathbb{N} \mid x < n \}\).
\end{definition}

\subsubsection{Einführungsregeln für endliche Teilmengen der natürlichen Zahlen}

Die Einführungsregeln ermöglichen es, die Zugehörigkeit eines Elements zu einem endlichen Segment der natürlichen Zahlen zu zeigen, wenn die jeweiligen Bedingungen erfüllt sind.

\paragraph{Einführungsregel für die Menge}
\label{rule:rSegmentI}
Die Einführungsregel für die Menge \(\{i, i+1, \dots, n\}\) besagt, dass ein Element \(x \in \mathbb{N}\) in \(\{i, i+1, \dots, n\}\) liegt, wenn \(i \leq x \leq n\) gilt.
\[
\begin{array}{llll}
    i   & (1) & i \leq x & \dots \\
    j   & (2) & x \leq n & \dots \\
    i,j & (3) & x \in \{i, i+1, \dots, n\} & \rSegmentI{1,2}
\end{array}
\]

\paragraph{Einführungsregel für die Menge}
\label{rule:rSegmentZeroI}
Die Einführungsregel für die Menge \(\{0, 1, \dots, n-1\}\) besagt, dass ein Element \(x \in \mathbb{N}\) in \(\{0, 1, \dots, n-1\}\) liegt, wenn \(x < n\) gilt.
\[
\begin{array}{llll}
    i   & (1) & x < n & \dots \\
    i   & (2) & x \in \{0, 1, \dots, n-1\} & \rSegmentZeroI{1}
\end{array}
\]

\[
\begin{array}{llll}
    i   & (1) & x \leq n-1 & \dots \\
    i   & (2) & x \in \{0, 1, \dots, n-1\} & \rSegmentZeroI{1}
\end{array}
\]

\(i\) und \(j\) sind dabei Listen von Annahmen.

\subsubsection{Eliminationsregeln für endliche Teilmengen der natürlichen Zahlen}
\label{rule:rSegmentE}
Die Eliminationsregeln ermöglichen es, aus der Zugehörigkeit eines Elements zu einer endlichen Menge die entsprechenden Bedingungen abzuleiten.

\paragraph{Eliminationsregel für die Menge}
Die Eliminationsregel besagt, dass wenn \(x \in \{i, i+1, \dots, n\}\) gilt, dann sowohl \(i \leq x\) als auch \(x \leq n\) folgt.
\[
\begin{array}{llll}
    i & (1) & x \in \{i, i+1, \dots, n\} & \dots \\
    i & (2) & i \leq x & \rSegmentE{1} \\
    i & (3) & x \leq n & \rSegmentE{1} \\
    i & (4) & i \leq n & \rSegmentE{1} \\
    i & (5) & i\in\mathbb{N} & \rSegmentE{1} \\
    i & (6) & n\in\mathbb{N} & \rSegmentE{1} \\
\end{array}
\]

\paragraph{Eliminationsregel für die Menge \(\{0, 1, \dots, n-1\}\)}
\label{rule:rSegmentZeroE}
Die Eliminationsregel besagt, dass wenn \(x \in \{0, 1, \dots, n-1\}\) gilt, dann \(x < n\) folgt.
\[
\begin{array}{llll}
    i & (1) & x \in \{0, 1, \dots, n-1\} & \dots \\
    i & (2) & x < n & \rSegmentZeroE{1} \\
    i & (3) & 0 \leq n-1 & \rSegmentZeroE{1} \\
    i & (4) & n-1\in\mathbb{N} & \rSegmentZeroE{1} \\
\end{array}
\]

\label{nInNaturalwnNotEqualsZeroImpZeroInLbZerowOnewDotswnMinusOneRb}
\begin{theorem}[\(n\in\mathbb{N},n\neq 0\vdash 0\in\{0, 1, \dots, n-1\}\)]
\end{theorem}
\begin{proof}
Sei \(n\in\mathbb{N}\), dann gilt:
        \[
	\begin{array}{llll}
            1       &  (1) & n\neq 0 & \rA \\
            1       &  (2) & n-1\in\mathbb{N} & \rPredecessorEa{1} \\
            1       &  (3) & 0\leq n-1\in\mathbb{N} & \ImpZeroLeqa{2} \\
            1       &  (4) & 0\leq n-1\in\mathbb{N} & \rSegmentZeroI{3} \\
        \end{array}
	\]
\end{proof}

\label{nInNaturalwnGneqOneImpZeroInLbZerowOnewDotswnMinusOneRb}
\begin{theorem}[\(n\in\mathbb{N},n>1\vdash 0\in\{0, 1, \dots, n-1\}\)]
\end{theorem}
\begin{proof}
Sei \(n\in\mathbb{N}\), dann gilt:
        \[
	\begin{array}{llll}
            1       &  (1) & n>1 & \rA \\
            1       &  (2) & n\neq 0 & \aInNaturalwaGneqOneImpaNotEqualsZero{1} \\
            1       &  (3) & 0\leq n-1\in\mathbb{N} & \nInNaturalwnNotEqualsZeroImpZeroInLbZerowOnewDotswnMinusOneRb{2} \\
        \end{array}
	\]
\end{proof}


\section{Prinzip der starken Induktion}

Um das Prinzip der starken Induktion zu zeigen, benötigen wir zunächst ein paar Hilfssätze.

\label{aInNaturalwbInNaturalwaLeqbPlusOneImpaLeqbOraEqualsbPlusOne}
\begin{theorem}[\(a\in\mathbb{N},b\in\mathbb{N},a\leq b+1\vdash a\leq b\lor a=b+1\)]
\end{theorem}
\begin{proof}
        Seien \(a,b\in\mathbb{N}\). Im Beweis benutzen wir die Eigenschaft \(\ImpLpNaturalwPluswZeroRpInAbelMonoid{}\). Hiermit gilt:
        \[
	\begin{array}{llll}
            1       &  (1) & a\leq b+1 & \rA \\
            1       &  (2) & (b+1)-a\in\mathbb{N} & \minusI{1} \\
            1       &  (3) & (b+1)-a=0\lor 0<(b+1)-a & \aInNaturalImpaEqualsZeroOrZeroLneqa{2} \\
            4       &  (4) & (b+1)-a=0 & \rA \\  
            4       &  (5) & a=(b+1) & \awbInNaturalLpbEqualsaEqvaMinusbEqualsZeroRp{4} \\ 
            4       &  (6) & a\leq b\lor a = b+1 & \rOIb{5}\\
            7       &  (7) & 0<(b+1)-a & \rA\\
            7       &  (8) & a<b+1 & \awbInNaturalLpaLneqbEqvZeroLneqbMinusaRp{7}\\
            7       &  (9) & a\leq b+1 & \aLneqbPlusOneEqvaLeqb{8}\\
            7       &  (10) & a\leq b\lor a=b+1 & \rOIa{9} \\
            1       &  (11) & a\leq b\lor a=b+1 & \rOE{3,4,6,7,10} \\
        \end{array}
	\]
\end{proof}



\label{FanInNaturalLpFakInNaturalLpkLeqnToPLpkRpImpFanInNaturalLpPLpnRpRpRpRp}
\begin{theorem}[\(\forall n\in\mathbb{N}(\forall k\in\mathbb{N}(k\leq n\rightarrow P(k))\vdash \forall n\in\mathbb{N}(P(n)))\)]
\end{theorem}
\begin{proof}
\[
\begin{array}{llll}
1 & (1) & \forall n\in\mathbb{N}\left(\forall k\in\mathbb{N}(k\leq n\rightarrow P(k)\right) & \rA \\
2 & (2) & m\in\mathbb{N} & \rA \\
1,2 & (3) & \forall k\in\mathbb{N}\left(k\leq m\rightarrow P(k)\right) & \rSetUEb{1,2} \\
1,2 & (4) & m\leq m\rightarrow P(m) & \rSetUEb{3,2} \\
    & (5) & m\leq m & \rReflexivityOrdRI{} \\
1,2 & (6) & P(m) & \rRE{5,4} \\
1  & (7) & \forall n\in\mathbb{N}(P(n)) & \rSetUIa{2,6} \\
\end{array}
\]
\end{proof}

\begin{lemma}[Induktionsanfang (\(IA\))]
\[P(0)\vdash \forall k\in\mathbb{N}(k\leq 0\rightarrow P(k))\]
\end{lemma}
\begin{proof}
\[
\begin{array}{llll}
1 & (1) & P(0) & \rA \\
2 & (2) & k\in\mathbb{N} & \rA \\
3 & (3) & k\leq 0 & \rA \\
2,3 & (4) & k=0 & \aInNaturalwaLeqZeroImpaEqualsZero{2,3} \\
1,2,3 & (5) & P(k) & \rIE{1,4} \\
1,2 & (6) & k\leq 0\rightarrow P(k) & \rRI{3,5} \\
1 & (7) & \forall k\in\mathbb{N}(k\leq 0\rightarrow P(k)) & \rSetUIa{2,6} \\
\\
\end{array}
\]
\end{proof}

\begin{lemma}[Induktionsschritt unter Bedingung \(P(m+1)\) (\(IS_{P(m+1)}\))]
\[m\in\mathbb{N}, P(m+1), \forall k\in\mathbb{N}(k\leq m\rightarrow P(k)) \vdash \forall k\in\mathbb{N}(k\leq m+1\rightarrow P(k))\]
\end{lemma}
\begin{proof}
   \[
\begin{array}{lll p{3.56cm}}
1 & (1) & m\in\mathbb{N} & \rA \\
2 & (2) & P(m+1) & \rA \\
3 & (3) & \forall k\in\mathbb{N}(k\leq m\rightarrow P(k)) & \rA \\
4 & (4) & k\in\mathbb{N} & \rA \\
5 & (5) & k\leq m+1 & \rA \\
1,4,5 & (6) & k\leq m\lor k=m+1 & \aInNaturalwbInNaturalwaLeqbPlusOneImpaLeqbOraEqualsbPlusOne{1,4,5} \\
7 & (7) & k\leq m & \rA \\
7 & (8) & k\leq m & \rA \\
4 & (9) & k\leq m\rightarrow P(k) & \rSetUEc{4,3} \\
4,7 & (10) & P(k) & \rRE{8,9} \\
11 & (11) & k=m+1 & \rA \\
2,3,11 & (12) & P(k) & \rIE{11,2} \\
1,2,3,4,5 & (13) & P(k) & \rOE{6,7,10,11,12} \\
1,2,3,4 & (14) & k\leq m+1\rightarrow P(k) & \rRI{5,13} \\
1,2,3 & (15) & \forall k\in\mathbb{N}(k\leq m+1\rightarrow P(k)) & \rSetUIa{4,14} \end{array}
\] 
\end{proof}


\label{PLpZeroRpwFanInNaturalLpLpFakInNaturalLpkLeqnToPLpkRpRpToPLpnPlusOneRpRpImpFanInNaturalLpPLpnRpRp}
\begin{theorem}[\(P(0), \forall n \in \mathbb{N} ((\forall k\in\mathbb{N}(k\leq n\rightarrow P(k)) \rightarrow P(n+1)) \vdash \forall n \in \mathbb{N}(P(n))\) (Starkes Induktionsprinzip)]
\end{theorem}
\begin{proof}

   \[
\begin{array}{lll p{3.56cm}}
1 & (1) & P(0) & \rA \\
2 & (2) & \forall n \in \mathbb{N}(\forall k\in\mathbb{N}(k\leq n\rightarrow P(k)) \rightarrow P(n+1)) & \rA \\
1 & (3) & \forall k\in\mathbb{N}(k\leq 0\rightarrow P(k)) & \ensuremath{IA(1)} \\
4 & (4) & m\in\mathbb{N} & \rA \\
5 & (5) & \forall k\in\mathbb{N}(k\leq m\rightarrow P(k)) & \rA \\
2,4 & (6) & \forall k\in\mathbb{N}(k\leq m\rightarrow P(k)) \rightarrow P(m+1) & \rSetUEc{2,4} \\
2,4,5 & (7) & P(m+1) & \rRE{6,5} \\
2,4,5 & (8) & \forall k\in\mathbb{N}(k\leq m+1\rightarrow P(k)) & \ensuremath{IS_{P(m+1)}(4,7,5)} \\
1,2 & (9) & \forall n\in\mathbb{N}(\forall k\in\mathbb{N}(k\leq n\rightarrow P(k)) & \rInductionN{3,4,5,8} \\
1,2 & (10) & \forall n\in\mathbb{N}(P(n))) & \FanInNaturalLpFakInNaturalLpkLeqnToPLpkRpImpFanInNaturalLpPLpnRpRpRpRp{9} \\
\end{array}
\] 
\end{proof}


\subsubsection{Regel des starken Induktionsprinzips über den natürlichen Zahlen}
\label{rule:rStrongInductionN}

Die Regel des starken Induktionsprinzips über den natürlichen Zahlen (\(\rStrongInductionN{}\)) erlaubt es, nach der Herleitung von \(P(0)\) und der Annahme, dass für ein beliebiges \(n \in \mathbb{N}\) die Aussage \(P(n+1)\) aus der Voraussetzung \(P(k)\) für alle \(k \leq n\) folgt, direkt auf die Aussage \(\forall n \in \mathbb{N} P(n)\) zu schließen. Es ist somit nicht mehr notwendig, die Zwischenschritte \(\forall n \in \mathbb{N} ((\forall k\in\mathbb{N}(k\leq n\rightarrow P(k)) \rightarrow P(n+1))\) explizit aufzuschreiben.

\[
\begin{array}{llll}
    i & (1) & P(0) & ... \\
    2 & (2) & \forall k \in \mathbb{N} (k \leq n \rightarrow P(k)) & \rA \\
    3 & (3) & n \in \mathbb{N} & \rA \\
    2,3,j & (4) & P(n+1) & ... \\
    i,j & (5) & \forall n \in \mathbb{N} P(n) & \rStrongInductionN{1,2,3,4}
\end{array}
\]

Hierbei beziehen sich \(n\) und \(P(n)\) auf ein beliebiges Element und eine beliebige Aussage über \(\mathbb{N}\). Der Ausdruck \(\rStrongInductionN{1,4}\) zeigt an, dass die Regel des starken Induktionsprinzips angewendet wurde, um die allgemeine Aussage \(\forall n \in \mathbb{N} P(n)\) abzuleiten.

\(i\) und \(j\) sind dabei Listen von Annahmen, und \(n\) kommt in keiner der Annahmen \(i\) und \(j\) vor, aus denen \(P(0)\) und \(P(n+1)\) abgeleitet werden.

\section{Extremale Elemente und Schranken der natürlichen Zahlen}
\begin{theorem}[\(\vdash \min(\mathbb{N}) = 0\)]
\end{theorem}
\begin{proof}
        Sei \(a\in\mathbb{N}\), dann gilt:
	\[
	\begin{array}{llll}
          & (1) & 0\leq \mathbb{N} & \zeroIsNaturalNumber{} \\
		 & (2) & 0\leq a & \ImpZeroLeqa{} \\
          & (3) & \forall a\in\mathbb{N}(0\leq a) & \rSetUIa{2} \\
          & (4) & \min(\mathbb{N}) = 0 & \rMinI{1,3} \\
	\end{array}
	\]
\end{proof}

\subsection{Das Wohlordnungsprinzip}
\begin{lemma}[1]
\[A\subseteq\mathbb{N},\forall n\in A\exists m\in A(m<n)\vdash 0\notin A\]
\end{lemma}
\begin{proof}
	\[
	\begin{array}{llll}
        1  & (1) & A\subseteq\mathbb{N} & \rA \\
	2  & (2) & \forall n\in A\exists m\in A(m<n) & \rA \\
        3  & (3) & 0\in A & \rA \\
        2,3& (4) & \exists m\in A(m<0) & \rSetUEc{3,2} \\
        5  & (5) & m\in A\land m<0 & \rA \\
        5  & (6) & m\in A          & \rAEa{5} \\
        5  & (7) & m<0             & \rAEb{5} \\
        1,5& (8) & m\in\mathbb{N}  & \subseteqE{6,1} \\
        1,5& (9) & 0\leq m & \ImpZeroLeqa{8} \\
        1,5& (10) & \neg(m<0) & \nLpaLneqbRpEqvbLeqa{9} \\
        1,5& (11) & \bot & \rBI{7,10} \\
        1,2,3& (12) & \bot & \rOE{4,5,12} \\
        1,2& (13) & 0\notin A & \rCI{3,12} \\
	\end{array}
	\]
\end{proof}

\begin{lemma}[2]
\[\neg(\forall n\in A\exists m\in A(m<n))\vdash \exists n\in A(n=\min(A))\]
\end{lemma}
\begin{proof}
	\[
	\begin{array}{llll}
        1  & (1) & \neg(\forall n\in A\exists m\in A(m<n)) & \rA \\
        1  & (2) & \exists n\in A \forall m\in A(\neg(m<n)) & \nLpFaxInAExyInBLpPLpxwyRpRpRpEqvExxInAFayInBLpnPLpxwyRpRp{1} \\
        3  & (3) & n\in A\land \forall m\in A(\neg(m<n) & \rA\\
        3  & (4) & n\in A & \rAEa{3}\\
        3  & (5) & \forall m\in A(\neg(m<n) & \rAEb{3}\\
        3  & (6) & m\in A\rightarrow \neg(m<n) & \rSetUEb{5}\\
        7  & (7) & m\in A & \rA\\
        3,7& (8) & \neg(m<n) & \rRE{7,6}\\
        3,7& (9) & n\leq m & \nLpaLneqbRpEqvbLeqa{8}\\
        3  & (10) & \forall m\in A(n\leq m) & \rSetUIa{7,9}\\
        3  & (11) & n=\min(A) & \rMinI{4,10}\\
        3  & (12) & \exists n\in A(n=\min(A)) & \rSetEIa{4,11}\\
        1  & (13) & \exists n\in A(n=\min(A)) & \rSetEEb{2,3,12}\\
	\end{array}
	\]
\end{proof}

\label{ASubseteqNaturalwANotEqualsEmptysetImpExnInALpnEqualsMinLpARpRp}
\begin{theorem}[\(A\subseteq\mathbb{N},A\neq\emptyset\vdash\exists n\in A(n=\min(A))\) (Wohlordnungsprinzip)]
\end{theorem}
Wir beweisen die Aussage mit Hilfe des Prinzips der starken Induktion über \(n\in\mathbb{N}\):
\begin{proof}
	\[
	\begin{array}{llll}
        1  & (1) & A\subseteq\mathbb{N}  & \rA \\
		2  & (2) & A\neq\emptyset        & \rA \\
        3  & (3) & \forall n\in A\exists m\in A(m<n) & \rA \\
        1,3& (4) & 0\notin A & \text{Lemma (1)}(1,3) \\
        5  & (5) & \forall k\in\mathbb{N}(k\leq n\rightarrow k\notin A) & \rA \\
        6  & (6) & n+1\in A & \rA \\
        3,6& (7) & \exists m\in A(m<n+1)    & \rSetUEc{3,6} \\
        8  & (8) & m\in A\land m<n+1        & \rA \\
        8  & (9) & m\in A                   & \rAEa{8} \\
        8  & (10) & m<n+1                   & \rAEb{8} \\
        8  & (11) & m\leq n                 & \aLneqbPlusOneEqvaLeqb{10} \\
        1,8& (12) & m\in\mathbb{N}          & \subseteqE{9,1} \\
        1,5,8& (13) & m\leq n\rightarrow m\notin A          & \rSetUEc{12,5} \\
        1,5,8& (14) & m\notin A          & \rRE{11,13} \\
        1,5,8& (15) & \bot          & \rBI{9,14} \\
        1,3,5,6& (16) & \bot          & \rEE{7,8,15} \\
        1,3,5& (17) & n+1\notin A      & \rCI{6,16} \\
        1,3& (18) & \forall n\in\mathbb{N}(n\notin A)      & \rStrongInductionN{4,5,17} \\
        1,3& (19) & A=\emptyset      & \ImpFaALpEmptysetSubseteqARp{1,18} \\
        1,2,3& (20) & \bot      & \rBI{2,19} \\
        1,2 & (21) & \neg(\forall n\in A\exists m\in A(m<n))      & \rCI{3,20} \\
        1,2 & (22) & \exists n\in A(n=\min(A))      & \text{Lemma (2)}(21) \\
	\end{array}
	\]
\end{proof}

\chapter{Multiplikation von natürlichen Zahlen}

\begin{definition}[Multiplikation]
    Die Multiplikation von zwei natürlichen Zahlen \( a \) und \( b \) ist eine binäre Operation \( \cdot: \mathbb{N} \times \mathbb{N} \to \mathbb{N} \), die rekursiv wie folgt definiert wird:
    
    \begin{itemize}
        \item \textbf{Basisfall}: Für jede natürliche Zahl \( a \) gilt:
        \[
        a \cdot 0 := 0.
        \]
        
        \item \textbf{Rekursionsschritt}: Für jede natürliche Zahl \( b \) gilt:
        \[
        a \cdot (b+1) := (a \cdot b) + a.
        \]
    \end{itemize}
\end{definition}
\begin{remark}
In der Mathematik ist es üblich, das Multiplikationszeichen \(\cdot\) zwischen zwei Operanden wegzulassen, wenn keine Verwechslung mit anderen Operationen möglich ist. Daher kann \( a \cdot b \) auch als \( ab \) geschrieben werden.

Zudem gilt in der Arithmetik allgemein die Regel, dass Multiplikationen vor Additionen ausgeführt werden (Punktrechnung vor Strichrechnung). Diese Konvention ermöglicht es uns, bestimmte Klammern wegzulassen, ohne die Bedeutung eines Ausdrucks zu verändern.

Zum Beispiel bedeutet der Ausdruck \( ab + a \) nach den Regeln der Arithmetik, dass zunächst das Produkt \( ab \) berechnet und anschließend \( a \) addiert wird. Es ist daher nicht erforderlich, den Ausdruck als \( (a \cdot b) + a \) zu schreiben, da die Reihenfolge der Operationen durch die Regel der Punktrechnung vor Strichrechnung bereits eindeutig festgelegt ist.

In der Definition der Multiplikation von natürlichen Zahlen nutzen wir diese Regel: Der Ausdruck \( a \cdot (b+1) := a \cdot b + a \) bedeutet, dass zuerst \( a \cdot b \) berechnet wird und das Ergebnis dann um \( a \) erhöht wird. Die Klammer um \( b+1 \) stellt sicher, dass \( b \) zuerst um 1 erhöht wird, bevor das Ergebnis mit \( a \) multipliziert wird.
\end{remark}

\paragraph{Beweisregeln für die Multiplikation}
\label{rule:rMultI} 
Basierend auf der rekursiven Definition der Multiplikation können wir die folgenden Regeln für die Multiplikation formulieren:

\[
\begin{array}{llll}
	i & (1) & a \in \mathbb{N} & ... \\
	j & (2) & b \in \mathbb{N} & ... \\                                    
        i & (3) & 0=a \cdot 0  & \rMultI{1} \\
	i,j & (4) & a \cdot (b+1) = a \cdot b + a & \rMultI{1,2} \\
         & (5) & \cdot:\mathbb{N}\times\mathbb{N}\rightarrow\mathbb{N} & \rMultI{} \\
\end{array}
\]
\(i\) und \(j\) sind dabei Listen von Annahmen.

\label{aInNaturalImpZeroEqualsZeroMulta}
\begin{theorem}[\(a\in\mathbb{N}\vdash 0=0\cdot a\)]
\end{theorem}
\begin{proof}
        \[
	\begin{array}{llll}
            1       &  (1) & a\in\mathbb{N} & \rA \\
                    &  (2) & 0\in\mathbb{N} & \zeroIsNaturalNumber{} \\
                    &  (3) & 0=0\cdot 0 & \rMultI{2} \\
            4       &  (4) & n\in\mathbb{N} & \rA \\
            5       &  (5) & 0=0\cdot n & \rA \\
            5       &  (6) & 0\cdot (n+1)=0\cdot n+0 & \rMultI{5} \\
            5       &  (7) & 0\cdot (n+1)=0+0 & \rIE{5,6} \\
            5       &  (8) & 0=0+0 & \rAddI{2} \\
            5       &  (9) & 0\cdot (n+1)=0 & \rIE{8,7} \\
            5       &  (10) & 0=0\cdot (n+1) & \aIdbImpbIda{9} \\
                    &  (11) & \forall n\in\mathbb{N}(0=0\cdot n) & \rInductionN{3,4,5,10} \\
                    &  (12) & a\in\mathbb{N}\rightarrow (0=0\cdot a) & \rSetUEb{11} \\
            1       &  (13) & 0=0\cdot a & \rRE{1,12} \\
	\end{array}
	\]
\end{proof}

\label{aInNaturalImpaEqualsaMultOne}
\begin{theorem}[\(a\in\mathbb{N}\vdash a=a\cdot 1\) (Neutrales Element)]
\end{theorem}
\begin{proof}
        \[
	\begin{array}{llll}
            1       &  (1) & a\in\mathbb{N} & \rA \\
                    &  (2) & 0\in\mathbb{N} & \zeroIsNaturalNumber{} \\
                    &  (3) & 1\in\mathbb{N} & \oneIsNaturalNumber{} \\
                    &  (4) & 1=0+1 & \aInNaturalImpaEqualsZeroPlusa{3} \\
            1       &  (5) & a\cdot (0+1) = a\cdot 0 + a & \rMultI{1,2} \\
            1       &  (6) & a\cdot 1 = a\cdot 0 + a & \rIE{5,6} \\
            1       &  (7) & 0=a\cdot 0 & \rMultI{1} \\
            1       &  (8) & a\cdot 1=0+a & \rIE{7,6} \\
            1       &  (9) & a=0+a & \aInNaturalImpaEqualsZeroPlusa{1} \\
            1       &  (10) & a\cdot 1=a & \rIE{9,8} \\
            1       &  (11) & a=a\cdot 1 & \aIdbImpbIda{10} \\
	\end{array}
	\]
\end{proof}

\label{aInNaturalImpaEqualsOneMulta}
\begin{theorem}[\(a\in\mathbb{N}\vdash a=1\cdot a\) (Neutrales Element)]
\end{theorem}
\begin{proof}
        \[
	\begin{array}{llll}
            1       &  (1) & a\in\mathbb{N} & \rA \\
                    &  (2) & 1\in\mathbb{N} & \oneIsNaturalNumber{} \\            
                    &  (3) & 0=1\cdot 0 & \rMultI{2} \\
            4       &  (4) & n\in\mathbb{N} & \rA \\
            5       &  (5) & n = 1\cdot n & \rA \\
            5       &  (6) & 1\cdot (n+1)=1\cdot n+1 & \rMultI{2,4} \\
            5       &  (7) & 1\cdot (n+1)=n+1 & \rIE{5,6} \\
            5       &  (8) & n+1=1\cdot (n+1) & \aIdbImpbIda{7} \\
                    &  (9) & \forall n\in\mathbb{N}(n=1\cdot n) & \rInductionN{3,4,5,8} \\
                    &  (10) & a\in\mathbb{N}\rightarrow a=1\cdot a & \rSetUEb{9} \\
            1       &  (11) & a=1\cdot a & \rRE{10} \\                    
	\end{array}
	\]
\end{proof}

\label{aInNaturalImpaMultOneEqualsOneMultaEqualsa}
\begin{theorem}[\(a\in\mathbb{N}\vdash a\cdot 1=1\cdot a = a\) (Neutrales Element)]
\end{theorem}
\begin{proof}
        \[
	\begin{array}{llll}
            1   &  (1) & a\in\mathbb{N} & \rA \\
            1   &  (2) & a=a\cdot 1 & \aInNaturalImpaEqualsaMultOne{1} \\
            1   &  (3) & a=1\cdot a & \aInNaturalImpaEqualsOneMulta{1} \\
            1   &  (4) & a\cdot 1=1\cdot a & \rIE{2,3} \\
            1   &  (5) & 1\cdot a=a & \aIdbImpbIda{3} \\
            1   &  (6) & a\cdot 1=1\cdot a\land 1\cdot a=a & \rAI{4,5} \\
            1   &  (7) & a\cdot 1=1\cdot a = a & \rIIb{6} \\
	\end{array}
	\]
\end{proof}

\label{aInNaturalwbInNaturalImpaMultbInNatural}
\begin{theorem}[\(a\in\mathbb{N},b\in\mathbb{N}\vdash a\cdot b\in\mathbb{N}\)]
\end{theorem}
\begin{proof}
        \[
	\begin{array}{llll}
            1         &  (1) & a\in\mathbb{N} & \rA \\
            2         &  (2) & b\in\mathbb{N} & \rA \\
                      &  (3) & 0\in\mathbb{N} & \zeroIsNaturalNumber{} \\            
            1         &  (4) & 0=a\cdot 0 & \rMultI{1} \\
            1         &  (5) & a\cdot 0\in\mathbb{N} & \rIE{4,3} \\
            6         &  (6) & n\in\mathbb{N} & \rA \\
            7         &  (7) & a\cdot n\in\mathbb{N} & \rA \\
            1,6       &  (8) & a\cdot (n+1) = a\cdot n+a & \rMultI{1,6} \\
            1,6,7     &  (9) & a\cdot n+a\in\mathbb{N} & \aInNaturalwbInNaturalImpaPlusbInNatural{7,1} \\
            1,6,7     &  (10) & a\cdot (n+1)\in\mathbb{N} & \rIE{8,9} \\
            1     &  (11) & \forall n\in\mathbb{N}(a\cdot n\in\mathbb{N} & \rInductionN{5,6,7,10} \\
            1     &  (12) & b\in\mathbb{N}\rightarrow a\cdot b\in\mathbb{N} & \rSetUEb{11} \\
            1,2     &  (13) & a\cdot b\in\mathbb{N} & \rRE{2,12} \\            
	\end{array}
	\]
\end{proof}


\label{aInNaturalwbInNaturalwcInNaturalImpaLpbPluscRpEqualsabPlusac}
\begin{theorem}[\(a\in\mathbb{N},b\in\mathbb{N},c\in\mathbb{N}\vdash a(b+c)=ab+ac\) (Linksdistributivität)]
\end{theorem}
\begin{proof}
        \[
	\begin{array}{llll}
            1           &  (1) & a\in\mathbb{N} & \rA \\
            2           &  (2) & b\in\mathbb{N} & \rA \\
            3           &  (3) & c\in\mathbb{N} & \rA \\
            2           &  (4) & b=b+0 & \rAddI{2} \\
                        &  (5) & ab=ab & \rII{} \\
            2           &  (6) & a(b+0)=ab & \rIE{4,5} \\
            1           &  (7) & 0=a\cdot 0 & \rMultI{1} \\
            1,2         &  (8) & ab\in\mathbb{N} & \aInNaturalwbInNaturalImpaMultbInNatural{1,2} \\
            1,2         &  (9) & ab=ab+0 & \rAddI{8} \\
            1,2         &  (10) & ab=ab+a\cdot 0 & \rIE{7,9} \\
            1,2         &  (11) & a(b+0)=ab+a\cdot 0 & \rIE{4,10} \\
            12          &  (12) & n\in\mathbb{N} & \rA \\
            13          &  (13) & a(b+n)=ab+an & \rA \\
                        &  (14) & 1\in\mathbb{N} & \oneIsNaturalNumber{} \\
            2,12        &  (15) & b+(n+1)=(b+n)+1 & \aInNaturalwbInNaturalwcInNaturalImpaPlusLpbPluscRpEqualsLpaPlusbRpPlusc{1,2,12} \\
                        &  (16) & a(b+(n+1))=a(b+(n+1)) & \rII{} \\
            2,12        &  (17) & a(b+(n+1))=a((b+n)+1) & \rIE{15,16} \\
            2,12        &  (18) & b+n\in\mathbb{N} & \aInNaturalwbInNaturalImpaPlusbInNatural{2,12} \\
            1,2,12      &  (19) & a((b+n)+1)=a(b+n)+a & \rMultI{1,18} \\
            1,2,12,13   &  (20) & a((b+n)+1)=ab+an+a & \rIE{12,19} \\
            1,12        &  (21) & a(n+1)=an+a & \rMultI{1,12} \\
            1,2,12,13   &  (22) & a((b+n)+1)=ab+a(n+1) & \rIE{21,20} \\
            1,2,12,13   &  (24) & a(b+(n+1))=ab+a(n+1) & \rIE{17,22} \\
            1,2         &  (25) & \forall n\in\mathbb{N}(a(b+n)=ab+an) & \rInductionN{11,12,13,24} \\
            1,2         &  (26) & c\in\mathbb{N}\rightarrow a(b+c)=ab+ac & \rSetUEb{25} \\
            1,2,3       &  (27) & a(b+c)=ab+ac & \rRE{3,26} \\
        \end{array}
	\]
\end{proof}

\label{aInNaturalwbInNaturalwcInNaturalImpLpaPlusbRpcEqualsacPlusbc}
\begin{theorem}[\(a\in\mathbb{N},b\in\mathbb{N},c\in\mathbb{N}\vdash (a+b)c=ac+bc\) (Rechtsdistributivität)]
\end{theorem}
\begin{proof}
        \[
	\begin{array}{lll p{5cm}}
            1           &  (1) & a\in\mathbb{N} & \rA \\
            2           &  (2) & b\in\mathbb{N} & \rA \\
            3           &  (3) & c\in\mathbb{N} & \rA \\
            1,2         &  (4) & a+b\in\mathbb{N} & \aInNaturalwbInNaturalImpaPlusbInNatural{1,2} \\
            1,2         &  (5) & 0=(a+b)\cdot 0 & \rMultI{4} \\
            1           &  (6) & 0=a\cdot 0 & \rMultI{1} \\
            2           &  (7) & 0=b\cdot 0 & \rMultI{2} \\
                        &  (8) & 0\in\mathbb{N} & \zeroIsNaturalNumber{} \\
                        &  (9) & 0=0+0 & \rAddI{8} \\
            1           &  (10) & 0=a\cdot 0+0 & \rIE{6,9} \\
            1,2         &  (11) & 0=a\cdot 0+b\cdot 0 & \rIE{7,9} \\
            1,2         &  (12) & (a+b)\cdot 0=a\cdot 0+b\cdot 0 & \rIE{5,11} \\
            13          &  (13) & n\in\mathbb{N} & \rA \\
            14          &  (14) & (a+b)n=an+bn & \rA \\
            1,2,13      &  (15) & (a+b)(n+1)=(a+b)n+(a+b) & \rMultI{4,13} \\
            1,2,13,14   &  (16) & (a+b)(n+1)=(an+bn)+(a+b) & \rIE{14,15} \\            
            1,13        &  (17) & an\in\mathbb{N} & \aInNaturalwbInNaturalImpaMultbInNatural{1,13} \\
            2,13        &  (18) & bn\in\mathbb{N} & \aInNaturalwbInNaturalImpaMultbInNatural{2,13} \\ 1,2,13      &  (19) & (an+bn)+(a+b)=(an+a)+(bn+b) & \aInNaturalwbInNaturalwcInNaturalwdInNaturalImpLpaPlusbRpPlusLpcPlusdRpEqualsLpaPluscRpPlusLpbPlusdRp{17,18,1,2} \\  
            1,13        &  (20) & a(n+1)=(an+a) & \rMultI{1,13}  \\  
            2,13        &  (21) & b(n+1)=(bn+b) & \rMultI{2,13}  \\  
            1,2,13      &  (22) & (an+bn)+(a+b) = a(n+1)+(bn+b) & \rIE{20,19}  \\  
            1,2,13      &  (23) & (an+bn)+(a+b) = a(n+1)+b(n+1) & \rIE{21,22}  \\  
            1,2,13,14   &  (24) & (a+b)(n+1) = a(n+1)+b(n+1) & \rIE{23,16}  \\
            1,2         &  (25) & \forall n\in\mathbb{N}((a+b)n = an+bn) & \rInductionN{12,13,14,24}\\
            1,2         &  (26) & c\in\mathbb{N}\rightarrow (a+b)c = ac+bc & \rSetUEb{25}\\
            1,2,3       &  (27) & (a+b)c = ac+bc & \rRE{3,26}\\
        \end{array}
	\]
\end{proof}

\label{aInNaturalwbInNaturalImpLpaPlusOneRpbEqualsabPlusb}
\begin{theorem}[\(a\in\mathbb{N},b\in\mathbb{N}\vdash (a+1)b=ab+b\)]
\end{theorem}
\begin{proof}
        \[
	\begin{array}{lll p{5cm}}
            1   &  (1) & a\in\mathbb{N} & \rA \\
            2   &  (2) & b\in\mathbb{N} & \rA \\
            2   &  (3) & 1\in\mathbb{N} & \oneIsNaturalNumber{} \\
            2   &  (4) & b=1\cdot b & \aInNaturalImpaEqualsOneMulta{2}  \\
            1,2 &  (5) & (a+1)b=ab+1\cdot b & \aInNaturalwbInNaturalwcInNaturalImpLpaPlusbRpcEqualsacPlusbc{1,2,4}  \\
            1,2 &  (6) & (a+1)b=ab+b & \rIE{4,5}  \\
        \end{array}
	\]
\end{proof}

\label{aInNaturalwbInNaturalwcInNaturalImpaLpbcRpEqualsLpabRpc}
\begin{theorem}[\(a\in\mathbb{N},b\in\mathbb{N},c\in\mathbb{N}\vdash a(bc)=(ab)c\) (Assoziativität)]
\end{theorem}
\begin{proof}
        \[
	\begin{array}{llll}
            1           &  (1) & a\in\mathbb{N} & \rA \\
            2           &  (2) & b\in\mathbb{N} & \rA \\
            3           &  (3) & c\in\mathbb{N} & \rA \\
            2           &  (4) & 0=b\cdot 0 & \rMultI{2} \\
            1           &  (5) & 0=a\cdot 0 & \rMultI{1} \\
                        &  (6) & a(b\cdot 0)=a(b\cdot 0) & \rII{} \\
            2           &  (7) & a(b\cdot 0)=a\cdot 0 & \rIE{4,6} \\
            1,2         &  (8) & a(b\cdot 0)= 0 & \rIE{5,7} \\
            1,2         &  (9) & ab\in\mathbb{N} & \aInNaturalwbInNaturalImpaMultbInNatural{1,2} \\
            1,2         &  (10) & 0=(ab)\cdot 0 & \rMultI{9} \\
            1,2         &  (11) & a(b\cdot 0)=(ab)\cdot 0 & \rIE{8,10} \\
            12          &  (12) & n\in\mathbb{N} & \rA \\
            13          &  (13) & a(bn)=(ab)n & \rA \\
            1,2         &  (14) & (ab)(n+1)=(ab)n+ab & \rMultI{9,12} \\
            1,2,13      &  (15) & (ab)(n+1)=a(bn)+ab & \rIE{13,14} \\
            2,12        &  (16) & bn\in\mathbb{N} & \aInNaturalwbInNaturalImpaMultbInNatural{2,12}\\
            1,2,12,13   &  (17) & a(bn)+ab=a(bn+b) & \aInNaturalwbInNaturalwcInNaturalImpaLpbPluscRpEqualsabPlusac{1,16,2} \\
            2,12        &  (18) & b(n+1)=bn+b & \rMultI{2,12} \\
            1,2,12,13   &  (19) & a(bn)+ab=a(b(n+1)) & \rIE{18,17} \\
            1,2,12,13   &  (20) & (ab)(n+1)=a(b(n+1)) & \rIE{19,15} \\
            1,2         &  (21) & \forall n\in\mathbb{N}((ab)n=a(bn) & \rInductionN{11,12,13,20} \\
            1,2         &  (22) & c\in\mathbb{N}\rightarrow(ab)c=a(bc) & \rSetUEb{21} \\
            1,2,3       &  (23) & (ab)c=a(bc) & \rRE{3,22} \\
    \end{array}
	\]
\end{proof}

\label{aInNaturalwbInNaturalImpabEqualsba}
\begin{theorem}[\(a\in\mathbb{N},b\in\mathbb{N}\vdash ab=ba\) (Kommutativität)]
\end{theorem}
\begin{proof}
        \[
	\begin{array}{llll}
            1       &  (1) & a\in\mathbb{N} & \rA \\
            2       &  (2) & b\in\mathbb{N} & \rA \\
            1       &  (3) & 0=a\cdot 0 & \rMultI{1} \\
            1       &  (4) & 0=0\cdot a & \aInNaturalImpZeroEqualsZeroMulta{1} \\
            1       &  (5) & a\cdot 0=0\cdot a & \rIE{3,4} \\
            6       &  (6) & n\in\mathbb{N} & \rA \\
            7       &  (7) & an=na & \rA \\
            1,6     &  (8) & a(n+1)=an+a & \rMultI{1,6} \\
            1,6,7   &  (9) & a(n+1)=na+a & \rIE{7,8} \\
            1,6     &  (10) & (n+1)a=na+a &  \aInNaturalwbInNaturalImpLpaPlusOneRpbEqualsabPlusb{6,1} \\
            1,6,7   &  (11) & a(n+1)=(n+1)a & \rIE{10,9} \\
            1       &  (12) & \forall n\in\mathbb{N}(an=na) & \rInductionN{5,6,7,11} \\
            1       &  (13) & b\in\mathbb{N}\rightarrow an=na & \rSetUEb{12} \\
            1,2     &  (14) & ab=ba & \rRE{2,13} \\
    \end{array}
	\]
\end{proof}

\label{ImpLpNaturalwMultwOneRpInMonoid}
\begin{theorem}[\(\vdash (\mathbb{N},\cdot,1) \text{ ist ein Monoid}\)]
\end{theorem}
\begin{proof}
        \[
	\begin{array}{llll}
                &  (1) & \cdot:\mathbb{N}\times\mathbb{N}\rightarrow\mathbb{N} & \rMultI{} \\
                &  (2) & \forall a,b,c\in\mathbb{N}(a(bc)=(ab)c) & \aInNaturalwbInNaturalwcInNaturalImpaLpbcRpEqualsLpabRpc{} \\
                &  (3) & \forall a\in\mathbb{N}(a\cdot 1=1\cdot a=a) & \aInNaturalImpaMultOneEqualsOneMultaEqualsa{} \\
                &  (4) & (\mathbb{N},+,0) \text{ ist ein Monoid.} & \rMonoidI{1,2,3} \\
	\end{array}
	\]
\end{proof}

\label{ImpLpNaturalwMultwOneRpInAbelMonoid}
\begin{theorem}[\(\vdash (\mathbb{N},\cdot,1) \text{ ist ein abelscher Monoid}\)]
\end{theorem}
\begin{proof}
        \[
	\begin{array}{llll}
                &  (1) & (\mathbb{N},\cdot,1) \text{ ist ein Monoid.} & \ImpLpNaturalwMultwOneRpInMonoid{} \\
                &  (2) & \forall a\in\mathbb{N}\forall b\in\mathbb{N}(ab=ba) & \aInNaturalwbInNaturalImpabEqualsba{} \\
                &  (3) & (\mathbb{N},\cdot,1) \text{ ist ein abelscher Monoid.} & \rAbelianMonoidI{1,2} \\
	\end{array}
	\]
\end{proof}

\label{ImpLpNaturalwMultwOneRpInAbelSemiRing}
\begin{theorem}[\(\vdash (\mathbb{N},+,\cdot) \text{ ist ein abelscher Halbring}\)]
\end{theorem}
\begin{proof}
        \[
	\begin{array}{llll}
                &  (1) & (\mathbb{N},\cdot,1) \text{ ist ein abelscher Monoid.} & \ImpLpNaturalwMultwOneRpInAbelMonoid{} \\
                &  (2) & (\mathbb{N},+,0) \text{ ist ein abelscher Monoid.} & \ImpLpNaturalwPluswZeroRpInAbelMonoid{} \\
                &  (3) & \forall a,b,c\in\mathbb{N}((a+b)c=ac+bc) & \aInNaturalwbInNaturalwcInNaturalImpLpaPlusbRpcEqualsacPlusbc{} \\
                &  (4) & \forall a,b,c\in\mathbb{N}(a(b+c)=ab+ac) & \aInNaturalwbInNaturalwcInNaturalImpaLpbPluscRpEqualsabPlusac{} \\
                &  (5) & (\mathbb{N},+,\cdot) \text{ ist ein abelscher Halbring.} & \rAbelianSemiringI{1,2,4,3} \\
	\end{array}
	\]
\end{proof}

\label{aInNaturalImpZeroEqualsaMultZero}
\begin{theorem}[\(a\in\mathbb{N}\vdash 0=a\cdot 0\)]
\end{theorem}
\begin{proof}
        \[
	\begin{array}{llclll}
                &  (1) & 0&=&0\cdot a & \aInNaturalImpZeroEqualsZeroMulta{} \\
                &  (2) &  &=&a\cdot 0 & \rCommutativeMonoid{} \\
	\end{array}
	\]
\end{proof}

\label{aInNaturalwaEqualsZeroImpabEqualsZero}
\begin{theorem}[\(a\in\mathbb{N},a=0\vdash ab=0\)]
\end{theorem}
\begin{proof}
    Sei \(a\in\mathbb{N}\), dann gilt:
        \[
	\begin{array}{llll}
              1  &  (1) & a=0 & \rA \\
                 &  (2) & 0\cdot b=0 & \aInNaturalImpZeroEqualsZeroMulta{} \\
                 &  (3) & ab=0 & \rIE{1,2} \\
	\end{array}
	\]
\end{proof}

\label{aInNaturalwabNotEqualsZeroImpaNotEqualsZero}
\begin{theorem}[\(a\in\mathbb{N},ab\neq 0\vdash a\neq 0\)]
\end{theorem}
\begin{proof}
    Sei \(a\in\mathbb{N}\), dann gilt:
        \[
	\begin{array}{llll}
              1  &  (1) & ab\neq 0 & \rA \\
                 &  (2) & a=0\rightarrow ab=0 & \aInNaturalwaEqualsZeroImpabEqualsZero{} \\
              1  &  (3) & a\neq 0 & \PToQwnQImpnP{2,1} \\
	\end{array}
	\]
\end{proof}

\label{aInNaturalwbEqualsZeroImpabEqualsZero}
\begin{theorem}[\(a\in\mathbb{N},b=0\vdash ab=0\)]
\end{theorem}
\begin{proof}
    Sei \(a\in\mathbb{N}\), dann gilt:
        \[
	\begin{array}{llll}
              1  &  (1) & b=0 & \rA \\
                 &  (2) & a\cdot 0=0 & \aInNaturalImpZeroEqualsaMultZero{} \\
              1  &  (3) & ab=0 & \rIE{1,2} \\
	\end{array}
	\]
\end{proof}

\label{aInNaturalwabNotEqualsZeroImpbNotEqualsZero}
\begin{theorem}[\(a\in\mathbb{N},ab\neq 0\vdash b\neq 0\)]
\end{theorem}
\begin{proof}
    Sei \(a\in\mathbb{N}\), dann gilt:
        \[
	\begin{array}{llll}
              1  &  (1) & ab\neq 0 & \rA \\
                 &  (2) & b=0\rightarrow ab=0 & \aInNaturalwbEqualsZeroImpabEqualsZero{} \\
              1  &  (3) & b\neq 0 & \PToQwnQImpnP{2,1} \\
	\end{array}
	\]
\end{proof}


\label{aInNaturalwbInNaturalwcInNaturalLpaEqualsbImpacEqualsbcRp}
\begin{theorem}[\(a,b,c\in\mathbb{N},a=b\vdash ac=bc\)]
\end{theorem}
\begin{proof}
Seien \(a,b,c\in\mathbb{N}\), dann gilt:
        \[
	\begin{array}{llll}
            1       &  (1)  & a=b & \rA \\
                    &  (2)  & ac=ac & \rII{} \\
            1       &  (3)  & ac=bc & \rIE{1,2} \\       
	\end{array}
        \]
\end{proof}

\label{awbwcInNaturalwacNotEqualsbcImpaNotEqualsb}
\begin{theorem}[\(a,b,c\in\mathbb{N},ac\neq bc\vdash a\neq b\)]
\end{theorem}
\begin{proof}
Seien \(a,b,c\in\mathbb{N}\), dann gilt:
        \[
	\begin{array}{llll}
            1       &  (1)  & ac\neq bc & \rA \\
                    &  (2)  & a=b\rightarrow ac=bc & \aInNaturalwbInNaturalwcInNaturalLpaEqualsbImpacEqualsbcRp{} \\
            1       &  (3)  & a\neq b & \PToQwnQImpnP{2,1} \\       
	\end{array}
        \]
\end{proof}

\label{awbwcInNaturalwaEqualsbImpcaEqualscb}
\begin{theorem}[\(a,b,c\in\mathbb{N},a=b\vdash ca=cb\)]
\end{theorem}
\begin{proof}
Seien \(a,b,c\in\mathbb{N}\), dann gilt:
        \[
	\begin{array}{llll}
            1       &  (1)  & a=b & \rA \\
                    &  (2)  & ca=ca & \rII{} \\
            1       &  (3)  & ca=cb & \rIE{1,2} \\       
	\end{array}
        \]
\end{proof}

\label{awbwcInNaturalwcaNotEqualscbImpaNotEqualsb}
\begin{theorem}[\(a,b,c\in\mathbb{N},ca\neq cb\vdash a\neq b\)]
\end{theorem}
\begin{proof}
Seien \(a,b,c\in\mathbb{N}\), dann gilt:
        \[
	\begin{array}{llll}
            1       &  (1)  & ca\neq cb & \rA \\
                    &  (2)  & a=b\rightarrow ca=cb & \awbwcInNaturalwaEqualsbImpcaEqualscb{} \\
            1       &  (3)  & a\neq b & \PToQwnQImpnP{2,1} \\       
	\end{array}
        \]
\end{proof}


\section{Eigenschaften der Multiplikation in Bezug auf Relationen}

\label{awbwcInNaturalImpaLeqbImpacLeqbc}
\begin{theorem}[\(a,b,c\in\mathbb{N},a\leq b\vdash ac\leq bc\)]
\end{theorem}
\begin{proof}
Seien \(a,b,c\in\mathbb{N}\). \(\ImpLpNaturalwMultwOneRpInAbelSemiRing{}\) und daher gilt:
       \[
	\begin{array}{lllcll}
            1       &  (1)  & \multicolumn{3}{l}{a\leq b} & \rA \\
            1       &  (2)  & \multicolumn{3}{l}{a+(b-a)=b} & \minusI{1} \\
            1       &  (3)  & ac+(b-a)c&=&(a+(b-a))c & \rRightDistributiveAbelianSemigroup{} \\       
            1       &  (4)  &  &=&bc & \aInNaturalwbInNaturalwcInNaturalLpaEqualsbImpacEqualsbcRp{2} \\ 
            1       &  (5)  & \multicolumn{3}{l}{ac+(b-a)c=bc} & \rTransitivityOrdRI{3,4} \\ 
            1       &  (6)  & \multicolumn{3}{l}{ac\leq bc} & \rLeqNI{5} \\ 
	\end{array}
        \]

\end{proof}

\label{awbwcInNaturalImpaLeqbImpcaLeqcb}
\begin{theorem}[\(a,b,c\in\mathbb{N}\vdash a\leq b\vdash ca\leq cb\)]
\end{theorem}
\begin{proof}
Seien \(a,b,c\in\mathbb{N}\). \(\ImpLpNaturalwMultwOneRpInAbelSemiRing{}\) und daher gilt:
       \[
	\begin{array}{lllcll}
            1       &  (1)  & \multicolumn{3}{l}{a\leq b} & \rA \\
                    &  (2)  & ca&=&ac & \rCommutativeMonoid{} \\       
            1       &  (3)  &  &\leq &bc & \awbwcInNaturalImpaLeqbImpacLeqbc{1} \\ 
                    &  (4)  &  &\leq &cb & \rCommutativeMonoid{} \\ 
                    &  (5)  & \multicolumn{3}{l}{ca\leq cb} & \rTransitivityEqRI{2,5} \\ 
	\end{array}
        \]
\end{proof}

\label{awbInNaturalwbNotEqualsZeroImpaLeqab}
\begin{theorem}[\(a,b\in\mathbb{N}, b\neq 0\vdash a\leq ab\)]
\end{theorem}
\begin{proof}
        Seien \(a,b\in\mathbb{N}\). \(\LeqIsTotalOrderOnNaturalNumbers{}\) und daher gilt
        \[
	\begin{array}{llclll}
        1   &  (1)  & \multicolumn{3}{l}{b\neq 0} & \rA \\
        1   &  (2)  & \multicolumn{3}{l}{1\leq b} & \aInNaturalwaNotEqualsZerowOneLeqa{1} \\
            &  (3)  & a&=&1\cdot a & \rNeutralElementMonoid{} \\
        1   &  (4)  & 1\cdot a&\leq & ba & \awbwcInNaturalImpaLeqbImpacLeqbc{2} \\
        1   &  (5)  & a&\leq & ba & \rTransitivityOrdRI{2,4} \\

	\end{array}
	\]
\end{proof}

\label{awbInNaturalwbNotEqualsZeroImpaLeqba}
\begin{theorem}[\(a,b\in\mathbb{N}, b\neq 0\vdash a\leq ba\)]
\end{theorem}
\begin{proof}
        Seien \(a,b\in\mathbb{N}\). \(\LeqIsTotalOrderOnNaturalNumbers{}\) und daher gilt
        \[
	\begin{array}{llclll}
        1   &  (1)  & \multicolumn{3}{l}{b\neq 0} & \rA \\
        1   &  (2)  & a&\leq &ab & \awbInNaturalwbNotEqualsZeroImpaLeqab{1} \\
            &  (3)  & &=& ba & \rCommutativeMonoid{} \\
        1   &  (4)  & a&\leq & ba & \rTransitivityOrdRI{2,3} \\

	\end{array}
	\]
\end{proof}


\section{Eigenschaften der Multiplikation und Differenz}

\label{awbwcInNaturalwcLeqbImpLpbMinuscRpaEqualsbaMinusca}
\begin{theorem}[\(a,b,c\in\mathbb{N},c\leq b\vdash (b-c)a=ba-ca\)]
\end{theorem}
\begin{proof}
Seien \(a,b,c\in\mathbb{N}\). \(\ImpLpNaturalwMultwOneRpInAbelSemiRing{}\) und daher gilt:
       \[
	\begin{array}{lllcll}
            1       &  (1)  & \multicolumn{3}{l}{c\leq b} & \rA \\
            1       &  (2)  & \multicolumn{3}{l}{ca\leq ba} & \awbwcInNaturalImpaLeqbImpacLeqbc{1} \\
            1       &  (3)  & \multicolumn{3}{l}{c+(b-c)=b} & \minusI{1} \\
            1       &  (4)  & \multicolumn{3}{l}{ca+(ba-ca)=ba} & \minusI{2} \\
                    &  (5)  & ca+(b-c)a&=&(c+(b-c))a & \rRightDistributiveAbelianSemigroup{} \\
            1       &  (6)  &          &=& ba & \rIE{3,5} \\
            1       &  (7)  &          &=& ca+(ba-ca) & \rIE{4,6} \\
            1       &  (8)  & \multicolumn{3}{l}{ca+(b-c)a=ca+(ba-ca)} & \rTransitivityEqRI{5,7} \\
            1       &  (9)  & \multicolumn{3}{l}{(b-c)a=ba-ca} & \aEqualsbEqvcPlusaEqualscPlusb{8} \\
	\end{array}
        \]
\end{proof}


\label{awbwcInNaturalwcLeqbImpaLpbMinuscRpEqualsabMinusac}
\begin{theorem}[\(a,b,c\in\mathbb{N},c\leq b\vdash a(b-c)=ab-ac\)]
\end{theorem}
\begin{proof}
Seien \(a,b,c\in\mathbb{N}\). \(\ImpLpNaturalwMultwOneRpInAbelSemiRing{}\) und daher gilt:
       \[
	\begin{array}{lllcll}
            1       &  (1)  & \multicolumn{3}{l}{c\leq b} & \rA \\
            1       &  (2)  & \multicolumn{3}{l}{ac\leq ab} & \awbwcInNaturalImpaLeqbImpacLeqbc{1} \\
            1       &  (3)  & \multicolumn{3}{l}{c+(b-c)=b} & \minusI{1} \\
            1       &  (4)  & \multicolumn{3}{l}{ac+(ab-ac)=ab} & \minusI{2} \\
                    &  (5)  & ac+a(b-c)&=&a(c+(b-c)) & \rRightDistributiveAbelianSemigroup{} \\
            1       &  (6)  &          &=& ab & \rIE{3,5} \\
            1       &  (7)  &          &=& ac+(ab-ac) & \rIE{4,6} \\
            1       &  (8)  & \multicolumn{3}{l}{ac+a(b-c)=ac+(ab-ac)} & \rTransitivityEqRI{5,7} \\
            1       &  (9)  & \multicolumn{3}{l}{a(b-c)=ab-ac} & \aEqualsbEqvcPlusaEqualscPlusb{8} \\
	\end{array}
        \]
\end{proof}

\section{Weitere Eigenschaften der Multiplikation}

\label{awbInNaturalwaNotEqualsZerowabEqualsZeroImpbEqualsZero}
\begin{theorem}[\(a,b\in\mathbb{N},a\neq 0,ab=0\vdash b=0\)]
\end{theorem}
\begin{proof}
        Seien \(a,b\in\mathbb{N}\). \(\ImpLpNaturalwMultwOneRpInAbelSemiRing{}\)
        \[
	\begin{array}{llclll}
            1       &  (1)  & \multicolumn{3}{l}{a\neq 0} & \rA \\
            1       &  (2)  & \multicolumn{3}{l}{a=(a-1)+1} & \rPredecessorI{1} \\
            3       &  (3)  & 0&=&ab & \rA \\
            1,3     &  (4)  &  &=&((a-1)+1)b & \rIE{2} \\
            1,3     &  (5)  &  &=&(a-1)b+1\cdot b & \rRightDistributiveAbelianSemigroup{} \\
            1,3     &  (6)  &  &=&(a-1)b+b & \rNeutralElementMonoid{} \\
            1,3     &  (7)  & \multicolumn{3}{l}{(a-1)b+b=0} & \rTransitivityOrdRI{3,6} \\
            1,3     &  (8)  & \multicolumn{3}{l}{(a-1)b=0\land b=0} & \aInNaturalwbInNaturalwaPlusbEqualsZeroImpaEqualsZeroAndbEqualsZero{7}\\  
            1,3     &  (9)  & \multicolumn{3}{l}{b=0} & \rAEb{8}\\  
	\end{array}
	\]
\end{proof}

\label{awbInNaturalwbNotEqualsZerowabEqualsZeroImpaEqualsZero}
\begin{theorem}[\(a,b\in\mathbb{N},b\neq 0,ab=0\vdash a=0\)]
\end{theorem}
\begin{proof}
        Seien \(a,b\in\mathbb{N}\). \(\ImpLpNaturalwMultwOneRpInAbelSemiRing{}\)
        \[
	\begin{array}{llclll}
            1       &  (1)  & \multicolumn{3}{l}{b\neq 0} & \rA \\
                    &  (2)  & ba&=&ab & \rCommutativeMonoid{} \\
            3       &  (3)  &   &=&0 & \rA \\
            3       &  (4)  & \multicolumn{3}{l}{ba=0} & \rTransitivityEqRI{2,3} \\
            1,3     &  (5)  & \multicolumn{3}{l}{a\neq 0} & \awbInNaturalwaNotEqualsZerowabEqualsZeroImpbEqualsZero{1,4} \\            
	\end{array}
	\]
\end{proof}

\label{awbInNaturalwaNotEqualsZerowbNotEqualsZeroImpabNotEqualsZero}
\begin{theorem}[\(a,b\in\mathbb{N},a\neq 0,b\neq 0\vdash ab\neq 0\)]
\end{theorem}
\begin{proof}
        Seien \(a,b\in\mathbb{N}\). 
        \[
	\begin{array}{llll}
            1       &  (1)  & a\neq 0 & \rA \\
            2       &  (2)  & b\neq 0 & \rA \\
            3       &  (3)  & ab=0 & \rA \\
            2,3     &  (4)  & a=0 & \awbInNaturalwbNotEqualsZerowabEqualsZeroImpaEqualsZero{2,3} \\
            1,2,3   &  (5)  & \bot & \rBI{1,4} \\
            1,2   &  (6)  & ab\neq 0 & \rCI{3,5} \\
	\end{array}
	\]
\end{proof}

\label{awbwcInNaturalwcNotEqualsZerowacEqualsbcImpaEqualsb}
\begin{theorem}[\(a,b,c\in\mathbb{N},c\neq 0,ac=bc\vdash a=b\)]
\end{theorem}
\begin{proof}
        Seien \(a,b,c\in\mathbb{N}\). \(\LeqIsTotalOrderOnNaturalNumbers{}\) und daher gilt
        \[
	\begin{array}{llclll}
            1       &  (1)  & \multicolumn{3}{l}{ac=bc} & \rA \\
            2       &  (2)  & \multicolumn{3}{l}{c\neq 0} & \rA \\
                    &  (3)  & \multicolumn{3}{l}{a\leq b\lor b\leq a} & \rTotalityOrdRI{} \\
            4       &  (4)  & \multicolumn{3}{l}{a\leq b} & \rA \\
            1       &  (5)  & 0&=&bc-ac & \awbInNaturalLpbEqualsaEqvaMinusbEqualsZeroRp{1} \\
            1,4     &  (6)  &  &=&(b-a)c & \awbwcInNaturalwcLeqbImpLpbMinuscRpaEqualsbaMinusca{4} \\
            1,4     &  (7)  & \multicolumn{3}{l}{(b-a)c=0} & \rTransitivityEqRI{5,6} \\        
            1,2,4   &  (8)  & \multicolumn{3}{l}{b-a=0} & \awbInNaturalwbNotEqualsZerowabEqualsZeroImpaEqualsZero{2,7} \\
            1,2,4   &  (9)  & \multicolumn{3}{l}{a=b} & \awbInNaturalLpbEqualsaEqvaMinusbEqualsZeroRp{8} \\
            10      &  (10)  & \multicolumn{3}{l}{b\leq a} & \rA \\
            1       &  (11)  & 0&=&ac-bc & \awbInNaturalLpaEqualsbEqvaMinusbEqualsZeroRp{1} \\
            1,10    &  (12)  &  &=&(a-b)c & \awbwcInNaturalwcLeqbImpLpbMinuscRpaEqualsbaMinusca{10} \\
            1,10     &  (13)  & \multicolumn{3}{l}{(a-b)c=0} & \rTransitivityEqRI{11,12} \\  
            1,2,10   &  (14)  & \multicolumn{3}{l}{a-b=0} & \awbInNaturalwbNotEqualsZerowabEqualsZeroImpaEqualsZero{2,13} \\  
            1,2,10     &  (15)  & \multicolumn{3}{l}{a=b} & \awbInNaturalLpaEqualsbEqvaMinusbEqualsZeroRp{14} \\  
            1,2        &  (16)  & \multicolumn{3}{l}{a=b} & \rOE{3,4,9,10,15} \\  
	\end{array}
	\]
\end{proof}

\label{awbwcInNaturalwcNotEqualsZerowaNotEqualsbImpacNotEqualsbc}
\begin{theorem}[\(a,b,c\in\mathbb{N},c\neq 0,a\neq b\vdash ac\neq bc\)]
\end{theorem}
\begin{proof}
        Seien \(a,b,c\in\mathbb{N}\).
        \[
	\begin{array}{llclll}
            1       &  (1)  & \multicolumn{3}{l}{a\neq b} & \rA \\
            2       &  (2)  & \multicolumn{3}{l}{c\neq 0} & \rA \\
            2       &  (3)  & \multicolumn{3}{l}{ac=bc\rightarrow a=b} & \awbwcInNaturalwcNotEqualsZerowacEqualsbcImpaEqualsb{2} \\
            1,2       &  (3)  & \multicolumn{3}{l}{ac\neq bc} & \PToQwnQImpnP{3,1} \\  
	\end{array}
	\]
\end{proof}

\label{awbwcInNaturalwcNotEqualsZerowcaEqualscbImpaEqualsb}
\begin{theorem}[\(a,b,c\in\mathbb{N},c\neq 0,ca=cb\vdash a=b\)]
\end{theorem}
\begin{proof}
        Seien \(a,b,c\in\mathbb{N}\). \(\LeqIsTotalOrderOnNaturalNumbers{}\) und daher gilt
        \[
	\begin{array}{llclll}
                    &  (1)  & ac&=&ca & \rCommutativeMonoid{} \\
                2   &  (2)  & &=&cb & \rA \\
                2   &  (3)  & &=&bc & \rCommutativeMonoid{} \\
                2   &  (4)  &\multicolumn{3}{l}{ac=bc} & \rTransitivityEqRI{1,3} \\
                5   &  (5)  &\multicolumn{3}{l}{c\neq 0} & \rA \\
                2,5 &  (6)  &\multicolumn{3}{l}{a=b} & \awbwcInNaturalwcNotEqualsZerowacEqualsbcImpaEqualsb{4,5} \\
	\end{array}
	\]
\end{proof}

\label{awbwcInNaturalwcNotEqualsZerowaNotEqualsbImpcaNotEqualscb}
\begin{theorem}[\(a,b,c\in\mathbb{N},c\neq 0,a\neq b\vdash ca\neq cb\)]
\end{theorem}
\begin{proof}
        Seien \(a,b,c\in\mathbb{N}\).
        \[
	\begin{array}{llclll}
            1       &  (1)  & \multicolumn{3}{l}{a\neq b} & \rA \\
            2       &  (2)  & \multicolumn{3}{l}{c\neq 0} & \rA \\
            2       &  (3)  & \multicolumn{3}{l}{ca=cb\rightarrow a=b} & \awbwcInNaturalwcNotEqualsZerowcaEqualscbImpaEqualsb{2} \\
            1,2     &  (4)  & \multicolumn{3}{l}{ca\neq cb} & \PToQwnQImpnP{3,1} \\  
	\end{array}
	\]
\end{proof}

\subsection{Ordnungsrelationen und Multiplikation in den natürlichen Zahlen}

\label{awbwcInNaturalwcNotEqualsZerowacLeqbcImpaLeqb}
\begin{theorem}[\(a,b,c\in\mathbb{N},c\neq 0, ac\leq bc\vdash a\leq b\)]
\end{theorem}
\begin{proof}
Seien \(a,b,c\in\mathbb{N}\).
       \[
	\begin{array}{lllcll}
            1       &  (1)  & \multicolumn{3}{l}{ac\leq bc} & \rA \\
            2       &  (2)  & \multicolumn{3}{l}{c\neq 0} & \rA \\
            3       &  (3)  & \multicolumn{3}{l}{b<a} & \rA \\       
            3       &  (4)  & \multicolumn{3}{l}{b\leq a} & \InducedStrictOrderE{3} \\   
            3       &  (5)  & \multicolumn{3}{l}{b\neq a} & \InducedStrictOrderE{3} \\   
            3       &  (6)  & \multicolumn{3}{l}{bc\leq ac} & \awbwcInNaturalImpaLeqbImpacLeqbc{4} \\ 
            1,3       &  (7)  & \multicolumn{3}{l}{bc=ac} & \rAntisymmetryOrdRI{6,1} \\  
            1,2,3     &  (8)  & \multicolumn{3}{l}{b=a} & \awbwcInNaturalwcNotEqualsZerowacEqualsbcImpaEqualsb{2,7} \\  
            1,2,3     &  (9)  & \multicolumn{3}{l}{\bot} & \rBI{8,5} \\  
            1,2       &  (10)  & \multicolumn{3}{l}{\neg(b<a)} & \rBI{8,5} \\  
            1,2       &  (11)  & \multicolumn{3}{l}{a\leq b)} & \nLpaLneqbRpEqvbLeqa{10} \\   
	\end{array}
        \]
\end{proof}


\label{awbwcInNaturalwcNotEqualsZerowcaLeqcbImpaLeqb}
\begin{theorem}[\(a,b,c\in\mathbb{N},c\neq 0, ca\leq cb\vdash a\leq b\)]
\end{theorem}
\begin{proof}
Seien \(a,b,c\in\mathbb{N}\). \(\ImpLpNaturalwMultwOneRpInAbelSemiRing{}\) und daher gilt:
       \[
	\begin{array}{lllcll}
            1       &  (1)  & \multicolumn{3}{l}{c\neq 0} & \rA \\
                    &  (2)  & ac&=&ca & \rCommutativeMonoid{} \\
            3       &  (3)  & &\leq &cb & \rA \\
                    &  (4)  & &= &bc & \rCommutativeMonoid{} \\
            3       &  (5)  & ac&=&bc & \rTransitivityOrdRI{2,4} \\     
            1,3       &  (6)  & a&\leq &b & \awbwcInNaturalwcNotEqualsZerowacLeqbcImpaLeqb{1,5} \\ 
	\end{array}
        \]
\end{proof}

\label{awbwcInNaturalwcNotEqualsZerowacLneqbcEqvaLneqb}
\begin{theorem}[\(a,b,c\in\mathbb{N},c\neq 0, ac<bc\dashv\vdash a<b\)]
Seien \(a,b,c\in\mathbb{N}\) und \(c\neq 0\), dann gilt:
\[ac<bc\dashv\vdash a<b\]
\end{theorem}
\begin{proof}
Seien \(a,b,c\in\mathbb{N}\). 
       \[
	\begin{array}{lllcll}
            1       &  (1)  & \multicolumn{3}{l}{c\neq 0} & \rA \\
            2       &  (2)  & \multicolumn{3}{l}{ac<bc} & \rA \\
            2       &  (3)  & \multicolumn{3}{l}{ac\leq bc} & \InducedStrictOrderE{2} \\
            2       &  (4)  & \multicolumn{3}{l}{ac\neq bc} & \InducedStrictOrderE{2} \\
            1,2     &  (5)  & \multicolumn{3}{l}{a\leq b} & \awbwcInNaturalwcNotEqualsZerowacLeqbcImpaLeqb{1,3} \\
            2     &  (6)  & \multicolumn{3}{l}{a\neq b} & \awbwcInNaturalwacNotEqualsbcImpaNotEqualsb{4} \\
            1,2     &  (7)  & \multicolumn{3}{l}{a<b} & \InducedStrictOrderI{5,6} \\

	\end{array}
        \]
\(\dashv\):
       \[
	\begin{array}{lllcll}
            1       &  (1)  & \multicolumn{3}{l}{c\neq 0} & \rA \\
            2       &  (2)  & \multicolumn{3}{l}{a<b} & \rA \\
            2       &  (3)  & \multicolumn{3}{l}{a\leq b} & \InducedStrictOrderE{2} \\
            2       &  (4)  & \multicolumn{3}{l}{ac\leq bc} & \awbwcInNaturalImpaLeqbImpacLeqbc{3} \\
            2       &  (5)  & \multicolumn{3}{l}{a\neq b} & \InducedStrictOrderE{2} \\
            1,2       &  (6)  & \multicolumn{3}{l}{ac\neq bc} & \awbwcInNaturalwcNotEqualsZerowaNotEqualsbImpacNotEqualsbc{1,5} \\
            1,2       &  (7)  & \multicolumn{3}{l}{ac<bc} & \InducedStrictOrderI{4,6} \\
	\end{array}
        \]
\end{proof}


\label{awbwcInNaturalwcNotEqualsZerowcaLneqcbEqvaLneqb}
\begin{theorem}[\(ca<cb\dashv\vdash a<b\)]
Seien \(a,b,c\in\mathbb{N}\) und \(c\neq 0\), dann gilt:
\[ca<cb\dashv\vdash a<b\]
\end{theorem}
\begin{proof}
Seien \(a,b,c\in\mathbb{N}\).
\(\vdash\):
       \[
	\begin{array}{lllcll}
            1       &  (1)  & \multicolumn{3}{l}{c\neq 0} & \rA \\
            2       &  (2)  & \multicolumn{3}{l}{ca<cb} & \rA \\
            2       &  (3)  & \multicolumn{3}{l}{ca\leq cb} & \InducedStrictOrderE{2} \\
            2       &  (4)  & \multicolumn{3}{l}{ca\neq cb} & \InducedStrictOrderE{2} \\
            1,2     &  (5)  & \multicolumn{3}{l}{a\leq b} & \awbwcInNaturalwcNotEqualsZerowcaLeqcbImpaLeqb{1,3} \\
            2     &  (6)  & \multicolumn{3}{l}{a\neq b} & \awbwcInNaturalwcaNotEqualscbImpaNotEqualsb{4} \\
            1,2     &  (7)  & \multicolumn{3}{l}{a<b} & \InducedStrictOrderI{5,6} \\
	\end{array}
        \]
\(\dashv\):
       \[
	\begin{array}{lllcll}
            1       &  (1)  & \multicolumn{3}{l}{c\neq 0} & \rA \\
            2       &  (2)  & \multicolumn{3}{l}{a<b} & \rA \\
            2       &  (3)  & \multicolumn{3}{l}{a\leq b} & \InducedStrictOrderE{2} \\
            2       &  (4)  & \multicolumn{3}{l}{ca\leq cb} & \awbwcInNaturalImpaLeqbImpcaLeqcb{3} \\
            2       &  (5)  & \multicolumn{3}{l}{a\neq b} & \InducedStrictOrderE{2} \\
            1,2       &  (6)  & \multicolumn{3}{l}{ca\neq cb} & \awbwcInNaturalwcNotEqualsZerowaNotEqualsbImpcaNotEqualscb{1,5} \\
            1,2       &  (7)  & \multicolumn{3}{l}{ca<cb} & \InducedStrictOrderI{4,6} \\
	\end{array}
        \]
\end{proof}

\label{awbInNaturalwbGneqOneImpaLneqba}
\begin{theorem}[\(a,b\in\mathbb{N},b>1\vdash a<ba\)]
\end{theorem}
\begin{proof}
Seien \(a,b,c\in\mathbb{N}\). \(\ImpLpNaturalwMultwOneRpInAbelSemiRing{}\) und daher gilt:
       \[
	\begin{array}{lllcll}
            1       &  (1)  & \multicolumn{3}{l}{b>1} & \rA \\
            1       &  (2)  & \multicolumn{3}{l}{1<b} & \rgtE{1} \\
                    &  (3)  & a&=&1\cdot a & \rNeutralElementMonoid{} \\
            1       &  (4)  &  &<&ba & \awbwcInNaturalwcNotEqualsZerowacLneqbcEqvaLneqb{2} \\
            1       &  (5)  &  a&<&ba & \rTransitivityOrdRI{3,4} \\
	\end{array}
        \]
\end{proof}


\chapter{Division von natürlichen Zahlen}

\begin{definition}[Teilbarkeit]
Seien \( a, b \in \mathbb{N} \) mit \( b \neq 0 \). Wir sagen, \( b \) teilt \( a \) (geschrieben \( b \mid a \)), wenn:
\[
b \mid a \quad := \quad \exists k \in \mathbb{N} \ (a = b \cdot k).
\]
\end{definition}

\begin{definition}[\(\nmid\)]
Seien \( a, b \in \mathbb{N} \) mit \( b \neq 0 \). Wir sagen, \( b \) teilt \( a \) nicht (geschrieben \( b \nmid a \)), wenn:
\[
b \nmid a \quad := \quad \neg(b \mid a).
\]
\end{definition}

\paragraph{Beweisregeln für die Teilbarkeit}
\label{rule:rDivisibilityI} \label{rule:rDivisibilityE}
Basierend auf der Definition der Teilbarkeit können wir folgende Regeln formulieren:

\[
\begin{array}{llll}
    i & (1) & a \in \mathbb{N} & ... \\
    j & (2) & b \in \mathbb{N} & ... \\
    k & (3) & b \mid a & ... \\
    i,j,k & (4) & \exists k \in \mathbb{N} \ (a = b \cdot k) & \rDivisibilityE{1,2,3} \\
    i,j,k & (5) & b\neq 0 & \rDivisibilityE{1,2,3} \\
\end{array}
\]

\[
\begin{array}{llll}
    i & (1) & a \in \mathbb{N} & ... \\
    j & (2) & b \in \mathbb{N} & ... \\
    k & (3) & b \neq 0 & ... \\
    l & (4) & \exists k \in \mathbb{N} \ (a = b \cdot k) & ... \\
    i,j,k,l & (5) & b \mid a & \rDivisibilityI{1,2,3,4} \\
\end{array}
\]

\[
\begin{array}{llll}
    i & (1) & a \in \mathbb{N} & ... \\
    j & (2) & b \in \mathbb{N} & ... \\
    k & (3) & k \in \mathbb{N} & ... \\
    l & (4) & b \neq 0 & ... \\
    m & (5) & a = b \cdot k & ... \\
    i,j,k,l,m & (6) & b \mid a & \rDivisibilityI{1,2,3,4,5} \\
\end{array}
\]

\[
\begin{array}{llll}
    i & (1) & a \in \mathbb{N} & ... \\
    j & (2) & b \in \mathbb{N} & ... \\
    k & (3) & k \in \mathbb{N} & ... \\
    l & (4) & b \neq 0 & ... \\
    m & (5) & a = k\cdot b  & ... \\
    i,j,k,l,m & (6) & b \mid a & \rDivisibilityI{1,2,3,4,5} \\
\end{array}
\]

Dabei sind \( i \), \( j \), \(k\), \(l\) und \(m\) Listen von Annahmen.

\section{Eindeutigkeit der Division}

\label{ExkSubOnewkSubTwoInNaturalLpaEqualsbMultkSubOneAndaEqualsbMultkSubTwoRpImpkSubOneEqualskSubTwo}
\begin{theorem}[\(\exists k_1, k_2 \in \mathbb{N} (a = b \cdot k_1 \land a = b \cdot k_2) \vdash k_1 = k_2\) (Eindeutigkeit der Teilbarkeit)]
Seien \( a, b \in \mathbb{N} \) mit \( b \neq 0 \), und es gelte \( a = b \cdot k_1 \) sowie \( a = b \cdot k_2 \) für \( k_1, k_2 \in \mathbb{N} \). Dann folgt, dass \( k_1 = k_2 \).
\end{theorem}

\begin{proof}
Seien \( a, b \in \mathbb{N} \), dann gilt:
\[
\begin{array}{llll}
    1 & (1) & \exists k_1, k_2 \in \mathbb{N} (a = b \cdot k_1 \land a = b \cdot k_2) & \rA \\
    1  & (2) & k_1 \in \mathbb{N} \land \exists k_2 \in \mathbb{N} (a = b \cdot k_1 \land a = b \cdot k_2) & \rSetEEm{1} \\
    1 & (3) & k_1 \in \mathbb{N} & \rAEa{2} \\
    1 & (4) & \exists k_2 \in \mathbb{N} (a = b \cdot k_1 \land a = b \cdot k_2) & \rAEb{2} \\
    1 & (5) & k_2 \in \mathbb{N} \land (a = b \cdot k_1 \land a = b \cdot k_2) & \rSetEEm{4} \\
    1 & (6) & k_2 \in \mathbb{N} & \rAEa{5} \\
    1 & (7) & a = b \cdot k_1 \land a = b \cdot k_2 & \rAEb{5} \\
    1 & (8) & a = b \cdot k_1 & \rAEa{7} \\
    1 & (9) & a = b \cdot k_2 & \rAEb{7} \\
    1 & (10) & b \cdot k_1 = b \cdot k_2 & \rIE{8,9} \\
    11 & (11) & b\neq 0 & \rA \\
    1,11 & (12) & k_1 = k_2 & \awbwcInNaturalwcNotEqualsZerowacEqualsbcImpaEqualsb{11,10} \\
\end{array}
\]
\end{proof}

\section{Definition der Division}

\begin{definition}[Division (\( \frac{a}{b} \))]
Seien \( a, b \in \mathbb{N} \) mit \( b \neq 0 \). Die Division \( a \div b \) ist definiert durch:
\[
\forall a, b \in \mathbb{N} \ [b \neq 0 \land b \mid a \rightarrow (\frac{a}{b} \coloneqq \iota k \, (k \in \mathbb{N} \land a = b \cdot k))].
\]
\end{definition}

\begin{remark}
Die Division \( \frac{a}{b} \) ist nur definiert, wenn \( b \neq 0 \) und \( b \) ein Teiler von \( a \) ist. Dies stellt sicher, dass \( k \) eindeutig existiert und die Operation wohldefiniert ist.
\end{remark}

\begin{definition}[Division]
Die Division zweier natürlicher Zahlen \( a \) und \( b \) (mit \( b \neq 0 \)) ist eine partielle binäre Operation
\[
\div : \{(a, b) \in \mathbb{N} \times \mathbb{N} \mid b \neq 0 \text{ und } b \mid a\} \to \mathbb{N},
\]
definiert durch
\[
\frac{a}{b} := k, \quad \text{wobei } k \in \mathbb{N} \text{ und } a = b \cdot k.
\]
\end{definition}

\begin{remark}
In dieser Definition bedeutet \(\frac{a}{b}\), dass wir die natürliche Zahl \( k \) finden, für die \( a = b \cdot k \) gilt. Da in den natürlichen Zahlen keine Brüche oder Dezimalzahlen existieren, ist die Division nur dann definiert, wenn \( b \) ein Teiler von \( a \) ist. Die Voraussetzung \( b \neq 0 \) stellt sicher, dass die Division durch Null ausgeschlossen ist.
\end{remark}

\paragraph{Beweisregeln für die Division}
\label{rule:rDivisionI}
Basierend auf dieser Definition können wir folgende Regeln für die Division formulieren:

\[
\begin{array}{llll}
    i & (1) & a \in \mathbb{N} & ... \\
    j & (2) & b \in \mathbb{N} & ... \\
    k & (3) & b \mid a & ... \\
    i,j,k & (4) & a = b \cdot \frac{a}{b} & \rDivisionI{1,2,3} \\
    i,j,k & (5) & \frac{a}{b} \in \mathbb{N} & \rDivisionI{1,2,3} \\
    i,j,k & (6) & b\cdot \frac{a}{b} \in \mathbb{N} & \rDivisionI{1,2,3} \\
\end{array}
\]

\[
\begin{array}{llll}
    i & (1) & a \in \mathbb{N} & ... \\
    j & (2) & b \in \mathbb{N} & ... \\
    k & (3) & c \in \mathbb{N} & ... \\
    l & (4) & b\neq 0 & ... \\
    m & (5) & a=b\cdot c & ... \\
    i,j,k,l,m & (6) & c = \frac{a}{b} & \rDivisionI{1,2,3,4,5} \\
\end{array}
\]

\[
\begin{array}{llll}
    i & (1) & a \in \mathbb{N} & ... \\
    j & (2) & b \in \mathbb{N} & ... \\
    k & (3) & c \in \mathbb{N} & ... \\
    l & (4) & b\neq 0 & ... \\
    m & (5) & a=c\cdot b & ... \\
    i,j,k,l,m & (6) & c = \frac{a}{b} & \rDivisionI{1,2,3,4,5} \\
\end{array}
\]

Dabei sind \( i \), \( j \), \(k\), \(l\) und \( m \) Listen von Annahmen.

\label{aInNaturalImpOneMida}
\begin{theorem}[\(a\in\mathbb{N}\vdash 1\mid a\)]
\end{theorem}
\begin{proof}
Sei \(a\in\mathbb{N}\), dann gilt:
        \[
	\begin{array}{llll}
                &  (1)  & a=1\cdot a & \rNeutralElementMonoid{} \\
                &  (2)  & 1\neq 0 & \nInNaturalImpnPlusOneNotEqualsZero{} \\
                &  (2)  & 1\mid a & \rDivisibilityI{1} \\
    \end{array}
	\]
\end{proof}

\label{aInNaturalImpLpaRpDurchLpOneRpEqualsa}
\begin{theorem}[\(a\in\mathbb{N}\vdash \frac{a}{1}=a\)]
\end{theorem}
\begin{proof}
Sei \(a\in\mathbb{N}\), dann gilt:
        \[
	\begin{array}{llll}
                &  (1)  & a=1\cdot a & \rNeutralElementMonoid{} \\
                &  (2)  & 1\mid a& \aInNaturalImpOneMida{} \\
                &  (2)  & \frac{a}{1}=a & \rDivisionI{2,1} \\
    \end{array}
	\]
\end{proof}


\label{aInNaturalwaNotEqualsZeroImpLpZeroRpDurchLpaRpEqualsZero}
\begin{theorem}[\(a\in\mathbb{N},a\neq 0\vdash 0=\frac{0}{a}\)]
\end{theorem}
\begin{proof}
Im Beweis nutzen wir das Theorem \(\ImpLpNaturalwMultwOneRpInAbelSemiRing{}\). 
Sei \(a\in\mathbb{N}\), dann gilt:
        \[
	\begin{array}{llll}
        1       &  (1)  & a\neq 0 & \rA \\
                &  (2)  & 0=a\cdot 0 & \rNeutralElementMonoid{} \\
        1       &  (3)  & 0=\frac{0}{a} & \rDivisionI{1} \\
    \end{array}
	\]
\end{proof}


\label{awbInNaturalwbNotEqualsZerowaMidbImpLpbRpDurchLpaRpNotEqualsZero}
\begin{theorem}[\(a,b \in\mathbb{N}, b\neq 0, a\mid b\vdash \frac{b}{a}\neq 0\) ]
\end{theorem}
\begin{proof}
Seien \(a,b\in\mathbb{N}\).
Im Beweis nutzen wir das Theorem \(\ImpLpNaturalwMultwOneRpInAbelSemiRing{}\). 
    \[
    \begin{array}{llclll}
    1       &  (1)  & \multicolumn{3}{l}{a\mid b} & \rA \\
    2       &  (2)  & \multicolumn{3}{l}{b\neq 0} & \rA \\
    1       &  (3)  & \multicolumn{3}{l}{b=a\cdot \frac{b}{a}} & \rDivisionI{1} \\
    4       &  (4)  & \multicolumn{3}{l}{\frac{b}{a}=0} & \rA \\
    4       &  (5)  & \multicolumn{3}{l}{\frac{b}{a}=0} & \rA \\
    1,4     &  (6)  & b&=&a\cdot 0 & \rIE{4,2} \\
    1,4     &  (7)  &  &=&0 & \aInNaturalImpZeroEqualsZeroMulta{} \\
    1,4     &  (9)  & \multicolumn{3}{l}{b=0} & \rTransitivityEqRI{6,7} \\
    1,2,4   &  (10)  & \multicolumn{3}{l}{\bot} & \rBI{2,9} \\
    1,2   &  (11)  & \multicolumn{3}{l}{\frac{b}{a}\neq 0} & \rCI{4,10} \\
    \end{array}
    \]
\end{proof}

\label{awbInNaturalwbNotEqualsZerowaMidbImpaLeqb}
\begin{theorem}[\(a,b \in\mathbb{N}, b\neq 0, a\mid b\vdash a\leq b\)]
\end{theorem}
\begin{proof}
Seien \(a,b\in\mathbb{N}\).
Im Beweis nutzen wir das Theorem \(\ImpLpNaturalwMultwOneRpInAbelSemiRing{}\). 
    \[
	\begin{array}{llclll}
    1       &  (1)  & \multicolumn{3}{l}{a\mid b} & \rA \\
    2       &  (2)  & \multicolumn{3}{l}{b\neq 0} & \rA \\
    1,2     &  (3)  & \multicolumn{3}{l}{\frac{b}{a}\neq 0} & \awbInNaturalwbNotEqualsZerowaMidbImpLpbRpDurchLpaRpNotEqualsZero{1,2} \\
    1       &  (4)  & \multicolumn{3}{l}{b=a\cdot \frac{b}{a}} & \rDivisionI{1} \\
    1,2     &  (5)  & \multicolumn{3}{l}{\frac{b}{a}=(\frac{b}{a}-1)+1} & \rPredecessorI{3} \\
    1       &  (6)  & b&=&a\cdot ((\frac{b}{a}-1)+1) & \rIE{5,4} \\
    1       &  (7)  & &=& a\cdot (\frac{b}{a}-1)+a & \rLeftDistributiveAbelianSemigroup{} \\
    1       &  (8)  & &=& a+a\cdot (\frac{b}{a}-1) & \rCommutativeMonoid{} \\
    1       &  (9)  & b &=& a+a\cdot (\frac{b}{a}-1) & \rTransitivityEqRI{3,6} \\
    1       &  (10)  &  \multicolumn{3}{l}{a\leq b} & \rLeqNI{6} \\
    \end{array}
	\]
\end{proof}

\label{awbwcInNaturalwcMidawcMidbImpcMidLbaPlusbRb}
\begin{theorem}[\(a,b,c\in\mathbb{N}, c\mid a, c\mid b\vdash c\mid{a+b}\)]
\end{theorem}
\begin{proof}
Im Beweis nutzen wir das Theorem \(\ImpLpNaturalwMultwOneRpInAbelSemiRing{}\). 
Seien \(a,b,c\in\mathbb{N}\), dann gilt:
        \[
	\begin{array}{lllcll}
            1       &  (1)  & \multicolumn{3}{l}{c\mid a} & \rA \\
            2       &  (2)  & \multicolumn{3}{l}{c\mid b} & \rA \\
            1       &  (3)  & \multicolumn{3}{l}{a=c\cdot \frac{a}{c}} & \rDivisionI{1}  \\
            2       &  (4)  & \multicolumn{3}{l}{b=c\cdot \frac{b}{c}} & \rDivisionI{2}  \\
                    &  (5)  & a+b&=&a+b & \rII{}  \\
            1       &  (6)  & &=&c\cdot \frac{a}{c}+b & \rIE{3,5}  \\
            1,2     &  (7)  & &=&c\cdot \frac{a}{c}+c\cdot \frac{b}{c} & \rIE{4,6}  \\
            1,2     &  (8)  & &=&c(\frac{a}{c}+\frac{b}{c}) & \rLeftDistributiveAbelianSemigroup{}  \\
            1,2     &  (9) & \multicolumn{3}{l}{a+b=c(\frac{a}{c}+\frac{b}{c})} & \rTransitivityEqRI{5,8}  \\
            1     &  (10) & \multicolumn{3}{l}{c\neq 0} & \rDivisibilityE{1}  \\
            1,2   &  (11) & \multicolumn{3}{l}{c\mid{a+b}} & \rDivisibilityI{10,9}  \\
        \end{array}
	\]
\end{proof}



\label{aInNaturalwbInNaturalwcInNaturalwcMidawcMidbImpLpaPlusbRpDurchLpcRpEqualsLpaRpDurchLpcRpPlusLpbRpDurchLpcRp}
\begin{theorem}[\(a,b,c\in\mathbb{N}, c\mid a, c\mid b\vdash \frac{a+b}{c}=\frac{a}{c}+\frac{b}{c}\)]
\end{theorem}
\begin{proof}
Im Beweis nutzen wir das Theorem \(\ImpLpNaturalwMultwOneRpInAbelSemiRing{}\). 
Seien \(a,b,c\in\mathbb{N}\), dann gilt:
    \[
	\begin{array}{llclll}
    1       &  (1)  & \multicolumn{3}{l}{c\mid a} & \rA \\
    2       &  (2)  & \multicolumn{3}{l}{c\mid b} & \rA \\
    1,2     &  (3)  & \multicolumn{3}{l}{c\mid a+b} & \awbwcInNaturalwcMidawcMidbImpcMidLbaPlusbRb{1,2} \\
    1       &  (4)  & \multicolumn{3}{l}{a=c\cdot \frac{a}{c}} & \rDivisionI{1}  \\
    2       &  (5)  & \multicolumn{3}{l}{b=c\cdot \frac{b}{c}} & \rDivisionI{2}  \\
    1,2     &  (6)  & \multicolumn{3}{l}{a+b=c\cdot \frac{a+b}{c}} & \rDivisionI{3}  \\
    1,2     &  (7)  & c\cdot \frac{a+b}{c}&=&a+b & \rSymmetryEqRI{6}  \\
    1,2     &  (8)  & &=&c\cdot \frac{a}{c}+b & \rIE{4,7}  \\
    1,2     &  (9)  & &=&c\cdot \frac{a}{c}+c\cdot \frac{b}{c} & \rIE{5,8} \\
    1,2     &  (10)  & &=&c\cdot (\frac{a}{c}+\frac{b}{c}) & \rLeftDistributiveAbelianSemigroup{} \\
    1,2     &  (11)  & \multicolumn{3}{l}{c\cdot \frac{a+b}{c}=c\cdot (\frac{a}{c}+\frac{b}{c})} & \rTransitivityEqRI{7,10} \\
    1       &  (12)  & \multicolumn{3}{l}{c\neq 0} & \rDivisibilityE{1} \\
    1,2     &  (13)  & \multicolumn{3}{l}{\frac{a+b}{c}=\frac{a}{c}+\frac{b}{c}} & \awbwcInNaturalwcNotEqualsZerowcaEqualscbImpaEqualsb{12,11} \\

    \end{array}
	\]
\end{proof}

\label{awbwcwdInNaturalwcMidawdMidbImpcdMidab}
\begin{theorem}[\(a,b,c,d\in\mathbb{N},c\mid a, d\mid b\vdash cd\mid ab\)]
\end{theorem}
\begin{proof}
Seien \(a,b,c,d\in\mathbb{N}\).
Im Beweis nutzen wir das Theorem \(\ImpLpNaturalwMultwOneRpInAbelSemiRing{}\). 
    \[
	\begin{array}{llclll}
    1       &  (1)  & \multicolumn{3}{l}{c\mid a} & \rA \\
    2       &  (2)  & \multicolumn{3}{l}{d\mid b} & \rA \\
    1       &  (3)  & \multicolumn{3}{l}{a=c\cdot \frac{a}{c}} & \rDivisionI{1} \\
    2       &  (4)  & \multicolumn{3}{l}{b=d\cdot \frac{b}{d}} & \rDivisionI{2} \\
            &  (5)  & ab&=&ab & \rII{} \\
    1       &  (6)  & &=&(c\cdot \frac{a}{c})\cdot b & \rIE{3,5} \\
    1,2     &  (7)  & &=&(c\cdot \frac{a}{c})\cdot (d\cdot \frac{b}{d}) & \rIE{3,5} \\
    1,2       &  (8)  & &=&(c\cdot d)\cdot (\frac{a}{c}\cdot \frac{b}{d}) & \aInMwbInMwcInMwdInMImpLpaPlusbRpPlusLpcPlusdRpEqualsLpaPluscRpPlusLpbPlusdRp{} \\
    1,2       &  (9) &\multicolumn{3}{l}{ab=(c\cdot d)\cdot (\frac{a}{c}\cdot \frac{b}{d})} & \rTransitivityEqRI{5,8} \\
    1       &  (10)  & \multicolumn{3}{l}{c\neq 0} & \rDivisibilityE{1} \\
    2       &  (11)  & \multicolumn{3}{l}{d\neq 0} & \rDivisibilityE{2} \\
    1,2     &  (12)  & \multicolumn{3}{l}{cd\neq 0} & \awbInNaturalwaNotEqualsZerowbNotEqualsZeroImpabNotEqualsZero{10,11} \\
    1,2     &  (13)  & \multicolumn{3}{l}{cd\neq 0} & \awbInNaturalwaNotEqualsZerowbNotEqualsZeroImpabNotEqualsZero{10,11} \\
    1,2     &  (14)  & \multicolumn{3}{l}{cd\mid ab} & \rDivisibilityI{13,9} \\
    \end{array}
	\]
\end{proof}

\label{awbwcwdInNaturalwcMidawdMidbImpLpabRpDurchLpcdRpEqualsLpaRpDurchLpcRpMultLpbRpDurchLpdRp}
\begin{theorem}[\(a,b,c,d\in\mathbb{N},c\mid a, d\mid b\vdash \frac{ab}{cd}=\frac{a}{c}\cdot\frac{b}{d}\)]
\end{theorem}
\begin{proof}
Seien \(a,b,c,d\in\mathbb{N}\).
Im Beweis nutzen wir das Theorem \(\ImpLpNaturalwMultwOneRpInAbelSemiRing{}\). 
    \[
	\begin{array}{llclll}
    1       &  (1)  & \multicolumn{3}{l}{c\mid a} & \rA \\
    2       &  (2)  & \multicolumn{3}{l}{d\mid b} & \rA \\
    1       &  (3)  & \multicolumn{3}{l}{a=c\cdot \frac{a}{c}} & \rDivisionI{1} \\
    2       &  (4)  & \multicolumn{3}{l}{b=d\cdot \frac{b}{d}} & \rDivisionI{2} \\
            &  (5)  & ab&=&ab & \rII{} \\
    1       &  (6)  & &=&(c\cdot \frac{a}{c})\cdot b & \rIE{3,5} \\
    1,2     &  (7)  & &=&(c\cdot \frac{a}{c})\cdot (d\cdot \frac{b}{d}) & \rIE{3,5} \\
    1,2       &  (8)  & &=&(cd)\cdot (\frac{a}{c}\cdot \frac{b}{d}) & \aInMwbInMwcInMwdInMImpLpaPlusbRpPlusLpcPlusdRpEqualsLpaPluscRpPlusLpbPlusdRp{} \\
    1,2       &  (9) &\multicolumn{3}{l}{ab=(cd)\cdot (\frac{a}{c}\cdot \frac{b}{d})} & \rTransitivityEqRI{5,8} \\
    1,2       &  (10)  & \multicolumn{3}{l}{cd\mid ab} & \awbwcwdInNaturalwcMidawdMidbImpcdMidab{1,2} \\
    1,2       &  (11)  & \multicolumn{3}{l}{\frac{ab}{cd}=\frac{a}{c}\cdot\frac{b}{d}} & \rDivisionI{10,9} \\
    \end{array}
	\]
\end{proof}

\label{aInNaturalwbInNaturalwcInNaturalwdInNaturalwcMidawdMidbImpdcMidadPlusbc}
\begin{theorem}[\(a\in\mathbb{N},b\in\mathbb{N}, c\in\mathbb{N}, d\in\mathbb{N},c\mid a, d\mid b\vdash dc\mid ad+bc\)]
\end{theorem}
\begin{proof}
Seien \(a,b,c,d\in\mathbb{N}\).
Im Beweis nutzen wir das Theorem \(\ImpLpNaturalwMultwOneRpInAbelSemiRing{}\). 
    \[
	\begin{array}{llclll}
    1       &  (1)  & \multicolumn{3}{l}{c\mid a} & \rA \\
    2       &  (2)  & \multicolumn{3}{l}{d\mid b} & \rA \\
    1       &  (3)  & \multicolumn{3}{l}{a=c\cdot \frac{a}{c}} & \rDivisionI{1} \\
    2       &  (4)  & \multicolumn{3}{l}{b=d\cdot \frac{b}{d}} & \rDivisionI{2} \\
            &  (5)  & ad+bc&=&ad+bc & \rII{} \\
    1       &  (6)  & &=&((c\cdot \frac{a}{c})\cdot d)+ bc & \rIE{3,5} \\
    1,2     &  (7)  & &=&((c\cdot \frac{a}{c})\cdot d)+((d\cdot \frac{b}{d})\cdot c) & \rIE{4,6} \\
    1,2     &  (8)  & &=&(dc\cdot \frac{a}{c})+((d\cdot \frac{b}{d})\cdot c) & \MInAbelMonoidwawbwcInMImpLpabRpcEqualsLpcaRpb{} \\
    1,2     &  (9)  & &=&(dc\cdot \frac{a}{c})+(dc\cdot \frac{b}{d}) & \awbwcwdInNaturalwcMidawdMidbImpcdMidab{} \\
    1,2     &  (10)  & &=&dc(\frac{a}{c}+\frac{b}{d}) & \rLeftDistributiveAbelianSemigroup{} \\
    1,2     &  (11) &\multicolumn{3}{l}{ad+bc=dc(\frac{a}{c}+\frac{b}{d})} & \rTransitivityEqRI{5,10} \\
    1,2     &  (12)  & \multicolumn{3}{l}{c\neq 0} & \rDivisibilityE{1} \\
    2       &  (13)  & \multicolumn{3}{l}{d\neq 0} & \rDivisibilityE{2} \\
    1,2     &  (14)  & \multicolumn{3}{l}{dc\neq 0} & \awbInNaturalwaNotEqualsZerowbNotEqualsZeroImpabNotEqualsZero{13,12} \\
    1,2     &  (15)  & \multicolumn{3}{l}{dc\mid ad+bc} & \rDivisibilityI{14,11} \\
    \end{array}
	\]
\end{proof}

\label{aInNaturalwbInNaturalwcInNaturalwdInNaturalwcMidawdMidbImpLpadPlusbcRpDurchLpdcRpEqualsLpaRpDurchLpcRpPlusLpbRpDurchLpdRp}
\begin{theorem}[\(a\in\mathbb{N},b\in\mathbb{N}, c\in\mathbb{N}, d\in\mathbb{N},c\mid a, d\mid b\vdash \frac{ad+bc}{dc}=\frac{a}{c}+\frac{b}{d}\)]
\end{theorem}
\begin{proof}
Seien \(a,b,c,d\in\mathbb{N}\).
Im Beweis nutzen wir das Theorem \(\ImpLpNaturalwMultwOneRpInAbelSemiRing{}\). 
    \[
	\begin{array}{llclll}
    1       &  (1)  & \multicolumn{3}{l}{c\mid a} & \rA \\
    2       &  (2)  & \multicolumn{3}{l}{d\mid b} & \rA \\
    1       &  (3)  & \multicolumn{3}{l}{a=c\cdot \frac{a}{c}} & \rDivisionI{1} \\
    2       &  (4)  & \multicolumn{3}{l}{b=d\cdot \frac{b}{d}} & \rDivisionI{2} \\
            &  (5)  & ad+bc&=&ad+bc & \rII{} \\
    1       &  (6)  & &=&((c\cdot \frac{a}{c})\cdot d)+ bc & \rIE{3,5} \\
    1,2     &  (7)  & &=&((c\cdot \frac{a}{c})\cdot d)+((d\cdot \frac{b}{d})\cdot c) & \rIE{4,6} \\
    1,2     &  (8)  & &=&(dc\cdot \frac{a}{c})+((d\cdot \frac{b}{d})\cdot c) & \MInAbelMonoidwawbwcInMImpLpabRpcEqualsLpcaRpb{} \\
    1,2     &  (9)  & &=&(dc\cdot \frac{a}{c})+(dc\cdot \frac{b}{d}) & \awbwcwdInNaturalwcMidawdMidbImpcdMidab{} \\
    1,2     &  (10)  & &=&dc(\frac{a}{c}+\frac{b}{d}) & \rLeftDistributiveAbelianSemigroup{} \\
    1,2     &  (11) &\multicolumn{3}{l}{ad+bc=dc(\frac{a}{c}+\frac{b}{d})} & \rTransitivityEqRI{5,10} \\
    1,2     &  (12)  & \multicolumn{3}{l}{dc\mid ad+bc} & \aInNaturalwbInNaturalwcInNaturalwdInNaturalwcMidawdMidbImpdcMidadPlusbc{1,2} \\
    1,2     &  (13)  & \multicolumn{3}{l}{\frac{ad+bc}{dc}=\frac{a}{c}+\frac{b}{d}} & \rDivisionI{12,11} \\
    \end{array}
	\]
\end{proof}

\section{Teilbarkeit und Relationen}

\label{awbInNaturalwaMidbwaMidcwbLeqcImpLpbRpDurchLpaRpLeqLpcRpDurchLpaRp}
\begin{theorem}[\(a,b\in\mathbb{N}, a\mid b, a\mid c, b\leq c\vdash \frac{b}{a}\leq \frac{c}{a}\)]    
\end{theorem}
\begin{proof}
Seien \(a,b,c \in\mathbb{N}\).
    \[
	\begin{array}{llclll}
    1       &  (1)  & \multicolumn{3}{l}{a\mid b} & \rA \\
    2       &  (2)  & \multicolumn{3}{l}{a\mid c} & \rA \\
    1       &  (3)  & \multicolumn{3}{l}{b=a\cdot\frac{b}{a}} & \rDivisionI{1} \\
    2       &  (4)  & \multicolumn{3}{l}{c=a\cdot\frac{c}{a}} & \rDivisionI{2} \\
    1       &  (5)  & a\cdot \frac{b}{a}&=&b & \rSymmetryEqRI{3} \\
    7       &  (6)  & &\leq &c & \rA \\
    2,7     &  (7)  & &= &a\cdot\frac{c}{a} & \rIE{4,6} \\
    1,2,7   &  (8)  & a\cdot \frac{b}{a}&\leq  &a\cdot\frac{c}{a} & \rIE{4,6} \\
    1       &  (9)  & \multicolumn{3}{l}{a\neq 0} & \rDivisibilityE{1} \\
    1,2,7   &  (10)  & \frac{b}{a}&\leq  &\frac{c}{a} & \awbwcInNaturalwcNotEqualsZerowcaLeqcbImpaLeqb{9,8} \\
    \end{array}
	\]
\end{proof}

\section{Eigenschaften der Halbordnung}

\label{aInNaturalwaNotEqualsZeroImpaMidLbaRb}
\begin{theorem}[\(a\in\mathbb{N}, a\neq 0 \vdash a\mid{a}\)]
\end{theorem}
\begin{proof}
Sei \(a\in\mathbb{N}\).
Im Beweis nutzen wir das Theorem \(\ImpLpNaturalwMultwOneRpInAbelSemiRing{}\). 
    \[
	\begin{array}{llclll}
    1       &  (1)  & \multicolumn{3}{l}{a\neq 0} & \rA \\
            &  (2)  & \multicolumn{3}{l}{a=a\cdot 1} & \rNeutralElementMonoid{} \\
    1       &  (3)  & \multicolumn{3}{l}{a\mid a} & \rDivisibilityI{1,2} \\
    \end{array}
	\]
\end{proof}

\label{aInNaturalwaNotEqualsZeroImpLpaRpDurchLpaRpEqualsOne}
\begin{theorem}[\(a\in\mathbb{N}, a\neq 0 \vdash \frac{a}{a}=1\)]
\end{theorem}
\begin{proof}
Sei \(a\in\mathbb{N}\).
Im Beweis nutzen wir das Theorem \(\ImpLpNaturalwMultwOneRpInAbelSemiRing{}\). 
    \[
	\begin{array}{llclll}
    1       &  (1)  & \multicolumn{3}{l}{a\neq 0} & \rA \\
            &  (2)  & \multicolumn{3}{l}{a=a\cdot 1} & \rNeutralElementMonoid{} \\
    1       &  (3)  & \multicolumn{3}{l}{a\mid a} & \rDivisibilityI{1,2} \\
    1       &  (4)  & \multicolumn{3}{l}{\frac{a}{a}=1} & \rDivisionI{2,3} \\
    \end{array}
	\]
\end{proof}


\label{awbInNaturalwaMidbwbMidaImpaEqualsb}
\begin{theorem}[\(a,b\in\mathbb{N}, a\mid b, b\mid a \vdash a=b\)]
\end{theorem}
\begin{proof}
Seien \(a,b\in\mathbb{N}\).
Im Beweis nutzen wir das Theorem \(\LeqIsHalfOrderOnNaturalNumbers{}\). 
    \[
	\begin{array}{llclll}
    1       &  (1)  & \multicolumn{3}{l}{a\mid b} & \rA \\
    2       &  (2)  & \multicolumn{3}{l}{b\mid a} & \rA \\
    1       &  (3)  & \multicolumn{3}{l}{a\neq 0} & \rDivisibilityE{1} \\
    2       &  (4)  & \multicolumn{3}{l}{b\neq 0} & \rDivisibilityE{2} \\
    1,2       &  (5)  & \multicolumn{3}{l}{a\leq b} & \awbInNaturalwbNotEqualsZerowaMidbImpaLeqb{1,4} \\
    1,2       &  (6)  & \multicolumn{3}{l}{b\leq a} & \awbInNaturalwbNotEqualsZerowaMidbImpaLeqb{2,3} \\
    1,2       &  (7)  & \multicolumn{3}{l}{a=b} & \rAntisymmetryOrdRI{5,6} \\
    \end{array}
	\]
\end{proof}

\label{awbwcInNaturalwaMidbwbMidcImpaMidc}
\begin{theorem}[\(a,b,c\in\mathbb{N}, a\mid b, b\mid c \vdash a\mid c\)]
\end{theorem}
\begin{proof}
Sei \(a\in\mathbb{N}\).
Im Beweis nutzen wir das Theorem \(\LeqIsHalfOrderOnNaturalNumbers{}\). 
    \[
	\begin{array}{llclll}
        1       &  (1)  & \multicolumn{3}{l}{a\mid b} & \rA \\
        2       &  (2)  & \multicolumn{3}{l}{b\mid c} & \rA \\
        1       &  (3)  & \multicolumn{3}{l}{b=a\cdot \frac{b}{a}} & \rDivisionI{1} \\
        2       &  (4)  & \multicolumn{3}{l}{c=b\cdot \frac{c}{b}} & \rDivisionI{2} \\
        1       &  (5)  & \multicolumn{3}{l}{a\neq 0} & \rDivisibilityE{1} \\
        1,2     &  (6)  & c&=&(a\cdot \frac{b}{a})\cdot \frac{c}{b} & \rIE{3,4} \\
        1,2     &  (7)  & &=&a\cdot (\frac{b}{a}\cdot \frac{c}{b}) & \rAssociativityMonoid{} \\
        1,2     &  (8)  & \multicolumn{3}{l}{c=a\cdot (\frac{b}{a}\cdot \frac{c}{b})} & \rTransitivityEqRI{6,7} \\
        1,2     &  (9)  & \multicolumn{3}{l}{a\mid c} & \rDivisibilityI{5,8} \\
        \end{array}
    \]
\end{proof}


\label{awbwcInNaturalwaMidbwbMidcImpLpaRpDurchLpcRpEqualsLpbRpDurchLpaRpMultLpcRpDurchLpbRp}
\begin{theorem}[\(a,b,c\in\mathbb{N}, a\mid b, b\mid c \vdash \frac{c}{a}=\frac{b}{a}\cdot\frac{c}{b}\)]
\end{theorem}
\begin{proof}
Seien \(a,b\in\mathbb{N}\).
Im Beweis nutzen wir das Theorem \(\LeqIsHalfOrderOnNaturalNumbers{}\). 
    \[
	\begin{array}{llclll}
        1       &  (1)  & \multicolumn{3}{l}{a\mid b} & \rA \\
        2       &  (2)  & \multicolumn{3}{l}{b\mid c} & \rA \\
        1       &  (3)  & \multicolumn{3}{l}{b=a\cdot \frac{b}{a}} & \rDivisionI{1} \\
        2       &  (4)  & \multicolumn{3}{l}{c=b\cdot \frac{c}{b}} & \rDivisionI{2} \\
        1       &  (5)  & \multicolumn{3}{l}{a\neq 0} & \rDivisibilityE{1} \\
        1,2     &  (6)  & c&=&(a\cdot \frac{b}{a})\cdot \frac{c}{b} & \rIE{3,4} \\
        1,2     &  (7)  & &=&a\cdot (\frac{b}{a}\cdot \frac{c}{b}) & \rAssociativityMonoid{} \\
        1,2     &  (8)  & \multicolumn{3}{l}{c=a\cdot (\frac{b}{a}\cdot \frac{c}{b})} & \rTransitivityEqRI{6,7} \\
        1,2     &  (9)  & \multicolumn{3}{l}{\frac{c}{a}=\frac{b}{a}\cdot \frac{c}{b}} & \rDivisionI{5,8} \\
        \end{array}
    \]
\end{proof}

\label{MidIsHalfOrderOnNaturalNumbers}
\begin{theorem}[\(\mid\) ist eine Halbordnung auf \(\mathbb{N}\) ]
\end{theorem}
\begin{proof}
        \[
	\begin{array}{llll}
                    & (1) & \forall a \in \mathbb{N}  (a \mid a) & \aInNaturalwaNotEqualsZeroImpaMidLbaRb{} \\
                    & (2) & \forall a, b \in \mathbb{N}  ((a \mid b \land b \mid a) \rightarrow a = b) & \awbInNaturalwaMidbwbMidaImpaEqualsb{} \\
                    & (3) & \forall a, b, c \in \mathbb{N}  ((a \mid b \land b \mid c) \rightarrow a \mid c) & \awbwcInNaturalwaMidbwbMidcImpaMidc{} \\
                    & (4) & \mid \text{ ist eine Halbordnung auf } \mathbb{N} &  \rPartialOrderRelationI{1,2,3}
    \end{array}
	\]
\end{proof}



\label{awbInNaturalwabEqualsOneImpaEqualsOneAndbEqualsOne}
\begin{theorem}[\(a,b\in\mathbb{N},ab=1\vdash a=1\land b=1\)]
\end{theorem}
\begin{proof}
Sei \(a\in\mathbb{N}\).
Im Beweis nutzen wir das Theorem \(\MidIsHalfOrderOnNaturalNumbers{}\). 
    \[
	\begin{array}{llclll}
        1       &  (1)  & \multicolumn{3}{l}{ab=1} & \rA \\
                &  (2)  & \multicolumn{3}{l}{1\neq 0} & \ImpOneNotEqualsZero{} \\
        1       &  (3)  & \multicolumn{3}{l}{ab\neq 0} & \rIE{1,2} \\
        1       &  (4)  & \multicolumn{3}{l}{a\neq 0} & \aInNaturalwabNotEqualsZeroImpaNotEqualsZero{3} \\
        1       &  (5)  & \multicolumn{3}{l}{b\neq 0} & \aInNaturalwabNotEqualsZeroImpbNotEqualsZero{3} \\
        1       &  (6)  & \multicolumn{3}{l}{a\mid 1} & \rDivisibilityI{1,4} \\
        1       &  (7)  & \multicolumn{3}{l}{b\mid 1} & \rDivisibilityI{1,5} \\
                &  (8)  & \multicolumn{3}{l}{1\mid a} & \aInNaturalImpOneMida{} \\
                &  (9)  & \multicolumn{3}{l}{1\mid b} & \aInNaturalImpOneMida{} \\
        1       &  (10)  & \multicolumn{3}{l}{a=1} & \rAntisymmetryOrdRI{6,8} \\
        1       &  (11)  & \multicolumn{3}{l}{b=1} & \rAntisymmetryOrdRI{7,9} \\
        1       &  (12)  & \multicolumn{3}{l}{a=1\land b=1} & \rAI{10,11} \\
        \end{array}
    \]
\end{proof}


\label{awbInNaturalwaNotEqualsOnewaMidbImpaNMidbPlusOne}
\begin{theorem}[\(a,b\in\mathbb{N}, a\neq 1, a\mid b \vdash a\nmid b+1\)]
\end{theorem}
\begin{proof}
Seien \(a,b\in\mathbb{N}\).
    \[
	\begin{array}{llclll}
        1       &  (1)  & \multicolumn{3}{l}{a\mid b} & \rA \\
        1       &  (2)  & \multicolumn{3}{l}{b=a\cdot \frac{b}{a}} & \rDivisionI{1} \\
        3       &  (3)  & \multicolumn{3}{l}{a\mid b+1} & \rA \\
        3       &  (4)  & \multicolumn{3}{l}{b+1=a\cdot \frac{b+1}{a}} & \rDivisionI{3} \\
                &  (5)  & \multicolumn{3}{l}{b\leq b+1} & \aInNaturalImpaLeqaPlusOne{} \\
        1,3     &  (6)  & \multicolumn{3}{l}{\frac{b}{a}\leq \frac{b+1}{a}} & \awbInNaturalwaMidbwaMidcwbLeqcImpLpbRpDurchLpaRpLeqLpcRpDurchLpaRp{1,3,5} \\
                &  (7)  & 1&=&1+0 & \rNeutralElementMonoid{} \\
                &  (8)  &  &=&1+(a\cdot\frac{b}{a}-a\cdot\frac{b}{a}) & \aInNaturalImpaMinusaEqualsZero{} \\
                &  (9)  &  &=&(1+a\cdot\frac{b}{a})-a\cdot\frac{b}{a} & \awbInNaturalImpLpaPlusbRpMinusbEqualsaPlusLpbMinusbRpEqualsa{} \\
        1       &  (10)  &  &=&(1+b)-a\cdot\frac{b}{a} & \rIE{2,9} \\
        1       &  (11)  &  &=&(b+1)-a\cdot\frac{b}{a} & \rCommutativeMonoid{} \\
        1,3     &  (12)  &  &=&a\cdot \frac{b+1}{a}-a\cdot\frac{b}{a} & \rIE{4,11} \\
        1,3     &  (13)  &  &=&a\cdot(\frac{b+1}{a}-\frac{b}{a}) & \awbwcInNaturalwcLeqbImpaLpbMinuscRpEqualsabMinusac{} \\
        1,3     &  (14)  & 1&=&a\cdot(\frac{b+1}{a}-\frac{b}{a}) & \rTransitivityEqRI{7,13} \\
        1,3     &  (15)  & \multicolumn{3}{l}{a=1\land (\frac{b+1}{a}-\frac{b}{a})=1 } & \awbInNaturalwabEqualsOneImpaEqualsOneAndbEqualsOne{14} \\
        1,3     &  (16)  & \multicolumn{3}{l}{a=1 } & \rAEa{15} \\
        17     &  (17)  & \multicolumn{3}{l}{a\neq 1 } & \rA \\
        1,3,17     &  (18)  & \multicolumn{3}{l}{\bot} & \rBI{16,17} \\
        1,17     &  (19)  & \multicolumn{3}{l}{a\nmid b+1} & \rCI{3,18} \\
        \end{array}
    \]
\end{proof}

\chapter{Darstellung natürlicher Zahlen durch Division mit Rest}

\begin{definition}[Division mit Rest]
Seien \( a, b \in \mathbb{N} \) mit \( b \neq 0 \). Wir nennen \( a \) den \emph{Dividend} und \( b \) den \emph{Divisor}. Wir sagen, \( q \) sei der Quotient und \( r \) der Rest der Division von \( a \) durch \( b \), wenn:
\[
\exists q, r \in \mathbb{N} \ \big(a = b \cdot q + r \land 0 \leq r < b\big).
\]
\end{definition}


\section{Existenz der Division mit Rest}
In diesem Abschnitt seien \(a,b\in\mathbb{N}\).

\begin{tempdefinition}
    \[M:=\{r\in\mathbb{N}\mid \exists q\in\mathbb{N}(r=a-bq)\}\]
\end{tempdefinition}


\label{tIMDefineEqualsLbrInNaturalMidExqInNaturalLprEqualsaMinusbqRpRb}
\paragraph{Einführungsregel für die Elementzugehörigkeit von \(r\) in \(M\)}
\[
\begin{array}{llll}
    i & (1) & r=a-bq & ... \\
    i & (2) & r\in M & \tIMDefineEqualsLbrInNaturalMidExqInNaturalLprEqualsaMinusbqRpRb{1}  \\
\end{array}
\]

\label{tEMDefineEqualsLbrInNaturalMidExqInNaturalLprEqualsaMinusbqRpRb}
\paragraph{Eliminierungsregel für die Elementzugehörigkeit von \(r\) in \(M\)}
\[
\begin{array}{llll}
    i & (1) & r\in M & ... \\
    i & (2) & \exists q\in\mathbb{N}(r=a-bq) & \tEMDefineEqualsLbrInNaturalMidExqInNaturalLprEqualsaMinusbqRpRb{1}  \\
\end{array}
\]

Sei \(q \in \mathbb{N}\) eine neue, frische Variable, die zuvor nicht im Beweis verwendet wurde. Dann kann auch folgende Regel verwendet werden:
\[
\begin{array}{llll}
    i & (1) & r\in M & ... \\
    2 & (2) & r=a-bq & \rA  \\
    2,j & (3) & Q & ...  \\
    i,j & (4) & Q & \tEMDefineEqualsLbrInNaturalMidExqInNaturalLprEqualsaMinusbqRpRb{1,2,3}  \\
\end{array}
\]

\label{tMDefineEqualsLbrInNaturalMidExqInNaturalLprEqualsaMinusbqRpRbSubseteqNatural}
\begin{lemma}[\(M\subseteq\mathbb{N}\)]
\end{lemma}
\begin{proof}
    \[
	\begin{array}{llcll p{4.3cm}}
            &  (1)  & \multicolumn{3}{l}{M\subseteq \mathbb{N}} & \ImpLbxInAMidPLpxRpRbSubseteqA{} \\
        \end{array}
    \]
\end{proof}

\label{tMDefineEqualsLbrInNaturalMidExqInNaturalLprEqualsaMinusbqRpNotEqualsEmptyset}
\begin{lemma}[\(M\neq\emptyset\)]
\end{lemma}
\begin{proof}
    \[
	\begin{array}{llcll p{4.3cm}}
            &  (1)  &  a&=&a-0 & \aInNaturalImpaMinusZeroEqualsa{} \\
            &  (2)  &   &=&a-b\cdot 0 & \rIE{2,5} \\
            &  (3)  &  a&=&a-b\cdot 0 & \rTransitivityEqRI{5,6} \\
            &  (4)  &  \multicolumn{3}{l}{a\in M} & \tIMDefineEqualsLbrInNaturalMidExqInNaturalLprEqualsaMinusbqRpRb{3} \\
            &  (5)  &  \multicolumn{3}{l}{\exists x(x\in M)} & \rEI{4} \\
            &  (6)  &  \multicolumn{3}{l}{M\neq \emptyset} & \ExxInSImpSNotEqualsEmptyset{5} \\
        \end{array}
    \]
\end{proof}

\label{tnInMDefineEqualsLbrInNaturalMidExqInNaturalLprEqualsaMinusbqRpRbwbLeqnImpnMinusbInM}
\begin{lemma}[\(n\in M,b\leq n\vdash n-b\in M\)]
\end{lemma}
\begin{proof}
Im Beweis nutzen wir das Theorem \(\ImpLpNaturalwMultwOneRpInAbelSemiRing{}\). Sei \(n\in\mathbb{N}\) mit:
    \[
	\begin{array}{llcll p{4.3cm}}
           1 &  (1)  & \multicolumn{3}{l}{n\in M} & \rA \\
           2 &  (2)  & \multicolumn{3}{l}{b\leq n} & \rA \\
           3 &  (3)  & \multicolumn{3}{l}{n=a-bq} & \rA \\
           3 &  (4)  & \multicolumn{3}{l}{bq\leq a} & \minusE{3} \\
           2,3 &  (5)  & \multicolumn{3}{l}{b\leq a-bq} & \rIE{3,2} \\
           2,3 &  (6)  & b+bq&\leq& (a-bq)+bq & \awbwcInNaturalLpaLeqbEqvaPluscLeqbPluscRp{5} \\
           3 &  (7)  & &=& a & \awbInNaturalwbLeqaImpLpaMinusbRpPlusbEqualsa{4} \\
           2,3 &  (8)  & b+bq&\leq & a & \rTransitivityEqRI{6,7} \\
           2 &  (9)  & n-b&=& n-b & \minusI{2} \\
           2,3 &  (10)  &  &=& (a-bq)-b & \rIE{3,9} \\
           2,3 &  (11)  &  &=& a-(bq+b) & \awbwcInNaturalwbPluscLeqaImpaMinusLpbPluscRpEqualsLpaMinusbRpMinusc{8} \\
             &  (12)  &  &=& a-(bq+b\cdot 1) & \rNeutralElementMonoid{} \\
             &  (13)  &  &=& a-(b(q+1)) & \rLeftDistributiveAbelianSemigroup{} \\
            2,3 &  (14)  &  n-b&=& a-(b(q+1)) & \rTransitivityEqRI{9,13} \\
            2,3 &  (15)  &  \multicolumn{3}{l}{n-b\in M} & \tIMDefineEqualsLbrInNaturalMidExqInNaturalLprEqualsaMinusbqRpRb{14} \\
            1,2 &  (16)  &  \multicolumn{3}{l}{n-b\in M} & \tEMDefineEqualsLbrInNaturalMidExqInNaturalLprEqualsaMinusbqRpRb{1,3,15} \\
        \end{array}
    \]
\end{proof}

\label{tExnInMDefineEqualsLbrInNaturalMidExqInNaturalLprEqualsaMinusbqRpRbLpnEqualsMinLpMRp}
\begin{lemma}[\(\exists n\in M(n=\min(M)\)]
\end{lemma}
\begin{proof}
    \[
	\begin{array}{llcll p{4.3cm}}
          &  (1)  & \multicolumn{3}{l}{M\subseteq\mathbb{N}} & \tMDefineEqualsLbrInNaturalMidExqInNaturalLprEqualsaMinusbqRpRbSubseteqNatural{} \\
          &  (2)  & \multicolumn{3}{l}{M\neq\emptyset} & \tMDefineEqualsLbrInNaturalMidExqInNaturalLprEqualsaMinusbqRpNotEqualsEmptyset{} \\
          &  (3)  & \multicolumn{3}{l}{\exists n\in\mathbb{N}(n=\min(M))} & \ASubseteqNaturalwANotEqualsEmptysetImpExnInALpnEqualsMinLpARpRp{1,2} \\
        \end{array}
    \]
\end{proof}

\label{MinLpMDefineEqualsLbrInNaturalMidExqInNaturalLprEqualsaMinusbqRpRbRpInNatural}
\begin{lemma}[\(\min(M)\in\mathbb{N}\)]
\end{lemma}
\begin{proof}
    \[
	\begin{array}{llcll p{4.3cm}}
          &  (1)  & \multicolumn{3}{l}{\exists n\in M(n=\min(M)} & \tExnInMDefineEqualsLbrInNaturalMidExqInNaturalLprEqualsaMinusbqRpRbLpnEqualsMinLpMRp{} \\
          &  (2)  & \multicolumn{3}{l}{\min(M)\in M} & \rMinE{1} \\
          &  (3)  & \multicolumn{3}{l}{\min(M)\in \mathbb{N}} & \inE{2} \\
        \end{array}
    \]
\end{proof}


\label{tbNotEqualsZeroImpMinLpMDefineEqualsLbrInNaturalMidExqInNaturalLprEqualsaMinusbqRpRbRpLneqb}
\begin{lemma}[\(b\neq 0\vdash \min(M)<b\)]
\end{lemma}
\begin{proof}
    \[
	\begin{array}{llcll p{4.3cm}}
        1 &  (1)  & \multicolumn{3}{l}{b\neq 0} & \rA \\
          &  (2)  & \multicolumn{3}{l}{\exists n\in\mathbb{N}(n=\min(M))} & \tExnInMDefineEqualsLbrInNaturalMidExqInNaturalLprEqualsaMinusbqRpRbLpnEqualsMinLpMRp{} \\
         3 &  (3)  & \multicolumn{3}{l}{b\leq min(M)} & \rA \\
           &  (4)  & \multicolumn{3}{l}{\min(M)\in M} & \rMinE{2} \\
         3 &  (5)  & \multicolumn{3}{l}{\min(M)-b\in M} & \tnInMDefineEqualsLbrInNaturalMidExqInNaturalLprEqualsaMinusbqRpRbwbLeqnImpnMinusbInM{3,4} \\
         3 &  (6)  & \multicolumn{3}{l}{\min(M)-b\leq \min(M)} & \awbInNaturalaLeqbImpbMinusaLeqb{5,3} \\
           &  (7)  & \multicolumn{3}{l}{\min(M)\leq \min(M)-b} & \rMinE{2} \\
         3 &  (8)  & \multicolumn{3}{l}{\min(M)=\min(M)-b} & \rAntisymmetryOrdRI{6,7} \\
         3 &  (9)  & \multicolumn{3}{l}{b=0} & \awbInNaturalwaMinusbEqualsaImpbEqualsZero{8} \\
         1,3 &  (10)  & \multicolumn{3}{l}{\bot} & \rBI{1,9} \\
         1 &  (11)  & \multicolumn{3}{l}{\neg(b\leq\min(M))} & \rBI{1,9} \\
         1 &  (12)  & \multicolumn{3}{l}{\min(M)<b} & \aLneqbEqvnLpbLeqaRp{11} \\

        \end{array}
    \]
\end{proof}


\label{tExqInNaturalLpaEqualsbqPlusMinLpMDefineEqualsLbrInNaturalMidExqInNaturalLprEqualsaMinusbqRpRbRpRp}
\begin{lemma}[\(\exists q\in\mathbb{N}(a=bq+\min(M))\)]
\end{lemma}
\begin{proof}
\(\ImpLpNaturalwPluswZeroRpInAbelMonoid{}\) und daher gilt:
    \[
	\begin{array}{llcll p{4.3cm}}
        1 &  (1)  & \multicolumn{3}{l}{\exists n\in M(n=\min(M))} & \tExnInMDefineEqualsLbrInNaturalMidExqInNaturalLprEqualsaMinusbqRpRbLpnEqualsMinLpMRp{} \\
        1 &  (2)  & \multicolumn{3}{l}{\min(M)\in \mathbb{N}} & \MinLpMDefineEqualsLbrInNaturalMidExqInNaturalLprEqualsaMinusbqRpRbRpInNatural{1} \\
        3 &  (3)  & \multicolumn{3}{l}{\min(M)=a-bq} & \rA \\
        1,3 &  (4)  & \multicolumn{3}{l}{a-bq\in\mathbb{N}} & \rIE{3,2} \\
        1,3 &  (5)  & \multicolumn{3}{l}{bq\leq a} & \minusE{4} \\
          &  (6)  & bq+\min(M)&=&\min(M)+bq & \rCommutativeMonoid{} \\
        3 &  (7)  &  &=&(a-bq)+bq & \aEqualsbEqvaPluscEqualsbPlusc{3} \\
        1,3 &  (8)  & &=&a & \awbInNaturalwbLeqaImpLpaMinusbRpPlusbEqualsa{5} \\
        1,3 &  (9)  & bq+\min(M)&=&a & \rTransitivityEqRI{7,8} \\
        1,3 &  (10)  & a&=&bq+\min(M) & \rSymmetryEqRI{9} \\
        1,3 &  (11)  & \multicolumn{3}{l}{\exists q\in\mathbb{N}(a=bq+\min(M))} & \rEI{10} \\
        1 &  (12)  & \multicolumn{3}{l}{\exists q\in\mathbb{N}(a=bq+\min(M))} & \tEMDefineEqualsLbrInNaturalMidExqInNaturalLprEqualsaMinusbqRpRb{11} \\

        \end{array}
    \]
\end{proof}

\label{bNotEqualsZeroImpExrInNaturalLprLneqbAndExqInNaturalLpaEqualsbqPlusrRpRp}
\begin{theorem}[\(b\neq 0\vdash \exists r\in\mathbb{N}(r<b\land \exists q\in\mathbb{N}(a=bq+r))\)]
\end{theorem}
\begin{proof}
    \[
	\begin{array}{llcll p{4.3cm}}
        1 &  (1)  & \multicolumn{3}{l}{b\neq 0} & \rA \\
          &  (2)  & \multicolumn{3}{l}{\exists q\in\mathbb{N}(a=bq+\min(M))} & \tExqInNaturalLpaEqualsbqPlusMinLpMDefineEqualsLbrInNaturalMidExqInNaturalLprEqualsaMinusbqRpRbRpRp{} \\
        1 &  (3)  & \multicolumn{3}{l}{\min(M)<b} & \tbNotEqualsZeroImpMinLpMDefineEqualsLbrInNaturalMidExqInNaturalLprEqualsaMinusbqRpRbRpLneqb{1} \\
        1 &  (4)  & \multicolumn{3}{l}{\min(M)<b\land \exists q\in\mathbb{N}(a=bq+\min(M))} & \rAI{3,2} \\
        1 &  (5)  & \multicolumn{3}{l}{\min(M)<b\land \exists q\in\mathbb{N}(a=bq+\min(M))} & \rAI{3,2} \\
          &  (6)  & \multicolumn{3}{l}{\min(M)\in\mathbb{N}} & \MinLpMDefineEqualsLbrInNaturalMidExqInNaturalLprEqualsaMinusbqRpRbRpInNatural{} \\
        1 &  (7)  & \multicolumn{3}{l}{\exists r\in\mathbb{N}(r<b\land \exists q\in\mathbb{N}(a=bq+r))} & \rSetEIa{6,5} \\
        \end{array}
    \]
\end{proof}


\section{Eindeutigkeit der Division mit Rest}


\label{awbwcwdwfInNaturalwaLeqcwbLeqdImpfLpcMinusaRpPlusLpdMinusbRpEqualsLpfcPlusdRpMinusLpfaPlusbRp}
\begin{theorem}[\(a,b,c,d,f\in\mathbb{N},a\leq c, b\leq d\vdash f(c-a)+(d-b)=(fc+d)-(fa+b)\)]
\end{theorem}
\begin{proof}
Seien \(a,b,c,d,f\in\mathbb{N}\).
    \[
	\begin{array}{llcll p{4.3cm}}
        1       &  (1)  & \multicolumn{3}{l}{a\leq c} & \rA \\
        2       &  (2)  & \multicolumn{3}{l}{b\leq d} & \rA \\
        1       &  (3)  & \multicolumn{3}{l}{fa\leq fd} & \awbwcInNaturalImpaLeqbImpcaLeqcb{1} \\
        1       &  (4)  & f(c-a)+(d-b)&=&(fc-fa)+(d-b) & \awbwcInNaturalwcLeqbImpaLpbMinuscRpEqualsabMinusac{1} \\
        1,2       &  (5)  &             &=&(fc+d)-(fa+b) & \awbwcwdInNaturalwaLeqcwbLeqdImpLpcMinusaRpPlusLpdMinusbRpEqualsLpcPlusdRpMinusLpaPlusbRp{3,2} \\
        1,2       &  (6)  & f(c-a)+(d-b)&=&(fc+d)-(fa+b) & \rTransitivityEqRI{4,5} \\
        \end{array}
    \]
\end{proof}

\label{awbwcwdwfInNaturalwfNotEqualsZerowfLpcMinusaRpEqualsZeroAndLpdMinusbRpEqualsZeroImpcEqualsaAnddEqualsb}
\begin{theorem}[\(a,b,c,d,f\in\mathbb{N},f\neq 0, f(c-a)=0\land (d-b)=0 \vdash c=a\land d=b\)]
\end{theorem}
\begin{proof}
Seien \(a,b,c,d,f\in\mathbb{N}\).
    \[
	\begin{array}{llclll}
        1       &  (1)  & \multicolumn{3}{l}{f(c-a)=0\land (d-b)=0} & \rA \\
        2       &  (2)  & \multicolumn{3}{l}{f\neq 0} & \rA \\
        1       &  (3)  & \multicolumn{3}{l}{f(c-a)=0} & \rAEa{1} \\
        1       &  (4)  & \multicolumn{3}{l}{d-b=0} & \rAEb{1} \\
        1       &  (5)  & \multicolumn{3}{l}{d=b} & \awbInNaturalLpaEqualsbEqvaMinusbEqualsZeroRp{3} \\
        1,2     &  (6)  & \multicolumn{3}{l}{c-a=0} & \awbInNaturalwaNotEqualsZerowabEqualsZeroImpbEqualsZero{2,3} \\
        1,2     &  (7)  & \multicolumn{3}{l}{c=a} & \awbInNaturalLpaEqualsbEqvaMinusbEqualsZeroRp{6} \\
        1,2     &  (8)  & \multicolumn{3}{l}{c=a\land d=b} & \rAI{5,7} \\
        \end{array}
    \]
\end{proof}

\label{awbwcwdwfInNaturalwcLeqawbLeqdwLpfaPlusbEqualsfcPlusdEqvfLpaMinuscRpEqualsdMinusbRp}
\begin{theorem}[\(fa+b=fc+d \dashv\vdash f(a-c)=d-b\)]
Seien \(a,b,c,d,f\in\mathbb{N}\), \(c\leq a\) und \(b\leq d\). Dann gilt:
\[fa+b=fc+d \dashv\vdash f(a-c)=d-b.\]
\end{theorem}
\begin{proof}
Seien \(a,b,c,d,f\in\mathbb{N}\).
\(\vdash\):
    \[
	\begin{array}{llcll p{5cm}}
        1       &  (1)  & \multicolumn{3}{l}{c\leq a} & \rA \\
        2       &  (2)  & \multicolumn{3}{l}{b\leq d} & \rA \\
        3       &  (3)  & \multicolumn{3}{l}{fa+b=fc+d} & \rA \\
                &  (4)  & f(a-c)&=&fa-fc & \rLeftDistributiveSemigroup{} \\
        1,2,3   &  (5)  &       &=&d-b & \awbwcwdInNaturalwcLeqawbLeqdLpaPlusbEqualscPlusdEqvaMinuscEqualsdMinusbRp{1,2,3} \\
        1,2,3   &  (6)  &  f(a-c)&=&d-b & \rTransitivityEqRI{4,5} \\
        \end{array}
    \]
\(\dashv\):
    \[
	\begin{array}{llcll p{5cm}}
        1       &  (1)  & \multicolumn{3}{l}{c\leq a} & \rA \\
        2       &  (2)  & \multicolumn{3}{l}{b\leq d} & \rA \\
                &  (3)  & fa-fc&=&f(a-c) & \rLeftDistributiveSemigroup{} \\
        4       &  (4)  &       &=&d-b & \rA \\
        4       &  (5)  & fa-fc &=&d-b & \rTransitivityEqRI{3,4} \\
        1,2,4   &  (6)  & \multicolumn{3}{l}{fa+b=fc+d} & \awbwcwdInNaturalwcLeqawbLeqdLpaPlusbEqualscPlusdEqvaMinuscEqualsdMinusbRp{1,2,5} \\
        \end{array}
    \]
\end{proof}

\label{awbwcwdwfInNaturalwfLpaMinusbRpEqualsLpcMinusdRpwLpcMinusdRpLneqfImpaEqualsbAndcEqualsd}
\begin{theorem}[\(a,b,c,d,f\in\mathbb{N}, f(a-b)=(c-d), (c-d)<f\vdash a=b\land c=d\)]
\end{theorem}
\begin{proof}
Seien \(a,b,c,d,f\in\mathbb{N}\).
    \[
	\begin{array}{llcll p{5cm}}
        1       &  (1)  & \multicolumn{3}{l}{f(a-b)=(c-d)} & \rA \\
        2       &  (2)  & \multicolumn{3}{l}{(c-d)<f} & \rA \\
        1,2     &  (3)  & \multicolumn{3}{l}{f(a-b)<f} & \rIE{2,1} \\
        1,2     &  (4)  & \multicolumn{3}{l}{(a-b)<1} & \awbwcInNaturalwcNotEqualsZerowcaLeqcbImpaLeqb{3} \\
        1,2     &  (5)  & \multicolumn{3}{l}{a-b=0} & \aInNaturalLpaLneqOneEqvaEqualsZeroRp{4} \\
        1,2     &  (6)  & \multicolumn{3}{l}{a=b} & \awbInNaturalLpaEqualsbEqvaMinusbEqualsZeroRp{5} \\
        1     &  (7)  & c-d&=&f(a-b) & \rSymmetryEqRI{1} \\
        1,2   &  (8)  &    &=&f\cdot 0 & \rIE{5,7} \\
              &  (9)  &    &=&0 & \aInNaturalImpZeroEqualsZeroMulta{} \\
        1,2   &  (9)  & c-d&=&0 & \rTransitivityEqRI{7,9} \\
        1,2   &  (10) &\multicolumn{3}{l}{c=d} & \awbInNaturalLpaEqualsbEqvaMinusbEqualsZeroRp{9} \\
        1,2   &  (11) &\multicolumn{3}{l}{a=b\land c=d} & \rAI{6,10} \\
        \end{array}
    \]
\end{proof}


\label{awbInNaturalwbNotEqualsZerowExqSubOnewrSubOnewqSubTwowrSubTwoInNaturalLpaEqualsbMultqSubOnePlusrSubOneAndrSubOneLneqbAndaEqualsbMultqSubTwoPlusrSubTwoAndrSubTwoLneqbRpImpqSubOneEqualsqSubTwoAndrSubOneEqualsrSubTwo}
\begin{theorem}[
\(
a,b\in\mathbb{N},b\neq 0,\exists q_1, r_1, q_2, r_2 \in \mathbb{N} \,
\big(
a = b \cdot q_1 + r_1 \land r_1 < b \land 
\)
\newline
\(
a = b \cdot q_2 + r_2 \land r_2 < b
\big)
\vdash q_1 = q_2 \land r_1 = r_2
\)
(Eindeutigkeit der Division mit Rest)]
Seien \( a, b \in \mathbb{N} \) mit \( b \neq 0 \), und es gelte:
\[
a = b \cdot q_1 + r_1, \quad r_1 < b,
\]
sowie
\[
a = b \cdot q_2 + r_2, \quad r_2 < b,
\]
für \( q_1, r_1, q_2, r_2 \in \mathbb{N} \). Dann folgt, dass \( q_1 = q_2 \) und \( r_1 = r_2 \).
\end{theorem}
\begin{proof}
Seien \(a,b\in\mathbb{N}\). Wir wählen \( q_1, r_1, q_2, r_2 \in \mathbb{N} \) so, dass folgende Annahmen gelten. Im Beweis nutzen wir das Theorem \(\LeqIsTotalOrderOnNaturalNumbers{}\)
    \[
	\begin{array}{llcll p{5cm}}
        1       &  (1)  & \multicolumn{3}{l}{a=bq_1+r_1} & \rA \\
        2       &  (2)  & \multicolumn{3}{l}{r_1<b} & \rA \\
        3       &  (3)  & \multicolumn{3}{l}{a=bq_2+r_2} & \rA \\
        4       &  (4)  & \multicolumn{3}{l}{r_2<b} & \rA \\
        5       &  (5)  & \multicolumn{3}{l}{b\neq 0} & \rA \\
        1,3     &  (6)  & \multicolumn{3}{l}{bq_1+r_1=bq_2+r_2} & \rIE{1,3} \\
                &  (7)  & \multicolumn{3}{l}{r_1\leq r_2\lor r_2\leq r_1} & \rTotalityOrdRI{} \\
                &  (8)  & \multicolumn{3}{l}{q_1\leq q_2\lor q_2\leq q_1} & \rTotalityOrdRI{} \\
        9       &  (9)  & \multicolumn{3}{l}{r_1\leq r_2} & \rA \\
        10      &  (10)  & \multicolumn{3}{l}{q_1\leq q_2} & \rA \\
        1,3     &  (11)  & 0&=&b\cdot q_2+r_2-(b\cdot q_1+r_1) & \awbInNaturalLpaEqualsbEqvaMinusbEqualsZeroRp{6}  \\
        9,10    &  (12)  & &=& b(q_2-q_1)+(r_2-r_1) & \awbwcwdwfInNaturalwaLeqcwbLeqdImpfLpcMinusaRpPlusLpdMinusbRpEqualsLpfcPlusdRpMinusLpfaPlusbRp{10,9}  \\
        1,3,9,10    &  (13)  & 0 &=& b(q_2-q_1)+(r_2-r_1) & \rTransitivityEqRI{11,12}  \\
        1,3,5,9,10  &  (14)  & \multicolumn{3}{l}{q_1=q_2\land r_1=r_2} & \awbwcwdwfInNaturalwfNotEqualsZerowfLpcMinusaRpEqualsZeroAndLpdMinusbRpEqualsZeroImpcEqualsaAnddEqualsb{5,13}  \\
        15    &  (15)  & \multicolumn{3}{l}{q_2\leq q_1} & \rA  \\
        1,3    &  (16)  & \multicolumn{3}{l}{bq_2+r_2=bq_1+r_1} & \rSymmetryEqRI{6}  \\
        1,3,9,15  &  (17)  & \multicolumn{3}{l}{b(q_1-q_2)=r_2-r_1} & \awbwcwdwfInNaturalwcLeqawbLeqdwLpfaPlusbEqualsfcPlusdEqvfLpaMinuscRpEqualsdMinusbRp{9,15,16}  \\
        4,9  &  (18)  & \multicolumn{3}{l}{r_2-r_1<b} & \bLneqcwaLeqbImpbMinusaLneqc{4,9}  \\
        1,3,4,9,15  &  (19)  & \multicolumn{3}{l}{q_1=q_2\land r_2=r_1} & \awbwcwdwfInNaturalwfLpaMinusbRpEqualsLpcMinusdRpwLpcMinusdRpLneqfImpaEqualsbAndcEqualsd{17,19}  \\
        1,3,4,5,9  &  (20)  & \multicolumn{3}{l}{q_1=q_2\land r_2=r_1} & \rOE{8,10,14,15,20}  \\ 
        21  &  (21)  & \multicolumn{3}{l}{r_2\leq r_1} & \rA  \\ 
        22  &  (22)  & \multicolumn{3}{l}{q_1\leq q_2} & \rA  \\
        1,3,21,22  &  (23)  & \multicolumn{3}{l}{b(q_2-q_1)=r_1-r_2} & \awbwcwdwfInNaturalwcLeqawbLeqdwLpfaPlusbEqualsfcPlusdEqvfLpaMinuscRpEqualsdMinusbRp{22,21,16}  \\ 
        2,21  &  (24)  & \multicolumn{3}{l}{r_1-r_2<b} & \bLneqcwaLeqbImpbMinusaLneqc{2,21}  \\ 
        1,2,3,21,22  &  (25)  & \multicolumn{3}{l}{q_1=q_2\land r_2=r_1} & \awbwcwdwfInNaturalwfLpaMinusbRpEqualsLpcMinusdRpwLpcMinusdRpLneqfImpaEqualsbAndcEqualsd{23,24}  \\
        26  &  (26)  & \multicolumn{3}{l}{q_2\leq q_1} & \rA  \\
        1,3     &  (27)  & 0&=&b\cdot q_1+r_1-(b\cdot q_2+r_2) & \awbInNaturalLpaEqualsbEqvaMinusbEqualsZeroRp{6}  \\
        21,26    &  (28)  & &=& b(q_1-q_2)+(r_1-r_2) & \awbwcwdwfInNaturalwaLeqcwbLeqdImpfLpcMinusaRpPlusLpdMinusbRpEqualsLpfcPlusdRpMinusLpfaPlusbRp{21,26}  \\
        1,3,21,26    &  (29)  & 0 &=& b(q_2-q_1)+(r_2-r_1) & \rTransitivityEqRI{11,12}  \\
        1,3,5,21,26  &  (30)  & \multicolumn{3}{l}{q_1=q_2\land r_1=r_2} & \awbwcwdwfInNaturalwfNotEqualsZerowfLpcMinusaRpEqualsZeroAndLpdMinusbRpEqualsZeroImpcEqualsaAnddEqualsb{5,29}  \\
        1,2,3,5,21  &  (31)  & \multicolumn{3}{l}{q_1=q_2\land r_1=r_2} & \rOE{8,22,25,26,30}  \\
        1,2,3,4,5  &  (32)  & \multicolumn{3}{l}{q_1=q_2\land r_1=r_2} & \rOE{7,9,20,21,31}  \\
        \end{array}
    \]
\end{proof}

\section{Eindeutig bestimmte Zahlen der Division mit Rest}

In diesem Abschnitt seien \( a, b \in \mathbb{N} \).

\begin{definition}[Division mit Rest]
Seien \( b \neq 0 \). Die Division mit Rest liefert eindeutig bestimmte Zahlen \( a \div b \) und \( a \bmod b \), wobei gilt:
\[
a = b \cdot (a \div b) + (a \bmod b) \quad \text{und} \quad (a \bmod b) < b.
\]
Diese Zahlen sind definiert durch:
\[
a \div b \coloneqq \iota k \, (k \in \mathbb{N} \land a - b \cdot k \in \mathbb{N} \land a - b \cdot k < b)
\]
und
\[
a \bmod b \coloneqq a - b \cdot (a \div b).
\]
\end{definition}

\paragraph{Beweisregeln für die Division mit Rest}
\label{rule:rDivisionWithRemainderI} 

Basierend auf der Definition der Division mit Rest können folgende Regeln formuliert werden:

\[
\begin{array}{llll}
    i & (1) & b \neq 0 & ... \\
    i & (2) & a = b \cdot (a \div b) + (a \bmod b)  & \rDivisionWithRemainderI{1} \\
    i & (3) & (a \bmod b) < b & \rDivisionWithRemainderI{1} \\
    i & (4) & (a \bmod b)=a-b \cdot (a \div b) & \rDivisionWithRemainderI{1} \\
\end{array}
\]

\[
\begin{array}{llll}
    i & (1) & b \neq 0 & ... \\
    j & (2) & a = bq + r  & ... \\
    i & (3) & a \bmod b = r & \rDivisionWithRemainderI{1,2} \\
    i & (4) & a \div b=q & \rDivisionWithRemainderI{1,2} \\
\end{array}
\]

Dabei sind \( i \) und \( j \) Listen von Annahmen.


\subsection{Grundlegende Eigenschaften}



\subsubsection{Division mit Rest bei kleinerem Dividend}
\label{awbInNaturalwbNotEqualsZerowaLneqbImpaDivbEqualsZero}
\begin{theorem}[\(a,b\in\mathbb{N},b\neq 0, a<b\vdash a\div b=0\)]
\end{theorem}
\begin{proof}
    \[
	\begin{array}{llcll p{5cm}}
            1 &  (1)  & \multicolumn{3}{l}{a<b} & \rA \\
            2 &  (2)  & \multicolumn{3}{l}{b\neq 0} & \rA \\
            3 &  (3)  & \multicolumn{3}{l}{a\div b\neq 0} & \rA \\
            2 &  (4)  & \multicolumn{3}{l}{ a = b \cdot (a \div b) + (a\bmod b)} & \rDivisionWithRemainderI{2} \\
            3 &  (5)  & a&\leq &a\cdot (a \div b)   & \awbInNaturalwbNotEqualsZeroImpaLeqab{3} \\
            3 &  (6)  &  &< &b\cdot (a \div b)   & \awbwcInNaturalwcNotEqualsZerowacLneqbcEqvaLneqb{1} \\
            3 &  (7)  &  &< &b\cdot (a \div b)   & \awbwcInNaturalwcNotEqualsZerowacLneqbcEqvaLneqb{1} \\
            1 &  (8)  &  &\leq &b\cdot (a \div b)+(a\bmod b)   & \awbInNaturalImpaLeqaPlusb{} \\
            1,3 &  (9)  & a &< &b\cdot (a \div b)+(a\bmod b)   & \rTransitivityOrdRI{5,8} \\
            1,3 &  (10)  & \multicolumn{3}{l}{a\neq b\cdot (a \div b)+(a\bmod b)}   & \rStrictOrderRelationI{9} \\
            1,2,3 &  (11)  & \multicolumn{3}{l}{\bot}   & \rBI{4,10} \\
            1,2 &  (12)  & \multicolumn{3}{l}{a\div b=0}   & \rCE{3,11} \\
        \end{array}
    \]
\end{proof}

\label{awbInNaturalwbNotEqualsZerowaLneqbImpaEqualsaModb}
\begin{theorem}[\(a,b\in\mathbb{N},b\neq 0, a<b\vdash a\bmod b=a\)]
\end{theorem}
\begin{proof}
\(\ImpLpNaturalwPluswZeroRpInMonoid{}\) und daher gilt:
    \[
	\begin{array}{llcll p{5cm}}
            1 &  (1)  & \multicolumn{3}{l}{a<b} & \rA \\
            2 &  (2)  & \multicolumn{3}{l}{b\neq 0} & \rA \\
            2 &  (3)  & \multicolumn{3}{l}{ a = b \cdot (a \div b) + (a\bmod b)} & \rDivisionWithRemainderI{2} \\
            1,2 &  (4)  & \multicolumn{3}{l}{(a \div b)=0} & \awbInNaturalwbNotEqualsZerowaLneqbImpaDivbEqualsZero{2,1} \\
            2 &  (5)  & a &=& b \cdot (a \div b) + (a\bmod b) & \rDivisionWithRemainderI{2} \\
            1,2 &  (6)  &   &=& b \cdot 0 + (a\bmod b) & \rIE{4,5} \\
              &  (7)  &   &=& 0 + (a\bmod b) & \aInNaturalImpZeroEqualsZeroMulta{} \\
              &  (8)  &   &=& a\bmod b & \rNeutralElementMonoid{} \\
            1,2 &  (9)  &  a &=& a\bmod b & \rTransitivityEqRI{5,8} \\
            1,2 &  (10)  &  a\bmod b &=& a & \rSymmetryEqRI{9} \\
        \end{array}
    \]
\end{proof}


\label{awbInNaturalwbNotEqualsZeroLpaLneqbEqvaDivbEqualsZeroAndaEqualsaModbRp}
\begin{theorem}[\(a<b\dashv\vdash a\div b=0\land  a\bmod b=a\)]
Seien \(a,b\in\mathbb{N}\) und \(b\neq 0\), dann gilt
\[a<b\dashv\vdash a\div b=0\land  a=a\bmod b\]
\end{theorem}
\begin{proof}
\(\vdash\):
    \[
	\begin{array}{llcll p{5cm}}
            1 &  (1)  & \multicolumn{3}{l}{a<b} & \rA \\
            2 &  (2)  & \multicolumn{3}{l}{b\neq 0} & \rA \\
            1,2 &  (3)  & \multicolumn{3}{l}{ a\bmod b=a} & \awbInNaturalwbNotEqualsZerowaLneqbImpaEqualsaModb{2,1} \\
            1,2 &  (4)  & \multicolumn{3}{l}{a\div b=0} & \awbInNaturalwbNotEqualsZerowaLneqbImpaDivbEqualsZero{2,1} \\
            1,2 &  (5)  &  \multicolumn{3}{l}{a\div b=0\land a\bmod b=a} & \rAI{4,3} \\
        \end{array}
    \]
\(\dashv\):
    \[
	\begin{array}{llcll p{5cm}}
            1 &  (1)  & \multicolumn{3}{l}{b\neq 0} & \rA \\
            2 &  (2)  & \multicolumn{3}{l}{ a\div b=0\land a\bmod b=a} & \rA \\
            2 &  (4)  & \multicolumn{3}{l}{ a\bmod b=a} & \rAEb{2} \\
            1 &  (5)  & \multicolumn{3}{l}{a\bmod b<b} & \rDivisionWithRemainderI{1} \\
            1,2 &  (6)  & \multicolumn{3}{l}{a<b} & \rIE{4,5} \\
        \end{array}
    \]
\end{proof}

\subsubsection{Ordnungsrelationen bei der Division}

\label{awbInNaturalwbNotEqualsZeroImpaDivbLeqa}
\begin{theorem}[\(a,b\in\mathbb{N}, b\neq 0\vdash a\div b\leq a\)]
\end{theorem}
\begin{proof}
Seien \(a,b\in\mathbb{N}\), dann gilt:
    \[
	\begin{array}{llcll p{4.5cm}}
             1 &  (1)  & \multicolumn{3}{l}{b\neq 0} & \rA \\
             1 &  (2)  & a\div b&\leq& b\cdot (a\div b) & \awbInNaturalwbNotEqualsZeroImpaLeqba{1} \\
               &  (3)  &     &\leq& b\cdot (a\div b)+(a\bmod b) & \awbInNaturalImpaLeqaPlusb{} \\
             1 &  (4)  &     &=& a & \rDivisionWithRemainderI{1} \\
             1 &  (5)  & a\div b &\leq & a & \rTransitivityOrdRI{2,4} \\
        \end{array}
    \]
\end{proof}

\label{awbInNaturalwbGneqOneImpaDivbLneqa}
\begin{theorem}[\(a,b\in\mathbb{N}, b > 1\vdash a\div b<a\)]
\end{theorem}
\begin{proof}
Seien \(a,b\in\mathbb{N}\), dann gilt:
    \[
	\begin{array}{llcll p{4.5cm}}
             1 &  (1)  & \multicolumn{3}{l}{b > 1} & \rA \\
             1 &  (2)  & \multicolumn{3}{l}{b \neq 0} & \aInNaturalwaGneqOneImpaNotEqualsZero{1} \\
             1 &  (3)  & a\div b&<& b\cdot (a\div b) & \awbInNaturalwbGneqOneImpaLneqba{1} \\
               &  (4)  &     &\leq& b\cdot (a\div b)+(a\bmod b) & \awbInNaturalImpaLeqaPlusb{} \\
             1 &  (5)  &     &=& a & \rDivisionWithRemainderI{2} \\
             1 &  (6)  & a\div b &< & a & \rTransitivityOrdRI{2,5} \\
        \end{array}
    \]
\end{proof}

\subsubsection{Teilbarkeit und Rest}

\label{awbInNaturalwaNotEqualsZerowaMidbEqvbDivaEqualsLpbRpDurchLpaRpAndbModaEqualsZero}
\begin{theorem}[\(a\mid b\dashv\vdash b\div a=\frac{b}{a}\land b\bmod a=0\)]
Seien \(a,b\in\mathbb{N}\) und \(a\neq 0\), dann gilt
\[a\mid b\dashv\vdash b\div a=\frac{b}{a}\land b\bmod a=0\]
\end{theorem}
\begin{proof}
\(\ImpLpNaturalwPluswZeroRpInAbelMonoid{}\) und daher gilt:
\(\vdash\):
    \[
	\begin{array}{llcll p{5cm}}
            1 &  (1)  & \multicolumn{3}{l}{a\mid b} & \rA \\
            1 &  (2)  & b&=&a\cdot\frac{b}{a} & \rDivisionI{1} \\
            1 &  (3)  &  &=&a\cdot\frac{b}{a}+0 & \rNeutralElementMonoid{} \\
            1 &  (4)  & b &=&a\cdot\frac{b}{a}+0 & \rTransitivityEqRI{2,3} \\
            1 &  (5)  &  \multicolumn{3}{l}{a\neq 0} & \rDivisibilityE{1} \\
            1,2 &  (6)  & \multicolumn{3}{l}{b\div a=\frac{b}{a}} & \rDivisionWithRemainderI{5,4} \\
            1,2 &  (7)  & \multicolumn{3}{l}{b\bmod a=0} & \rDivisionWithRemainderI{5,4} \\
            1,2 &  (8)  &  \multicolumn{3}{l}{b\div a=\frac{b}{a}\land a\bmod b=0} & \rAI{6,7} \\
        \end{array}
    \]
\(\dashv\):
    \[
	\begin{array}{llcll p{5cm}}
            1 &  (1)  & \multicolumn{3}{l}{a\neq 0} & \rA \\
            2 &  (2)  & \multicolumn{3}{l}{ b\div a=\frac{b}{a}\land b\bmod a=0} & \rA \\
            2 &  (3)  & \multicolumn{3}{l}{ b\div a=\frac{b}{a}} & \rAEa{2} \\
            2 &  (4)  & \multicolumn{3}{l}{ b\bmod a=0} & \rAEb{2} \\
            1 &  (5)  & b&=&a\cdot (b\div a)+ (b\bmod a) & \rDivisionWithRemainderI{1} \\
            1,2 &  (6)  &  &=&a\cdot (b\div a)+ 0 & \rIE{4,5} \\
            1 &  (7)  &  &=&a\cdot (b\div a) & \rNeutralElementMonoid{} \\
            1,2 &  (8)  &  &=&a\cdot \frac{b}{a} & \rIE{3,7} \\
            1,2 &  (9)  &  b&=&a\cdot \frac{b}{a} & \rTransitivityEqRI{5,8} \\
            1,2 &  (10)  &  b&=&a\cdot \frac{b}{a} & \rDivisibilityI{1,9} \\
        \end{array}
    \]
\end{proof}

\label{awbInNaturalwaMidbImpbDivaEqualsLpbRpDurchLpaRp}
\begin{theorem}[\(a,b\in\mathbb{N},a\mid b\vdash b\div a=\frac{b}{a}\)]
\end{theorem}
\begin{proof}
    \[
	\begin{array}{llcll p{5cm}}
            1 &  (1)  & \multicolumn{3}{l}{a\mid b} & \rA \\
            1 &  (2)  & \multicolumn{3}{l}{b\neq 0} & \rDivisibilityE{1} \\
            1 &  (3)  & \multicolumn{3}{l}{b\div a=\frac{b}{a}\land b\bmod a=0} & \awbInNaturalwaNotEqualsZerowaMidbEqvbDivaEqualsLpbRpDurchLpaRpAndbModaEqualsZero{1,2} \\
            1 &  (4)  & \multicolumn{3}{l}{b\div a=\frac{b}{a}} & \rAEa{3} \\
        \end{array}
    \]
\end{proof}

\label{awbInNaturalwaMidbImpbModaEqualsZero}
\begin{theorem}[\(a,b\in\mathbb{N},a\mid b\vdash b\bmod a=0\)]
\end{theorem}
\begin{proof}
    \[
	\begin{array}{llcll p{5cm}}
            1 &  (1)  & \multicolumn{3}{l}{a\mid b} & \rA \\
            1 &  (2)  & \multicolumn{3}{l}{a\neq 0} & \rA \\
            1 &  (3)  & \multicolumn{3}{l}{b\div a=\frac{b}{a}\land b\bmod a=0} & \awbInNaturalwaNotEqualsZerowaMidbEqvbDivaEqualsLpbRpDurchLpaRpAndbModaEqualsZero{1,2} \\
            1 &  (4)  & \multicolumn{3}{l}{b\bmod a=0} & \rAEb{3} \\
        \end{array}
    \]
\end{proof}

\subsubsection{Invarianz des Restes bei Vielfachen des Divisors}

\label{awbwkInNaturalwbNotEqualsZeroImpLpaPlusbkRpModbEqualsaModb}
\begin{theorem}[\(a,b,k\in\mathbb{N},b\neq 0\vdash (a+bk)\bmod b=a\bmod b\)]
\end{theorem}
\begin{proof}
Seien \(a,b,k\in\mathbb{N}\). \(\ImpLpNaturalwMultwOneRpInAbelSemiRing{}\) und daher gilt:
    \[
	\begin{array}{llcll p{5cm}}
            1 &  (1)  & \multicolumn{3}{l}{b\neq 0} & \rA \\
            1 &  (2)  & \multicolumn{3}{l}{a=b\cdot (a\div b)+(a\bmod b)} & \rDivisionWithRemainderI{1} \\
            1 &  (3)  & a+bk&=&(b\cdot (a\div b)+(a\bmod b))+bk & \rIE{2,3} \\
              &  (4)  &     &=&(b\cdot (a\div b)+bk)+(a\bmod b) & \aInMwbInMwcInMImpLpaPlusbRpPluscEqualsLpaPluscRpPlusb{} \\
              &  (5)  &     &=&(b\cdot ((a\div b)+k)+(a\bmod b) & \rLeftDistributiveAbelianSemigroup{} \\
            1 &  (6)  & a+bk &=&(b\cdot ((a\div b)+k)+(a\bmod b) & \rTransitivityEqRI{3,5} \\
            1 &  (7)  & (a+bk)\bmod b &=&a\bmod b & \rDivisionWithRemainderI{1,6} \\
        \end{array}
    \]
\end{proof}

\label{awbwkInNaturalwbNotEqualsZeroImpLpaPlusbkRpDivbEqualsLpaDivbRpPlusk}
\begin{theorem}[\(a,b,k\in\mathbb{N},b\neq 0\vdash (a+bk)\div b=(a\div b)+k\)]
\end{theorem}
\begin{proof}
Seien \(a,b,k\in\mathbb{N}\). \(\ImpLpNaturalwMultwOneRpInAbelSemiRing{}\) und daher gilt:
    \[
	\begin{array}{llcll p{5cm}}
            1 &  (1)  & \multicolumn{3}{l}{b\neq 0} & \rA \\
            1 &  (2)  & \multicolumn{3}{l}{a=b\cdot (a\div b)+(a\bmod b)} & \rDivisionWithRemainderI{1} \\
            1 &  (3)  & a+bk&=&(b\cdot (a\div b)+(a\bmod b))+bk & \rIE{2,3} \\
              &  (4)  &     &=&(b\cdot (a\div b)+bk)+(a\bmod b) & \aInMwbInMwcInMImpLpaPlusbRpPluscEqualsLpaPluscRpPlusb{} \\
              &  (5)  &     &=&(b\cdot ((a\div b)+k)+(a\bmod b) & \rLeftDistributiveAbelianSemigroup{} \\
            1 &  (6)  & a+bk &=&(b\cdot ((a\div b)+k)+(a\bmod b) & \rTransitivityEqRI{3,5} \\
            1 &  (7)  & (a+bk)\div b &=&(a\div b)+k & \rDivisionWithRemainderI{1,6} \\
        \end{array}
    \]
\end{proof}

\chapter{Summen- und Produktzeichen}

\begin{definition}[Summe über eine endliche Indexmenge]
Sei \(\bigl(S,+,0\bigr)\) ein abelsches Monoid. Für eine endliche Indexmenge \(I\) mit \(\lvert I \rvert = n\) und eine Familie \(\{s_i\}_{i \in I}\subseteq S\) definieren wir die \emph{Summe} 
\[
  \sum_{i \in I} s_i
\]
durch folgende Schritte:
\begin{itemize}
    \item Falls \(I = \varnothing\), setzen wir
    \[
       \sum_{i \in \varnothing} s_i := 0.
    \]
    \item Falls \(I = \{i_1,i_2,\dots,i_n\}\) mit \(n \ge 1\), so wählen wir eine beliebige Anordnung 
    \[
      i_1, i_2, \dots, i_n
    \]
    der Indizes und definieren rekursiv:
    \[
      \sum_{k=1}^1 s_{i_k} \;:=\; s_{i_1}, 
      \quad\text{und}\quad
      \sum_{k=1}^{m+1} s_{i_k} \;:=\; \Bigl(\sum_{k=1}^m s_{i_k}\Bigr) \;+\; s_{i_{m+1}}
      \quad \text{für } m \in \{1,\dots,n-1\}.
    \]
    Wegen der Kommutativität und Assoziativität der Addition in \(S\) ist diese Definition unabhängig von der gewählten Reihenfolge.
\end{itemize}
\end{definition}

???

\paragraph{Wohldefiniertheit von \(\text{Sum}_I\)}
\begin{theorem}[Wohldefiniertheit der Summenabbildung]
Sei \((S, +, 0)\) ein abelscher Monoid, \(I\) eine endliche Indexmenge und \(a \colon I \to S\) eine Familie. Dann gilt:
\[
\forall I \, (\text{endlich}(I) \land a \colon I \to S \implies \exists! b \in S \, (\Sigma_I(a) = b)).
\]
\end{theorem}
\begin{proof}
Wir zeigen die Aussage durch vollständige Induktion über die Anzahl der Elemente der Indexmenge \(I\), also \(|I|\).

\paragraph{Basisfall: \(|I| = 0\)}
Falls \(I = \varnothing\), ist die Summenabbildung \(\Sigma_I(a)\) durch Definition gegeben als:
\[
\Sigma_I(a) := 0.
\]
Da \(0\) das neutrale Element des abelschen Monoids \((S, +, 0)\) ist, existiert genau ein \(b \in S\) mit \(\Sigma_I(a) = b\). Damit ist die Aussage für den Basisfall gezeigt.

\paragraph{Induktionsschritt: \(|I| = n+1\)}
Angenommen, die Aussage gilt für alle endlichen Indexmengen \(I'\) mit \(|I'| = n\). Sei nun \(I\) eine Indexmenge mit \(|I| = n+1\), und sei \(a \colon I \to S\) eine Familie. Nach Definition gilt:
\[
\Sigma_I(a) = a(j) + \Sigma_{I \setminus \{j\}}(a\restriction_{I \setminus \{j\}}),
\]
wobei \(j \in I\) ein beliebiges Element der Indexmenge ist und \(a\restriction_{I \setminus \{j\}}\) die Einschränkung der Familie \(a\) auf die Menge \(I \setminus \{j\}\) bezeichnet.

Es bleibt zu zeigen, dass \(\Sigma_I(a)\) unabhängig von der Wahl von \(j \in I\) ist. Sei hierzu \(j, k \in I\) mit \(j \neq k\). Dann gilt:
\[
\Sigma_I(a) = a(j) + \Sigma_{I \setminus \{j\}}(a\restriction_{I \setminus \{j\}})
\quad \text{und} \quad
\Sigma_I(a) = a(k) + \Sigma_{I \setminus \{k\}}(a\restriction_{I \setminus \{k\}}).
\]

Nach Induktionsannahme ist die Summenabbildung \(\Sigma_{I \setminus \{j\}}\) für die Menge \(I \setminus \{j\}\) wohldefiniert. Insbesondere gilt, dass für jede beliebige Permutation der Elemente von \(I \setminus \{j\}\) das Ergebnis der Summation gleich bleibt. Dasselbe gilt für \(\Sigma_{I \setminus \{k\}}\).

Die rekursive Definition und die Eigenschaften der abelschen Monoidstruktur (\(+\) ist kommutativ und assoziativ) gewährleisten, dass die Reihenfolge der Addition keinen Einfluss hat. Somit gilt:
\[
\Sigma_{I \setminus \{j\}}(a\restriction_{I \setminus \{j\}}) = \Sigma_{I \setminus \{k\}}(a\restriction_{I \setminus \{k\}}).
\]

Einsetzen in die Ausdrücke für \(\Sigma_I(a)\) ergibt:
\[
a(j) + \Sigma_{I \setminus \{j\}}(a\restriction_{I \setminus \{j\}})
= a(k) + \Sigma_{I \setminus \{k\}}(a\restriction_{I \setminus \{k\}}).
\]

Damit ist gezeigt, dass \(\Sigma_I(a)\) unabhängig von der Wahl von \(j \in I\) ist.

\paragraph{Eindeutigkeit}
Die Eindeutigkeit von \(\Sigma_I(a)\) folgt unmittelbar aus der rekursiven Definition. In jedem Schritt wird ein Element \(a(j)\) addiert, und die verbleibende Summe wird eindeutig durch die Induktionsannahme bestimmt. Somit existiert genau ein \(b \in S\), sodass \(\Sigma_I(a) = b\).

\paragraph{Abschließende Folgerung}
Da sowohl die Existenz als auch die Eindeutigkeit von \(\Sigma_I(a)\) gezeigt wurden, ist die Summenabbildung \(\Sigma_I(a)\) wohldefiniert.
\end{proof}


?????
\section{Induktive Definition von Summen- und Produktzeichen}
\begin{definition}[Summenzeichen und Produktzeichen, induktiv]
    Sei \((S, \ast)\) eine Halbgruppe, d.h., \(S\) ist eine Menge und \(\ast\) eine assoziative binäre Operation \(\ast: S \times S \to S\). Zur Vereinfachung der Notation wird die Operation \(\ast\) konventionell als \enquote{Addition} oder \enquote{Multiplikation} bezeichnet und erhält entsprechend das Symbol \(+\) oder \(\cdot\).

    \begin{itemize}
        \item \textbf{Addition} (Symbol \(+\)): Falls die Operation \(\ast\) als Addition bezeichnet wird und das Symbol \(+\) erhält, definieren wir das Summenzeichen für die Elemente \(a_{i_0}, \dots, a_{i_0+n} \in S\) induktiv durch:
        \[
        \sum_{i=i_0}^{i_0} a_i := a_{i_0},
        \]
        und für einen Übergang von \(n\) auf \(n+1\):
        \[
        \sum_{i=i_0}^{n+1} a_i := \sum_{i=i_0}^n a_i + a_{n+1}.
        \]

        \item \textbf{Multiplikation} (Symbol \(\cdot\)): Falls die Operation \(\ast\) als Multiplikation bezeichnet wird und das Symbol \(\cdot\) erhält, definieren wir das Produktzeichen für die Elemente \(a_{i_0}, \dots, a_{i_0+n} \in S\) induktiv durch:
        \[
        \prod_{i=i_0}^{i_0} a_i := a_{i_0},
        \]
        und für einen Übergang von \(n\) auf \(n+1\):
        \[
        \prod_{i=i_0}^{n+1} a_i := \prod_{i=i_0}^n a_i \cdot a_{n+1}.
        \]
    \end{itemize}

    Diese Definitionen gelten für beliebige assoziative Strukturen, in denen eine binäre Operation \(\ast\) als Addition oder Multiplikation festgelegt wird. Die spezifische Zuweisung der Symbole \(+\) und \(\cdot\) erleichtert die Notation und Interpretation in der jeweiligen Struktur.
\end{definition}

\subsubsection{Einführungsregeln für Summen- und Produktzeichen}
\label{rule:sumIntro} \label{rule:prodIntro}

\paragraph{Einführungsregel für das Summenzeichen \(\sum_{i=i_0}^n a_i\)}
Falls die Operation \(\ast\) als Addition bezeichnet wird, gilt für \(a_{i_0}, \dots, a_n \in S\):
\[
\begin{array}{llll}
    i   & (1) & a_1,...,a_n \in S & \dots \\
    j   & (2) & i_0 < n & \dots \\
    i,j   & (3) & \sum_{i=i_0}^n a_i = \sum_{i=i_0}^{n-1} a_i + a_n & \rSumI{1,2} \\
\end{array}
\]

\[
\begin{array}{llll}
    i   & (1) & a_1,...,a_n \in S & \dots \\
    j   & (2) & i_0 < n & \dots \\
    i,j   & (3) & \sum_{i=i_0}^n a_i = a_{i_0}+\sum_{i=i_0+1}^{n} a_i  & \rSumI{1,2} 
\end{array}
\]

Falls die Operation \(\ast\) als Addition bezeichnet wird, gilt für \(a_{i_0} \in S\):
\[
\begin{array}{llll}
    i & (1) & a_{i_0} \in S & \dots \\
    i & (2) & \sum_{i=i_0}^{i_0} a_i = a_{i_0} & \rSumI{1}
\end{array}
\]

\(i\) und \(j\) sind dabei Listen von Annahmen.

\paragraph{Einführungsregel für das Produktzeichen \(\prod_{i=i_0}^n a_i\)}
Falls die Operation \(\ast\) als Multiplikation bezeichnet wird, gilt für \(a_{i_0}, \dots, a_n \in S\):
\[
\begin{array}{llll}
    i   & (1) & a_n \in S & \dots \\
    j   & (2) & i_0 < n & \dots \\
    i,j & (3) & \prod_{i=i_0}^n a_i = \prod_{i=i_0}^{n-1} a_i \cdot a_n & \rProdI{1,2}
\end{array}
\]

\[
\begin{array}{llll}
    i   & (1) & a_n \in S & \dots \\
    j   & (2) & i_0 < n & \dots \\
    i,j & (3) & \prod_{i=i_0}^n a_i = a_{i_0}\cdot \prod_{i=i_0+1}^{n-1} a_i & \rProdI{1,2}
\end{array}
\]

Falls die Operation \(\ast\) als Multiplikation bezeichnet wird, gilt für \(a_{i_0} \in S\):
\[
\begin{array}{llll}
    i & (1) & a_{i_0} \in S & \dots \\
    i & (2) & \prod_{i=i_0}^{i_0} a_i = a_{i_0} & \rProdI{1}
\end{array}
\]

\(i\) und \(j\) sind dabei Listen von Annahmen.

\chapter{Endliche Summen natürlicher Zahlen}

\section{Linksdistributivität endlicher Summen}

\label{awsSubLbiSubZeroRbInNaturalImpaSumSubLbiEqualsiSubZeroRbPowerLbiSubZeroRbsSubiEqualsSumSubLbiEqualsiSubZeroRbPowerLbiSubZeroRbasSubi}
\begin{lemma}[\(a,s_{i_0}\in\mathbb{N}\vdash a\sum_{i=i_0}^{i_0} s_i=\sum_{i=i_0}^{i_0} as_i\) (Induktionsanfang)]
\end{lemma}
\begin{proof}
    Seien \(a,s_{i_0}\in\mathbb{N}\), dann gilt:
    \[
	\begin{array}{llcll p{5cm}}
              &  (1)  & a\sum_{i=i_0}^{i_0} s_i &=& as_{i_0} & \rSumI{} \\
              &  (2)  &  &=& \sum_{i=i_0}^{i_0} as_i & \rSumI{} \\
              &  (3)  &  a\sum_{i=i_0}^{i_0} s_i&=& \sum_{i=i_0}^{i_0} as_i & \rTransitivityEqRI{1,2} \\
        \end{array}
    \]
\end{proof}

\label{awnwsSubLbiSubZeroRbwDotswsSubLbnPlusOneRbInNaturalwaSumSubLbiEqualsiSubZeroRbPowerLbnRbsSubiEqualsSumSubLbiEqualsiSubZeroRbPowerLbnRbasSubiImpaSumSubLbiEqualsiSubZeroRbPowerLbnPlusOneRbsSubiEqualsSumSubLbiEqualsiSubZeroRbPowerLbnPlusOneRbasSubi}
\begin{lemma}[\(a,n,s_{i_0},\dots,s_{n+1}\in\mathbb{N},a\sum_{i=i_0}^{n} s_i=\sum_{i=i_0}^{n} as_i\vdash a\sum_{i=i_0}^{n+1} s_i=\sum_{i=i_0}^{n+1} as_i\) (Induktionsschritt)]
\end{lemma}
\begin{proof}
    Seien \(a,n,s_{i_0},\dots,s_{n+1}\in\mathbb{N}\). \(\ImpLpNaturalwMultwOneRpInAbelSemiRing{}\) und daher gilt:
    \[
	\begin{array}{llcll p{5cm}}
            1 &  (1)  & \multicolumn{3}{l}{a\sum_{i=i_0}^{n} s_i=\sum_{i=i_0}^{n} as_i} & \rA \\
              &  (2)  & a\sum_{i=i_0}^{n+1} s_i &=& a(\sum_{i=i_0}^{n} s_i+s_{n+1}) & \rSumI{} \\
              &  (3)  &  &=& a\sum_{i=i_0}^{n} s_i+as_{n+1} & \rLeftDistributiveAbelianSemigroup{} \\
            1 &  (4)  &  &=& \sum_{i=i_0}^{n} as_i+as_{n+1} & \rIE{1,3} \\
              &  (5)  &  &=& \sum_{i=i_0}^{n+1} as_i & \rSumI{} \\
            1 &  (6)  & a\sum_{i=i_0}^{n+1} s_i &=& \sum_{i=i_0}^{n+1} as_i & \rTransitivityEqRI{2,5} \\
        \end{array}
    \]
\end{proof}

\label{awnInNaturalwsSubLbiSubZeroRbwDotswsSubnInNaturalImpaSumSubLbiEqualsiSubZeroRbPowernsSubiEqualsSumSubLbiEqualsiSubZeroRbPowernasSubi}
\begin{theorem}[\(a,n\in\mathbb{N},s_{i_0},\dots, s_n\in\mathbb{N}\vdash a\sum_{i=i_0}^n s_i=\sum_{i=i_0}^n as_i\)]
\end{theorem}
\begin{proof}
    Seien \(a,s_{i_0},\dots,s_{n}\in\mathbb{N}\), dann gilt:
    \[
	\begin{array}{llcll p{5cm}}
               &  (1)  & \multicolumn{3}{l}{a\sum_{i=i_0}^{i_0} s_i=\sum_{i=i_0}^{i_0} as_i} & \awsSubLbiSubZeroRbInNaturalImpaSumSubLbiEqualsiSubZeroRbPowerLbiSubZeroRbsSubiEqualsSumSubLbiEqualsiSubZeroRbPowerLbiSubZeroRbasSubi{} \\
             2 &  (2)  & \multicolumn{3}{l}{n\in\mathbb{N}} & \rA \\
             3 &  (3)  & \multicolumn{3}{l}{a\sum_{i=i_0}^{n} s_i=\sum_{i=i_0}^{n} as_i} & \rA \\ 
             2,3 &  (4)  & \multicolumn{3}{l}{a\sum_{i=i_0}^{n+1} s_i=\sum_{i=i_0}^{n+1} as_i} & \awnwsSubLbiSubZeroRbwDotswsSubLbnPlusOneRbInNaturalwaSumSubLbiEqualsiSubZeroRbPowerLbnRbsSubiEqualsSumSubLbiEqualsiSubZeroRbPowerLbnRbasSubiImpaSumSubLbiEqualsiSubZeroRbPowerLbnPlusOneRbsSubiEqualsSumSubLbiEqualsiSubZeroRbPowerLbnPlusOneRbasSubi{2,3} \\ 
                &  (5)  & \multicolumn{3}{l}{\forall n\in\mathbb{N}(a\sum_{i=i_0}^{n} s_i=\sum_{i=i_0}^{n} as_i)} & \rInductionN{1,2,3,4} \\ 
                &  (6)  & \multicolumn{3}{l}{n\in\mathbb{N}\rightarrow a\sum_{i=i_0}^{n} s_i=\sum_{i=i_0}^{n} as_i} & \rSetEEb{5} \\ 
            2&  (7)  & \multicolumn{3}{l}{a\sum_{i=i_0}^{n} s_i=\sum_{i=i_0}^{n} as_i} & \rRE{2,6} \\ 
        \end{array}
    \]
\end{proof}

\section{Rechtssdistributivität endlicher Summen}

\label{awsSubLbiSubZeroRbInNaturalImpLpSumSubLbiEqualsiSubZeroRbPowerLbiSubZeroRbsSubiRpaEqualsSumSubLbiEqualsiSubZeroRbPowerLbiSubZeroRbsSubia}
\begin{lemma}[\(a,s_{i_0}\in\mathbb{N}\vdash (\sum_{i=i_0}^{i_0} s_i)a=\sum_{i=i_0}^{i_0} s_ia\) (Induktionsanfang)]
\end{lemma}
\begin{proof}
    Seien \(a,s_{i_0}\in\mathbb{N}\), dann gilt:
    \[
	\begin{array}{llcll p{5cm}}
              &  (1)  & (\sum_{i=i_0}^{i_0} s_i)a &=& s_{i_0}a & \rSumI{} \\
              &  (2)  &  &=& \sum_{i=i_0}^{i_0} s_ia & \rSumI{} \\
              &  (3)  &  (\sum_{i=i_0}^{i_0} s_i)a&=& \sum_{i=i_0}^{i_0} s_ia & \rTransitivityEqRI{1,2} \\
        \end{array}
    \]
\end{proof}

\label{awnwsSubLbiSubZeroRbwDotswsSubLbnPlusOneRbInNaturalwLpSumSubLbiEqualsiSubZeroRbPowerLbnRbsSubiRpaEqualsSumSubLbiEqualsiSubZeroRbPowerLbnRbsSubiaImpLpSumSubLbiEqualsiSubZeroRbPowerLbnPlusOneRbsSubiRpaEqualsSumSubLbiEqualsiSubZeroRbPowerLbnPlusOneRbsSubia}
\begin{lemma}[\(a,n,s_{i_0},\dots,s_{n+1}\in\mathbb{N},(\sum_{i=i_0}^{n} s_i)a=\sum_{i=i_0}^{n} s_ia\vdash (\sum_{i=i_0}^{n+1} s_i)a=\sum_{i=i_0}^{n+1} s_ia\) (Induktionsschritt)]
\end{lemma}
\begin{proof}
    Seien \(a,n,s_{i_0},\dots,s_{n+1}\in\mathbb{N}\). \(\ImpLpNaturalwMultwOneRpInAbelSemiRing{}\) und daher gilt:
    \[
	\begin{array}{llcll p{5cm}}
            1 &  (1)  & \multicolumn{3}{l}{(\sum_{i=i_0}^{n} s_i)a=\sum_{i=i_0}^{n} s_ia} & \rA \\
              &  (2)  & (\sum_{i=i_0}^{n+1} s_i)a &=& (\sum_{i=i_0}^{n} s_i+s_{n+1})a & \rSumI{} \\
              &  (3)  &  &=& (\sum_{i=i_0}^{n} s_i)a+s_{n+1}a & \rRightDistributiveAbelianSemigroup{} \\
            1 &  (4)  &  &=& \sum_{i=i_0}^{n} s_ia+s_{n+1}a & \rIE{1,3} \\
              &  (5)  &  &=& \sum_{i=i_0}^{n+1} s_ia & \rSumI{} \\
            1 &  (6)  & (\sum_{i=i_0}^{n+1} s_i)a &=& \sum_{i=i_0}^{n+1} s_ia & \rTransitivityEqRI{2,5} \\
        \end{array}
    \]
\end{proof}

\label{awnInNaturalwsSubLbiSubZeroRbwDotswsSubnInNaturalImpLpSumSubLbiEqualsiSubZeroRbPowernsSubiRpaEqualsSumSubLbiEqualsiSubZeroRbPowernsSubia}
\begin{theorem}[\(a,n\in\mathbb{N},s_{i_0},\dots, s_n\in\mathbb{N}\vdash (\sum_{i=i_0}^n s_i)a=\sum_{i=i_0}^n s_ia\)]
\end{theorem}
\begin{proof}
    Seien \(a,s_{i_0},\dots,s_{n}\in\mathbb{N}\), dann gilt:
    \[
	\begin{array}{llcll p{5cm}}
               &  (1)  & \multicolumn{3}{l}{(\sum_{i=i_0}^{i_0} s_i)a=\sum_{i=i_0}^{i_0} s_ia} & \awsSubLbiSubZeroRbInNaturalImpLpSumSubLbiEqualsiSubZeroRbPowerLbiSubZeroRbsSubiRpaEqualsSumSubLbiEqualsiSubZeroRbPowerLbiSubZeroRbsSubia{} \\
             2 &  (2)  & \multicolumn{3}{l}{n\in\mathbb{N}} & \rA \\
             3 &  (3)  & \multicolumn{3}{l}{(\sum_{i=i_0}^{n} s_i)a=\sum_{i=i_0}^{n} s_ia} & \rA \\ 
             2,3 &  (4)  & \multicolumn{3}{l}{(\sum_{i=i_0}^{n+1} s_i)a=\sum_{i=i_0}^{n+1} s_ia} & \awnwsSubLbiSubZeroRbwDotswsSubLbnPlusOneRbInNaturalwLpSumSubLbiEqualsiSubZeroRbPowerLbnRbsSubiRpaEqualsSumSubLbiEqualsiSubZeroRbPowerLbnRbsSubiaImpLpSumSubLbiEqualsiSubZeroRbPowerLbnPlusOneRbsSubiRpaEqualsSumSubLbiEqualsiSubZeroRbPowerLbnPlusOneRbsSubia{2,3} \\ 
                &  (5)  & \multicolumn{3}{l}{\forall n\in\mathbb{N}((\sum_{i=i_0}^{n} s_i)a=\sum_{i=i_0}^{n} s_ia)} & \rInductionN{1,2,3,4} \\ 
                &  (6)  & \multicolumn{3}{l}{n\in\mathbb{N}\rightarrow (\sum_{i=i_0}^{n} s_i)a=\sum_{i=i_0}^{n} s_ia} & \rSetEEb{5} \\ 
            2&  (7)  & \multicolumn{3}{l}{(\sum_{i=i_0}^{n} s_i)a=\sum_{i=i_0}^{n} s_ia} & \rRE{2,6} \\ 
        \end{array}
    \]
\end{proof}

\section{Rechtssdistributivität endlicher Summen mit Potenzen}

\label{awsSubLbiSubZeroRbInNaturalImpLpSumSubLbiEqualsiSubZeroRbPowerLbiSubZeroRbsSubiaPoweriRpaEqualsSumSubLbiEqualsiSubZeroRbPowerLbiSubZeroRbsSubiaPowerLbiPlusOneRb}
\begin{lemma}[\(a,s_{i_0}\in\mathbb{N}\vdash (\sum_{i=i_0}^{i_0} s_ia^i)a=\sum_{i=i_0}^{i_0} s_ia^{i+1}\) (Induktionsanfang)]
\end{lemma}
\begin{proof}
    Seien \(a,s_{i_0}\in\mathbb{N}\), dann gilt:
    \[
	\begin{array}{llcll p{5cm}}
              &  (1)  & (\sum_{i=i_0}^{i_0} s_ia^i)a &=& s_{i_0}a^ia & \rSumI{} \\
              &  (2)  &  &=& s_{i_0}a^{i+1} & \rPowerI{} \\
              &  (3)  &  &=& \sum_{i=i_0}^{i_0} s_ia^{i+1} & \rSumI{} \\
              &  (4)  &  (\sum_{i=i_0}^{i_0} s_ia^i)a&=& \sum_{i=i_0}^{i_0} s_ia^{i+1} & \rTransitivityEqRI{1,3} \\
        \end{array}
    \]
\end{proof}

\label{awnwsSubLbiSubZeroRbwDotswsSubLbnPlusOneRbInNaturalwLpSumSubLbiEqualsiSubZeroRbPowerLbnRbsSubiaPoweriRpaEqualsSumSubLbiEqualsiSubZeroRbPowerLbnRbsSubiaPowerLbiPlusOneRbImpLpSumSubLbiEqualsiSubZeroRbPowerLbnPlusOneRbsSubiaPoweriRpaEqualsSumSubLbiEqualsiSubZeroRbPowerLbnPlusOneRbsSubiaPowerLbiPlusOneRb}
\begin{lemma}[\(a,n,s_{i_0},\dots,s_{n+1}\in\mathbb{N},(\sum_{i=i_0}^{n} s_ia^i)a=\sum_{i=i_0}^{n} s_ia^{i+1}\vdash (\sum_{i=i_0}^{n+1} s_ia^i)a=\sum_{i=i_0}^{n+1} s_ia^{i+1}\) (Induktionsschritt)]
\end{lemma}
\begin{proof}
    Seien \(a,n,s_{i_0},\dots,s_{n+1}\in\mathbb{N}\). \(\ImpLpNaturalwMultwOneRpInAbelSemiRing{}\) und daher gilt:
    \[
	\begin{array}{llcll p{5cm}}
            1 &  (1)  & \multicolumn{3}{l}{(\sum_{i=i_0}^{n} s_ia^i)a=\sum_{i=i_0}^{n} s_ia^{i+1}} & \rA \\
              &  (2)  & (\sum_{i=i_0}^{n+1} s_ia^i)a &=& (\sum_{i=i_0}^{n} s_ia^i+s_{n+1}a^{n+1})a & \rSumI{} \\
              &  (3)  &  &=& (\sum_{i=i_0}^{n} s_ia^i)a+s_{n+1}a^{n+1}a & \rRightDistributiveAbelianSemigroup{} \\
            1 &  (4)  &  &=& \sum_{i=i_0}^{n} s_ia^{i+1}+s_{n+1}a^{n+1}a & \rIE{1,3} \\
              &  (5)  &  &=& \sum_{i=i_0}^{n} s_ia^{i+1}+s_{n+1}a^{n+2} & \rPowerI{} \\
              &  (6)  &  &=& \sum_{i=i_0}^{n+1} s_ia^{i+1} & \rSumI{} \\
            1 &  (7)  & (\sum_{i=i_0}^{n+1} s_ia^i)a &=& \sum_{i=i_0}^{n+1} s_ia^{i+1} & \rTransitivityEqRI{2,6} \\
        \end{array}
    \]
\end{proof}

\label{awnInNaturalwsSubLbiSubZeroRbwDotswsSubnInNaturalImpLpSumSubLbiEqualsiSubZeroRbPowernsSubiaPoweriRpaEqualsSumSubLbiEqualsiSubZeroRbPowernsSubiaPowerLbiPlusOneRb}
\begin{theorem}[\(a,n\in\mathbb{N},s_{i_0},\dots, s_n\in\mathbb{N}\vdash (\sum_{i=i_0}^n s_ia^i)a=\sum_{i=i_0}^n s_ia^{i+1}\)]
\end{theorem}
\begin{proof}
    Seien \(a,s_{i_0},\dots,s_{n}\in\mathbb{N}\), dann gilt:
    \[
	\begin{array}{llcll p{5cm}}
               &  (1)  & \multicolumn{3}{l}{(\sum_{i=i_0}^{i_0} s_ia^i)a=\sum_{i=i_0}^{i_0} s_ia^{i+1}} & \awsSubLbiSubZeroRbInNaturalImpLpSumSubLbiEqualsiSubZeroRbPowerLbiSubZeroRbsSubiaPoweriRpaEqualsSumSubLbiEqualsiSubZeroRbPowerLbiSubZeroRbsSubiaPowerLbiPlusOneRb{} \\
             2 &  (2)  & \multicolumn{3}{l}{n\in\mathbb{N}} & \rA \\
             3 &  (3)  & \multicolumn{3}{l}{(\sum_{i=i_0}^{n} s_ia^i)a=\sum_{i=i_0}^{n} s_ia^{i+1}} & \rA \\ 
             2,3 &  (4)  & \multicolumn{3}{l}{(\sum_{i=i_0}^{n+1} s_ia^i)a=\sum_{i=i_0}^{n+1} s_ia^{i+1}} & \awnwsSubLbiSubZeroRbwDotswsSubLbnPlusOneRbInNaturalwLpSumSubLbiEqualsiSubZeroRbPowerLbnRbsSubiaPoweriRpaEqualsSumSubLbiEqualsiSubZeroRbPowerLbnRbsSubiaPowerLbiPlusOneRbImpLpSumSubLbiEqualsiSubZeroRbPowerLbnPlusOneRbsSubiaPoweriRpaEqualsSumSubLbiEqualsiSubZeroRbPowerLbnPlusOneRbsSubiaPowerLbiPlusOneRb{2,3} \\ 
                &  (5)  & \multicolumn{3}{l}{\forall n\in\mathbb{N}(\sum_{i=i_0}^{n} s_ia^i)a=\sum_{i=i_0}^{n} s_ia^{i+1}} & \rInductionN{1,2,3,4} \\ 
                &  (6)  & \multicolumn{3}{l}{n\in\mathbb{N}\rightarrow (\sum_{i=i_0}^{n} s_ia^i)a=\sum_{i=i_0}^{n} s_ia^{i+1}} & \rSetEEb{5} \\ 
            2&  (7)  & \multicolumn{3}{l}{(\sum_{i=i_0}^{n} s_ia^i)a=\sum_{i=i_0}^{n} s_ia^{i+1}} & \rRE{2,6} \\ 
        \end{array}
    \]
\end{proof}

\section{Linksdistributivität endlicher Summen mit Potenzen}


\label{awnInNaturalwsSubLbiSubZeroRbwDotswsSubnInNaturalImpaSumSubLbiEqualsiSubZeroRbPowernsSubiaPoweriEqualsSumSubLbiEqualsiSubZeroRbPowernsSubiaPowerLbiPlusOneRb}
\begin{theorem}[\(a,n\in\mathbb{N},s_{i_0},\dots, s_n\in\mathbb{N}\vdash a\sum_{i=i_0}^n s_ia^i=\sum_{i=i_0}^n s_ia^{i+1}\)]
\end{theorem}
\begin{proof}
    Seien \(a,n,s_{i_0},\dots,s_{n}\in\mathbb{N}\). \(\ImpLpNaturalwMultwOneRpInAbelSemiRing{}\) und es gilt:
    \[
	\begin{array}{llcll p{5cm}}
               &  (1)  & a\sum_{i=i_0}^{i_0} s_ia^i&=&(\sum_{i=i_0}^{i_0} s_ia^i)a & \rCommutativeMonoid{} \\
               &  (2)  & &=&\sum_{i=i_0}^{i_0} s_ia^{i+1} & \awnInNaturalwsSubLbiSubZeroRbwDotswsSubnInNaturalImpLpSumSubLbiEqualsiSubZeroRbPowernsSubiaPoweriRpaEqualsSumSubLbiEqualsiSubZeroRbPowernsSubiaPowerLbiPlusOneRb{} \\
               &  (3)  & a\sum_{i=i_0}^{i_0} s_ia^i&=&\sum_{i=i_0}^{i_0} s_ia^{i+1} & \rTransitivityEqRI{1,2} \\
        \end{array}
    \]
\end{proof}




\chapter{n-adische Zahlendarstellung}

\section{Existenz der n-adischen Darstellung}

\subsection{Existenz der n-adischen Darstellung}

\paragraph{Induktionsanfang}

\label{FanInNaturalLpnGneqOneRpToExkInNaturalExaSubZerowaSubOnewDotswaSubkInLbZerowOnewDotswnMinusOneRbLpZeroEqualsSumSubLbiEqualsZeroRbPowerkaSubiMultnPoweriRp}
\begin{lemma}[\(n\in\mathbb{N}, n > 1 \vdash \exists k\in\mathbb{N}\exists a_0, a_1, \dots, a_k \in \{0,1,\dots,n-1\} ( 0 = \sum_{i=0}^k a_i \cdot n^i)\)]
\end{lemma}
\begin{proof}
Sei \(n\in\mathbb{N}\), dann gilt:
    \[
	\begin{array}{llcll p{4.5cm}}
            1 &  (1)  & \multicolumn{3}{l}{n > 1} & \rA \\
            1 &  (2)  &  \multicolumn{3}{l}{0\in\{0,1,...,n-1\}} & \nInNaturalwnGneqOneImpZeroInLbZerowOnewDotswnMinusOneRb{1} \\ 
              &  (3)  & 0&=&0\cdot 1 & \aInNaturalImpZeroEqualsZeroMulta{} \\
              &  (4)  & &=&0\cdot n^{0} & \rPowerI{} \\
              &  (5)  &  &=&\sum_{i=0}^0 0 \cdot n^{i} & \rSumI{} \\
              &  (6)  &  0&=&\sum_{i=0}^0 0 \cdot n^{i} & \rTransitivityEqRI{4,6} \\
             1&  (7)  & \multicolumn{3}{p{6cm}}{\(\exists a_0\in \{0,1,\dots,n-1\} ( 0 = \sum_{i=0}^0 a_i \cdot n^i)\)} & \rSetEIa{3,6} \\
              &  (8)  & \multicolumn{3}{l}{0\in\mathbb{N}} & \zeroIsNaturalNumber{} \\
             1 &  (9)  & \multicolumn{3}{p{6cm}}{\(\exists k\in\mathbb{N}\exists a_0, a_1, \dots, a_k \in \{0,1,\dots,n-1\} ( 0 = \sum_{i=0}^k a_i \cdot n^i)\)} & \rSetEIa{9,8} \\
        \end{array}
    \]
\end{proof}

\begin{lemma}[Induktionsschritt]
Seien \(n,m,a\in\mathbb{N}\) und
\[
n > 1, \exists k \in \mathbb{N}\exists a_0, a_1, \dots, a_k \in \{0, 1, \dots, n-1\} 
    (m+1)\div n = \sum_{i=0}^k a_i \cdot n^i\]
\[
     \vdash\exists k' \in \mathbb{N}, \, \exists a_0', a_1', \dots, a_{k'}' \in \{0, 1, \dots, n-1\}m + 1 = \sum_{i=0}^{k'} a_i' \cdot n^i.
\]
\end{lemma}
\begin{proof}
Seien \(n,m,a\in\mathbb{N}\), dann gilt:
    \[
	\begin{array}{llcll p{4.5cm}}
             1 &  (1)  & \multicolumn{3}{l}{n > 1} & \rA \\
             1 &  (2)  & \multicolumn{3}{l}{n \neq 1} & \aInNaturalwaGeqOneEqvaNotEqualsZero{1} \\
             4 &  (3)  & \multicolumn{3}{p{7cm}}{\(\exists k \in \mathbb{N}\exists a_0, a_1, \dots, a_k \in \{0, 1, \dots, n-1\} 
    (m+1)\div n = \sum_{i=0}^k a_i \cdot n^i\)}& \rA \\
    \multicolumn{6}{l}{\text{Wähle \(k\in\mathbb{N}\), sodass:}}\\
             4 &  (4)  & \multicolumn{3}{p{7cm}}{\(\exists a_0, a_1, \dots, a_k \in \{0, 1, \dots, n-1\} 
    ((m+1)\div n = \sum_{i=0}^k (a_i \cdot n^i))\)}& \rA \\
             5 &  (5)  & \multicolumn{3}{l}{(m+1)\div n = \sum_{i=0}^k a_i \cdot n^{i}}& \rA \\
             1 &  (6)  & \multicolumn{3}{l}{m+1 = n\cdot ((m+1)\div n)+((m+1)\bmod n)}& \rDivisionWithRemainderI{2} \\
             1,5 &  (7)  & m+1 &=& n\cdot (\sum_{i=0}^k a_i \cdot n^{i})+((m+1)\bmod n)& \rIE{5,6} \\
             1 &  (8)  &  &=& \sum_{i=0}^k a_{i} \cdot n^{i+1}+((m+1)\bmod n)& \awnInNaturalwsSubLbiSubZeroRbwDotswsSubnInNaturalImpaSumSubLbiEqualsiSubZeroRbPowernsSubiaPoweriEqualsSumSubLbiEqualsiSubZeroRbPowernsSubiaPowerLbiPlusOneRb{} \\
               &  (9)  &  &=& \sum_{i=0}^k a_{i} \cdot n^{i+1}+((m+1)\bmod n)\cdot 1& \rNeutralElementMonoid{} \\
               &  (10)  &  &=& \sum_{i=0}^k a_{i} \cdot n^{i+1}+((m+1)\bmod n)\cdot n^0& \rPowerI{} \\
               &  (11)  &  &=& ((m+1)\bmod n)\cdot n^0+\sum_{i=0}^k a_{i} \cdot n^{i+1}& \rCommutativeMonoid{} \\
             1,5 &  (12)  & m+1 &=& ((m+1)\bmod n)\cdot n^0+\sum_{i=0}^k a_{i} \cdot n^{i+1}& \rTransitivityEqRI{6,11} \\
             1 &  (13)  &\multicolumn{3}{l}{((m+1)\bmod n)<n}& \rDivisionWithRemainderI{2} \\
             1 &  (14)  &\multicolumn{3}{l}{((m+1)\bmod n)\in \{0,1,\dots,n-1\}}& \rSegmentZeroI{13} \\
             1,5 &  (15)  &\multicolumn{3}{p{7cm}}{\(\exists a_0'\in \{0,1,\dots,n-1\} (m+1 = a_0'\cdot n^0+\sum_{i=0}^k a_{i} \cdot n^{i+1}) \)}& \rSetEIa{13,12} \\
             1,5 &  (16)  &\multicolumn{3}{p{7cm}}{\(\exists a_0',\dots a_k'\in \{0,1,\dots,n-1\} (m+1 = a_0'\cdot n^0+\sum_{i=0}^k a_{i+1}' \cdot n^{i+1}) \)}& \rSetEm{15} \\
        \end{array}
    \]
\end{proof}
????
aInNaturalwaGeqOneEqvaNotEqualsZero

\begin{theorem}[\(\forall a \in \mathbb{N}\forall n \in \mathbb{N} \left( n > 1 \right) \rightarrow \exists k\in\mathbb{N}\exists a_0, a_1, \dots, a_k \in \{0,1,\dots,n-1\} \left( a = \sum_{i=0}^k a_i \cdot n^i \right)\)]
\end{theorem}

\begin{proof}
    \[
	\begin{array}{llcll p{4.5cm}}
              &  (1)  & \multicolumn{3}{p{6cm}}{\(\forall n \in \mathbb{N}( n > 1) \rightarrow \exists k\in\mathbb{N}\exists a_0, a_1, \dots, a_k \in \{0,1,\dots,n-1\} ( 0 = \sum_{i=0}^k a_i \cdot n^i)\)} & \FanInNaturalLpnGneqOneRpToExkInNaturalExaSubZerowaSubOnewDotswaSubkInLbZerowOnewDotswnMinusOneRbLpZeroEqualsSumSubLbiEqualsZeroRbPowerkaSubiMultnPoweriRp{} \\
        \end{array}
    \]
\end{proof}




\subsection{Eindeutigkeit der n-adischen Darstellung}

\begin{theorem}[Eindeutigkeit der n-adischen Darstellung]
    \(
    \forall a, n \in \mathbb{N}, n > 1, 
    \forall a_0, \dots, a_k, b_0, \dots, b_m \in \mathbb{N}, 
    \big(
        0 \leq a_i < n \land 
        0 \leq b_j < n \land 
        a = \sum_{i=0}^{k} a_i \cdot n^i \land 
        a = \sum_{j=0}^{m} b_j \cdot n^j
    \big) 
    \vdash 
    (k = m) \land (\forall i \leq k: a_i = b_i)
    \)
\end{theorem}



\subsection{Definition der n-adischen Darstellung}

\begin{definition}[n-adische Darstellung]
    Seien \( a \in \mathbb{N} \) und \( n \in \mathbb{N} \) mit \( n > 1 \). Die \textbf{n-adische Darstellung} von \( a \) ist die eindeutige Folge von Ziffern \( (a_k, a_{k-1}, \dots, a_0) \), sodass
    \[
    a = \sum_{i=0}^{k} a_i \cdot n^i,
    \]
    wobei \( 0 \leq a_i < n \) für alle \( i \) gilt.
\end{definition}

\subsection{Einführung des Dezimalsystems}

\begin{theorem}[Dezimalsystem als n-adische Darstellung]
    \(
    \forall a \in \mathbb{N} \vdash \exists a_0, a_1, \dots, a_k \in \mathbb{N} \,
    \big(
        0 \leq a_i < 10 \land 
        a = a_0 + a_1 \cdot 10 + a_2 \cdot 10^2 + \dots + a_k \cdot 10^k
    \big)
    \)
\end{theorem}

\begin{proof}
    \[
    \begin{array}{llcll p{5cm}}
        1 &  (1)  & a \in \mathbb{N} & \rA & \\
        2 &  (2)  & n = 10 & \rA & \\
        3 &  (3)  & \text{Existenz folgt aus Theorem 1} & & \\
        \multicolumn{5}{l}{\text{Damit ist die Existenz im Dezimalsystem gezeigt.}} \\
    \end{array}
    \]
\end{proof}

\begin{example}
    \(
    345 = 5 \cdot 10^0 + 4 \cdot 10^1 + 3 \cdot 10^2
    \)
\end{example}

\end{document}