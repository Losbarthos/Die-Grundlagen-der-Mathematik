%============================================================
%  Bd. 08 - Algebraische Strukturen %============================================================

\documentclass{book}
\usepackage{graphicx}
\usepackage[utf8]{inputenc}
\usepackage[ngerman]{babel}
\usepackage{amsmath,amssymb,amsthm}
\usepackage{mathtools}
\usepackage{microtype}
\usepackage{hyperref}
\usepackage{csquotes}
\usepackage{enumitem}
\usepackage{thmtools}
\usepackage{etoolbox}
\usepackage{longtable}
\usepackage{xparse}
\usepackage{imakeidx}
\usepackage{bussproofs}
\tolerance=9999
\MakeAutoQuote{„}{“}
\hypersetup{
    bookmarksopen=false,
    bookmarksnumbered=true
}



% ----------------------------------------------------------------------------
% Syntax-Abkürzungen für logische und mengentheoretische Operatoren
% ----------------------------------------------------------------------------
% Dieser Abschnitt definiert Kurzformen (Abkürzungen) für häufig verwendete logische
% und mengentheoretische Operatoren, die in diesem Dokumentin Form von Commands 
% verwendet werden. Diese Kurzformen sind entworfen, um die Lesbarkeit des Codes zu 
% verbessern und eine effiziente Bearbeitung zu ermöglichen.
%
% ----------------------------------------------------------------------------

    \newcommand{\rA}{\hyperref[rule:A]{\ensuremath{A}}}
    % Regeln zum Umgang mit dem ∧-Symbol
    \newcommand{\rAI}[1]{\hyperref[rule:AI]{\ensuremath{\land}I(#1)}}
    \newcommand{\rAEa}[1]{\hyperref[rule:AE1]{\ensuremath{\land}E1(#1)}}
    \newcommand{\rAEb}[1]{\hyperref[rule:AE2]{\ensuremath{\land}E2(#1)}}
    \newcommand{\rAEn}[1]{\hyperref[rule:AEn]{\ensuremath{\land}E(#1)}}	
	
    % Regeln zum Umgang mit dem ∨-Symbol
    \newcommand{\rOIa}[1]{\hyperref[rule:OI1]{\ensuremath{\lor}I1(#1)}}
    \newcommand{\rOIb}[1]{\hyperref[rule:OI2]{\ensuremath{\lor}I2(#1)}}
    \newcommand{\rOE}[1]{\hyperref[rule:OE]{\ensuremath{\lor}E(#1)}}	
    \newcommand{\rOEn}[1]{\hyperref[rule:OEn]{\ensuremath{\lor}E(#1)}}	
	
    % Regeln zum Umgang mit dem →-Symbol
    \newcommand{\rRI}[1]{\hyperref[rule:RI]{\ensuremath{\rightarrow}I(#1)}}	
    \newcommand{\rRE}[1]{\hyperref[rule:RE]{\ensuremath{\rightarrow}E(#1)}}	
	
    % Regeln zu Umgang mit dem ↔.Symbol
    \newcommand{\rLRI}[1]{\hyperref[rule:LRI]{\ensuremath{\leftrightarrow}I(#1)}}	
    \newcommand{\rLREa}[1]{\hyperref[rule:LRE1]
    {\ensuremath{\leftrightarrow}E1(#1)}}	
    \newcommand{\rLREb}[1]{\hyperref[rule:LRE2]{\ensuremath{\leftrightarrow}E2(#1)}}

    \newcommand{\rLRS}[1]{\hyperref[rule:LRSubst]{\ensuremath{\leftrightarrow}S(#1)}}

	
    % Regeln zum Umgang mit dem ∀-Symbol
    \newcommand{\rUI}[1]{\hyperref[rule:UI]{\ensuremath{\forall}I(#1)}}		
    \newcommand{\rUE}[1]{\hyperref[rule:UE]{\ensuremath{\forall}E(#1)}}	
	
	
    % Regeln zum Umgang mit dem ∃-Symbol	
    \newcommand{\rEI}[1]{\hyperref[rule:EI]{\ensuremath{\exists}I(#1)}}				
    \newcommand{\rEE}[1]{\hyperref[rule:EE]{\ensuremath{\exists}E(#1)}}	
	
    % Regeln zum Umgang mit dem ∃!-Symbol
    \newcommand{\UEI}[1]{\hyperref[rule:UEI]{\ensuremath{\exists!}I(#1)}}			
    \newcommand{\UEE}[1]{\hyperref[rule:UEE]{\ensuremath{\exists!}E(#1)}}	

    % Regeln zum Umgang mit dem =-Symbol
    \newcommand{\rII}{\hyperref[rule:II]{\ensuremath{=}I}}
    \newcommand{\rIE}[1]{\hyperref[rule:IE]{\ensuremath{=}E(#1)}}	

    % Regeln zum Umgang mit dem =-Symbol dreier Gleichheiten
    \newcommand{\rIIb}[1]{\hyperref[rule:rIIb]{\ensuremath{=}I(#1)}}
    \newcommand{\rIEb}[1]{\hyperref[rule:rIEb]{\ensuremath{=}E(#1)}}	
	
    \newcommand{\rNeq}{\hyperref[rule:Neq]{\ensuremath{\neq}}}	
	
    % Regeln zum Umgang mit dem ¬-Symbol als auch dem ⊥-Symbol	
    \newcommand{\rBI}[1]{\hyperref[rule:BI]{\ensuremath{\bot}I(#1)}}
    \newcommand{\rCI}[1]{\hyperref[rule:CI]{\ensuremath{\neg}I(#1)}}
    \newcommand{\rCE}[1]{\hyperref[rule:CE]{\ensuremath{\neg}E(#1)}}
    \newcommand{\rDN}[1]{\hyperref[rule:DN]{\ensuremath{}DN(#1)}}


    % Regel zur Kettennotation
    \newcommand{\rChain}[1]{\hyperref[rule:Chain]{\ensuremath{\mathsf{Tr.}}(#1)}}

    %Xor
    \newcommand*\lxor{\mathbin{\veebar}}
    
    \newcommand{\eqvdash}{\dashv\vdash}

    % Powerset
    \newcommand{\powerset}{\mathcal{P}}

    %Induktionsprinzip
    \newcommand{\rInduktion}[1]{\hyperref[rule:Induktion]{\ensuremath{\mathrm{Induktion}}(#1)}}

    



\makeindex[name=satz,title=Sätze und Definitionen zu diesem Kapitel]

\theoremstyle{plain}
\newtheorem{notation*}{Notation}
\newtheorem{theorem}{Theorem}
\newtheorem*{theorem*}{Theorem}
\newtheorem{corollary}[theorem]{Korollar}
\newtheorem*{lemma}{Lemma} 
\theoremstyle{remark}
\newtheorem*{remark}{Bemerkung}
\newtheorem*{bemerkung}{Bemerkung}
\theoremstyle{definition}
\newtheorem{definition}{Definition}[section]
\newtheorem{hilfsdefinition}{Hilfsdefinition}[section]
\newtheorem*{tempdefinition}{Temporäre Definition}
\newtheorem*{example}{Beispiel}
\newtheorem*{hint}{Hinweis}

\title{Bd. 09 - Algebraische Strukturen}
\author{Martin Kunze}
\date{}

\begin{document}
\maketitle
\tableofcontents
\listoftheorems


\chapter{Wichtige Begriffe und Sätze aus Band 1}
\label{sec:band1-essentials}

In Band~1 haben wir die Grundlagen der Aussagen- und Prädikatenlogik ausführlich behandelt. Dabei entstanden Schlussregeln für das Kalkül des natürlichen Schließens, die wir im Folgenden kurz vorstellen wollen. In diesem Band wenden wir die Schlussregeln an, um wichtige Sätze der Logik herzuleiten.

\vspace{1em}

\section{Regeln des natürlichen Schließens (Kurzfassung)}

\subsection{Annahmeregel}
\label{rule:A}

\begin{description}
\item[Regel (\(\mathrm{A}\)) \quad (Annahmeregel)]
  Aus dem Nichts darf eine beliebige Formel \(P\) als Annahme eingeführt werden:
  \[
    \frac{\quad}{P}
    \quad\text{(A)}
  \]
\end{description}

\subsection{Konjunktion}
\label{sec:land-rules}

\paragraph{Einführung der Konjunktion}
\label{rule:AI}

\[
\frac{P \quad Q}{P \land Q}
\quad (\land I)
\]

\paragraph{Eliminierung der Konjunktion}
\label{rule:AE1}\label{rule:AE2}

\[
\frac{P \land Q}{P}
\quad (\land E1)
\qquad
\frac{P \land Q}{Q}
\quad (\land E2)
\]

\subsection{Disjunktion}
\label{sec:lor-rules}

\paragraph{Einführung der Disjunktion}
\label{rule:OI1}\label{rule:OI2}

\[
\frac{P}{P \lor Q}
\quad (\lor I1)
\qquad
\frac{Q}{P \lor Q}
\quad (\lor I2)
\]

\paragraph{Eliminierung der Disjunktion}
\label{rule:OE}

\[
\frac{P \lor Q \quad
   \begin{array}{c}
   [P]\\
   \vdots\\
   R
   \end{array}
   \quad
   \begin{array}{c}
   [Q]\\
   \vdots\\
   R
   \end{array}
}{R}
\quad (\lor E)
\]

\subsection{Implikation}
\label{sec:to-rules}

\paragraph{Einführung der Implikation}
\label{rule:RI}

\[
\frac{
  \begin{array}{c}
    [P]\\
    \vdots\\
    Q
  \end{array}
}{P \to Q}
\quad (\to I)
\]

\paragraph{Eliminierung der Implikation}
\label{rule:RE}

\[
\frac{P \to Q \quad P}{Q}
\quad (\to E)
\]

\subsection{Äquivalenz}
\label{sec:leftrightarrow-rules}

\paragraph{Einführung der Äquivalenz}
\label{rule:LRI}

\[
\frac{P \to Q \quad Q \to P}{P \leftrightarrow Q}
\quad (\leftrightarrow I)
\]

\paragraph{Eliminierung der Äquivalenz}
\label{rule:LRE1}\label{rule:LRE2}

\[
\frac{P \leftrightarrow Q}{P \to Q}
\quad (\leftrightarrow E1)
\qquad
\frac{P \leftrightarrow Q}{Q \to P}
\quad (\leftrightarrow E2)
\]

\subsection{Widerspruch und Negation}
\label{sec:bot-neg-rules}

\paragraph{Widerspruchs-Einführung}
\label{rule:BI}

\[
\frac{P \quad \neg P}{\bot}
\quad (\bot I)
\]

\paragraph{Negations-Einführung}
\label{rule:CI}

\[
\frac{
   \begin{array}{c}
     [P]\\
     \vdots\\
     \bot
   \end{array}
}{\neg P}
\quad (\neg I)
\]

\paragraph{Negations-Eliminierung}
\label{rule:CE}

\[
\frac{
\begin{array}{c}
\text{[\ensuremath{\neg P}]} \\
\vdots \\
\bot
\end{array}
}{P}
\quad \neg I
\]

\subsection{Allquantor}
\label{sec:forall-rules}

\paragraph{Einführung}
\label{rule:UI}

\[
\frac{P(x)}{\forall x\,P(x)}
\quad (\forall I)
\]

\paragraph{Eliminierung}
\label{rule:UE}

\[
\frac{\forall x\,P(x)}{P(t)}
\quad (\forall E)
\]

\subsection{Existenzquantor}
\label{sec:exists-rules}

\paragraph{Einführung}
\label{rule:EI}

\[
\frac{P(t)}{\exists x\,P(x)}
\quad (\exists I)
\]

\paragraph{Eliminierung}
\label{rule:EE}

\[
\frac{
  \exists x\,P(x)
  \quad
  \begin{array}{c}
    [P(x)]\\
    \vdots\\
    Q
  \end{array}
}{Q}
\quad (\exists E)
\]

\subsection{Regeln für das Identitätssymbol}
\label{sec:identity-rules}

\paragraph{Einführung}
\label{rule:II}

\[
\frac{\quad}{t=t}
\quad (=I)
\]

\paragraph{Eliminierung}
\label{rule:IE}

\[
\frac{t = u \quad P(t)}{P(u)}
\quad (=E)
\]

\label{rule:Neq}
\[\forall t,u(t\neq u\coloneqq \neg(t=u))\]

\section{Regeln für die eindeutige Existenz}
\paragraph{Einführungsregel für die eindeutige Existenz}
\label{rule:UEI}

\[
\frac{
  \exists x\,P(x)
  \quad
  \begin{array}{c}
    [P(a)]\quad [P(b)] \\
    \vdots \\
    a = b 
  \end{array}
}{\exists! x\,P(x)}
\quad \exists! I
\]

\paragraph{Eliminierungsregel  für die eindeutige Existenz}
\label{rule:UEE}


\[
\frac{
  \exists! x\,P(x)
  \quad
  \begin{array}{c}
    [P(a)]\quad [\forall y\,(P(y)\rightarrow a=y)]\\
    \vdots \\
    Q
  \end{array}
}{
  Q
}
\quad \exists! E
\]

\chapter{Einführung in algebraische Strukturen}

\section{Halbgruppen}

\begin{definition}[Halbgruppe]
    Der \textbf{Begriff der Halbgruppe} wird durch das \textbf{Symbol} \((S, \cdot)\) \textbf{implizit definiert}. Dabei gelten die folgenden Axiome:
    
    \begin{itemize}
     
        \item \textbf{Existenz der binären Operation \(\cdot\)}:
        \[
        \cdot \colon S \times S \to S \text{ ist eine binäre Operation auf } S.
        \]
        
        \item \textbf{Assoziativität}: 
        \[
        \forall a, b, c \in S \colon (a \cdot b) \cdot c = a \cdot (b \cdot c).
        \]
    \end{itemize}
\end{definition}
\begin{remark}
    In Halbgruppen wird der Operator \(\cdot\) häufig weggelassen, sodass die Operation zwischen zwei Elementen \(a\) und \(b\) einfach als \(ab:=a\cdot b\) geschrieben wird. Dies dient der Vereinfachung der Notation, insbesondere in formalen Ausdrücken.
\end{remark}

\subsubsection*{Einführungsregel für Halbgruppen (HG\textsubscript{I})}
\label{rule:rSemigroupI}
Die Einführungsregel für Halbgruppen \((S, \cdot)\) ermöglicht es, eine Halbgruppe zu definieren, indem die Menge \(S\), die binäre Operation \(\cdot\) und die Assoziativität der Operation gezeigt werden. Diese Regel basiert auf den Axiomen der Halbgruppe.

\[
\begin{array}{llll}
    i       & (1) & \cdot \colon S \times S \to S & \dots \\
    j       & (2) & \forall a, b, c \in S ((a \cdot b) \cdot c = a \cdot (b \cdot c)) & \dots \\
    i,j     & (3) & (S, \cdot) \text{ ist eine Halbgruppe} & \rSemigroupI{1,2}
\end{array}
\]

\(i\) und \(j\) sind dabei Listen von Annahmen.

\subsubsection*{Eliminierungsregel für Halbgruppen (HG\textsubscript{E})}
\label{rule:rSemigroupE}
Die Eliminierungsregel für Halbgruppen \((S, \cdot)\) erlaubt es, aus der Tatsache, dass \((S, \cdot)\) eine Halbgruppe ist, die Existenz der Menge \(S\), der binären Operation \(\cdot\) sowie die Assoziativität der Operation abzuleiten.

\[
\begin{array}{llll}
    i       & (1) & (S, \cdot) \text{ ist eine Halbgruppe.} & \dots \\
    i       & (2) & \cdot \colon S \times S \to S& \rSemigroupE{1} \\
    i       & (3) & \forall a, b, c \in S \colon (a \cdot b) \cdot c = a \cdot (b \cdot c). & \rSemigroupE{1}
\end{array}
\]

\(i\) ist dabei die Liste der Annahmen.

\subsubsection*{Regel der Assoziativität}
\label{rule:rAssociativityHG}
Die Regel der Assoziativität ermöglicht es, aus den Elementen einer Halbgruppe die Assoziativität der binären Operation abzuleiten. Dies ist eine direkte Anwendung des Assoziativitätsaxioms der Halbgruppe.

\[
\begin{array}{llll}
        & (1) & (ab)c = a(bc) & \rAssociativityHG{}
\end{array}
\]


\subsection{Abelsche Halbgruppen}

\begin{definition}[Abelsche Halbgruppe]
    Eine \textbf{abelsche Halbgruppe} ist eine Halbgruppe \((S, \cdot)\), in der zusätzlich zur Assoziativität die Kommutativität der binären Operation \(\cdot\) gilt. Das bedeutet, dass für alle Elemente \(a, b \in S\) gilt:
    
    \begin{itemize}
        \item \textbf{Kommutativität}:
        \[
        \forall a, b \in S(a \cdot b = b \cdot a).
        \]
    \end{itemize}
\end{definition}

\begin{remark}
    In abelschen Halbgruppen wird der Operator \(\cdot\) häufig weggelassen, sodass die Operation zwischen zwei Elementen \(a\) und \(b\) einfach als \(ab := a \cdot b\) geschrieben wird. Dies dient der Vereinfachung der Notation, insbesondere in formalen Ausdrücken.
\end{remark}

\subsubsection*{Einführungsregel für abelsche Halbgruppen (AHG\textsubscript{I})}
\label{rule:rAbelianSemigroupI}
Die Einführungsregel für abelsche Halbgruppen ermöglicht es, eine abelsche Halbgruppe zu definieren, indem die Menge \(S\), die binäre Operation \(\cdot\), die Assoziativität und die Kommutativität der Operation gezeigt werden.


\[
\begin{array}{llll}
    i       & (1) &  (S, \cdot) \text{ ist eine Halbgruppe.} & \dots \\
    j       & (2) & \forall a, b \in S (ab = ba) & \dots \\
    i,j   & (3) & (S, \cdot) \text{ ist eine abelsche Halbgruppe.} & \rAbelianSemigroupI{1,2}
\end{array}
\]



\subsubsection*{Eliminierungsregel für abelsche Halbgruppen (AHG\textsubscript{E})}
\label{rule:rAbelianSemigroupE}
Die Eliminierungsregel für abelsche Halbgruppen ermöglicht es, aus der Tatsache, dass \((S, \cdot)\) eine abelsche Halbgruppe ist, die Assoziativität und Kommutativität der binären Operation \(\cdot\) sowie die Halbgruppenstruktur von \(S\) abzuleiten.

\[
\begin{array}{llll}
    i       & (1) &  (S, \cdot) \text{ ist eine abelsche Halbgruppe.} & \dots \\
    i       & (2) & \forall a, b \in S(ab = ba) & \rAbelianSemigroupE{1} \\
    i       & (3) & (S, \cdot) \text{ ist eine Halbgruppe.} & \rAbelianSemigroupE{1}
\end{array}
\]

\(i\) ist dabei eine Liste von Annahmen.

\subsubsection*{Kommutativität der abelschen Halbgruppe}
\label{rule:rCommutativeSemigroup}
Die Kommutativität der abelschen Halbgruppe ermöglicht es, in einer abelschen Halbgruppe \((S, \cdot)\) die Reihenfolge der Operanden der binären Operation \(\cdot\) zu vertauschen. Dies folgt direkt aus der Definition einer abelschen Halbgruppe, bei der die Operation \(\cdot\) sowohl assoziativ als auch kommutativ ist.

\[
\begin{array}{llll}
          & (1) & ab = ba & \rCommutativeSemigroup{} \\
\end{array}
\]


Die Regel der Kommutativität erlaubt es, die Argumente \(a\) und \(b\) der Operation zu vertauschen, ohne dass sich das Ergebnis ändert, was eine grundlegende Eigenschaft von abelschen Halbgruppen ist.

\section{Monoide}

\begin{definition}[Monoid]
    Der \textbf{Begriff des Monoids} wird durch das \textbf{Symbol} \((M, \cdot)\) \textbf{implizit definiert}. Dabei gelten die folgenden Axiome:
    
    \begin{itemize}
        \item \textbf{Existenz der binären Operation \(\cdot\)}:
        \[
        \cdot \colon M \times M \to M \text{ ist eine binäre Operation auf } M.
        \]
        
        \item \textbf{Assoziativität}: 
        \[
        \forall a, b, c \in M \colon (a \cdot b) \cdot c = a \cdot (b \cdot c).
        \]
        
        \item \textbf{Existenz eines neutralen Elements}:
        \[
        \exists e \in M\forall a \in M \colon e \cdot a = a \cdot e = a.
        \]
    \end{itemize}
\end{definition}

\begin{remark}
    In Monoiden wird der Operator \(\cdot\) häufig weggelassen, sodass die Operation zwischen zwei Elementen \(a\) und \(b\) einfach als \(ab := a \cdot b\) geschrieben wird. Dies dient der Vereinfachung der Notation, insbesondere in formalen Ausdrücken.
\end{remark}

\subsubsection*{Einführungsregel für Monoide (M\textsubscript{I})}
\label{rule:rMonoidI}
Die Einführungsregel für Monoide \((M, \cdot)\) ermöglicht es, ein Monoid zu definieren, indem die Menge \(M\), die binäre Operation \(\cdot\), die Assoziativität der Operation und die Existenz eines neutralen Elements gezeigt werden. Diese Regel basiert auf den Axiomen des Monoids.

\[
\begin{array}{llll}
    i       & (1) & \cdot \colon M \times M \to M & \dots \\
    j       & (2) & \forall a, b, c \in M ((ab)c = a(bc)) & \dots \\
    k       & (3) & \exists e \in M \forall a \in M(e \cdot a = a \cdot e = a) & \dots \\
    i,j,k   & (4) & (M, \cdot, e) \text{ ist ein Monoid.} & \rMonoidI{1,2,3}
\end{array}
\]

\(i\), \(j\) und \(k\) sind dabei Listen von Annahmen.

\subsubsection*{Eliminierungsregel für Monoide (M\textsubscript{E})}
\label{rule:rMonoidE}
Die Eliminierungsregel für Monoide \((M, \cdot)\) erlaubt es, aus der Tatsache, dass \((M, \cdot)\) ein Monoid ist, die Existenz der Menge \(M\), der binären Operation \(\cdot\), die Assoziativität der Operation sowie die Existenz eines neutralen Elements abzuleiten.

\[
\begin{array}{llll}
    i       & (1) & (M, \cdot, e) \text{ ist ein Monoid.} & \dots \\
    i       & (2) & \cdot \colon M \times M \to M & \rMonoidE{1} \\
    i       & (3) & \forall a, b, c \in M \colon (ab)c = a(bc) & \rMonoidE{1} \\
    i       & (4) & \exists e \in M \forall a \in M \colon e \cdot a = a \cdot e = a & \rMonoidE{1}
\end{array}
\]

\(i\) ist dabei die Liste der Annahmen.

\subsubsection*{Regel der Assoziativität}
\label{rule:rAssociativityMonoid}
Die Regel der Assoziativität ermöglicht es, aus den Elementen einer Halbgruppe die Assoziativität der binären Operation abzuleiten. Dies ist eine direkte Anwendung des Assoziativitätsaxioms der Halbgruppe.

\[
\begin{array}{llll}
         & (1) & (ab)c = a(bc) & \rAssociativityMonoid{}
\end{array}
\]

\(i\) ist dabei eine Liste von Annahmen.

\subsubsection*{Regel des neutralen Elements}
\label{rule:rNeutralElementMonoid}
Die Regel des neutralen Elements ermöglicht es, aus der Tatsache, dass \((M, \cdot, e)\) ein Monoid ist, das neutrale Element \(e \in M\) und dessen Eigenschaften abzuleiten. Dies ist eine direkte Anwendung des Axioms des neutralen Elements im Monoid. Für \(a\in M\) gilt somit:

\[
\begin{array}{llll}
        & (1) & e \cdot a = a \cdot e = a & \rNeutralElementMonoid{} \\
\end{array}
\]

\(i\) ist dabei eine Liste von Annahmen.


\subsubsection*{Eindeutigkeit des neutralen Elements}
\label{ExeweApostropheInMLpFaaInMLpeMultaEqualsaMulteEqualsaAndeApostropheMultaEqualsaMulteApostropheEqualsaRpRpImpeEqualseApostrophe}
\begin{theorem}[\(\exists e, e' \in M (\forall a \in M (e \cdot a = a \cdot e = a \land e' \cdot a = a \cdot e' = a)) \vdash e = e'\) (Eindeutigkeit des neutralen Elements)]
Sei \((M, \cdot)\) ein Monoid, das durch die Menge \(M\), die binäre Operation \(\cdot\) und ein neutrales Element \(e \in M\) definiert ist, sodass gilt:
\[
\forall a \in M \colon e \cdot a = a \cdot e = a.
\]
Wenn es ein weiteres Element \(e' \in M\) gibt, das ebenfalls die neutrale Eigenschaft besitzt, d. h.,
\[
\forall a \in M \colon e' \cdot a = a \cdot e' = a,
\]
dann folgt, dass \(e = e'\).
\end{theorem}
\begin{proof}
    Seien \(e,e'\in M\) neutrale Elemente von M. Dann gilt:
    \[
	\begin{array}{llll}
		1 &  (1) & \forall a \in M(e \cdot a = a \cdot e = a \land e' \cdot a = a \cdot e' = a)  & \rA \\
		1 &  (2) & e \cdot e = e \cdot e = e \land e' \cdot e = e \cdot e' = e & \rUE{1} \\	
            1 &  (3) & e' \cdot e = e \cdot e' = e & \rAEb{2} \\	
            1 &  (4) & e' \cdot e = e & \rIEb{3} \\	
		1 &  (5) & e \cdot e' = e' \cdot e = e' \land e' \cdot e' = e' \cdot e' = e' & \rUE{1} \\	
		1 &  (6) & e \cdot e' = e' \cdot e = e' & \rAEa{5} \\	
            1 &  (7) & e' \cdot e = e' & \rIEb{6} \\	
            1 &  (8) & e = e' & \rIE{4,7} \\	
	\end{array}
	\]
\end{proof}



\label{LpMwMultweRpInMonoidImpLpMwMultRpInSemiGroup}
\begin{theorem}[\((M,\cdot, e) \text{ ist ein Monoid} \vdash (M,\cdot) \text{ ist eine Halbgruppe}\)]
\end{theorem}
\begin{proof}
	\[
	\begin{array}{llll}
		1 &  (1) & (M, \cdot, e) \text{ ist ein Monoid.} & \rA \\
		1 &  (2) & \cdot \colon M \times M \to M & \rMonoidE{1} \\			
		1 &  (3) & \forall a, b, c \in M \colon (ab)c = a(bc) & \rMonoidE{1} \\
		1 &  (4) & (M, \cdot) \text{ ist eine Halbgruppe.} & \rSemigroupI{2,3} \\	
	\end{array}
	\]
\end{proof}

\subsection{Abelsche Monoide}

\begin{definition}[Abelsches Monoid]
    Ein \textbf{abelsches Monoid} ist ein Monoid \((M, \cdot, e)\), in dem zusätzlich zur Assoziativität und der Existenz eines neutralen Elements die Kommutativität der binären Operation \(\cdot\) gilt. Das bedeutet, dass für alle Elemente \(a, b \in M\) gilt:
    
    \begin{itemize}
        \item \textbf{Kommutativität}:
        \[
        \forall a, b \in M \colon a \cdot b = b \cdot a.
        \]
    \end{itemize}
\end{definition}

\begin{remark}
    In abelschen Monoiden wird der Operator \(\cdot\) häufig weggelassen, sodass die Operation zwischen zwei Elementen \(a\) und \(b\) einfach als \(ab := a \cdot b\) geschrieben wird. Dies dient der Vereinfachung der Notation, insbesondere in formalen Ausdrücken.
\end{remark}

\subsubsection*{Einführungsregel für abelsche Monoide (AM\textsubscript{I})}
\label{rule:rAbelianMonoidE}
Die Einführungsregel für abelsche Monoide ermöglicht es, aus der Tatsache, dass \((M, \cdot, e)\) ein abelsches Monoid ist, sowohl die Kommutativität der binären Operation \(\cdot\) als auch die Monoidstruktur abzuleiten. Das bedeutet, dass sowohl die Assoziativität als auch die Existenz eines neutralen Elements erfüllt sein müssen.

\[
\begin{array}{llll}
    i       & (1) & (M, \cdot, e) \text{ ist ein abelsches Monoid.} & \dots \\
    i       & (2) & \forall a, b \in M (ab = ba) & \rAbelianMonoidE{1}\\
    i       & (3) & (M, \cdot, e) \text{ ist ein  Monoid.} & \rAbelianMonoidE{1}
\end{array}
\]

\(i\) ist dabei eine Liste von Annahmen.


\subsubsection*{Eliminierungsregel für abelsche Monoide (AM\textsubscript{E})}
\label{rule:rAbelianMonoidI}
Die Eliminierungsregel für abelsche Monoide ermöglicht es, aus der Tatsache, dass \((M, \cdot, e)\) ein Monoid ist und die Kommutativität der Operation gezeigt wird, die abelsche Monoidstruktur abzuleiten.

\[
\begin{array}{llll}
    i       & (1) & (M, \cdot, e) \text{ ist ein Monoid.} & \dots \\
    j       & (2) & \forall a, b \in M (ab = ba) & \dots \\
    i,j     & (3) & (M, \cdot, e) \text{ ist ein abelscher Monoid.} & \rAbelianMonoidI{1,2}
\end{array}
\]

\(i\) und \(j\) sind dabei Listen von Annahmen.

\subsubsection*{Kommutativität des Monoids}
\label{rule:rCommutativeMonoid}
Die Kommutativität des Monoids ermöglicht es, in einem abelschen Monoid \((M, \cdot, e)\) die Reihenfolge der Operanden der binären Operation \(\cdot\) zu vertauschen. Dies folgt direkt aus der Definition eines abelschen Monoids, bei dem die Operation \(\cdot\) sowohl assoziativ als auch kommutativ ist.


\[
\begin{array}{llll}
          & (1) & ab = ba & \rCommutativeMonoid{} \\
\end{array}
\]

\label{LpMwMultweRpInAbelMonoidImpLpMwMultRpInAbelSemiGroup}
\begin{theorem}[\((M,\cdot, e) \text{ ist ein abelscher Monoid} \vdash (M,\cdot) \text{ ist eine abelsche Halbgruppe}\)]
\end{theorem}
\begin{proof}
	\[
	\begin{array}{llll}
		1 &  (1) & (M, \cdot, e) \text{ ist ein abelscher Monoid.} & \rA \\
		1 &  (2) & (M, \cdot, e) \text{ ist ein Monoid.} & \rAbelianMonoidE{1} \\	
		1 &  (3) & \forall a, b \in M (ab = ba) & \rAbelianMonoidE{1} \\
		1 &  (4) & (M, \cdot) \text{ ist eine Halbgruppe.} & \LpMwMultweRpInMonoidImpLpMwMultRpInSemiGroup{1} \\	
  		1 &  (5) & (M, \cdot) \text{ ist eine abelsche Halbgruppe.} & \rAbelianSemigroupI{4,3} \\	
	\end{array}
	\]
\end{proof}

\subsection{Erweiterte Vertauschungsgesetze abelscher Monoide}
Im folgenden sei \((M, \cdot, e)\) ein abelscher Monoid mit der binären Operation \(\cdot :M\times M\rightarrow M\).

\label{aInMwbInMwcInMImpLpaPlusbRpPluscEqualsLpaPluscRpPlusb}
\begin{theorem}[\(a\in M, b\in M, c\in M \vdash (ab)c=(ac)b\)]
\end{theorem}
\begin{proof}
        Seien \(a,b,c\in M\), dann gilt:
        \[
	\begin{array}{lllcll}
		  &  (1) & (ab)c &=& a(bc) & \rAssociativityMonoid{} \\
            &  (2) &             &=& a(cb) & \rCommutativeMonoid{} \\
    	&  (3) & &=&(ac)b & \rAssociativityMonoid{} \\
	\end{array}
	\]
\end{proof}

\label{MInAbelMonoidwawbwcInMImpLpabRpcEqualsLpcaRpb}
\begin{theorem}[\(a,b,c\in M \vdash (ab)c=(ca)b\)]
\end{theorem}
\begin{proof}
        Seien \(a,b,c\in M\), dann gilt:
        \[
	\begin{array}{lllcll}
		  &  (1) & (ab)c &=& (ac)b & \aInMwbInMwcInMImpLpaPlusbRpPluscEqualsLpaPluscRpPlusb{} \\
            &  (2) &             &=& (ca)b & \rCommutativeMonoid{} \\
	\end{array}
	\]
\end{proof}

\label{aInMwbInMwcInMImpaPlusLpbPluscRpEqualsLpaPluscRpPlusb}
\begin{theorem}[\(a\in M, b\in M, c\in M \vdash a(bc)=(ac)b\)]
\end{theorem}
\begin{proof}
        Seien \(a,b,c\in M\), dann gilt:
        \[
	\begin{array}{lllcll}
		  &  (1) & a(bc) &=& (ab)c & \rAssociativityMonoid{} \\
            &  (2) &             &=& (ac)b & \aInMwbInMwcInMImpLpaPlusbRpPluscEqualsLpaPluscRpPlusb{} \\
	\end{array}
	\]
\end{proof}

\label{aInMwbInMwcInMwdInMImpLpaPlusbRpPlusLpcPlusdRpEqualsLpaPluscRpPlusLpbPlusdRp}
\begin{theorem}[\(a\in M, b\in M, c\in M, d\in M \vdash (ab)(cd)=(ac)(bd)\)]
\end{theorem}
\begin{proof}
    Seien \(a,b,c,d\in M\), dann gilt:
	\[
	\begin{array}{lllcll}
		  &  (1) & (ab)(cd) &=&((ab)c)d & \rAssociativityMonoid{} \\
            &  (2) &             &=& ((ac)b)d & \aInMwbInMwcInMImpLpaPlusbRpPluscEqualsLpaPluscRpPlusb{1,2,3} \\
    	&  (3) & &=&(ac)(bd) & \rAssociativityMonoid{} \\
	\end{array}
	\]
\end{proof}


\section{Halbringe}

\begin{definition}[Halbring]
    Ein \textbf{Halbring} ist eine algebraische Struktur \((R, +, \cdot)\), wobei \(R\) eine Menge ist und \(+\) und \(\cdot\) zwei binäre Operationen auf \(R\) sind, die die folgenden Axiome erfüllen:
    
    \begin{itemize}
        \item \((R, +)\) ist eine abelsche Halbgruppe.        
        \item \((R, \cdot)\) ist eine Halbgruppe.
        \item \textbf{Linksdistributivität der Multiplikation über Addition}:
        \[
        \forall a, b, c \in R (a \cdot (b + c) = (a \cdot b) + (a \cdot c)).
        \]
        \item \textbf{Rechtsdistributivität der Multiplikation über Addition}:
        \[
        \forall a, b, c \in R ((a + b) \cdot c = (a \cdot c) + (b \cdot c)).
        \]
    \end{itemize}
\end{definition}

\begin{remark}
    In Halbringen wird die Operation \(\cdot\) oft weggelassen, sodass \(a \cdot b\) als \(ab\) geschrieben wird. Diese Vereinfachung dient der Notation und Übersichtlichkeit.
\end{remark}

\subsubsection*{Einführungsregel für Halbringe (HR\textsubscript{I})}
\label{rule:rSemiringI}
Die Einführungsregel für Halbringe \((R, +, \cdot)\) ermöglicht es, einen Halbring zu definieren, indem die Menge \(R\), die Operationen \(+\) und \(\cdot\) sowie die entsprechenden Axiome (abelsche Halbgruppe, Halbgruppe und Distributivität) gezeigt werden.

\[
\begin{array}{llll}
    i       & (1) & (R, +) \text{ ist eine abelsche Halbgruppe.} & \dots \\
    j       & (2) & (R, \cdot) \text{ ist eine Halbgruppe.} & \dots \\
    k       & (3) & \forall a, b, c \in R (a(b + c)=ab+ac) & \dots \\
    l       & (4) & \forall a, b, c \in R ((a + b)c=ac+bc) & \dots \\
    i,j,k,l   & (5) & (R, +, \cdot) \text{ ist ein Halbring.} & \rSemiringI{1,2,3,4}
\end{array}
\]

\(i\), \(j\), \(k\) und \(l\) sind Listen von Annahmen.

\subsubsection*{Eliminierungsregel für Halbringe (HR\textsubscript{E})}
\label{rule:rSemiringE}
Die Eliminierungsregel für Halbringe erlaubt es, aus der Tatsache, dass \((R, +, \cdot)\) ein Halbring ist, die abelsche Halbgruppe \((R, +)\), die Halbgruppe \((R, \cdot)\) sowie die Distributivität der Multiplikation über Addition abzuleiten.

\[
\begin{array}{llll}
    i       & (1) & (R, +, \cdot) \text{ ist ein Halbring.} & \dots \\
    i       & (2) & (R, +) \text{ ist eine abelsche Halbgruppe.} & \rSemiringE{1} \\
    i       & (3) & (R, \cdot) \text{ ist eine Halbgruppe.} & \rSemiringE{1} \\
    i       & (4) & \forall a, b, c \in R (a(b + c)=ab+ac) & \rSemiringE{1} \\
    i       & (5) & \forall a, b, c \in R ((a + b)c=ac+bc) & \rSemiringE{1}
\end{array}
\]

\(i\) ist dabei die Liste der Annahmen.

\subsubsection*{Regel der Linksdistributivität}
\label{rule:rLeftDistributiveSemigroup}
Die Regel der Linksdistributivität ermöglicht es, die Eigenschaft der Linksdistributivität der Multiplikation über Addition in Halbringen zu nutzen. Diese Regel basiert auf dem Axiom der Linksdistributivität in Halbringen. Für Elemente \(a,b,c\in R\) gilt somit:

\[
\begin{array}{llll}
     & (1) & a(b + c) = ab+ac & \rLeftDistributiveSemigroup{}
\end{array}
\]


\subsubsection*{Regel der Rechtsdistributivität}
\label{rule:rRightDistributiveSemigroup}
Die Regel der Rechtsdistributivität ermöglicht es, die Eigenschaft der Rechtsdistributivität der Multiplikation über Addition in Halbringen zu nutzen. Diese Regel basiert auf dem Axiom der Rechtsdistributivität in Halbringen.

\[
\begin{array}{llll}
      & (1) & (a + b)c = ac+bc & \rRightDistributiveSemigroup{}
\end{array}
\]


\subsection{Abelsche Halbringe}

\begin{definition}[Abelscher Halbring]
    Ein \textbf{abelscher Halbring} ist ein Halbring \((R, +, \cdot)\), wobei die folgenden Bedingungen erfüllt sind:
    
    \begin{itemize}
        \item \((R, +)\) ist eine abelsche Halbgruppe.
        \item \((R, \cdot)\) ist eine abelsche Halbgruppe.
        \item \textbf{Linksdistributivität der Multiplikation über Addition}:
        \[
        \forall a, b, c \in R \colon a \cdot (b + c) = (a \cdot b) + (a \cdot c).
        \]
        \item \textbf{Rechtsdistributivität der Multiplikation über Addition}:
        \[
        \forall a, b, c \in R \colon (a + b) \cdot c = (a \cdot c) + (b \cdot c).
        \]
    \end{itemize}
\end{definition}

\begin{remark}
    In abelschen Halbringen wird der Operator \(\cdot\) häufig weggelassen, sodass die Multiplikation zwischen zwei Elementen \(a\) und \(b\) einfach als \(ab := a \cdot b\) geschrieben wird. Diese Vereinfachung dient der Notation und Übersichtlichkeit.
\end{remark}

\subsubsection*{Einführungsregel für abelsche Halbringe (AHR\textsubscript{I})}
\label{rule:rAbelianSemiringI}
Die Einführungsregel für abelsche Halbringe \((R, +, \cdot)\) ermöglicht es, einen abelschen Halbring zu definieren, indem die Menge \(R\), die Operationen \(+\) und \(\cdot\), die Distributivität sowie die abelschen Halbgruppenstrukturen gezeigt werden.

\[
\begin{array}{llll}
    i       & (1) & (R, +) \text{ ist eine abelsche Halbgruppe.} & \dots \\
    j       & (2) & (R, \cdot) \text{ ist eine abelsche Halbgruppe.} & \dots \\
    k       & (3) & \forall a, b, c \in R \colon a(b + c) = ab + ac & \dots \\
    l       & (4) & \forall a, b, c \in R \colon (a + b)c = ac + bc & \dots \\
    i,j,k,l & (5) & (R, +, \cdot) \text{ ist ein abelscher Halbring.} & \rAbelianSemiringI{1,2,3,4}
\end{array}
\]

\subsubsection*{Eliminierungsregel für abelsche Halbringe (AHR\textsubscript{E})}
\label{rule:rAbelianSemiringE}
Die Eliminierungsregel für abelsche Halbringe erlaubt es, aus der Tatsache, dass \((R, +, \cdot)\) ein abelscher Halbring ist, die abelschen Halbgruppenstrukturen für \(+\) und \(\cdot\) sowie die Distributivität der Multiplikation über Addition abzuleiten.

\[
\begin{array}{llll}
    i       & (1) & (R, +, \cdot) \text{ ist ein abelscher Halbring.} & \dots \\
    i       & (2) & (R, +) \text{ ist eine abelsche Halbgruppe.} & \rAbelianSemiringE{1} \\
    i       & (3) & (R, \cdot) \text{ ist eine abelsche Halbgruppe.} & \rAbelianSemiringE{1} \\
    i       & (4) & \forall a, b, c \in R \colon a(b + c) = ab + ac & \rAbelianSemiringE{1} \\
    i       & (5) & \forall a, b, c \in R \colon (a + b)c = ac + bc & \rAbelianSemiringE{1}
\end{array}
\]

\subsubsection*{Regel der Linksdistributivität}
\label{rule:rLeftDistributiveAbelianSemigroup}
Die Regel der Linksdistributivität ermöglicht es, die Eigenschaft der Linksdistributivität der Multiplikation über Addition in abelschen Halbringen zu nutzen. Diese Regel basiert auf dem Axiom der Linksdistributivität in abelschen Halbringen.

\[
\begin{array}{llll}
      & (1) & a(b + c) = ab+ac & \rLeftDistributiveAbelianSemigroup{}
\end{array}
\]
\subsubsection*{Regel der Rechtsdistributivität}
\label{rule:rRightDistributiveAbelianSemigroup}
Die Regel der Rechtsdistributivität ermöglicht es, die Eigenschaft der Rechtsdistributivität der Multiplikation über Addition in abelschen Halbringen zu nutzen. Diese Regel basiert auf dem Axiom der Rechtsdistributivität in abelschen Halbringen.

\[
\begin{array}{llll}
       & (1) & (a + b)c = ac+bc & \rRightDistributiveAbelianSemigroup{}
\end{array}
\]



\subsection{Potenzen in Halbgruppen und Monoiden}

\begin{definition}[Potenz eines Elements in einer Halbgruppe]
    Sei \((S, \cdot)\) eine Halbgruppe mit einer \textbf{multiplikativen} Operation \(\cdot\), und \(a \in S\) ein beliebiges Element. Für \(n \in \mathbb{N}\) definieren wir die \textbf{Potenz} \(a^n\) des Elements \(a\) induktiv durch:
    \[
    a^0 := 0,
    a^1 := a,
    \]
    und für \(n > 1\):
    \[
    a^{n+1} := a^n \cdot a.
    \]
\end{definition}

\begin{remark}
    Die Potenzierung \(a^n\) setzt voraus, dass die Operation \(\cdot\) assoziativ ist, wie es in Halbgruppen und Monoiden gefordert ist. Die Definition lässt sich in Monoiden erweitern, indem man \(a^0 := e\) setzt, wobei \(e\) das neutrale Element des Monoids ist. In abelschen Strukturen wie abelschen Monoiden oder abelschen Gruppen ist die Reihenfolge der Multiplikation irrelevant, da die Operation kommutativ ist.
\end{remark}

\subsubsection*{Einführungsregel für Potenzen}
\label{rule:rPowerI}
Die Einführungsregel für Potenzen eines Elements in einer Halbgruppe erlaubt es, Potenzen \(a^n\) induktiv zu definieren, indem die Basis \(a\) und die rekursive Definition der Potenzierung verwendet werden.

\[
\begin{array}{llll}
    i   & (1) & a \in S & \dots \\
    i   & (2) & a^n = a^{n-1} \cdot a & \rPowerI{1} \\
    i   & (3) & a^n\in S & \rPowerI{1} \\
\end{array}
\]

\[
\begin{array}{llll}
    i   & (1) & a \in S & \dots \\
    i   & (2) & 0 = a^{0} & \rPowerI{1} \\
    i   & (3) & a^{0}\in S & \rPowerI{1} \\
\end{array}
\]

\[
\begin{array}{llll}
    i   & (1) & a \in S & \dots \\
    i   & (2) & a = a^{1} & \rPowerI{1} \\
    i   & (3) & a^{1}\in S & \rPowerI{1} \\
\end{array}
\]

\(i\) ist die Liste der Annahmen.




\end{document}