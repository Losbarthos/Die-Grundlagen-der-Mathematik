%============================================================
%  Bd. 03 - Mengenlehre %============================================================

\documentclass[main.tex]{subfiles}


\ifSubfilesClassLoaded{
    \usepackage{xr}
    \externaldocument{_B01}
  \externaldocument[B02-]{_B02}
}{
   % Code für als Subfile eingebunden
}

\title{Bd. 03 - Mengenlehre}
\author{Martin Kunze}
\date{}

\begin{document}

\maketitle
\tableofcontents
%\listoftheorems

\setcounter{file}{2}

\chapter{Einführung}

Die Mengenlehre ist ein fundamentaler Teil der Mathematik, der die Grundlage für viele andere Bereiche bildet. In diesem Kapitel werden wir die Zermelo-Fraenkel (ZF) Axiome der Mengenlehre einführen und diskutieren. Dabei bezeichnen \( A \), \( B \), \( C \) und \( D \) stets Mengen, es sei denn, es wird ausdrücklich etwas anderes angegeben. Alle Variablen, die als Mengen bezeichnet werden, sind implizit durch Allquantoren gebunden, es sei denn, es wird ein anderer Quantor verwendet. Das bedeutet, dass Aussagen wie „\( A = B \)“ oder „\( A \neq B \)“ für alle Mengen \( A \) und \( B \) gelten, ohne dass dies explizit angegeben werden muss.

\chapter{Die Zermelo-Fraenkel-Axiome}
Nachdem wir die grundlegenden Begriffe und Notationen eingeführt haben, wenden wir uns nun den Zermelo-Fraenkel-Axiomen zu, die das Fundament der modernen Mengenlehre bilden. Diese Axiome definieren, wie Mengen gebildet werden können und welche Eigenschaften sie besitzen.
\begin{definition}[Begriff der Menge]
Der \textbf{Begriff der Menge} wird durch das \textbf{Element-Symbol} \(\in\) \textbf{implizit definiert}. 
Das Symbol \(\in\) ist ein binäres Prädikat, das die Mitgliedschaft zwischen einem Element und einer Menge 
ausdrückt, also \(x \in y\) bedeutet, dass \(x\) ein Element von \(y\) ist. Die Eigenschaften von \(\in\) 
werden durch die folgenden Axiome der Zermelo-Fraenkel-Mengenlehre festgelegt, welche zusammen die Menge von 
Aussagen \(\Phi(\in)\) bilden:
\end{definition}

\FormulaAxiomAuto[Extensionalität]{\forall x\, (x \in A \leftrightarrow x \in B) \dashv\vdash A = B}

\FormulaAxiomAuto[Leere Menge]{\exists O\;\bigl(\forall x\,(x \not\in O)\bigr)}

\FormulaDefAuto[Leere Menge]{\emptyset := \iota O\bigl(\forall x\,(x \not\in O)\bigr)}
\begin{remark}
    Hieraus gewinnen wir für alle \(x\):
    \[
    x \not\in \emptyset.
    \]
\end{remark}

\FormulaThmAuto{ \exists! O\forall x (x \not\in O) }
\begin{tabproof}
  \proofstep{2}{ \forall x (x \not\in O) }{ \rA }
  \proofstep{3}{ \forall x (x \not\in P) }{ \rA }
  \proofstep{2}{ \forall x (x \not\in O \lor x \in P) }{ \FormulaRefAuto{\forall x(F(x))\lor\forall x(G(x))\vdash\forall x(F(x)\lor G(x))} }
\end{tabproof}

\edef\tempa{\forall x(F(x))\lor\forall x(G(x))\vdash\forall x(F(x)\lor G(x))}%
\sanitize{\tempa}%
\edef\labFull{2:\temp}%
\texttt{thm:\labFull}
\end{document}